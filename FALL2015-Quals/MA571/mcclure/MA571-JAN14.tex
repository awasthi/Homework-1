\chapter{MA 571 January 2014}
\begin{problem}
Let $X$ be a topological space, let $A$ be a subset of $X$, and let $U$ be
an open subset of $X$. Prove that $U\cap\bar A\subset\overline{U\cap A}$.
\end{problem}
\begin{proof}
The proof is simple and we have shown this before in the August 2014
quals, it goes as follows: If $U\cap\bar A=\emptyset$, there is nothing to
show. Let $x\in U\cap\bar A$. Then $x\in U$ and $x\in\bar A$. Since $x\in
U$ and $U$ is open, by Lemma C, there exists a neighborhood $V$ of $x$ such
that $V\subset U$; in particular, note that $V\cap U\neq\emptyset$. But
$x\in\bar A$ so $V\cap A\neq\emptyset$. Thus, $V\cap(U\cap
A)\neq\emptyset$. Thus, $x\in\overline{U\cap A}$.
\end{proof}
\begin{problem}
Let $\sim$ be an equivalence relation on $\bbR^2$ defined by
$(x,y)\sim(x',y')$ if and only if there is a nonzero $t$ with
$(x,y)=(tx',ty')$. Prove that the quotient space $\bbR^2/{\sim}$ is compact
but not Hausdorff.
\end{problem}
\begin{proof}
To show that $\bbR^2/{\sim}$ is compact, we need to show that for every
open covering $\calA$ of $\bbR^2/{\sim}$, there is a finite subcover
$\calA'\subset\calA$. Let $q\colon\bbR^2\to\bbR^2/{\sim}$ denote the
quotient map. Then, since $q$ is continuous and onto $\bbR^2/{\sim}$, the
set $\left\{q^{-1}(A_\alpha)\right\}_{A_\alpha\in\calA}$ is an open cover
of $\bbR^2$. In particular, there exists at least one $A_\alpha$ such that
$q^{-1}(A_\alpha)$ is a neighborhood of $(0,0)$. By Lemma C, there exists a
basic open neighborhood, i.e., an open ball $B((0,0),\varepsilon)\subset
q^{-1}(A_\alpha)$ for $\varepsilon>0$. Now, for any point
$[(x,y)]\in\bbR^2$ pick a representative $(x,y)\in\bbR^2$. Then, by the
Archimedean principle, there exists a positive real numbers $t',t''>0$ such
that $t'x<\sqrt{\varepsilon}$ and $t''y<\sqrt{\varepsilon}$. Define
$t\coloneqq\min\{t',t''\}$. Then $tx<\sqrt{\varepsilon}$ and
$ty<\varepsilon$. Thus, $(tx,ty)\in A_\alpha$ (since
$t^2x^2+t^2y^2<\varepsilon$). Since we can do this for any point
$[x]\in\bbR^2/{\sim}$, it follows that
$A_\alpha\supset\bbR^2/{\sim}$. Thus,
$\calA'\coloneqq\left\{A_\alpha\right\}$ is a finite subset of $\calA$
which covers $\bbR^2/{\sim}$. Thus, $\bbR^2/{\sim}$ is compact.

To show that $\bbR^2/{\sim}$ is not compact, we will employ a very similar
strategy, that is, we will show that every neighborhood of the point
$[0,0]\in\bbR^2/{\sim}$, contains every point $[x,y]\in\bbR^2/{\sim}$. Let
$[x,y]\in\bbR^2/{\sim}$ and let $U$ be a neighborhood of $[0,0]$. Then
$q^{-1}(U)$ is an open neighborhood of $(0,0)$, i.e., there exists an open
ball $B((0,0),\varepsilon)\subset q^{-1}(U)$. But as we have just shown,
for sufficiently small values of $t>0$, $(tx,ty)\in
B((0,0),\varepsilon)\subset q^{-1}(U)$. Thus, $[x,y]\in U$. In particular,
for any open neighborhood $V$ of $[x,y]$, $V\cap U\neq\emptyset$. Thus,
$\bbR^2/{\sim}$ is not Hausdorff.
\end{proof}
\begin{problem}
Let $X$ and $Y$ be topological spaces. Let $x_0\in X$ and let $C$ be a
compact subset of $Y$. Let $N$ be an open set in $X\times Y$ containing
$\left\{x_0\right\}\times C$. Prove that there is an open set $U$
containing $x_0$ and an open set $V$ containing $C$ such that $U\times
V\subset N$.
\end{problem}
\begin{proof}
This is a classical theorem called the tube lemma. We shall prove first in
the style of Munkres and second in the style of McClure (if I can find the
proof or somehow reconstruct it).

Let $X$, $Y$, $x_0$, $N$, and $C$ be as above. Note that since $C$ is
compact and the injection $\iota_{x_0}\colon X\hookrightarrow X\times Y$
given by $\iota_{x_0}(y)\coloneqq(x_0,y)$ is continuous by Theorem 18.4 (since
its components, i.e., projections to $X$ and $Y$, are continuous these are
$\pi_1(\iota_{x_0})(x)=x_0$ and $\pi_1(\iota_{x_0})(y)=y$ a constant map
and identity map, respectively) so the image of $C$ under $\iota_{x_0}$,
$\{x_0\}\times C$, is compact by Theorem 23.5. For every point
$x\in\left\{x_0\right\}\times C$, let $U_x\times V_x$ be a basic open
neighborhood of $x$ contained in $N$ (this can be arranged by Lemma
C). Then the collection $\calA\coloneqq\left\{U_x\times
  V_x\right\}_{x\in\left\{x_0\right\}\times   Y}$ forms an open covering of
$\left\{x_0\right\}\times C$. Thus, there exists a finite subcover
$\left\{U_{x_i}\times V_{x_i}\right\}_{i=1}^n$ of $\calA$.

Define $W\coloneqq U_{x_1}\cap\cdots\cap U_{x_n}$. This set is clearly open
since it is a finite intersection of open sets and contains $x_0$ since
every $U_{x_i}\times V_{x_i}$ intersects $\left\{x_0\right\}\times
Y$. Define $W'\coloneqq\pi_2(N)\cap Y$. This set is open since it is a
finite intersection of open sets in $Y$. The $W\times W'\subset N$. This is
clear since every point $(x,y)\in W\times W'$ is in $N$ ($x\in
W\subset U_{x_i}$ for all $i$ which in turn is a subset of $\pi_1(N)$ and
$y\in W'=\pi_2(N)$). Lastly, $W\times W'\supset \{x_0\}\times C$ since
$x_0\in W$ and $W'=\pi_1(N)\supset C$. Thus, $W\times W'\subset N$
containing $\left\{x_0\right\}\times C$ as desired.
\end{proof}
\begin{problem}
Let $X$ be a locally compact Hausdorff space and let $A$ be a subset with
the property that $A\cap K$ is closed for every compact $K$. Prove that $A$
is closed.
\end{problem}
\begin{proof}
Here's what I have so far:

We will try to show that $\bar A\subset A$. Let $x\in\bar A$. Then, for
every neighborhood $U$ of $x$, $U\cap A\neq\emptyset$. Now, since $X$ is
locally compact, there exists a neighborhood $V$ of $x$ such that $\bar V$
is compact and is a subset of $U$. Since $X$ is Hausdorff, $\bar V$ is
compact so $\bar V\cap A$ is closed.
\end{proof}
\begin{problem}
Let $X$ and $Y$ be path-connected and let $h\colon X\to Y$ be a continuous
function which induces the trivial homomorphism of fundamental groups. Let
$x_0,x_1\in X$ and let $f$ and $g$ be paths from $x_0$ to $x_1$. Prove that
$h\circ f$ and $h\circ g$ are homotopic.
\end{problem}
\begin{proof}
Consider the path-product $\gamma\coloneqq f*\bar g$. $\gamma$ is a loop
based at $x_0$ since $\gamma(0)=f(0)=x_0$ and $\gamma(1)=\bar g(2-1)=\bar
g(1)=x_0$. Thus, $[\gamma]\in\pi_1(X,x_0)$. Now, since
$h_*\colon\pi_1(X,x_0)\to\pi_1(Y,h(x_0))$ induces the trivial homomorphism,
i.e., $h(\gamma)\simeq_p e_{x_0}$, there exists a homotopy $H\colon
[0,1]\times[0,1]\to Y$ such that $H(s,0)=h\circ\gamma(s)$ and
$H(s,1)=e_{x_0}(s)$. Now, since $Y$ is path-connected, there exists a path
$\delta\colon[0,1]\to Y$ from $h(x_0)$ to $h(x_1)$.
\end{proof}
\begin{problem}
Let $X$ be the quotient space obtained from an $8$-sided polygonal region
$P$ by pasting its edges together according to the labelling scheme
$aabbcdc^{-1}d^{-1}$.
\begin{enumerate}[noitemsep,label=(\roman*)]
\item Calculate $H_1(X)$.
\item Assuming $X$ is homeomorphic to one of the standard surfaces in the
  classification theorem, which surface is it?
\end{enumerate}
\end{problem}
\begin{proof}
\end{proof}
\begin{problem}
Let $p\colon E\to B$ be a covering map with $B$ locally connected, and let
$x\in B$. Prove that $x$ has a neighborhood $W$ with the following
property: for every connected component $C$ of $p^{-1}(W)$, the map
$p\colon C\to W$ is a homeomorphism.
\end{problem}
\begin{proof}
\end{proof}

%%% Local Variables:
%%% mode: latex
%%% TeX-master: "../MA571-Quals"
%%% End:
