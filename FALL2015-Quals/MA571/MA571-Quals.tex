\def\documentauthor{Carlos Salinas}
\def\documenttitle{MA571: Qual Problems}
% \def\hwnum{1}
\def\shorttitle{MA571 Quals}
\def\coursename{MA571}
\def\documentsubject{point-set topology}
\def\authoremail{salinac@purdue.edu}

\documentclass[article,oneside,10pt]{memoir}
\usepackage{geometry}
\usepackage[dvipsnames]{xcolor}
\usepackage[
    breaklinks,
    bookmarks=true,
    colorlinks=true,
    pageanchor=false,
    linkcolor=black,
    anchorcolor=black,
    citecolor=black,
    filecolor=black,
    menucolor=black,
    runcolor=black,
    urlcolor=black,
    hyperindex=false,
    hyperfootnotes=true,
    pdftitle={\shorttitle},
    pdfauthor={\documentauthor},
    pdfkeywords={\documentsubject},
    pdfsubject={\coursename}
    ]{hyperref}

% Use symbols instead of numbers
\renewcommand*{\thefootnote}{\fnsymbol{footnote}}

%% Math
\usepackage{amsthm}
\usepackage{amssymb}
\usepackage{mathtools}
% \usepackage{unicode-math}

%% PDFTeX specific
\usepackage[mathcal]{euscript}
\usepackage{mathrsfs}

\usepackage[LAE,LFE,T2A,T1]{fontenc}
\usepackage[utf8]{inputenc}
\usepackage[farsi,french,german,spanish,russian,english]{babel}
\babeltags{fr=french,
           de=german,
           en=english,
           es=spanish,
           pa=farsi,
           ru=russian
           }
\def\spanishoptions{mexico}

\selectlanguage{english}

\newcommand{\textfa}[1]{\beginR\textpa{#1}\endR}

\usepackage{cmap}
\usepackage{CJKutf8}
\newcommand{\textkr}[1]{\begin{CJK}{UTF8}{mj}#1\end{CJK}}
\newcommand{\textjp}[1]{\begin{CJK}{UTF8}{min}#1\end{CJK}}
\newcommand{\textzh}[1]{\begin{CJK}{UTF8}{bsmi}#1\end{CJK}}

\usepackage{graphicx}
\graphicspath{{figures/}}

% Misc
\usepackage{microtype}
\usepackage{multicol}
\usepackage[inline]{enumitem}
\usepackage{listings}
\usepackage{mleftright}
\mleftright

%% Theorems and definitions
%% remove parentheses
\makeatletter
\def\thmhead@plain#1#2#3{%
  \thmname{#1}\thmnumber{\@ifnotempty{#1}{ }\@upn{#2}}%
  \thmnote{ {\the\thm@notefont#3}}}
\let\thmhead\thmhead@plain
\makeatother

\theoremstyle{plain}
\newtheorem{theorem}{Theorem}
\newtheorem{proposition}[theorem]{Proposition}
\newtheorem{corollary}[theorem]{Corollary}
\newtheorem{claim}[theorem]{Claim}
\newtheorem{lemma}[theorem]{Lemma}
\newtheorem{axiom}[theorem]{Axiom}

\newtheorem*{corollary*}{Corollary}
\newtheorem*{claim*}{Claim}
\newtheorem*{lemma*}{Lemma}
\newtheorem*{proposition*}{Proposition}
\newtheorem*{theorem*}{Theorem}

\theoremstyle{definition}
\newtheorem{definition}{Definition}
\newtheorem{example}{Examples}
\newtheorem{examples}[example]{Examples}
\newtheorem{exercise}{Exercise}[chapter]
\newtheorem{problem}[exercise]{Problem}

% \newcounter{problem}
% \newenvironment{problem}[1][]% environment name
% {% begin code
%   \stepcounter{problem}
%   \par\vspace{\baselineskip}\noindent
%   \ifx &#1&%
%   {\normalfont\Large\bfseries\scshape Problem~\hwnum.\theproblem}
%   \global\def\exercisename{Problem~\hwnum.\theproblem}%
%   \else
%   {\normalfont\Large\bfseries\scshape Problem~\hwnum.\theproblem~(#1)}
%   \global\def\exercisename{Problem~\hwnum.\theproblem(#1)}
%   \fi
%   \par\vspace{\baselineskip}%
%   \noindent\ignorespaces
% }%
% {% end code
%   \par\vspace{\baselineskip}%
%   \noindent\ignorespacesafterend
% }

%% Redefinitions & commands
% \newcommand\restr[2]{{% we make the whole thing an ordinary symbol
%   \left.\kern-\nulldelimiterspace % automatically resize the bar with \right
%   {#1} % the function
%   % \vphantom{\big|} % pretend it's a little taller at normal size
%   \right|{#2} % this is the delimiter
%   }}

\newcommand{\nsubset}{\ensuremath{\not\subset}}
\newcommand{\nsupset}{\ensuremath{\not\supset}}
\renewcommand\qedsymbol{\ensuremath{\null\hfill\blacksquare}}

%% Commands and operators
\DeclareMathOperator{\id}{id}
\DeclareMathOperator{\im}{im}
\DeclareMathOperator{\Int}{int}
\DeclareMathOperator{\Cl}{cl}

\newcommand{\clsr}[1]{\overbar{#1}}
\newcommand{\bbC}{\mathbb{C}}
\newcommand{\bbN}{\mathbb{N}}
\newcommand{\bbQ}{\mathbb{Q}}
\newcommand{\bbR}{\mathbb{R}}
\newcommand{\bbZ}{\mathbb{Z}}
\newcommand{\bfC}{\mathbf{C}}
\newcommand{\bfN}{\mathbf{N}}
\newcommand{\bfQ}{\mathbf{Q}}
\newcommand{\bfR}{\mathbf{R}}
\newcommand{\bfZ}{\mathbf{Z}}

\begin{document}
\author{\href{mailto:\authoremail}{\documentauthor}}
\title{\documenttitle}
\date{\today}
\maketitle
% \chapter{MA571 (Midterm 2015)}
\begin{problem}
Prove that a function to a product space is continuous if and only if its
components are.
\end{problem}
\begin{proof}
\end{proof}

\begin{problem}
Prove that a subspace is closed if and only if it contains all of its limit
points.
\end{problem}
\begin{proof}
\end{proof}

\begin{problem}
Prove that the projection maps for a product are open maps.
\end{problem}
\begin{proof}
\end{proof}

\begin{problem}
Prove that $\partial A=\emptyset$ if and only if $A$ is open and closed.
\end{problem}
\begin{proof}
\end{proof}

\begin{problem}
Prove that a metric space satisfies the 1st countability axiom.
\end{problem}
\begin{proof}
\end{proof}

\begin{problem}
Prove that $\bfR^\omega$ is not metrizable in the box topology.
\end{problem}
\begin{proof}
\end{proof}

\begin{problem}
Show that the diagonal map is not continuous in the box topology, but it is
in the product topology.
\end{problem}
\begin{proof}
\end{proof}

\begin{problem}
Prove the sequence lemma.
\end{problem}
\begin{proof}
\end{proof}

\begin{problem}
Give an example of a surjective map of spaces that is not a quotient map.
\end{problem}
\begin{proof}
\end{proof}

\begin{problem}
Prove that if $f_n$ is a sequence of functions $X\to\bfR$ considered as
elements of $X^{\bfR}$ with the product topology, then $f_n\to f$ if and
only if for each $x\in X$ the sequence $f_n(x)$ converges to the point
$f_n(x)$.
\end{problem}
\begin{proof}
\end{proof}

\begin{problem}
Prove that if $f_n$ is a sequence of functions $X\to\bfR$ considered as
elements of $X^{\bfR}$ with the topology induced by the uniform metric
$\bar\rho$, then $f_n\to f$ if and only if the sequence of functions
$f_n$ converges uniformly to the point $f$. (Recall that $f_n\colon X\to
Y$, with $Y$ a metric space, uniformly converges to $f$ if for any
$\varepsilon>0$ there exists an integer $N$ such that for all $n>N$ and
$x\in D$, $d_y(f_n(x),f(x))<\varepsilon$.)
\end{problem}
\begin{proof}
\end{proof}

\begin{problem}
Give an example of a surjective map of spaces that is not a quotient map.
\end{problem}
\begin{proof}
\end{proof}

\begin{problem}
\end{problem}
\begin{proof}
\end{proof}

\begin{problem}
\end{problem}
\begin{proof}
\end{proof}

\begin{problem}
\end{problem}
\begin{proof}
\end{proof}

\begin{problem}
\end{problem}
\begin{proof}
\end{proof}

\begin{problem}
\end{problem}
\begin{proof}
\end{proof}

\begin{problem}
\end{problem}
\begin{proof}
\end{proof}

\begin{problem}
\end{problem}
\begin{proof}
\end{proof}

\begin{problem}
\end{problem}
\begin{proof}
\end{proof}

\begin{problem}
\end{problem}
\begin{proof}
\end{proof}

\begin{problem}
\end{problem}
\begin{proof}
\end{proof}


%%% Local Variables:
%%% mode: latex
%%% TeX-master: "../MA571-Quals"
%%% End:

% \chapter{MA571 (Final 2015)}
\begin{problem}
\end{problem}
\begin{proof}
\end{proof}

\begin{problem}
\end{problem}
\begin{proof}
\end{proof}

\begin{problem}
\end{problem}
\begin{proof}
\end{proof}

\begin{problem}
\end{problem}
\begin{proof}
\end{proof}

\begin{problem}
\end{problem}
\begin{proof}
\end{proof}

\begin{problem}
\end{problem}
\begin{proof}
\end{proof}

\begin{problem}
\end{problem}
\begin{proof}
\end{proof}

\begin{problem}
\end{problem}
\begin{proof}
\end{proof}

\begin{problem}
\end{problem}
\begin{proof}
\end{proof}

\begin{problem}
\end{problem}
\begin{proof}
\end{proof}

\begin{problem}
\end{problem}
\begin{proof}
\end{proof}

\begin{problem}
\end{problem}
\begin{proof}
\end{proof}


%%% Local Variables:
%%% mode: latex
%%% TeX-master: "../MA571-Quals"
%%% End:

\chapter{Covering Space Problems}
Compiled from Prof.\,McClure's old quals.
\begin{problem}

\end{problem}
\begin{proof}
\end{proof}

\begin{problem}
\end{problem}
\begin{proof}
\end{proof}

\begin{problem}
\end{problem}
\begin{proof}
\end{proof}

\begin{problem}
\end{problem}
\begin{proof}
\end{proof}

\begin{problem}
\end{problem}
\begin{proof}
\end{proof}

\begin{problem}
\end{problem}
\begin{proof}
\end{proof}

\begin{problem}
\end{problem}
\begin{proof}
\end{proof}
\section{Kyle's Stuff}
\begin{problem}[No.\,5]
Let $X$ be a topological space and let $x_0\in X$. Let $U$ and $V$ be open
sets containing $x_0$, and suppose that the hypotheses of the seifert--van
Kampen theorem are satisfied. Let $i_1\colon U\cap V\to U$, $i_2\colon
U\cap V\to V$, $j_1\colon U\to X$, and $j_2\colon V\to X$ be the inclusion
maps. Suppose that $\left(i_1\right)_*\colon \pi_1(U\cap
V,x_0)\to\pi_1(U,x_0)$ is onto. Prove, using the Seifert--van Kampen
theorem, that $\left( j_2 \right)_*\colon\pi_1(V,x_0)\to\pi_1(X,x_0)$ is
onto.
\end{problem}
\begin{proof}
We use the classical Seifert--van Kampen theorem (Theorem 70.2). Suppose
$\left(i_1\right)_*\colon\pi_1(U\cap V,x_0)\to\pi_1(U,x_0)$ is onto. Then
for every element $\gamma\in\pi_1(U,x_0)$,
$\gamma=\left(i_1\right)_*(\gamma')$ for some element
$\gamma'\in\pi_1(U\cap V,x_0)$. Now, let $\gamma''\in\pi_1(X,x_0)$. By the
classical Seifert--van Kapmen theorem, the map
\[
j\colon\pi_1(U,x_0)*\pi_1(V,x_0)\longrightarrow\pi_1(X,x_0)
\]
is surjective and its kernel is the leans normal subgroup $N$ of the free
product that contanis all elements represented by words of the form
$(i_1(g)^{-1},i_2(g))$.
\end{proof}
\begin{problem}[No.\,6]
As in 5., but instead suppose that $\left( i_1 \right)_*\colon\pi_1(U\cap
V,x_0)\to\pi_1(X,x_0)$ is an isomorphism. Prove, using the Seifert--van
Kampen theorem, that there is a homomorphism
$\Phi\colon\pi_1(X,x_0)\to\pi_1(V,x_0)$ for which
$\Phi\circ\left(j_2\right)_*$ is the identity map of $\pi_1(V,x_0)$.
\end{problem}
\begin{proof}

\end{proof}

%%% Local Variables:
%%% mode: latex
%%% TeX-master: "../MA571-Quals"
%%% End:

\chapter{August 2014}
\begin{problem}
Let $X$ be a topological space, let $A$ be a subset of $X$, and let $U$ be
an open subset of $X$. Prove that $U\cap \bar A\subset\overline{U\cap A}$.
\end{problem}
\begin{proof}
Let $x\in U\cap\bar A$. Then $x\in U$ and $x\in\bar A$. This means that,
since $U$ is open, by Lemma C there exist an open neighborhood $V$ of $x$
such that $V\subset U$. Moreover, since $x\in\bar A$, $V'\cap
A\neq\emptyset$ for every open neighborhood $V'$ of $x$. In particular,
$V\cap A\neq\emptyset$. Thus, we have $V\cap U\neq\emptyset$ and $V\cap
A\neq\emptyset$ so $V\cap(U\cap A)\neq\emptyset$.
\end{proof}

\begin{problem}
\end{problem}
\begin{proof}
\end{proof}

\begin{problem}
\end{problem}
\begin{proof}
\end{proof}

\begin{problem}
\end{problem}
\begin{proof}
\end{proof}

\begin{problem}
\end{problem}
\begin{proof}
\end{proof}

\begin{problem}
\end{problem}
\begin{proof}
\end{proof}

\begin{problem}
\end{problem}
\begin{proof}
\end{proof}

%%% Local Variables:
%%% mode: latex
%%% TeX-master: "../MA571-Quals"
%%% End:

\end{document}

%%% Local Variables:
%%% mode: latex
%%% TeX-master: t
%%% End:
