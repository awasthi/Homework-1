\chapter{Qualifying Exam, January 2000}
\begin{problem}
Find all groups of order $7\cdot 11^3$ which have a cyclic subgroup of
order $11^3$.
\end{problem}
\begin{proof}
Suppose $G$ is a group of order $7\cdot 11^3$. By Sylow's theorem,
$n_{11}\equiv 1\pmod{11}$ and $n_{11}\mid 7$, thus $n_{11}=1$ and we see
that $G$ must have a unique, therefore normal, Sylow $11$-subgroup $P$ of
order $11^3$. Also by Sylow's theorem, we see than $n_7=1$ or $11^3=1331$
(what an outrageous number!!!).

If $n_7=1$, again the Sylow $7$-subgroup $Q$ is unique hence, normal in $G$
and we must have $PQ=QP=G$ (since $P\cap Q=\{e\}$ and
$|PQ|=|P||Q|/|P\cap Q|=11^3\cdot 7/1=|G|$). Thus,
$G\cong Z_7\times Z_{11^3}$.

Otherwise, $n_7=11^3$. Thus, there are $6\cdot 11^3+1$ elements of order
$7$ plus the identity plus $11^3-1$ elements in $P$. Thus, there are a
total of $6\cdot 11^3+1+11^3-1=7\cdot 11^3$ elements of order $7$, in $Q$,
plus the identity. No contradiction here. But we still have $P\cap Q=\{e\}$
for any $Q\in\Syl_7(G)$. Therefore, I suspect that the only other
(nonabelian) group that has a cyclic subgroup of order $11^3$ must be the
semidirect product $Z_7\rtimes Z_{11^3}$.

This is what \textkr{성준} had to say about the matter:

Suppose $Q$ is a group of order $7$ and $P$ is a cyclic group of order
$11^3$. If $\varphi\colon Q\to\Aut(P)$ is a homomorphism, then
$\varphi(Q)\mid 7$ so $\varphi(Q)=1$ or $\varphi(Q)=7$. But if
$\psi\in\Aut(P)$, then $\psi$ must send a generator $g$ of $P$ to another
generator of $P$. Since there are only
$\varphi(11^3)=11^3-11^2=1331-121=1220$ which is not divisible by $7$. Thus
$\varphi$ can only be the trivial homomorphism and $Q\rtimes_\varphi P\cong
Z_7\times Z_{11^3}$.
\end{proof}

\begin{problem}
Let $R$ be a ring with identity $1$ and consider the following two
conditions:
\begin{center}
\begin{enumerate}[label=(\roman*)]
\item If $a,b\in R$ and $ab=0$, then $ba=0$;
\item If $a,b\in R$ and $ab=1$, then $ba=1$;
\end{enumerate}
\end{center}
\begin{enumerate}[label=(\alph*)]
\item Show that (i) implies (ii).
\item Show by example that (ii) does not imply (i).
\end{enumerate}
\end{problem}
\begin{proof}
(i) $\implies$ (ii) Let $a,b\in R$ such that $ab=1$. Consider the product
of elements of $R$
\[
b(ab-1)a=baba-ba=0.
\]
Then, by (i), we have $ba(ba-1)=0$ so
\[
a(ba-1)b=(ba-1)ab=ba-1=0
\]
so $ba=1$.
\\\\
(ii) $\nimplies$ (ii) Consider the ring of $2\times 2$ matrices over
$\bfR$, $M(2,\bfR)$. Then, for any matrices $A,B\in M(2,\bfR)$ such that
$AB=1$ we have $BA=1$. However,
\[
\begin{bmatrix}
0&0\\
0&1
\end{bmatrix}
\begin{bmatrix}
1&1\\
0&0
\end{bmatrix}
=
\begin{bmatrix}
0&0\\
0&0
\end{bmatrix},
\]
but
\[
\begin{bmatrix}
1&1\\
0&0
\end{bmatrix}
\begin{bmatrix}
0&0\\
0&1
\end{bmatrix}
=
\begin{bmatrix}
0&1\\
0&0
\end{bmatrix}
\neq
\begin{bmatrix}
0&0\\
0&0
\end{bmatrix}.
\qedhere
\]
\end{proof}

\begin{problem}
Let $F$ be a field. Suppose that $E/F$ is a Galois extension, and that
$L/F$ is an algebraic extension with $L\cap E=F$. Let $EL$ be the composite
field, i.e., the subfield of an algebraic closer $\bar F$ of $F$ generated
by $E$ and $L$.
\begin{enumerate}[label=(\alph*)]
\item Show $EL/L$ is a Galois extension.
\item Show that there is an injective homomorphism
\[\varphi\colon\Gal(EL/L)\hookrightarrow\Gal(E/F).\]
Find the fixed field of the image of $\varphi$.
\item Show that $[EL:L]=[E:F]$.
\item Give an example to show that the conclusion of (c) is false if we do
  not assume that $E/F$ is Galois.
\end{enumerate}
\end{problem}
\begin{proof}
This is standard and is found in Dummit \& Foote\footnote{I'm not saying that I
remembered how to do it, but I can start the proof at the very least.}

(a) Since $E/F$ is Galois, $E$ is the splitting field of some separable
polynomial $f\in F[X]$. Hence, $EL/L$ is the splitting field of $f$ viewed
as a polynomial in $L[X]$ (since $F\subset L$, we can just look at the
embedding $F[X]\hookrightarrow L[X]$). Thus, $EL/L$ is Galois.
\\\\
(b) Define the map $\varphi\colon\Gal(EL/L)\to\Gal(E/F)$ to be
$\varphi(\sigma)\coloneqq\left.\sigma\right|_{E}$, i.e., $\varphi$ is the
restriction of $\sigma\in\Gal(EL/L)$ to $E$. Since $E/F$ is Galois, every
embedding of $E$ fixing $F$ is an automorphism of $E$ hence, $\varphi$ is
well defined and because of the properties of restriction, trivially
$\varphi$ is a homomorphism, i.e., for any $\sigma_1,\sigma_2\in\Gal(EL/L)$
we have
\[
\varphi(\sigma_1\circ\sigma_2)
=\left.\sigma_1\circ\sigma_2\right|_K
=\left(\left.\sigma_1\right|_K\right)\circ\left(\left.\sigma_2\right|_K\right)
=\varphi(\sigma_1)\circ\varphi(\sigma_2).
\]

Now, let us show $\varphi$ is injective. To that end, we will show that
$\ker\varphi=\{\id_{EL}\}$. Let $\sigma\in\ker\varphi$. Then
$\left.\sigma\right|_K=\id_K$. But $\left.\sigma\right|_L=\id_L$. Thus,
$\sigma=\id_{EL}$. Hence, $\varphi$ is injective.
\\\\
(c)
\\\\
(d) Consider the extensions $E\coloneqq\bfQ\left(\sqrt[3]{2}\right)$ and
$L\coloneqq\bfQ\left(\omega\sqrt[3]{2}\right)$ where $\omega$ is a primitive $3$rd
root of unity, i.e., satisfies $\omega^2+\omega+1=0$. Then, the minimal
polynomial for both extensions over $\bfQ$ is $X^3-2$. In particular,
$[E:F]=3$. However, the composite
\[
EL=\bfQ\bigl(\omega,\sqrt[3]{2}\bigr)
\]
is a degree $2$ extension over $L$ (since the minimal polynomial for
$\omega$ over $L$ is $X^2+X+1$). Thus, $[EL:L]=2\neq 3=[E:F]$ since neither
the extension $E$ or $L$ is Galois.
\end{proof}

\begin{problem}
Let $G$ be a finite group. Let $p$ be a prime and suppose that $|G|=p^km$,
with $k\geq 1$ and $p\nmid m$. Let $X$ be the collection of all subsets of
$G$ of order $p^k$. Then $G$ acts on $X$ by left multiplication, i.e.,
$g\cdot A=\left\{\,ga\;\middle|\;a\in A\,\right\}$. For $A\in X$< denote by
$H_A$ the stabilizer in $G$ of $A$. Show that $|H_A|\mid p^k$.
\end{problem}
\begin{proof}
\end{proof}

\begin{problem}
Let $R=\bfZ+X\bfQ[X]\subset\bfQ[X]$ be the ring consisting of polynomials
with rational coefficients whose constant term is an integer.
\begin{enumerate}[label=(\alph*)]
\item Prove that $R$ is an integral domain, with units $1$ and $-1$.
\item Show that $X$ is not an irreducible element of $R$.
\item Let $(X)\coloneqq RX$ be the ideal of $R$ generated by $X$. Describe
  $R/(X)$ and show that $R/(X)$ is not an integral domain. What can you
  conclude about $X$?
\end{enumerate}
\end{problem}
\begin{proof}
(a) Let $f,g\in R$ such that $fg=0$. Then
\[
f(X)g(X)=(a_n/b_n)X^n+(a_{n-1}/b_{n-1})X^{n-1}+\cdots+a_0b_0=0
\]
where the coefficients $a_i/b_i$ for $1\leq i\leq n$ are sums of
products of coefficients of $f$ and $g$. Since the right handside of the
equation above has degree $0$, each of the coefficients $a_i/b_i=0$ and
since $\bfZ$ is a Euclidean domain, $a_0b_0=0$ implies that either $a_0=0$
or $b_0=0$. Hence, either $f=0$ or $g=0$ (that is, WLOG, supposing $b_0=0$
then the next smallest coefficient of $g$, say $b_1'=0$ for otherwise
$b_1'a_0X\neq 0$ is in the product). Thus, $R$ is an integral domain.
\\\\
(b) Consider the decomposition $X=2((1/2)X)$. Thus, $X$ is not
irreducible.
\\\\
(c) Consider the ideal generated by $X$, i.e., the set $(X)$ of all
polynomials in $\bfQ$ of degree $\geq 1$ such that the coefficient of the
degree $1$ part of the polynomial is in $\bfZ$, i.e., all $f$ of the form
$(a_n/b_n)X^n+\cdots+a_1X$ where $a_i/b_i\in\bfQ$ and $a_1\in\bfZ$.

Now, consider the quotient $R/(X)$. Under this quotient, every polynomial
is of degree at most $1$ with the coefficient being a number between $0$
and $1$. Let $\bar f,\bar g\in R/(X)$ where the $X$ coefficient of $f$ is
$1/2$ and the constant coefficient of $g$ is $2$. Then $\bar f\neq\bar 0$ and
$\bar g\neq\bar 0$, but $\bar f\bar g=\bar 0$.
\end{proof}

%%% Local Variables:
%%% mode: latex
%%% TeX-master: "../MA553-Quals"
%%% End:
