\chapter{Qualifying Exam, January 2000}
\begin{problem}
Find all groups of order $7\cdot 11^3$ which have a cyclic subgroup of
order $11^3$.
\end{problem}
\begin{proof}
Suppose $G$ is a group of order $7\cdot 11^3$. By Sylow's theorem,
$n_{11}\equiv 1\pmod{11}$ and $n_{11}\mid 7$, thus $n_{11}=1$ and we see
that $G$ must have a unique, therefore normal, Sylow $11$-subgroup $P$ of
order $11^3$. Also by Sylow's theorem, we see than $n_7=1$ or $11^3=1331$
(what an outrageous number!!!).

If $n_7=1$, again the Sylow $7$-subgroup $Q$ is unique hence, normal in $G$
and we must have $PQ=QP=G$ (since $P\cap Q=\{e\}$ and
$|PQ|=|P||Q|/|P\cap Q|=11^3\cdot 7/1=|G|$). Thus,
$G\cong Z_7\times Z_{11^3}$.

Otherwise, $n_7=11^3$. Thus, there are $6\cdot 11^3+1$ elements of order
$7$ plus the identity plus $11^3-1$ elements in $P$. Thus, there are a
total of $6\cdot 11^3+1+11^3-1=7\cdot 11^3$ elements of order $7$, in $Q$,
plus the identity. No contradiction here. But we still have $P\cap Q=\{e\}$
for any $Q\in\Syl_7(G)$. Therefore, I suspect that the only other
(nonabelian) group that has a cyclic subgroup of order $11^3$ must be the
semidirect product $Z_7\rtimes Z_{11^3}$.
\end{proof}

\begin{problem}
Let $R$ be a ring with identity $1$ and consider the following two
conditions:
\begin{center}
\begin{enumerate}[label=(\roman*)]
\item If $a,b\in R$ and $ab=0$, then $ba=0$;
\item If $a,b\in R$ and $ab=1$, then $ba=1$;
\end{enumerate}
\end{center}
\begin{enumerate}[label=(\alph*)]
\item Show that (i) implies (ii).
\item Show by example that (ii) does not imply (i).
\end{enumerate}
\end{problem}
\begin{proof}

(ii) $\nimplies$ (ii)
\end{proof}

\begin{problem}
Let $F$ be a field. Suppose that $E/F$ is a Galois extension, and that
$L/F$ is an algebraic extension with $L\cap E=F$. Let $EL$ be the composite
field, i.e., the subfield of an algebraic closer $\bar F$ of $F$ generated
by $E$ and $L$.
\begin{enumerate}[label=(\alph*)]
\item Show $EL/L$ is a Galois extension.
\item Show that there is an injective homomorphism
\[\varphi\colon\Gal(EL/L)\hookrightarrow\Gal(E/F).\]
Find the fixed field of the image of $\varphi$.
\item Show that $[EL:L]=[E:F]$.
\item Give an example to show that the conclusion of (c) is false if we do
  not assume that $E/F$ is Galois.
\end{enumerate}
\end{problem}
\begin{proof}
\end{proof}

\begin{problem}
Let $G$ be a finite group. Let $p$ be a prime and suppose that $|G|=p^km$,
with $k\geq 1$ and $p\nmid m$. Let $X$ be the collection of all subsets of
$G$ of order $p^k$. Then $G$ acts on $X$ by left multiplication, i.e.,
$g\cdot A=\left\{\,ga\;\middle|\;a\in A\,\right\}$. For $A\in X$< denote by
$H_A$ the stabilizer in $G$ of $A$. Show that $|H_A|\mid p^k$.
\end{problem}
\begin{proof}
\end{proof}

\begin{problem}
Let $R=\bfZ+X\bfQ[X]\subset\bfQ[X]$ be the ring consisting of polynomials
with rational coefficients whose constant term is an integer.
\begin{enumerate}[label=(\alph*)]
\item Prove that $R$ is an integral domain, with units $1$ and $-1$.
\item Show that $x$ is not an irreducible element of $R$.
\item Let $(X)\coloneqq Rx$ be the ideal of $R$ generated by $X$. Describe
  $R/(X)$ and show that $R/(X)$ is not an integral domain. What can you
  conclude about $X$?
\end{enumerate}
\end{problem}
\begin{proof}
\end{proof}

%%% Local Variables:
%%% mode: latex
%%% TeX-master: "../MA553-Quals"
%%% End:
