\chapter{MA 553: Final, Fall 2015}
\begin{problem}
Let $G$ be a finite non-Abelian group, and let $Z(G)$ be the center of
$G$. Prove that $|Z(G)|\leq |G|/4$.
\end{problem}
\begin{proof}
Seeking a contradiction, suppose $4>[G:Z(G)]$. Since $Z(G)\lhd G$, we have
$G/Z(G)$ is a group of order $1$, $2$, or $3$. Thus, $G/Z(G)\cong Z_1$,
$Z_2$, or $Z_3$ all of which are cyclic. This implies that $G$ is
Abelian. This is a contradiction.
\end{proof}

\begin{problem}
Let
\[
G=\SL_2(\bbZ/(5))\coloneqq
\left\{\,\begin{bmatrix}a&b\\c&d\end{bmatrix}\;\middle|\;
\text{$a,b,c,d\in\bbZ/(5)$, and $ad-bc\equiv 1\pmod{5}$}\,\right\}.
\]
\begin{enumerate}[label=(\alph*)]
\item Show $|G|=120$.
\item Show
\[
N\coloneqq\left\{\,\begin{bmatrix}1&b\\0&1\end{bmatrix}\;\middle|\;b\in\bbZ/(5)\,\right\}
\]
is a Sylow $5$-subgroup of $G$.
\item Find the number of Sylow $5$-subgroups of $G$.
\end{enumerate}
\end{problem}
\begin{proof}
(a) Let us count the number of elements in $\SL_2(\bbZ/(5))$.
\\\\
(b)
\\\\
(b)
\end{proof}

\begin{problem}
Suppose $R$ is a UFD and $F$ is the quotient field of $R$. Let $f(X)\in
R[X]$ and suppose $f(X)$ factors as a product of lower degree polynomials
in $F[X]$. Show $f(X)$ factors as a product of lower degree polynomials in
$R[X]$.
\end{problem}
\begin{proof}
\end{proof}

\begin{problem}
Let $R$ be a commutative ring. Recall an element $a\in R$ is
\emph{nilpotent} if $r^n=0$ for some $n\geq 1$. Let $I=\left\{\,a\in
  R\;\middle|\;\text{$a$ is nilpotent}\,\right\}$.
\begin{enumerate}[label=(\alph*)]
\item Show $I$ is an ideal. (\emph{Hint:} To show $I$ is an additive
  subgroup, show if $x,y\in I$ there is an $N>0$ so that $(x-y)^N=0$ using
  the binomial expansion of $(x-y)^N$.)
\item Show $I$ is contained in any prime ideal of $R$.
\end{enumerate}
\end{problem}
\begin{proof}
\end{proof}

\begin{problem}
Let $\alpha\in\bbC$ be algebraic over $\bbQ$, and let $f(X)\in\bbQ[x]$ be
its minimal polynomial. Let $\sqrt{\alpha}$ be a square root of $\alpha$,
and let $g(X)\in\bbQ[X]$ be its minimal polynomial.
\begin{enumerate}[label=(\alph*)]
\item Show $\deg f(X)$ divides $\deg g(X)$.
\item Show $\sqrt{\alpha}\in\bbQ(\alpha)$ if and only if $f(X^2)$ is
  reducible in $\bbQ[X]$.
\end{enumerate}
\end{problem}
\begin{proof}
\end{proof}

\begin{problem}
Let $f(X)=X^6+3\in\bbQ[X]$.
\begin{enumerate}[label=(\alph*)]
\item Let $\alpha$ be a root of $f(X)$. Prove $(\alpha^3+1)/2$ is a
  primitive 6th root of unity.
\item Determine the Galois group of $f(X)$ over $\bbQ$.
\end{enumerate}
\end{problem}
\begin{proof}
(a) Put $\zeta_6\coloneqq(\alpha^3+1)/2$.
\\\\
(b)
\end{proof}

\begin{problem}
Let $R\coloneqq(\bbZ/(3))[X]$. Consider the ideals $I_1\coloneqq(X^2+1)$,
and $I_2\coloneqq(X^2+X+2)$. For $i=1,2$ we set $F_i=R/I_i$.
\begin{enumerate}[label=(\alph*)]
\item Show $F_1$ and $F_2$ are fields.
\item Are $F_1$ and $F_2$ isomorphic? If not, why not, and if so give an
  isomorphism from $F_1$ to $F_2$.
\end{enumerate}
\end{problem}
\begin{proof}
\end{proof}

\begin{problem}
Suppose $F$ is a field, $K=F(\alpha)$ is a Galois extension, with cyclic
Galois group generated by $\sigma(\alpha)\coloneqq\alpha+1$. Show that
$\Ch(K)=p\neq 0$, and $\alpha^p-\alpha\in F$.
\end{problem}
\begin{proof}
\end{proof}

%%% Local Variables:
%%% mode: latex
%%% TeX-master: "../MA553-Quals"
%%% End:
