\chapter{MA 553: Final, Fall 2015}
\begin{problem}
Let $G$ be a finite non-Abelian group, and let $Z(G)$ be the center of
$G$. Prove that $|Z(G)|\leq |G|/4$.
\end{problem}
\begin{proof}
Seeking a contradiction, suppose $4>[G:Z(G)]$. Since $Z(G)\lhd G$, we have
$G/Z(G)$ is a group of order $1$, $2$, or $3$. Thus, $G/Z(G)\cong Z_1$,
$Z_2$, or $Z_3$ all of which are cyclic. This implies that $G$ is
Abelian. This is a contradiction.
\end{proof}

\begin{problem}
Let
\[
G=\SL_2(\bfZ/(5))\coloneqq
\left\{\,\begin{bmatrix}a&b\\c&d\end{bmatrix}\;\middle|\;
\text{$a,b,c,d\in\bfZ/(5)$, and $ad-bc\equiv 1\pmod{5}$}\,\right\}.
\]
\begin{enumerate}[label=(\alph*)]
\item Show $|G|=120$.
\item Show
  $N\coloneqq\left\{\,\left[\begin{smallmatrix}1&b\\0&1\end{smallmatrix}\right]\;\middle|\;b\in\bfZ/(5)\,\right\}$
  is a Sylow $5$-subgroup of $G$.
\item Find the number of Sylow $5$-subgroups of $G$.
\end{enumerate}
\end{problem}
\begin{proof}
(a) We'll do a case by case analysis. First, suppose that $a=0$. Then
$bc\equiv 1\pmod{5}$ and we have elements of the form
\[
\begin{bmatrix}
0&b\\
c&d
\end{bmatrix}.
\]
Hence, we have $5$ choices for $c$ and $4$ choices for $b$ ($d$ is
determined by the equivalence $bd\equiv 1\pmod{5}$). So there are $5\cdot
4=20$ elements with $a=0$.

Now, suppose $a\neq 0$. Then $d\equiv(1+bc)a^{-1}\pmod{5}$. Hence there are
$4$ choices for $a$ and $5$ choices for both $b$ and $c$. Hence, there are
$4\cdot 5\cdot 5=100$ elements of the form
\[
\begin{bmatrix}
a&b\\
c&(1+bc)a^{-1}
\end{bmatrix}.
\]
with $a\neq 0$. Tallying up this total, we have $20+100=120$, as was to be
shown.
\\\\
% (b) It suffices to show that the order of $N$ is $5$ since $1$ is the
% largest exponent of $5$ dividing $120=2^3\cdot 3\cdot 5$. But this is
% clear, since $N$ must satisfy $1-b\cdot 0\equiv 1\equiv 1\pmod{5}$ which is
% true for any $b\in\bfZ$. Hence, there are $5$ elements in $N$. Thus, $N$ is
% a Sylow $5$-subgroup.
(b) First, note that $|G|=120=2^3\cdot 3\cdot 5$ and since $5$ is the smallest
power of $5$ dividing $|G|$, it suffices to show that $|N|=5$. Now, note
that
\[
\begin{bmatrix}
1&1\\
0&1
\end{bmatrix}^b
=
\begin{bmatrix}
1&b\\
0&1
\end{bmatrix}
\]
hence, $N$ is generated by
$g\coloneqq\left[\begin{smallmatrix}1&1\\0&1\end{smallmatrix}\right]$. Moreover,
\[
g^5=
\begin{bmatrix}
1&1\\
0&1
\end{bmatrix}^5
=
\begin{bmatrix}
1&5\\
0&1
\end{bmatrix}
=
\begin{bmatrix}
1&0\\
0&1
\end{bmatrix}.
\]
Hence, $|N|=|g|=5$ so $N\in\Syl_5(G)$.
\\\\
(c) By Sylow's theorem, there are $n_5=1$ or $6$. We will show that $N$ is
not normal in $\SL_2(\bfZ/(5))$ so that $n_5\neq 1$. Let
$\left[\begin{smallmatrix}a&b\\c&d\end{smallmatrix}\right]\in\SL_2(\bfZ/(5))$. Then,
for any matrix in $N$ we have
\[
\begin{bmatrix}
a&b\\c&d
\end{bmatrix}
\begin{bmatrix}
1&1\\0&1
\end{bmatrix}
\begin{bmatrix}
d&-b\\
-c&a
\end{bmatrix}
=
\begin{bmatrix}
1-ac&a^2\\-bc&1+ba
\end{bmatrix}
\]
is in $N$ if and only if $ac=ba=0$ and $-bc=0$. But $ad\equiv
1+bc\pmod{5}$. Implies $bc=0$ so $b=0$ or $c=0$ so either $b=0$ and $c=0$
or $c=0$. The former implies that $ad=1\equiv\pmod{5}$ so $a=d=1$. This
would imply that
$\left[\begin{smallmatrix}a&b\\c&d\end{smallmatrix}\right]$. Thus,
$N\nlhd\SL_2(\bfZ/(5))$ so $n_5=6$.
\end{proof}

\begin{problem}
Suppose $R$ is a UFD and $F$ is the quotient field of $R$. Let $f(X)\in
R[X]$ and suppose $f(X)$ factors as a product of lower degree polynomials
in $F[X]$. Show $f(X)$ factors as a product of lower degree polynomials in
$R[X]$.
\end{problem}
\begin{proof}
This is an important result called \emph{\textde{Gauß}'s lemma} and is proven in
Dummit and Foote more or less as follows:

Suppose $f(X)$ factors as $f(X)=g(X)h(X)$ for polynomials $g,h\in F[X]$
with $\deg(g),\deg(h)<\deg(f)$. Then each coefficient $\left\{a_i\right\}$,
$\left\{b_i\right\}$ of $g$ and $h$, respectively, is in $F$. Thus,
clearing denominators, we have $df(X)=g'(X)h'(X)$ for $g'(X),h'(X)\in
R[X]$. If $d$ is a unit in $R$ we are done since
$f(X)=d^{-1}df(X)=d^{-1}g'(X)h'(X)$.

Suppose $d$ is not a unit. Then, since $R$ is a UFD, we may write $d$ as
the product $d=d_1\cdots d_n$ of irreducible elements $d_i\in R$. Since
$d_1$ is irreducible and $R$ is a UFD, then $d_1$ is prime so the ideal
generated by $d_1$ is prime. Thus, $\left(R/(d_1)\right)[X]$ is a domain
and
\[
\bar 0=\overline{df(X)}=
{\bar d}\cdot \overline{f(X)}=\overline{g'(X)h'(X)}=
\overline{g'(X)}\cdot\overline{h'(X)}.
\]
Thus, either $\overline{g'(X)}=\bar 0$ or $\overline{h'(X)}=\bar 0$ since
$\left(R/(d_1)\right)[X]$ is a domain. Without loss of generality, suppose
$\overline{g'(X)}=0$. Then, $(1/d_1)g'(X)\in R[X]$ so, dividing over
$F$, we have $(d_2\cdots d_n)f(X)=\left((1/d_1)g'(X)\right)h'(X)$ in
$R[X]$. Proceeding recursively in this fashion until, we may arrive at
$f(X)=G(X)H(X)$ where $G(X),H(X)\in R[X]$. Since we reduced by elements in
the subring $R$, $\deg(G)=\deg(g)$ and $\deg(H)=\deg(h)$ so that $f(X)$
factors as a product of polynomials of lower degree in $R[X]$, as desired.
\end{proof}

\begin{problem}
Let $R$ be a commutative ring. Recall an element $a\in R$ is
\emph{nilpotent} if $r^n=0$ for some $n\geq 1$. Let $I=\left\{\,a\in
  R\;\middle|\;\text{$a$ is nilpotent}\,\right\}$.
\begin{enumerate}[label=(\alph*)]
\item Show $I$ is an ideal. (\emph{Hint:} To show $I$ is an additive
  subgroup, show if $x,y\in I$ there is an $N>0$ so that $(x-y)^N=0$ using
  the binomial expansion of $(x-y)^N$.)
\item Show $I$ is contained in any prime ideal of $R$.
\end{enumerate}
\end{problem}
\begin{proof}
(a) In fact, one can show that $I=\Nil(R)=\bigcap_{\frakp\in\Spec(R)}\frakp$,
i.e., $I$ is the intersection of all prime ideals in $R$ hence, an ideal.

First, we show that $R$ is multiplicatively closed. Let $r\in R$ and $a\in
I$. Then $(ar)^n=a^nr^n$ since $R$ is commutative. But $r^n=0$, so
$(ar)^n=a^n\cdot 0=0$. Thus $ar\in I$.

Next, we show that it is additively closed. Suppose $a,b\in I$. Then
$a^m=0$ and $b^n=0$ for some positive integer $m$ and $n$. Suppose, without
loss of generality, that $n\geq m$. Let $N=n+m$. Then
\begin{align*}
(a+b)^N&=(a+b)^{n+m}\\
       &=\sum_{i=1}^{n+m}\tbinom{n+m}{i}a^ib^{n+m-i}.
\end{align*}
Now, note that if $k\geq n$, $x^k=0$ so $\binom{n+m}{k}a^kb^{n+m-k}=0$. On
the other hand, if $k<n$, $n+m-k>m$ so $b^{n+m-k}=0$ so
$\binom{n+m}{k}a^kb^{n+m-k}=0$. In either case, we see that
$\binom{n+m}{k}a^kb^{n+m-k}=0$ so $(a+b)^N=0$. Thus, $a+b\in I$. Hence,
$I$ is an ideal.
\\\\
(b) Let $\frakp$ be a maximal ideal of $R$. Now, since $\frakp$ is an ideal
of $R$, $0\in R$. Moreover, for any $a\in I$, $a^n=0$ for some positive
integer $n$. Thus, $a^n=0\in\frakp$. But $\frakp$ is a prime ideal. Thus,
$a\in\frakp$ or $a^{n-1}\in\frakp$. If the former, we are done. In the
later, $a^{n-1}\in\frakp$ so $a\in\frakp$ or $a^{n-2}\in\frakp$. Proceeding
recursively in this manner, we have $a\in\frakp$. Thus, $I\subset\frakp$,
as desired.
\end{proof}

\begin{problem}
Let $\alpha\in\bfC$ be algebraic over $\bfQ$, and let $f(X)\in\bfQ[x]$ be
its minimal polynomial. Let $\sqrt{\alpha}$ be a square root of $\alpha$,
and let $g(X)\in\bfQ[X]$ be its minimal polynomial.
\begin{enumerate}[label=(\alph*)]
\item Show $\deg f(X)$ divides $\deg g(X)$.
\item Show $\sqrt{\alpha}\in\bfQ(\alpha)$ if and only if $f(X^2)$ is
  reducible in $\bfQ[X]$.
\end{enumerate}
\end{problem}
\begin{proof}
(a) This follows directly from the tower of fields theorem. Let $\bfQ(f)$
denote the splitting field of $f$. Then, $\alpha\in\bfQ(f)$ so that
$\bfQ(g)\supset\bfQ(f)$. Thus, we have
\[
\left[\bfQ(g):\bfQ\right]=\left[\bfQ(g):\bfQ(f)\right]\left[\bfQ(f):\bfQ\right]=k\cdot\deg(f)
\]
Thus, $\deg(f)\mid\deg(g)$.
\\\\
(b) $\implies$ Suppose that $\sqrt{\alpha}\in\bfQ(\alpha)$. Then
$f({\sqrt{\alpha}}^2)=f(\alpha)=0$ hence, $f(X^2)$ has a root in $\bbQ$
hence, is reducible.

$\impliedby$ Conversely, suppose that $f(X^2)$ is reducible. Then, we may
write $f(X^2)=\prod_{i=1}^k f_i(X)$ where $f_i\in\bfQ[X]$ is
irreducible. Now, each of these factors, $f_i$, have degree less than $2n$
where $n\coloneqq\deg\left(f(X^2)\right)$. Suppose
\[
f_i(X)=X^k+a_{k-1}X^{k-1}+\cdots+a_0
\]
for $a_{k-1},...,a_0\in\bfQ$. Then
\[
f_i(\sqrt{\alpha})=\alpha^{k/2}+a_{k-1}\alpha^{(k-1)/2}+\cdots+a_0.
\]
\end{proof}

\begin{problem}
Let $f(X)=X^6+3\in\bfQ[X]$.
\begin{enumerate}[label=(\alph*)]
\item Let $\alpha$ be a root of $f(X)$. Prove $(\alpha^3+1)/2$ is a
  primitive 6th root of unity.
\item Determine the Galois group of $f(X)$ over $\bfQ$.
\end{enumerate}
\end{problem}
\begin{proof}
(a) To show that $(\alpha^3+1)/2$ is a $6$th root of unity, suffices to
show that $\Phi_6((\alpha^3+1)/2)=0$ where $\Phi_6$ is the $6$th cyclotomic
polynomial. Recall that we may derive the $n$th cyclotomic polynomial via
the formula
\[
X^n-1=\prod_{d\mid n}\Phi_d(X)
\]
so that
\[
X^6-1=\Phi_1(X)\Phi_2(X)\Phi_3(X)\Phi_6(X)=(X-1)(X+1)(X^2+X+1)
\]
and we have
\begin{align*}
\Phi_6(X)&=\frac{X^6-1}{(X-1)(X+1)(X^2+X+1)}\\
         &=X^2-X+1.
\end{align*}
Thus,
\begin{align*}
\Phi_6((\alpha^3+1)/2)
&=\tfrac{1}{4}(\alpha^3+1)^2-\tfrac{1}{2}(\alpha^3+1)+1\\
&=\tfrac{1}{4}\alpha^6+\tfrac{1}{2}\alpha^3+\tfrac{1}{4}
  -\tfrac{1}{2}\alpha^3-\tfrac{1}{2}+1\\
&=\tfrac{1}{4}\alpha^6+\tfrac{3}{4}\\
&=\tfrac{1}{4}(\alpha^6+3)\\
&=0.
\end{align*}
Thus, $(\alpha^3+1)/2$ is $6$th root of unity.

To show that $(\alpha^3+1)/2$ is in fact a primitive root of unity, we need
to show that $6$ is the smallest integer such that
$((\alpha^3+1)/2)^6=1$. And that is too much work.
\\\\
(b) Put $\zeta_6\coloneqq(\alpha^3+1)/2$. The roots of the polynomial
are $\sqrt[6]{3},\zeta_6\sqrt[6]{3},...,{\zeta_6}^5\sqrt[6]{3}$. Hence, the
splitting field of $f$ contains $\sqrt[6]{3}$ and a primitive sixth root of
unity $(\alpha^3+1)/2$. Since $\deg(\Phi_6)=2$, and
$\sqrt{3}\in\bfQ(\Phi_6)$, the minimal polynomial of $\sqrt[6]{3}$ over
$\bfQ(Phi_6)$ is $X^3-\sqrt{3}$. Hence, the degree of the extension
\[
\left[\bfQ(f):\bfQ\right]=\left[\bfQ(f):\bfQ(\Phi_6)\right]\left[\bfQ(\Phi_6):\bfQ\right]=3\cdot 2=6.
\]
Thus, the Galois group of $\bfQ(f)/\bfQ$ is order $6$.

Moreover, the Galois group acts transitively on the roots of $f$ so there
are automorphism of the splitting field fixing the subfields $\bfQ(\Phi_6)$
and $\bfQ$. These are the automorphism
\[
\sigma\colon\alpha\mapsto-\alpha\qquad\text{and}\qquad\tau\colon\alpha\mapsto\zeta_6\alpha.
\]
Note that $\sigma$ has order $2$ and $\tau$ has order $3$ so that
$\Gal(\bfQ(f)/\bfQ)\cong D_6$.
\end{proof}

\begin{problem}
Let $R\coloneqq(\bfZ/(3))[X]$. Consider the ideals $I_1\coloneqq(X^2+1)$,
and $I_2\coloneqq(X^2+X+2)$. For $i=1,2$ we set $F_i=R/I_i$.
\begin{enumerate}[label=(\alph*)]
\item Show $F_1$ and $F_2$ are fields.
\item Are $F_1$ and $F_2$ isomorphic? If not, why not, and if so give an
  isomorphism from $F_1$ to $F_2$.
\end{enumerate}
\end{problem}
\begin{proof}
(a) Recall by some theorem in chapter 13 that $F[X]/(f)$ is a field
if and only if $f$ is irreducible. Therefore, it suffices to show that
$X^2+1$ and $X^2+X+2$ are irreducible over $\bfZ/(3)$. To that end, since
the degree of these polynomials is two, it suffices to show that they have
no roots over $\bfZ/(3)$.

In the case of $X^2+1$, we have $0^2+1\neq 0$, $1^2+1=1\neq 0$, and
$2^2+1=4+1=1+1=2\neq 0$. Thus, $X^2+1$ is irreducible.

In the case of $X^2+X+2$, we have $0^2+0+2=2\neq 0$, $1+1+2=1\neq 0$, and
$4+2+2=8=2\neq 0$.

Thus, $F_1$ and $F_2$ are fields.
\\\\
(b) By the classification theorem for finite fields, both $F_1$ and $F_2$
are an extension over $\bfF_3=\bfZ/(3)$ of degree $2$ hence, both are
isomorphic to $\bfF_{3^2}$. In particular, they are isomorphic to each
other. Let $\alpha$ be a root of $X^2+1$ and $\beta$ be a root of
$X^2+X+2$. Then the map $\alpha\mapsto\beta$ which fixes $\bfF_3$ is an
isomorphism. It suffices to show that this is an injective
homomorphism. First, this is a homomorphism since for any $x,y\in F_1$,
if $x,y\in\bfF_3$, $\varphi(x+y)=x+y=\varphi(x)+\varphi(y)$. If one of
$x$ or $y$ not in $\bfF_3$, suppose $x$, then $x=\alpha^k+x'$ for
$x'\in\bfF_3$ so
\[
\varphi(\alpha^k+x'+y)=\beta^k+x'+y=\varphi(\alpha^k+x')+\varphi(y)
\]
etc., thus this is an isomorphism.

To see that this map is injective, note that $\ker\varphi=\{0\}$. Thus,
$\varphi$ is an isomorphism.
\end{proof}

\begin{problem}
Suppose $F$ is a field, $K=F(\alpha)$ is a Galois extension, with cyclic
Galois group generated by $\sigma(\alpha)\coloneqq\alpha+1$. Show that
$\Ch(K)=p\neq 0$, and $\alpha^p-\alpha\in F$.
\end{problem}
\begin{proof}
Suppose that the Galois group of $K$ is cyclic of order $n>1$. Then,
\[
\sigma^n(\alpha)=\alpha=\alpha+n.
\]
Thus, $0=\alpha-\alpha=n\in F$ so $\Ch(F)$ is prime since the order of a
field is always prime.

Lastly, note that $\alpha^p-\alpha=\alpha(\alpha^{p-1}-1)$ since $\alpha$
is the root of the polynomial $x^p-x$.
\end{proof}

%%% Local Variables:
%%% mode: latex
%%% TeX-master: "../MA553-Quals"
%%% End:
