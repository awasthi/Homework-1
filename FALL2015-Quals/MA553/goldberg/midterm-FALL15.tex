\chapter{MA 553: Midterm, Fall 2015}
In full detail now:
\begin{problem}
\begin{enumerate}[label=(\alph*)]
\item Show, for any abelian group, the map $x\mapsto x^{-1}$ is an
  automorphism.
\item Show, for any $n$, the dihedral group $D_{2n}$ of order $2n$,
  satisfies $D_{2n}\cong Z_2\ltimes Z_n$.
\end{enumerate}
\end{problem}
\begin{proof}
(a) Let $G$ be an abelian group and define the map $\varphi\colon G\to G$
by $\varphi(x)\coloneqq x^{-1}$. Then, for any $x,y\in G$, we have
\begin{align*}
\varphi(xy)&=(xy)^{-1}\\
           &=y^{-1}x^{-1}\\
           &=x^{-1}y^{-1}\\
           &=\varphi(x)\varphi(x).
\end{align*}
Hence, $\varphi$ is a homomorphism.

Next, we will show that $\varphi$ is in fact an automorphism. To that end,
we must show that $\varphi$ is one-to-one and onto.

First, we show $\varphi$ is one-to-one. Let $x\in\ker\varphi$. Then
$\varphi(x)=x^{-1}=e$. Then we have $x^{-1}x=x$. But $x^{-1}x=e$ so
$x=e$. Thus, $\ker\varphi=\{e\}$ and $\varphi$ must be injective.

To see than that $\varphi$ is onto, take $x\in G$ then
$\varphi(x^{-1})=\left(x^{-1}\right)^{-1}=x$. Thus, $\varphi$ is surjective
and we conclude that $\varphi\in\Aut(G)$.
\\\\
(b) Recall that the dihedral group of order $2n$ is the group
\[
G\coloneqq
D_{2n}=
\left<\,r,s\;\middle|\;\text{$r^n=s^2=e$ and $srs^{-1}=r^{-1}$}\,\right>.
\]
Now, note that the subgroup generated by $r$, $K\coloneqq\langle r\rangle$,
is order $n$ hence, $K\lhd G$ since $[G:H]=2$ is the smallest prime
dividing the order of $G$. Let $H\coloneqq\langle s\rangle$. This is a
subgroup of order $2$. Note that $H\cap K=\{e\}$ and $HK<G$ since $K$ is
normal in $G$. Moreover, $|HK|=|H||K|/|H\cap K|=2n=|G|$ so $HK=G$ so we
have $G=H\ltimes K$. Moreover, since $H$ and $K$ are cyclic of order $2$
and $n$, respectively, we have $H\cong Z_2$ and $K\cong Z_n$ so $G\cong
Z_2\ltimes Z_n$.
\end{proof}

\begin{problem}
Show that there is no simple group of order $306=2\cdot 3^2\cdot 17$.
\end{problem}
\begin{proof}
Suppose $G$ is a finite group of order $306=2\cdot 3^2\cdot 17$. We will
show that one of $n_2$, $n_3$, or $n_{17}$ equals $1$.

By Sylow's theorem, $n_p\equiv 1\pmod{p}$ and $n_p\mid m$ where
$|G|=p^\alpha m$. Thus, we have:
\begin{itemize}[noitemsep]
\item $n_2=1$, $3$, $3^2$, $17$, $3\cdot 17$, or $3^2\cdot 17$;
\item $n_3=1$, $34$;
\item $n_{17}=1$, $18$.
\end{itemize}
Seeking a contradiction, suppose that none of $n_2$, $n_3$, or $n_{17}$
equal $1$. Then, at least, $n_2=3$, $n_3=34$, and $n_{17}$. This means that
there are $1+3+16\cdot 18=302$ elements of order $1$, $2$, and $17$. But
there are at least $8$ elements of order $3$ in the remaining Sylow
$3$-subgroups, pushing this total to $310$ which is absurd. Thus, at least
one of $n_2$, $n_3$, or $n_{17}$ equals $1$.
\end{proof}

\begin{problem}
Suppose $R$ is a ring with identity, and $I$, $J$, and $K$ are (two-sided)
ideals of $R$ with $K\subset I\cup J$. Prove that either $K\subset I$ or
$K\subset J$.
\end{problem}
\begin{proof}
We shall proceed by contradiction. Suppose that $K\nsubset I$ and
$K\nsubset J$. Then there exists elements $a,b\in K$ such that $a\notin I$
and $b\notin J$. Now, consider the element $a-b\in K$. Since $K\subset
I\cup J$, then $a-b\in I$ or $a-b\in J$. Without loss of generality,
suppose that $a-b\in I$. Then $(a-b)+b=a\in I$ since $I$ is additively
closed. This is a contradiction. Thus, $K\subset I$ or $K\subset J$.
\end{proof}

\begin{problem}
Let $R$ and $S$ be rings and suppose that $\varphi\colon R\to S$ is a ring
homomorphism. Let $I$ be an ideal of $R$ and $J$ and ideal of $S$.
\begin{enumerate}[label=(\alph*)]
\item Show that $\varphi^{-1}(J)\coloneqq\left\{\,r\in
    R\;\middle|\;\varphi(r)\in J\right\}$ is an ideal in $R$.
\item Show that if $\varphi$ is surjective, then
  $\varphi(I)\coloneqq\left\{\,\varphi(r)\;\middle|\;r\in I\,\right\}$ is
  an ideal in $S$.
\item Given an example where $\varphi$ is not surjective and $\varphi(I)$
  is not an ideal in $S$.
\end{enumerate}
\end{problem}
\begin{proof}
(a) We need to show two things: Let $r\in R$ and $a\in\varphi^{-1}(J)$ then
$\varphi(ra)=\varphi(r)\varphi(a)$, but $\varphi(a)\in J$ so
$\varphi(r)\varphi(a)\in J$. Thus, $ra\in\varphi^{-1}(J)$. Lastly, we show
$\varphi^{-1}(J)$ is an additive subgroup, namely, for
$a_1,a_2\in\varphi^{-1}(J)$, we have $\varphi(a_1),\varphi(a_2)\in J$ so
$\varphi(a_1)+\varphi(a_2)=\varphi(a_1+a_2)\in J$. Thus,
$a_1+a_2\in\varphi^{-1}(J)$. Thus, $\varphi^{-1}(J)$ is an ideal in $R$.
\\\\
(b) Suppose $\varphi$ is surjective. Then, for every element $s\in S$,
there exist an element $r\in R$ such that $s=\varphi(r)$. Now, let
$a\in\varphi(I)$ and $s\in S$. Then $\varphi(b)=a$ for some $b\in I$ and
$\varphi(r)=s$ for some $r\in R$. Thus,
$\varphi(rb)=sa\in\varphi(I)$. Lastly, if $a_1,a_2\in\varphi(I)$ then
$\varphi(b_1)=a_1$ and $\varphi(b_2)=b_2$ for $b_1,b_2\in I$ so $b_1+b_2\in
I$ implies that
$\varphi(b_1+b_2)=\varphi(b_1)+\varphi(b_2)\in\varphi(I)$. Thus,
$\varphi(I)$ is an ideal of $S$.
\\\\
(c) Consider the map $\varphi\colon Z_4\to Z_2\times Z_2$ given by the rule
$\varphi(s)=(s,s)$. This map is a homomorphism since for any $s_1,s_2\in
Z_4$, we have
\begin{align*}
\varphi(s_1+s_2)&=(s_1+s_2,s_1+s_2)&
\varphi(s_1s_2)&=(s_1s_2,s_1s_2)\\
                &=(s_1,s_1)+(s_2,s_2)&
               &=(s_1,s_1)(s_2,s_2)\\
                &=\varphi(s_1)+\varphi(s_2)&
               &=\varphi(s_1)\varphi(s_2).
\end{align*}
But note that $\varphi$ is not surjective since
$\varphi(Z_4)=\{(0,0),(1,1)\}$. Moreover, the latter is not an ideal since
for $(1,0)\in Z_2\times Z_2$, $(1,0)(1,1)=(1,0)\notin\varphi(Z_4)$.
\end{proof}

\begin{problem}
\begin{enumerate}[label=(\alph*)]
\item Let $R$ be a commutative ring with identity $1\neq 0$. Suppose that,
  for every $r\in R$, there is some $n=n_r\geq 2$ so that $r^n=r$. Prove
  that every prime ideal of $R$ is maximal.
\item Suppose $R$ is a unique factorization domain, $p\in R$ is
  irreducible, and $\frakp$ is a prime ideal with
  $0\subsetneq\frakp\subset(p)$. Show $\frakp=(p)$. (\emph{Hint:} Prove
  that $\frakp$ can be generated by irreducible elements.)
\end{enumerate}
\end{problem}
\begin{proof}
(a) Let $\frakp\in\Spec(R)$. Then $R/\frakp$ is an integral domain. Now,
let $r\in R\minus\frakp$ and $\pi\colon R\to R/\frakp$ be the canonical
projection. Put $\bar r\coloneqq\pi(r)$. Then since $r^n=r$ for some $n\geq
2$ we have
\[
\pi(r^n)=(\bar r)^n(\bar r)^n=\bar r=\pi(r).
\]
Thus, $\bar r(\bar r^{n-1}-\bar 1)=0$ implies $\bar r=\bar 0$ or $\bar
r^{n-1}=\bar 1$. But $r\notin\frakp$ so $\bar r\neq\bar 0$. Thus, $\bar
r^{n-1}=\bar 1$ and we see that $\bar r$ is a unit. Thus, $R/\frakp$ is a
field which implies that $\frakp$ is maximal.
\\\\
(b) First note that if $p$ is irreducible in $R$ then it is prime. We will
show that $\frakp$ contains a principal prime ideal. Let
$a\in\frakp$. Then, since $R$ is a UFD, we may write $a=p_1\cdots p_n$ for
$p_1,...,p_n$ irreducible in $R$. Hence, each $p_i$ is prime in $R$ and
$(p_i)$ is a prime ideal. Moreover, since $a=p_1\cdots p_n\in\frakp$,
$p_k\in\frakp$ for some $1\leq k\leq n$. Thus, $(p_k)\subset\frakp$. Hence,
we have $(p_k)\subset\frakp\subset(p)$. But this implies $p_k=rp$ for some
$r\in R$. Since $p_k$ is irreducible, $r$ must be a unit so $(p_k)=(p)$
which implies that $\frakp=(p)$.
\end{proof}

%%% Local Variables:
%%% mode: latex
%%% TeX-master: "../MA553-Quals"
%%% End:
