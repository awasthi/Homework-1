\chapter{Some Group Theory}
We will not start from the very basics. Instead, we supply important
theorems from D.\,\& F.
\section{Quotient Groups and Homomorphisms}
\subsection{The Isomorphism Theorems}
\begin{theorem*}[16]
If $\varphi\colon G\to H$ is a homomorphism of groups, then
$\ker\varphi\lhd G$ and $G/{\ker\varphi}\cong\varphi(G)$.
\end{theorem*}
\begin{theorem*}[17]
Let $\varphi\colon G\to H$ be a homomorphism of groups.
\begin{enumerate}[noitemsep,label=(\alph*)]
\item $\varphi$ is injective if and only if $\ker\varphi=1$.
\item $|G:\ker\varphi|=|\varphi(G)|$.
\end{enumerate}
\end{theorem*}
\begin{theorem*}[18]
Let $G$ be a group, let $A$ and $B$ be subgroups of $G$ and assume
$A<N_G(B)$. Then $AB$ is a subgroup of $G$, $B\lhd AB$, $A\cap B\lhd A$ and
$AB/B\cong A/A\cap B$.
\end{theorem*}
recall that $n_g(b)$, i.e., the normalizer of $b$ in $g$ is the set
\[
n_g(b)\coloneqq\left\{\,g\in g\;\middle|\;gbg^{-1}\subset a\,\right\}.
\]
incidentally, $n_g(b)$ is a subgroup of $g$: Associativity follows easily
from associativity of the binary operation on $G$ and the identity element,
$e_G$, is trivially in $N_G(B)$ since $e_GBe_G^{-1}=B\subse B$. Next we
check that it is closed under multiplication (the binary operation): let
$g_1,g_2\in n_g(b)$ then $g_1bg_1^{-1}\subset b$ and $g_2bg_2^{-1}\subset
b$ so $g_1(g_2 bg_2^{-1})g_1^{-1}\subset b$ so $g_1g_2\in n_g(b)$. Next we
show $N_G(B)$ is closed under inverses

While we are at it, let us define the following important subgroups

%%% Local Variables:
%%% mode: latex
%%% TeX-master: "../MA553-Quals"
%%% End:
