\chapter{January 2007}
\begin{problem}
Let $(G,\cdot)$  be a group. Show that $G$ is Abelian whenever $\Aut(G)$ is
a cyclic group under composition.
\end{problem}
\begin{proof}
Suppose that $\Aut(G)$ is cyclic. Then $\Inn(G)<\Aut(G)$ is cyclic. But
$\Inn(G)\cong G/Z(G)$. Thus, $G$ is Abelian by the following lemma.
\begin{lemma}
Let $(G,\cdot)$ be a group. If $G/Z(G)$ is cyclic, then $G$ is Abelian.
\end{lemma}
\begin{proof}[Proof of lemma]
\renewcommand\qedsymbol{$\clubsuit$}
Suppose that $G/Z(G)$ is cyclic. Then $G/Z(G)=\langle \bar x \rangle$ for
some representative $x\in G$. This means that for any $g\in G$, we can
write $g=x^kz$ for some positive integer $k$, for some $z\in Z(G)$. Let
$g_1,g_2\in G$. Then, by the following obvious algebraic manipulations
\[
g_1g_2=x^{k_1}z_1x^{k_2}z_2=z_1x^{k_1+k_2}z_2=z_2x^{k_2+k_1}z_1=z_2x^{k_2}x^{k_1}z_1=(x^{k_2}z_2)(x^{k_1}z_1)=g_2g_1,
\]
we see that $G$ is Abelian.
\end{proof}
\end{proof}

\begin{problem}
Let $(G,\cdot)$ be an Abelian group. The \emph{torsion subgroup of $G$} is
defined as the collection of elements of finite order:
\[
\Tor(G)\coloneqq
\left\{\,g\in G\;\middle|\;\text{$g^m=e$ for some integer $m>0$}\,\right\}.
\]
\begin{enumerate}[noitemsep,label=(\alph*)]
\item Show that the quotient group $G/{\Tor(G)}$ is \emph{torsion free},
  i.e., it contains no nontrivial elements of finite order.
\item Show that $\Tor(G)$ is finite whenever $G$ is finitely generated. (Do
  not assume that $G$ is finite.)
\end{enumerate}
\end{problem}
\begin{proof}
(a) (Presumably the torsion subgroup is a normal subgroup of $G$.) Define
$T\coloneqq\Tor(G/{\Tor(G)})$. We will show that $T=\bar e$. It is clear
that $\langle\bar e\rangle\subset T$ thus, we need only show that $T\subset
\langle\bar e\rangle$, i.e., if $t\in T$ then $g=\bar e$. Let $\bar g\in
T$. Then $\bar g\in G/{\Tor(G)}$ and $\bar g^m=\bar e$ for some positive
integer $m$. But $\bar g^m=\bar e$ implies that $g^m\Tor(G)=\Tor(G)$, i.e.,
$g^m\in\Tor(G)$. Thus, $(g^m)^n=g^{mn}e$ for some positive integer
$n$. Thus, $g\in\Tor(G)$ so we must have $\bar g=\bar e$.
\\\\
(b) Suppose that $G$ is finitely generated. By the fundamental theorem of
finitely generated Abelian groups, $G\cong\bbZ^r{\times}
Z_{s_1}{\times}\cdots{\times}Z_{s_n}$ for positive integers
$r,s_1,...,s_n$. It suffices to show that $\mathbf{1}\times
Z_{s_1}\times\cdots\times Z_{s_n}=\Tor(G)$ (once we have demonstrated this,
note that $\left|\mathbf{1}\times Z_{s_1}\times\cdots\times
  Z_{s_n}\right|=s_1\cdots s_n<\infty$). It is clear that $\mathbf{1}\times
Z_{s_1}\times\cdots\times Z_{s_n}\subset\Tor(G)$ since every element of
$\mathbf{1}\times Z_{s_1}\times\cdots\times Z_{s_n}$ has finite order,
i.e., for any $(\mathbf{1},z_1,...,z_n)\in \mathbf{1}\times
Z_{s_1}\times\cdots\times Z_{s_n}$, we have
$z=(\mathbf{1},z_1,...,z_n)^{s_1\cdots s_n}=(\mathbf{1},1,...,1)$ (as a
consequence of Lagrange's theorem). Now, suppose $z\coloneqq
(\mathbf{z},z_1,...,z_n)\in\Tor(G)$. Then $z^m=(\mathbf{1},1,...,1)$ for
some positive integer $m$. Since every non-identity element of $\bbZ^r$ has
infinite order, $\mathbf{z}=\mathbf{1}$ and $s_i\mid k$ for all $i$. Thus
$z\in\mathbf{1}\times Z_{s_1}\times\cdots Z_{s_n}$. Thus,
$|\Tor(G)|=s_1\cdots s_n$ so $\Tor(G)$ is indeed finite.
\end{proof}

\begin{problem}
Let $(G,\cdot)$ be a group of order $|G|=351$. Show that $G$ is solvable.
\end{problem}
\begin{proof}
The best plan of attack is to use Sylow's theorem. First, let us factor the
order of $G$ into powers of primes, $|G|=351=3^3\cdot 13$. In light of this
factorization, it suffices to show that either $|\Syl_{13}(G)|=1$ or
$|\Syl_3(G)|=1$ and hence, the unique Sylow-$13$ (or Sylow-$3$) subgroup
will be a normal subgroup of $G$. By Sylow's theorem, $n_{13}\equiv
1\pmod{13}$ and $n_{13}\mid 3^3$. Thus, $n_{13}=1$ or $27$. Suppose
$n_{13}=27$. Then $G$ contains $12\times 27=324$ elements of order $13$ so
there are $351-324-1=26$ elements remaining. This implies that
$n_3=1$. Thus, $P_3\in\Syl_3(G)$ is the unique Sylow-$3$ subgroup of $G$
hence, is normal. Thus, $G\rhd P_3$ so $G/P_3$ is a group. Incidentally,
$G/P_3\cong Z_{13}$ hence, solvable and $P_3$ is a $p$-group, hence
solvable. Thus, $G$ is solvable.

On the other hand, if $n_{13}=1$ then $P_{13}\in\Syl_{13}(G)$ is the unique
Sylow-$13$ subgroup of $G$ hence, normal in $G$. Since $P_{13}$ is a
$p$-group, it is solvable. Moreover, $G/P_{13}$ is a group of order $3^3$,
i.e., a $p$-group, hence, solvable. Thus, $G$ is solvable.

In either case, we have shown that $G$ must be solvable.
\end{proof}

\begin{problem}
Let $(G,\cdot)$ be a group, and $H<G$ a subgroup of finite index. Show that
there exists a normal subgroup $N\lhd G$ contained in $H$ which is also of
finite index. (Do not assume that $G$ is finite.)
\end{problem}
\begin{proof}
Suppose $H<G$ is a subgroup of finite index, i.e., $H$ partitions $G$ into
a finite number of cosets, say $G/H\coloneqq
\{H,g_1H,...,g_{k-1}H\}$. Define a homomorphism$\varphi\colon G\to
S_{G/H}$ by $g\mapsto gH$ (this is clearly a homomorphism: take $g_1,g_2\in
G$ then
$\varphi(g_1g_2)=g_1g_2H=(g_1H)(g_2H)=\varphi(g_1)\varphi(g_2)$). Thus,
$\ker\varphi\lhd G$ of finite index (in particular, by the 1st isomorphism
theorem and Lagrange's theorem $|G:\ker\varphi|\mid
|S_{G/H}|=|S_k|=k!$). Thus, it suffices to show that $\ker\varphi<H$. But
this is clear since, if $g\in\ker\varphi$ then $gH=H$ hence, $g\in H$.
\end{proof}

\begin{problem}
Let $(G,\cdot)$ be a finite group, and $\varphi\colon G\to G$ be a group
homomorphism. Show that for all normal Sylow $p$-subgroups $P\lhd G$ we
have $\varphi(P)<P$.
\end{problem}
\begin{proof}

\end{proof}

\begin{problem}
Let $(R,+,\cdot)$ be a commutative ring with $1\neq 0$.
\begin{enumerate}[noitemsep,label=(\alph*)]
\item Show that $R$ is an integral domain if and only if $(0)$ is a prime
  ideal.
\item Show that $R$ is a field if and only if $(0)$ is a maximal ideal.
\end{enumerate}
\end{problem}
\begin{proof}
\end{proof}

\begin{problem}
let $(R,+,\cdot)$ be a unique factorization domain. Choose an irreducible
element $p\in R$, and define the \emph{localization at $p$} as the ring of
fractions $R_p=D^{-1}R$ with respect to the multiplicative set
$D=R-(p)$. Show that $R_p$ is a principal ideal domain.
\end{problem}
\begin{proof}
\end{proof}

\begin{problem}
Let $(F,+,\cdot)$ be a field, and $F(\theta)/F$ be a finite, separable
extension. Let $L$ be the splitting field of the minimal polynomial
$m_{\theta,F}(x)\in F[x]$. Prove that for every prime $p$ dividing the
degree $[L:F]$, there exists a field $K$ such that $F\subset K\subset L$,
$[L:K]=p$, and $L=K(\theta)$.
\end{problem}
\begin{proof}
\end{proof}

\begin{problem}
Let $(\bbF_p,+,\cdot)$ be a finite field whose Cardinality $p$ is
prime. Fix a positive integer $n$ which is not divisible by $p$, and let
$\zeta_n$ be a primitive $n$th root of unity. Show that
$\left[\bbF_p(\zeta_n):\bbF_p\right]=\alpha$  is the least positive integer
such that $p^\alpha\equiv 1\pmod{n}$.
\end{problem}
\begin{proof}
\end{proof}

\begin{problem}
Prove that the Galois group of the splitting field over $\bbQ$ of
$f(x)=x^4+4x^2+2$ is a cyclic group.
\end{problem}
\begin{proof}
\end{proof}

%%% Local Variables:
%%% mode: latex
%%% TeX-master: "../MA553-Quals"
%%% End:
