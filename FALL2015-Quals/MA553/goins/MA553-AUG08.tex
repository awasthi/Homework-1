\chapter{Spring 2008}
\begin{problem}
Let $(G,\cdot)$ be a group, $(H,+)$ be an Abelian group, and
$\varphi\colon G\to H$ be a group homomorphism. If $N$ is a subgroup such
that $\ker\varphi<N<G$, show that $N\lhd G$ is a normal subgroup.
\end{problem}
\begin{proof}
Let $N$ be a subgroup of $G$ containing $\ker\varphi$. Then we must show
that for any $g\in G$, $gNg^{-1}\subset N$. First we observe that, since
$\ker\varphi\lhd G$, then $\ker\varphi\lhd N$ since for any $g\in N$, $g$
is also in $G$ so that $g(\ker\varphi)g^{-1}=\ker\varphi\subset N$. Thus,
$\ker\varphi\lhd N$. By the first isomorphism theorem\footnote{Theorem 16
  of Dummit and Foote \S3, p.\,99.}, $G/\ker\varphi\cong H$ hence,
$G/\ker\varphi$ is Abelian. Moreover, $N/\ker\varphi<G/\ker\varphi$ hence,
$N/\ker\varphi\lhd G/\ker\varphi$. It follows immediately from the lattice
isomorphism theorem\footnote{Theorem 20 of Dummit and Foote \S3, p.\,99.} (this
is essentially the UMP of the quotient by a group) that $N\lhd G$.
\end{proof}
\begin{problem}
Let $(G,\cdot)$ be a finite Abelian group of even order, i.e., $|G|=2k$ for
some $k\in\bfN$.
\begin{enumerate}[noitemsep,label=(\alph*)]
\item For $k$ odd, show that $G$ has exactly one element of order $2$.
\item Does the same happen for $k$ even? Prove or give a counterexample.
\end{enumerate}
\end{problem}
\begin{proof}
(a) This problem is most easily proven using Cauchy's
theorem\footnote{Theorem 11 of Dummit and Foote \S3, p.\,93}. Suppose that
$k$ is odd. If $k=1$, $G\cong Z_2$ and we are done ($Z_2$ contains only one
nontrivial element and its order is $2$). Otherwise $k>2$. Then by Cauchy's
theorem we are guaranteed that there exists an element $g\in G$ of order
$2$. Suppose $h$ is another element (distinct from $g$) of order $2$. Since
$2$ is the smallest prime number dividing the order of $G$, by a corollary
to Cayley's theorem\footnote{Corollary 5 of Dummit and Foote \S4, p.\,121},
$\langle  g \rangle$ is a normal subgroup of $G$ so $G/\langle g \rangle$
is a group. Moreover, since $h\neq g$, then $\bar h\neq\bar e$ and
$2\geq|\bar h|>1$ implies that $|\bar h|=2$. But $2\nmid k=|G/\langle g
\rangle|$ contradicting Lagrange's theorem. It follows that $G$ must have
exactly one element of order $2$.
\\\\
(b) No. Here is the simplest counterexample: Consider the direct product
$Z_2\times Z_2$. The elements $(1,0)$ and $(0,1)$ are elements of order
$2$, but are not equivalent.
\end{proof}
\begin{problem}
Let $(G,\cdot)$ be a finite group of odd order, and $H\lhd G$ be a normal
subgroup of prime order $|H|=17$. Show that $H<Z(G)$.
\end{problem}
\begin{proof}
Let $G$ act on $H$ by conjugation, i.e., the map $\varphi\colon G\times
H\to H$ defined by the rule $\varphi(g,h)\coloneqq ghg^{-1}$ determines a
group action on $H$. First, we verify that $\varphi$ indeed defines a group
action on $H$: First, observe that for $e_G\in G$ the identity element,
$\varphi(e_G,h)=e_Ghe_G^{-1}=h$; next, if $g_1,g_2\in G$ then
\[
\varphi(g_1,\varphi(g_2,h))=\varphi(g_1,g_2hg^{-1})=g_1g_2hg_2^{-1}g_1=g_1g_2h(g_1g_2)^{-1}=\varphi(g_1g_2,h).
\]
Lastly, $\varphi$ is clearly well-defined in the sense $\varphi(g,h)\in H$
for all $g\in G$, $h\in H$. Thus, $\varphi$ is a group action. Now, let us
ask what the kernel of this action is. Thus group action $\varphi$, induces
a group homomorphism $\varphi'\colon G\to\Aut(H)$ given by
$\varphi'(g)\coloneqq\Eval(\varphi,g)$. Now, since $|H|=17$, $H\cong
Z_{17}$, hence is cyclic. Thus, $\Aut(H)\cong(\bfZ/17\bfZ)^{\times}\cong
Z_{16}$. Now, since $|\varphi'(G)|\mid |G|$, $|\varphi'(G)|$ is odd. But
$\varphi'(G)<\Aut(H)$ so, by Lagrange's theorem, $|\varphi'(G)|\mid
16$. Thus, $|\varphi'(G)|=1$, i.e., $\varphi'$ is the trivial homomorphism,
i.e., $\varphi(g,h)=ghg^{-1}=h=\varphi(1,h)$. Thus, $H<Z(G)$.
\end{proof}
\begin{problem}
Let $(G,\cdot)$ be a finite group. Show that there exists a positive
integer $n$ such that $G$ is isomorphic to a subgroup of $A_n$, the
alternating group on $n$ letters. [\emph{Hint:} Show that $A_n$ contains a
copy of $S_{n-1}$ when $n\geq 3$.]
\end{problem}
\begin{proof}
% Let $n\coloneqq |G|$. If $n=1$ or $2$ we are done as $1$ (the trivial
% group) and $Z_2$ are exactly $A_1$ and $A_2$. Now suppose $n\geq 3$. By
% Cayley's theorem, $G$ imbeds into $S_n$.
Let $n-2\coloneqq |G|$. If $n-2=1$ or $2$, $G\cong 0$ (the trivial group) or
$G\cong Z_2$, both of which are exactly $A_1$ and $A_2$. Suppose $n-2\geq
3$. By Cayley's theorem, $G$ imbeds into $S_{n-1}$. Now, define a
homomorphism
\[
\varphi(\sigma)\coloneqq
\begin{cases}
\sigma&\text{if $\sigma$ is even}\\
\sigma(n+1\;n+2)&\text{if $\sigma$ is odd}
\end{cases}.
\]
We check that this is in fact a homomorphism. Let $\sigma,\tau\in G$. Then
\[
\varphi(\sigma\tau)=
\begin{cases}
\sigma\tau&\text{if $\sigma\tau$ is even}\\
\sigma\tau(n+1\;n+2)&\text{if $\sigma\tau$ is odd}
\end{cases}.
\]
But $\sigma\tau$ is odd if and only if $\sigma$ or $\tau$ is odd and
$\sigma\tau$ is even if and only if $\tau$ is even.
\end{proof}
\begin{problem}
Let $(G,\cdot)$ be a group of order $|G|=200$.
\begin{enumerate}[noitemsep,label=(\alph*)]
\item Show that $G$ is solvable.
\item Show that $G$ is the semidirect product of two $p$-subgroups.
\end{enumerate}
\end{problem}
\begin{proof}
(a) First we factor the order of the group $G$, $|G|=200=2^3\cdot
5^2$. Now we will make use of Sylow's theorem to show that $G$ has at least
one normal $p$-subgroup.
\\\\
(b)
\end{proof}
\begin{problem}
Let $(R,+,\cdot)$ and $(S,+,\cdot)$ be commutative rings with $1\neq 0$,
and let $\varphi\colon R\to S$ be a surjective ring homomorphism. Assuming
that $R$ is local, i.e., it has a unique maximal ideal, show that $S$ is
also local.
\end{problem}
\begin{proof}
\end{proof}
\begin{problem}
Let $(R,+,\cdot)$ be a principal ideal domain.
\begin{enumerate}[noitemsep,label=(\alph*)]
\item Show that every maximal ideal in $R$ is a prime ideal.
\item Must every prime ideal in $R$ be a maximal ideal? Prove or give a
  counterexample.
\end{enumerate}
\end{problem}
\begin{proof}
\end{proof}
\begin{problem}
Let $L/F$ be a Galois extension of degree $[L:F]=2p$ where $p$ is an odd
prime.
\begin{enumerate}[noitemsep,label=(\alph*)]
\item Show that there exists a unique quadratic subfield $E$, i.e.,
  $F\subset E\subset L$ and $[E:F]=2$.
\item Does there exist a unique subfield $K$ of index $2$, i.e., $F\subset
  K\subset L$ and $[L:K]=2$? Prove or give a counterexample.
\end{enumerate}
\end{problem}
\begin{proof}
\end{proof}
\begin{problem}
Fix a prime $p$, and consider the Artin--Schreier polynomial
$f(x)=x^p-x-1$.
\begin{enumerate}[noitemsep,label=(\alph*)]
\item Let $\bfF_p(f)$ be the splitting field of $f(x)$ over $\bfF_p$. Show
  that $\Gal\left(\bfF_p(f)/\bfF_p\right)\cong Z_p$.
\item Prove that $f(x)$ is irreducible in $\bfZ[x]$.
\end{enumerate}
\end{problem}
\begin{proof}
\end{proof}
\begin{problem}
Determine the Galois group of the splitting field over $\bfQ$ of
$f(x)=x^4+4$.
\end{problem}
\begin{proof}
\end{proof}


%%% Local Variables:
%%% mode: latex
%%% TeX-master: "../MA553-Quals"
%%% End:
