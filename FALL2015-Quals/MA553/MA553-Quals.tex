\def\documentauthor{Carlos Salinas}
\def\documenttitle{MA553 Past Qualifying Examinations}
\def\hwnum{Quals}
\def\shorttitle{MA553 \hwnum}
\def\coursename{MA553}
\def\documentsubject{abstract algebra}
\def\authoremail{salinac@purdue.edu}

\documentclass[article,oneside,10pt]{memoir}
\usepackage{geometry}
\usepackage[dvipsnames]{xcolor}
\usepackage[
    breaklinks,
    bookmarks=true,
    colorlinks=true,
    pageanchor=false,
    linkcolor=black,
    anchorcolor=black,
    citecolor=black,
    filecolor=black,
    menucolor=black,
    runcolor=black,
    urlcolor=black,
    hyperindex=false,
    hyperfootnotes=true,
    pdftitle={\shorttitle},
    pdfauthor={\documentauthor},
    pdfkeywords={\documentsubject},
    pdfsubject={\coursename}
    ]{hyperref}

% Use symbols instead of numbers
\renewcommand*{\thefootnote}{\fnsymbol{footnote}}

%% Math
\usepackage{amsthm}
\usepackage{amssymb}
\usepackage{mathtools}
% \usepackage{unicode-math}

%% PDFTeX specific
\usepackage[mathcal]{euscript}
\usepackage{mathrsfs}

\usepackage[LAE,LFE,T2A,T1]{fontenc}
\usepackage[utf8]{inputenc}
\usepackage[farsi,french,german,spanish,russian,english]{babel}
\babeltags{fr=french,
           de=german,
           en=english,
           es=spanish,
           pa=farsi,
           ru=russian
           }
\def\spanishoptions{mexico}

\selectlanguage{english}

\newcommand{\textfa}[1]{\beginR\textpa{#1}\endR}

\usepackage{cmap}
\usepackage{CJKutf8}
\newcommand{\textkr}[1]{\begin{CJK}{UTF8}{mj}#1\end{CJK}}
\newcommand{\textjp}[1]{\begin{CJK}{UTF8}{min}#1\end{CJK}}
\newcommand{\textzh}[1]{\begin{CJK}{UTF8}{bsmi}#1\end{CJK}}

\usepackage{graphicx}
\graphicspath{{figures/}}

% Misc
\usepackage{microtype}
\usepackage{multicol}
\usepackage[inline]{enumitem}
\usepackage{listings}
\usepackage{mleftright}
\mleftright

%% Theorems and definitions
%% remove parentheses
\makeatletter
\def\thmhead@plain#1#2#3{%
  \thmname{#1}\thmnumber{\@ifnotempty{#1}{ }\@upn{#2}}%
  \thmnote{{\the\thm@notefont\bfseries{#3}}}}
\let\thmhead\thmhead@plain
\makeatother

\theoremstyle{plain}
\newtheorem{theorem}{Theorem}
\newtheorem{proposition}[theorem]{Proposition}
\newtheorem{corollary}[theorem]{Corollary}
\newtheorem{claim}[theorem]{Claim}
\newtheorem{lemma}[theorem]{Lemma}
\newtheorem{axiom}[theorem]{Axiom}

\newtheorem*{corollary*}{Corollary}
\newtheorem*{claim*}{Claim}
\newtheorem*{lemma*}{Lemma}
\newtheorem*{proposition*}{Proposition}
\newtheorem*{theorem*}{Theorem}

\theoremstyle{definition}
\newtheorem{definition}{Definition}
\newtheorem{example}{Examples}
\newtheorem{examples}[example]{Examples}
\newtheorem{exercise}{Exercise}[chapter]
\newtheorem{problem}[exercise]{Problem}

\newtheorem*{definition*}{Definition}
\newtheorem*{example*}{Examples}
\newtheorem*{examples*}{Examples}
\newtheorem*{exercise*}{Exercise}
\newtheorem*{problem*}{Problem}

\theoremstyle{remark}
\newtheorem{remark}{Remark}
\newtheorem{remarks}[remark]{Remarks}
\newtheorem{observation}[remark]{Observation}
\newtheorem{observations}[remark]{Observations}

\newtheorem*{remark*}{**Remark**}
\newtheorem*{remarks*}{**Remarks**}
\newtheorem*{observation*}{**Observation**}
\newtheorem*{observations*}{**Observations**}

%% Redefinitions & commands
% \newcommand\restr[2]{{% we make the whole thing an ordinary symbol
%   \left.\kern-\nulldelimiterspace % automatically resize the bar with \right
%   {#1} % the function
%   % \vphantom{\big|} % pretend it's a little taller at normal size
%   \right|{#2} % this is the delimiter
%   }}

% \newcommand{\nle}{\ensuremath{\triangleleft}}
% \newcommand{\nge}{\ensuremath{\triangleright}}
\newcommand{\nsubset}{\ensuremath{\not\subset}}
\newcommand{\nsupset}{\ensuremath{\not\supset}}
\renewcommand\qedsymbol{\ensuremath{\null\hfill\blacksquare}}

%% Commands and operators
\DeclareMathOperator{\id}{id}
\DeclareMathOperator{\im}{im}
\DeclareMathOperator{\lcm}{lcm}

\DeclareMathOperator{\Aut}{Aut}
\DeclareMathOperator{\Eval}{Eval}
\DeclareMathOperator{\Gal}{Gal}
\DeclareMathOperator{\GL}{GL}
\DeclareMathOperator{\Inn}{Inn}
\DeclareMathOperator{\SL}{SL}
\DeclareMathOperator{\Syl}{Syl}
\DeclareMathOperator{\Tor}{Tor}

\newcommand{\bbC}{\mathbb{C}}
\newcommand{\bbF}{\mathbb{F}}
\newcommand{\bbN}{\mathbb{N}}
\newcommand{\bbQ}{\mathbb{Q}}
\newcommand{\bbR}{\mathbb{R}}
\newcommand{\bbZ}{\mathbb{Z}}
\newcommand{\bfC}{\mathbf{C}}
\newcommand{\bfF}{\mathbf{F}}
\newcommand{\bfN}{\mathbf{N}}
\newcommand{\bfQ}{\mathbf{Q}}
\newcommand{\bfR}{\mathbf{R}}
\newcommand{\bfZ}{\mathbf{Z}}

\begin{document}
\author{\href{mailto:\authoremail}{\documentauthor}}
\title{\documenttitle}
\date{\today}
\maketitle

% \chapter{Some Group Theory}
We will not start from the very basics. Instead, we supply important
theorems from D.\,\& F.
\section{Quotient Groups and Homomorphisms}
\subsection{The Isomorphism Theorems}
\begin{theorem*}[16]
If $\varphi\colon G\to H$ is a homomorphism of groups, then
$\ker\varphi\lhd G$ and $G/{\ker\varphi}\cong\varphi(G)$.
\end{theorem*}
\begin{theorem*}[17]
Let $\varphi\colon G\to H$ be a homomorphism of groups.
\begin{enumerate}[noitemsep,label=(\alph*)]
\item $\varphi$ is injective if and only if $\ker\varphi=1$.
\item $|G:\ker\varphi|=|\varphi(G)|$.
\end{enumerate}
\end{theorem*}
\begin{theorem*}[18]
Let $G$ be a group, let $A$ and $B$ be subgroups of $G$ and assume
$A<N_G(B)$. Then $AB$ is a subgroup of $G$, $B\lhd AB$, $A\cap B\lhd A$ and
$AB/B\cong A/A\cap B$.
\end{theorem*}
recall that $n_g(b)$, i.e., the normalizer of $b$ in $g$ is the set
\[
n_g(b)\coloneqq\left\{\,g\in g\;\middle|\;gbg^{-1}\subset a\,\right\}.
\]
incidentally, $n_g(b)$ is a subgroup of $g$: Associativity follows easily
from associativity of the binary operation on $G$ and the identity element,
$e_G$, is trivially in $N_G(B)$ since $e_GBe_G^{-1}=B\subse B$. Next we
check that it is closed under multiplication (the binary operation): let
$g_1,g_2\in n_g(b)$ then $g_1bg_1^{-1}\subset b$ and $g_2bg_2^{-1}\subset
b$ so $g_1(g_2 bg_2^{-1})g_1^{-1}\subset b$ so $g_1g_2\in n_g(b)$. Next we
show $N_G(B)$ is closed under inverses

While we are at it, let us define the following important subgroups

%%% Local Variables:
%%% mode: latex
%%% TeX-master: "../MA553-Quals"
%%% End:

% \include{rings/notes-on-rings}
% \include{fields/notes-on-fields}

\chapter{January 2007}
\begin{problem}
Let $(G,\cdot)$  be a group. Show that $G$ is Abelian whenever $\Aut(G)$ is
a cyclic group under composition.
\end{problem}
\begin{proof}
Suppose that $\Aut(G)$ is cyclic. Then $\Inn(G)<\Aut(G)$ is cyclic. But
$\Inn(G)\cong G/Z(G)$. Thus, $G$ is Abelian by the following lemma.
\begin{lemma}
Let $(G,\cdot)$ be a group. If $G/Z(G)$ is cyclic, then $G$ is Abelian.
\end{lemma}
\begin{proof}[Proof of lemma]
\renewcommand\qedsymbol{$\clubsuit$}
Suppose that $G/Z(G)$ is cyclic. Then $G/Z(G)=\langle \bar x \rangle$ for
some representative $x\in G$. This means that for any $g\in G$, we can
write $g=x^kz$ for some positive integer $k$, for some $z\in Z(G)$. Let
$g_1,g_2\in G$. Then, by the following obvious algebraic manipulations
\[
g_1g_2=x^{k_1}z_1x^{k_2}z_2=z_1x^{k_1+k_2}z_2=z_2x^{k_2+k_1}z_1=z_2x^{k_2}x^{k_1}z_1=(x^{k_2}z_2)(x^{k_1}z_1)=g_2g_1,
\]
we see that $G$ is Abelian.
\end{proof}
\end{proof}

\begin{problem}
Let $(G,\cdot)$ be an Abelian group. The \emph{torsion subgroup of $G$} is
defined as the collection of elements of finite order:
\[
\Tor(G)\coloneqq
\left\{\,g\in G\;\middle|\;\text{$g^m=e$ for some integer $m>0$}\,\right\}.
\]
\begin{enumerate}[noitemsep,label=(\alph*)]
\item Show that the quotient group $G/{\Tor(G)}$ is \emph{torsion free},
  i.e., it contains no nontrivial elements of finite order.
\item Show that $\Tor(G)$ is finite whenever $G$ is finitely generated. (Do
  not assume that $G$ is finite.)
\end{enumerate}
\end{problem}
\begin{proof}
(a) (Presumably the torsion subgroup is a normal subgroup of $G$.) Define
$T\coloneqq\Tor(G/{\Tor(G)})$. We will show that $T=\bar e$. It is clear
that $\langle\bar e\rangle\subset T$ thus, we need only show that $T\subset
\langle\bar e\rangle$, i.e., if $t\in T$ then $g=\bar e$. Let $\bar g\in
T$. Then $\bar g\in G/{\Tor(G)}$ and $\bar g^m=\bar e$ for some positive
integer $m$. But $\bar g^m=\bar e$ implies that $g^m\Tor(G)=\Tor(G)$, i.e.,
$g^m\in\Tor(G)$. Thus, $(g^m)^n=g^{mn}e$ for some positive integer
$n$. Thus, $g\in\Tor(G)$ so we must have $\bar g=\bar e$.
\\\\
(b) Suppose that $G$ is finitely generated. By the fundamental theorem of
finitely generated Abelian groups, $G\cong\bbZ^r{\times}
Z_{s_1}{\times}\cdots{\times}Z_{s_n}$ for positive integers
$r,s_1,...,s_n$. It suffices to show that $\mathbf{1}\times
Z_{s_1}\times\cdots\times Z_{s_n}=\Tor(G)$ (once we have demonstrated this,
note that $\left|\mathbf{1}\times Z_{s_1}\times\cdots\times
  Z_{s_n}\right|=s_1\cdots s_n<\infty$). It is clear that $\mathbf{1}\times
Z_{s_1}\times\cdots\times Z_{s_n}\subset\Tor(G)$ since every element of
$\mathbf{1}\times Z_{s_1}\times\cdots\times Z_{s_n}$ has finite order,
i.e., for any $(\mathbf{1},z_1,...,z_n)\in \mathbf{1}\times
Z_{s_1}\times\cdots\times Z_{s_n}$, we have
$z=(\mathbf{1},z_1,...,z_n)^{s_1\cdots s_n}=(\mathbf{1},1,...,1)$ (as a
consequence of Lagrange's theorem). Now, suppose $z\coloneqq
(\mathbf{z},z_1,...,z_n)\in\Tor(G)$. Then $z^m=(\mathbf{1},1,...,1)$ for
some positive integer $m$. Since every non-identity element of $\bbZ^r$ has
infinite order, $\mathbf{z}=\mathbf{1}$ and $s_i\mid k$ for all $i$. Thus
$z\in\mathbf{1}\times Z_{s_1}\times\cdots Z_{s_n}$. Thus,
$|\Tor(G)|=s_1\cdots s_n$ so $\Tor(G)$ is indeed finite.
\end{proof}

\begin{problem}
Let $(G,\cdot)$ be a group of order $|G|=351$. Show that $G$ is solvable.
\end{problem}
\begin{proof}
The best plan of attack is to use Sylow's theorem. First, let us factor the
order of $G$ into powers of primes, $|G|=351=3^3\cdot 13$. In light of this
factorization, it suffices to show that either $|\Syl_{13}(G)|=1$ or
$|\Syl_3(G)|=1$ and hence, the unique Sylow-$13$ (or Sylow-$3$) subgroup
will be a normal subgroup of $G$. By Sylow's theorem, $n_{13}\equiv
1\pmod{13}$ and $n_{13}\mid 3^3$. Thus, $n_{13}=1$ or $27$. Suppose
$n_{13}=27$. Then $G$ contains $12\times 27=324$ elements of order $13$ so
there are $351-324-1=26$ elements remaining. This implies that
$n_3=1$. Thus, $P_3\in\Syl_3(G)$ is the unique Sylow-$3$ subgroup of $G$
hence, is normal. Thus, $G\rhd P_3$ so $G/P_3$ is a group. Incidentally,
$G/P_3\cong Z_{13}$ hence, solvable and $P_3$ is a $p$-group, hence
solvable. Thus, $G$ is solvable.

On the other hand, if $n_{13}=1$ then $P_{13}\in\Syl_{13}(G)$ is the unique
Sylow-$13$ subgroup of $G$ hence, normal in $G$. Since $P_{13}$ is a
$p$-group, it is solvable. Moreover, $G/P_{13}$ is a group of order $3^3$,
i.e., a $p$-group, hence, solvable. Thus, $G$ is solvable.

In either case, we have shown that $G$ must be solvable.
\end{proof}

\begin{problem}
Let $(G,\cdot)$ be a group, and $H<G$ a subgroup of finite index. Show that
there exists a normal subgroup $N\lhd G$ contained in $H$ which is also of
finite index. (Do not assume that $G$ is finite.)
\end{problem}
\begin{proof}
Suppose $H<G$ is a subgroup of finite index, i.e., $H$ partitions $G$ into
a finite number of cosets, say $G/H\coloneqq
\{H,g_1H,...,g_{k-1}H\}$. Define a homomorphism$\varphi\colon G\to
S_{G/H}$ by $g\mapsto gH$ (this is clearly a homomorphism: take $g_1,g_2\in
G$ then
$\varphi(g_1g_2)=g_1g_2H=(g_1H)(g_2H)=\varphi(g_1)\varphi(g_2)$). Thus,
$\ker\varphi\lhd G$ of finite index (in particular, by the 1st isomorphism
theorem and Lagrange's theorem $|G:\ker\varphi|\mid
|S_{G/H}|=|S_k|=k!$). Thus, it suffices to show that $\ker\varphi<H$. But
this is clear since, if $g\in\ker\varphi$ then $gH=H$ hence, $g\in H$.
\end{proof}

\begin{problem}
Let $(G,\cdot)$ be a finite group, and $\varphi\colon G\to G$ be a group
homomorphism. Show that for all normal Sylow $p$-subgroups $P\lhd G$ we
have $\varphi(P)<P$.
\end{problem}
\begin{proof}

\end{proof}

\begin{problem}
Let $(R,+,\cdot)$ be a commutative ring with $1\neq 0$.
\begin{enumerate}[noitemsep,label=(\alph*)]
\item Show that $R$ is an integral domain if and only if $(0)$ is a prime
  ideal.
\item Show that $R$ is a field if and only if $(0)$ is a maximal ideal.
\end{enumerate}
\end{problem}
\begin{proof}
\end{proof}

\begin{problem}
let $(R,+,\cdot)$ be a unique factorization domain. Choose an irreducible
element $p\in R$, and define the \emph{localization at $p$} as the ring of
fractions $R_p=D^{-1}R$ with respect to the multiplicative set
$D=R-(p)$. Show that $R_p$ is a principal ideal domain.
\end{problem}
\begin{proof}
\end{proof}

\begin{problem}
Let $(F,+,\cdot)$ be a field, and $F(\theta)/F$ be a finite, separable
extension. Let $L$ be the splitting field of the minimal polynomial
$m_{\theta,F}(x)\in F[x]$. Prove that for every prime $p$ dividing the
degree $[L:F]$, there exists a field $K$ such that $F\subset K\subset L$,
$[L:K]=p$, and $L=K(\theta)$.
\end{problem}
\begin{proof}
\end{proof}

\begin{problem}
Let $(\bbF_p,+,\cdot)$ be a finite field whose Cardinality $p$ is
prime. Fix a positive integer $n$ which is not divisible by $p$, and let
$\zeta_n$ be a primitive $n$th root of unity. Show that
$\left[\bbF_p(\zeta_n):\bbF_p\right]=\alpha$  is the least positive integer
such that $p^\alpha\equiv 1\pmod{n}$.
\end{problem}
\begin{proof}
\end{proof}

\begin{problem}
Prove that the Galois group of the splitting field over $\bbQ$ of
$f(x)=x^4+4x^2+2$ is a cyclic group.
\end{problem}
\begin{proof}
\end{proof}

%%% Local Variables:
%%% mode: latex
%%% TeX-master: "../MA553-Quals"
%%% End:

\chapter{Spring 2008}
\begin{problem}
Let $(G,\cdot)$ be a group, $(H,+)$ be an Abelian group, and
$\varphi\colon G\to H$ be a group homomorphism. If $N$ is a subgroup such
that $\ker\varphi<N<G$, show that $N\lhd G$ is a normal subgroup.
\end{problem}
\begin{proof}
Let $N$ be a subgroup of $G$ containing $\ker\varphi$. Then we must show
that for any $g\in G$, $gNg^{-1}\subset N$. First we observe that, since
$\ker\varphi\lhd G$, then $\ker\varphi\lhd N$ since for any $g\in N$, $g$
is also in $G$ so that $g(\ker\varphi)g^{-1}=\ker\varphi\subset N$. Thus,
$\ker\varphi\lhd N$. By the first isomorphism theorem\footnote{Theorem 16
  of Dummit and Foote \S3, p.\,99.}, $G/\ker\varphi\cong H$ hence,
$G/\ker\varphi$ is Abelian. Moreover, $N/\ker\varphi<G/\ker\varphi$ hence,
$N/\ker\varphi\lhd G/\ker\varphi$. It follows immediately from the lattice
isomorphism theorem\footnote{Theorem 20 of Dummit and Foote \S3, p.\,99.} (this
is essentially the UMP of the quotient by a group) that $N\lhd G$.
\end{proof}
\begin{problem}
Let $(G,\cdot)$ be a finite Abelian group of even order, i.e., $|G|=2k$ for
some $k\in\bfN$.
\begin{enumerate}[noitemsep,label=(\alph*)]
\item For $k$ odd, show that $G$ has exactly one element of order $2$.
\item Does the same happen for $k$ even? Prove or give a counterexample.
\end{enumerate}
\end{problem}
\begin{proof}
(a) This problem is most easily proven using Cauchy's
theorem\footnote{Theorem 11 of Dummit and Foote \S3, p.\,93}. Suppose that
$k$ is odd. If $k=1$, $G\cong Z_2$ and we are done ($Z_2$ contains only one
nontrivial element and its order is $2$). Otherwise $k>2$. Then by Cauchy's
theorem we are guaranteed that there exists an element $g\in G$ of order
$2$. Suppose $h$ is another element (distinct from $g$) of order $2$. Since
$2$ is the smallest prime number dividing the order of $G$, by a corollary
to Cayley's theorem\footnote{Corollary 5 of Dummit and Foote \S4, p.\,121},
$\langle  g \rangle$ is a normal subgroup of $G$ so $G/\langle g \rangle$
is a group. Moreover, since $h\neq g$, then $\bar h\neq\bar e$ and
$2\geq|\bar h|>1$ implies that $|\bar h|=2$. But $2\nmid k=|G/\langle g
\rangle|$ contradicting Lagrange's theorem. It follows that $G$ must have
exactly one element of order $2$.
\\\\
(b) No. Here is the simplest counterexample: Consider the direct product
$Z_2\times Z_2$. The elements $(1,0)$ and $(0,1)$ are elements of order
$2$, but are not equivalent.
\end{proof}
\begin{problem}
Let $(G,\cdot)$ be a finite group of odd order, and $H\lhd G$ be a normal
subgroup of prime order $|H|=17$. Show that $H<Z(G)$.
\end{problem}
\begin{proof}
Let $G$ act on $H$ by conjugation, i.e., the map $\varphi\colon G\times
H\to H$ defined by the rule $\varphi(g,h)\coloneqq ghg^{-1}$ determines a
group action on $H$. First, we verify that $\varphi$ indeed defines a group
action on $H$: First, observe that for $e_G\in G$ the identity element,
$\varphi(e_G,h)=e_Ghe_G^{-1}=h$; next, if $g_1,g_2\in G$ then
\[
\varphi(g_1,\varphi(g_2,h))=\varphi(g_1,g_2hg^{-1})=g_1g_2hg_2^{-1}g_1=g_1g_2h(g_1g_2)^{-1}=\varphi(g_1g_2,h).
\]
Lastly, $\varphi$ is clearly well-defined in the sense $\varphi(g,h)\in H$
for all $g\in G$, $h\in H$. Thus, $\varphi$ is a group action. Now, let us
ask what the kernel of this action is. Thus group action $\varphi$, induces
a group homomorphism $\varphi'\colon G\to\Aut(H)$ given by
$\varphi'(g)\coloneqq\Eval(\varphi,g)$. Now, since $|H|=17$, $H\cong
Z_{17}$, hence is cyclic. Thus, $\Aut(H)\cong(\bfZ/17\bfZ)^{\times}\cong
Z_{16}$. Now, since $|\varphi'(G)|\mid |G|$, $|\varphi'(G)|$ is odd. But
$\varphi'(G)<\Aut(H)$ so, by Lagrange's theorem, $|\varphi'(G)|\mid
16$. Thus, $|\varphi'(G)|=1$, i.e., $\varphi'$ is the trivial homomorphism,
i.e., $\varphi(g,h)=ghg^{-1}=h=\varphi(1,h)$. Thus, $H<Z(G)$.
\end{proof}
\begin{problem}
Let $(G,\cdot)$ be a finite group. Show that there exists a positive
integer $n$ such that $G$ is isomorphic to a subgroup of $A_n$, the
alternating group on $n$ letters. [\emph{Hint:} Show that $A_n$ contains a
copy of $S_{n-1}$ when $n\geq 3$.]
\end{problem}
\begin{proof}
% Let $n\coloneqq |G|$. If $n=1$ or $2$ we are done as $1$ (the trivial
% group) and $Z_2$ are exactly $A_1$ and $A_2$. Now suppose $n\geq 3$. By
% Cayley's theorem, $G$ imbeds into $S_n$.
Let $n-2\coloneqq |G|$. If $n-2=1$ or $2$, $G\cong 0$ (the trivial group) or
$G\cong Z_2$, both of which are exactly $A_1$ and $A_2$. Suppose $n-2\geq
3$. By Cayley's theorem, $G$ imbeds into $S_{n-1}$. Now, define a
homomorphism
\[
\varphi(\sigma)\coloneqq
\begin{cases}
\sigma&\text{if $\sigma$ is even}\\
\sigma(n+1\;n+2)&\text{if $\sigma$ is odd}
\end{cases}.
\]
We check that this is in fact a homomorphism. Let $\sigma,\tau\in G$. Then
\[
\varphi(\sigma\tau)=
\begin{cases}
\sigma\tau&\text{if $\sigma\tau$ is even}\\
\sigma\tau(n+1\;n+2)&\text{if $\sigma\tau$ is odd}
\end{cases}.
\]
But $\sigma\tau$ is odd if and only if $\sigma$ or $\tau$ is odd and
$\sigma\tau$ is even if and only if $\tau$ is even.
\end{proof}
\begin{problem}
Let $(G,\cdot)$ be a group of order $|G|=200$.
\begin{enumerate}[noitemsep,label=(\alph*)]
\item Show that $G$ is solvable.
\item Show that $G$ is the semidirect product of two $p$-subgroups.
\end{enumerate}
\end{problem}
\begin{proof}
(a) First we factor the order of the group $G$, $|G|=200=2^3\cdot
5^2$. Now we will make use of Sylow's theorem to show that $G$ has at least
one normal $p$-subgroup.
\\\\
(b)
\end{proof}
\begin{problem}
Let $(R,+,\cdot)$ and $(S,+,\cdot)$ be commutative rings with $1\neq 0$,
and let $\varphi\colon R\to S$ be a surjective ring homomorphism. Assuming
that $R$ is local, i.e., it has a unique maximal ideal, show that $S$ is
also local.
\end{problem}
\begin{proof}
\end{proof}
\begin{problem}
Let $(R,+,\cdot)$ be a principal ideal domain.
\begin{enumerate}[noitemsep,label=(\alph*)]
\item Show that every maximal ideal in $R$ is a prime ideal.
\item Must every prime ideal in $R$ be a maximal ideal? Prove or give a
  counterexample.
\end{enumerate}
\end{problem}
\begin{proof}
\end{proof}
\begin{problem}
Let $L/F$ be a Galois extension of degree $[L:F]=2p$ where $p$ is an odd
prime.
\begin{enumerate}[noitemsep,label=(\alph*)]
\item Show that there exists a unique quadratic subfield $E$, i.e.,
  $F\subset E\subset L$ and $[E:F]=2$.
\item Does there exist a unique subfield $K$ of index $2$, i.e., $F\subset
  K\subset L$ and $[L:K]=2$? Prove or give a counterexample.
\end{enumerate}
\end{problem}
\begin{proof}
\end{proof}
\begin{problem}
Fix a prime $p$, and consider the Artin--Schreier polynomial
$f(x)=x^p-x-1$.
\begin{enumerate}[noitemsep,label=(\alph*)]
\item Let $\bfF_p(f)$ be the splitting field of $f(x)$ over $\bfF_p$. Show
  that $\Gal\left(\bfF_p(f)/\bfF_p\right)\cong Z_p$.
\item Prove that $f(x)$ is irreducible in $\bfZ[x]$.
\end{enumerate}
\end{problem}
\begin{proof}
\end{proof}
\begin{problem}
Determine the Galois group of the splitting field over $\bfQ$ of
$f(x)=x^4+4$.
\end{problem}
\begin{proof}
\end{proof}


%%% Local Variables:
%%% mode: latex
%%% TeX-master: "../MA553-Quals"
%%% End:

% \include{goldberg/}
% \include{heinzer/}
\section{August, 2015}
\begin{problem}

\textfa{جوشحال}
\end{problem}
\begin{proof}
\end{proof}

%%% Local Variables:
%%% mode: latex
%%% TeX-master: "../MA553-Quals"
%%% End:

\section{August 2010}

%%% Local Variables:
%%% mode: latex
%%% TeX-master: "../MA553-Quals"
%%% End:

% \include{ulrich/}
\end{document}

%%% Local Variables:
%%% mode: latex
%%% TeX-master: t
%%% End:
