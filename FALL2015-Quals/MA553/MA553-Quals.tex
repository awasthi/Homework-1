\def\documentauthor{Carlos Salinas}
\def\documenttitle{MA553 Past Qualifying Examinations}
\def\hwnum{Quals}
\def\shorttitle{MA553 \hwnum}
\def\coursename{MA553}
\def\documentsubject{abstract algebra}
\def\authoremail{salinac@purdue.edu}

\documentclass[article,oneside,10pt]{memoir}
\usepackage{geometry}
\usepackage[dvipsnames]{xcolor}
\usepackage[
    breaklinks,
    bookmarks=true,
    colorlinks=true,
    pageanchor=false,
    linkcolor=black,
    anchorcolor=black,
    citecolor=black,
    filecolor=black,
    menucolor=black,
    runcolor=black,
    urlcolor=black,
    hyperindex=false,
    hyperfootnotes=true,
    pdftitle={\shorttitle},
    pdfauthor={\documentauthor},
    pdfkeywords={\documentsubject},
    pdfsubject={\coursename}
    ]{hyperref}

% Use symbols instead of numbers
\renewcommand*{\thefootnote}{\fnsymbol{footnote}}

%% Math
\usepackage{amsthm}
\usepackage{amssymb}
\usepackage{centernot}
\usepackage{mathtools}
% \usepackage{unicode-math}

%% PDFTeX specific
\usepackage[mathcal]{euscript}
\usepackage{mathrsfs}

\usepackage[LAE,LFE,T2A,T1]{fontenc}
\usepackage[utf8]{inputenc}
\usepackage[farsi,french,german,spanish,russian,english]{babel}
\babeltags{fr=french,
           de=german,
           en=english,
           es=spanish,
           pa=farsi,
           ru=russian
           }
\def\spanishoptions{mexico}

\selectlanguage{english}

\newcommand{\textfa}[1]{\beginR\textpa{#1}\endR}

\usepackage{cmap}
\usepackage{CJKutf8}
\newcommand{\textkr}[1]{\begin{CJK}{UTF8}{mj}#1\end{CJK}}
\newcommand{\textjp}[1]{\begin{CJK}{UTF8}{min}#1\end{CJK}}
\newcommand{\textzh}[1]{\begin{CJK}{UTF8}{bsmi}#1\end{CJK}}

\usepackage{graphicx}
\graphicspath{{figures/}}

% Misc
\usepackage{microtype}
\usepackage{multicol}
\usepackage[inline]{enumitem}
\usepackage{listings}
\usepackage{mleftright}
\mleftright

%% Theorems and definitions
%% remove parentheses
% \makeatletter
% \def\thmhead@plain#1#2#3{%
%   \thmname{#1}\thmnumber{\@ifnotempty{#1}{ }\@upn{#2}}%
%   \thmnote{{\the\thm@notefont\bfseries{#3}}}}
% \let\thmhead\thmhead@plain
% \makeatother

\theoremstyle{plain}
\newtheorem{theorem}{Theorem}
\newtheorem{proposition}[theorem]{Proposition}
\newtheorem{corollary}[theorem]{Corollary}
\newtheorem{claim}[theorem]{Claim}
\newtheorem{lemma}[theorem]{Lemma}
\newtheorem{axiom}[theorem]{Axiom}

\newtheorem*{corollary*}{Corollary}
\newtheorem*{claim*}{Claim}
\newtheorem*{lemma*}{Lemma}
\newtheorem*{proposition*}{Proposition}
\newtheorem*{theorem*}{Theorem}

\theoremstyle{definition}
\newtheorem{definition}{Definition}
\newtheorem{example}{Examples}
\newtheorem{examples}[example]{Examples}
\newtheorem{exercise}{Exercise}[chapter]
\newtheorem{problem}[exercise]{Problem}

\newtheorem*{definition*}{Definition}
\newtheorem*{example*}{Examples}
\newtheorem*{examples*}{Examples}
\newtheorem*{exercise*}{Exercise}
\newtheorem*{problem*}{Problem}

\theoremstyle{remark}
\newtheorem{remark}{Remark}
\newtheorem{remarks}[remark]{Remarks}
\newtheorem{observation}[remark]{Observation}
\newtheorem{observations}[remark]{Observations}

\newtheorem*{remark*}{**Remark**}
\newtheorem*{remarks*}{**Remarks**}
\newtheorem*{observation*}{**Observation**}
\newtheorem*{observations*}{**Observations**}

%% Redefinitions & commands
% \newcommand\restr[2]{{% we make the whole thing an ordinary symbol
%   \left.\kern-\nulldelimiterspace % automatically resize the bar with \right
%   {#1} % the function
%   % \vphantom{\big|} % pretend it's a little taller at normal size
%   \right|{#2} % this is the delimiter
%   }}

% \newcommand{\nle}{\ensuremath{\triangleleft}}
% \newcommand{\nge}{\ensuremath{\triangleright}}
\newcommand{\nlhd}{\ensuremath{\centernot\lhd}}
\newcommand{\nsubset}{\ensuremath{\not\subset}}
\newcommand{\nsupset}{\ensuremath{\not\supset}}
\newcommand\minus{\ensuremath{\null\smallsetminus}}
\renewcommand\qedsymbol{\ensuremath{\null\hfill\blacksquare}}

%% Commands and operators
\DeclareMathOperator{\Ch}{ch}
\DeclareMathOperator{\id}{id}
\DeclareMathOperator{\im}{im}
\DeclareMathOperator{\lcm}{lcm}
\DeclareMathOperator{\nil}{nil}
\DeclareMathOperator{\rad}{rad}

\DeclareMathOperator{\Aut}{Aut}
\DeclareMathOperator{\Eval}{Eval}
\DeclareMathOperator{\Gal}{Gal}
\DeclareMathOperator{\GL}{GL}
\DeclareMathOperator{\Inn}{Inn}
\DeclareMathOperator{\SL}{SL}
\DeclareMathOperator{\Stab}{Stab}
\DeclareMathOperator{\Spec}{Spec}
\DeclareMathOperator{\Syl}{Syl}
\DeclareMathOperator{\Tor}{Tor}

\newcommand{\bbC}{\mathbb{C}}
\newcommand{\bbF}{\mathbb{F}}
\newcommand{\bbN}{\mathbb{N}}
\newcommand{\bbQ}{\mathbb{Q}}
\newcommand{\bbR}{\mathbb{R}}
\newcommand{\bbZ}{\mathbb{Z}}

\newcommand{\bfC}{\mathbf{C}}
\newcommand{\bfF}{\mathbf{F}}
\newcommand{\bfN}{\mathbf{N}}
\newcommand{\bfQ}{\mathbf{Q}}
\newcommand{\bfR}{\mathbf{R}}
\newcommand{\bfZ}{\mathbf{Z}}


\newcommand{\frakm}{\mathfrak{m}}
\newcommand{\frakM}{\mathfrak{M}}
\newcommand{\frakp}{\mathfrak{p}}
\newcommand{\frakP}{\mathfrak{P}}
\newcommand{\frakq}{\mathfrak{q}}
\newcommand{\frakQ}{\mathfrak{Q}}

\begin{document}
\author{\href{mailto:\authoremail}{\documentauthor}}
\title{\documenttitle}
\date{\today}
\maketitle

\chapter{Course Notes}
\thispagestyle{empty}

These notes roughly correspond to the three sections (by the same name) on
groups, rings and fields from Dummit and Foote's \emph{Abstract Algebra}
\cite{dummit-foote:abstract-algebra}. I also make nominal use of Herstein's
\emph{Topics in Abstract Algebra} \cite{herstein:topics-in-algebra}

\section{Group Theory}


%%% Local Variables:
%%% mode: latex
%%% TeX-master: "../MA553-HW-Current"
%%% End:

\section{Rings}
\begin{problem}
Let $R$ be a commutative ring with $1\neq 0$ and let $\frakp$ be a
prime ideal of $R$. Let $I$ and $J$ be ideals of $R$ such that $I\cap
J\subset\frakp$, prove that either $I\subset P$ or $J\subset P$.
\end{problem}
\begin{proof}
Without loss of generality, suppose that $I\nsubset J$. We show that
$J\subset\frakp$. Let $x\in I$. Then $x\notin\frakp$. But for any $y\in J$,
$xy\in I\cap J$. Thus, $xy\in\frakp$. Since $\frakp$ is prime, $x\in\frakp$
or $y\in\frakp$. But $x\notin\frakp$ hence, $y\in\frakp$. This is true for
any $y\in J$. Thus, $J\subset\frakp$.
\end{proof}
\begin{problem}
Prove that a finite integral domain is a field.
\end{problem}
\begin{proof}
Let $a\in R$ be a nonzero element. Define the map $\varphi_a\colon R\to R$
by $\varphi_a(x)\coloneqq ax$. Then $\varphi_a$ defines a group
homomorphism on $R$ viewed as an additive Abelian group: Let $x,y\in R$
then
\begin{align*}
\varphi_a(x+y)&=a(x+y)\\
&=ax+ay\\
&=\varphi_a(x)+\varphi_a(y).
\end{align*}
Now, let $x\in\ker\varphi$. Then $\varphi_a(x)=ax=0$. Since $R$ is a domain
and $a\neq 0$, $x=0$. Thus, $\varphi$ is injective. Since $R$ is finite and
$\varphi_a\colon R\to R$ is injective, $\varphi_a$ is surjective (by the
pigeonhole principle). Thus, there exists an element $b\in R$ such that
$\varphi_a(b)=ab=1$. Thus, $a$ is a unit. Since $\varphi_a$ chosen
arbitrarily, it follows that every nonzero element $a\in R$ is a
unit. Thus, $R$ is a field.
\end{proof}

\begin{problem}
An element $x$ of a ring $R$ is called nilpotent if some power of $x$ is
zero. Prove that if $x$ is nilpotent, then $1+x$ is a unit in $R$.
\end{problem}
\begin{proof}
First we will prove the following:
\begin{lemma}
If $x$ is nilpotent, then $-x$ is nilpotent.
\end{lemma}
\begin{proof}
\renewcommand\qedsymbol{$\clubsuit$}
Suppose that $x$ is nilpotent. Then $x^n=0$ for some positive integer
$n$. Then
\[
(-x)^n=(-1)^n\cdot x^n=(-1)^n\cdot 0=0.
\]
Thus, $-x$ is nilpotent.
\end{proof}
Now, since $x$ is nilpotent, by the preceding lemma, $-x$ is
nilpotent. Thus
\[
(-x)^n-1=(-x-1)((-x)^{n-1}+\cdots+1).
\]
Since $x^n=0$, we have
\[
-1=((-x)-1)((-x)^{n-1}+\cdots+1)
\]
or
\[
1=(1+x)((-x)^{n-1}+\cdots+1).
\]
Thus, $1+x$ is a unit.
\end{proof}

\begin{problem}
Let $R$ be a nonzero commutative ring with $1$. Show that if $I$ is an
ideal of $R$ such that $1+a$ is a unit in $R$ for all $a\in I$, then $I$ is
contained in every maximal ideal of $R$.
\end{problem}
\begin{proof}
Seeking a contradiction, assume otherwise. Then there exists a maximal
ideal $\frakm$ such that $\frakm\nsupset I$, i.e., for some $a\in I$,
$a\notin\frakm$. Consider the ideal generated by $(a)$. Since $a\in I$,
$(a)\neq R$ since $I$ is a proper ideal of $R$, in particular, since $a$ is
a nonunit. Consider the ideal $\frakm+(a)$. Since $a\notin\frakm$,
$\frakm\subset\frakm+(a)$. But since $\frakm$ is maximal, it follows that
$\frakm+(a)=R$. Hence, there exists an element $m\in\frakm$ such that
$m+ra=1$ for some $r\in r$. Then we have $m=1-ra$. Since $-r\in R$ and
$a\in I$, we have $-ra\in I$ so $m=1+(-ra)$ is a unit thus,
$\frakm=R$. This contradicts that $\frakm$ is a maximal ideals. Thus, $I$
is contained in every maximal ideal of $R$.
\end{proof}

\begin{problem}
Let $R$ be an integral domain and $F$ be its field of fractions. Let
$\frakp$ be a prime ideal in $R$ and
\[
R_\frakp\coloneqq
\left\{\,\tfrac{a}{b}\;\middle|\;a,b\in R,\,b\notin\frakp\,\right\}\subset F.
\]
Show that $R_\frakp$ has a unique maximal ideal.
\end{problem}
\begin{proof}
We will show that
\[
\frakp
R_\frakp\coloneqq\left\{\,\tfrac{a}{b}\;\middle|\;a\in\frakp,\,b\notin\frakp\,\right\}
\]
is the unique maximal ideal of $R$. We will show that $a/b\in R_\frakp$ is
a unit if and only if $a/b\notin\frakp R_\frakp$.

$\implies$ Suppose that $a/b$ is a unit. Then there exists an element
$a'/b'$ such that
\[
\left(\frac{a}{b}\right)\left(\frac{c}{d}\right)=\frac{ac}{bd}=\frac{1}{1}.
\]
That is, there exists an element $s\in R\minus\frakp$ such that
$s(ac-bd)=0$. Since $R$ is an integral domain, $s\neq 0$ so $ac-bd=0$
implies $ac=bd$. Since $b,d\notin\frakp$, $bd\notin\frakp$ (since $\frakp$
is prime) and, in particular, $ac\notin\frakp$ so $a/b\notin\frakp
R_\frakp$.

$\impliedby$ Conversely, suppose that $a/b\notin\frakp R_\frakp$. Then
$a\notin\frakp$. Thus, $b/a\in R_\frakp$ and
\[
\left(\frac{a}{b}\right)\left(\frac{b}{a}\right)=\frac{ab}{ba}=\frac{1}{1}.
\]
Thus, $a/b$ is a unit in $R_\frakp$.

Now, since $\frakp R_\frakp$ does not contain any units, it is a proper
ideal of $R_\frakp$. Morevore, for every $a/b\notin\frakp R_\frakp$,
$\frakp R_\frakp +(a/b)=R_\frakp$ so $\frak R_\frakp$ is a maximal ideal,
i.e., is not contained in any proper ideal of $R_\frakp$. Any other ideal
must contain a unit or is strictly contained in $\frakp R_\frakp$. Thus,
$\frakp R_\frakp$ is the unique maximal ideal of $R_\frakp$.
\end{proof}

\begin{problem}
Let $m$ and $n$ be relatively prime integers. Show that there is an
isomorphism $Z_{mn}^\times\cong Z_m^\times\times Z_n^\times$.
\end{problem}
\begin{proof}
Suppose $m$ and $n$ are relatively prime. Then $(m)+(n)=\bbZ$, i.e., $(m)$
and $(n)$ are comaximal. By the Chinese remainder theorem there is a ring
isomorphism
\[
Z_{mn}\cong Z_m\times Z_n.
\]
which gives an isomorphism of the group of units
\[
Z_{mn}^\times\cong\left(Z_m\times Z_n\right)^\times.
\]
Thus, it suffices to show that $\left(Z_m\times
  Z_n\right)^\times=Z_m^\times\times Z_m^\times$.

Suppose $(a,b)\in\left(Z_m\times Z_n\right)^\times$. Then $(a,b)$ is a unit
in $Z_m\times Z_n$, i.e., there exists $(c,d)$ such that
$(a,b)(c,d)=(1,1)$. But $(a,b)(c,d)=(1,1)$ if and only if $ac=1$ and
$bd=1$. Thus, $a\in Z_m^\times$ and $b\in Z_n^\times$ so $(a,b)\in
Z_m^\times\times Z_n^\times$. Conversely, if $(a,b)\in Z_m^\times\times
Z_n^\times$ then $a$ is a unit in $Z_m$ and $b$ is a unit in $Z_n$. Thus,
there exists elements $c\in Z_m$ and $d\in Z_n$ such that $ac=1$ and
$bd=1$ so $(a,b)(c,d)=(ac,bd)=(1,1)$. Thus, $(a,b)\in\left(Z_m\times
  Z_n\right)^\times$.
\end{proof}

\begin{problem}
Show that if $x$ is non-nilpotent in $R$ then a maximal ideal $\frakp$ of
$R$, which does not contain $x^n$ for $n=1,2,...$, is prime.
\end{problem}
\begin{proof}
I think what the professor had in mind was to prove this: ``Show that if
$x$ is non-nilpotent in $R$ then the ideal $\frakp$, which is maximal with
respect to not containing $x^n$ for any $n\in\bbZ$, is prime.''

This looks like a standard commutative algebra problem. Let
$S\coloneqq\left\{\,x^k\;\middle|\;k\geq 1\,\right\}$, i.e., the
multiplicative set generated by $x$ and suppose that $\frakp$ is an ideal
maximal with respect to $\frakp\cap S=\emptyset$. Seeking a contradiction
suppose $a,b\in R$ with $ab\in\frakp$ but $a,b\notin\frakp$. Then, the
ideals $\frakp+(a)$ and $\frakp+(b)$ contain $\frakp$ and therefore must
contain a power of $x$, say $x^m$ and $x^n$, respectively. Thus, we have
\[
x^mx^n=x^{m+n}\in(\frakp+(a))(\frakp+(b))\subset\frakp+(ab)\subset\frakp.
\]
But $\frakp$ is maximal with respect to not containing any power of
$x$. This is a contradiction. Thus, we must have $a\in\frakp$ or
$b\in\frakp$ which implies $\frakp$ is prime.
\end{proof}

\begin{problem}
Let $\bbQ$ be the field of rational numbers and
$D=\left\{\,a+b\sqrt{2}\;\middle|\;a,b\in\bbQ\,\right\}$.
\begin{enumerate}[label=(\alph*)]
\item Show that $D$ is a principal ideal domain.
\item Show that $\sqrt{3}$ is not an element of $D$.
\end{enumerate}
\end{problem}
\begin{proof}
(a) We prove the following stronger result (which is, incidentally, easier
to prove than what we are asked to prove): $D$ is a field (in fact, it is
the extension $\bbQ(\sqrt{2})$). Let $a+b\sqrt{2}\in D$ be a nonzero
element. To show that $a+b\sqrt{2}$ is a unit, it suffices to find an
inverse for it. Hence, we have
\[
\frac{1}{a+b\sqrt{2}}=\frac{a-b\sqrt{2}}{a^2-2b^2}=\frac{a}{a^2-2b^2}-\frac{b}{a^2-2b^2}\sqrt{2}.
\]
Note that $a^2-2b^2\neq 0$ if and only if $a^2=2b^2$, but this implies that
$a=\sqrt{2}b$ which is impossible since $\sqrt{2}\notin\bbQ$ so that the
above is indeed in $D$. Now, we have
\begin{align*}
(a+b\sqrt{2})\left(\frac{a}{a^2-2b^2}-\frac{b}{a^2-2b^2}\sqrt{2}\right)
&=\frac{1}{a^2-2b^2}\left(a^2+ab\sqrt{2}-2b^2+-ba\sqrt{2}\right)\\
&=\frac{a^2-2b^2}{a^2-2b^2}\\
&=1.
\end{align*}
Thus, $D$ is a field.
\\\\
(b) We shall proceed by contradiction. Suppose that $\sqrt{3}\in D$. Then
\[
\sqrt{3}=a+b\sqrt{2}
\]
for some $a,b\in\bbQ$. Squaring both sides, we have
\begin{align*}
3&=a^2+2b^2+2ab\sqrt{2}\\
3-a^2-2b^2&=2ab\sqrt{2}\\
\sqrt{2}&=\frac{3-a^2-2b^2}{2ab}.
\end{align*}
This implies that $\sqrt{2}\in\bbQ$, which is a contradiction.
\end{proof}

\begin{problem}
Show that if $p$ is a prime such that $p\equiv 1\pmod{4}$, then $x^2+1$ is
not irreducible in $\bbF_p[x]$.
\end{problem}
\begin{proof}
Since $p\equiv 1\pmod{4}$, $p=a^2+b^2$ for some integers $a$ and $b$. It
follows that $b\neq 0\pmod{p}$ or else $a=\sqrt{p}$ or $a^2+b^2>p$, a
contradiction. Thus $b$ is a unit in $\bbF_p$. We claim that $ab^{-1}$ is a
root of $x^2+1$. First note that
\[
(ab^{-1})^2+1=a^2b^{-2}+1.
\]
Since $a^2+b^2\equiv 0\pmod{p}$, it follows that $b^{-2}(a^2+b^2)\equiv
0\pmod{p}$, but $b^{-2}(a^2+b^2)=a^2b^{-2}+1$. Thus, $a^2b^{-2}+1=0$ in
$\bbF_p$. Thus, $a^2b^{-2}+1=0$ in $\bbF_p$ so $x^2+1$ has a root in
$\bbF_p[x]$ and hence, is reducible.
\end{proof}

\begin{problem}
Show that if $p$ is a prime such that $p\equiv 3\pmod{4}$, then $x^2+1$ is
irreducible in $\bbF_p[x]$.
\end{problem}
\begin{proof}
Note that $p-1\equiv 2\pmod{4}$. In particular, we see that $4\nmid p-1$
for all primes $p$ satisfying the conditions above. Now, consider
multiplicative subgroup $(\bbF_p[x])^\times\cong Z_{p-1}$ of
$\bbF_p[x]$, this is a cyclic group of order $p-1$. If $F_p^\times$ had an
element of order $4$ then, by Lagrange's theorem, $4\mid p-1$. But this is
false. Now suppose there exists $a\in\bbF_p$ such that $a^2=-1$. Then
$a^4=(-1)^2=1$. It follows that $a\neq 1$ and $a^3\neq 1$, so $a$ is an
element of order $4$ in $\bbF_p^\times$. Thus, $x^2$ does not have a root
in $\bbF_p[x]$. Since $x^2+1$ is of degree $2$, it follows that $x^2+1$ is
irreducible in $\bbF_p[x]$ for $p\equiv 3\pmod{4}$.
\end{proof}

\begin{problem}
Find a simpler description for each of the following rings:
\begin{enumerate}
\item $\bbZ[x]/(x^2-3,2x+4)$;
\item $\bbZ[i]/(2+i)$ $(i^2=-1)$.
\end{enumerate}
\end{problem}
\begin{proof}
\end{proof}

\begin{problem}
Show that $\bbZ[\sqrt{-13}]$ is not a principal ideal domain.
\end{problem}
\begin{proof}
It suffices to exhibit an ideal that is not generated by a single
element. To that end, consider the ideal generated by
\end{proof}

\begin{problem}
Let $D$ be a principal ideal domain. Prove that every nonzero prime ideal
of $D$ is a maximal ideal.
\end{problem}
\begin{proof}
\end{proof}

\begin{problem}
Prove or disprove that a nonzero prime ideal $P$ of a principal ideal
domain $R$ is a maximal ideal.
\end{problem}
\begin{proof}
\end{proof}

\begin{problem}
Consider the polynomial $f(x)=x^4+1$.
\begin{enumerate}[label=(\alph*)]
\item Use the Eisenstein Criterion to show that $f(x)$ is irreducible in
  $\bbZ[x]$.
\item Prove that $f(x)$ is reducible in $\bbF_p[x]$ for every prime
  $p$.
\end{enumerate}
\end{problem}
\begin{proof}
\end{proof}

\begin{problem}
Assume that $f(x)$ and $g(x)$ are polynomials in $\bbQ[x]$ and that
$f(x)g(x)\in\bbZ[x]$. Prove that the product of any coefficient of $f(x)$
with any coefficient of $g(x)$ is an integer.
\end{problem}
\begin{proof}
\end{proof}

\begin{problem}
Let $k$ be a field, $x,y$, indeterminates. Let $f(x)$ and $g(x)$ be
relatively prime polynomials in $k[x]$. Show that in the polynomial ring
$k(y)[x]$, $f(x)-yg(x)$ is irreducible.
\end{problem}
\begin{proof}
\end{proof}

%%% Local Variables:
%%% mode: latex
%%% TeX-master: "../MA553-Quals"
%%% End:

\section{Fields}
\begin{problem}
Let $F$ be a field with prime characteristic $\ch(F)=p$. Let $L/F$ be a
finite extension such that $p$ does not divide $[L:F]$. Show that $L/F$ is
a separable extension.
\begin{proof}
Seeking a contradiction, suppose that $L/F$ is not separable. The there
exists an element $\alpha\in L$ such that its minimal polynomial
$m_{\alpha,F}(X)$ is not separable, i.e., $m_{\alpha,F}$ has a multiple
root. But recall that an irreducible polynomial $g(X)$ is separable if
$\deg(D(g))=\deg(g)-1$. Thus, we must have
$\deg\left(D(m_{\alpha,F})\right)<\deg\left(m_{\alpha,F}\right)-1$ (since
for any polynomial $f$, $\deg(D(f))\leq \deg(f)-1$). But since $\Ch(F)=p$,
this is true only if $p\mid\deg(m_{\alpha,F})$. For suppose not. Then
$m_{\alpha,F}(X)=X^n+a_{n-1}X^{n-1}+\cdots+a_0$ and
\[
D(m_{\alpha,F})=nX^{n-1}+\text{some terms of lower degree.}
\]
so that $\deg(D(m_{\alpha,F}))=n-1=\deg(m_{\alpha,F})-1$. Hence, we have
$p\mid[F(\alpha):L]$ and by the tower theorem,
\[
[L:F]=[L:F(\alpha)][F(\alpha):L]
\]
implies that $p\mid[L:F]$. This is a contradiction. Thus, $L/F$ is
separable.
\end{proof}
\end{problem}

\begin{problem}
Let $\zeta_5$ be a primitive $5$-th root of unity, and denote
$\theta=\zeta_5+\zeta_5^{-1}$ as an element of the cyclotomic field
$\bbQ(\zeta_5)$. Show that the minimal polynomial of $\theta$ over $\bbQ$
is $m_{\theta,\bbQ}(X)=X^2+X-1$.
\end{problem}
\begin{proof}
Via some algebra, \textbf{:\textasciicircum)}, we have
\begin{align*}
(\zeta_5+{\zeta_5}^{-1})^2+(\zeta_5+{\zeta_5}^{-1})-1
&={\zeta_5}^2+2+{\zeta_5}^{-2}+\zeta_5+{\zeta_5}^{-1}-1,
\shortintertext{but since ${\zeta_p}^{-k}={\zeta_p}^{p-k}$ we have}
&={\zeta_5}^2+2+{\zeta_5}^3+\zeta_5+{\zeta_5}^4-1\\
&={\zeta_5}^4+{\zeta_5}^3+{\zeta_5}^2+\zeta_5+1\\
&=0.
\end{align*}
Thus, $m_{\theta,\bbQ}$ satisfies $\theta$. This implies that the
minimal polynomial of $\theta$ divides $m_{\theta,\bbQ}$. Therefore, to
show that the minimal polynomial of $\theta$ is in fact $m_{\theta,\bbQ}$
we must show that $m_{\theta,\bbQ}$ is irreducible.

To see that $m_{\theta,\bbQ}$ is irreducible we employ Eisetnstein's
criterion. Consider the shifted polynomial
\[
m_{\theta,\bbQ}(X+2)=(X+2)^2+(X+2)-1=X^2+4X+4+X+2-1=X^2+5X+5.
\]
By Eisenstein's criterion, $5\mid 5$ and $5\mid 5X$, but $5^2\nmid
5$. Thus, $m_{\theta,\bbQ}(X+2)$ is irreducible so $m_{\theta,\bbQ}(X)$ is
irreducible. Therefore, the minimal polynomial of $\theta$ is
$m_{\theta,\bbQ}$.

Now, since $\bbQ$ is characteristic $0$, $\Gal(\bbQ(\zeta_5)/\bbQ)\cong
(\bbZ/(5))^\times\cong Z_4$. Since $Z_4$ has a unique subgroup of order
$2$, by he fundamental theorem of Galois theory, $\bbQ(\theta)$ is the only
extension of degree $2$ under $\bbQ(\zeta_5)$. Similarly, $\bbQ$ is the
only other proper subfield since the only other subgroup of $Z_4$ is the
trivial subgroup.
\end{proof}

\begin{problem}
Prove or disprove the following: If $f(x),g(x)\in\bbQ[x]$ are irreducible
polynomials that have the same splitting field, then $\deg f=\deg g$.
\end{problem}
\begin{proof}
This is false. Consider the polynomial $f(X)=X^3-2$. The splitting field of
this polynomial is $\bbQ\left(\sqrt[3]{2},\zeta_3\right)$. However, by the
primitive element theorem, there exists
$\alpha\in\bbQ(\sqrt[3]{2},\zeta_3)$ such that
$\bbQ(\sqrt[3]{2},\zeta_3)=\bbQ(\alpha)$ and the
$\deg(m_{\alpha,\bbQ})=\left[\bbQ\left(\sqrt[3]{2},\zeta_3\right):\bbQ\right]=6$.
\end{proof}

\begin{problem}
Prove or disprove that every finite algebraic extension field of
$\bbF_{p^n}$ is Galois.
\end{problem}
\begin{proof}
The adjective \emph{algebraic} is redundant in the above for every finite
extension is necessarily algebraic.
\\\\
Let $F$ be a finite extension of $\bbF_{p^n}$. Then $F$ must be a finite
field of characteristic $p$ since $\bbF_p\subset\bbF_{p^n}\subset F$. By
the uniqueness theorem for finite fields, $F\cong\bbF_{p^m}$ for some
positive integer $m$. Hence, $\bbF_{p^m}/\bbF_p$ is Galois, being the
splitting field of the separable polynomial $X^{p^m}-X$.

By the fundamental theorem of Galois theory, since $F$ is Galois over
$\bbF_p$, $F$ is Galois over any subfield containing $\bbF_p$. Thus,
$F/\bbF_p$ is Galois.
\end{proof}

\begin{problem}
If $[K:\bbF_p]$ divides $[L:\bbF_p]$, does it follow that $K$ is isomorphic
to a subfield of $L$?
\end{problem}
\begin{proof}
Yes. Put $n\coloneqq\left[K:\bbF_p\right]$, $m\coloneqq[L:\bbF_p]$, and
suppose $n\mid m$. By the fundamental theorem for finite fields,
$K\cong\bbF_{p^n}$ and $L\cong\bbF_{p^m}$. Now, $\Gal(L/\bbF_p)\cong Z_m$
(generated by the Frobenius automorphism). Since $n\mid m$, $Z_m$ has a
subgroup of order $Z_{m/n}$. Thus, by the fundamental theorem of Galois
theory, $L$ has a subfield $E$ such that
\[
\left[E:\bbF_p\right]=\left[Z_m:Z_{m/n}\right]=m/(m/n)=n.
\]
Thus, by the fundamental theorem for finite fields, $E\cong\bbF_{p^n}\cong
K$.
\end{proof}

\begin{problem}
Let $\bbF_p$ be a finite field whose cardinality $p$ is prime. Fix a
positive integer $n$ which is not divisible by $p$, and let $\zeta_n$ be a
primitive $n$th root of unity. Show that
$\left[\bbF_p(\zeta_n):\bbF_p\right]=a$ is the least positive integer such
that $p^a\equiv 1\pmod{n}$. (\emph{Hint:} the Galois group of the extension
of $\bbF_p$ is generated by the Frobenius automorphism.)
\end{problem}
\begin{proof}
By the fundamental theorem of finitely fields,
$G\coloneqq\Gal(\bbF_p(\zeta_n)/\bbF_p)=\langle\sigma\rangle$ where $\sigma$
is the Frobenius automorphism. Since
$\left[\bbF_p(\zeta_n):\bbF_p\right]=a$, the order of $\sigma$ is
$a$. Since $\zeta_n$ generates $\bbF_p(\zeta_n)$, by the fundamental
theorem of Galois theory, the identity automorphism is the only
automorphism in $G$ which fixes $\zeta_n$: If $b$ is a positive integer
with $b<a$, then ${\zeta_n}^{p^b}=\sigma^b(\zeta_n)\neq\zeta_n$. Hence,
$p^b\nequiv 1\pmod{n}$.

Since $\sigma_a=\id_{\bbF_{p}(\zeta_n)}$, we have that
$\sigma_a(\zeta_n)=\zeta_n$. But $\sigma_a(\zeta_n)={\zeta_n}^{p^a}$. Hence
${\zeta_n}^{p^n}=\zeta_n$. Since the $n$th roots of unity form a cyclic
multiplicative group generated by $\zeta_n$ of order $n$, it follows from
${\zeta_n}^{p^a}=\zeta_n$ that $p^a\equiv 1\pmod{n}$.
\end{proof}

\begin{problem}
Fix a prime $p$, and consider the polynomial $f(x)=x^p-x-1$. Let
$\bbF_p(f)$ be the splitting field of $f(x)$ over $\bbF_p$. Let
$a\in\bbF_p(f)$ be a root of $f$. Show that $a\mapsto a+1$ defines an
automorphism of $\bbF_p(f)$. Show that
$\Gal(\bbF_p(f)/\bbF_p)\cong\bbZ_p$. Prove that $f(x)$ is irreducible in
$\bbZ[x]$. $\bbF_p(f)/\bbF_p$ is called an Artin--Schreier Extension.
\end{problem}
\begin{proof}
Since $\bbF_p$ is of characteristic $p$, the \emph{freshman's dream holds},
i.e.,
\[
(a+1)^p-(a+1)-1=a^p+1^p-a-1-1=a^p-a-1=0.
\]
Thus, $a+1$ is a root of $f$. Note that if $a\in\bbF_p$, then $0$ is a root
of $f$ since $a+1,a+2,\cdots a+(p-a)=0$ would be the roots of this
polynomial. But $f(0)=0^p-0-1=-1\neq 0$. Thus, $a\notin\bbF_p$.

Now, we note that $\bbF_p(a)=\bbF_p(a+1)$: $1,a\in\bbF_p(a)$ so
$a+1\in\bbF_p(a)$ and $a,-1\in\bbF_p(a+1)$ so
$(a+1)-1=a\in\bbF_p(a+1)$. Thus,
\[
\bbF(a)=\bbF(a+1)=\bbF(a+2)=\cdots=\bbF(a+p-1).
\]
Since all of $a,a+1,...,a+p-1$ are roots of $f$, and all of these fields
are equal, $\bbF_p(a)=\bbF_p(f)$, i.e., $\bbF_p(a)$ is the splitting field
of $f$. Hence, any map $a\mapsto a+i$, for $0\leq i\leq p-1$, determines an
automorphism of $\bbF_p(f)$. Note that $a\mapsto a+i$ is just $i-1$
applications of the map $a\mapsto a+1$. hence,
$\Gal\left(\bbF_p(f)/\bbF_p\right)$ is cyclic generated by $a\mapsto
a+1$. Moreover, this is a group of order $p$ since $a+p=a$ but $a+i\neq a$
for all $1\leq i\leq p-1$. Thus, $\Gal\left(\bbF_p(f)/\bbF_p\right)\cong
Z_p$.

Since $f$ is a monic polynomial of degree
$p=\left[\bbF_p(a):\bbF_p\right]$, with $a$ as root, it follows that
$f(X)=m_{\alpha,\bbF_p}(X)$.

Hence, $f$ is irreducible in $\bbF_p[X]$.

Since $\bbZ$ is an integral domain, $f$ is a nonconstant monic polynomial
in $\bbZ[X]$ and $(p)$ is a proper ideol of $\bbZ$, and $\bar f=f$ is
irreducible in $\bbF_p[X]\cong(\bbZ/(p))[X]$, if $f$ is irreducible in
$\bbZ[X]$.
\end{proof}

\begin{problem}
Let $x$ and $y$ be indeterminates over the field $\bbF_2$. Prove that there
exists infinitely many subfields of $L=\bbF_2(x,y)$ that contain the field
$K=\bbF_2(x^2,y^2)$.
\end{problem}
\begin{proof}
\end{proof}

\begin{problem}
Let $K/F$ be an algebraic field extension. If $K=F(a)$ for some $a\in K$,
prove that there are only finitely many subfields of $K$ that contain $F$.
\end{problem}
\begin{proof}
\end{proof}

\begin{problem}
Let $p$ be a prime integer. Recall that a field extension $K/F$ is called a
$p$-extension if $K/F$ is Galois and $[K:F]$ is a power of $p$. If $K/F$
and $L/K$ are $p$-extensions, prove that the Galois closure of $L/F$ is a
$p$-extension.
\end{problem}
\begin{proof}
\end{proof}

\begin{problem}
Give an example where $K/F$ and $L/K$ are $p$-extensions, but $L/F$ is not
Galois.
\end{problem}
\begin{proof}
\end{proof}

\begin{problem}
Let $L/\bbQ$ be the splitting field of the polynomial $x^6-2\in\bbQ[x]$.
\begin{enumerate}[label=(\alph*)]
\item If $a$ is one root of $x^6-2$, draw the subfield lattice of the
  extension $\bbQ(a)$ over $\bbQ$.
% \begin{proof}[Subfield lattice]
% Alright. Let's crank it out! Let $f(x)=x^6-2$. The
% splitting field of this polynomial is just
% $L=\bbQ(\sqrt[6]{2},\zeta_6)$ with index
% $[L:\bbQ]=6\cdot\varphi(6)=6\cdot 2=12$. First, we'll
% calculate the Galois group of this extension. To that end,
% it suffices to look at the automorphisms on the generators
% of $L$.

% Clearly
% \[\Gal(L/\bbQ)=\left<\,
% \sigma,\tau\;\middle|\;
% \sigma^6=\tau^2=1,\,\tau\sigma=\sigma^5\tau\, \right>,\]
% where
% \begin{align*}
% \sigma
% &\colon
% \begin{cases}
% \sqrt[6]{2}&\longmapsto\zeta_6\sqrt[6]{2},\\
% \zeta_6&\longmapsto\zeta_6,
% \end{cases},
% &\tau
% &\colon
% \begin{cases}
% \sqrt[6]{2}&\longmapsto\sqrt[6]{2},\\
% \zeta_6&\longmapsto\zeta_6^5.
% \end{cases}
% \end{align*}
% Clearly $\sigma^6=\tau^2=1$. What is less trivial is
% showing $\sigma\tau=\tau\sigma^5$. Observe
% \begin{align*}
% \sigma^5
% &\colon
% \begin{cases}
% \sqrt[6]{2}&\longmapsto\zeta_6^5\sqrt[6]{2},\\
% \zeta_6&\longmapsto\zeta_6,
% \end{cases},\\
% \sigma\tau
% &\colon
% \begin{cases}
% \sqrt[6]{2}&\longmapsto\zeta_6\sqrt[6]{2},\\
% \zeta_6&\longmapsto\zeta_6^5,
% \end{cases},
% &\tau\sigma^5
% &\colon
% \begin{cases}
% \sqrt[6]{2}&\longmapsto(\zeta_6^5)^5\sqrt[6]{2}
% =\zeta_6\sqrt[6]{2},\\
% \zeta_6&\longmapsto\zeta_6^5.
% \end{cases}
% \end{align*}
% Thus $\Gal(L/\bbQ)\cong D_{12}$. From here, we simply use
% the Fundamental Theorem of Galois Theory and observe the
% correspondence between subfields of $L$ and subgroups of
% $D_{12}$. (If only I knew the subgroup lattice of
% $D_{12}$).
% \end{proof}
\item Give generators for each subfield $K$ of $L$ for which
  $[K:\bbQ]=2$. How many are there?
% \begin{proof}[Solution]
% There is at least one and it corresponds to the subgroup
% $\langle \sigma \rangle\leq D_{12}$ whose index
% $[D_{12}:\langle \sigma \rangle]=2$. Therefore, the only
% subfield is $K=\bbQ(\zeta_6)=\bbQ(\sqrt{-3})$ (a degree $2$
% extension over $\bbQ$).at
% \end{proof}
\item Give generators for each subfield $K$ of $L$ for which
  $[K:\bbQ]=3$. How many are there?
\item Give generators for each subfield $K$ of $L$ for which
  $[K:\bbQ]=4$. How many are there?
\item How many subfields $K$ of $L$ have index $[L:K]=2$?
% \begin{proof}[Solution]
% This is also has at least one such subfield corresponding
% to the subgroup $\langle \tau \rangle\leq D_{12}$. The
% field is $\bbQ(\sqrt[6]{2})$. The extension to $L$ is
% certainly degree $2$.
% \end{proof}
\end{enumerate}
\end{problem}

\begin{problem}
Give an example of a field $F$ having characteristic $p>0$ and irreducible
monic polynomial $f(x)\in F[x]$ that has a multiple root.
\begin{proof}

\end{proof}
\end{problem}

\begin{problem}
Let $f$ be an irreducible polynomial of degree $k$ over $\bbF_p$. Find the
splitting field of $f$ and its Galois group.
\end{problem}
\begin{proof}
\end{proof}

\begin{problem}
Let $n$ be a positive integer and $d$ a positive integer that divides
$n$. Suppose $a\in\bbR$ is a root of the polynomial
$x^n-2\in\bbQ[x]$. Prove that there is precisely one subfield $F$ of
$\bbQ(a)$ with $[F:\bbQ]=d$.
\end{problem}
\begin{proof}
\end{proof}

\begin{problem}
Let $a=\sqrt[3]{5-\sqrt{7}}$.
\begin{enumerate}[label=(\alph*)]
\item Find the minimal polynomial of $a$, and the conjugates of $a$.
\item Determine the Galois closure of $F$ of $\bbQ(a)$.
\item Show that $F/\bbQ$ is an extension by radicals.
\item Conclude that $\Gal(F/\bbQ)$ is solvable.
\end{enumerate}
\end{problem}
\begin{proof}
\end{proof}

\begin{problem}
Let $F$ be a field of characteristic $p>0$. Fix an element $c$ in
$F$. Prove that $f(x)=x^p-c$ is irreducible in $F[x]$ if and only if $f(x)$
has no roots in $F$.
\end{problem}
\begin{proof}
\end{proof}

\begin{problem}
Determine the Galois group of the splitting field over $\bbQ$ and all its
subfields for
\begin{enumerate}[label=(\alph*)]
\item $f(x)=x^3-2$
\item $f(x)=x^4+2$
\item $f(x)=x^4+4$
\item $f(x)=x^4+4x+2$
\end{enumerate}
\end{problem}
\begin{proof}
\end{proof}

\begin{problem}
Show that $\sqrt{2}\notin\bbQ(\sqrt[3]{2},\zeta_3)$, where
$\zeta_3^2+\zeta_3+1=0$.
\end{problem}
\begin{proof}
\end{proof}

\begin{problem}
Let $L/F$ be a Galois extension of degree $[L:F]=2p$, where $p$ is aan odd
prime.
\begin{enumerate}[label=(\alph*)]
\item Show that hhere exits a unique queadratic subfield $E$, i.e.,
  $F\subseteq E\subseteq L$ and $[E:F]=2$.
\item Does there exist a unique subfield $K$ of index $2$, i.e.,
  $F\subseteq E\subseteq L$ and $[E:F]=2$.
\end{enumerate}
\end{problem}
\begin{proof}
\end{proof}

\begin{problem}
Let $L/F$ be a Galois extension of degree $[L:F]=p^2$ for some prime
$p$. Let $K$ be a subfield satisfying $F\subset K\subset L$. Must $K/F$ be
a normal extension?
\end{problem}
\begin{proof}
\end{proof}

\begin{problem}
Let $L/F$ be the Galois closure of he separable algebraic field extension
$F(\theta)/F$. Let $p$ be a prime that divides $[L:F]$. Prove that there
exists a subfield $K$ of $L$ such that $[L:K]=p$ and $L=K(\theta)$.
\end{problem}
\begin{proof}
Since $p$ divides $[L:K]$, $[L:K]=pn$ for some positive integer
$n$.
\end{proof}
\begin{problem}
Suppose $L/\bbQ$ is a finite field extension with $[L:\bbQ]=4$. Is it
possible that there exist precisely two subfields $K_1$ and $K_2$ of $L$
for which $[L:K_i]=2$? Justify your answer.
\end{problem}
\begin{proof}
\end{proof}

%%% Local Variables:
%%% mode: latex
%%% TeX-master: "../MA553-Quals"
%%% End:


% \chapter{January 2007}
\begin{problem}
Let $(G,\cdot)$  be a group. Show that $G$ is Abelian whenever $\Aut(G)$ is
a cyclic group under composition.
\end{problem}
\begin{proof}
Suppose that $\Aut(G)$ is cyclic. Then $\Inn(G)<\Aut(G)$ is cyclic. But
$\Inn(G)\cong G/Z(G)$. Thus, $G$ is Abelian by the following lemma.
\begin{lemma}
Let $(G,\cdot)$ be a group. If $G/Z(G)$ is cyclic, then $G$ is Abelian.
\end{lemma}
\begin{proof}[Proof of lemma]
\renewcommand\qedsymbol{$\clubsuit$}
Suppose that $G/Z(G)$ is cyclic. Then $G/Z(G)=\langle \bar x \rangle$ for
some representative $x\in G$. This means that for any $g\in G$, we can
write $g=x^kz$ for some positive integer $k$, for some $z\in Z(G)$. Let
$g_1,g_2\in G$. Then, by the following obvious algebraic manipulations
\[
g_1g_2=x^{k_1}z_1x^{k_2}z_2=z_1x^{k_1+k_2}z_2=z_2x^{k_2+k_1}z_1=z_2x^{k_2}x^{k_1}z_1=(x^{k_2}z_2)(x^{k_1}z_1)=g_2g_1,
\]
we see that $G$ is Abelian.
\end{proof}
\end{proof}

\begin{problem}
Let $(G,\cdot)$ be an Abelian group. The \emph{torsion subgroup of $G$} is
defined as the collection of elements of finite order:
\[
\Tor(G)\coloneqq
\left\{\,g\in G\;\middle|\;\text{$g^m=e$ for some integer $m>0$}\,\right\}.
\]
\begin{enumerate}[noitemsep,label=(\alph*)]
\item Show that the quotient group $G/{\Tor(G)}$ is \emph{torsion free},
  i.e., it contains no nontrivial elements of finite order.
\item Show that $\Tor(G)$ is finite whenever $G$ is finitely generated. (Do
  not assume that $G$ is finite.)
\end{enumerate}
\end{problem}
\begin{proof}
(a) (Presumably the torsion subgroup is a normal subgroup of $G$.) Define
$T\coloneqq\Tor(G/{\Tor(G)})$. We will show that $T=\bar e$. It is clear
that $\langle\bar e\rangle\subset T$ thus, we need only show that $T\subset
\langle\bar e\rangle$, i.e., if $t\in T$ then $g=\bar e$. Let $\bar g\in
T$. Then $\bar g\in G/{\Tor(G)}$ and $\bar g^m=\bar e$ for some positive
integer $m$. But $\bar g^m=\bar e$ implies that $g^m\Tor(G)=\Tor(G)$, i.e.,
$g^m\in\Tor(G)$. Thus, $(g^m)^n=g^{mn}e$ for some positive integer
$n$. Thus, $g\in\Tor(G)$ so we must have $\bar g=\bar e$.
\\\\
(b) Suppose that $G$ is finitely generated. By the fundamental theorem of
finitely generated Abelian groups, $G\cong\bbZ^r{\times}
Z_{s_1}{\times}\cdots{\times}Z_{s_n}$ for positive integers
$r,s_1,...,s_n$. It suffices to show that $\mathbf{1}\times
Z_{s_1}\times\cdots\times Z_{s_n}=\Tor(G)$ (once we have demonstrated this,
note that $\left|\mathbf{1}\times Z_{s_1}\times\cdots\times
  Z_{s_n}\right|=s_1\cdots s_n<\infty$). It is clear that $\mathbf{1}\times
Z_{s_1}\times\cdots\times Z_{s_n}\subset\Tor(G)$ since every element of
$\mathbf{1}\times Z_{s_1}\times\cdots\times Z_{s_n}$ has finite order,
i.e., for any $(\mathbf{1},z_1,...,z_n)\in \mathbf{1}\times
Z_{s_1}\times\cdots\times Z_{s_n}$, we have
$z=(\mathbf{1},z_1,...,z_n)^{s_1\cdots s_n}=(\mathbf{1},1,...,1)$ (as a
consequence of Lagrange's theorem). Now, suppose $z\coloneqq
(\mathbf{z},z_1,...,z_n)\in\Tor(G)$. Then $z^m=(\mathbf{1},1,...,1)$ for
some positive integer $m$. Since every non-identity element of $\bbZ^r$ has
infinite order, $\mathbf{z}=\mathbf{1}$ and $s_i\mid k$ for all $i$. Thus
$z\in\mathbf{1}\times Z_{s_1}\times\cdots Z_{s_n}$. Thus,
$|\Tor(G)|=s_1\cdots s_n$ so $\Tor(G)$ is indeed finite.
\end{proof}

\begin{problem}
Let $(G,\cdot)$ be a group of order $|G|=351$. Show that $G$ is solvable.
\end{problem}
\begin{proof}
The best plan of attack is to use Sylow's theorem. First, let us factor the
order of $G$ into powers of primes, $|G|=351=3^3\cdot 13$. In light of this
factorization, it suffices to show that either $|\Syl_{13}(G)|=1$ or
$|\Syl_3(G)|=1$ and hence, the unique Sylow-$13$ (or Sylow-$3$) subgroup
will be a normal subgroup of $G$. By Sylow's theorem, $n_{13}\equiv
1\pmod{13}$ and $n_{13}\mid 3^3$. Thus, $n_{13}=1$ or $27$. Suppose
$n_{13}=27$. Then $G$ contains $12\times 27=324$ elements of order $13$ so
there are $351-324-1=26$ elements remaining. This implies that
$n_3=1$. Thus, $P_3\in\Syl_3(G)$ is the unique Sylow-$3$ subgroup of $G$
hence, is normal. Thus, $G\rhd P_3$ so $G/P_3$ is a group. Incidentally,
$G/P_3\cong Z_{13}$ hence, solvable and $P_3$ is a $p$-group, hence
solvable. Thus, $G$ is solvable.

On the other hand, if $n_{13}=1$ then $P_{13}\in\Syl_{13}(G)$ is the unique
Sylow-$13$ subgroup of $G$ hence, normal in $G$. Since $P_{13}$ is a
$p$-group, it is solvable. Moreover, $G/P_{13}$ is a group of order $3^3$,
i.e., a $p$-group, hence, solvable. Thus, $G$ is solvable.

In either case, we have shown that $G$ must be solvable.
\end{proof}

\begin{problem}
Let $(G,\cdot)$ be a group, and $H<G$ a subgroup of finite index. Show that
there exists a normal subgroup $N\lhd G$ contained in $H$ which is also of
finite index. (Do not assume that $G$ is finite.)
\end{problem}
\begin{proof}
Suppose $H<G$ is a subgroup of finite index, i.e., $H$ partitions $G$ into
a finite number of cosets, say $G/H\coloneqq
\{H,g_1H,...,g_{k-1}H\}$. Define a homomorphism$\varphi\colon G\to
S_{G/H}$ by $g\mapsto gH$ (this is clearly a homomorphism: take $g_1,g_2\in
G$ then
$\varphi(g_1g_2)=g_1g_2H=(g_1H)(g_2H)=\varphi(g_1)\varphi(g_2)$). Thus,
$\ker\varphi\lhd G$ of finite index (in particular, by the 1st isomorphism
theorem and Lagrange's theorem $|G:\ker\varphi|\mid
|S_{G/H}|=|S_k|=k!$). Thus, it suffices to show that $\ker\varphi<H$. But
this is clear since, if $g\in\ker\varphi$ then $gH=H$ hence, $g\in H$.
\end{proof}

\begin{problem}
Let $(G,\cdot)$ be a finite group, and $\varphi\colon G\to G$ be a group
homomorphism. Show that for all normal Sylow $p$-subgroups $P\lhd G$ we
have $\varphi(P)<P$.
\end{problem}
\begin{proof}

\end{proof}

\begin{problem}
Let $(R,+,\cdot)$ be a commutative ring with $1\neq 0$.
\begin{enumerate}[noitemsep,label=(\alph*)]
\item Show that $R$ is an integral domain if and only if $(0)$ is a prime
  ideal.
\item Show that $R$ is a field if and only if $(0)$ is a maximal ideal.
\end{enumerate}
\end{problem}
\begin{proof}
\end{proof}

\begin{problem}
let $(R,+,\cdot)$ be a unique factorization domain. Choose an irreducible
element $p\in R$, and define the \emph{localization at $p$} as the ring of
fractions $R_p=D^{-1}R$ with respect to the multiplicative set
$D=R-(p)$. Show that $R_p$ is a principal ideal domain.
\end{problem}
\begin{proof}
\end{proof}

\begin{problem}
Let $(F,+,\cdot)$ be a field, and $F(\theta)/F$ be a finite, separable
extension. Let $L$ be the splitting field of the minimal polynomial
$m_{\theta,F}(x)\in F[x]$. Prove that for every prime $p$ dividing the
degree $[L:F]$, there exists a field $K$ such that $F\subset K\subset L$,
$[L:K]=p$, and $L=K(\theta)$.
\end{problem}
\begin{proof}
\end{proof}

\begin{problem}
Let $(\bbF_p,+,\cdot)$ be a finite field whose Cardinality $p$ is
prime. Fix a positive integer $n$ which is not divisible by $p$, and let
$\zeta_n$ be a primitive $n$th root of unity. Show that
$\left[\bbF_p(\zeta_n):\bbF_p\right]=\alpha$  is the least positive integer
such that $p^\alpha\equiv 1\pmod{n}$.
\end{problem}
\begin{proof}
\end{proof}

\begin{problem}
Prove that the Galois group of the splitting field over $\bbQ$ of
$f(x)=x^4+4x^2+2$ is a cyclic group.
\end{problem}
\begin{proof}
\end{proof}

%%% Local Variables:
%%% mode: latex
%%% TeX-master: "../MA553-Quals"
%%% End:

% \chapter{Spring 2008}
\begin{problem}
Let $(G,\cdot)$ be a group, $(H,+)$ be an Abelian group, and
$\varphi\colon G\to H$ be a group homomorphism. If $N$ is a subgroup such
that $\ker\varphi<N<G$, show that $N\lhd G$ is a normal subgroup.
\end{problem}
\begin{proof}
Let $N$ be a subgroup of $G$ containing $\ker\varphi$. Then we must show
that for any $g\in G$, $gNg^{-1}\subset N$. First we observe that, since
$\ker\varphi\lhd G$, then $\ker\varphi\lhd N$ since for any $g\in N$, $g$
is also in $G$ so that $g(\ker\varphi)g^{-1}=\ker\varphi\subset N$. Thus,
$\ker\varphi\lhd N$. By the first isomorphism theorem\footnote{Theorem 16
  of Dummit and Foote \S3, p.\,99.}, $G/\ker\varphi\cong H$ hence,
$G/\ker\varphi$ is Abelian. Moreover, $N/\ker\varphi<G/\ker\varphi$ hence,
$N/\ker\varphi\lhd G/\ker\varphi$. It follows immediately from the lattice
isomorphism theorem\footnote{Theorem 20 of Dummit and Foote \S3, p.\,99.} (this
is essentially the UMP of the quotient by a group) that $N\lhd G$.
\end{proof}
\begin{problem}
Let $(G,\cdot)$ be a finite Abelian group of even order, i.e., $|G|=2k$ for
some $k\in\bfN$.
\begin{enumerate}[noitemsep,label=(\alph*)]
\item For $k$ odd, show that $G$ has exactly one element of order $2$.
\item Does the same happen for $k$ even? Prove or give a counterexample.
\end{enumerate}
\end{problem}
\begin{proof}
(a) This problem is most easily proven using Cauchy's
theorem\footnote{Theorem 11 of Dummit and Foote \S3, p.\,93}. Suppose that
$k$ is odd. If $k=1$, $G\cong Z_2$ and we are done ($Z_2$ contains only one
nontrivial element and its order is $2$). Otherwise $k>2$. Then by Cauchy's
theorem we are guaranteed that there exists an element $g\in G$ of order
$2$. Suppose $h$ is another element (distinct from $g$) of order $2$. Since
$2$ is the smallest prime number dividing the order of $G$, by a corollary
to Cayley's theorem\footnote{Corollary 5 of Dummit and Foote \S4, p.\,121},
$\langle  g \rangle$ is a normal subgroup of $G$ so $G/\langle g \rangle$
is a group. Moreover, since $h\neq g$, then $\bar h\neq\bar e$ and
$2\geq|\bar h|>1$ implies that $|\bar h|=2$. But $2\nmid k=|G/\langle g
\rangle|$ contradicting Lagrange's theorem. It follows that $G$ must have
exactly one element of order $2$.
\\\\
(b) No. Here is the simplest counterexample: Consider the direct product
$Z_2\times Z_2$. The elements $(1,0)$ and $(0,1)$ are elements of order
$2$, but are not equivalent.
\end{proof}
\begin{problem}
Let $(G,\cdot)$ be a finite group of odd order, and $H\lhd G$ be a normal
subgroup of prime order $|H|=17$. Show that $H<Z(G)$.
\end{problem}
\begin{proof}
Let $G$ act on $H$ by conjugation, i.e., the map $\varphi\colon G\times
H\to H$ defined by the rule $\varphi(g,h)\coloneqq ghg^{-1}$ determines a
group action on $H$. First, we verify that $\varphi$ indeed defines a group
action on $H$: First, observe that for $e_G\in G$ the identity element,
$\varphi(e_G,h)=e_Ghe_G^{-1}=h$; next, if $g_1,g_2\in G$ then
\[
\varphi(g_1,\varphi(g_2,h))=\varphi(g_1,g_2hg^{-1})=g_1g_2hg_2^{-1}g_1=g_1g_2h(g_1g_2)^{-1}=\varphi(g_1g_2,h).
\]
Lastly, $\varphi$ is clearly well-defined in the sense $\varphi(g,h)\in H$
for all $g\in G$, $h\in H$. Thus, $\varphi$ is a group action. Now, let us
ask what the kernel of this action is. Thus group action $\varphi$, induces
a group homomorphism $\varphi'\colon G\to\Aut(H)$ given by
$\varphi'(g)\coloneqq\Eval(\varphi,g)$. Now, since $|H|=17$, $H\cong
Z_{17}$, hence is cyclic. Thus, $\Aut(H)\cong(\bfZ/17\bfZ)^{\times}\cong
Z_{16}$. Now, since $|\varphi'(G)|\mid |G|$, $|\varphi'(G)|$ is odd. But
$\varphi'(G)<\Aut(H)$ so, by Lagrange's theorem, $|\varphi'(G)|\mid
16$. Thus, $|\varphi'(G)|=1$, i.e., $\varphi'$ is the trivial homomorphism,
i.e., $\varphi(g,h)=ghg^{-1}=h=\varphi(1,h)$. Thus, $H<Z(G)$.
\end{proof}
\begin{problem}
Let $(G,\cdot)$ be a finite group. Show that there exists a positive
integer $n$ such that $G$ is isomorphic to a subgroup of $A_n$, the
alternating group on $n$ letters. [\emph{Hint:} Show that $A_n$ contains a
copy of $S_{n-1}$ when $n\geq 3$.]
\end{problem}
\begin{proof}
% Let $n\coloneqq |G|$. If $n=1$ or $2$ we are done as $1$ (the trivial
% group) and $Z_2$ are exactly $A_1$ and $A_2$. Now suppose $n\geq 3$. By
% Cayley's theorem, $G$ imbeds into $S_n$.
Let $n-2\coloneqq |G|$. If $n-2=1$ or $2$, $G\cong 0$ (the trivial group) or
$G\cong Z_2$, both of which are exactly $A_1$ and $A_2$. Suppose $n-2\geq
3$. By Cayley's theorem, $G$ imbeds into $S_{n-1}$. Now, define a
homomorphism
\[
\varphi(\sigma)\coloneqq
\begin{cases}
\sigma&\text{if $\sigma$ is even}\\
\sigma(n+1\;n+2)&\text{if $\sigma$ is odd}
\end{cases}.
\]
We check that this is in fact a homomorphism. Let $\sigma,\tau\in G$. Then
\[
\varphi(\sigma\tau)=
\begin{cases}
\sigma\tau&\text{if $\sigma\tau$ is even}\\
\sigma\tau(n+1\;n+2)&\text{if $\sigma\tau$ is odd}
\end{cases}.
\]
But $\sigma\tau$ is odd if and only if $\sigma$ or $\tau$ is odd and
$\sigma\tau$ is even if and only if $\tau$ is even.
\end{proof}
\begin{problem}
Let $(G,\cdot)$ be a group of order $|G|=200$.
\begin{enumerate}[noitemsep,label=(\alph*)]
\item Show that $G$ is solvable.
\item Show that $G$ is the semidirect product of two $p$-subgroups.
\end{enumerate}
\end{problem}
\begin{proof}
(a) First we factor the order of the group $G$, $|G|=200=2^3\cdot
5^2$. Now we will make use of Sylow's theorem to show that $G$ has at least
one normal $p$-subgroup.
\\\\
(b)
\end{proof}
\begin{problem}
Let $(R,+,\cdot)$ and $(S,+,\cdot)$ be commutative rings with $1\neq 0$,
and let $\varphi\colon R\to S$ be a surjective ring homomorphism. Assuming
that $R$ is local, i.e., it has a unique maximal ideal, show that $S$ is
also local.
\end{problem}
\begin{proof}
\end{proof}
\begin{problem}
Let $(R,+,\cdot)$ be a principal ideal domain.
\begin{enumerate}[noitemsep,label=(\alph*)]
\item Show that every maximal ideal in $R$ is a prime ideal.
\item Must every prime ideal in $R$ be a maximal ideal? Prove or give a
  counterexample.
\end{enumerate}
\end{problem}
\begin{proof}
\end{proof}
\begin{problem}
Let $L/F$ be a Galois extension of degree $[L:F]=2p$ where $p$ is an odd
prime.
\begin{enumerate}[noitemsep,label=(\alph*)]
\item Show that there exists a unique quadratic subfield $E$, i.e.,
  $F\subset E\subset L$ and $[E:F]=2$.
\item Does there exist a unique subfield $K$ of index $2$, i.e., $F\subset
  K\subset L$ and $[L:K]=2$? Prove or give a counterexample.
\end{enumerate}
\end{problem}
\begin{proof}
\end{proof}
\begin{problem}
Fix a prime $p$, and consider the Artin--Schreier polynomial
$f(x)=x^p-x-1$.
\begin{enumerate}[noitemsep,label=(\alph*)]
\item Let $\bfF_p(f)$ be the splitting field of $f(x)$ over $\bfF_p$. Show
  that $\Gal\left(\bfF_p(f)/\bfF_p\right)\cong Z_p$.
\item Prove that $f(x)$ is irreducible in $\bfZ[x]$.
\end{enumerate}
\end{problem}
\begin{proof}
\end{proof}
\begin{problem}
Determine the Galois group of the splitting field over $\bfQ$ of
$f(x)=x^4+4$.
\end{problem}
\begin{proof}
\end{proof}


%%% Local Variables:
%%% mode: latex
%%% TeX-master: "../MA553-Quals"
%%% End:

\chapter{MA 553: Midterm, Fall 2015}
In full detail now:
\begin{problem}
\begin{enumerate}[label=(\alph*)]
\item Show, for any abelian group, the map $x\mapsto x^{-1}$ is an
  automorphism.
\item Show, for any $n$, the dihedral group $D_{2n}$ of order $2n$,
  satisfies $D_{2n}\cong Z_2\ltimes Z_n$.
\end{enumerate}
\end{problem}
\begin{proof}
(a) Let $G$ be an abelian group and define the map $\varphi\colon G\to G$
by $\varphi(x)\coloneqq x^{-1}$. Then, for any $x,y\in G$, we have
\begin{align*}
\varphi(xy)&=(xy)^{-1}\\
           &=y^{-1}x^{-1}\\
           &=x^{-1}y^{-1}\\
           &=\varphi(x)\varphi(x).
\end{align*}
Hence, $\varphi$ is a homomorphism.

Next, we will show that $\varphi$ is in fact an automorphism. To that end,
we must show that $\varphi$ is one-to-one and onto.

First, we show $\varphi$ is one-to-one. Let $x\in\ker\varphi$. Then
$\varphi(x)=x^{-1}=e$. Then we have $x^{-1}x=x$. But $x^{-1}x=e$ so
$x=e$. Thus, $\ker\varphi=\{e\}$ and $\varphi$ must be injective.

To see than that $\varphi$ is onto, take $x\in G$ then
$\varphi(x^{-1})=\left(x^{-1}\right)^{-1}=x$. Thus, $\varphi$ is surjective
and we conclude that $\varphi\in\Aut(G)$.
\\\\
(b) Recall that the dihedral group of order $2n$ is the group
\[
G\coloneqq
D_{2n}=
\left<\,r,s\;\middle|\;\text{$r^n=s^2=e$ and $srs^{-1}=r^{-1}$}\,\right>.
\]
Now, note that the subgroup generated by $r$, $K\coloneqq\langle r\rangle$,
is order $n$ hence, $K\lhd G$ since $[G:H]=2$ is the smallest prime
dividing the order of $G$. Let $H\coloneqq\langle s\rangle$. This is a
subgroup of order $2$. Note that $H\cap K=\{e\}$ and $HK<G$ since $K$ is
normal in $G$. Moreover, $|HK|=|H||K|/|H\cap K|=2n=|G|$ so $HK=G$ so we
have $G=H\ltimes K$. Moreover, since $H$ and $K$ are cyclic of order $2$
and $n$, respectively, we have $H\cong Z_2$ and $K\cong Z_n$ so $G\cong
Z_2\ltimes Z_n$.
\end{proof}

\begin{problem}
Show that there is no simple group of order $306=2\cdot 3^2\cdot 17$.
\end{problem}
\begin{proof}
Suppose $G$ is a finite group of order $306=2\cdot 3^2\cdot 17$. We will
show that one of $n_2$, $n_3$, or $n_{17}$ equals $1$.

By Sylow's theorem, $n_p\equiv 1\pmod{p}$ and $n_p\mid m$ where
$|G|=p^\alpha m$. Thus, we have:
\begin{itemize}[noitemsep]
\item $n_2=1$, $3$, $3^2$, $17$, $3\cdot 17$, or $3^2\cdot 17$;
\item $n_3=1$, $34$;
\item $n_{17}=1$, $18$.
\end{itemize}
Seeking a contradiction, suppose that none of $n_2$, $n_3$, or $n_{17}$
equal $1$. Then, at least, $n_2=3$, $n_3=34$, and $n_{17}$. This means that
there are $1+3+16\cdot 18=302$ elements of order $1$, $2$, and $17$. But
there are at least $8$ elements of order $3$ in the remaining Sylow
$3$-subgroups, pushing this total to $310$ which is absurd. Thus, at least
one of $n_2$, $n_3$, or $n_{17}$ equals $1$.
\end{proof}

\begin{problem}
Suppose $R$ is a ring with identity, and $I$, $J$, and $K$ are (two-sided)
ideals of $R$ with $K\subset I\cup J$. Prove that either $K\subset I$ or
$K\subset J$.
\end{problem}
\begin{proof}
We shall proceed by contradiction. Suppose that $K\nsubset I$ and
$K\nsubset J$. Then there exists elements $a,b\in K$ such that $a\notin I$
and $b\notin J$. Now, consider the element $a-b\in K$. Since $K\subset
I\cup J$, then $a-b\in I$ or $a-b\in J$. Without loss of generality,
suppose that $a-b\in I$. Then $(a-b)+b=a\in I$ since $I$ is additively
closed. This is a contradiction. Thus, $K\subset I$ or $K\subset J$.
\end{proof}

\begin{problem}
Let $R$ and $S$ be rings and suppose that $\varphi\colon R\to S$ is a ring
homomorphism. Let $I$ be an ideal of $R$ and $J$ and ideal of $S$.
\begin{enumerate}[label=(\alph*)]
\item Show that $\varphi^{-1}(J)\coloneqq\left\{\,r\in
    R\;\middle|\;\varphi(r)\in J\right\}$ is an ideal in $R$.
\item Show that if $\varphi$ is surjective, then
  $\varphi(I)\coloneqq\left\{\,\varphi(r)\;\middle|\;r\in I\,\right\}$ is
  an ideal in $S$.
\item Given an example where $\varphi$ is not surjective and $\varphi(I)$
  is not an ideal in $S$.
\end{enumerate}
\end{problem}
\begin{proof}
(a) We need to show two things: Let $r\in R$ and $a\in\varphi^{-1}(J)$ then
$\varphi(ra)=\varphi(r)\varphi(a)$, but $\varphi(a)\in J$ so
$\varphi(r)\varphi(a)\in J$. Thus, $ra\in\varphi^{-1}(J)$. Lastly, we show
$\varphi^{-1}(J)$ is an additive subgroup, namely, for
$a_1,a_2\in\varphi^{-1}(J)$, we have $\varphi(a_1),\varphi(a_2)\in J$ so
$\varphi(a_1)+\varphi(a_2)=\varphi(a_1+a_2)\in J$. Thus,
$a_1+a_2\in\varphi^{-1}(J)$. Thus, $\varphi^{-1}(J)$ is an ideal in $R$.
\\\\
(b) Suppose $\varphi$ is surjective. Then, for every element $s\in S$,
there exist an element $r\in R$ such that $s=\varphi(r)$. Now, let
$a\in\varphi(I)$ and $s\in S$. Then $\varphi(b)=a$ for some $b\in I$ and
$\varphi(r)=s$ for some $r\in R$. Thus,
$\varphi(rb)=sa\in\varphi(I)$. Lastly, if $a_1,a_2\in\varphi(I)$ then
$\varphi(b_1)=a_1$ and $\varphi(b_2)=b_2$ for $b_1,b_2\in I$ so $b_1+b_2\in
I$ implies that
$\varphi(b_1+b_2)=\varphi(b_1)+\varphi(b_2)\in\varphi(I)$. Thus,
$\varphi(I)$ is an ideal of $S$.
\\\\
(c) Consider the map $\varphi\colon Z_4\to Z_2\times Z_2$ given by the rule
$\varphi(s)=(s,s)$. This map is a homomorphism since for any $s_1,s_2\in
Z_4$, we have
\begin{align*}
\varphi(s_1+s_2)&=(s_1+s_2,s_1+s_2)&
\varphi(s_1s_2)&=(s_1s_2,s_1s_2)\\
                &=(s_1,s_1)+(s_2,s_2)&
               &=(s_1,s_1)(s_2,s_2)\\
                &=\varphi(s_1)+\varphi(s_2)&
               &=\varphi(s_1)\varphi(s_2).
\end{align*}
But note that $\varphi$ is not surjective since
$\varphi(Z_4)=\{(0,0),(1,1)\}$. Moreover, the latter is not an ideal since
for $(1,0)\in Z_2\times Z_2$, $(1,0)(1,1)=(1,0)\notin\varphi(Z_4)$.
\end{proof}

\begin{problem}
\begin{enumerate}[label=(\alph*)]
\item Let $R$ be a commutative ring with identity $1\neq 0$. Suppose that,
  for every $r\in R$, there is some $n=n_r\geq 2$ so that $r^n=r$. Prove
  that every prime ideal of $R$ is maximal.
\item Suppose $R$ is a unique factorization domain, $p\in R$ is
  irreducible, and $\frakp$ is a prime ideal with
  $0\subsetneq\frakp\subset(p)$. Show $\frakp=(p)$. (\emph{Hint:} Prove
  that $\frakp$ can be generated by irreducible elements.)
\end{enumerate}
\end{problem}
\begin{proof}
(a) Let $\frakp\in\Spec(R)$. Then $R/\frakp$ is an integral domain. Now,
let $r\in R\minus\frakp$ and $\pi\colon R\to R/\frakp$ be the canonical
projection. Put $\bar r\coloneqq\pi(r)$. Then since $r^n=r$ for some $n\geq
2$ we have
\[
\pi(r^n)=(\bar r)^n(\bar r)^n=\bar r=\pi(r).
\]
Thus, $\bar r(\bar r^{n-1}-\bar 1)=0$ implies $\bar r=\bar 0$ or $\bar
r^{n-1}=\bar 1$. But $r\notin\frakp$ so $\bar r\neq\bar 0$. Thus, $\bar
r^{n-1}=\bar 1$ and we see that $\bar r$ is a unit. Thus, $R/\frakp$ is a
field which implies that $\frakp$ is maximal.
\\\\
(b) First note that if $p$ is irreducible in $R$ then it is prime. We will
show that $\frakp$ contains a principal prime ideal. Let
$a\in\frakp$. Then, since $R$ is a UFD, we may write $a=p_1\cdots p_n$ for
$p_1,...,p_n$ irreducible in $R$. Hence, each $p_i$ is prime in $R$ and
$(p_i)$ is a prime ideal. Moreover, since $a=p_1\cdots p_n\in\frakp$,
$p_k\in\frakp$ for some $1\leq k\leq n$. Thus, $(p_k)\subset\frakp$. Hence,
we have $(p_k)\subset\frakp\subset(p)$. But this implies $p_k=rp$ for some
$r\in R$. Since $p_k$ is irreducible, $r$ must be a unit so $(p_k)=(p)$
which implies that $\frakp=(p)$.
\end{proof}

%%% Local Variables:
%%% mode: latex
%%% TeX-master: "../MA553-Quals"
%%% End:

\chapter{MA 553: Final, Fall 2015}
\begin{problem}
Let $G$ be a finite non-Abelian group, and let $Z(G)$ be the center of
$G$. Prove that $|Z(G)|\leq |G|/4$.
\end{problem}
\begin{proof}
Seeking a contradiction, suppose $4>[G:Z(G)]$. Since $Z(G)\lhd G$, we have
$G/Z(G)$ is a group of order $1$, $2$, or $3$. Thus, $G/Z(G)\cong Z_1$,
$Z_2$, or $Z_3$ all of which are cyclic. This implies that $G$ is
Abelian. This is a contradiction.
\end{proof}

\begin{problem}
Let
\[
G=\SL_2(\bfZ/(5))\coloneqq
\left\{\,\begin{bmatrix}a&b\\c&d\end{bmatrix}\;\middle|\;
\text{$a,b,c,d\in\bfZ/(5)$, and $ad-bc\equiv 1\pmod{5}$}\,\right\}.
\]
\begin{enumerate}[label=(\alph*)]
\item Show $|G|=120$.
\item Show
  $N\coloneqq\left\{\,\left[\begin{smallmatrix}1&b\\0&1\end{smallmatrix}\right]\;\middle|\;b\in\bfZ/(5)\,\right\}$
  is a Sylow $5$-subgroup of $G$.

Suppose $a,b,c$ have been chosen. Then $d\equiv
(ab-1)d^{-1}\pmod{5}$. Then, there are $5^2+4=28$ possible choices for
these and $4$ possible ways to choose fix one of $a,b,c,d$. Thus, there are
at least $4\cdot 28=2^2\cdot 3\cdot 7$.
\item Find the number of Sylow $5$-subgroups of $G$.
\end{enumerate}
\end{problem}
\begin{proof}
(a) We know the since of $\GL_2(\bfZ/(5))$. This is $(5^2-1)(5^2-5)=24\cdot
20=2^5\cdot 3\cdot 5$.
\\\\
(b) It suffices to show that the order of $N$ is $5$ since $1$ is the
largest exponent of $5$ dividing $120=2^3\cdot 3\cdot 5$. But this is
clear, since $N$ must satisfy $1-b\cdot 0\equiv 1\equiv 1\pmod{5}$ which is
true for any $b\in\bfZ$. Hence, there are $5$ elements in $N$. Thus, $N$ is
a Sylow $5$-subgroup.
\\\\
(c) By Sylow's theorem, there are $n_5=1$ or $6$. We will show that $N$ is
not normal in $\SL_2(\bfZ/(5))$ so that $n_5\neq 1$. Let
$\left[\begin{smallmatrix}a&b\\c&d\end{smallmatrix}\right]\in\SL_2(\bfZ/(5))$. Then,
for any matrix in $N$ we have
\[
\begin{bmatrix}
a&b\\c&d
\end{bmatrix}
\begin{bmatrix}
1&1\\0&1
\end{bmatrix}
\begin{bmatrix}
d&-b\\
-c&a
\end{bmatrix}
=
\begin{bmatrix}
1-ac&a^2\\-bc&1+ba
\end{bmatrix}
\]
is in $N$ if and only if $ac=ba=0$ and $-bc=0$. But $ad\equiv
1+bc\pmod{5}$. Implies $bc=0$ so $b=0$ or $c=0$ so either $b=0$ and $c=0$
or $c=0$. The former implies that $ad=1\equiv\pmod{5}$ so $a=d=1$. This
would imply that
$\left[\begin{smallmatrix}a&b\\c&d\end{smallmatrix}\right]$. Thus,
$N\centernot\nlhd\SL_2(\bfZ/(5))$ so $n_5=6$.
\end{proof}

\begin{problem}
Suppose $R$ is a UFD and $F$ is the quotient field of $R$. Let $f(X)\in
R[X]$ and suppose $f(X)$ factors as a product of lower degree polynomials
in $F[X]$. Show $f(X)$ factors as a product of lower degree polynomials in
$R[X]$.
\end{problem}
\begin{proof}
This is an important result called \emph{\textde{Gauß}'s lemma} and is proven in
Dummit and Foote more or less as follows:

Suppose $f(X)$ factors as $f(X)=g(X)h(X)$ for polynomials $g,h\in F[X]$
with $\deg(g),\deg(h)<\deg(f)$. Then each coefficient $\left\{a_i\right\}$,
$\left\{b_i\right\}$ of $g$ and $h$, respectively, is in $F$. Thus,
clearing denominators, we have $df(X)=g'(X)h'(X)$ for $g'(X),h'(X)\in
R[X]$. If $d$ is a unit in $R$ we are done since
$f(X)=d^{-1}df(X)=d^{-1}g'(X)h'(X)$.

Suppose $d$ is not a unit. Then, since $R$ is a UFD, we may write $d$ as
the product $d=d_1\cdots d_n$ of irreducible elements $d_i\in R$. Since
$d_1$ is irreducible and $R$ is a UFD, then $d_1$ is prime so the ideal
generated by $d_1$ is prime. Thus, $\left(R/(d_1)\right)[X]$ is a domain
and
\[
\bar 0=\overline{df(X)}=
{\bar d}\cdot \overline{f(X)}=\overline{g'(X)h'(X)}=
\overline{g'(X)}\cdot\overline{h'(X)}.
\]
Thus, either $\overline{g'(X)}=\bar 0$ or $\overline{h'(X)}=\bar 0$ since
$\left(R/(d_1)\right)[X]$ is a domain. Without loss of generality, suppose
$\overline{g'(X)}=0$. Then, $(1/d_1)g'(X)\in R[X]$ so, dividing over
$F$, we have $(d_2\cdots d_n)f(X)=\left((1/d_1)g'(X)\right)h'(X)$ in
$R[X]$. Proceeding recursively in this fashion until, we may arrive at
$f(X)=G(X)H(X)$ where $G(X),H(X)\in R[X]$. Since we reduced by elements in
the subring $R$, $\deg(G)=\deg(g)$ and $\deg(H)=\deg(h)$ so that $f(X)$
factors as a product of polynomials of lower degree in $R[X]$, as desired.
\end{proof}

\begin{problem}
Let $R$ be a commutative ring. Recall an element $a\in R$ is
\emph{nilpotent} if $r^n=0$ for some $n\geq 1$. Let $I=\left\{\,a\in
  R\;\middle|\;\text{$a$ is nilpotent}\,\right\}$.
\begin{enumerate}[label=(\alph*)]
\item Show $I$ is an ideal. (\emph{Hint:} To show $I$ is an additive
  subgroup, show if $x,y\in I$ there is an $N>0$ so that $(x-y)^N=0$ using
  the binomial expansion of $(x-y)^N$.)
\item Show $I$ is contained in any prime ideal of $R$.
\end{enumerate}
\end{problem}
\begin{proof}
(a) In fact, one can show that $I=\nil(R)=\bigcap_{\frakp\in\Spec(R)}\frakp$,
i.e., $I$ is the intersection of all prime ideals in $R$ hence, an ideal.

First, we show that $R$ is multiplicatively closed. Let $r\in R$ and $a\in
I$. Then $(ar)^n=a^nr^n$ since $R$ is commutative. But $r^n=0$, so
$(ar)^n=a^n\cdot 0=0$. Thus $ar\in I$.

Next, we show that it is additively closed. Suppose $a,b\in I$. Then
$a^m=0$ and $b^n=0$ for some positive integer $m$ and $n$. Suppose, without
loss of generality, that $n\geq m$. Let $N=n+m$. Then
\begin{align*}
(a+b)^N&=(a+b)^{n+m}\\
       &=\sum_{i=1}^{n+m}\tbinom{n+m}{i}a^ib^{n+m-i}.
\end{align*}
Now, note that if $k\geq n$, $x^k=0$ so $\binom{n+m}{k}a^kb^{n+m-k}=0$. On
the other hand, if $k<n$, $n+m-k>m$ so $b^{n+m-k}=0$ so
$\binom{n+m}{k}a^kb^{n+m-k}=0$. In either case, we see that
$\binom{n+m}{k}a^kb^{n+m-k}=0$ so $(a+b)^N=0$. Thus, $a+b\in I$. Hence,
$I$ is an ideal.
\\\\
(b) Let $\frakp$ be a maximal ideal of $R$. Now, since $\frakp$ is an ideal
of $R$, $0\in R$. Moreover, for any $a\in I$, $a^n=0$ for some positive
integer $n$. Thus, $a^n=0\in\frakp$. But $\frakp$ is a prime ideal. Thus,
$a\in\frakp$ or $a^{n-1}\in\frakp$. If the former, we are done. In the
later, $a^{n-1}\in\frakp$ so $a\in\frakp$ or $a^{n-2}\in\frakp$. Proceeding
recursively in this manner, we have $a\in\frakp$. Thus, $I\subset\frakp$,
as desired.
\end{proof}

\begin{problem}
Let $\alpha\in\bfC$ be algebraic over $\bfQ$, and let $f(X)\in\bfQ[x]$ be
its minimal polynomial. Let $\sqrt{\alpha}$ be a square root of $\alpha$,
and let $g(X)\in\bfQ[X]$ be its minimal polynomial.
\begin{enumerate}[label=(\alph*)]
\item Show $\deg f(X)$ divides $\deg g(X)$.
\item Show $\sqrt{\alpha}\in\bfQ(\alpha)$ if and only if $f(X^2)$ is
  reducible in $\bfQ[X]$.
\end{enumerate}
\end{problem}
\begin{proof}
(a) This follows directly from the tower of fields theorem. Let $\bfQ(f)$
denote the splitting field of $f$. Then, $\alpha\in\bfQ(f)$ so that
$\bfQ(g)\supset\bfQ(f)$. Thus, we have
\[
\left[\bfQ(g):\bfQ\right]=\left[\bfQ(g):\bfQ(f)\right]\left[\bfQ(f):\bfQ\right]=k\cdot\deg(f)
\]
Thus, $\deg(f)\mid\deg(g)$.
\\\\
(b) $\implies$ Suppose that $\sqrt{\alpha}\in\bfQ(\alpha)$. Then
$f({\sqrt{\alpha}}^2)=f(\alpha)=0$ hence, $f(X^2)$ has a root in $\bbQ$
hence, is reducible.

$\impliedby$ Conversely, suppose that $f(X^2)$ is reducible. Then, we may
write $f(X^2)=\prod_{i=1}^k f_i(X)$ where $f_i\in\bfQ[X]$ is
irreducible. Now, each of these factors, $f_i$, have degree less than $2n$
where $n\coloneqq\deg\left(f(X^2)\right)$. Suppose
\[
f_i(X)=X^k+a_{k-1}X^{k-1}+\cdots+a_0
\]
for $a_{k-1},...,a_0\in\bfQ$. Then
\[
f_i(\sqrt{\alpha})=\alpha^{k/2}+a_{k-1}\alpha^{(k-1)/2}+\cdots+a_0.
\]
\end{proof}

\begin{problem}
Let $f(X)=X^6+3\in\bfQ[X]$.
\begin{enumerate}[label=(\alph*)]
\item Let $\alpha$ be a root of $f(X)$. Prove $(\alpha^3+1)/2$ is a
  primitive 6th root of unity.
\item Determine the Galois group of $f(X)$ over $\bfQ$.
\end{enumerate}
\end{problem}
\begin{proof}
(a) To show that $(\alpha^3+1)/2$ is a $6$th root of unity, suffices to
show that $\Phi_6((\alpha^3+1)/2)=0$ where $\Phi_6$ is the $6$th cyclotomic
polynomial. Recall that we may derive the $n$th cyclotomic polynomial via
the formula
\[
X^n-1=\prod_{d\mid n}\Phi_d(X)
\]
so that
\[
X^6-1=\Phi_1(X)\Phi_2(X)\Phi_3(X)\Phi_6(X)=(X-1)(X+1)(X^2+X+1)
\]
and we have
\begin{align*}
\Phi_6(X)&=\frac{X^6-1}{(X-1)(X+1)(X^2+X+1)}\\
         &=X^2-X+1.
\end{align*}
Thus,
\begin{align*}
\Phi_6((\alpha^3+1)/2)
&=\tfrac{1}{4}(\alpha^3+1)^2-\tfrac{1}{2}(\alpha^3+1)+1\\
&=\tfrac{1}{4}\alpha^6+\tfrac{1}{2}\alpha^3+\tfrac{1}{4}
  -\tfrac{1}{2}\alpha^3-\tfrac{1}{2}+1\\
&=\tfrac{1}{4}\alpha^6+\tfrac{3}{4}\\
&=\tfrac{1}{4}(\alpha^6+3)\\
&=0.
\end{align*}
Thus, $(\alpha^3+1)/2$ is $6$th root of unity.

To show that $(\alpha^3+1)/2$ is in fact a primitive root of unity, we need
to show that $6$ is the smallest integer such that
$((\alpha^3+1)/2)^6=1$. And that is too much work.
\\\\
(b) Put $\zeta_6\coloneqq(\alpha^3+1)/2$. The roots of the polynomial
are $\sqrt[6]{3},\zeta_6\sqrt[6]{3},...,{\zeta_6}^5\sqrt[6]{3}$. Hence, the
splitting field of $f$ contains $\sqrt[6]{3}$ and a primitive sixth root of
unity $(\alpha^3+1)/2$. Since $\deg(\Phi_6)=2$, and
$\sqrt{3}\in\bfQ(\Phi_6)$, the minimal polynomial of $\sqrt[6]{3}$ over
$\bfQ(Phi_6)$ is $X^3-\sqrt{3}$. Hence, the degree of the extension
\[
\left[\bfQ(f):\bfQ\right]=\left[\bfQ(f):\bfQ(\Phi_6)\right]\left[\bfQ(\Phi_6):\bfQ\right]=3\cdot 2=6.
\]
Thus, the Galois group of $\bfQ(f)/\bfQ$ is order $6$.

Moreover, the Galois group acts transitively on the roots of $f$ so there
are automorphism of the splitting field fixing the subfields $\bfQ(\Phi_6)$
and $\bfQ$. These are the automorphism
\[
\sigma\colon\alpha\mapsto-\alpha\qquad\text{and}\qquad\tau\colon\alpha\mapsto\zeta_6\alpha.
\]
Note that $\sigma$ has order $2$ and $\tau$ has order $3$ so that
$\Gal(\bfQ(f)/\bfQ)\cong D_6$.
\end{proof}

\begin{problem}
Let $R\coloneqq(\bfZ/(3))[X]$. Consider the ideals $I_1\coloneqq(X^2+1)$,
and $I_2\coloneqq(X^2+X+2)$. For $i=1,2$ we set $F_i=R/I_i$.
\begin{enumerate}[label=(\alph*)]
\item Show $F_1$ and $F_2$ are fields.
\item Are $F_1$ and $F_2$ isomorphic? If not, why not, and if so give an
  isomorphism from $F_1$ to $F_2$.
\end{enumerate}
\end{problem}
\begin{proof}
(a) Recall by some theorem in chapter 13 that $F[X]/(f)$ is a field
if and only if $f$ is irreducible. Therefore, it suffices to show that
$X^2+1$ and $X^2+X+2$ are irreducible over $\bfZ/(3)$. To that end, since
the degree of these polynomials is two, it suffices to show that they have
no roots over $\bfZ/(3)$.

In the case of $X^2+1$, we have $0^2+1\neq 0$, $1^2+1=1\neq 0$, and
$2^2+1=4+1=1+1=2\neq 0$. Thus, $X^2+1$ is irreducible.

In the case of $X^2+X+2$, we have $0^2+0+2=2\neq 0$, $1+1+2=1\neq 0$, and
$4+2+2=8=2\neq 0$.

Thus, $F_1$ and $F_2$ are fields.
\\\\
(b) By the classification theorem for finite fields, both $F_1$ and $F_2$
are an extension over $\bfF_3=\bfZ/(3)$ of degree $2$ hence, both are
isomorphic to $\bfF_{3^2}$. In particular, they are isomorphic to each
other. Let $\alpha$ be a root of $X^2+1$ and $\beta$ be a root of
$X^2+X+2$. Then the map $\alpha\mapsto\beta$ which fixes $\bfF_3$ is an
isomorphism. It suffices to show that this is an injective
homomorphism. First, this is a homomorphism since for any $x,y\in F_1$,
if $x,y\in\bfF_3$, $\varphi(x+y)=x+y=\varphi(x)+\varphi(y)$. If one of
$x$ or $y$ not in $\bfF_3$, suppose $x$, then $x=\alpha^k+x'$ for
$x'\in\bfF_3$ so
\[
\varphi(\alpha^k+x'+y)=\beta^k+x'+y=\varphi(\alpha^k+x')+\varphi(y)
\]
etc., thus this is an isomorphism.

To see that this map is injective, note that $\ker\varphi=\{0\}$. Thus,
$\varphi$ is an isomorphism.
\end{proof}

\begin{problem}
Suppose $F$ is a field, $K=F(\alpha)$ is a Galois extension, with cyclic
Galois group generated by $\sigma(\alpha)\coloneqq\alpha+1$. Show that
$\Ch(K)=p\neq 0$, and $\alpha^p-\alpha\in F$.
\end{problem}
\begin{proof}
Suppose that the Galois group of $K$ is cyclic of order $n>1$. Then,
\[
\sigma^n(\alpha)=\alpha=\alpha+n.
\]
Thus, $0=\alpha-\alpha=n\in F$ so $\Ch(F)$ is prime since the order of a
field is always prime.

Lastly, note that $\alpha^p-\alpha=\alpha(\alpha^{p-1}-1)$ since $\alpha$
is the root of the polynomial $x^p-x$.
\end{proof}

%%% Local Variables:
%%% mode: latex
%%% TeX-master: "../MA553-Quals"
%%% End:

\chapter{Qualifying Exam, January 2000}
\begin{problem}
Find all groups of order $7\cdot 11^3$ which have a cyclic subgroup of
order $11^3$.
\end{problem}
\begin{proof}
\end{proof}

\begin{problem}
Let $R$ be a ring with identity $1$ and consider the following two
conditions:
\begin{center}
\begin{enumerate}[label=(\roman*)]
\item If $a,b\in R$ and $ab=0$, then $ba=0$;
\item If $a,b\in R$ and $ab=1$, then $ba=1$;
\end{enumerate}
\end{center}
\begin{enumerate}[label=(\alph*)]
\item Show that (i) implies (ii).
\item Show by example that (ii) does not imply (i).
\end{enumerate}
\end{problem}
\begin{proof}
\end{proof}

\begin{problem}
Let $F$ be a field. Suppose that $E/F$ is a Galois extension, and that
$L/F$ is an algebraic extension with $L\cap E=F$. Let $EL$ be the composite
field, i.e., the subfield of an algebraic closer $\bar F$ of $F$ generated
by $E$ and $L$.
\begin{enumerate}[label=(\alph*)]
\item Show $EL/L$ is a Galois extension.
\item Show that there is an injective homomorphism
\[\varphi\colon\Gal(EL/L)\hookrightarrow\Gal(E/F).\]
Find the fixed field of the image of $\varphi$.
\item Show that $[EL:L]=[E:F]$.
\item Give an example to show that the conclusion of (c) is false if we do
  not assume that $E/F$ is Galois.
\end{enumerate}
\end{problem}
\begin{proof}
\end{proof}

\begin{problem}
Let $G$ be a finite group. Let $p$ be a prime and suppose that $|G|=p^km$,
with $k\geq 1$ and $p\nmid m$. Let $X$ be the collection of all subsets of
$G$ of order $p^k$. Then $G$ acts on $X$ by left multiplication, i.e.,
$g\cdot A=\left\{\,ga\;\middle|\;a\in A\,\right\}$. For $A\in X$< denote by
$H_A$ the stabilizer in $G$ of $A$. Show that $|H_A|\mid p^k$.
\end{problem}
\begin{proof}
\end{proof}

\begin{problem}
Let $R=\bfZ+X\bfQ[X]\subset\bfQ[X]$ be the ring consisting of polynomials
with rational coefficients whose constant term is an integer.
\begin{enumerate}[label=(\alph*)]
\item Prove that $R$ is an integral domain, with units $1$ and $-1$.
\item Show that $x$ is not an irreducible element of $R$.
\item Let $(X)\coloneqq Rx$ be the ideal of $R$ generated by $X$. Describe
  $R/(X)$ and show that $R/(X)$ is not an integral domain. What can you
  conclude about $X$?
\end{enumerate}
\end{problem}
\begin{proof}
\end{proof}

%%% Local Variables:
%%% mode: latex
%%% TeX-master: "../MA553-Quals"
%%% End:

\chapter{Qualifying Exam, January 2011}
\begin{problem}
Let
\[
G=\SL_2(\bfZ/(5))\coloneqq
\left\{\,\begin{bmatrix}a&b\\c&d\end{bmatrix}\;\middle|\;
\text{$a,b,c,d\in\bfZ/(5)$, and $ad-bc\equiv 1\pmod{5}$}\,\right\}.
\]
\begin{enumerate}[label=(\alph*)]
\item Show $|G|=120$.
\item Show
\[
N\coloneqq\left\{\,\begin{bmatrix}1&b\\0&1\end{bmatrix}\;\middle|\;b\in\bfZ/(5)\,\right\}
\]
is a Sylow $5$-subgroup of $G$.
\item Find the number of Sylow $5$-subgroups of $G$.
\end{enumerate}
\end{problem}
\begin{proof}
\end{proof}

\begin{problem}
\begin{enumerate}[label=(\alph*)]
\item Let $G$ be a group, $H$ a subgroup of $G$ with $[G:H]=2$. Suppose $K$
  is a subgroup of $G$ of odd order. Show $K\subset H$.
\item Let $G$ be a finite group and suppose there is a sequence of
  subgroups
\[
  G_0\coloneqq G\supset G_1\supset G_2\supset\cdots\supset  G_n\coloneqq H,
\]
with $[G_i:G_{i+1}]=2$ for $i\in\{1,...,n-1\}$. Suppose $H$ has odd
order. Show $H\lhd G$.
\item Suppose $|G|=2^nm$, with $m$ odd. Suppose $G$ has a normal subgroup
  $H$ of order $m$. Show there is a sequence of subgroups $G_0\coloneqq
  G\supset G_1\supset\cdots\supset G_n\coloneqq H$, with $[G_i:G_{i+1}]=2$,
  for all $i$.
\end{enumerate}
\end{problem}
\begin{proof}
\end{proof}

\begin{problem}
Let $R$ be a commutative ring with identity $1\neq 0$, and let $I$ be an
ideal of $R$. Define $\rad(I)$ to be the intersection of all maximal ideals
containing $I$, with the convention $\rad(R)=R$. Let
$\sqrt{I}\coloneqq\left\{\,r\in R\;\middle|\;\text{$r^n\in R$ for some
    $n>0$}\,\right\}$.
\begin{enumerate}[label=(\alph*)]
\item Prove $\rad(I)$ is an ideal of $R$ containing $I$.
\item Prove $\sqrt{I}\subset\rad(I)$.
\item Let $F$ be a field, set $R=F[X]$, and let $I=(f)$, for some nonzero
  polynomial $f(X)\in R$. Describe $\rad(I)$ in this intstance.
\end{enumerate}
\end{problem}
\begin{proof}
\end{proof}

\begin{problem}
Let $S$ be the subring of $\bfC[X]\times\bfC[Y]$ consisting of pairs
$(f,g)$ with $f(0)=g(0)$.
\begin{enumerate}[label=(\alph*)]
\item Let $\varphi\colon\bfC[X,Y]\to S$ be defined by $\varphi(h)=(f,g)$,
  where $f(X)=h(x,0)$, and $g(Y)=g(0,Y)$. Prove $\varphi$ is a surjective
  homomorphism.
\item Prove $\bfC[X,Y]/(X,Y)\cong S$.
\item Use (b) to describe the prime ideals of $S$. Be sure to justify your
  answer.
\end{enumerate}
\end{problem}
\begin{proof}
\end{proof}

\begin{problem}
Let $p$ be a prime, let $F=\bfF_p$ be the field of $p$ elements and
$K=\bfF_{p^{10}}$ be the unique extension of $F$ with $p^{10}$ elements.
\begin{enumerate}[label=(\alph*)]
\item Find all subfields of $K$. Make sure to justify your answer.
\item Find a formula for the number of monic irreducible polynomials of
  degree $10$ in $F[X]$. Justify your answer.
\end{enumerate}
\end{problem}
\begin{proof}
\end{proof}

\begin{problem}
Let $f(X)=(X^2-3)(X^3-7)\in\bfQ[X]$. Let $K$ be the splitting field of
$f(X)$ over $\bfQ$.
\begin{enumerate}[label=(\alph*)]
\item Find the degree of $K$ over $\bfQ$.
\item Classify the Galois group $\Gal(K/\bfQ)$.
\item Find all subfields $E$ of $K$ so that $E/\bfQ$ is a quadratic
  extension.
\end{enumerate}
\end{problem}
\begin{proof}
\end{proof}

%%% Local Variables:
%%% mode: latex
%%% TeX-master: "../MA553-Quals"
%%% End:

% \include{heinzer/}
% \section{August, 2015}
\begin{problem}

\textfa{جوشحال}
\end{problem}
\begin{proof}
\end{proof}

%%% Local Variables:
%%% mode: latex
%%% TeX-master: "../MA553-Quals"
%%% End:

% \section{August 2010}

%%% Local Variables:
%%% mode: latex
%%% TeX-master: "../MA553-Quals"
%%% End:

% \include{ulrich/}
\end{document}

%%% Local Variables:
%%% mode: latex
%%% TeX-master: t
%%% End:
