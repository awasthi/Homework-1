\section{Rings}
\begin{problem}
Let $R$ be a commutative ring with $1\neq 0$ and let $\frakp$ be a
prime ideal of $R$. Let $I$ and $J$ be ideals of $R$ such that $I\cap
J\subset\frakp$, prove that either $I\subset P$ or $J\subset P$.
\end{problem}
\begin{proof}
Without loss of generality, suppose that $I\nsubset J$. We show that
$J\subset\frakp$. Let $x\in I$. Then $x\notin\frakp$. But for any $y\in J$,
$xy\in I\cap J$. Thus, $xy\in\frakp$. Since $\frakp$ is prime, $x\in\frakp$
or $y\in\frakp$. But $x\notin\frakp$ hence, $y\in\frakp$. This is true for
any $y\in J$. Thus, $J\subset\frakp$.
\end{proof}
\begin{problem}
Prove that a finite integral domain is a field.
\end{problem}
\begin{proof}
Let $a\in R$ be a nonzero element. Define the map $\varphi_a\colon R\to R$
by $\varphi_a(x)\coloneqq ax$. Then $\varphi_a$ defines a group
homomorphism on $R$ viewed as an additive Abelian group: Let $x,y\in R$
then
\begin{align*}
\varphi_a(x+y)&=a(x+y)\\
&=ax+ay\\
&=\varphi_a(x)+\varphi_a(y).
\end{align*}
Now, let $x\in\ker\varphi$. Then $\varphi_a(x)=ax=0$. Since $R$ is a domain
and $a\neq 0$, $x=0$. Thus, $\varphi$ is injective. Since $R$ is finite and
$\varphi_a\colon R\to R$ is injective, $\varphi_a$ is surjective (by the
pigeonhole principle). Thus, there exists an element $b\in R$ such that
$\varphi_a(b)=ab=1$. Thus, $a$ is a unit. Since $\varphi_a$ chosen
arbitrarily, it follows that every nonzero element $a\in R$ is a
unit. Thus, $R$ is a field.
\end{proof}

\begin{problem}
An element $x$ of a ring $R$ is called nilpotent if some power of $x$ is
zero. Prove that if $x$ is nilpotent, then $1+x$ is a unit in $R$.
\end{problem}
\begin{proof}
First we will prove the following:
\begin{lemma}
If $x$ is nilpotent, then $-x$ is nilpotent.
\end{lemma}
\begin{proof}
\renewcommand\qedsymbol{$\clubsuit$}
Suppose that $x$ is nilpotent. Then $x^n=0$ for some positive integer
$n$. Then
\[
(-x)^n=(-1)^n\cdot x^n=(-1)^n\cdot 0=0.
\]
Thus, $-x$ is nilpotent.
\end{proof}
Now, since $x$ is nilpotent, by the preceding lemma, $-x$ is
nilpotent. Thus
\[
(-x)^n-1=(-x-1)((-x)^{n-1}+\cdots+1).
\]
Since $x^n=0$, we have
\[
-1=((-x)-1)((-x)^{n-1}+\cdots+1)
\]
or
\[
1=(1+x)((-x)^{n-1}+\cdots+1).
\]
Thus, $1+x$ is a unit.
\end{proof}

\begin{problem}
Let $R$ be a nonzero commutative ring with $1$. Show that if $I$ is an
ideal of $R$ such that $1+a$ is a unit in $R$ for all $a\in I$, then $I$ is
contained in every maximal ideal of $R$.
\end{problem}
\begin{proof}
Seeking a contradiction, assume otherwise. Then there exists a maximal
ideal $\frakm$ such that $\frakm\nsupset I$, i.e., for some $a\in I$,
$a\notin\frakm$. Consider the ideal generated by $(a)$. Since $a\in I$,
$(a)\neq R$ since $I$ is a proper ideal of $R$, in particular, since $a$ is
a nonunit. Consider the ideal $\frakm+(a)$. Since $a\notin\frakm$,
$\frakm\subset\frakm+(a)$. But since $\frakm$ is maximal, it follows that
$\frakm+(a)=R$. Hence, there exists an element $m\in\frakm$ such that
$m+ra=1$ for some $r\in r$. Then we have $m=1-ra$. Since $-r\in R$ and
$a\in I$, we have $-ra\in I$ so $m=1+(-ra)$ is a unit thus,
$\frakm=R$. This contradicts that $\frakm$ is a maximal ideals. Thus, $I$
is contained in every maximal ideal of $R$.
\end{proof}

\begin{problem}
Let $R$ be an integral domain and $F$ be its field of fractions. Let
$\frakp$ be a prime ideal in $R$ and
\[
R_\frakp\coloneqq
\left\{\,\tfrac{a}{b}\;\middle|\;a,b\in R,\,b\notin\frakp\,\right\}\subset F.
\]
Show that $R_\frakp$ has a unique maximal ideal.
\end{problem}
\begin{proof}
We will show that
\[
\frakp
R_\frakp\coloneqq\left\{\,\tfrac{a}{b}\;\middle|\;a\in\frakp,\,b\notin\frakp\,\right\}
\]
is the unique maximal ideal of $R$. We will show that $a/b\in R_\frakp$ is
a unit if and only if $a/b\notin\frakp R_\frakp$.

$\implies$ Suppose that $a/b$ is a unit. Then there exists an element
$a'/b'$ such that
\[
\left(\frac{a}{b}\right)\left(\frac{c}{d}\right)=\frac{ac}{bd}=\frac{1}{1}.
\]
That is, there exists an element $s\in R\minus\frakp$ such that
$s(ac-bd)=0$. Since $R$ is an integral domain, $s\neq 0$ so $ac-bd=0$
implies $ac=bd$. Since $b,d\notin\frakp$, $bd\notin\frakp$ (since $\frakp$
is prime) and, in particular, $ac\notin\frakp$ so $a/b\notin\frakp
R_\frakp$.

$\impliedby$ Conversely, suppose that $a/b\notin\frakp R_\frakp$. Then
$a\notin\frakp$. Thus, $b/a\in R_\frakp$ and
\[
\left(\frac{a}{b}\right)\left(\frac{b}{a}\right)=\frac{ab}{ba}=\frac{1}{1}.
\]
Thus, $a/b$ is a unit in $R_\frakp$.

Now, since $\frakp R_\frakp$ does not contain any units, it is a proper
ideal of $R_\frakp$. Morevore, for every $a/b\notin\frakp R_\frakp$,
$\frakp R_\frakp +(a/b)=R_\frakp$ so $\frak R_\frakp$ is a maximal ideal,
i.e., is not contained in any proper ideal of $R_\frakp$. Any other ideal
must contain a unit or is strictly contained in $\frakp R_\frakp$. Thus,
$\frakp R_\frakp$ is the unique maximal ideal of $R_\frakp$.
\end{proof}

\begin{problem}
Let $m$ and $n$ be relatively prime integers. Show that there is an
isomorphism $Z_{mn}^\times\cong Z_m^\times\times Z_n^\times$.
\end{problem}
\begin{proof}
Suppose $m$ and $n$ are relatively prime. Then $(m)+(n)=\bbZ$, i.e., $(m)$
and $(n)$ are comaximal. By the Chinese remainder theorem there is a ring
isomorphism
\[
Z_{mn}\cong Z_m\times Z_n.
\]
which gives an isomorphism of the group of units
\[
Z_{mn}^\times\cong\left(Z_m\times Z_n\right)^\times.
\]
Thus, it suffices to show that $\left(Z_m\times
  Z_n\right)^\times=Z_m^\times\times Z_m^\times$.

Suppose $(a,b)\in\left(Z_m\times Z_n\right)^\times$. Then $(a,b)$ is a unit
in $Z_m\times Z_n$, i.e., there exists $(c,d)$ such that
$(a,b)(c,d)=(1,1)$. But $(a,b)(c,d)=(1,1)$ if and only if $ac=1$ and
$bd=1$. Thus, $a\in Z_m^\times$ and $b\in Z_n^\times$ so $(a,b)\in
Z_m^\times\times Z_n^\times$. Conversely, if $(a,b)\in Z_m^\times\times
Z_n^\times$ then $a$ is a unit in $Z_m$ and $b$ is a unit in $Z_n$. Thus,
there exists elements $c\in Z_m$ and $d\in Z_n$ such that $ac=1$ and
$bd=1$ so $(a,b)(c,d)=(ac,bd)=(1,1)$. Thus, $(a,b)\in\left(Z_m\times
  Z_n\right)^\times$.
\end{proof}

\begin{problem}
Show that if $x$ is non-nilpotent in $R$ then a maximal ideal $\frakp$ of
$R$, which does not contain $x^n$ for $n=1,2,...$, is prime.
\end{problem}
\begin{proof}
I think what the professor had in mind was to prove this: ``Show that if
$x$ is non-nilpotent in $R$ then the ideal $\frakp$, which is maximal with
respect to not containing $x^n$ for any $n\in\bbZ$, is prime.''

This looks like a standard commutative algebra problem. Let
$S\coloneqq\left\{\,x^k\;\middle|\;k\geq 1\,\right\}$, i.e., the
multiplicative set generated by $x$ and suppose that $\frakp$ is an ideal
maximal with respect to $\frakp\cap S=\emptyset$. Seeking a contradiction
suppose $a,b\in R$ with $ab\in\frakp$ but $a,b\notin\frakp$. Then, the
ideals $\frakp+(a)$ and $\frakp+(b)$ contain $\frakp$ and therefore must
contain a power of $x$, say $x^m$ and $x^n$, respectively. Thus, we have
\[
x^mx^n=x^{m+n}\in(\frakp+(a))(\frakp+(b))\subset\frakp+(ab)\subset\frakp.
\]
But $\frakp$ is maximal with respect to not containing any power of
$x$. This is a contradiction. Thus, we must have $a\in\frakp$ or
$b\in\frakp$ which implies $\frakp$ is prime.
\end{proof}

\begin{problem}
Let $\bbQ$ be the field of rational numbers and
$D=\left\{\,a+b\sqrt{2}\;\middle|\;a,b\in\bbQ\,\right\}$.
\begin{enumerate}[label=(\alph*)]
\item Show that $D$ is a principal ideal domain.
\item Show that $\sqrt{3}$ is not an element of $D$.
\end{enumerate}
\end{problem}
\begin{proof}
(a) First we define a Euclidean norm $d\colon D\to\bbZ_{\geq 0}$ via
$d\left(a+b\sqrt{2}\right)\coloneqq a^2+2b^2$. Let $I\subset D$ be a proper
ideal. Let $x=a+b\sqrt{2}\in I$ be minimal with respect to the Euclidean
norm, i.e., $d(x)\leq d(y)$ for all $y\in I$. We wi
\\\\
(b)
\end{proof}

\begin{problem}
Show that if $p$ is a prime such that $p\equiv 1\mod 4$, then $x^2+1$ is
not irreducible in $\bbZ_p[x]$.
\end{problem}
\begin{proof}
\end{proof}

\begin{problem}
Show that if $p$ is a prime such that $p\equiv 3\mod 4$, then $x^2+1$ is
irreducible in $\bbZ_p[x]$.
\end{problem}
\begin{proof}
\end{proof}

\begin{problem}
Find a simpler description for each of the following rings:
\begin{enumerate}
\item $\bbZ[x]/(x^2-3,2x+4)$;
\item $\bbZ[i]/(2+i)$ $(i^2=-1)$.
\end{enumerate}
\end{problem}
\begin{proof}
\end{proof}

\begin{problem}
Show that $\bbZ[\sqrt{-13}]$ is not a principal ideal domain.
\end{problem}
\begin{proof}
\end{proof}

\begin{problem}
Let $D$ be a principal ideal domain. Prove that every nonzero prime ideal
of $D$ is a maximal ideal.
\end{problem}
\begin{proof}
\end{proof}

\begin{problem}
Prove or disprove that a nonzero prime ideal $P$ of a principal ideal
domain $R$ is a maximal ideal.
\end{problem}
\begin{proof}
\end{proof}

\begin{problem}
Consider the polynomial $f(x)=x^4+1$.
\begin{enumerate}[label=(\alph*)]
\item Use the Eisenstein Criterion to show that $f(x)$ is irreducible in
  $\bbZ[x]$.
\item Prove that $f(x)$ is reducible in $\mathbf{F}_p[x]$ for every prime
  $p$.
\end{enumerate}
\end{problem}
\begin{proof}
\end{proof}

\begin{problem}
Assume that $f(x)$ and $g(x)$ are polynomials in $\bbQ[x]$ and that
$f(x)g(x)\in\bbZ[x]$. Prove that the product of any coefficient of $f(x)$
with any coefficient of $g(x)$ is an integer.
\end{problem}
\begin{proof}
\end{proof}

\begin{problem}
Let $k$ be a field, $x,y$, indeterminates. Let $f(x)$ and $g(x)$ be
relatively prime polynomials in $k[x]$. Show that in the polynomial ring
$k(y)[x]$, $f(x)-yg(x)$ is irreducible.
\end{problem}
\begin{proof}
\end{proof}

%%% Local Variables:
%%% mode: latex
%%% TeX-master: "../MA553-Quals"
%%% End:
