\section{Fields}
\chapter{Field Theory}
\begin{problem}
Let $F$ be a field with prime characteristic $\ch(F)=p$. Let $L/F$ be a
finite extension such that $p$ does not divide $[L:F]$. Show that $L/F$ is
a separable extension.
\begin{proof}
\end{proof}
\end{problem}

\begin{problem}
Let $\zeta_5$ be a primitive $5$-th root of unity, and denote
$\theta=\zeta_5+\zeta_5^{-1}$ as an element of the cyclotomic field
$\bbQ(\zeta_5)$. Show that the minimal polynomial of $\theta$ over $\bbQ$
is $m_{\theta,\bbQ}(x)=x^2+x-1$.
\begin{proof}
\end{proof}
\end{problem}

\begin{problem}
Prove or disprove the following: If $f(x),g(x)\in\bbQ[x]$ are irreducible
polynomials that have the same splitting field, then $\deg f=\deg g$.
\begin{proof}
\end{proof}
\end{problem}

\begin{problem}
Prove or disprove that every finite algebraic extension field of
$\bbF_{p^n}$ is Galois.
\begin{proof}
\end{proof}
\end{problem}

\begin{problem}
If $[K:\bbF_p]$ divides $[L:\bbF_p]$, does it follow that $K$ is isomorphic
to a subfield of $L$.
\begin{proof}
\end{proof}
\end{problem}

\begin{problem}
Let $\bbF_p$ be a finite field whose cardinality $p$ is prime. Fix a
positive integer $n$ which is not divisible by $p$, and let $\zeta_n$ be a
primitive $n$-th root of unity. Show that $[\bbF_p(\zeta_n):\bbF_p]=a$ is
the least positive integer such that $p^a\equiv 1\mod n$. [\emph{Hint:} the
Galois group of the extension of $\bbF_p$ is generated by the Frobenius
automorphism.]
\begin{proof}

\end{proof}
\end{problem}

\begin{problem}
Fix a prime $p$, and consider the polynomial $f(x)=x^p-x-1$. Let
$\bbF_p(f)$ be the splitting field of $f(x)$ over $\bbF_p$. Let
$a\in\bbF_p(f)$ be a root of $f$.
\begin{enumerate}[label=(\alph*)]
\item Show that $a\mapsto a+1$ defines an automorphism of $\bbF_p(f)$.
\begin{proof}
Let
\end{proof}
\item Show that $\Gal(\bbF_p(f)/\bbF_p)\cong\bbZ_p$.
\begin{proof}
\end{proof}
\item Prove that $f(x)$ is irreducible in $\bbZ[x]$.
\begin{proof}
\end{proof}
\end{enumerate}
$\bbF_p(f)/\bbF_p$ is called an Artin--Schreier Extension.
\end{problem}

\begin{problem}
Let $x$ and $y$ be indeterminates over the field $\bbF_2$. Prove that there
exists infinitely many subfields of $L=\bbF_2(x,y)$ that contain the field
$K=\bbF_2(x^2,y^2)$.
\begin{proof}
\end{proof}
\end{problem}

\begin{problem}
Let $K/F$ be an algebraic field extension. If $K=F(a)$ for some $a\in K$,
prove that there are only finitely many subfields of $K$ that contain $F$.
\begin{proof}
\end{proof}
\end{problem}

\begin{problem}
Let $p$ be a prime integer. Recall that a field extension $K/F$ is called a
$p$-extension if $K/F$ is Galois and $[K:F]$ is a power of $p$. If $K/F$
and $L/K$ are $p$-extensions, prove that the Galois closure of $L/F$ is a
$p$-extension.
\begin{proof}
\end{proof}
\end{problem}

\begin{problem}
Give an example where $K/F$ and $L/K$ are $p$-extensions, but $L/F$ is not
Galois.
\begin{proof}
\end{proof}
\end{problem}

\begin{problem}
Let $L/\bbQ$ be the splitting field of the polynomial $x^6-2\in\bbQ[x]$.
\begin{enumerate}[label=(\alph*)]
\item If $a$ is one root of $x^6-2$, draw the subfield lattice of the
  extension $\bbQ(a)$ over $\bbQ$.
% \begin{proof}[Subfield lattice]
% Alright. Let's crank it out! Let $f(x)=x^6-2$. The
% splitting field of this polynomial is just
% $L=\bbQ(\sqrt[6]{2},\zeta_6)$ with index
% $[L:\bbQ]=6\cdot\varphi(6)=6\cdot 2=12$. First, we'll
% calculate the Galois group of this extension. To that end,
% it suffices to look at the automorphisms on the generators
% of $L$.

% Clearly
% \[\Gal(L/\bbQ)=\left<\,
% \sigma,\tau\;\middle|\;
% \sigma^6=\tau^2=1,\,\tau\sigma=\sigma^5\tau\, \right>,\]
% where
% \begin{align*}
% \sigma
% &\colon
% \begin{cases}
% \sqrt[6]{2}&\longmapsto\zeta_6\sqrt[6]{2},\\
% \zeta_6&\longmapsto\zeta_6,
% \end{cases},
% &\tau
% &\colon
% \begin{cases}
% \sqrt[6]{2}&\longmapsto\sqrt[6]{2},\\
% \zeta_6&\longmapsto\zeta_6^5.
% \end{cases}
% \end{align*}
% Clearly $\sigma^6=\tau^2=1$. What is less trivial is
% showing $\sigma\tau=\tau\sigma^5$. Observe
% \begin{align*}
% \sigma^5
% &\colon
% \begin{cases}
% \sqrt[6]{2}&\longmapsto\zeta_6^5\sqrt[6]{2},\\
% \zeta_6&\longmapsto\zeta_6,
% \end{cases},\\
% \sigma\tau
% &\colon
% \begin{cases}
% \sqrt[6]{2}&\longmapsto\zeta_6\sqrt[6]{2},\\
% \zeta_6&\longmapsto\zeta_6^5,
% \end{cases},
% &\tau\sigma^5
% &\colon
% \begin{cases}
% \sqrt[6]{2}&\longmapsto(\zeta_6^5)^5\sqrt[6]{2}
% =\zeta_6\sqrt[6]{2},\\
% \zeta_6&\longmapsto\zeta_6^5.
% \end{cases}
% \end{align*}
% Thus $\Gal(L/\bbQ)\cong D_{12}$. From here, we simply use
% the Fundamental Theorem of Galois Theory and observe the
% correspondence between subfields of $L$ and subgroups of
% $D_{12}$. (If only I knew the subgroup lattice of
% $D_{12}$).
% \end{proof}
\item Give generators for each subfield $K$ of $L$ for which
  $[K:\bbQ]=2$. How many are there?
% \begin{proof}[Solution]
% There is at least one and it corresponds to the subgroup
% $\langle \sigma \rangle\leq D_{12}$ whose index
% $[D_{12}:\langle \sigma \rangle]=2$. Therefore, the only
% subfield is $K=\bbQ(\zeta_6)=\bbQ(\sqrt{-3})$ (a degree $2$
% extension over $\bbQ$).at
% \end{proof}
\item Give generators for each subfield $K$ of $L$ for which
  $[K:\bbQ]=3$. How many are there?
\item Give generators for each subfield $K$ of $L$ for which
  $[K:\bbQ]=4$. How many are there?
\item How many subfields $K$ of $L$ have index $[L:K]=2$?
% \begin{proof}[Solution]
% This is also has at least one such subfield corresponding
% to the subgroup $\langle \tau \rangle\leq D_{12}$. The
% field is $\bbQ(\sqrt[6]{2})$. The extension to $L$ is
% certainly degree $2$.
% \end{proof}
\end{enumerate}
\end{problem}

\begin{problem}
Give an example of a field $F$ having characteristic $p>0$ and irreducible
monic polynomial $f(x)\in F[x]$ that has a multiple root.
\begin{proof}

\end{proof}
\end{problem}

\begin{problem}
Let $f$ be an irreducible polynomial of degree $k$ over $\bbF_p$. Find the
splitting field of $f$ and its Galois group.
\begin{proof}
\end{proof}
\end{problem}

\begin{problem}
Let $n$ be a positive integer and $d$ a positive integer that divides
$n$. Suppose $a\in\bbR$ is a root of the polynomial
$x^n-2\in\bbQ[x]$. Prove that there is precisely one subfield $F$ of
$\bbQ(a)$ with $[F:\bbQ]=d$.
\begin{proof}
\end{proof}
\end{problem}

\begin{problem}
Let $a=\sqrt[3]{5-\sqrt{7}}$.
\begin{enumerate}[label=(\alph*)]
\item Find the minimal polynomial of $a$, and the conjugates of $a$.
\item Determine the Galois closure of $F$ of $\bbQ(a)$.
\item Show that $F/\bbQ$ is an extension by radicals.
\item Conclude that $\Gal(F/\bbQ)$ is solvable.
\end{enumerate}
\begin{proof}
\end{proof}
\end{problem}

\begin{problem}
Let $F$ be a field of characteristic $p>0$. Fix an element $c$ in
$F$. Prove that $f(x)=x^p-c$ is irreducible in $F[x]$ if and only if $f(x)$
has no roots in $F$.
\begin{proof}
\end{proof}
\end{problem}

\begin{problem}
Determine the Galois group of the splitting field over $\bbQ$ and all its
subfields for
\begin{enumerate}[label=(\alph*)]
\item $f(x)=x^3-2$
\item $f(x)=x^4+2$
\item $f(x)=x^4+4$
\item $f(x)=x^4+4x+2$
\end{enumerate}
\begin{proof}
\end{proof}
\end{problem}

\begin{problem}
Show that $\sqrt{2}\notin\bbQ(\sqrt[3]{2},\zeta_3)$, where
$\zeta_3^2+\zeta_3+1=0$.
\begin{proof}
\end{proof}
\end{problem}

\begin{problem}
Let $L/F$ be a Galois extension of degree $[L:F]=2p$, where $p$ is aan odd
prime.
\begin{enumerate}[label=(\alph*)]
\item Show that hhere exits a unique queadratic subfield $E$, i.e.,
  $F\subseteq E\subseteq L$ and $[E:F]=2$.
\item Does there exist a unique subfield $K$ of index $2$, i.e.,
  $F\subseteq E\subseteq L$ and $[E:F]=2$.
\end{enumerate}
\begin{proof}
\end{proof}
\end{problem}

\begin{problem}
Let $L/F$ be a Galois extension of degree $[L:F]=p^2$ for some prime
$p$. Let $K$ be a subfield satisfying $F\subset K\subset L$. Must $K/F$ be
a normal extension?
\begin{proof}
\end{proof}
\end{problem}

\begin{problem}
Let $L/F$ be the Galois closure of he separable algebraic field extension
$F(\theta)/F$. Let $p$ be a prime that divides $[L:F]$. Prove that there
exists a subfield $K$ of $L$ such that $[L:K]=p$ and $L=K(\theta)$.
\end{problem}
\begin{proof}
Since $p$ divides $[L:K]$, $[L:K]=pn$ for some positive integer
$n$.
\end{proof}
\begin{problem}
Suppose $L/\bbQ$ is a finite field extension with $[L:\bbQ]=4$. Is it
possible that there exist precisely two subfields $K_1$ and $K_2$ of $L$
for which $[L:K_i]=2$? Justify your answer.
\begin{proof}
\end{proof}
\end{problem}

%%% Local Variables:
%%% mode: latex
%%% TeX-master: "../MA553-Quals"
%%% End:
