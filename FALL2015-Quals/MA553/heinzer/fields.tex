\section{Fields}
\begin{problem}
Let $F$ be a field with prime characteristic $\ch(F)=p$. Let $L/F$ be a
finite extension such that $p$ does not divide $[L:F]$. Show that $L/F$ is
a separable extension.
\begin{proof}
Seeking a contradiction, suppose that $L/F$ is not separable. The there
exists an element $\alpha\in L$ such that its minimal polynomial
$m_{\alpha,F}(X)$ is not separable, i.e., $m_{\alpha,F}$ has a multiple
root. But recall that an irreducible polynomial $g(X)$ is separable if
$\deg(D(g))=\deg(g)-1$. Thus, we must have
$\deg\left(D(m_{\alpha,F})\right)<\deg\left(m_{\alpha,F}\right)-1$ (since
for any polynomial $f$, $\deg(D(f))\leq \deg(f)-1$). But since $\Ch(F)=p$,
this is true only if $p\mid\deg(m_{\alpha,F})$. For suppose not. Then
$m_{\alpha,F}(X)=X^n+a_{n-1}X^{n-1}+\cdots+a_0$ and
\[
D(m_{\alpha,F})=nX^{n-1}+\text{some terms of lower degree.}
\]
so that $\deg(D(m_{\alpha,F}))=n-1=\deg(m_{\alpha,F})-1$. Hence, we have
$p\mid[F(\alpha):L]$ and by the tower theorem,
\[
[L:F]=[L:F(\alpha)][F(\alpha):L]
\]
implies that $p\mid[L:F]$. This is a contradiction. Thus, $L/F$ is
separable.
\end{proof}
\end{problem}

\begin{problem}
Let $\zeta_5$ be a primitive $5$-th root of unity, and denote
$\theta=\zeta_5+\zeta_5^{-1}$ as an element of the cyclotomic field
$\bbQ(\zeta_5)$. Show that the minimal polynomial of $\theta$ over $\bbQ$
is $m_{\theta,\bbQ}(X)=X^2+X-1$.
\end{problem}
\begin{proof}
Via some algebra, \textbf{:\textasciicircum)}, we have
\begin{align*}
(\zeta_5+{\zeta_5}^{-1})^2+(\zeta_5+{\zeta_5}^{-1})-1
&={\zeta_5}^2+2+{\zeta_5}^{-2}+\zeta_5+{\zeta_5}^{-1}-1,
\shortintertext{but since ${\zeta_p}^{-k}={\zeta_p}^{p-k}$ we have}
&={\zeta_5}^2+2+{\zeta_5}^3+\zeta_5+{\zeta_5}^4-1\\
&={\zeta_5}^4+{\zeta_5}^3+{\zeta_5}^2+\zeta_5+1\\
&=0.
\end{align*}
Thus, $m_{\theta,\bbQ}$ satisfies $\theta$. This implies that the
minimal polynomial of $\theta$ divides $m_{\theta,\bbQ}$. Therefore, to
show that the minimal polynomial of $\theta$ is in fact $m_{\theta,\bbQ}$
we must show that $m_{\theta,\bbQ}$ is irreducible.

To see that $m_{\theta,\bbQ}$ is irreducible we employ Eisetnstein's
criterion. Consider the shifted polynomial
\[
m_{\theta,\bbQ}(X+2)=(X+2)^2+(X+2)-1=X^2+4X+4+X+2-1=X^2+5X+5.
\]
By Eisenstein's criterion, $5\mid 5$ and $5\mid 5X$, but $5^2\nmid
5$. Thus, $m_{\theta,\bbQ}(X+2)$ is irreducible so $m_{\theta,\bbQ}(X)$ is
irreducible. Therefore, the minimal polynomial of $\theta$ is
$m_{\theta,\bbQ}$.

Now, since $\bbQ$ is characteristic $0$, $\Gal(\bbQ(\zeta_5)/\bbQ)\cong
(\bbZ/(5))^\times\cong Z_4$. Since $Z_4$ has a unique subgroup of order
$2$, by he fundamental theorem of Galois theory, $\bbQ(\theta)$ is the only
extension of degree $2$ under $\bbQ(\zeta_5)$. Similarly, $\bbQ$ is the
only other proper subfield since the only other subgroup of $Z_4$ is the
trivial subgroup.
\end{proof}

\begin{problem}
Prove or disprove the following: If $f(x),g(x)\in\bbQ[x]$ are irreducible
polynomials that have the same splitting field, then $\deg f=\deg g$.
\end{problem}
\begin{proof}
This is false. Consider the polynomial $f(X)=X^3-2$. The splitting field of
this polynomial is $\bbQ\left(\sqrt[3]{2},\zeta_3\right)$. However, by the
primitive element theorem, there exists
$\alpha\in\bbQ(\sqrt[3]{2},\zeta_3)$ such that
$\bbQ(\sqrt[3]{2},\zeta_3)=\bbQ(\alpha)$ and the
$\deg(m_{\alpha,\bbQ})=\left[\bbQ\left(\sqrt[3]{2},\zeta_3\right):\bbQ\right]=6$.
\end{proof}

\begin{problem}
Prove or disprove that every finite algebraic extension field of
$\bbF_{p^n}$ is Galois.
\end{problem}
\begin{proof}
The adjective \emph{algebraic} is redundant in the above for every finite
extension is necessarily algebraic.
\\\\
Let $F$ be a finite extension of $\bbF_{p^n}$. Then $F$ must be a finite
field of characteristic $p$ since $\bbF_p\subset\bbF_{p^n}\subset F$. By
the uniqueness theorem for finite fields, $F\cong\bbF_{p^m}$ for some
positive integer $m$. Hence, $\bbF_{p^m}/\bbF_p$ is Galois, being the
splitting field of the separable polynomial $X^{p^m}-X$.

By the fundamental theorem of Galois theory, since $F$ is Galois over
$\bbF_p$, $F$ is Galois over any subfield containing $\bbF_p$. Thus,
$F/\bbF_p$ is Galois.
\end{proof}

\begin{problem}
If $[K:\bbF_p]$ divides $[L:\bbF_p]$, does it follow that $K$ is isomorphic
to a subfield of $L$?
\end{problem}
\begin{proof}
Yes. Put $n\coloneqq\left[K:\bbF_p\right]$, $m\coloneqq[L:\bbF_p]$, and
suppose $n\mid m$. By the fundamental theorem for finite fields,
$K\cong\bbF_{p^n}$ and $L\cong\bbF_{p^m}$. Now, $\Gal(L/\bbF_p)\cong Z_m$
(generated by the Frobenius automorphism). Since $n\mid m$, $Z_m$ has a
subgroup of order $Z_{m/n}$. Thus, by the fundamental theorem of Galois
theory, $L$ has a subfield $E$ such that
\[
\left[E:\bbF_p\right]=\left[Z_m:Z_{m/n}\right]=m/(m/n)=n.
\]
Thus, by the fundamental theorem for finite fields, $E\cong\bbF_{p^n}\cong
K$.
\end{proof}

\begin{problem}
Let $\bbF_p$ be a finite field whose cardinality $p$ is prime. Fix a
positive integer $n$ which is not divisible by $p$, and let $\zeta_n$ be a
primitive $n$th root of unity. Show that
$\left[\bbF_p(\zeta_n):\bbF_p\right]=a$ is the least positive integer such
that $p^a\equiv 1\pmod{n}$. (\emph{Hint:} the Galois group of the extension
of $\bbF_p$ is generated by the Frobenius automorphism.)
\end{problem}
\begin{proof}
By the fundamental theorem of finitely fields,
$G\coloneqq\Gal(\bbF_p(\zeta_n)/\bbF_p)=\langle\sigma\rangle$ where $\sigma$
is the Frobenius automorphism. Since
$\left[\bbF_p(\zeta_n):\bbF_p\right]=a$, the order of $\sigma$ is
$a$. Since $\zeta_n$ generates $\bbF_p(\zeta_n)$, by the fundamental
theorem of Galois theory, the identity automorphism is the only
automorphism in $G$ which fixes $\zeta_n$: If $b$ is a positive integer
with $b<a$, then ${\zeta_n}^{p^b}=\sigma^b(\zeta_n)\neq\zeta_n$. Hence,
$p^b\nequiv 1\pmod{n}$.

Since $\sigma_a=\id_{\bbF_{p}(\zeta_n)}$, we have that
$\sigma_a(\zeta_n)=\zeta_n$. But $\sigma_a(\zeta_n)={\zeta_n}^{p^a}$. Hence
${\zeta_n}^{p^n}=\zeta_n$. Since the $n$th roots of unity form a cyclic
multiplicative group generated by $\zeta_n$ of order $n$, it follows from
${\zeta_n}^{p^a}=\zeta_n$ that $p^a\equiv 1\pmod{n}$.
\end{proof}

\begin{problem}
Fix a prime $p$, and consider the polynomial $f(x)=x^p-x-1$. Let
$\bbF_p(f)$ be the splitting field of $f(x)$ over $\bbF_p$. Let
$a\in\bbF_p(f)$ be a root of $f$. Show that $a\mapsto a+1$ defines an
automorphism of $\bbF_p(f)$. Show that
$\Gal(\bbF_p(f)/\bbF_p)\cong\bbZ_p$. Prove that $f(x)$ is irreducible in
$\bbZ[x]$. $\bbF_p(f)/\bbF_p$ is called an Artin--Schreier Extension.
\end{problem}
\begin{proof}
Since $\bbF_p$ is of characteristic $p$, the \emph{freshman's dream holds},
i.e.,
\[
(a+1)^p-(a+1)-1=a^p+1^p-a-1-1=a^p-a-1=0.
\]
Thus, $a+1$ is a root of $f$. Note that if $a\in\bbF_p$, then $0$ is a root
of $f$ since $a+1,a+2,\cdots a+(p-a)=0$ would be the roots of this
polynomial. But $f(0)=0^p-0-1=-1\neq 0$. Thus, $a\notin\bbF_p$.

Now, we note that $\bbF_p(a)=\bbF_p(a+1)$: $1,a\in\bbF_p(a)$ so
$a+1\in\bbF_p(a)$ and $a,-1\in\bbF_p(a+1)$ so
$(a+1)-1=a\in\bbF_p(a+1)$. Thus,
\[
\bbF(a)=\bbF(a+1)=\bbF(a+2)=\cdots=\bbF(a+p-1).
\]
Since all of $a,a+1,...,a+p-1$ are roots of $f$, and all of these fields
are equal, $\bbF_p(a)=\bbF_p(f)$, i.e., $\bbF_p(a)$ is the splitting field
of $f$. Hence, any map $a\mapsto a+i$, for $0\leq i\leq p-1$, determines an
automorphism of $\bbF_p(f)$. Note that $a\mapsto a+i$ is just $i-1$
applications of the map $a\mapsto a+1$. hence,
$\Gal\left(\bbF_p(f)/\bbF_p\right)$ is cyclic generated by $a\mapsto
a+1$. Moreover, this is a group of order $p$ since $a+p=a$ but $a+i\neq a$
for all $1\leq i\leq p-1$. Thus, $\Gal\left(\bbF_p(f)/\bbF_p\right)\cong
Z_p$.

Since $f$ is a monic polynomial of degree
$p=\left[\bbF_p(a):\bbF_p\right]$, with $a$ as root, it follows that
$f(X)=m_{\alpha,\bbF_p}(X)$. Hence, $f$ is irreducible in $\bbF_p[X]$.

Since $\bbZ$ is an integral domain, $f$ is a nonconstant monic polynomial
in $\bbZ[X]$ and $(p)$ is a proper ideol of $\bbZ$, and $\bar f=f$ is
irreducible in $\bbF_p[X]\cong(\bbZ/(p))[X]$, if $f$ is irreducible in
$\bbZ[X]$.
\end{proof}

\begin{problem}
Let $x$ and $y$ be indeterminates over the field $\bbF_2$. Prove that there
exists infinitely many subfields of $L=\bbF_2(X,Y)$ that contain the field
$K=\bbF_2(X^2,Y^2)$.
\end{problem}
\begin{proof}
This is from Dummit and Foote:
\\\\
Consider the polynomial $f(T)=T^2-X^2\in\bbF_2(X^2,Y^2)[T]$. The roots of
this polynomial are $X$, $-X$, neither of which are contained in
$\bbF_2(X^2,Y^2)$. Thus, $T^2-X^2$ is irreducible in
$\bbF_2(X^2,Y^2)[T]$. Thus,
$\left[\bbF_2(X,Y):\bbF_2(X^2,Y^2)\right]=2$. Similarly, $T^2-Y^2$ is
irreducible over $\bbF_2(X,Y^2)[T]$, so by the tower theorem, we have
\[
\left[\bbF_2(X,Y):\bbF_2(X^2,Y^2)\right]=\left[\bbF_2(X,Y):\bbF_2(X,Y^2)\right]\left[\bbF_2(X,Y^2):\bbF_2(X^2,Y^2)\right]=2\cdot 2=4.
\]

for $c\in\bbF_2(X^2,Y^2)$, consider the subfield $\bbF_2(X+cY)$. Since
$(X+cY)^2=X^2+c^2Y^2$ (by the freshman's dream), we have
\[
\bbF_2(X^2,Y^2)\subset\bbF_2(X+cY),\qquad\text{and}\qquad
\bbF_2(X+cY)\subset\bbF_2(X,Y).
\]
Now, $T^2-X^2-c^2-Y^2$ has $X+CY$ as a root, so
\[
\left[\bbF_2(X+cY):\bbF_2(X^2,Y^2)\right]\leq 2.
\]
Bit if there were only finitely many subfields, then for some $c\neq
c'\in\bbF_2(X^2,Y^2)$, $\bbF_2(X+cY)=\bbF_2(X+c'Y)$, so
$x+cy,x+c'y\in\bbF_2(X^2,Y^2)$. Thus,
$(X+cY)-(X+c'Y)=(c-c')Y\in\bbF_2(X+cY)$ so $Y\in\bbF_2(X+cY)$. Thus,
$X\in\bbF_2(X+cY)$. Thus,
\[
\bbF_2(X+cY)\subset\bbF_2(X,Y)\subset\bbF_2(X+cY),
\]
so $\bbF_2(X,Y)=\bbF_2(X+cY)$. But this is absurd since
\[
\left[\bbF_2(X,Y):\bbF_2(X^2,Y^2)\right]=4\neq 2=
\left[\bbF_2(X+cY):\bbF_2(X^2,Y^2)\right],
\]
so $\bbF_2(X,Y)=\bbF_2(X+cY)$. This is a contradiction. Thus, there are
infinitely many intermediate subfields.
\end{proof}

\begin{problem}
Let $K/F$ be an algebraic field extension. If $K=F(a)$ for some $a\in K$,
prove that there are only finitely many subfields of $K$ that contain $F$.
\end{problem}
\begin{proof}

\end{proof}

\begin{problem}
Let $p$ be a prime integer. Recall that a field extension $K/F$ is called a
$p$-extension if $K/F$ is Galois and $[K:F]$ is a power of $p$. If $K/F$
and $L/K$ are $p$-extensions, prove that the Galois closure of $L/F$ is a
$p$-extension.
\end{problem}
\begin{proof}
\end{proof}

\begin{problem}
Give an example where $K/F$ and $L/K$ are $p$-extensions, but $L/F$ is not
Galois.
\end{problem}
\begin{proof}
\end{proof}

\begin{problem}
Let $L/\bbQ$ be the splitting field of the polynomial $x^6-2\in\bbQ[x]$.
\begin{enumerate}[label=(\alph*)]
\item If $a$ is one root of $x^6-2$, draw the subfield lattice of the
  extension $\bbQ(a)$ over $\bbQ$.
% \begin{proof}[Subfield lattice]
% Alright. Let's crank it out! Let $f(x)=x^6-2$. The
% splitting field of this polynomial is just
% $L=\bbQ(\sqrt[6]{2},\zeta_6)$ with index
% $[L:\bbQ]=6\cdot\varphi(6)=6\cdot 2=12$. First, we'll
% calculate the Galois group of this extension. To that end,
% it suffices to look at the automorphisms on the generators
% of $L$.

% Clearly
% \[\Gal(L/\bbQ)=\left<\,
% \sigma,\tau\;\middle|\;
% \sigma^6=\tau^2=1,\,\tau\sigma=\sigma^5\tau\, \right>,\]
% where
% \begin{align*}
% \sigma
% &\colon
% \begin{cases}
% \sqrt[6]{2}&\longmapsto\zeta_6\sqrt[6]{2},\\
% \zeta_6&\longmapsto\zeta_6,
% \end{cases},
% &\tau
% &\colon
% \begin{cases}
% \sqrt[6]{2}&\longmapsto\sqrt[6]{2},\\
% \zeta_6&\longmapsto\zeta_6^5.
% \end{cases}
% \end{align*}
% Clearly $\sigma^6=\tau^2=1$. What is less trivial is
% showing $\sigma\tau=\tau\sigma^5$. Observe
% \begin{align*}
% \sigma^5
% &\colon
% \begin{cases}
% \sqrt[6]{2}&\longmapsto\zeta_6^5\sqrt[6]{2},\\
% \zeta_6&\longmapsto\zeta_6,
% \end{cases},\\
% \sigma\tau
% &\colon
% \begin{cases}
% \sqrt[6]{2}&\longmapsto\zeta_6\sqrt[6]{2},\\
% \zeta_6&\longmapsto\zeta_6^5,
% \end{cases},
% &\tau\sigma^5
% &\colon
% \begin{cases}
% \sqrt[6]{2}&\longmapsto(\zeta_6^5)^5\sqrt[6]{2}
% =\zeta_6\sqrt[6]{2},\\
% \zeta_6&\longmapsto\zeta_6^5.
% \end{cases}
% \end{align*}
% Thus $\Gal(L/\bbQ)\cong D_{12}$. From here, we simply use
% the Fundamental Theorem of Galois Theory and observe the
% correspondence between subfields of $L$ and subgroups of
% $D_{12}$. (If only I knew the subgroup lattice of
% $D_{12}$).
% \end{proof}
\item Give generators for each subfield $K$ of $L$ for which
  $[K:\bbQ]=2$. How many are there?
% \begin{proof}[Solution]
% There is at least one and it corresponds to the subgroup
% $\langle \sigma \rangle\leq D_{12}$ whose index
% $[D_{12}:\langle \sigma \rangle]=2$. Therefore, the only
% subfield is $K=\bbQ(\zeta_6)=\bbQ(\sqrt{-3})$ (a degree $2$
% extension over $\bbQ$).at
% \end{proof}
\item Give generators for each subfield $K$ of $L$ for which
  $[K:\bbQ]=3$. How many are there?
\item Give generators for each subfield $K$ of $L$ for which
  $[K:\bbQ]=4$. How many are there?
\item How many subfields $K$ of $L$ have index $[L:K]=2$?
% \begin{proof}[Solution]
% This is also has at least one such subfield corresponding
% to the subgroup $\langle \tau \rangle\leq D_{12}$. The
% field is $\bbQ(\sqrt[6]{2})$. The extension to $L$ is
% certainly degree $2$.
% \end{proof}
\end{enumerate}
\end{problem}

\begin{problem}
Give an example of a field $F$ having characteristic $p>0$ and irreducible
monic polynomial $f(x)\in F[x]$ that has a multiple root.
\begin{proof}

\end{proof}
\end{problem}

\begin{problem}
Let $f$ be an irreducible polynomial of degree $k$ over $\bbF_p$. Find the
splitting field of $f$ and its Galois group.
\end{problem}
\begin{proof}
\end{proof}

\begin{problem}
Let $n$ be a positive integer and $d$ a positive integer that divides
$n$. Suppose $a\in\bbR$ is a root of the polynomial
$x^n-2\in\bbQ[x]$. Prove that there is precisely one subfield $F$ of
$\bbQ(a)$ with $[F:\bbQ]=d$.
\end{problem}
\begin{proof}
\end{proof}

\begin{problem}
Let $a=\sqrt[3]{5-\sqrt{7}}$.
\begin{enumerate}[label=(\alph*)]
\item Find the minimal polynomial of $a$, and the conjugates of $a$.
\item Determine the Galois closure of $F$ of $\bbQ(a)$.
\item Show that $F/\bbQ$ is an extension by radicals.
\item Conclude that $\Gal(F/\bbQ)$ is solvable.
\end{enumerate}
\end{problem}
\begin{proof}
\end{proof}

\begin{problem}
Let $F$ be a field of characteristic $p>0$. Fix an element $c$ in
$F$. Prove that $f(x)=x^p-c$ is irreducible in $F[x]$ if and only if $f(x)$
has no roots in $F$.
\end{problem}
\begin{proof}
\end{proof}

\begin{problem}
Determine the Galois group of the splitting field over $\bbQ$ and all its
subfields for
\begin{enumerate}[label=(\alph*)]
\item $f(x)=x^3-2$
\item $f(x)=x^4+2$
\item $f(x)=x^4+4$
\item $f(x)=x^4+4x+2$
\end{enumerate}
\end{problem}
\begin{proof}
\end{proof}

\begin{problem}
Show that $\sqrt{2}\notin\bbQ(\sqrt[3]{2},\zeta_3)$, where
$\zeta_3^2+\zeta_3+1=0$.
\end{problem}
\begin{proof}
\end{proof}

\begin{problem}
Let $L/F$ be a Galois extension of degree $[L:F]=2p$, where $p$ is aan odd
prime.
\begin{enumerate}[label=(\alph*)]
\item Show that hhere exits a unique queadratic subfield $E$, i.e.,
  $F\subseteq E\subseteq L$ and $[E:F]=2$.
\item Does there exist a unique subfield $K$ of index $2$, i.e.,
  $F\subseteq E\subseteq L$ and $[E:F]=2$.
\end{enumerate}
\end{problem}
\begin{proof}
\end{proof}

\begin{problem}
Let $L/F$ be a Galois extension of degree $[L:F]=p^2$ for some prime
$p$. Let $K$ be a subfield satisfying $F\subset K\subset L$. Must $K/F$ be
a normal extension?
\end{problem}
\begin{proof}
\end{proof}

\begin{problem}
Let $L/F$ be the Galois closure of he separable algebraic field extension
$F(\theta)/F$. Let $p$ be a prime that divides $[L:F]$. Prove that there
exists a subfield $K$ of $L$ such that $[L:K]=p$ and $L=K(\theta)$.
\end{problem}
\begin{proof}
Since $p$ divides $[L:K]$, $[L:K]=pn$ for some positive integer
$n$.
\end{proof}
\begin{problem}
Suppose $L/\bbQ$ is a finite field extension with $[L:\bbQ]=4$. Is it
possible that there exist precisely two subfields $K_1$ and $K_2$ of $L$
for which $[L:K_i]=2$? Justify your answer.
\end{problem}
\begin{proof}
\end{proof}

%%% Local Variables:
%%% mode: latex
%%% TeX-master: "../MA553-Quals"
%%% End:
