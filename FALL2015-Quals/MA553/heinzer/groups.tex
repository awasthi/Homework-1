\chapter{Heinzer MA 553 Problems}
Past Heinzer and Włodarczyk problems with proofs to the theorems,
corrolaries, and lemmas where I believe they would benefit me.
\section{Groups}
\begin{problem}
Does the symmetric group $S_5$ have a subgroup of order $10$? Justify your
answer.
\end{problem}
\begin{proof}
Yes. In fact, the following more general result holds.
\begin{lemma}
The group $D_{2n}$ acts transitively on the set $A$ consisting of the
vertices of a regular $n$-gon.
\end{lemma}
\begin{proof}[Proof of lemma]
\renewcommand\qedsymbol{$\clubsuit$}
Labeling these vertices $0,...,n-1$ in a clockwise fashion, let $r$ be the
rotation of the $n$-polygon clockwise by $2\pi/n$ radians and let $s$ be
the reflection of the regular $n$-gon by any line which passes through the
center of the $n$-gon. This defines an action on $A$ since for any vertex
$a\in A$ and we have $r\cdot a\in A$ (that is, $r\cdot a\mapsto a+1\mod n$)
and $s\cdot a\in A$ (that is, $s\cdot a\mapsto n-1\mod n$ or something like
that) and $r,s$ are generators for $D_{2n}$.

Next, it is easy to see that the action is transitive for $r^k\cdot
a\mapsto a+k\mod n$ traverses (goes through every element of) the set $A$.

Lastly, we claim that this action is faithful. That is, we claim that the
stabilizer of $A$ consists of the identity subgroup. First $\langle
e\rangle\subset\Stab_{D_{2n}}(A)$ (this is always true). Let
$g\in\Stab_{D_{2n}}(A)$. Then, $g\cdot a=a\mod n$ for all $a\in A$. This
cannot be an element of the form $sr^k$ or $r^k$ since $r^k$ does not fix
any vertices. Thus, it can only be an element of the form $s$ or $e$. But
likewise $s$ only fixes at most two vertices (vertices which intersect the
line we are reflecting about). Thus, $g=e$ and we see that the action is
indeed faithful.

Thus, there is an induced homomorphism $\varphi\colon D_{2n}\hookrightarrow
S_{n}$ with kernel $\langle e \rangle$ the identity element, i.e.,
$\varphi$ is a monomorphism so $D_{2n}\cong\varphi(D_{2n})<S_n$. This shows
that $S_n$ always contains a subgroup of order $2n$, namely, a subgroup
isomorphic to the dihedral group $D_{2n}$.
\end{proof}
From the lemma above, we see that $D_{10}\hookrightarrow S_5$ so that $S_5$
has a subgroup of order $10$.
\end{proof}

\begin{problem}
Let $G$ be a subgroup generated by the $5$-cycles in $S_5$. Find the order
of $N_{S_5}(G)$.
\end{problem}
\begin{proof}
This is a thinly disguised Sylow's theorem problem. The $5$-cycles of $S_5$
are order the order $5$ premutations of $S_5$ hence, are contained in some
Sylow $5$-subgroup $P$. Since $G$ is the larges subgroup containing these
$5$-cycles and $P$ is a maximal subgroup of $S_5$ then $G=P$. First, let us
factor the order of $S_5$ into primes, $|S_5|=5!=2^3\cdot 3\cdot 5$. By
Sylow's theorem, we have that the index of the normalizer of $G$ in $S_5$
is $n_5=[S_5:N_{S_5}(G)]$ and $n_5\equiv 1\pmod{5}$ and $n_5\mid 2^3\cdot
3$. Running through all of the possibilities, we see that $n_5=1$ or
$n_5=6$.

If $n_5=1$ then $G$ is the unique Sylow $5$-subgroup of $G$ and hence, a
normal subgroup of $S_5$. Moreover, since all of the $5$-cycles are even
permutations $G<A_5$. Since $G$ is a characteristic subgroup of $S_5$ this
would imply that $G\lhd A_5$, but $A_5$ is simple. Thus, $n_5=6$.

Hence, $n_5=6$ and we have that
\[
|N_{S_5}(G)|=\frac{5!}{6}=\frac{1\cdot 2\cdot 3\cdot 4\cdot 5}{6}=4\cdot 5=20.\qedhere
\]
\end{proof}

\begin{problem}
Show that for any element $\sigma$ of order $2$ in the
alternating group $A_n$, there exists $\tau\in S_n$ such that
$\tau^2=\sigma$.
\end{problem}
\begin{proof}
Consider the unique representation of $\sigma$ as a product of disjoint
cycles
\[
\sigma=(a^1_1\,\cdots\,a^1_{k_1})\cdots (a^\ell_1\,\cdots\,a^\ell_{k_\ell}).
\]
since disjoint cycles commute, $|\sigma|$ is the least common multiple of
the order of each of the cycles in the representation above. Since every
$n$-cycle has order $n$ and $|\sigma|=2$, it follows that $\sigma$ must be
a product of disjoint transposition, i.e., disjoint $2$-cycles.

Now, since $\sigma\in A_n$, $\sigma$ is an even permutation so consists of
an even number of disjoint transpositions, say
\[
\sigma=(a_1\,b_1)\cdots(a_{2k}\,b_{2k})
\]
for some positive integer $k$. Now, note that the product of transpositions
\[
(a\,b)(c\,d)=(a\,c\,b\,d)^2
\]
so that
\[
\sigma=(a_1\,a_2\,b_1\,b_2)^2\cdots(a_{2k-1}\,a_{2k}\,b_{2k-1}\,b_{2k})^2.
\]
Since each of these cycles are disjoint from one another, they commute so
that
\[
\sigma=\left[(a_1\,a_2\,b_1\,b_2)\cdots(a_{2k-1}\,a_{2k}\,b_{2k-1}\,b_{2k})\right]^2.
\]
Define
\[
\tau\coloneqq(a_1\,a_2\,b_1\,b_2)\cdots(a_{2k-1}\,a_{2k}\,b_{2k-1}\,b_{2k}).
\]
Then $\tau^2=\sigma$ as desired.
\end{proof}

\begin{problem}
Let $G$ be a finite group, $p>0$ a prime number. Show that a
subgroup $H<G$ contains a Sylow $p$-subgroup of $G$ if and only
if $p$ does not divide $[G:H]$.
\end{problem}
\begin{proof}
$\implies$ Put $|G|=p^\alpha m$ for positive integer $m$ and $\alpha$,
where $m$ is not divisible by $p$. Suppose that $P\in\Syl_p(G)$ is
contained in $H$. Then, by Lagrange's theorem, we have $p^\alpha\mid H$ and
$|H|\mid p^\alpha m|G|$. Thus, $|H|=p^\alpha n$ for some $n\mid m$ not
divisible by $p$. Hence,
\[
[G:H]=\frac{p^\alpha m}{p^\alpha n}=\frac{m}{n}
\]
which is not divisible by $p$ since $m$ and $n$ are not divisible by $p$.

$\impliedby$ Conversely, suppose that $p\nmid [G:H]$. Then $|H|=p^\alpha
m/[G:H]$. Since $p\nmid [G:H]$, $[G:H]\mid m$. Put $|H|=p^\alpha n$. Let
$P\in\Syl_p(H)$. Then $P$ is a $p$-subgroup of $G$ hence, must be contained
in a Sylow $p$-subgroup $Q$ of $G$. Thus, $P<Q$, but
$|P|=p^\alpha=|Q|$. Hence, $P=Q$, i.e., $H$ contains a Sylow $p$-subgroup
of $G$.
\end{proof}

\begin{problem}
Let $G$ be a finite group, $p>0$ a prime number, and $H$ a
normal subgroup of $G$. Prove the following assertions.
\begin{enumerate}[label=(\alph*)]
\item Any Sylow $p$-subgroup of $H$ is the intersection
$P\cap H$ of a Sylow $p$-subgroup of $G$ and $H$.
\item Any Sylow $p$-subgroup of $G/H$ is the quotient $PH/H$,
where $P$ is a Sylow $p$-subgroup of $G$.
\end{enumerate}
\end{problem}
\begin{proof}
(a) Let $Q\in\Syl_p(H)$. Then $Q$ is a $p$-subgroup of $G$ hence, it is
contained in a Sylow $p$-subgroup $P$ of $G$. Hence, $Q<P\cap
H$. Conversely, since $P\cap H<P$, $P\cap H$ is a $p$-subgroup of $H$
hence, it is contained in a Sylow $p$-subgroup $R$ of $H$. Thus, $Q<P\cap
H<R$. But since $|Q|=|R|$ and $|Q|\mid |P\cap H|$ and $|P\cap H|\mid |R|$,
we must have that $Q=P\cap H$.
\\\\
(b)
\end{proof}

\begin{problem}
Let $H$ be a normal subgroup of a finite group $G$, and let
$N\subset H$ be a normal Sylow subgroup of $H$. Prove that $N$
is a normal subgroup of $G$.
\end{problem}
\begin{proof}
\end{proof}

\begin{problem}
Let $G$ be a finite group, $p>0$ a prime number, and $H$ a
normal $p$-subgroup of $G$. Prove the following assertions.
\begin{enumerate}[label=(\alph*)]
\item $H$ is contained in each Sylow $p$-subgroup of $G$.
\item If $K$ is any normal $p$-subgroup of $G$, then $HK$ is a
normal $p$-subgroup of $G$.
\end{enumerate}
\end{problem}
\begin{proof}
\end{proof}

\begin{problem}
Prove that the order of the automorphism group $(\bbZ/3\bbZ)^4$
is $80\times 78\times 72\times 54$.
\end{problem}
\begin{proof}
\end{proof}

\begin{problem}
Prove, for fixed $n$, that the following conditions are
equivalent:
\begin{enumerate}[label=(\alph*)]
\item Every abelian group of order $n$ is cyclic.
\item $n$ is square free (i.e., not divisible by any square
integer $>1$).
\end{enumerate}
\end{problem}
\begin{proof}
\end{proof}

\begin{problem}
Prove that there is no simple group of order $4125$.
\end{problem}
\begin{proof}
\end{proof}

\begin{problem}
Show that $P$ is abelian whenever $\Aut(P)$ is cyclic.
\end{problem}
\begin{proof}
\end{proof}

\begin{problem}
Let $G$ be a finite group of order $pqr$, where $p>q>r$ are
prime.
\begin{enumerate}[label=(\alph*)]
\item If $G$ fails to have a normal subgroup of order $p$,
determine the number of elements in $G$ of order $p$.
\item If $G$ fails to have a normal subgroup of order $q$,
prove that $G$ has at least $q^2$ elements of order $q$.
\end{enumerate}
\end{problem}
\begin{proof}
\end{proof}

\begin{problem}
Find all abelian groups of order $60$. Find the number of
elements of order $6$ in each group. % order $6$, a in each ?
\end{problem}
\begin{proof}
\end{proof}

\begin{problem}
Show that any group $G$ of order $80$ is solvable.
\end{problem}
\begin{proof}
\end{proof}

\begin{problem}
Let $G$ be a finite group and suppose that $\Aut(G)$ is
solvable. Show that $G$ is solvable.
\end{problem}
\begin{proof}
\end{proof}

%%% Local Variables:
%%% mode: latex
%%% TeX-master: "../MA553-Quals"
%%% End:
