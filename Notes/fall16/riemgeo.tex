\chapter{Riemannian Geometry}
Notes compiled from Do Carmo's \emph{Riemannian Geometry} book.
\section{Differentiable manifolds}
\begin{definition}
  A \emph{differentiable manifold} of dimension \(n\) is a set \(M\) and a
  family of injective mappings
  \(\bfx_\alpha\colon U_\alpha\subset\bbR^n\to M\) of open sets
  \(U_\alpha\) of \(\bbR^n\) into \(M\) such that:
  \begin{enumerate}[label=(\arabic*),noitemsep]
  \item \(\bigcup_\alpha\bfx_\alpha(U_\alpha)=M\);
  \item for any pair \(\alpha\), \(\beta\), with
    \(\bfx_\alpha(U_\alpha)\cap\bfx_\beta(U_\beta)=W\neq\emptyset\), the
    sets \(\bfx_\alpha^{-1}(W)\) and \(\bfx_\beta^{-1}(W)\) are open sets
    in \(\bbR^n\) and the mappings \(\bfx_\beta^{-1}\circ\bfx_\alpha\) are
    differentiable;
  \item the family \(\{(U_\alpha,\bfx_\alpha)\}\) is maximal relative to
    the conditions (1) and (2).
  \end{enumerate}
\end{definition}

The pair \((U_\alpha,\bfx_\alpha)\) with \(p\in\bfx_\alpha(U_\alpha)\) is
called a \emph{parametrization} (or \emph{system of coordinates}) of \(M\)
at \(p\); \(\bfx_\alpha(U_\alpha)\) is then called a \emph{coordinate
  neighborhood} at \(p\). A family \(\{(U_\alpha,\bfx_\alpha)\}\)
satisfying (1) and (2) is called a \emph{differentiable structure} on
\(M\).

\begin{remark}
  A differentiable structure on a set \(M\) induces a natural topology on
  \(M\). It suffices to define \(A\subset M\) to be an \emph{open set} in
  \(M\) if and only if \(\bfx_\alpha^{-1}(A\cap\bfx_\alpha(U_\alpha))\) is
  an open set in \(\bbR^n\) for all \(\alpha\). Observe that the topology
  is defined in such a way that the sets \(\bfx_\alpha(U_\alpha)\) are open
  and that the mappings \(\bfx_\alpha\) are continuous.
\end{remark}

\begin{example}
  The \emph{real projective space \(\varbbRP^n\)}. Let us denote by
  \(\varbbRP\) the set of straight lines of \(\bbR^{n+1}\) which pass
  through the origin \(\mathbf{0}=(0,\dotsc,0)\in\bbR^{n+1}\); that is,
  \(\varbbRP^n\) is the set of ``directions'' of \(\bbR^{n+1}\).

  Let us introduce a differentiable structure on \(\varbbRP^n\). For this,
  let \((x_1,\dotsc,x_{n+1})\in\bbR^{n+1}\) and observe, to begin with,
  that \(\varbbRP^n\) is the quotient space of
  \(\bbR^{n+1}\setminus\{\mathbf{0}\}\) by the equivalence relation:
  \[
    (x_1,\dotsc,x_{n+1})\sim\lambda(x_1,\dotsc,x_{n+1}), \qquad
    (\lambda\neq 0).
  \]
  The points of \(\varbbRP^n\) will be denoted by
  \([x_1,\dotsc,x_{n+1}]\). Observe that, if \(x_i\neq 0\),
  \[
    [x_1,\dotsc,x_{n+1}]=
    \left[\frac{x_1}{x_i},\dotsc,\frac{x_{i-1}}{x_i},1,\frac{x_{i+1}}{x_i},\dotsc,\frac{x_{n+1}}{x_i}\right].
  \]
  Define subsets \(V_1,\dotsc,V_{n+1}\), of \(\varbbRP^n\), by:
  \[
    V_i=\bigl\{\,[x_1,\dotsc,x_{n+1}]:x_i\neq 0\,\bigr\},\qquad i=1,\dotsc,n+1.
  \]

  Geometrically, \(V_i\) is the set of straight lines in \(\bbR^{n+1}\)
  which pass through the origin and do not belong to the hyperplane
  \(x_i=0\). We are now going to show that we can take the \(V_i\) as
  coordinate neighborhoods, where the coordinates on \(V_i\) are
  \[
    y_1=\frac{x_1}{x_i},\dotsc,y_{i-1}=\frac{x_{i-1}}{x_i},
    y_i=\frac{x_{i+1}}{x_i},\dotsc,y_n=\frac{x_{n+1}}{x_i}.
  \]
  For this, we will define mappings \(\bfx_i\colon\bbR^n\to V_i\) by
  \[
    \bfx_i(y_1,\dotsc,y_n)=[y_1,\dotsc,y_{i-1},1,y_{i+1},\dotsc,y_n]
    ,\qquad (y_1,\dotsc,y_n)\in\bbR^n,
  \]
  and will show that the family \(\{(\bbR^n,\bfx_i)\}\) is a differentiable
  structure on \(\varbbRP^n\).

  Indeed, any mapping \(\bfx_i\) is clearly bijective while
  \(\bigcup\bfx_i(\bbR^n)=\varbbRP^n\). It remains to show that
  \(\bfx_i^{-1}(V_i\cap V_j)\) is an open set in \(\bbR^n\) and that
  \(\bfx_j^{-1}\circ\bfx_i\), \(j=1,\dotsc,n+1\), is differentiable
  there. Now, if \(i>j\), the points in \(\bfx_i^{-1}(V_i\cap V_j)\) are of
  the form:
  \[
    \bigl\{\,(y_1,\dotsc,y_n)\in\bbR^n:y_j\neq 0\,\bigr\}.
  \]
  Therefore \(\bfx_i^{-1}(V_i\cap V_j)\) is an open set in \(\bbR^n\), and
  supposing that \(i>j\) (the case \(i<j\) is similar),
  \begin{align*}
    \bfx_j^{-1}\circ\bfx_i(y_1,\dotsc,y_n)
    &=\bfx_j^{-1}[y_1,\dotsc,y_{i-1},1,y_i,\dotsc,y_n]\\
    &=\bfx_j^{-1}\left[\frac{y_1}{y_j},\dotsc,\frac{y_{j-1}}{y_j},1,
      \frac{y_{j+1}}{y_j},\dotsc,\frac{y_{i-1}}{y_j},\frac{1}{y_j},
      \frac{y_i}{y_j},\dotsc,\frac{y_n}{y_j}\right]\\
    &=\left(\frac{y_1}{y_j},\dotsc,\frac{y_{j-1}}{y_j},\frac{y_{j+1}}{y_j},
      \dotsc,\frac{y_{i-1}}{y_j},\frac{1}{y_j},\frac{y_i}{y_j},\dotsc,
      \frac{y_n}{y_j}\right),
  \end{align*}
  which is clearly differentiable.

  In summary, the space of directions of \(\bbR^{n+1}\) (real projective
  space \(\varbbRP^n\)) can ve covered by \(n+1\) coordinate neighborhoods
  \(V_i\), where the \(V_i\) are made up of those directions of
  \(\varbbRP^{n+1}\) that are not in the hyperplane \(x_i=0\); in addition,
  in each \(V_i\) we have coordinates
  \[
    \left(\frac{x_1}{x_i},\dotsc,\frac{x_{i-1}}{x_i},
    \frac{x_{i+1}}{x_i},\dotsc,\frac{x_{n+1}}{x_i}\right),
  \]
  where \((x_1,\dotsc,x_{n+1})\) are coordinates of \(\bbR^{n+1}\). It is
  customary, in the classical terminology to call the coordinates of
  \(V_i\) ``inhomogeneous coordinates'' corresponding to the ``homogeneous
  coordinates'' \((x_1,\dotsc,x_{n+1})\in\bbR^{n+1}\).
\end{example}

\begin{definition}
  Let \(M_1^n\) and \(M_2^m\) be differentiable manifolds. A mapping
  \(\varphi\colon M_1\to M_2\) is \emph{differentibale at \(p\in M_1\)} if
  given a parametrization \(\bfy\colon V\subset\bbR^m\to M_2\) at
  \(\varphi(p)\) there exists a paremtrization \(\bfx\colon
  U\subset\bbR^n\to M_1\) at \(p\) such that
  \(\varphi(\bfx(U))\subset\bfy(V)\) and the mapping
  \[
    \bfy^{-1}\circ\varphi\circ\bfx\colon U\subset\bbR^n\To\bbR^m
  \]
  is differentiable at \(\bfx^{-1}(p)\). \(\varphi\) is differentiable on
  an open set of \(M_1\) if it is differentiable at all of the points of
  this open set.
\end{definition}

%%% Local Variables:
%%% mode: latex
%%% TeX-master: "../Fall16-Notes"
%%% End:
