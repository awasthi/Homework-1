\chapter{Classical Mechanics}
This section is devoted to notes and problems from \textru{Владимир
  Арнольд}'s \emph{\textru{Математические методы классической механики}}
\cite{arnold}.

\begin{otherlanguage}{russian}
\section{Ньютонова Механика}
Ньютонова механика изучает движение системы материальных точек в терхмерном
евклидовом пространстве. В евклидовом пространстве действует шестимерная
группа движений пространства. Основые понятия и теоремы ньютоновой механики
(даже если они и формулируются в терминах декарвотых координат) инварианты
относительно этой группы.

Ньютонова потенциальная механическая система задается массами точек и
потенциальной энергией. Движениям пространства, оставляющим потенциальную
энергию неизменной, соответствуют законы сохранения.

Уравнения Ньютона позволяют исследовать до конца ряд важных задач механики,
например задачу о движении в центральном поле.

\section{Экспериментальные фаткы}
В этой главе  описаны основые экспериментальные факты, лежащие в основе
механики: принцип относительности Галилея
\end{otherlanguage}

%%% Local Variables:
%%% mode: latex
%%% TeX-master: "../Fall16-Notes"
%%% End:
