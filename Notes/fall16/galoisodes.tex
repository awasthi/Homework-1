\chapter{Group Theory and Differential Equations}
This is a summary of Kuga's \emph{Galois' Dream: Group Theory and
  Differential Equations} book.

\section{The First Week: Sets and Maps}
This week I will explain the concepts of ``set'' and ``map'', which are
fundamental in mathematics. I am sure that many of you know these terms
already, so my explanation will be just an appetizer before the main
course.

\subsection{Sets}
We call any collection of objects that is clearly determined a set. Each
object in this collection (\(=\) set) is called an element. We use capital
letters \(\calM\), \(\calN\), etc., to denote various sets, and small
letters \(x\), \(y\), etc., to denote elements of these sets. When \(x\) is
an element of (say) \(\calM\), we sometimes use expressions such as ``\(x\)
belongs to \(\calM\)'' and ``\(x\) is contained in \(\calM\)''. In
mathematical notation,
\[
  \calM\ni x\qquad\text{or}\qquad x\in\calM.
\]
If \(y\) is not an element of \(\calM\), we use the notations
\[
  \calM\not\ni y\qquad\text{or}\qquad y\notin\calM.
\]

\begin{example}
  Consider all the natural numbers. Of course, you know what they are:
  \(1,2,3,4,5,\dotsc\). These numbers are obtained by starting from \(1\)
  and adding \(1\) some number of times. Although there are infinitely many
  natural numbers, we will consider them all at once. This collection of
  all natural number is a set. We use the letter \(\bbN\) to represent this
  set. That is,
  \[
    \bbN=\{\,1,2,3,4,5,\dotsc\,\}.
  \]
  All the natural numbers belong to \(\bbN\), and nothing but natural
  numbers belong to \(\bbN\). The number \(1966\) is a natural number; so
  \(1966\in\bbN\). But a negative number \(-7031\) is not a natural number;
  so \(-7031\notin\bbN\).
\end{example}

\begin{example}
  No let's consider all the integers. An integer is any of
  \begin{enumerate}[label=\arabic*),noitemsep]
  \item Natural numbers: \(1,2,3,4,5,\dotsc\);
  \item Natural numbers with a negative sign: \(-1,-2,-3,-4,-5,\dotsc\);
    and
  \item \(0\)
  \end{enumerate}
\end{example}

In mathematics, the letter \(\bbZ\) is used to denote the set of integers.
\[
  -7031\in\bbZ,
\]
because \(-7031\) is an integer.

%%% Local Variables:
%%% mode: latex
%%% TeX-master: "../Fall16-Notes"
%%% End:
