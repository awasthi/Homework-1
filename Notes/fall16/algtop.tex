\chapter{Algebraic Topology}
From my meetings with Mark. We reference Hatcher's \emph{Algebraic
  Topology} \cite{hatcher} freely available here
\url{https://www.math.cornell.edu/~hatcher/#ATI}.

\section{Cohomology}
Let's look at some examples to get an idea of what cohomolgy is all
about. Take the simplest case: Let \(X\) be a \(1\)-dimensional
\(\Delta\)-complex, i.e., an oriented graph. For a fixed abelian group
\(G\), the set of all functions from vertices of \(X\) to \(G\) forms an
abelian group, which we denote by \(\Delta^0(X;G)\) in the natural sense,
i.e., by point-wise addition. Similarly the set of all functions assigning
an element of \(G\) to each edge of \(X\) forms an abelian group
\(\Delta^1(X;G)\). We are concerned about homomorphisms
\(\delta\colon\Delta^0(X;G)\to\Delta^1(X;G)\) sending
\(\varphi\in\Delta^0\) to the function \(\delta\varphi\in\Delta^1(X;G)\)
whose value on an oriented edge \([v_0,v_1]\) is the difference
\(\varphi(v_1)-\varphi(v_0)\). For example, \(X\) here might be the graph
formed by a system of trails on a mountain, with vertices at the junctions
between trails. The function \(\varphi\) could assign to each junction its
elevation above sea level, in which case \(\delta\varphi\) would measure
the net change in elevation along the trail from one junction to the next.

Regarding the map \(\delta\colon\Delta^0(X;G)\to\Delta^1(X;G)\) as a chain
complex with \(0\)s before and after the two terms, the homology of groups
of this chain complex are by definition the simplicial cohomolgy groups of
\(X\), namely \(H^0(X;G)=\Ker\delta\subset\Delta^0(X;G)\) and
\(H^1(X;G)=\Delta^1(X;G)/{\Img\delta}=\Coker\delta\). For simplicity we are
using here the same notation as will be used for singular cohomolgy; we
later prove that for \(\Delta\)-complexes, the two theories in fact
coincide.

The group \(H^0(X;G)\) is easy to describe explicitly. A function
\(\varphi\in\Delta^0(X;G)\) has \(\delta\varphix=0\) if and only if
\(\varphi\) takes the same value at both ends of each edge of \(X\). This
is equivalent to saying that \(\varphi\) is constant on each component of
\(X\). So \(H^0(X;G)\) is the group of all functions from the set of
components of \(X\) to \(G\). This is a direct product of copies of \(G\),
one for each component of \(X\).

The cohomology group \(H^1(X:G)=\Delta^1(X;G)/{\Img\delta}\) will be
trivial if and only if \(\delta\varphi=\psi\) has a solution
\(\varphi\in\Delta^0(X;G)\) for each \(\varphi\in\Delta^1(X;G)\). Solving
this equation means deciding whether specifying the change in \(\varphi\)
across each edge of \(X\) determines an actual function
\(\varphi\in\Delta^0(X;G)\). This is rather like the calculus problem of
finding a function having a specified derivative, with the difference
operator \(\delta\) playing the role of differentiation. As in calculus, if
a solution of \(\delta\varphi=\psi\) exists, it will be unique up to adding
an element of the kernel of \(\delta\), i.e., a f unction constant on each
component of \(X\).

The equation \(\delta\varphi=\psi\) is always solvable if \(X\) is a tree
since if we choose arbitrarily a value for \(\varphi\) at a base point
vertex \(v_0\), then if the change in \(\varphi\) across each edge of \(X\)
is specified, this uniquely determines the value of \(\varphi\) at ever
other vertex \(v\) by induction along the unique path from \(v_0\) to \(v\)
in a tree. Then, since every vertex lies in one of these maximal trees, the
values of \(\psi\) on the edges of the maximal trees determine \(\varphi\)
uniquely up to a constant on each component of \(X\). But in order for the
equation \(\delta\varphi=\psi\) to hold, the value of \(\psi\) on each edge
is not in any of the maximal trees must equal the difference in the
already-determined values of \(\varphi\) at the two ends of the edge. This
condition need not be satisfied since \(\psi\) can have arbitrary values on
these edges. Thus we see that the cohomology group \(H^1(X;G)\) is a direct
product of copies of the group \(G\), one copy for each edge of \(X\) not
in one of the chosen maximal trees. This can be compared with the homology
group \(H_1(X;G)\) which consists of a direct \emph{sum} of copies


%%% Local Variables:
%%% mode: latex
%%% TeX-master: "../Fall16-Notes"
%%% End:
