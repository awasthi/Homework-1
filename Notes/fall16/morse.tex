\chapter{Morse Theory}
These are notes from Milnor's \emph{Morse Theory.}

\section{Introduction}
In this section we will illustrate by a specific example the situation that
we will investigate later for arbitrary manifolds. Let us consider a torus
\(M\), tangent to the plane \(V\), as indicated in the diagram (that I will
not bother making ... yet)

Let \(f\colon M\to\bbR\) be the height above the \(V\) plane, and let
\(M^a\) be the set of all points \(x\in M\) such that \(f(x)\leq a\). Then
the following things are true:
\begin{enumerate}[label=(\arabic*),noitemsep]
\item If \(a<0=f(p)\), then \(M^a\) is vacuous.
\item If \(f(p)<a<f(q)\), then \(M^a\) is homeomorphic to a \(2\)-cell.
\item If \(f(q)<a<f(r)\), then \(M^a\) is homeomorphic to a cylinder.
\item If \(f(r)<a<f(s)\), then \(M^a\) is homeomorphic to a compact
  manifold of genus one having a circle as a boundary.
\item If \(f(s)<a\), then \(M^a\) is a full torus.
\end{enumerate}

In order to describe the change in \(M^a\) as it passes through one of the
points \(f(p)\), \(f(q)\), \(f(r)\), \(fs\) it is convenient to consider
homotopy rather than homeomorphism type. In terms of homotopy types: (1)
\(\to\) (2) is the operation of attaching a \(0\)-cell. For as far as
homotopy type is concerned, the space \(M^a\), \(f(p)<a<f(q)\), cannot be
distinguished from a \(0\)-cell.

Here \(\approx\) means ``is of the same homotopy type as.''

(2) \(\to\) (3) is the operation of attaching a \(1\)-cell.

(3) \(\to\) (4) is again the operation of attaching a \(1\)-cell.

(4) \(\to\) (5) is the operation of attaching a \(2\)-cell.

The precise definition of ``attaching a \(k\)-cell'' can be given as
follows. Let \(Y\) be any topological space, and let
\[
  \bbD^k=\bigl\{\,x\in\bbR^k: |x|\leq 1\,\bigr\}
\]
be the \(k\)-cell consisting of all vectors in Euclidean \(k\)-space with
length less than or equal to \(1\). The boundary
\[
  \partial \bbD^k=\bigl\{\,x\in\bbR^k:|x|=1\,\bigr\}
\]
will be denoted by \(\bbS^{k-1}\). If \(g\colon \bbS^{k-1}\to Y\) is a continuous
map then
\[
  Y\sqcup_g \bbD^k
\]
(\(Y\) with a \(k\)-cell attached by \(g\)) is obtained by first taking the
topological sum (i.e., disjoint union) of \(Y\) and \(\bbD^k\), and
identifying each \(x\in \bbS^{k-1}\) with \(g(x)\in Y\). To take care of
the case \(k=0\) let \(\bbD^0\) be a point and let
\(\partial\bbD^0=\bbS^{-1}\) be vacuous, so that \(Y\) is a \(0\)-cell
attached to just the union of \(Y\) and a disjoint point.

As one might expect, the point \(p\), \(q\), \(r\), and \(s\) at which the
homotopy type of \(M^a\) changes, have as simple characterization in terms
of \(f\). They are the critical points of the function. If we choose any
coordinate system \((x,y)\) near these points, then the derivatives
\(\frac{\partial f}{\partial x}\) and \(\frac{\partial f}{\partial y}\) are
both zero. At \(p\) we can choose \((x,y)\) so that \(f=x^2+y^2\), at \(s\)
so that \(f=C-x^2-y^2\), and at \(q\) and \(r\) so that
\(f=C+x^2-y^2\). Note that the number of minus signs in the expression for
\(f\) at eeach point is the dimension of the cell we must attach to go from
\(M^a\) to \(M^b\), where \(a<f(\cdot)<b\). Our first theorems will
generalize these facts for any differentiable function on a manifold.

\section{Non-degenerate functions}
\subsection{Definitions and lemmas}
The word \emph{smooth} and \emph{differentiable} will be used
interchangeably to mean differentiable of class \(C^\infty\). The tangent
space of a smooth manifold \(M\) at a point \(p\) will be denoted by
\(T_pM\). If \(g\colon M\to N\) is a smooth map with \(g(p)=q\), then the
induced linear map of tangent spaces will be denoted by \(g_*\colon T_pM\to
T_qN\).

Now let \(f\) be a smooth real valued function on a manifold \(M\). A point
\(p\in M\) is called a \emph{critical point} of \(f\) if the induced map
\(f_*\colon T_pM\to T_{f(p)}\bbR\) is zero. If we choose local coordinates
\((x^1,\dotsc,x^n)\) in a neighborhood \(U\) of \(p\) this means
\[
  \frac{\partial f}{\partial x^1}(p)=\dotsb=\frac{\partial f}{\partial x^n}=0.
\]
The real number \(f(p)\) is called a \emph{critical value} of \(f\).

We denote by \(M^a\) the set of all points \(x\in M\) such that \(f(x)\leq
a\). If \(a\) is not a critical value of \(f\) then it follows from the
implicit function theorem that \(M^a\) is a smooth manifold with
boundary. The boundary \(f^{-1}(a)\) is a smooth submanifold of \(M\).

A critical point \(p\) is called \emph{non-degenerate} if and only if the
\[
  \frac{\partial^2 f}{\partial x^i\partial x^j}(p)
\]
is non-singular. It can be checked directly that non-degeneracy does not
depend on the coordinate system. This will follow also from the following
intrinsic definition.

If \(p\) is a critical point of \(f\) we define a symmetric bilinear
functional \(f_{**}\) on \(T_pM\), called the \emph{Hessian} of \(f\) at
\(p\). If \(v,w\in T_pM\) then \(v\) and \(w\) have extensions \(\tilde v\)
and \(\tilde w\) to vector fields. We let \(f_{**}(v,w)\defeq \tilde
\(v\). We must show that this is symmetric and well-defined.

%%% Local Variables:
%%% mode: latex
%%% TeX-master: "../Fall16-Notes"
%%% End:
