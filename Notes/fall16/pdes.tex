\chapter{Introduction to Partial Differential Equations}
Here we summarize some important points about PDEs. The material is mostly
taken from Evans's \emph{Partial Differential Equations} \cite{evans} with
occasional detours to Strauss's \emph{Partial Differential Equations: An
  Introduction} \cite{strauss}. We will be following Dr.\@ Petrosyan's
\textsf{Course Log} which can be found here
\url{https://www.math.purdue.edu/~arshak/F16/MA523/courselog/}, i.e.,
summarizing the appropriate chapters from \cite{evans}.

\section{Introduction}
\subsection{Partial differential equations}
\begin{definition}
  An expression of the form
  \begin{equation}
    \label{eq:pde:def-of-pde}
    F\bigl(D^ku(x),D^{k-1}u(x),\dotsc,Du(x),u(x),x\bigr)=0
    \qquad (x\in U)
  \end{equation}
  is called a \emph{\(k\)th-order partial differential equation (PDE)},
  where
  \[
    F\colon\bfR^{n^k}\times\bfR^{n^{k-1}}\times\dotsb\times\bfR^n\times U\too\bfR
  \]
  is given, and
  \[
    u\colon U\too\bfR
  \]
  is the unknown.
\end{definition}

Here are some more definitions,
\begin{definition}
  \hfill
  \begin{enumerate}[label=(\roman*)]
  \item The partial differential equation \eqref{eq:pde:def-of-pde} is
    called \emph{linear} if it has the form
    \[
      \sum_{|\alpha|\leq k}a_\alpha(x)D^\alpha u=f(x)
    \]
    for given functions \(a_\alpha(|\alpha|\leq k)\), \(f\). This linear
    PDE is \emph{homogeneous} if \(f=0\).
  \item The PDE \eqref{eq:pde:def-of-pde} is \emph{semilinear} if it has
    the form
    \[
      \sum_{|\alpha|=k}a_\alpha D^\alpha u
      +a_0\bigl(D^{k-1}u,\dotsc,Du,u,x\bigr)=0.
    \]
  \item The PDE \eqref{eq:pde:def-of-pde} is \emph{quasilinear} if it has
    the form
    \[
      \sum_{|\alpha|=k}a_\alpha\bigl(D^{k-1}u,\dotsc,Du,u,x\bigr)D^\alpha u
      +a_0\bigl(D^{k-1}u,\dotsc,Du,u,x\bigr)=0.
    \]
  \item The PDE \eqref{eq:pde:def-of-pde} is \emph{fully nonlinear} if it
    depends upon the highest order derivatives.
  \end{enumerate}
\end{definition}

A \emph{system} of partial differential equations is, informally speaking,
a collection of several PDEs for several unknown functions.

\begin{definition}
  An expression of the form
\end{definition}

%%% Local Variables:
%%% mode: latex
%%% TeX-master: "../Fall16-Notes"
%%% End:
