\chapter{Introduction to Partial Differential Equations}
Here we summarize some important points about PDEs. The material is mostly
taken from Evans's \emph{Partial Differential Equations} \cite{evans} with
occasional detours to Strauss's \emph{Partial Differential Equations: An
  Introduction} \cite{strauss}. We will be following Dr.\@ Petrosyan's
\textsf{Course Log} which can be found here
\url{https://www.math.purdue.edu/~arshak/F16/MA523/courselog/}, i.e.,
summarizing the appropriate chapters from \cite{evans}.

\section{First-Order PDEs}
\subsection{The transport equation}
In this section, we consider the simplest first-order PDE, the
\emph{transport equation} with constant coefficients, i.e., the PDE
\begin{equation}
  \label{eq:pde:transport-equation}
  u_t+b\cdot Du=0\qquad\text{in \(\bbR\times(0,\infty)\),}
\end{equation}
where \(b\) is a fixed vector in \(\bbR^n\), and
\(u\colon\bbR^n\times[0,\infty)\to\bbR\) is the solution to the PDE. Our
task is to find solutions \(u\) which satisfy the equation
\eqref{eq:pde:transport-equation}.

To address this task, let us suppose for a moment that we have a (smooth)
solution \(u\) and try to compute it using the PDE
\eqref{eq:pde:transport-equation}. First, note that
\eqref{eq:pde:transport-equation} asserts that the directional derivative
\(D_{(b,1)}u=0\). Fix a point \((x,t)\in\bbR^n\times(0,\infty)\) and define
\[
  z(s)\defeq u(x+sb,t+s)
\]
for \(s\in\bbR\). Then
\[
  \dot z(s)=Du(x+sb,t+s)\cdot b+u_t(x+sb,t+s)=0.
\]
Thus, \(z\) is a constant function of \(s\) and, consequently for each
\((x,t)\), \(u\) is constant on the line through \((x,t)\) with direction
\((b,1)\in\bbR^{n+1}\). Hence if we know the value of \(u\) at any point on
each such line, we know its value everywhere in \(\bbR^n\times(0,\infty)\).

\subsubsection{Initial-value problem}
Let's now look at the transport equation with \emph{initial conditions}
\begin{equation}
  \label{eq:pde:transport-equation-init}
  \left\{
    \begin{aligned}
      u_t+b\cdot Du&=0&&\text{in \(\bbR^n\times(0,\infty)\),}\\
      u&=g&&\text{on \(\bbR^n\times\{\,t=0\,\}\).}
    \end{aligned}
  \right.
\end{equation}
Here \(b\in\bbR^n\) and \(g\colon\bbR^n\to\bbR\) are known, and \(u\) is
the unknown. Given \((x,t)\), the line through \((x,t)\) with direction
\((b,1)\) is represented parametrically by \((x+sb,t+s)\) for
\(s\in\bbR\). This line hits the plane
\(\Gamma\defeq\bbR^n\times\{\,t=0\,\}\) when \(s=-t\), at the point
\((x-tb,0)\). Since \(u\) is constant on the line and
\(u(x-tb,0)=g(x-tb)\), we deduce
\begin{equation}
  \label{eq:pde:transport-equation-init-sol}
  u(x,t)=g(x-tb)
\end{equation}
for \(x\in\bbR^n\), \(t\geq 0\). So if
\eqref{eq:pde:transport-equation-init} has a sufficiently regular solution
\(u\) (at least \(C^1\)), it must certainly be given by
\eqref{eq:pde:transport-equation-init-sol}.

\section{Characteristics}
We now turn our attention to a very important method for solving
first-order PDEs, the method of \emph{characteristics}.
\subsection{Derivation of characteristic ODEs}
Consider the first-order (possibly non-linear) PDE
\begin{equation}
  \label{eq:pde:first-order}
  \begin{aligned}
    F(Du,u,x)&=0&&\text{in \(U\),}
  \end{aligned}
\end{equation}
subject to the \emph{boundary condition}
\begin{equation}
  \label{eq:pde:boundary-condition}
  \begin{aligned}
    u&=g&&\text{on \(\Gamma\),}
  \end{aligned}
\end{equation}
where \(\Gamma\subset\partial U\) and \(g\colon\Gamma\to\bbR\) are
known. We shall assume, for simplicity, that \(F\) and \(g\) are smooth.

We now develop the \emph{method of characteristics} to solve
\eqref{eq:pde:first-order}, \eqref{eq:pde:boundary-condition} by converting
the PDE into a system of ODEs. We proceed as follows: Suppose \(u\) solves
\eqref{eq:pde:first-order}, \eqref{eq:pde:boundary-condition} and fix a
point \(x\in U\). We would like to calculate \(u(x)\) by finding some curve
lying within \(U\), connecting \(x\) with a point \(x^0\in\Gamma\) and
along which we can compute \(u\). Since \eqref{eq:pde:boundary-condition}
says \(u=g\) on \(\Gamma\), we know the value of \(u\) at one end
\(x^0\). We hope to be able to calculate \(u\) all along the curve, and so
in particular at \(x\).

\subsubsection{Finding the characteristic curve}

But how do we choose a path in \(U\) so all of this will work? Suppose the
curve is described parametrically by the function
\(\bfx(x)=(x^1(s),\dotsc,x^n(s))\), the parameter \(s\) lying in some
subinterval \(I\subset\bbR\). Assuming \(u\) is a \(C^2\) solution of
\eqref{eq:pde:first-order}, we define also
\begin{equation}
  \label{eq:pde:parametrized-sol}
  z(s)\defeq u(\bfx(s)).
\end{equation}
In addition, set
\begin{equation}
  \label{eq:pde:parametrized-grad}
  \bfp(s)\defeq Du(\bfx(s));
\end{equation}
that is, \(\bfp(s)=(p^1(s),\dotsc,p^n(s))\), where
\begin{equation}
  \label{eq:pde:parametrized-grad-comp}
  p^i(s)=u_{xi}(\bfx(s)),\qquad 1\leq i\leq n..
\end{equation}
So \(z(\blank)\) gives us the values of \(u\) along the curve and
\(\bfp(\blank)\) records the values of gradient \(Du\). We must choose a
function \(\bfx(\blank)\) that will allow us to compute \(z(\blank)\) and
\(\bfp(\blank)\).

Differentiating \eqref{eq:pde:parametrized-grad-comp}, we have
\begin{equation}
  \label{eq:pde:parametrized-grad-derivative}
  \dot p^i(s)=\sum_{j=1}^n u_{x_ix_j}(\bfx(s))\dot x^j(s).
\end{equation}
But this expression is not too promising since it involves second order
derivatives of \(u\) which we do not know (in fact, our solution need not
be so regular as to have second order derivatives). On the other hand, if
we differentiate \eqref{eq:pde:first-order} with respect to \(x_i\), we
have
\begin{equation}
  \label{eq:pde:first-order-derivative}
  \sum_{j=1}^n F_{p_j}(Du,u,x)u_{x_jx_i}+F_z(Du,u,x)u_{x_i}+F_{x_i}(Du,u,x)=0.
\end{equation}
We can use this identity to get rid of the second order derivatives in
\eqref{eq:pde:parametrized-sol-derivative}, provided we first set
\begin{equation}
  \label{eq:pde:parametrized-curve-derivative}
  \dot x^j(s)=F_{p_j}(\bfp(s),z(s),\bfx(s)),\qquad 1\leq j\leq n.
\end{equation}
Assuming \eqref{eq:pde:parametrized-curve-derivative} holds, we evaluate
\eqref{eq:pde:first-order-derivative} at \(x=\bfx(s)\), thereby obtaining
from \eqref{eq:pde:parametrized-sol} and \eqref{eq:pde:parametrized-grad}
the identity
\[
  \sum_{j=1}^n F_{p_j}(\bfp(s),z(s),\bfx(s))u_{x_ix_j}(\bfx(s))+
  F_z(\bfp(s),z(s),\bfx(s))p^i(s)+F_{x_i}(\bfp(s),z(s),\bfx(s))=0
\]
Finally, we differentiate \eqref{eq:pde:parametrized-sol} to give us
\begin{equation}
  \label{eq:pde:parametrized-sol-derivative}
  \begin{aligned}
    \dot z(s)
    &=\sum_{j=1}^n u_{x_j}(\bfx(s))\dot x^j(s)\\
    &=\sum_{j=1}^n p^j(s)F_{p_j}(\bfp(s),z(s),\bfx(s)),
  \end{aligned}
\end{equation}
the second equality holding by \eqref{eq:pde:parametrized-grad-comp} and
\eqref{eq:pde:parametrized-grad-derivative}.

We summarize our results by rewriting equations
\eqref{eq:pde:parametrized-curve-derivative},
\eqref{eq:pde:parametrized-curve-derivative}, and
\eqref{eq:pde:parametrized-sol-derivative} as
\begin{equation}
  \label{eq:pde:first-order-characteristics}
  \left\{
    \begin{aligned}
      \text{(a) }
      \dot\bfp(s)&=
      -D_xF(\bfp(s),z(s),\bfx(s))-D_zF(\bfp(s),z(s),\bfx(s))\bfp(s),\\
      \text{(b) }
      \dot z(s)&=
      D_pF(\bfp(s),z(s),\bfx(s))\cdot\bfp(s),\\
      \text{(c) }
      \dot\bfx(s)&=D_pF(\bfp(s),z(s),\bfx(s)).
    \end{aligned}
  \right.
\end{equation}
Furthermore,
\begin{equation}
  \label{eq:pde:first-order-pzx}
  F(\bfp(s),z(s),\bfx(s))=0.
\end{equation}
These identities hold for \(s\in I\).

The system \eqref{eq:pde:first-order-characteristics} of \(2n+1\) first
order ODEs comprises the \emph{characteristic equotions/ODEs} of the
nonlinear first-order PDE \eqref{eq:pde:first-order}. The functions
\(\bfp(\blank)\), \(\bfx(\blank)\) are called \emph{characteristics} and
\(\bfx(\blank)\) is called the \emph{projected characteristic} (it is the
projection of the full characteristics \((\bfp,z,\bfx)\subset\bbR^{2n+1}\)
onto the physical region \(U\subset\bbR^n\)).

\begin{theorem}[Structure of characteristic ODEs]
  Let \(u\in C^2(U)\) solve the nonlinear, first-order partial differential
  equation \eqref{eq:pde:first-order} in \(U\). Assume \(\bfx(\blank)\)
  solves the ODE \emph{\eqref{eq:pde:first-order-characteristics}(c)},
  where \(\bfp(\blank)=Du(\bfx(\blank))\),
  \(z(\blank)=u(\bfx(\blank))\). Then \(\bfp(\blank)\) solves the ODE
  \emph{\eqref{eq:pde:first-order-characteristics}(a)} and \(z(\blank)\)
  solves the ODE \emph{\eqref{eq:pde:first-order-characteristics}(b)}, for
  those \(s\) such that \(\bfx(s)\in U\).
\end{theorem}

We still need to discover appropriate initial conditions for
\eqref{eq:pde:first-order-characteristics} to be useful. We do that in the
following section.

\subsection{Examples}
But before we move on, we look at some examples to show you how to use
\eqref{eq:pde:first-order-characteristics} to find solutions to
\eqref{eq:pde:first-order}.

\subsubsection{The linear case}
Suppose \eqref{eq:pde:first-order} is linear, i.e., has the form
\begin{equation}
  \label{eq:pde:first-order-linear}
  F(D,u,x)=\bfx\cdot Du(x)+c(x)u(x)=0,\qquad x\in U.
\end{equation}
Then, rewriting \eqref{eq:pde:first-order-linear} in terms of \(p\), \(z\),
and \(x\), we have\(F(p,z,x)=\bfb(x)\cdot p+c(x)z\), so
\[
  D_pF=\bfb(x)
\]
so \eqref{eq:pde:first-order-characteristics}(c) becomes
\begin{equation}
  \label{eq:pde:first-order-linear-dotx}
  \dot\bfx(s)=\bfx(\bfx(s)),
\end{equation}
an ODE involving only the function \(\bfx(\blank)\). Furthermore
\eqref{eq:pde:first-order-characteristics}(b) becomes
\begin{equation}
  \label{eq:pde:first-order-linear-dotz}
  \dot z(s)=\bfb(\bfx(s))\cdot\bfp(s).
\end{equation}
Then equation \eqref{eq:pde:first-order-pzx} simplifies
\eqref{eq:pde:first-order-linear-dotz}, yielding
\[
  \dot z(s)=-c(\bfx(s))z(s).
\]
This ODE is linear in \(z(\blank)\), noce we know the function
\(\bfx(\blank)\) by solving \eqref{eq:pde:first-order-linear-dotx}. In
summary, we have
\begin{equation}
  \label{eq:pde:first-order-linear-characteristics}
  \left\{
    \begin{aligned}
      \text{(a) }\dot\bfx(s)&=\bfb(\bfx(s)),\\
      \text{(b) }\dot z(s)&=-c(\bfx(s))z(s).
    \end{aligned}
  \right.
\end{equation}

\begin{example}
  Let's now look at a simple example to see how to use
  \eqref{eq:pde:first-order-linear-characteristics} to solve a PDE.
  Consider the PDE
  \[
    \tag{\(*\)}
    \left\{
      \begin{aligned}
        x_1u_{x_2}-x_2u_{x_1}&=u&&\text{in \(U\),}\\
        u&=g&&\text{on \(\Gamma\),}
      \end{aligned}
    \right.
  \]
  where \(U=\{\,x_1>0,x_2>0\,\}\) and
  \(\Gamma=\{\,x_1>0,x_2=0\,\}\subset\partial U\).
\end{example}
The PDE (\(*\)) is of the form \eqref{eq:pde:first-order-linear} with
\(\bfb=(-x_2,x_1)\) and \(c=-1\). Thus the equations
\eqref{eq:pde:first-order-linear-characteristics} read
\[
  \left\{
    \tag{\(**\)}
    \begin{aligned}
      \dot x^1&=-x^2,&\dot x^2&=x^1,\\
      \dot z&=z.
    \end{aligned}
  \right.
\]
Solving this system of ODEs we have
\[
  \left\{
    \begin{aligned}
      x^1(s)&=x^0\cos s,&x^2(s)&=x^0\sin x,\\
      z(s)&=z^0e^s=g(x^0)e^s,
    \end{aligned}
  \right.
\]
where \(x^0\geq 0\), \(0\leq s\leq \pi/2\). Now, fix \((x_1,x_2)\in
U\). Select \(s>0\), \(x^0>0\) so that
\((x_1,x_2)=(x^1(s),x^2(s))=(x^0\cos s,x^0\sin s)\) and solve for \(x^0\),
in this case, \(x^0=\sqrt{x_1^2+x_2^2}\), \(s=\arctan(x_2/x_1)\), and
therefore
\begin{align*}
  u(x)&=u(x^1(s),x^2(s))\\
      &=z(s)\\
      &=g(x^0)e^s\\
      &=g\Bigl({\textstyle\sqrt{x_1^2+x_2^2}}\Bigr)e^{\arctan(x_2/x_1)}.
\end{align*}

\subsubsection{The quasilinear case}

%%% Local Variables:
%%% mode: latex
%%% TeX-master: "../Fall16-Notes"
%%% End:
