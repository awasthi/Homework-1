\chapter{Introduction to Partial Differential Equations}
Here we summarize some important points about PDEs. The material is mostly
taken from Evans's \emph{Partial Differential Equations} \cite{evans} with
occasional detours to Strauss's \emph{Partial Differential Equations: An
  Introduction} \cite{strauss}. We will be following Dr.\@ Petrosyan's
\textsf{Course Log} which can be found here
\url{https://www.math.purdue.edu/~arshak/F16/MA523/courselog/}, i.e.,
summarizing the appropriate chapters from \cite{evans}.

\section{First-Order PDEs}
\subsection{The transport equation}
In this section, we consider the simplest first-order PDE, the
\emph{transport equation}\footnote{For more details, refer to \cite[\S
  2.1]{evans}.} with constant coefficients, i.e., the PDE
\begin{equation}
  \label{eq:pde:transport-equation}
  u_t+b\cdot Du=0\qquad\text{in \(\bbR\times(0,\infty)\),}
\end{equation}
where \(b\) is a fixed vector in \(\bbR^n\), and
\(u\colon\bbR^n\times[0,\infty)\to\bbR\) is the solution to the PDE. Our
task is to find solutions \(u\) which satisfy the equation
\eqref{eq:pde:transport-equation}.

To address this task, let us suppose for a moment that we have a (smooth)
solution \(u\) and try to compute it using the PDE
\eqref{eq:pde:transport-equation}. First, note that
\eqref{eq:pde:transport-equation} asserts that the directional derivative
\(D_{(b,1)}u=0\). Fix a point \((x,t)\in\bbR^n\times(0,\infty)\) and define
\[
  z(s)\defeq u(x+sb,t+s)
\]
for \(s\in\bbR\). Then
\[
  \dot z(s)=Du(x+sb,t+s)\cdot b+u_t(x+sb,t+s)=0.
\]
Thus, \(z\) is a constant function of \(s\) and, consequently for each
\((x,t)\), \(u\) is constant on the line through \((x,t)\) with direction
\((b,1)\in\bbR^{n+1}\). Hence if we know the value of \(u\) at any point on
each such line, we know its value everywhere in \(\bbR^n\times(0,\infty)\).

\subsubsection{Initial-value problem}
Let's now look at the transport equation with \emph{initial conditions}
\begin{equation}
  \label{eq:pde:transport-equation-init}
  \left\{
    \begin{aligned}
      u_t+b\cdot Du&=0&&\text{in \(\bbR^n\times(0,\infty)\),}\\
      u&=g&&\text{on \(\bbR^n\times\{\,t=0\,\}\).}
    \end{aligned}
  \right.
\end{equation}
Here \(b\in\bbR^n\) and \(g\colon\bbR^n\to\bbR\) are known, and \(u\) is
the unknown. Given \((x,t)\), the line through \((x,t)\) with direction
\((b,1)\) is represented parametrically by \((x+sb,t+s)\) for
\(s\in\bbR\). This line hits the plane
\(\Gamma\defeq\bbR^n\times\{\,t=0\,\}\) when \(s=-t\), at the point
\((x-tb,0)\). Since \(u\) is constant on the line and
\(u(x-tb,0)=g(x-tb)\), we deduce
\begin{equation}
  \label{eq:pde:transport-equation-init-sol}
  u(x,t)=g(x-tb)
\end{equation}
for \(x\in\bbR^n\), \(t\geq 0\). So if
\eqref{eq:pde:transport-equation-init} has a sufficiently regular solution
\(u\) (at least \(C^1\)), it must certainly be given by
\eqref{eq:pde:transport-equation-init-sol}.

\section{Characteristics}
We now turn our attention to a very important method for solving
first-order PDEs, the method of \emph{characteristics}. This section is
paraphrased if copied not verbatim from \cite[\S 3.2]{evans}.
\subsection{Derivation of characteristic ODEs}
Consider the first-order (possibly non-linear) PDE
\begin{equation}
  \label{eq:pde:first-order}
  \begin{aligned}
    F(Du,u,x)&=0&&\text{in \(U\),}
  \end{aligned}
\end{equation}
subject to the \emph{boundary condition}
\begin{equation}
  \label{eq:pde:boundary-condition}
  \begin{aligned}
    u&=g&&\text{on \(\Gamma\),}
  \end{aligned}
\end{equation}
where \(\Gamma\subset\partial U\) and \(g\colon\Gamma\to\bbR\) are
known. We shall assume, for simplicity, that \(F\) and \(g\) are smooth.

We now develop the \emph{method of characteristics} to solve
\eqref{eq:pde:first-order}, \eqref{eq:pde:boundary-condition} by converting
the PDE into a system of ODEs. We proceed as follows: Suppose \(u\) solves
\eqref{eq:pde:first-order}, \eqref{eq:pde:boundary-condition} and fix a
point \(x\in U\). We would like to calculate \(u(x)\) by finding some curve
lying within \(U\), connecting \(x\) with a point \(x^0\in\Gamma\) and
along which we can compute \(u\). Since \eqref{eq:pde:boundary-condition}
says \(u=g\) on \(\Gamma\), we know the value of \(u\) at one end
\(x^0\). We hope to be able to calculate \(u\) all along the curve, and so
in particular at \(x\).

\subsubsection{Finding the characteristic curve}

But how do we choose a path in \(U\) so all of this will work? Suppose the
curve is described parametrically by the function
\(\bfx(x)=(x^1(s),\dotsc,x^n(s))\), the parameter \(s\) lying in some
subinterval \(I\subset\bbR\). Assuming \(u\) is a \(C^2\) solution of
\eqref{eq:pde:first-order}, we define also
\begin{equation}
  \label{eq:pde:parametrized-sol}
  z(s)\defeq u(\bfx(s)).
\end{equation}
In addition, set
\begin{equation}
  \label{eq:pde:parametrized-grad}
  \bfp(s)\defeq Du(\bfx(s));
\end{equation}
that is, \(\bfp(s)=(p^1(s),\dotsc,p^n(s))\), where
\begin{equation}
  \label{eq:pde:parametrized-grad-comp}
  p^i(s)=u_{xi}(\bfx(s)),\qquad 1\leq i\leq n..
\end{equation}
So \(z(\blank)\) gives us the values of \(u\) along the curve and
\(\bfp(\blank)\) records the values of gradient \(Du\). We must choose a
function \(\bfx(\blank)\) that will allow us to compute \(z(\blank)\) and
\(\bfp(\blank)\).

Differentiating \eqref{eq:pde:parametrized-grad-comp}, we have
\begin{equation}
  \label{eq:pde:parametrized-grad-derivative}
  \dot p^i(s)=\sum_{j=1}^n u_{x_ix_j}(\bfx(s))\dot x^j(s).
\end{equation}
But this expression is not too promising since it involves second order
derivatives of \(u\) which we do not know (in fact, our solution need not
be so regular as to have second order derivatives). On the other hand, if
we differentiate \eqref{eq:pde:first-order} with respect to \(x_i\), we
have
\begin{equation}
  \label{eq:pde:first-order-derivative}
  \sum_{j=1}^n F_{p_j}(Du,u,x)u_{x_jx_i}+F_z(Du,u,x)u_{x_i}+F_{x_i}(Du,u,x)=0.
\end{equation}
We can use this identity to get rid of the second order derivatives in
\eqref{eq:pde:parametrized-sol-derivative}, provided we first set
\begin{equation}
  \label{eq:pde:parametrized-curve-derivative}
  \dot x^j(s)=F_{p_j}(\bfp(s),z(s),\bfx(s)),\qquad 1\leq j\leq n.
\end{equation}
Assuming \eqref{eq:pde:parametrized-curve-derivative} holds, we evaluate
\eqref{eq:pde:first-order-derivative} at \(x=\bfx(s)\), thereby obtaining
from \eqref{eq:pde:parametrized-sol} and \eqref{eq:pde:parametrized-grad}
the identity
\[
  \sum_{j=1}^n F_{p_j}(\bfp(s),z(s),\bfx(s))u_{x_ix_j}(\bfx(s))+
  F_z(\bfp(s),z(s),\bfx(s))p^i(s)+F_{x_i}(\bfp(s),z(s),\bfx(s))=0
\]
Finally, we differentiate \eqref{eq:pde:parametrized-sol} to give us
\begin{equation}
  \label{eq:pde:parametrized-sol-derivative}
  \begin{aligned}
    \dot z(s)
    &=\sum_{j=1}^n u_{x_j}(\bfx(s))\dot x^j(s)\\
    &=\sum_{j=1}^n p^j(s)F_{p_j}(\bfp(s),z(s),\bfx(s)),
  \end{aligned}
\end{equation}
the second equality holding by \eqref{eq:pde:parametrized-grad-comp} and
\eqref{eq:pde:parametrized-grad-derivative}.

We summarize our results by rewriting equations
\eqref{eq:pde:parametrized-curve-derivative},
\eqref{eq:pde:parametrized-curve-derivative}, and
\eqref{eq:pde:parametrized-sol-derivative} as
\begin{equation}
  \label{eq:pde:first-order-characteristics}
  \left\{
    \begin{aligned}
      \text{(a) }
      \dot\bfp(s)&=
      -D_xF(\bfp(s),z(s),\bfx(s))-D_zF(\bfp(s),z(s),\bfx(s))\bfp(s),\\
      \text{(b) }
      \dot z(s)&=
      D_pF(\bfp(s),z(s),\bfx(s))\cdot\bfp(s),\\
      \text{(c) }
      \dot\bfx(s)&=D_pF(\bfp(s),z(s),\bfx(s)).
    \end{aligned}
  \right.
\end{equation}
Furthermore,
\begin{equation}
  \label{eq:pde:first-order-pzx}
  F(\bfp(s),z(s),\bfx(s))=0.
\end{equation}
These identities hold for \(s\in I\).

The system \eqref{eq:pde:first-order-characteristics} of \(2n+1\) first
order ODEs comprises the \emph{characteristic equotions/ODEs} of the
nonlinear first-order PDE \eqref{eq:pde:first-order}. The functions
\(\bfp(\blank)\), \(\bfx(\blank)\) are called \emph{characteristics} and
\(\bfx(\blank)\) is called the \emph{projected characteristic} (it is the
projection of the full characteristics \((\bfp,z,\bfx)\subset\bbR^{2n+1}\)
onto the physical region \(U\subset\bbR^n\)).

\begin{theorem}[Structure of characteristic ODEs]
  Let \(u\in C^2(U)\) solve the nonlinear, first-order partial differential
  equation \eqref{eq:pde:first-order} in \(U\). Assume \(\bfx(\blank)\)
  solves the ODE \emph{\eqref{eq:pde:first-order-characteristics}(c)},
  where \(\bfp(\blank)=Du(\bfx(\blank))\),
  \(z(\blank)=u(\bfx(\blank))\). Then \(\bfp(\blank)\) solves the ODE
  \emph{\eqref{eq:pde:first-order-characteristics}(a)} and \(z(\blank)\)
  solves the ODE \emph{\eqref{eq:pde:first-order-characteristics}(b)}, for
  those \(s\) such that \(\bfx(s)\in U\).
\end{theorem}

We still need to discover appropriate initial conditions for
\eqref{eq:pde:first-order-characteristics} to be useful. We do that in the
following section.

\subsection{Examples}
But before we move on, we look at some examples to show you how to use
\eqref{eq:pde:first-order-characteristics} to find solutions to
\eqref{eq:pde:first-order}.

\subsubsection{The linear case}
Suppose \eqref{eq:pde:first-order} is linear, i.e., has the form
\begin{equation}
  \label{eq:pde:first-order-linear}
  F(D,u,x)=\bfx\cdot Du(x)+c(x)u(x)=0,\qquad x\in U.
\end{equation}
Then, rewriting \eqref{eq:pde:first-order-linear} in terms of \(p\), \(z\),
and \(x\), we have\(F(p,z,x)=\bfb(x)\cdot p+c(x)z\), so
\[
  D_pF=\bfb(x)
\]
so \eqref{eq:pde:first-order-characteristics}(c) becomes
\begin{equation}
  \label{eq:pde:first-order-linear-dotx}
  \dot\bfx(s)=\bfx(\bfx(s)),
\end{equation}
an ODE involving only the function \(\bfx(\blank)\). Furthermore
\eqref{eq:pde:first-order-characteristics}(b) becomes
\begin{equation}
  \label{eq:pde:first-order-linear-dotz}
  \dot z(s)=\bfb(\bfx(s))\cdot\bfp(s).
\end{equation}
Then equation \eqref{eq:pde:first-order-pzx} simplifies
\eqref{eq:pde:first-order-linear-dotz}, yielding
\[
  \dot z(s)=-c(\bfx(s))z(s).
\]
This ODE is linear in \(z(\blank)\), noce we know the function
\(\bfx(\blank)\) by solving \eqref{eq:pde:first-order-linear-dotx}. In
summary, we have
\begin{equation}
  \label{eq:pde:first-order-linear-characteristics}
  \left\{
    \begin{aligned}
      \text{(a) }\dot\bfx(s)&=\bfb(\bfx(s)),\\
      \text{(b) }\dot z(s)&=-c(\bfx(s))z(s).
    \end{aligned}
  \right.
\end{equation}

\begin{example}
  Let's now look at a simple example to see how to use
  \eqref{eq:pde:first-order-linear-characteristics} to solve a PDE.
  Consider the PDE
  \[
    \tag{\(*\)}
    \left\{
      \begin{aligned}
        x_1u_{x_2}-x_2u_{x_1}&=u&&\text{in \(U\),}\\
        u&=g&&\text{on \(\Gamma\),}
      \end{aligned}
    \right.
  \]
  where \(U=\{\,x_1>0,x_2>0\,\}\) and
  \(\Gamma=\{\,x_1>0,x_2=0\,\}\subset\partial U\).  The PDE (\(*\)) is of
  the form \eqref{eq:pde:first-order-linear} with \(\bfb=(-x_2,x_1)\) and
  \(c=-1\). Thus the equations
  \eqref{eq:pde:first-order-linear-characteristics} read
  \[
    \left\{ \tag{\(**\)}
      \begin{aligned}
        \dot x^1&=-x^2,&\dot x^2&=x^1,\\
        \dot z&=z.
      \end{aligned}
    \right.
  \]
  Solving this system of ODEs we have
  \[
    \left\{
      \begin{aligned}
        x^1(s)&=x^0\cos s,&x^2(s)&=x^0\sin x,\\
        z(s)&=z^0e^s=g(x^0)e^s,
      \end{aligned}
    \right.
  \]
  where \(x^0\geq 0\), \(0\leq s\leq \pi/2\). Now, fix \((x_1,x_2)\in
  U\). Select \(s>0\), \(x^0>0\) so that
  \((x_1,x_2)=(x^1(s),x^2(s))=(x^0\cos s,x^0\sin s)\) and solve for
  \(x^0\), in this case, \(x^0=\sqrt{x_1^2+x_2^2}\),
  \(s=\arctan(x_2/x_1)\), and therefore
  \begin{align*}
    u(x)&=u(x^1(s),x^2(s))\\
        &=z(s)\\
        &=g(x^0)e^s\\
        &=g\Bigl({\textstyle\sqrt{x_1^2+x_2^2}}\Bigr)e^{\arctan(x_2/x_1)}.
  \end{align*}
\end{example}

\subsubsection{The quasilinear case}
Let's look at the quasilinear case now, i.e., \eqref{eq:pde:first-order}
with the form
\begin{equation}
  \label{eq:pde:first-order-quasilinear}
  F(Du,u,x)=\bfb(x,u(x))\cdot Du(x)+c(x,u(x))=0.
\end{equation}
In this circumstance \(F(p,z,x)=\bfb(x,z)\cdot p+c(x,z)\), whence
\[
  D_pF=\bfb(x,z).
\]
Hence equation \eqref{eq:pde:first-order-characteristics}(c) reads
\begin{equation}
  \label{eq:pde:first-order-quasilinear-dotx}
  \dot\bfx(s)=\bfb(\bfx(s),z(s)),
\end{equation}
an ODE involving only the function \(\bfx\). Furthermore,
\eqref{eq:pde:first-order-characteristics}(c) becomes
\begin{equation}
  \label{eq:pde:first-order-quasilinear-dotz}
  \dot z(s)=\bfb(\bfx(s))\cdot\bfp(s),
\end{equation}
which, after applying \eqref{eq:pde:first-order-pzx}, turns into
\begin{equation}
  \label{eq:pde:first-order-quasilinear-dotz-simp}
  \dot z(s)=-c(\bfx(s))z(s).
\end{equation}
In summary, we have
\begin{equation}
  \label{eq:pde:first-order-quasilinear-characteristics}
  \left\{
    \begin{aligned}
      \text{(a) }
      \dot\bfx(s)&=\bfb(\bfx(s)),
      \\
      \text{(b) }
      \dot z(s)&=-c(\bfx(s))z(s).
    \end{aligned}
  \right.
\end{equation}
We will see later that the equation for \(\bfp(\blank)\) is in fact not
needed (at least in the linear and quasilinear cases).

\begin{example}
  Let's look at an example of a quasilinear PDE. Consider the PDE
  \[
    \tag{\(*\)}
    \left\{
      \begin{aligned}
        u_{x_2}+u_{x_1}&=u&&\text{in \(U\),}\\
        u&=g&&\text{on \(\Gamma\).}
      \end{aligned}
    \right.
  \]
  Here \(U=\{\,x_2>0\,\}\) and \(\Gamma=\{\,x_2=0\,\}=\partial U\) with
  \(\bfb=(1,1)\) and \(c=-z^2\). Thus, the equations
  \eqref{eq:pde:first-order-quasilinear-characteristics} yield
  \[
    \left\{
      \begin{aligned}
        \dot x^1&=1,&\dot x^2&=1,\\
        \dot z&=z^2.
      \end{aligned}
    \right.
  \]
  Consequently
  \[
    \left\{
      \begin{aligned}
        x^1(s)&=x^0+s,&x^2(s)&=s,\\
        z(s)&=\frac{z^0}{1-sz^0}=\frac{g(x^0)}{1-sg(x^0)},
      \end{aligned}
    \right.
  \]
  where \(x^0\in\bbR\), \(s\geq 0\), provided the denominator is not zero.

  Now, fix a point \((x_1,x_2)\in U\) and select \(s>0\) and \(x^0\in\bbR\)
  so \((x_1,x_2)=(x^1(s),x^2(s))=(x^0+s,s)\), i.e., \(x^0=x_1-x_2\),
  \(s=x_2\). Then
  \begin{align*}
    u(x)&=u(x^1(s),x^2(s))\\
        &=z(s)\\
        &=\frac{g(x^0)}{1-sg(x^0)}\\
        &=\frac{g(x_1-x_2)}{1-x_2g(x_1-x_2)}.
  \end{align*}
  This solution of course only makes sense if \(1-x_2g(x_1-x_2)\neq 0\).
\end{example}

\subsubsection{The fully nonlinear case}
In the general case, we must integrate the full characteristics
\eqref{eq:pde:first-order-characteristcs} if possible. In this case, we
cannot generally reduce \eqref{eq:pde:first-order-characteristics} and we
must look at the PDE on a case-by-case basis.

\begin{example}
  Let's look at an example. Consider the fully nonlinear PDE
  \[
    \tag{\(*\)}
    \left\{
      \begin{aligned}
        u_{x_1}u_{x_2}&=u&&\text{in \(U\),}\\
        u&=x_2^2&&\text{on \(\Gamma\),}
      \end{aligned}
    \right.
  \]
  where \(U=\{\,x_2>0\,\}\), \(\Gamma=\partial U=\{\,x_1=0\,\}\). Here,
  \(F(p,z,x)=p_1p_2-z\), and hence
  \eqref{eq:pde:first-order-characteristics} yield
  \[
    \left\{
      \begin{aligned}
        \dot p^1&=p^1,&\dot p^2&=p^2,\\
        \dot z&=2p^1p^2,\\
        \dot x^1&=p^2,&\dot x^2&=p^1.
      \end{aligned}
    \right.
  \]
After integrating these equations, we have
\[
  \left\{
    \begin{aligned}
      x^1(s)&=p_2^0(e^s-1),&x^2(s)&=x^0+p_1^0(e^s-1),\\
      z(s)&=z^0+p_1^0p_2^0(e^{2s}-1),\\
      p^1(s)&=p_1^0e^s,&p^2(s)&=p_2^0e^s,
    \end{aligned}
  \right.
\]
where \(x^0\in\bbR\), \(s\in\bbR\), and \(z^0={(x^0)}^2\).

We must determine \(p^0=(p_1^0,p_2^0)\). Since \(u=x_2^2\) on \(\Gamma\),
\(p_2^0=u_{x_2}(0,x^0)=2x^0\). Furthermore the PDE \(u_{x_1x_2}=u\) itself
implies \(p_1^0p_2^0=z^0={(x^0)}^2\), and so \(p_1^0=x^0/2\). Consequently
the formulas above become
\[
  \left\{
    \begin{aligned}
      x^1(s)&=2x^0(e^s-1),&x^2(s)&=\frac{x^0}{2}(e^s+1),\\
      z(s)&={(x^0)}^{2}e^{2s},\\
      p^1(s)&=\frac{x^0}{2}e^s,&p^2(s)&=2x^0e^s.
    \end{aligned}
  \right.
\]

Fix a point \((x_1,x_2)\in U\). Select \(s\) and \(x^0\) so that
\((x_1,x_2)=(x^1(s),x^2(s))=(2x^0(e^s-1),\frac{x^0}{2}(e^s+1))\). This
equality implies
\[
  x^0=\frac{4x_2-x_1}{4};\qquad e^s=\frac{x_1+4x_2}{4x_2-x_1};
\]
so
\begin{align*}
  u(x)
  &=u(x^1(s),x^2(s))\\
  &=z(s)\\
  &={(x^0)}^2e^{2s}\\
  &=\frac{(x_1+4x_2)^2}{16}.
\end{align*}
\end{example}

\subsection{Compatibility conditions on boundary data}
Let \(x^0\in\Gamma\). We intend to use the characteristic ODEs
\eqref{eq:pde:first-order-characteristics} to construct a solution \(u\) to
\eqref{eq:pde:first-order}, \eqref{eq:pde:boundary-condition}, at least
near \(x^0\), and for this, we must determine appropriate initial
conditions
\begin{equation}
  \label{eq:pde:first-order-linear-characteristics}
  \bfp(0)=p^0,\qquad
  z(0)=z^0,\qquad
  \bfx(0)=x^0.
\end{equation}
We will assume throughout that \(\Gamma\) is flat (i.e., isometric to
\(\{\,x_n=0\,\}\)) at least near \(x^0\).\footnote{This can always be
  achieved, assuming some regularity of \(\Gamma\).}

Clearly if the curve \(\bfx(\blank)\) passes through \(x^0\), we should
insist that
\begin{equation}
  \label{eq:pde:initial-sol}
  z^0=g(x^0).
\end{equation}

What should we require concerning \(\bfp(0)=p^0\)? Since
\eqref{eq:pde:boundary-condition} implies
\[
  u(x_1,\dotsc,x_{n-1},0)=g(x_1,\dotsc,x_{n-1})
\]
near \(x^0\), we may differentiate this to find
\[
  u_{x_i}(x^0)=g_{x_i}(x^0).
\]
This, along with the PDE \eqref{eq:pde:first-order}, forces \(p^0\) to
satisfy
\begin{equation}
  \label{eq:pde:initial-grad}
  \left\{
    \begin{aligned}
      p_i^0&=g_{x_i}(x^0)&&1\leq i\leq n-1,\\
      F(p^0,z^0,x^0)&=0.
    \end{aligned}
  \right.
\end{equation}
These identities provide \(n\) equations for the \(n\) quantities
\(p^0=(p_1^0,\dotsc,p_n^0)\).

We call \eqref{eq:pde:initial-sol} and \eqref{eq:pde:initial-grad}
\emph{compatibility conditions}. A triple \(p^0,z^0,x^0\in\bbR^{2n+1}\)
satisfying \eqref{eq:pde:initial-sol}, \eqref{eq:pde:initial-grad}, is
called \emph{admissible}. Note \(z^0\) is uniquely determined by the
boundary condition and our choice of \(x^0\), but \(p^0\) satisfying
\eqref{eq:pde:initial-grad} may not exist or be unique.

Having ascertained what are appropriate boundary conditions for the
characteristic ODEs with \(\bfx(\blank)\) intersecting \(\Gamma\) at
\(x^0\), we proceed to construct a solution to \eqref{eq:pde:first-order},
\eqref{eq:pde:boundary-condition}, near \(x^0\). We now ask, can we somehow
appropriately perturb \((p^0,z^0,x^0)\), keeping the compatibility
conditions?

In other words, given a point \(y=(y_1,\dotsc,y_{n-1},0)\in\Gamma\), with
\(y\) close enough to \(x^0\), we intend to solve the ODEs
\eqref{eq:pde:first-order-characteristics} with initial conditions
\begin{equation}
  \label{eq:pde:first-order-inits-y}
  \bfp(0)=\bfq(y),\qquad
  z(0)=g(y),\qquad
  \bfx(0)=y.
\end{equation}

Our task is now to find a function
\(\bfq(\blank)=(q^1(\blank),\dotsc,q^n(\blank))\), so that
\begin{equation}
  \label{eq:pde:condition-grad-q}
  \bfq(x^0)=p^0
\end{equation}
and \(\bfq(y),g(y),y\) is admissible; i.e., so
\begin{equation}
  \label{eq:pde:condition-grad-q-init}
  \left\{
    \begin{aligned}
      q^i(y)&=g_{x_i}(y)&&1\leq i\leq n-1,\\
      F(\bfq(y),g(y),y)&=0,
    \end{aligned}
  \right.
\end{equation}
hold for all \(y\in\Gamma\) close to \(x^0\).

\begin{lemma}[Noncharacteristic boundary conditions]
  There exists a unique solution \(\bfq(\blank)\) of
  \eqref{eq:pde:condition-grad-q}, \eqref{eq:pde:condition-grad-q-init},
  for all \(y\in\Gamma\) sufficiently close to \(x^0\), provided
  \[
    F_{p_n}(p^0,z^0,x^0)\neq 0.
  \]
\end{lemma}
See \cite[\S 3.2 c]{evans} for proof.

More generally,
\[
  D_pF(p^0,z^0,x^0)\cdot\bfnu(x^0)\neq 0,
\]
\(\bfnu(x^0)\) denoting the outward unit normal to \(\partial U\) at
\(x^0\).

Now, given a point \(y=(y_1,\dotsc,y_{n-1},0)\) sufficiently close to
\(x^0\) we solve the characteristic ODEs \eqref{eq:pde:condition-grad-q}
subject to \eqref{eq:pde:condition-grad-q-init}.

Write
\[
  \left\{
    \begin{aligned}
      \bfp(s)&=\bfp(y,s)=\bfp(y_1,\dotsc,y_{n-1},s),\\
      z(s)&=z(y,s)=z(y_1,\dotsc,y_{n-1},s),\\
      \bfx(s)&=\bfx(y,s)=\bfx(y_1,\dotsc,y_{n-1},s),
    \end{aligned}
  \right.
\]
displaying the dependence of the solution of
\eqref{eq:pde:condition-grad-q}, \eqref{eq:pde:condition-grad-q-init}, with
respect to \(s\) and \(y\).

\begin{lemma}[Local invertibility]
  Assume we have the noncharacteristic condition \(F_{p_n}(p^0,z^0,x^0)\neq
  0\). Then there exists an open interval \(I\subset\bbR\) containing
  \(0\), a neighborhood \(W\) of \(x^0\) in \(\Gamma\subset\bbR^{n-1}\),
  and a neighborhood \(V\) of \(x^0\) in \(\bbR^n\), such that for each
  \(x\in V\) there exists a unique \(s\in I\), \(y\in W\) such that
  \[
    x=\bfx(y,s).
  \]
  The mappings \(x\mapsto s,y\) are \(C^2\).
\end{lemma}

In view of this lemma, for each \(x\in V\), we can uniquely solve (at least
locally) the equation
\begin{equation}
  \label{eq:pde:local-equations-sy}
  \left\{
    \begin{aligned}
      x&=\bfx(y,s),\\
      y&=\bfy(x),&s&=s(x).
    \end{aligned}
  \right.
\end{equation}

Define
\begin{equation}
  \label{eq:pde:solution-in-sy}
  \left\{
    \begin{aligned}
      u(x)&\defeq z(\bfy(x),s(x)),\\
      \bfp(x)&=\bfp(\bfy(x),s(x)),
    \end{aligned}
  \right.
\end{equation}
for \(x\in V\), \(s\) and \(y\) as in \eqref{eq:pde:local-equations-sy}.

Putting all of this together, we have the following result:
\begin{theorem}[Local existence theorem]
  The function \(u\) defined above is \(C^2\) and solves the PDE
  \[
    F(Du(x),u(x),x)=0\qquad x\in V,
  \]
  with the boundary condition
  \[
    u(x)=g(x)\qquad x\in \Gamma\cap V.
  \]
\end{theorem}
See \cite[\S 3.2.4]{evans} for more details.

\section{Power Series}
We now turn our attention to an important class of solutions to PDEs,
analytic solutions. We begin this section with a brief discussion on
noncharacteristic surfaces.

\subsection{Noncharacteristic surfaces}
Consider the \(k\)\textsup{th}-order quasilinear PDE
\begin{equation}
  \label{eq:pde:k-order-quasilinear}
  \sum_{|\alpha|=k}a_\alpha(D^{k-1}u,\dotsc,u,x)D^\alpha
  u+a_0(D^{k-1}u,\dotsc,u,x)=0,
\end{equation}
in some region in \(U\subset\bbR^n\). Let us assume for simplicity that
\(\Gamma\) is a smooth \((n-1)\)-dimensional hypersurface in \(U\). For any
\(x^0\in\Gamma\) let \(\bfnu(x^0)=\bfnu=(\nu_1,\dotsc,\nu_n)\) denote the
unit normal to \(\Gamma\) at \(x^0\). Furthermore, we define the
\emph{\(j\)\textsup{th} normal derivative} of \(u\) at \(x^0\in\Gamma\) by
\[
  \frac{\partial^j u}{\partial\nu^j}
  \defeq\sum_{|\alpha|=j}\binom{j}{\alpha}D^\alpha u\bfnu^\alpha
  =\sum_{\alpha_1+\dotsb+\alpha_n=j}\binom{j}{\alpha}
  \frac{\partial^j u}{\partial^{\alpha_1}x_1\dotsm\partial^{\alpha_n}x_n}
  \nu_1^{\alpha_1}\dotsm\nu_n^{\alpha_n}.
\]

Let \(g_0,\dotsc,g_{k-1}\colon\Gamma\to\bbR\) be \(k\) given functions. The
\emph{Cauchy problem} is to find a function \(u\) solving the PDE
\eqref{eq:pde:k-order-quasilinear}, subject to the boundary conditions
\begin{equation}
  \label{eq:pde:k-order-boundary-conditions}
  \left\{
    \begin{aligned}
      u&=g_0,\\
      \frac{\partial u}{\partial\nu}
      &=g_1,\\
      &\vdotswithin{=}\\
      \frac{\partial^{k-1} u}{\partial\nu^{k-1}} &=g_{k-1}
    \end{aligned}
  \right.
  \qquad\text{on \(\Gamma\).}
\end{equation}
We say that \eqref{eq:pde:k-order-boundary-conditions} prescribe the
\emph{Cauchy data} on \(\Gamma\).

%%% Local Variables:
%%% mode: latex
%%% TeX-master: "../Fall16-Notes"
%%% End:
