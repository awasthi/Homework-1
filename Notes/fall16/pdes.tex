\chapter{Introduction to Partial Differential Equations}
Here we summarize some important points about PDEs. The material is mostly
taken from Evans's \emph{Partial Differential Equations} \cite{evans} with
occasional detours to Strauss's \emph{Partial Differential Equations: An
  Introduction} \cite{strauss}. We will be following Dr.\@ Petrosyan's
\textsf{Course Log} which can be found here
\url{https://www.math.purdue.edu/~arshak/F16/MA523/courselog/}, i.e.,
summarizing the appropriate chapters from \cite{evans}.

\section{First-Order PDEs}
\subsection{The transport equation}
In this section, we consider the simplest first-order PDE, the
\emph{transport equation} with constant coefficients, i.e., the PDE
\begin{equation}
  \label{eq:pde:transport-equation}
  u_t+b\cdot Du=0\qquad\text{in \(\bbR\times(0,\infty)\),}
\end{equation}
where \(b\) is a fixed vector in \(\bbR^n\), and
\(u\colon\bbR^n\times[0,\infty)\to\bbR\) is the solution to the PDE. Our
task is to find solutions \(u\) which satisfy the equation
\eqref{eq:pde:transport-equation}.

To address this task, let us suppose for a moment that we have a (smooth)
solution \(u\) and try to compute it using the PDE
\eqref{eq:pde:transport-equation}. First, note that
\eqref{eq:pde:transport-equation} asserts that the directional derivative
\(D_{(b,1)}u=0\).




%%% Local Variables:
%%% mode: latex
%%% TeX-master: "../Fall16-Notes"
%%% End:
