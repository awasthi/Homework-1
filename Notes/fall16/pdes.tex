\chapter{Introduction to Partial Differential Equations}
Here we summarize some important points about PDEs. The material is mostly
taken from Evans's \emph{Partial Differential Equations} \cite{evans} with
occasional detours to Strauss's \emph{Partial Differential Equations: An
  Introduction} \cite{strauss}. We will be following Dr.\@ Petrosyan's
\textsf{Course Log} which can be found here
\url{https://www.math.purdue.edu/~arshak/F16/MA523/courselog/}, i.e.,
summarizing the appropriate chapters from \cite{evans}.

\section{Introduction}
\subsection{Partial differential equations}
\begin{definition}
  An expression of the form
  \begin{equation}
    \label{eq:pde:def-of-pde}
    F\bigl(D^ku(x),D^{k-1}u(x),\dotsc,Du(x),u(x),x\bigr)=0,
    \quad x\in U,
  \end{equation}
  is called a \emph{\(k\)th-order partial differential equation (PDE)},
  where
  \[
    F\colon\bbR^{n^k}\times\bbR^{n^{k-1}}\times\dotsb\times\bbR^n\times U\too\bbR
  \]
  is given, and
  \[
    u\colon U\too\bbR
  \]
  is the unknown.
\end{definition}

Here are some more definitions,
\begin{definition}
  \hfill
  \begin{enumerate}[label=(\roman*)]
  \item The partial differential equation \eqref{eq:pde:def-of-pde} is
    called \emph{linear} if it has the form
    \[
      \sum_{|\alpha|\leq k}a_\alpha(x)D^\alpha u=f(x)
    \]
    for given functions \(a_\alpha(|\alpha|\leq k)\), \(f\). This linear
    PDE is \emph{homogeneous} if \(f=0\).
  \item The PDE \eqref{eq:pde:def-of-pde} is \emph{semilinear} if it has
    the form
    \[
      \sum_{|\alpha|=k}a_\alpha D^\alpha u
      +a_0\bigl(D^{k-1}u,\dotsc,Du,u,x\bigr)=0.
    \]
  \item The PDE \eqref{eq:pde:def-of-pde} is \emph{quasilinear} if it has
    the form
    \[
      \sum_{|\alpha|=k}a_\alpha\bigl(D^{k-1}u,\dotsc,Du,u,x\bigr)D^\alpha u
      +a_0\bigl(D^{k-1}u,\dotsc,Du,u,x\bigr)=0.
    \]
  \item The PDE \eqref{eq:pde:def-of-pde} is \emph{fully nonlinear} if it
    depends upon the highest order derivatives.
  \end{enumerate}
\end{definition}

A \emph{system} of partial differential equations is, informally speaking,
a collection of several PDEs for several unknown functions.

\begin{definition}
  An expression of the form
  \begin{equation}
    \label{eq:pde:k-order-system}
    \bfF\bigl(D^k\bfu(x),D^{k-1}\bfu(x),\dotsc,D\bfu(x),\bfu(x),x\bigr)=0,
    \quad x\in U,
  \end{equation}
  is called a \emph{\(k\)th-order system of PDEs}, where
  \[
    \bfF\colon\bbR^{mn^k}\times\bbR^{mn^{k-1}}\times\dotsb\times\bbR^{mn}\times\bbR^m\times
    U\too\bbR^m
  \]
  is given and
  \[
    \bfu\colon U\too\bbR^m,\quad\bfu=(u^1,\dotsc,u^m)
  \]
  is the unknown.
\end{definition}
\begin{remark}
  We haven't talked much about systems of PDEs and I suspect we will not do
  so very much in this course.
\end{remark}
\subsection{Examples}
This is only a fraction of the PDEs listed in Evan's chapter.

\subsubsection{Linear equations}
\begin{enumerate}[label=\arabic*.,noitemsep]
\item Laplace's equation
  \[
    \Delta u=\sum_{i=1}^n u_{x_ix_i}=0.
  \]
\item Helmholtz's (or eigenvalue) equation
  \[
    -\Delta u=\lambda u.
  \]
\item Linear transport equation
  \[
    u_t+\sum_{i=1}^n b^iu_{x_i}=0.
  \]
\item Liouville's equation
  \[
    u_t-\sum_{i=1}^n(b^iu)_{x_i}=0.
  \]
\item Heat (or diffusion) equation
  \[
    u_t-\Delta u=0.
  \]
\item Wave equation
  \[
    u_{tt}-\Delta u=0.
  \]
\item Telegraph equation
  \[
    u_{tt}+du_t-u_{xx}=0.
  \]
\end{enumerate}
\subsubsection{Nonlinear equations}
\begin{enumerate}[label=\arabic*.,noitemsep]
\item Eikonal equation
  \[
    |Du|=1.
  \]
\item Nonlinear Poisson equation
  \[
    -\Delta u=f(u).
  \]
\item Inviscid Burgers' equation
  \[
    u_t+uu_x=0.
  \]
\end{enumerate}
and so on.

\section{The transport equation}
We begin our study with one of the simplest PDEs, the \emph{transport
  equation} with constant coefficients. This is the PDE
\begin{equation}
  \label{eq:pde:ptrans}
  u_t+b\cdot Du=0,\quad \text{in \(\bbR^n\times(0,\infty)\),}
\end{equation}
where \(b\) is a fixed vector in \(\bbR^n\), \(b=(b_1,\dotsc,b_n)\),
\(x=(x_1,\dotsc,x_n)\in\bbR^n\) is a typical point in space, \(t\geq 0\)
denotes a typical time and \(u\colon\bbR\times[0,\infty)\to\bbR\) is the
unknown, \(u=u(x,t)\). We write \(Du=D_xu=(u_{x_1},\dotsc,u_{x_n})\) for
the gradient of \(u\) with respect to the spatial variable \(x\).

So, which functions solve \eqref{eq:pde:ptrans}? Well, let us suppose for a
moment that \(u\) is a smooth solution to the PDE and let us try to compute
it. To do so, we first recognize that \eqref{eq:pde:ptrans} asserts that a
particular directional derivative of \(u\) vanishes, namely, \(D_bu=0\). We
exploit this by fixing a point \((x,t)\in\bbR^n\times(0,\infty)\) and
defining
\[
  z(s)\defeq u(x+sb,t+s),\quad s\in\bbR.
\]
Then we calculate
\begin{align*}
  \dot z(s)&=Du(x+sb,t+s)\cdot b+u_t(x+sb,t+s)\\
           &=0,
\end{align*}
the second equality holding by \eqref{eq:pde:ptrans}. Thus, \(z\) is a
constant function of \(s\), and consequently for each \((x,t)\), \(u\) is
constant on the line through \((x,t)\) with direction
\((b,1)\in\bbR^{n+1}\). Hence, if we know the value of \(u\) at any point
on ecah such line, we know its value everywhere in
\(\bbR^n\times(0,\infty)\).

\section{Characteristics}
\subsection{Derivation of characteristic ODEs}
Consider the nonlinear first-order PDE
\begin{equation}
  \label{eq:pde:pde-1}
  F(Du,u,x)=0\quad\text{in \(U\),}
\end{equation}
subject now to the boundary condition
\begin{equation}
  \label{eq:pde:initial-cond-1}
  u=g\quad\text{on \(\Gamma\),}
\end{equation}
where \(\Gamma\subseteq\partial U\) and \(g\colon\Gamma\to\bbR\) are
given. We hereafter suppose that \(F\), \(g\) are smooth functions.

We now develop the method of \emph{characteristics}, which solves
\eqref{eq:pde:pde-1} and \eqref{eq:pde:initial-cond-1} by converting the
PDE into an appropriate system of ODEs. Suppose \(u\) solves the
\eqref{eq:pde:pde-1}, \eqref{eq:pde:initial-cond-1} and fix any point
\(x\in U\). We would like to calculate \(u(x)\) by finding some curve lying
within \(U\), connecting \(x\) with a point \(x^0\in\Gamma\) and along
which we can compute \(u\). Since \eqref{eq:pde:initial-cond-1} says \(u=g\) on
\(\Gamma\), we know the value of \(u\) at the one end \(x^0\). We hope then
to be able to calculate \(u\) all along the curve, and so in particular at
\(x\).
\subsubsection{Finding the characteristic ODEs}
How can we choose the curve so all this will work? Let us suppose it is
described parametrically by the function
\(\bfx(s)=\bigl(x^1(s),\dotsc,x^n(s)\bigr)\), the parameter \(s\) lying in
some subinterval of \(\bbR\). Assuming \(u\) is a \(C^2\) solution of
\eqref{eq:pde:pde-1}, we define also
\[
  z(s)\defeq u\bigl(\bfx(s)\bigr).
\]
In addition, set
\[
  \bfp(s)\defeq Du(\bfx(s));
\]
that is, \(\bfp(s)=\bigl(p^1(s),\dotsc,p^n(s)\bigr)\), where
\begin{equation}
  \label{eq:pde:char-curve-p-i}
  p^i(s)=u_{x_i}\bigl(\bfx(s)\bigr),
\end{equation}
\(1\leq i\leq n\). So \(z\) gives the values of \(u\) along the curve and
\(\bfp\) records the values of the gradient \(Du\). We must choose a
function \(\bfx\) in such a way that we can compute \(z\) and \(\bfp\).

For this, first differentiate \eqref{eq:pde:char-curve-p-i}
\[
  \dot{p}^i(s)=\sum_{j=1}^n u_{x_ix_j}\bigl(\bfx(s)\bigr)\dot{x}^j(s)
\]
This expression is not too promising, since it involves the second
derivatives of \(u\). On the other hand, we can also differentiate the PDE
\eqref{eq:pde:pde-1} with respect to \(x_i\) to get
\[
  \sum_{j=1}^n
  \frac{\partial}{\partial p_j}F(Du,u,x)u_{x_jx_i}
  +\frac{\partial}{\partial z}F(Du,u,x)u_{x_i}
  +\frac{\partial}{\partial x_i}F(Du,u,x)=0.
\]
We are able to employ this identity to get rid of the \emph{dangerous}
second derivative terms provided we first set
\[
  \dot{x}^j(s)=\frac{\partial}{\partial
    p_j}F\bigl(\bfp(s),z(s),\bfx(s)\bigr).
\]
Assuming now that the above equation holds, we can evaluate the partials
\[
  \begin{aligned}
    \sum_{j=1}^n\frac{\partial}{\partial
      p_j}F\bigl(&\bfp(s),z(s),\bfx(s)\bigr)\\
    &+\frac{\partial}{\partial z}F\bigl(\bfp(s),z(s),\bfx(s)\bigr)p^{i}(s)
    +\frac{\partial}{\partial x_i}F\bigl(\bfp(s),z(s),\bfx(s)\bigr)=0.
  \end{aligned}
\]
Substitute this expression and the previous one into the derivative for
\(\dot{p}^i\) and we get
\[
  \begin{aligned}
    \dot{p}^i(s)=\frac{\partial}{\partial x_i}F\bigl(\bfp(s),z(s),\bfx(s)\bigr)
  \end{aligned}
\]
Finally, we differentiate \(z\) to get
\[
  \dot z(s)=
  \sum_{j=1}^n\frac{\partial}{\partial x_j}u\bigl(\bfx(s)\bigr)
  \dot{x}^j(s)=
  \sum_{j=1}^n p^j(s)\frac{\partial}{\partial
    p_j}F\bigl(\bfp(s),z(s),\bfx(s)\bigr),
\]
the second equality holding by --fuck this guy for numbering every
expression--(5) and (8)--whatever they are.

We summarize by rewriting equations (8)--(10) in vector notation:
\begin{equation}
  \label{eq:pde:characteristics}
  \left\{
    \begin{aligned}
      \text{(a) }\dot\bfp(s)&=-D_xF\bigl(\bfp(s),z(s),\bfx(s)\bigr)
      -D_zF\bigl(\bfp(s),z(s),\bfx(s)\bigr)\bfp(s),\\
      \text{(b) }\dot z(s)&=D_pF\bigl(\bfp(s),z(s),\bfx(s)\bigr)\cdot\bfp(s),\\
      \text{(c) }\dot\bfx(s)&=D_pF\bigl(\bfp(s),z(s),\bfx(s)\bigr).
    \end{aligned}
  \right.
\end{equation}

This important system of \(2n+1\) first-order ODEs comprises the
\emph{characteristic equations} of the nonlinear first-order PDE
\eqref{eq:pde:pde-1}. The functions \(\bfp=(p^1,\dotsc,p^n)\), \(z\),
\(\bfx=(x^1,\dotsc,x^n)\) are a called the \emph{characteristics}. We will
sometimes refer to \(\bfx\) as the \emph{projected characteristics}: it is
the projection of the full characteristics
\((\bfp,z,\bfx)\subseteq\bbR^{2n+1}\) onto the physical region
\(U\subseteq\bbR^n\).

\begin{theorem}[Structure of characteristic ODEs]
  Let \(u\in C^2(U)\) solve the nonlinear, first-order partial differential
  equation \eqref{eq:pde:pde-1} in \(U\). Assume \(\bfx\) solves the ODEs
  \eqref{eq:pde:characteristics}\textnormal{(c)}, where \(\bfp=Du\),
  \(z=u\). Then \(\bfp\) solves the ODE
  \eqref{eq:pde:characteristics}\textnormal{(a)} and \(z\) solves the ODE
  \eqref{eq:pde:characteristics}\textnormal{(b)}, for those \(s\) such that
  \(\bfx\in U\).
\end{theorem}
\subsubsection{Examples}
\subsubsection[\(F\) linear]{\(\bm F\)  linear}
Consider first the situation that \eqref{eq:pde:pde-1} is linear and
homogeneous, and thus has the form
\[
  F(Du,u,x)=\bfb(x)\cdot D u(x)+c(x)u(x)=0,\qquad x\in U.
\]
Then \(F(p,z,x)=\bfb(x)\cdot p+c(x)z\), and so
\[
  D_pF=\bfb(x).
\]
In this circumstance \eqref{eq:pde:characteristics}(c) becomes
\[
  \bfx(s)=\bfb\bigl(\bfx(s)\bigr),
\]
an ODE involving only the function \(\bfx\). Furthermore
\eqref{eq:pde:characteristics}(b) becomes
\begin{equation}
  \label{eq:pde:1st-order-linear}
  \dot z(s)=\bfb\bigl(\bfx(s)\bigr)\cdot\bfp(s).
\end{equation}
Since \(\bfp(\cdot)=Du\bigl(\bfx(\cdot)\bigr)\), the PDE simplifies the
above to
\[
  \dot z(s)=-c\bigl(\bfx(s)\bigr)z(s).
\]
This ODE is linear in \(z\), once we know the function \(\bfx\) by solving
its ODE.

In summary,
\begin{equation}
  \label{eq:pde:1st-order-linear-chars}
  \left\{
    \begin{aligned}
      \text{(a) }\dot\bfx(s)&=\bfb\bigl( \bfx(s) \bigr)\\
      \text{(b) }\dot z(s)&=-c\bigl(\bfx(s)\bigr)z(s)
    \end{aligned}
  \right.
\end{equation}
comprise the characteristic equations for the linear, first-order PDE
\eqref{eq:pde:1st-order-linear}.

\begin{example}
  We demonstrate the utility of equations
  \eqref{eq:pde:1st-order-linear-chars} by explicitly solving the problem
  \begin{equation}
    \label{eq:pde:linear-ex-1}
    \left\{
      \begin{aligned}
        x_1u_{x_2}-x_2u_{x_1}&=u\quad\text{in \(U\)}\\
        u&=g\quad\text{on \(\Gamma\)},
      \end{aligned}
    \right.
  \end{equation}
  where \(U\) is the quadrant \(\{\,x_1>0,x_2>0\,\}\) and
  \(\Gamma=\{\,x_1>0,x_2=0\,\}\subseteq\partial U\). The PDE in
  \eqref{eq:pde:linear-ex-1} is of the form
  \eqref{eq:pde:1st-order-linear}, for \(\bfb=(-x_2,x_1)\) and
  \(c=-1\). Thus the equations \eqref{eq:pde:1st-order-linear-chars} read
  \begin{equation}
    \label{eq:pde:linear-ex-1-chars}
    \left\{
      \begin{aligned}
        (x^1,x^2)(s)&=(x^0\cos s,x^0\sin s)\\
        z(s)&=z^0\rme^s=g(x^0)\rme^s,
      \end{aligned}
    \right.
  \end{equation}
  where \(x^0\geq 0\), \(0\leq s\leq \pi/2\). Fix a point \((x_1,x_2)\in
  U\). We select \(s>0\), \(x^0>0\) so that
  \((x_1,x_2)=\bigl(x^1(s),x^2(s)\bigr)=(x^0\cos s,x^0\sin s)\). That is,
  \(x^0=\bigl({x_1}^2+{x_2}^2\bigr)^{1/2}\),
  \(s=\arctan(x_2/x_1)\). Therefore,
  \begin{align*}
    u(x_1,x_2)
    &=u\bigl( x^1(s),x^2(s) \bigr)\\
    &=z(s)\\
    &=g(x^0)\rme^s\\
    &=g\bigl(({x_1}^2+{x_2}^2)^{1/2}\bigr)\rme^{\arctan(x_2/x_1)}.
  \end{align*}
\end{example}

\subsubsection[\(F\) quasilinear]{\(\bm F\)  quasilinear}
The PDE \eqref{eq:pde:pde-1} is quasilinear if it has the form
\begin{equation}
  \label{eq:pde:1st-order-quasilinear}
  F(Du,u,x)=\bfb\bigl(x,u(x)\bigr)\cdot Du(x)+c\bigl(x,u(x)\bigr)=0.
\end{equation}
In this circumstance \(F(p,z,x)=\bfb(x,z)\cdot p+c(x,z)\); whence
\[
  D_pF=\bfb(x,z).
\]
Hence equation \eqref{eq:pde:1st-order-linear-chars}(c) reads
\[
  \dot\bfx=\bfb\bigl( \bfx(s),z(s) \bigr),
\]
and \eqref{eq:pde:1st-order-linear-chars}(b) becomes
\begin{align*}
  \dot z(s)&=\bfb\bigl(\bfx(s),z(s)\bigr)\cdot\bfp(s)\\
           &=-c\bigl(\bfx(s),z(s)\bigr)
\end{align*}
by \eqref{eq:pde:1st-order-quasilinear}. Consequently
\begin{equation}
  \label{eq:pde:1st-order-quasilinear-chars}
  \left\{
    \begin{aligned}
      \text{(a) } \dot\bfx(s)&=\bfb\bigl( \bfx(s),z(s) \bigr),\\
      \text{(b) } \dot z(s)&=-c\bigl( \bfx(s),z(s) \bigr)
    \end{aligned}
  \right.
\end{equation}
are the characteristic equations for the quasilinear first-order PDE.

\begin{example}
  The characteristic ODEs \eqref{eq:pde:1st-order-quasilinear-chars} are in
  general difficult to solve, and so we work out in this example the
  simpler case of a boundary-value problem for a semilinear PDE:
  \begin{equation}
    \label{eq:pde:quasilinear-ex-2}
    \left\{
      \begin{aligned}
        u_{x_1}+u_{x_2}&=u^2\quad\text{in \(U\)}\\
        u&=g\quad\text{on \(\Gamma\).}
      \end{aligned}
    \right.
  \end{equation}
  Now \(U\) is the half-space \(\{\,x_2>0\,\}\) and
  \(\Gamma=\{\,x_2=0\,h\}=\partial U\). Here \(\bfb=(1,1)\) and
  \(c=-z^2\). Then \eqref{eq:pde:1st-order-quasilinear-chars} becomes
  \[
    \left\{
      \begin{aligned}
        (\dot{x}^1,\dot{x}^2)&=1\\
        \dot z&=z^2.
      \end{aligned}
    \right.
  \]
  Consequently
  \[
    \left\{
    \begin{aligned}
      (x^1,x^2)(s)&=(x^0+s,s)\\
      z(s)&=\frac{z^0}{1-sz^0}\\
      &=\frac{g(x^0)}{1-sg(x^0)},
    \end{aligned}
    \right.
  \]
  where \(x^0\in\bbR\), \(s\geq 0\), provided the denominator is not zero.

  Fix a point \(x_1,x_2\in U\). We select \(s>0\) and \(x^0\in\bbR\) so
  that \((x_1,x_2)=\bigl( x^1(s),x^2(s) \bigr)=(x^0+s,s)\); that is,
  \(x^0=x_1-x_2\), \(s=x_2\). Then
  \begin{align*}
    u(x_1,x_2)&=u\bigl( x^1(s),x^2(s) \bigr)\\
              &=z(s)\\
              &=\frac{g(x^0)}{1-sg(x^0)}\\
              &=\frac{g(x_1-x_2)}{1-x_2g(x_1-x_2)}.
  \end{align*}
  This solution of course make sense only if \(1-x_2g(x_1-x_2)\neq 0\).
\end{example}

\subsubsection[\(F\) fully nonlinear]{\(\bm F\) fully nonlinear}
In the general case, the full characteristic equations
\eqref{eq:pde:characteristics} must be integrated, if possible.
\begin{example}
  Consider the fully nonlinear problem
  \begin{equation}
    \label{eq:pde:nonlinear-ex-3}
    \left\{
    \begin{aligned}
      u_{x_1}u_{x_2}&=u\quad\text{on \(U\)}\\
      u&={x_2}^2\quad\text{on \(\Gamma\)}
    \end{aligned}
    \right.
  \end{equation}
  where \(U=\{\,x_1>0\,\}\), \(\Gamma=\{\,x_1=0\,\}=\partial U\). Here
  \(F(p,z,x)=p_1p_2-z\), and hence the characteristic ODEs
  \eqref{eq:pde:characteristics} become
  \[
    \left\{
    \begin{aligned}
      (\dot p^1,\dot p^2)&=(p^1,p^2)\\
      \dot z&=2p^1p^2\\
      (\dot x^1,\dot x^2)&=(p^2,p^1).
    \end{aligned}
    \right.
  \]
  We integrate these equations to find
  \[
    \left\{
    \begin{aligned}
      (x^1,x^2)(s)&=\bigl(p_2^0(\rme^s-1),p_1^0(\rme^s-1)\bigr)\\
      z(s)&=z^0+p_1^0p_2^0(\rme^{2s}-1)\\
      (p^1,p^2)(s)&=(p_1^0\rme^s,p_2^0\rme^s),
    \end{aligned}
    \right.
  \]
  where \(x^0\in\bbR\), \(s\in\bbR\), and \(z^0={(x^0)}^2\).

  We must determine \(p^0=(p_1^0,p_2^0)\). Since \(u={x_2}^2\) on
  \(\Gamma\), \(p_2^0=u_{x_2}(0,x^0)=2x^0\). Furthermore the PDE
  \(u_{x_1}u_{x_2}=u\) itself implies \(p_1^0p_2^0=z^0={(x^0)}^2\), and so
  \(p_1^0=x^2/2\). Consequently the formulas above become
  \[
    \left\{
      \begin{aligned}
        (x^1,x^2)(s)&=\bigl(2x^9(\rme^s-1),x^0(\rme^s+1)/2\bigr)\\
        z(s)&={(x^0)}^2\rme^{2s}\\
        (p^1,p^2)(s)&=(x^0\rme^s/2,2x^0\rme^s).
      \end{aligned}
    \right.
  \]

  Fix a point \((x_1,x_2)\in U\). Select \(s\) and \(x^0\) so that
  \[
    (x_1,x_2)=\bigl( x^1(s),x^2(s) \bigr)=\bigl(
    2x^0(\rme^s-1),x^0(\rme^s+1)/2 \bigr).
  \]
  This equality implies \(x^0=(4x_2-x_1)/4\),
  \(\rme^s=(x_1+4x_2)/(4x_2-x_1)\); and so
  \begin{align*}
    u(x_1,x_2)&=u\bigl( x^1(s),x^2(s) \bigr)\\
              &=\frac{(x_1+4x_2)^2}{16}.
  \end{align*}
\end{example}

\section{Boundary conditions}
\subsection{Straightening the boundary}
We intend in the following section to invoke the characteristic ODE
\eqref{eq:pde:characteristics} to actually solve the boundary-value problem
\eqref{eq:pde:pde-1}, \eqref{eq:pde:initial-cond-1}, at least in a small
region near an appropriate portion \(\Gamma\) of \(\partial U\). In order
to simplify the relevant calculations, it is convenient first fix any point
\(x^0\in\partial U\). Then utilizing the notation from the appendix \S C.1
of \cite{evans}, we find smooth mappings
\(\bfPhi,\bfPsi\colon\bbR^n\to\bbR^n\) such that \(\bfPsi=\bfPhi^{-1}\) and
\(\bfPhi\) straightens \(\partial U\) near \(x^0\).

Given a function \(u\colon U\to\bbR\), let us write \(V\defeq\bfPhi(U)\)
and set
\begin{equation}
  \label{eq:pde:straight-sol}
  v(y)\defeq u\bigl( \bfPsi(y) \bigr)\qquad\text{\(y\in V\).}
\end{equation}
Then
\begin{equation}
  \label{eq:pde:unstraight-sol}
  u(x)=v\bigl( \bfPhi(x) \bigr)\qquad\text{\(x\in U\)}.
\end{equation}
Now suppose that \(u\) is a \(C^1\) solution of our boundary-value problem
\eqref{eq:pde:pde-1}, \eqref{eq:pde:initial-cond-1} in \(U\). What PDE does
\(v\) then satisfy in \(V\)?

According to \eqref{eq:pde:unstraight-sol}, we have
\[
  u_{x_i}(x)=\sum_{k=1}^n v_{y_k}\bigl( \bfPhi(x) \bigr)\Phi_{x_i}^k(x)
\]
i.e.,
\[
  Du(x)=Dv(y)D\bfPhi(x).
\]
Thus,
\begin{align*}
  0&=F\bigl(Du(x),u(x),x\bigr)\\
   &=F\bigl( Dv(y),D\bfPhi\bigl(\bfPsi(y)\bigr),v(y),\Psi(y) \bigr).
\end{align*}
In addition \(v=h\) on \(\Delta\), where \(\Delta\defeq\bfPhi(\Gamma)\) and
\(h(y)\defeq g\bigl(\bfPsi(y)\bigr)\).

In summary, our problem \eqref{eq:pde:pde-1}, \eqref{eq:pde:initial-cond-1}
converts into a problem having the same form.

\subsubsection{Compatibility conditions on boundary}
In view of the foregoing computations, if we are given a point
\(x^0\in\Gamma\) we may as well assume that the outset that \(\Gamma\) is
flat near \(x^0\), lying in the plane \(\{\,x_n=0\,\}\).

We intend now to utilize the characteristic ODE to construct a solution to
\eqref{eq:pde:pde-1}, at least near \(x^0\), and for this we must discover
appropriate initial conditions
\begin{equation}
  \label{eq:pde:initial-conds-2}
  \bfp(0)=p^0,\quad
  z(0)=z^0,\quad
  \bfx(0)=x^0.
\end{equation}

Now clearly if the curve \(\bfx\) passes through \(x^0\), we should insist
that
\begin{equation}
  \label{eq:pde:initial-z}
  z^0=g(x^0).
\end{equation}

What should we require concerning \(\bfp(0)=p^0\)? Since
\eqref{eq:pde:initial-cond-1} implies
\(u(x_1,\dotsc,x_{n-1},0)=g(x_1,\dotsc,x_{n-1})\) near \(x^0\), we may
differentiate to find
\[
  u_{x_i}(x^0)=g_{x_i}(x^0)\quad \text{\(i=1,\dotsc,n-1\).}
\]
As we also want the PDE \eqref{eq:pde:pde-1} to hold, we should therefore
insist \(p^0=(p_1^0,\dotsc,p_n^0)\) satisfies these relations
\begin{equation}
  \label{eq:pde:initial-p}
  \left\{
    \begin{aligned}
      p_i^0&=g_{x_i}(x^0)\qquad\text{\(i=1,\dotsc,n-1\)}\\
      F(p^0,z^0,x^0)&=0.
    \end{aligned}
  \right.
\end{equation}
These identities provide \(n\) equations for the \(n\) quantities
\(p^0=(p_1^0,\dotsc,p_n^0)\).

We call \eqref{eq:pde:initial-z} and \eqref{eq:pde:initial-p} the
\emph{compatibility conditions.} A triple \((p^0,z^0,x^0)\in\bbR^{2n+1}\)
verifying \eqref{eq:pde:initial-z}, \eqref{eq:pde:initial-p} is
\emph{admissible}. Note that \(z^0\) is uniquely determined by the boundary
condition and our choice of the point \(x^0\), but a vector \(p^0\)
satisfying \eqref{eq:pde:initial-p} may not exist and it may not be
unique.

\subsubsection{Noncharacteristic boundary data}
So now assume as above that \(x^0\in\Gamma\), that \(\Gamma\) near \(x^0\)
lies in the plane \(\{\,x_n=0\,\}\), and that the triple \((p^0,z^0,x^0)\)
is admissible. We are planning to construct a solution \(u\) of
\eqref{eq:pde:pde-1}, \eqref{eq:pde:initial-cond-1} in \(U\) near \(x^0\)
by integrating by parts the characteristic ODE
\eqref{eq:pde:characteristics}. So far we have ascertained \(\bfx(0)=x^0\),
\(z(0)=z^0\), \(\bfp(0)=p^0\) are appropriate boundary conditions for the
characteristic ODE, with \(\bfx\) intersecting \(\Gamma\) at \(x^0\). But
we will need in fact to solve these ODEs for \emph{nearby} initial points
as well, and must consequently now ask if we can somehow appropriately
perturb \((p^0,z^0,x^0)\), keeping the compatibility conditions.

In other words, given a point \(y=(y_1,\dotsc,y_{n-1},0)\in\Gamma\), with
\(y\) close to \(x^0\), we intend to solve the characteristic ODE
\begin{equation}
  \label{eq:pde:chars-2}
  \left\{
    \begin{aligned}
      \text{(a) }\dot\bfp(s)
      &=-D_xF\bigl(\bfp(s),z(s),\bfx(s)\bigr)
      -D_zF\bigl(\bfp(s),z(s),\bfx(s)\bigr)\\
      \text{(b) }\dot z(s)&=D_pF\bigl(\bfp(s),z(s),\bfx(s)\bigr)\cdot\bfp(s)\\
      \text{(c) }\dot\bfx(s)&=D_pF\bigl(\bfp(s),z(s),\bfx(s)\bigr),
    \end{aligned}
  \right.
\end{equation}
with the initial conditions
\begin{equation}
  \label{eq:pde:initial-cond-2}
  \bfp(0)=\bfq(y),\quad z(0)=g(y),\quad\bfx(0)=y.
\end{equation}

Our task then is to find a function \(\bfq=(q^1,\dotsc,q^n)\), so that
\begin{equation}
  \label{eq:pde:initial-q}
  \bfq(x^0)=p^0
\end{equation}
and \(\bigl(\bfq(y),g(y),y\bigr)\) is admissible; that is, the
compatibility conditions
\begin{equation}
  \label{eq:pde:compat-cond-q}
  \left\{
    \begin{aligned}
      q^i(y)&=g_{x_i}(y)\qquad 1\leq i\leq n-1\\
      F\bigl( \bfq(y),g(y),y \bigr)&=0
    \end{aligned}
  \right.
\end{equation}
hold for all \(y\in\Gamma\) close to \(x^0\).

\begin{lemma}[Noncharacteristic boundary conditions]
  There exists a unique solution \(\bfq\) of \eqref{eq:pde:initial-q},
  \eqref{eq:pde:compat-cond-q} for all \(y\in\Gamma\) sufficiently close to
  \(x^0\), provided
  \begin{equation}
    \label{eq:pde:nonchar-cond}
    F_{p_n}(p^0,z^0,x^0)\neq 0.
  \end{equation}
\end{lemma}

We say the admissible triple \((p^0,z^0,x^0)\) is \emph{noncharacteristic}
if \eqref{eq:pde:nonchar-cond} holds. We henceforth assume this condition.
\begin{proof}
  To simplify notation, let us now temporarily write
  \(y=(y_1,\dotsc,y_n)\in\bbR^n\). We apply the implicit function theorem
  to the mapping
  \[
    \bfG\colon\bbR^n\times\bbR^n\too\bbR^n,\qquad
    \bfG(p,y)=\bigl(G^1(p,y),\dotsc,G^n(p,y)\bigr),
  \]
  where
  \[
    \left\{
      \begin{aligned}
        G^i(p,y)&=p_i-g_{x_i}(y)\qquad 1\leq i\leq n-1,\\
        G^n(p,y)&=F\bigl(p,g(y),y\bigr).
      \end{aligned}
    \right.
  \]
  Now \(\bfG(p^0,x^0)=0\), according to \eqref{eq:pde:initial-z},
  \eqref{eq:pde:initial-conds-2}. Also
  \[
    D_p\bfG(p^0,x^0)
    =
    \begin{bmatrix}
      1&\cdots&0&0\\
      \vdots&\ddots&\vdots&\vdots\\
      0&\cdots&1&0\\
      F_{p_1}(p^0,z^0,x^0)&\cdots&F_{p_{n-1}}(p^0,z^0,x^0)&F_{p_n}(p^0,z^0,x^0)
    \end{bmatrix}
  \]
  and thus
  \[
    \det D_p\bfG(p^0,x^0)=F_{p_n}(p^0,z^0,x^0)\neq 0,
  \]
  in view of the noncharacteristic condition \eqref{eq:pde:nonchar-cond}.
\end{proof}

\section{The One-Dimensional Wave Equation}
We now turn our attention to second-order partial differential equations.

%%% Local Variables:
%%% mode: latex
%%% TeX-master: "../Fall16-Notes"
%%% End:
