\chapter{Introduction to Partial Differential Equations}
Here we summarize some important points about PDEs. The material is mostly
taken from Evans's \emph{Partial Differential Equations} \cite{evans} with
occasional detours to Strauss's \emph{Partial Differential Equations: An
  Introduction} \cite{strauss}. We will be following Dr.\@ Petrosyan's
\textsf{Course Log} which can be found here
\url{https://www.math.purdue.edu/~arshak/F16/MA523/courselog/}, i.e.,
summarizing the appropriate chapters from \cite{evans}.

\section{Introduction}
\subsection{Partial differential equations}
\begin{definition}
  An expression of the form
  \begin{equation}
    \label{eq:pde:def-of-pde}
    F\bigl(D^ku(x),D^{k-1}u(x),\dotsc,Du(x),u(x),x\bigr)=0,
    \quad x\in U,
  \end{equation}
  is called a \emph{\(k\)th-order partial differential equation (PDE)},
  where
  \[
    F\colon\bfR^{n^k}\times\bfR^{n^{k-1}}\times\dotsb\times\bfR^n\times U\too\bfR
  \]
  is given, and
  \[
    u\colon U\too\bfR
  \]
  is the unknown.
\end{definition}

Here are some more definitions,
\begin{definition}
  \hfill
  \begin{enumerate}[label=(\roman*)]
  \item The partial differential equation \eqref{eq:pde:def-of-pde} is
    called \emph{linear} if it has the form
    \[
      \sum_{|\alpha|\leq k}a_\alpha(x)D^\alpha u=f(x)
    \]
    for given functions \(a_\alpha(|\alpha|\leq k)\), \(f\). This linear
    PDE is \emph{homogeneous} if \(f=0\).
  \item The PDE \eqref{eq:pde:def-of-pde} is \emph{semilinear} if it has
    the form
    \[
      \sum_{|\alpha|=k}a_\alpha D^\alpha u
      +a_0\bigl(D^{k-1}u,\dotsc,Du,u,x\bigr)=0.
    \]
  \item The PDE \eqref{eq:pde:def-of-pde} is \emph{quasilinear} if it has
    the form
    \[
      \sum_{|\alpha|=k}a_\alpha\bigl(D^{k-1}u,\dotsc,Du,u,x\bigr)D^\alpha u
      +a_0\bigl(D^{k-1}u,\dotsc,Du,u,x\bigr)=0.
    \]
  \item The PDE \eqref{eq:pde:def-of-pde} is \emph{fully nonlinear} if it
    depends upon the highest order derivatives.
  \end{enumerate}
\end{definition}

A \emph{system} of partial differential equations is, informally speaking,
a collection of several PDEs for several unknown functions.

\begin{definition}
  An expression of the form
  \begin{equation}
    \label{eq:pde:k-order-system}
    \bfF\bigl(D^k\bfu(x),D^{k-1}\bfu(x),\dotsc,D\bfu(x),\bfu(x),x\bigr)=0,
    \quad x\in U,
  \end{equation}
  is called a \emph{\(k\)th-order system of PDEs}, where
  \[
    \bfF\colon\bfR^{mn^k}\times\bfR^{mn^{k-1}}\times\dotsb\times\bfR^{mn}\times\bfR^m\times
    U\too\bfR^m
  \]
  is given and
  \[
    \bfu\colon U\too\bfR^m,\quad\bfu=(u^1,\dotsc,u^m)
  \]
  is the unknown.
\end{definition}
\begin{remark}
  We haven't talked much about systems of PDEs and I suspect we will not do
  so very much in this course.
\end{remark}
\subsection{Examples}
This is only a fraction of the PDEs listed in Evan's chapter.

\subsubsection{Linear equations}
\begin{enumerate}[label=\arabic*.,noitemsep]
\item Laplace's equation
  \[
    \Delta u=\sum_{i=1}^n u_{x_ix_i}=0.
  \]
\item Helmholtz's (or eigenvalue) equation
  \[
    -\Delta u=\lambda u.
  \]
\item Linear transport equation
  \[
    u_t+\sum_{i=1}^n b^iu_{x_i}=0.
  \]
\item Liouville's equation
  \[
    u_t-\sum_{i=1}^n(b^iu)_{x_i}=0.
  \]
\item Heat (or diffusion) equation
  \[
    u_t-\Delta u=0.
  \]
\item Wave equation
  \[
    u_{tt}-\Delta u=0.
  \]
\item Telegraph equation
  \[
    u_{tt}+du_t-u_{xx}=0.
  \]
\end{enumerate}
\subsubsection{Nonlinear equations}
\begin{enumerate}[label=\arabic*.,noitemsep]
\item Eikonal equation
  \[
    |Du|=1.
  \]
\item Nonlinear Poisson equation
  \[
    -\Delta u=f(u).
  \]
\item Inviscid Burgers' equation
  \[
    u_t+uu_x=0.
  \]
\end{enumerate}
and so on.

\section{The transport equation}
We begin our study with one of the simplest PDEs, the \emph{transport
  equation} with constant coefficients. This is the PDE
\begin{equation}
  \label{eq:pde:ptrans}
  u_t+b\cdot Du=0,\quad \text{in \(\bfR^n\times(0,\infty)\),}
\end{equation}
where \(b\) is a fixed vector in \(\bfR^n\), \(b=(b_1,\dotsc,b_n)\),
\(x=(x_1,\dotsc,x_n)\in\bfR^n\) is a typical point in space, \(t\geq 0\)
denotes a typical time and \(u\colon\bfR\times[0,\infty)\to\bfR\) is the
unknown, \(u=u(x,t)\). We write \(Du=D_xu=(u_{x_1},\dotsc,u_{x_n})\) for
the gradient of \(u\) with respect to the spatial variable \(x\).

So, which functions solve \eqref{eq:pde:ptrans}? Well, let us suppose for a
moment that \(u\) is a smooth solution to the PDE and let us try to compute
it. To do so, we first recognize that \eqref{eq:pde:ptrans} asserts that a
particular directional derivative of \(u\) vanishes, namely, \(D_bu=0\). We
exploit this by fixing a point \((x,t)\in\bfR^n\times(0,\infty)\) and
defining
\[
  z(s)\defeq u(x+sb,t+s),\quad s\in\bfR.
\]
Then we calculate
\begin{align*}
  \dot z(s)&=Du(x+sb,t+s)\cdot b+u_t(x+sb,t+s)\\
           &=0,
\end{align*}
the second equality holding by \eqref{eq:pde:ptrans}. Thus, \(z\) is a
constant function of \(s\), and consequently for each \((x,t)\), \(u\) is
constant on the line through \((x,t)\) with direction
\((b,1)\in\bfR^{n+1}\). Hence, if we know the value of \(u\) at any point
on ecah such line, we know its value everywhere in
\(\bfR^n\times(0,\infty)\).

\section{Characteristics}
\subsection{Derivation of characteristic ODEs}
Consider the nonlinear first-order PDE
\begin{equation}
  \label{eq:pde:pde-1}
  F(Du,u,x)=0\quad\text{in \(U\),}
\end{equation}
subject now to the boundary condition
\begin{equation}
  \label{eq:pde:initial-cond-1}
  u=g\quad\text{on \(\Gamma\),}
\end{equation}
where \(\Gamma\subseteq\partial U\) and \(g\colon\Gamma\to\bfR\) are
given. We hereafter suppose that \(F\), \(g\) are smooth functions.

We now develop the method of \emph{characteristics}, which solves
\eqref{eq:pde:pde-1} and \eqref{eq:pde:initial-cond-1} by converting the
PDE into an appropriate system of ODEs. Suppose \(u\) solves the
\eqref{eq:pde:pde-1}, \eqref{eq:pde:initial-cond-1} and fix any point
\(x\in U\). We would like to calculate \(u(x)\) by finding some curve lying
within \(U\), connecting \(x\) with a point \(x^0\in\Gamma\) and along
which we can compute \(u\). Since \eqref{eq:initial-cond-1} says \(u=g\) on
\(\Gamma\), we know the value of \(u\) at the one end \(x^0\). We hope then
to be able to calculate \(u\) all along the curve, and so in particular at
\(x\).
\subsubsection{Finding the characteristic ODEs}
How can we choose the curve so all this will work? Let us suppose it is
described parametrically by the function
\(\bfx(s)=\bigl(x^1(s),\dotsc,x^n(s)\bigr)\), the parameter \(s\) lying in
some subinterval of \(\bfR\). Assuming \(u\) is a \(C^2\) solution of
\eqref{eq:pde:pde-1}, we define also
\[
  z(s)\defeq u\bigl(\bfx(s)\bigr).
\]
In addition, set
\[
  \bfp(s)\defeq Du(\bfx(s));
\]
that is, \(\bfp(s)=\bigl(p^1(s),\dotsc,p^n(s)\bigr)\), where
\begin{equation}
  \label{eq:pde:char-curve-p-i}
  p^i(s)=u_{x_i}\bigl(\bfx(s)\bigr),
\end{equation}
\(1\leq i\leq n\). So \(z\) gives the values of \(u\) along the curve and
\(\bfp\) records the values of the gradient \(Du\). We must choose a
function \(\bfx\) in such a way that we can compute \(z\) and \(\bfp\).

For this, first differentiate \eqref{eq:pde:char-curve-p-i}
\[
  \dot{p}^i(s)=\sum_{j=1}^n u_{x_ix_j}\bigl(\bfx(s)\bigr)\dot{x}^j(s)
\]
This expression is not too promising, since it involves the second
derivatives of \(u\). On the other hand, we can also differentiate the PDE
\eqref{eq:pde:pde-1} with respect to \(x_i\) to get
\[
  \sum_{j=1}^n
  \frac{\partial}{\partial p_j}F(Du,u,x)u_{x_jx_i}
  +\frac{\partial}{\partial z}F(Du,u,x)u_{x_i}
  +\frac{\partial}{\partial x_i}F(Du,u,x)=0.
\]
We are able to employ this identity to get rid of the \emph{dangerous}
second derivative terms provided we first set
\[
  \dot{x}^j(s)=\frac{\partial}{\partial
    p_j}F\bigl(\bfp(s),z(s),\bfx(s)\bigr).
\]

%%% Local Variables:
%%% mode: latex
%%% TeX-master: "../Fall16-Notes"
%%% End:
