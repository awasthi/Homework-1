\chapter{Introduction to Partial Differential Equations}
Here we summarize some important points about PDEs. The material is mostly
taken from Evans's \emph{Partial Differential Equations} \cite{evans} with
occasional detours to Strauss's \emph{Partial Differential Equations: An
  Introduction} \cite{strauss}. We will be following Dr.\@ Petrosyan's
\textsf{Course Log} which can be found here
\url{https://www.math.purdue.edu/~arshak/F16/MA523/courselog/}, i.e.,
summarizing the appropriate chapters from \cite{evans}.

\section{First-Order PDEs}
\subsection{The transport equation}
In this section, we consider the simplest first-order PDE, the
\emph{transport equation} with constant coefficients, i.e., the PDE
\begin{equation}
  \label{eq:pde:transport-equation}
  u_t+b\cdot Du=0\qquad\text{in \(\bbR\times(0,\infty)\),}
\end{equation}
where \(b\) is a fixed vector in \(\bbR^n\), and
\(u\colon\bbR^n\times[0,\infty)\to\bbR\) is the solution to the PDE. Our
task is to find solutions \(u\) which satisfy the equation
\eqref{eq:pde:transport-equation}.

To address this task, let us suppose for a moment that we have a (smooth)
solution \(u\) and try to compute it using the PDE
\eqref{eq:pde:transport-equation}. First, note that
\eqref{eq:pde:transport-equation} asserts that the directional derivative
\(D_{(b,1)}u=0\). Fix a point \((x,t)\in\bbR^n\times(0,\infty)\) and define
\[
  z(s)\defeq u(x+sb,t+s)
\]
for \(s\in\bbR\). Then
\[
  \dot z(s)=Du(x+sb,t+s)\cdot b+u_t(x+sb,t+s)=0.
\]
Thus, \(z\) is a constant function of \(s\) and, consequently for each
\((x,t)\), \(u\) is constant on the line through \((x,t)\) with direction
\((b,1)\in\bbR^{n+1}\). Hence if we know the value of \(u\) at any point on
each such line, we know its value everywhere in \(\bbR^n\times(0,\infty)\).

\subsubsection{Initial-value problem}
Let's now look at the transport equation with \emph{initial conditions}
\begin{equation}
  \label{eq:pde:transport-equation-init}
  \left\{
    \begin{aligned}
      u_t+b\cdot Du&=0&&\text{in \(\bbR^n\times(0,\infty)\),}\\
      u&=g&&\text{on \(\bbR^n\times\{\,t=0\,\}\).}
    \end{aligned}
  \right.
\end{equation}
Here \(b\in\bbR^n\) and \(g\colon\bbR^n\to\bbR\) are known, and \(u\) is
the unknown. Given \((x,t)\), the line through \((x,t)\) with direction
\((b,1)\) is represented parametrically by \((x+sb,t+s)\) for
\(s\in\bbR\). This line hits the plane
\(\Gamma\defeq\bbR^n\times\{\,t=0\,\}\) when \(s=-t\), at the point
\((x-tb,0)\). Since \(u\) is constant on the line and
\(u(x-tb,0)=g(x-tb)\), we deduce
\begin{equation}
  \label{eq:pde:transport-equation-init-sol}
  u(x,t)=g(x-tb)
\end{equation}
for \(x\in\bbR^n\), \(t\geq 0\). So if
\eqref{eq:pde:transport-equation-init} has a sufficiently regular solution
\(u\) (at least \(C^1\)), it must certainly be given by
\eqref{eq:pde:transport-equation-init-sol}.

\section{Characteristics}
We now turn our attention to a very important method for solving
first-order PDEs, the method of \emph{characteristics}.
\subsection{Derivation of characteristic ODEs}
Consider the first-order (possibly non-linear) PDE
\begin{equation}
  \label{eq:pde:first-order}
  \begin{aligned}
    F(Du,u,x)&=0&&\text{in \(U\),}
  \end{aligned}
\end{equation}
subject to the \emph{boundary condition}
\begin{equation}
  \label{eq:pde:boundary-condition}
  \begin{aligned}
    u&=g&&\text{on \(\Gamma\),}
  \end{aligned}
\end{equation}
where \(\Gamma\subset\partial U\) and \(g\colon\Gamma\to\bbR\) are
known. We shall assume, for simplicity, that \(F\) and \(g\) are smooth.

We now develop the \emph{method of characteristics} to solve
\eqref{eq:pde:first-order}, \eqref{eq:pde:boundary-condition} by converting
the PDE into a system of ODEs. We proceed as follows: Suppose \(u\) solves
\eqref{eq:pde:first-order}, \eqref{eq:pde:boundary-condition} and fix a
point \(x\in U\). We would like to calculate \(u(x)\) by finding some curve
lying within \(U\), connecting \(x\) with a point \(x^0\in\Gamma\) and
along which we can compute \(u\).

%%% Local Variables:
%%% mode: latex
%%% TeX-master: "../Fall16-Notes"
%%% End:
