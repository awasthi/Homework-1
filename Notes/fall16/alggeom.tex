\chapter{Algebraic Geometry}
A summary to a course on an introduction to sheaf cohomology. We will
mostly reference Donu's notes available here
\url{https://www.math.purdue.edu/~dvb/classroom.html}, but also cite Ravi
Vakil's \emph{Fundamentals of Algebraic Geometry} \cite{vakil} available
here \url{https://math216.wordpress.com/}.

\section{The statement of de Rham's theorem}
These are almost verbatim Arapura's notes on the de Rham Complex and
cohomology.

Before doing anything fancy, let's start at the beginning. Let
\(U\subseteq\bbR^3\) be an open set. In calculus class, we learn about
operations
\[
  \{\,\text{functions}\,\}
  \xrightarrow{\,\nabla\,}
  \{\,\text{vector fields}\,\}
  \xrightarrow{\,\nabla\times\,}
  \{\,\text{vector fields}\,\}
  \xrightarrow{\,\nabla\cdot\,}
  \{\,\text{functions}\,\}
\]
such that \((\nabla\times)(\nabla)=0\) and
\((\nabla\cdot)(\nabla\times)=0\). This is a prototype for a
\emph{complex.} An obvious question: does \(\nabla\times v=0\) imply that
\(v\) is a gradient? Answer: sometimes yes (e.g.\@ if \(U=\bbR^3\)) and
sometimes no (e.g.\@ if \(U=\bbR^3\) minus a line). To quantify the failure
we introduce the first de Rham cohomology
\[
  H_{\dR}^1(U)=\frac{\bigl\{\,\text{\(v\) a vector field on
      \(U\)}:\nabla\times v=0\,\bigr\}}{\bigl\{\,\nabla f\,\bigr\}}.
\]
Contrary to first appearances, for reasonable \(U\) this is finite
dimensional and computable. This follows from the de Rham's theorem, which
we now explain. First, let's generalize this to an open set
\(U\subset\bbR^n\). Once \(n>3\) vector calculus is useless, but there is a
good replacement. A differential form of degree \(p\), or \(p\)-form, is an
expression
\[
  \alpha=\sum f_{i_1,\dotsc,i_p}(x_1,\dotsc,x_n)\diff
  x_{i_1}\wedge\dotsb\wedge dx_{i_p}
\]
such that the \(x_i\) are coordinates, the \(f\) are \(C^\infty\)
functions, \(dx_{i_1}\wedge\dotsb\wedge dx_{i_p}\) are symbols where
\(\wedge\) is an anticommutative product. Let \(\calE^p(U)\) denote the
vector space of \(p\)-forms. Define the exterior derivative by
\[
  d\alpha=\ssum_j\frac{\partial f_{i_1,\dotsc,i_p}}{\partial
    x_j}\diff x_j\wedge\dotsb\wedge dx_{i_p}.
\]
This is a \((p+1)\)-form.

\begin{lemma}
  \(d^2=0\).
\end{lemma}
\begin{proof}
  We prove it for \(p=0\). In this case, we have
  \begin{align*}
    df&=\sum_i\frac{\partial f}{\partial x_i}\diff x_i\\
    d(df)&=\ssum_{i,j}\frac{\partial^2}{\partial x_j\partial x_i}\diff
           x_j\wedge dx_i.
  \end{align*}
  Using anticommutativity, we can rewrite this as
  \[
    \sum_{j<i}\left(\frac{\partial^2 f}{\partial x_j\partial x_i}
    -\frac{\partial^2 f}{\partial x_i\partial x_j}\right) dx_j\wedge dx_i=0.
  \]
\end{proof}

A cochain complex is a collection of Abelian groups \(M^i\) and
homomorphisms \(d\colon M^i\to M^{i+1}\) such that \(d^2=0\). We define the
\(p\)th cohomology of this by
\[
  H^p(M^\bullet,d)=\frac{\Ker d\colon M^p\to M^{p+1}}{\Img d\colon
    M^{p-1}\to M^p}.
\]
So we have an example of a complex \((\calE^\bullet(U),d)\) called the de
Rham complex of \(U\). It's cohomology is the de Rham cohomology
\(H_{\dR}^p(U)=H^p\bigl(\calE^\bullet(U),d\bigr)\). Here is a basic
computation.

\begin{theorem}[Poincaré's lemma]
  \[
    H_{\dR}^p(\bbR^n)=
    \begin{cases}
      \bbR&\text{if \(p=0\),}\\
      0&\text{otherwise}.
    \end{cases}
  \]
\end{theorem}
\begin{proof}
  We show this for \(n\leq 2\). We first treat the case \(n=1\). Clearly
  \(H_\dR^p(\bbR)\) consists of constant functions. If \(\alpha=f(x)\diff
  x\), then
  \[
    d\left(\int_0^x f(t)\diff t\right)=\alpha.
  \]
  There are no \(p\)-forms for \(p>1\).

  Next, we treat \(n=2\) which contains all of the ideas
\end{proof}

%%% Local Variables:
%%% mode: latex
%%% TeX-master: "../Fall16-Notes"
%%% End:
