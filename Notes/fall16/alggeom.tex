\chapter{Algebraic Geometry}
A summary to a course on an introduction to sheaf cohomology. We will
mostly reference Donu's notes available here
\url{https://www.math.purdue.edu/~dvb/classroom.html}, but also cite Ravi
Vakil's \emph{Fundamentals of Algebraic Geometry} \cite{vakil} available
here \url{https://math216.wordpress.com/}.

\section{The statement of de Rham's theorem}
These are almost verbatim Arapura's notes on the de Rham Complex and
cohomology.

Before doing anything fancy, let's start at the beginning. Let
\(U\subseteq\bbR^3\) be an open set. In calculus class, we learn about
operations
\[
  \{\,\text{functions}\,\}
  \xrightarrow{\,\nabla\,}
  \{\,\text{vector fields}\,\}
  \xrightarrow{\,\nabla\times\,}
  \{\,\text{vector fields}\,\}
  \xrightarrow{\,\nabla\cdot\,}
  \{\,\text{functions}\,\}
\]
such that \((\nabla\times)(\nabla)=0\) and
\((\nabla\cdot)(\nabla\times)=0\). This is a prototype for a
\emph{complex.} An obvious question: does \(\nabla\times v=0\) imply that
\(v\) is a gradient? Answer: sometimes yes (e.g.\@ if \(U=\bbR^3\)) and
sometimes no (e.g.\@ if \(U=\bbR^3\) minus a line).

%%% Local Variables:
%%% mode: latex
%%% TeX-master: "../Fall16-Notes"
%%% End:
