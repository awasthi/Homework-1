\chapter{Probability}
Some (mostly discrete) probability theory for MA 51900.

\section{Basics}
In this section we will talk about concepts related to discrete
probability. Before we begin, we have to define the concepts we will be
working with throughout the rest of this section (ye this chapter). First
and foremost, to do probability we need a \emph{sample space \(\Omega\)}
and a probability function \(p\colon\calP(\Omega)\to[0,1]\) which assigns
values between \(0\) and \(1\) to subsets of \(\Omega\) (usually, one needs
to specify a \(\sigma\)-algebra on \(\Omega\), but for the rest of the
section, since \(\card\Omega<\infty\), we need only consider the power set
of \(\Omega\), \(\calP(\Omega)\) since its cardinality is also finite).


A large part of discrete probability theory comes down to combinatorics.

%%% Local Variables:
%%% mode: latex
%%% TeX-master: "../Fall16-Notes"
%%% End:
