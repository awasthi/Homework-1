\chapter{Probability}
We will devote this chapter to the material that is covered in MA 51900
(discrete probability) as it was covered in DasGupta's class. We will, for
the most part, reference Feller's \emph{An introduction to probability
  theory and its applications, Volume 1} \cite{feller} (especially for the
discrete noncalculus portion of the class) and DasGupta's own book
\emph{Fundamentals of Probability: A First Course} \cite{dasgupta}.

\section{Discrete Probability}
The material in this chapter is mostly pulled from Sheldon Ross's \emph{A
  First Course in Probability Theory} with some examples from
\cite{dasgupta} and \cite{feller}. I find Ross's book to be better
structured than the latter two.

\subsection{Combinatorial Analysis}
These are the main results from this section.
\begin{theorem}[The basic principle of counting]
  Suppose that two experiments are to be performed. Then if experiment 1
  can result in any one of \(m\) possible outcomes and if, for each outcome
  of experiment 1, there are \(n\) possible outcomes of experiment 2, then
  together there are \(mn\) possible outcomes of the two experiments.
\end{theorem}
\begin{theorem}[The generalized principle of counting]
  If \(r\) experiments that are to be performed are such that the first one
  may result in any of \(n_1\) possible outcomes; and if, for each of these
  \(n_1\) possible outcomes, there are \(n_2\) possible outcomes for the
  second experiment; and if, for each of the possible outcomes of the first
  two experiments, there are \(n_3\) possible outcomes for the third
  experiment; etc.\@ ..., then there is a total of \(n_1n_2\dotsm n_r\)
  possible outcomes of the \(r\) experiments.
\end{theorem}

Using notation as in \cite{feller}, the number
\[
  (n)_r=n(n-1)\dotsm(n-r+1)
\]
represents the number of different ways that a group of \(r\) items could
be selected from \(n\) items when the order of selection is relevant, and
as each group of \(r\) items will be counted \(r!\) times in this count, it
follows that the number of different groups of \(r\) items that could be
formed from a set of \(n\) items is
\[
  \frac{(n)_r}{r!}=\frac{n!}{(n-r)!r!}
\]
for which we reserve the notation
\[
  \binom{n}{r}
\]
read \(n\) choose \(r\). (This is called a binomial coefficient since it
appears in the binomial expansion \((a+b)^n\).)

A useful combinatorial identity on binomial coefficients is the following
\[
  \binom{n}{r}=\binom{n-1}{r-1}+\binom{n-1}{r}
\]
for \(1\leq r\leq n\).

\begin{theorem}[The binomial theorem]
  \[
    (a+b)^n=\sum_{i=1}^n\binom{n}{i}x^iy^{n-i}.
  \]
\end{theorem}
\begin{proof}
  We provide a combinatiorial proof of the theorem. Consider the product
  \[
    (a_1+b_1)\dotsm(a_n+b_n).
  \]
  Its expansion consists of the sum of \(2^n\) terms, each term being the
  product of \(n\) factors. Furthermore, each of the \(2^n\) terms in the
  sum will contain as a factor either \(a_i\) or \(b_i\) for each
  \(1\leq i\leq n\). Now, how many of the \(2^n\) terms in the sum will
  have \(k\) of the \(a_i\) and \(n-k\) of the \(b_i\) as factors? As each
  term consisting of \(k\) of the \(a_i\) and \(n-k\) of the \(b_i\)
  correspond to a choice of a group of \(k\) from the values
  \(a_1,\dotsc,a_n\), there are \(\binom{n}{k}\) such terms. Thus, letting
  \(a_i=a\), \(b_i=b\), \(1\leq i\leq n\), we see that
  \[
    (a+b)^n=\sum_{i=0}^n\binom{n}{i}x^iy^{n-i}.
  \]
\end{proof}

%%% Local Variables:
%%% mode: latex
%%% TeX-master: "../Fall16-Notes"
%%% End:
