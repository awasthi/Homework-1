\chapter{Probability}
We will devote this chapter to the material that is covered in MA 51900
(discrete probability) as it was covered in DasGupta's class. We will, for
the most part, reference Feller's \emph{An introduction to probability
  theory and its applications, Volume 1} \cite{feller} (especially for the
discrete noncalculus portion of the class) and DasGupta's own book
\emph{Fundamentals of Probability: A First Course} \cite{dasgupta}.

\section{Discrete Probability}
The material in this section is pulled almost entirely from \cite{feller}
with minor detours to \cite{dasgupta}. We will not reference any particular
pages in either book (unless we feel particularly lazy).

\subsection{Background}
Given a discrete sample space \(\Omega\) with sample points
\(\omega_1,\omega_2,\dotsc\), we shall assume that with each point
\(\omega_j\) there is associated a number, called the probability of
\(\omega_j\) and denoted by \(P(\omega_i)\). It is nonnegative and such
that
\begin{equation}
  \label{eq:prob-sum}
  \sum_{i\in\bfN} P(\omega_i)=1.
\end{equation}

\begin{definition}
  The probability \(P(A)\) of an event \(A\) is the sum of the
  probabilities of all sample points in it.
\end{definition}

Since the probability of \(\Omega\) is \(1\) by \eqref{eq:prob-sum}, it
follows that for any event \(A\)
\begin{equation}
  \label{eq:prob-event}
  0\leq P(A)\leq 1.
\end{equation}

Let \(A_1\) and \(A_2\) be arbitrary events. To compute the probability
\(P(A_1\cup A_2)\) that either \(A_1\) or \(A_2\) or both occur, we have to
add the probabilities of the sample points contained either in \(A_1\) or
in \(A_2\), but each point is to be counted only once. Therefore, we have
\begin{equation}
  \label{eq:prob-union-bound}
  P(A_1\cup A_2)\leq P(A_1)+P(A_2).
\end{equation}
Now, if \(\omega\) is any point contained in both \(A_1\) and \(A_2\) the
probability of \(\omega\), \(P(\omega)\), appears on the right-hand side of
\eqref{ep:prob-union-bound} twice but only once in the left-hand side. This
analysis leads us to conclude that the probability \(P(A_1\cap A_2)\)
occurs twice on right-hand side of \eqref{eq:prob-union-bound}, and we have
the important result
\begin{theorem}
  For any two events \(A_1\) and \(A_2\) the probability that either
  \(A_1\) or \(A_2\) or both occur is given by
  \begin{equation}
    \label{eq:prob-union-exact}
    P(A_1\cup A_2)=P(A_1)+P(A_2)-P(A_1\cap A_2).
  \end{equation}
  If \(A_1\cap A_2=\emptyset\), that is, if \(A_1\) and \(A_2\) are
  mutually exclusive, then \eqref{eq:prob-union-exact} reduces to
  \[
    P(A_1\cup A_2)=P(A_1)+P(A_2).
  \]
\end{theorem}

We may similarly continue to consider the probability of (countably)
arbitrarily many events \(A_1,A_2,\dotsc\),
\begin{equation}
  \label{eq:booles-inequality}
  P\left(\bigcup\nolimits_{i\in\bfN} A_i\right)\leq\sum_{i\in\bfN} P(A_i).
\end{equation}
This equation is referred to as \emph{Boole's inequality}. In the special
case where the events \(A_1,A_2,\dotsc\) are mutually exclusive, we have
\[
  P\left( \bigcup\nolimits_{i\in\bfN} A_i\right)=\sum_{i\in\bfN} P(A_i).
\]

%%% Local Variables:
%%% mode: latex
%%% TeX-master: "../Fall16-Notes"
%%% End:
