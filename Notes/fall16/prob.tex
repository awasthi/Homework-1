\chapter{Probability}
We will devote this chapter to the material that is covered in MA 51900
(discrete probability) as it was covered in DasGupta's class. We will, for
the most part, reference Feller's \emph{An introduction to probability
  theory and its applications, Volume 1} \cite{feller} (especially for the
discrete noncalculus portion of the class) and DasGupta's own book
\emph{Fundamentals of Probability: A First Course} \cite{dasgupta}.

\section{Discrete Probability}
The material in this section is pulled almost entirely from \cite{feller}
with minor detours to \cite{dasgupta}. We will not reference any particular
pages in either book (unless we feel particularly lazy).

\subsection{Background}
We begin with axioms and terminology. We shall call the result of an
experiment (which shall not be defined) as an \emph{event}. Thus we shall
speak of the event that of five coins tossed more than three fell heads. We
also make a distinction between \emph{compound} (or decomposable) and
\emph{simple} (or indecomposable) \emph{events}. E.g., saying that two
tosses of a coin resulted in one head and one tail amounts to saying that
it resulted in \((H,T)\) or \((T,H)\) and this enumeration decomposes the
event ``two tosses of a coin resulted in one head and one tail'' into two
indecomposable events.

Now comes the rigor. If we want to speak about the outcome of experiments
in a theoretical way with no ambiguity, we must first agree on the simple
events representing the thinkable outcomes; \emph{they define the idealized
experiment}.

%%% Local Variables:
%%% mode: latex
%%% TeX-master: "../Fall16-Notes"
%%% End:
