\chapter{Probability}
Some (mostly discrete) probability theory for MA 51900.

\section{Basics}
In this section we will talk about concepts related to discrete
probability. Before we begin, we have to define the concepts we will be
working with throughout the rest of this section (ye this chapter). First
and foremost, to do probability we need a \emph{sample space \(\Omega\)}
and a probability function \(p\colon\calP(\Omega)\to[0,1]\) which assigns
values between \(0\) and \(1\) to subsets of \(\Omega\) (usually, one needs
to specify a \(\sigma\)-algebra on \(\Omega\), but for the rest of the
section, since \(\Omega\) is at least countable, the power set
\(\calP(\Omega)\) is a \(\sigma\)-algebra by default). A \emph{sample
  point} \(\omega\) is a point of \(\Omega\) and an event
\(A\subseteq\calP(\Omega)\) is a collection of sample points.

In this section we will talk about concepts related to discrete
probability. Before we begin, we introduce the objects we will be working
with. First and foremost, to do probability we need a \emph{sample space
  \(\Omega\)} and a \emph{probability \(p\colon\calM\to[0,1]\)} which
assigns values between \(0\) and \(1\) to \emph{special} subsets of
\(\Omega\) which we denote by \(\calM\) (more formally, this \(\calM\) is
called a
\href{https://en.wikipedia.org/wiki/Sigma-algebra}{\emph{\(\sigma\)-algebra}}
and \(p\) is called a
\href{https://en.wikipedia.org/wiki/Probability_measure}{\emph{probability
    measure}} and there are certain axioms it must satisfy for us to be
able to assign consistent values to subsets of \(\Omega\) with \(p\)). An
element \(\omega\in\Omega\) is called a \emph{sample point} and a (special)
collection of \(\omega\), \(A\in\calM\), is called an event. We call the
triplet \((\Omega,\calM,p)\) a probability space.

With that out of the way, let us get down to the crux of the matter (at
least at this point in the class): counting. Since our sample spaces will
be finite (at least for now), we need to be able to count sample points in
\(\Omega\) by way of combinatorics (this is in my opinion, a lot tougher
than working with infinite sample spaces for which we must make certain
assumptions about the sample points and the probability measure -- it is
less tedious to solve problems with sane assumptions than it is to count
points).

%%% Local Variables:
%%% mode: latex
%%% TeX-master: "../Fall16-Notes"
%%% End:
