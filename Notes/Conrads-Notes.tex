\def\documentauthor{Carlos Salinas}
\def\documenttitle{Mathematics Notes}
% \def\hwnum{1}
\def\shorttitle{Math-Notes}
\def\coursename{mathematics}
\def\documentsubject{abstract algebra, real analysis, complex analysis,
  topology, hyperbolic geometry}
\def\authoremail{salinac@purdue.edu}

\documentclass[article,oneside,10pt]{memoir}
\usepackage{geometry}
\usepackage[dvipsnames]{xcolor}
\usepackage[
    breaklinks,
    bookmarks=true,
    colorlinks=true,
    pageanchor=false,
    linkcolor=black,
    anchorcolor=black,
    citecolor=black,
    filecolor=black,
    menucolor=black,
    runcolor=black,
    urlcolor=black,
    hyperindex=false,
    hyperfootnotes=true,
    pdftitle={\shorttitle},
    pdfauthor={\documentauthor},
    pdfkeywords={\documentsubject},
    pdfsubject={\coursename}
    ]{hyperref}

% Use symbols instead of numbers
\renewcommand*{\thefootnote}{\fnsymbol{footnote}}

%% Math
\usepackage{amsthm}
\usepackage{amssymb}
\usepackage{mathtools}
% \usepackage{unicode-math}

%% PDFTeX specific
\usepackage[mathcal]{euscript}
\usepackage{mathrsfs}

\usepackage[LAE,LFE,T2A,T1]{fontenc}
\usepackage[utf8]{inputenc}
\usepackage[farsi,french,german,spanish,russian,english]{babel}
\babeltags{fr=french,
           de=german,
           en=english,
           es=spanish,
           pa=farsi,
           ru=russian
           }
\def\spanishoptions{mexico}

\selectlanguage{english}

\newcommand{\textfa}[1]{\beginR\textpa{#1}\endR}

\usepackage{cmap}
\usepackage{CJKutf8}
\newcommand{\textkr}[1]{\begin{CJK}{UTF8}{mj}#1\end{CJK}}
\newcommand{\textjp}[1]{\begin{CJK}{UTF8}{min}#1\end{CJK}}
\newcommand{\textzh}[1]{\begin{CJK}{UTF8}{bsmi}#1\end{CJK}}

\usepackage{graphicx}
\graphicspath{{figures/}}

% Misc
\usepackage{microtype}
\usepackage{multicol}
\usepackage[inline]{enumitem}
\usepackage{listings}
\usepackage{mleftright}
\mleftright

%% Theorems and definitions
%% remove parentheses
% \makeatletter
% \def\thmhead@plain#1#2#3{%
%   \thmname{#1}\thmnumber{\@ifnotempty{#1}{ }\@upn{#2}}%
%   \thmnote{ {\the\thm@notefont#3}}}
% \let\thmhead\thmhead@plain
% \makeatother

\theoremstyle{plain}
\newtheorem{theorem}{Theorem}
\newtheorem{proposition}[theorem]{Proposition}
\newtheorem{corollary}[theorem]{Corollary}
\newtheorem{claim}[theorem]{Claim}
\newtheorem{lemma}[theorem]{Lemma}
\newtheorem{axiom}[theorem]{Axiom}

\newtheorem*{corollary*}{Corollary}
\newtheorem*{claim*}{Claim}
\newtheorem*{lemma*}{Lemma}
\newtheorem*{proposition*}{Proposition}
\newtheorem*{theorem*}{Theorem}

\theoremstyle{definition}
\newtheorem{definition}{Definition}
\newtheorem{example}{Examples}
\newtheorem{examples}[example]{Examples}
\newtheorem{exercise}{Exercise}[chapter]
\newtheorem{problem}[exercise]{Problem}

\newtheorem*{example*}{Example}
\newtheorem*{exercise*}{Exercise}
\newtheorem*{problem*}{Problem}

%% Redefinitions & commands
\newcommand{\nsubset}{\ensuremath{\not\subset}}
\newcommand{\nsupset}{\ensuremath{\not\supset}}
\newcommand\minus{\ensuremath{\null\smallsetminus}}
\renewcommand\qedsymbol{\ensuremath{\null\hfill\blacksquare}}

%% Commands and operators
\DeclareMathOperator{\id}{id}
\DeclareMathOperator{\im}{im}
\DeclareMathOperator{\lcm}{lcm}

\DeclareMathOperator{\Aut}{Aut}
\DeclareMathOperator{\Gal}{Gal}
\DeclareMathOperator{\Rad}{rad}
\DeclareMathOperator{\Nil}{nil}


%% Symbols
\newcommand{\bbC}{\mathbb{C}}
\newcommand{\bbCP}{\mathbb{C}\mathrm{P}}
\newcommand{\bbH}{\mathbb{H}}
\newcommand{\bbN}{\mathbb{N}}
\newcommand{\bbQ}{\mathbb{Q}}
\newcommand{\bbR}{\mathbb{R}}
\newcommand{\bbRP}{\mathbb{R}\mathrm{P}}
\newcommand{\bbZ}{\mathbb{Z}}

\newcommand{\bfC}{\mathbf{C}}
\newcommand{\bfCP}{\mathbf{C}\mathrm{P}}
\newcommand{\bfH}{\mathbf{H}}
\newcommand{\bfN}{\mathbf{N}}
\newcommand{\bfQ}{\mathbf{Q}}
\newcommand{\bfR}{\mathbf{R}}
\newcommand{\bfRP}{\mathbf{R}\mathrm{P}}
\newcommand{\bfZ}{\mathbf{Z}}

\newcommand{\bfu}{\mathbf{u}}
\newcommand{\bfv}{\mathbf{v}}
\newcommand{\bfw}{\mathbf{w}}
\newcommand{\bfx}{\mathbf{x}}
\newcommand{\bfy}{\mathbf{y}}
\newcommand{\bfz}{\mathbf{z}}

\newcommand{\calA}{\mathcal{A}}
\newcommand{\calB}{\mathcal{B}}
\newcommand{\calC}{\mathcal{C}}
\newcommand{\calS}{\mathcal{S}}
\newcommand{\calT}{\mathcal{T}}
\newcommand{\calU}{\mathcal{U}}
\newcommand{\calV}{\mathcal{V}}

\newcommand{\scrL}{\mathscr{L}}
\newcommand{\scrO}{\mathscr{O}}
\newcommand{\scrS}{\mathscr{S}}

\begin{document}
\author{\href{mailto:\authoremail}{\documentauthor}}
\title{\documenttitle}
\date{\today}
\maketitle
\chapter{Group Theory}
\section{The Sign of a Permutation}
Summary of Keith Conrad's blurb by the same name.

Throughout this discussion $n\geq 2$. A cycle in $S_n$ is the (non-unique)
product of transpositions: the identity $(1)$, $(1\;2)(1\;2)$, and a
$k$-cycle with $k\geq 2$ can be written as
\[
(i_1\;i_2\;\cdots\;i_k)=(i_1\;i_2)(i_2\;i_3)\cdots(i_{k-1}\;i_k).
\]
For example a $3$-cycle $(a\;b\;c)$ -- which means $a$, $b$ and $c$ are
distict -- can be written as
\[
(a\;b\;c)=(a\;b)(b\;c).
\]
This is not the only way to write $(a\;b\;c)$ using transpositions, e.g.,
$(a\;b\;c)=(b\;c)(a\;c)=(a\;c)(a\;b)$.

Since any permutation in $S_n$ is the product of cycles and any cycle is a
product of transpositions, any permutation in $S_n$ is a product of
transpositions. Unlike the unique decomposition of a permutation into
products of disjoint cycles, the decomposition of a permutation is

%%% Local Variables:
%%% TeX-master: "../Conrads-Notes"
%%% End:


% \include{rings/}

% \include{fields/}

\chapter{Real Analysis}

%%% Local Variables:
%%% mode: latex
%%% TeX-master: "../Basic-Math"
%%% End:

\chapter{McMullen's Complex Analysis Notes}
\section{Basic Complex Analysis}
\subsection{Some Notation}
The complex numbers will be denoted $\bbC$. We let $\Delta$, $\bbH$, and
$\widehat\bbC$ denote the unit disk $|z|<1$, the upper half-plane $\Im z>1$
and the Riemann sphere $\bbC\cup\{\infty\}$. We write $S^1(r)$ for the
circle $|z|=r$ and $S^1$ for the unit circle, each oriented
counter-clockwise. We also set $\Delta^*=\Delta\minus\{0\}$ and
$\bbC^*=\bbC\minus\{0\}$.
\subsection{Algebra and complex numbers}
The complex numbers are formally defined as the field $\bbC=\bbR[i]$, where
$i^2=-1$. They are represented in the Euclidean plane by
$z=(x,y)=x+iy$. There are two square-roots of $-1$ in $\bbC$; the number
$i$ is the one with positive imaginary part.

An important role is played by the Galois involution $z\mapsto\bar z$. We
define $|z|^2=N(z)=z\bar z=x^2+y^2$. Compatibility of $|z|$ with the
Euclidean metric justifies the identification of $\bbC$ and $\bbR^2$. We
also see that $\bbC$ is a field as $1/z=\bar z/|z|$.

It is also convenient to describe complex numbers by polar coordinates
\[
z=[r,\theta]=r(\cos\theta+i\sin\theta).
\]
Here $r=|z|$ and $\theta=\arg z$ in $\bbR/2\pi\bbZ$. (The multivaluedness
of $\arg z$ requires care but is also the  source of powerful results like
Cauchy's integral formula.) We then have
\[
\left[r_1,\theta_1\right]\left[r_2,\theta_2\right]=
\left[r_1r_2,\theta_1+\theta_2\right].
\]
In particular, the linear maps $f(z)=az+b$, $a\neq 0$, of $\bbC$ to itself,
preserve angles and orientations.

This formula should be proved geometrically; in fact, it is a consequence
of the formula $|ab|=|a||b|$ and similar triangles. It can then be used to
derive the addition formulas for sine and cosine.

\subsection{Algebraic closure}
A critical feature of the complex numbers is that they are
\emph{algebraically closed}, i.e., every polynomial has a root.

Classically, the complex numbers were introduced in the course of solving
real cubic equations. Starting with $x^3+ax+b=0$ one can make a Tschirnhaus
transformation so $a=0$. This is done by introducing a new variable,
$y=cx^2+d$ such that $\sum y_i=\sum {y_i}^2=0$; even when $a$ and $b$ are
real, it may be necessary to choose $c$ complex (the discriminant of the
equation for $c$ is $27b^2+4a^3$). It is negative when the cubic has only
one real root; this can be checked by looking at the product of of the
values of the cubic at its $\min$ and $\max$.

\subsection{Polynomials and rational functions}
Using addition and multiplication we obtain naturally the polynomial
functions $f(z)=\sum_0^n a_nz^2\colon\bbC\to\bbC$. The ring of polynomials
$\bbC[z]$ is an integral domain and a unique factorization domain, since
$\bbC$ is a field. Indeed, since $\bbC$ is algebraically closed, every
polynomial factors into linear terms.

It is useful to add the allowed value $\infty$ to obtain the Riemann sphere
$\widehat\bbC=\bbC\cup\{\infty\}$. Then the rational functions determine
rational maps $f\colon\bbC\to\bbC$. The rational functions $\bbC(z)$ are
the same as the field of fractions for the domain $\bbC[z]$. We set
$f(z)=\infty$ if $q(z)=0$; these points are called the \emph{poles} of $f$.

\subsection{Analysis functions}
Let $U$ be an open set in $\bbC$ and $f\colon U\to\bbC$ a function. We say
that $f$ is \emph{analytic} if
\[
f'(z)=\lim_{t\to\infty}\frac{f(z+)-f(z)}{t}
\]
exists for all $z\in U$. It is important that $t$ approaches $0$ through
arbitrary values of $\bbC$. Remarkably, this condition implies that $f\in
C^\infty$.

%%% Local Variables:
%%% mode: latex
%%% TeX-master: "../Conrads-Notes"
%%% End:

\chapter{McMullen's Topology Notes}

%%% Local Variables:
%%% mode: latex
%%% TeX-master: "../Conrads-Notes"
%%% End:

\end{document}

%%% Local Variables:
%%% mode: latex
%%% TeX-master: t
%%% End:
