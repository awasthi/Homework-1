\section{6.041: Probabilistic Systems Analysis}
Because the notation \(A^\rmc\) is too ugly to our eyes and we will often
be working with set complements we shall set aside the notation
\(\tilde A\) for the complement of \(A\).

\subsection{Problem-Set 1}
\begin{problem}
  Express each of the following events in terms of the events \(A\), \(B\),
  and \(C\) as well as the operations of complementation, union, and
  intersection:
  \begin{alphlist}
  \item at least one of the events \(A\), \(B\), \(C\) occurs;
  \item at most one of the events \(A\), \(B\), \(C\) occurs;
  \item none of the events \(A\), \(B\), \(C\) occurs;
  \item exactly one of the events \(A\), \(B\), \(C\) occurs;
  \item events \(A\) and \(B\) occur, but not \(C\);
  \item either event \(A\) occurs or, if not, then \(B\) also does not
    occur.
  \end{alphlist}
  In each case draw the corresponding Venn diagram.
\end{problem}
\begin{solution*}
  We present only one of the many possible expressions for (a)-(g) and we
  shall omit the finer details; suffice it to say, these are all
  consequences of elementary set theory. We also omit the Venn diagrams the
  problem is asking us to draw as it would be a bad investment of our time
  to trace them out using PGF/Ti\textit{k}Z.
  \\\\
  For part (a): the event, call it \(E\), that at least one of \(A\),
  \(B\), \(C\) occurs is the expression
  \[
    E=A\cup B\cup C.
  \]
  \\\\
  For part (b): the event \(E\) that at most one of \(A\), \(B\), \(C\)
  occurs is the expression
  \[
    E=\widetilde{\left[(A\cap B)\cup (A\cap C)\cup (B\cap C)\cup (A\cap
      B\cap C)\right]}.
  \]
  \\\\
  For part (c): the event \(E\) that none of \(A\), \(B\), \(C\) occur is
  the expression
  \[
    E=\widetilde{A\cup B\cup C}
  \]
  \\\\
  For part (d): the event \(E\) that all three events \(A\), \(B\), \(C\)
  occur is the expression
  \[
    E=A\cap B\cap C.
  \]
  \\\\
  For part (e): the event \(E\) that exactly one of the events \(A\),
  \(B\), \(C\) occurs is
  \[
    E=(A\cup B\cup C)\setminus \left[(A\cap B)\cup (A\cap C)\cup (B\cap
      C)\cup (A\cap B\cap C)\right].
  \]
  \\\\
  For part (f):
  \\\\
  For part (g):
\end{solution*}

\begin{problem}
\end{problem}
\begin{solution*}
\end{solution*}

\begin{problem}
\end{problem}
\begin{solution*}
\end{solution*}

\begin{problem}
\end{problem}
\begin{solution*}
\end{solution*}

\begin{problem}
\end{problem}
\begin{solution*}
\end{solution*}

\begin{problem}
\end{problem}
\begin{solution*}
\end{solution*}

\begin{problem}
\end{problem}
\begin{solution*}
\end{solution*}

\subsection{Problem-Set 2}
\subsection{Problem-Set 3}
\subsection{Problem-Set 4}
\subsection{Problem-Set 5}
\subsection{Problem-Set 6}
\subsection{Problem-Set 7}
\subsection{Problem-Set 8}
\subsection{Problem-Set 9}
\subsection{Problem-Set 10}
\subsection{Problem-Set 11}

%%% Local Variables:
%%% mode: latex
%%% TeX-master: "../OCW-Notes"
%%% End:
