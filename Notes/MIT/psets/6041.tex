\section{6.041: Probabilistic Systems Analysis}
Because the notation \(A^\rmc\) is too ugly to our eyes and we will often
be working with set complements we shall set aside the notation
\(\tilde A\) for the complement of \(A\).

\subsection{Problem-Set 1}
\begin{problem}
  Express each of the following events in terms of the events \(A\), \(B\),
  and \(C\) as well as the operations of complementation, union, and
  intersection:
  \begin{alphlist}
  \item at least one of the events \(A\), \(B\), \(C\) occurs;
  \item at most one of the events \(A\), \(B\), \(C\) occurs;
  \item none of the events \(A\), \(B\), \(C\) occurs;
  \item exactly one of the events \(A\), \(B\), \(C\) occurs;
  \item events \(A\) and \(B\) occur, but not \(C\);
  \item either event \(A\) occurs or, if not, then \(B\) also does not
    occur.
  \end{alphlist}
  In each case draw the corresponding Venn diagram.
\end{problem}
\begin{solution*}
  We present only one of the many possible expressions for (a)-(g) and we
  shall omit the finer details; suffice it to say, these are all
  consequences of elementary set theory. We also omit the Venn diagrams the
  problem is asking us to draw as it would be a bad investment of our time
  to trace them out using PGF/Ti\textit{k}Z.
  \\\\
  For part (a): the event, call it \(E\), that at least one of \(A\),
  \(B\), \(C\) occurs is the expression
  \[
    E=A\cup B\cup C.
  \]
  \\\\
  For part (b): the event \(E\) that at most one of \(A\), \(B\), \(C\)
  occurs is the expression
  \[
    E=\widetilde{\left[(A\cap B)\cup (A\cap C)\cup (B\cap C)\cup (A\cap
      B\cap C)\right]}.
  \]
  \\\\
  For part (c): the event \(E\) that none of \(A\), \(B\), \(C\) occur is
  the expression
  \[
    E=\widetilde{A\cup B\cup C}
  \]
  \\\\
  For part (d): the event \(E\) that all three events \(A\), \(B\), \(C\)
  occur is the expression
  \[
    E=A\cap B\cap C.
  \]
  \\\\
  For part (e): the event \(E\) that exactly one of the events \(A\),
  \(B\), \(C\) occurs is the expression
  \[
    E=(A\cup B\cup C)\setminus \left[(A\cap B)\cup (A\cap C)\cup (B\cap
      C)\cup (A\cap B\cap C)\right].
  \]
  \\\\
  For part (f): the event \(E\) that \(A\) and \(B\) occur, but not \(C\)
  is the expression
  \[
    E=(A\cup B)\cap \tilde C.
  \]
  \\\\
  For part (g): the event \(E\) that \(A\) occurs or, if not, then \(B\)
  also does not occur is the expression
  \[
    E=A\cup(C\setminus B).\qedhere
  \]
\end{solution*}

\begin{problem}
  You flip a fair coin three times, determine the probability of the below
  events. Assume all sequences are equally likely.
  \begin{alphlist}
  \item Three heads: \(\rmH\rmH\rmH\).
  \item The sequence hea, tail, head: \(\rmH\rmT\rmH\).
  \item Any sequence with two heads and one tail.
  \item Any sequence where the number of heads is greater than or eqal to
    the number of tails.
  \end{alphlist}
\end{problem}
\begin{solution*}
  For part (a): Under the equally likely hypothesis, the probability of
  getting three heads, assuming independence of each throw, is
  \(\frac{1}{8}\). (We can justify this by writing a table with all the
  possible outcomes of three tosses of a coin. There are eight of them and
  \(\rmH\rmH\rmH\) is precisely one sample point of this sample space.)
  \\\\
  For part (b): By the same reasoning as above, \(\rmH\rmT\rmH\) has a
  probability of \(\frac{1}{8}\) of occurring.
  \\\\
  For part (c): There are precisely \(3\) such sequences \(\rmH\rmH\rmT\),
  \(\rmH\rmT\rmH\) and \(\rmT\rmH\rmH\). Therefore, the probability of any
  sequence with two heads and one tail occurring is \(\frac{3}{8}\).
  \\\\
  For part (d): We see almost immediately that this event is a superset of
  the one considered in part (c) since sequences having two tails and one
  head are forbidden; therefore, we expect the probability of this event to
  be greater than or equal to \(\frac{3}{8}\). In fact, the probability is
  \(\frac{1}{2}\) which comes about froms the addition of the event
  \(\rmH\rmH\rmH\) to part (c).
\end{solution*}

\begin{problem}
  Bob has a peculiar pair of four-sided dice. When he rolls the dice, the
  probability of any particular outcome is proportional to the sum of the
  results of each die. All outcomes that result in a particular sum are
  equally likely.
  \begin{alphlist}
  \item What is the probability of the sum being even?
  \item What is the probability of Bob rolling a \(2\) and a \(3\), in any
    order?
  \end{alphlist}
\end{problem}
\begin{solution*}
  For part (a): Since this question concerns only the sum of the faces of
  the two dice and we know that the probability of rolling a certain sum is
  proportional to the probability
  \\\\
  For part (b):
\end{solution*}

\begin{problem}
  Alice and Bob each choose at random a number in the interval
  \([0,2]\). We assume a uniform probability law under which the
  probability of an event is proportional to its area. Consider the
  following events
  \begin{align*}
    A&=\bigl\{\,\text{the magnitude of the difference of the events is
       greater than \(\tfrac{1}{3}\)}\,\bigr\},\\
    B&=\bigl\{\,\text{at least one of the numbers is greater that
       \(\tfrac{1}{3}\)}\,\bigr\},\\
    C&=\bigl\{\,\text{the two numbers are equal}\,\bigr\},\\
    D&=\bigl\{\,\text{Alice's number is greater that \(\tfrac{1}{3}\)}\,\bigr\}.
  \end{align*}
  Find the probabilities \(P(B)\), \(P(C)\), and \(P(A\cap D)\).
\end{problem}
\begin{solution*}
\end{solution*}

\begin{problem}
  Mike and John are playing a friendly game of darts where the dart board
  is a disk with radius \(10\) inches.

  Whenever a dart falls within \(1\) inch of the center, \(50\) points are
  scored. If the point of impact is between \(1\) and \(3\) inches from the
  center, \(30\) points are scored, if it is at a distance of \(3\) to
  \(5\) inches, \(20\) points are scored and if it is further than \(5\)
  inches, \(10\) points are scored.

  Assume that both players are skilled enough to be able to throw the dart
  within the boundaries of the board.

  Mike can place the dart uniformly on the board (i.e., the probability of
  the dart falling in a given region is proportional to its area).
  \begin{alphlist}
  \item What is the probability that Mike scores \(50\) points on one
    throw?
  \item What is the probability of him scoring \(30\) points on one throw?
  \item John is right handed and twice more likely to throw in the right
    half of the board than it the left half. Across each half, the dart
    falls uniformly in that region. Answer the previous questions for
    John's throw.
  \end{alphlist}
\end{problem}
\begin{solution*}
\end{solution*}

\begin{problem}
  Prove that for three events \(A\), \(B\), and \(C\), we have
  \[
    P(A\cap B\cap C)\geq P(A)+P(B)+P(C)-2.
  \]
\end{problem}
\begin{solution*}
\end{solution*}

\begin{problem}
  Consider an experiment whose sample space is the real line.
  \begin{alphlist}
  \item Let \(\{a_n\}\) be an increasing sequence of numbers that converges
    to \(a\) and \(\{b_n\}\) a decreasing sequence of numbers that
    converges to \(b\). Show that
    \[
      \lim_{n\to\infty} P([a_n,b_n])=P([a,b]).
    \]
    Here, the notation \([a,b]\) stands for the closed interval \(\{x:a\leq
    x\leq b\}\).

    \noindent\emph{Note:} This result seems intuitively obvious. The issue is to
    derive it using he axioms of probability theory.
  \item Let \(\{a_n\}\) be a decreasing sequence that converges to \(a\)
    and \(\{b_n\}\) an increasing sequence that converges to \(b\). Is it
    true that
    \[
      \lim_{n\to\infty} P([a_n,b_n])=P([a,b])?
    \]

    \noindent\emph{Note:} You may use freely the results from the problems
    in the text in your proofs.
  \end{alphlist}
\end{problem}
\begin{solution*}
\end{solution*}

\subsection{Problem-Set 2}
\subsection{Problem-Set 3}
\subsection{Problem-Set 4}
\subsection{Problem-Set 5}
\subsection{Problem-Set 6}
\subsection{Problem-Set 7}
\subsection{Problem-Set 8}
\subsection{Problem-Set 9}
\subsection{Problem-Set 10}
\subsection{Problem-Set 11}

%%% Local Variables:
%%% mode: latex
%%% TeX-master: "../OCW-Notes"
%%% End:
