\subsection{Qualifying Exam, January `06}
\begin{problem}
  The birthdays of \(5\) people are known to fall in exactly \(3\) calendar
  months. What is the probability that exactly two of the \(5\) were born
  in January?
\end{problem}
\begin{solution*}
\end{solution*}

\begin{problem}
  Coupons are drawn, independently, with replacement, one at a time, from a
  set of \(10\) coupons. Find, explicitly, the expected number of draws
  \begin{enumerate}[label=(\alph*),noitemsep]
  \item until the first draw coupon is drawn again;
  \item until a duplicate occurs.
  \end{enumerate}
\end{problem}
\begin{solution*}
\end{solution*}

\begin{problem}
  Let \(N\) be a positive integer. Choose an integer at random from
  \(\{1,\dotsc,N\}\). Let \(E\) be the event that your chosen random number
  is divisible by \(3\), and divisible by at least one of \(4\) and \(6\),
  but not divisible by \(5\). Find, explicitly, \(\lim_{N\to\infty}P(E)\).
\end{problem}
\begin{solution*}
\end{solution*}

\begin{problem}
  Anirban is driving his Dodge on a highway with \(4\) lanes each way. He
  is wired to change lanes every minute on the minute. He changes with
  equal probability to either adjacent lane if there are two adjacent
  lanes, and the successive changes are mutually independent. Find,
  explicitly, the probability that after \(4\) minutes, Anirban is back to
  the lane he started from
  \begin{enumerate}[label=(\alph*),noitemsep]
  \item if he started at an outside lane;
  \item if he started at an inside lane.
  \end{enumerate}
\end{problem}
\begin{solution*}
\end{solution*}

\begin{problem}
  Burgess is going to Moose Pass, Alaska. He is driving his Dodge. He puts
  his car on cruise control at \(\SI{70}{\mph}\). Gas stations are located
  every \(30\) miles, starting from his home. His car runs out of gas at a
  time distributed as an exponential with mean \(4\) hours. When that
  happens, he gets out, takes his bik out of his trunk, and bikes to the
  next gas station say \(M\), at \(\SI{10}{\mph}\). Let the time elapsed
  between when Burgess starts his trip and when he arrives at the gas
  station \(M\) be \(T\). Find \(E(T)\).
\end{problem}
\begin{solution*}
\end{solution*}

\begin{problem}
  A fair coin is tossed \(n\) times. Suppose \(X\) heads are
  obtained. Given \(X=x\), let \(Y\) be generated according to the Poisson
  distribution with mean \(x\). Find the unconditional variance of \(Y\),
  and then find the limit of the probability
  \(P\bigl(|Y-n/2|>n^{3/4}\bigr)\), as \(n\to\infty\).
\end{problem}
\begin{solution*}
\end{solution*}

\begin{problem}
  Anirban plays a game repeatedly. On each play he wins an amount uniformly
  distributed in \((0,1)\) dollars, and then he tips the lady in charge of the
  game the square of the amount he has won. Then he plays again, tips
  again, and so on. Approximately calculate the probability that if he
  plays and tips six hundred times, his total winnings minus his total tips
  will exceed \(\$105\).
\end{problem}
\begin{solution*}
\end{solution*}

\begin{problem}
  Anirban's dog got mad at him and broke his walking cane, first uniformly
  into two peices, and then the long piece again uniformly into two
  pieces. Find the probability that Anirban can make a triangle out of the
  three pieces of his cane.
\end{problem}
\begin{solution*}
\end{solution*}

\begin{problem}
  Suppose \(X\), \(Y\), \(Z\) are identically independently distributed
  \(\Exp(1)\) random variables. Find the joint density of \((X,XY,XYZ)\).
\end{problem}
\begin{solution*}
\end{solution*}

\begin{problem}
  Let \(X\) be the number of kings and \(Y\) the number of hearts in a
  Bridge hand. Find the correlation between \(X\) and \(Y\).
\end{problem}
\begin{solution*}
\end{solution*}

%%% Local Variables:
%%% mode: latex
%%% TeX-master: "../MA519-HW-ALL"
%%% End:
