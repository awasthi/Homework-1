\begin{problem}[Handout 17, \# 16]
  Suppose \(X\sim\Exp(1)\), \(Y\sim U[0,1]\), and \(X\), \(Y\) are
  independent.
  \begin{enumerate}[label=(\alph*),noitemsep]
  \item Find the density of \(X+Y\).
  \item Find the density of \(XY\).
  \end{enumerate}
\end{problem}
\begin{solution}
  For part (a): Since \(X\) and \(Y\) are independent, the distribution of
  \(X+Y\) is given by the convolution
  \[
    f_{X+Y}(x)=\int_{-\infty}^\infty f_X(x-y)f_Y(y)\diff y,
  \]
  where
  \begin{align*}
    f_X(x)
      &=\begin{cases}
        \rme^{-x}&\text{for \(x\geq 0\),}\\
        0&\text{otherwise,}
      \end{cases}
    &f_Y(x)
       &=\begin{cases}
         1&\text{for \(0\leq x\leq 1\),}\\
         0&\text{otherwise.}
       \end{cases}
  \end{align*}
  Therefore, a straight forward calculation gives us
  \begin{align*}
    f_{X+Y}(x)
    &=\int_{-\infty}^\infty
      \chi_{[0,\infty)}(x-y)\rme^{-(x-y)}\chi_{[0,1]}(y)\diff y\\
    &=\rme^{-x}\int_{-\infty}^{\infty}\rme^y\chi_{[0,\infty)}(x-y)\chi_{[0,1]}(y)\diff
      y\\
    &=\begin{cases}
        0&\text{for \(x<0\),}\\
        1-\rme^{-x}&\text{for \(0\leq x\leq 1\),}\\
        (\rme-1)\rme^{-x}&\text{for \(x>1\).}
      \end{cases}
  \end{align*}
  Now let us run a sanity check by demonstrating that
  \(\int_{-\infty}^\infty f_{X+Y}(x)\diff x=1\),
  \begin{align*}
    \int_{-\infty}^\infty f_{X+Y}(x)\diff x
    &=\int_0^1 [1-\rme^{-x}]\diff x+(\rme-1)\int_1^\infty\rme^{-x}\diff x\\
    &=\bigl[1+\rme^{-1}-1-0\bigr]+(\rme-1)\bigl[0-(-\rme^{-1})\bigr]\\
    &=\rme^{-1}+1-\rme^{-1}\\
    &=1.
  \end{align*}

  For part (b): Let us find the CDF of \(XY\). By a direct computation
  \begin{align*}
    P(XY\leq z)
    &=\iint\limits_{\{\,(x,y):xy\leq z\,\}} f_X(x)f_Y(y)\diff x\diff y\\
    &=\iint\limits_{\{\,(x,y):xy\leq z\,\}} \rme^{-x}\chi_{[0,\infty)}(x)\chi_{[0,1]}(y)
      \diff x\diff y\\
    &=\int_0^1\int_0^{y/z}\rme^{-x}\diff x\diff y\\
    &=\int_0^1 1-\rme^{-z/y}\diff y
  \end{align*}
  so
  \[
    f_{XY}(z)=\frac{d}{dz}\int_0^1 1-\rme^{-z/y}\diff
    y=\int_0^1\frac{\rme^{-z/y}}{y}\diff y.\qedhere
  \]
\end{solution}
\newpage

\begin{problem}[Handout 17, \# 18]
  Two points \(A\), \(B\) are chosen at random from the unit circle. Find
  the probability that the circle centered at \(A\) with radius \(AB\) is
  fully contained within the original unit circle.
\end{problem}
\begin{solution}
  The probability that a circle centered at \(A\) with radius \(AB\) is
  contained in the original circle is zero. What the professor means is
  ``two points \(A\), \(B\) are chosen at random from \emph{inside} the
  unit circle''. We can think of choosing \(A\) as choosing a random
  variable \(0<R<1\) representing the distance of \(A\) from the origin and
  we ask what is the probability that the point \(B\) lands inside the
  circle of radius \(1-R\) centered at \(A\).

  First, let us find the distribution for the radius \(R\). We can find the
  CDF of \(R\) as the ratio of the area of the circle centered at the
  origin with radius \(x\) and the unit circle; i.e.,
  \[
    P(R\leq x)=\frac{\pi x^2}{\pi\cdot 1^2}=x^2\qquad \text{for \(0<x<1\).}
  \]
  Thus the PDF of \(R\) is
  \[
    f_R(x)=2x\qquad\text{for \(0<x<1\).}
  \]

  Now, the probability that \(B\) lands in the set
  \(D_{1-R}(A)=\{\,x:|x-A|<1-R\,\}\) given that \(|A|=R\) is given by the
  ratio of the area
  \[
    P(B\in D_{1-R}(A)\,|\,|A|=x)=(1-x)^2\qquad\text{for \(0<x<1\).}
  \]
  Then by the total probability formula,
  \begin{align*}
    P(B\in D_{1-R}(A))
    &=\int_0^1 (1-x)^2\cdot 2x\diff x\\
    &=2\int_0^1 x-2x^2+x^3\\
    &=2\left(1+\frac{1}{4}-\frac{2}{3}\right)\\
    &=\frac{1}{6}.\qedhere
  \end{align*}
\end{solution}
\newpage

\begin{problem}[Handout 17, \# 19]
  Let \(X\), \(Y\) be i.i.d.\@ \(U[0,1]\) random variables. Find the
  correlation between \(\max\{X,Y\}\) and \(\min\{X,Y\}\).
\end{problem}
\begin{solution}
  First, let us find the CDF of \(W\defeq \max\{X,Y\}\) and \(Z\defeq
  \min\{X,Y\}\). These are
  \begin{align*}
    P(W\leq x)
    &=P(\max\{X,Y\}\leq x)\\
    &=P(\text{\(X\leq x\) and \(Y\leq x\)})\\
    &=P(X\leq x)P(Y\leq x)\\
    &=\begin{cases}
      x^2&\text{for \(0\leq x\leq 1\),}\\
      1&\text{for \(x>1\),}\\
      0&\text{otherwise.}
    \end{cases}\\
  \end{align*}
  and
  \begin{align*}
    P(Z\leq x)
    &=1-P(Z\geq x)\\
    &=1-P(\min\{X,Y\}\geq x)\\
    &=1-P(\text{\(X\geq x\) and \(Y\geq x\)})\\
    &=1-P(X\geq x)P(Y\geq x)\\
    &=\begin{cases}
      1-(1-x)^2&\text{for \(0\leq x\leq 1\),}\\
      1&\text{for \(x>1\),}\\
      0&\text{otherwise.}
    \end{cases}\\  \end{align*}
  Thus, their corresponding PDFs are
  \[
    f_W(x)=
    \begin{cases}
      2x&\text{for \(0\leq x\leq 1\),}\\
      0&\text{otherwise}
    \end{cases}\qquad
    f_Z(x)=
    \begin{cases}
      2(1-x)&\text{for \(0\leq x\leq 1\),}\\
      0&\text{otherwise.}
    \end{cases}
  \]

  With this data we can now compute the mean of \(W\) and \(Z\) and find
  the correlation between \(W\) and \(Z\). First, let us find the first and
  second moments of \(W\), \(Z\).

  The first moment of \(W\) and \(Z\) are
  \begin{align*}
    E(W)
    &=\int_0^12 x^2\diff x\\
    &=\frac{2}{3},
  \end{align*}
  and
  \begin{align*}
    E(Z)
    &=\int_0^1 2x(1-x)\diff x\\
    &=\frac{1}{3}.
  \end{align*}
  The second moment of \(W\) and \(Z\) are
  \begin{align*}
    E(W)
    &=\int_0^12 x^3\diff x\\
    &=\frac{1}{2},
  \end{align*}
  and
  \begin{align*}
    E(Z)
    &=\int_0^1 2x^2(1-x)\diff x\\
    &=\frac{2}{3}-\frac{1}{2}\\
    &=\frac{1}{6}.
  \end{align*}
  Thus, the variances are
  \begin{align*}
    \Var(W)&=\frac{1}{2}-\frac{4}{9}\\
           &=\frac{1}{18}
  \end{align*}
  and
  \begin{align*}
    \Var(Z)&=\frac{1}{6}-\frac{1}{9}\\
           &=\frac{1}{18}.
  \end{align*}
  Thus,
  \[
    \sigma_W=\sigma_Z=\sqrt{1/18}.
  \]

  Lastly, we must find the covariance of \(W\) and \(Z\). That is,
  \begin{align*}
    \Cov(W,Z)
    &=E(WZ)-E(W)E(Z)\\
    &=E(\max\{X,Y\}\min\{X,Y\})-E(W)E(Z)\\
    &=E(XY)-E(W)E(Z)
    \shortintertext{since \(X\) and \(Y\) are independent}
    &=E(X)E(Y)-E(W)E(Z)\\
    &=\frac{1}{4}-\frac{2}{9}\\
    &=\frac{1}{36}.
  \end{align*}

  At last we come to the correlation of \(W\) and \(Z\),
  \[
    \rho_{WZ}=%
    \frac{\Cov(W,Z)}{\sigma_W\sigma_Z}=%
    \frac{1/36}{1/18}=%
    \frac{1}{2}.\qedhere
  \]
\end{solution}

%%% Local Variables:
%%% mode: latex
%%% TeX-master: "../MA519-HW-Current"
%%% End:
