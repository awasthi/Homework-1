\chapter{Group Theory}
\section{The Sign of a Permutation}
Summary of Keith Conrad's blurb by the same name.

Throughout this discussion $n\geq 2$. A cycle in $S_n$ is the (non-unique)
product of transpositions: the identity $(1)$, $(1\;2)(1\;2)$, and a
$k$-cycle with $k\geq 2$ can be written as
\[
(i_1\;i_2\;\cdots\;i_k)=(i_1\;i_2)(i_2\;i_3)\cdots(i_{k-1}\;i_k).
\]
For example a $3$-cycle $(a\;b\;c)$ -- which means $a$, $b$ and $c$ are
distict -- can be written as
\[
(a\;b\;c)=(a\;b)(b\;c).
\]
This is not the only way to write $(a\;b\;c)$ using transpositions, e.g.,
$(a\;b\;c)=(b\;c)(a\;c)=(a\;c)(a\;b)$.

Since any permutation in $S_n$ is the product of cycles and any cycle is a
product of transpositions, any permutation in $S_n$ is a product of
transpositions. Unlike the unique decomposition of a permutation into
products of disjoint cycles, the decomposition of a permutation is

%%% Local Variables:
%%% TeX-master: "../Conrads-Notes"
%%% End:
