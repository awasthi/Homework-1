\def\auth{Carlos Salinas}%
\def\tight{Micro Teaching Session}%
\def\short{MCSess}%
\def\due{October 3, 2016}%
\def\class{mathematics}%
\def\subject{calculus ii, recitation}
\def\email{salinac@purdue.edu}%

\documentclass[article,oneside]{memoir}
\usepackage{geometry}

%% Base customization and variables
\usepackage[eulermath]{general-setup}%

%% Footnote style
\renewcommand*{\thefootnote}{\fnsymbol{footnote}}%

%% Hyperref setup
\hypersetup{%
  breaklinks,%
  colorlinks=true,%
  linkcolor=black,%
  citecolor=black,%
  filecolor=black,%
  menucolor=black,%
  runcolor=black,%
  urlcolor=black,%
  pdftitle={\short},%
  pdfauthor={\auth},%
  pdfkeywords={\subject},%
  pdfsubject={\class},%
  % pageanchor={false},%
  unicode%
}

%% Graphics path as it says
\graphicspath{{figures/}}%

\begin{document}%

\author{\textti{\auth}}%
\title{\textti{\tight}}%
\date{\textti{\due}}%

\thispagestyle{empty}
\frontmatter
\maketitle%
\tableofcontents*%
\newpage

\mainmatter
\chapter{Script}
This is my script for the \emph{Micro-teaching recitation presentation} on
Monday, October 3, 2016. I have attached a sample 15-minute quiz at the end
of this script.

\section{A review of l'Hôpital's rule}
Suppose we are trying to analyze the behavior of the function
\[
  F(x)=\frac{\ln x}{x-1}.
\]
Although \(F\) is not defined when \(x=1\), we need to know how \(F\)
behaves \emph{near \(1\)}. In particular, we would like to evaluate the
limit
\begin{equation}
  \label{eq:rec:1}
  \lim_{x\to 1}\frac{\ln x}{x-1}.
\end{equation}
In computing this limit, we cannot proceed with the typical strategy which
is to compute the limit of the numerator and the limit of the denominator
and take their quotients because in this case \(\lim_{x\to 1}x-1=0\) and
the quotient \(0/0\) is not defined.

In general, if we have a limit of the form
\[
  \lim_{x\to a}\frac{f(x)}{g(x)}
\]
where both \(f(x)\to 0\) and \(g(x)\to 0\) as \(x\to a\), the limit may or
may not exist; we will call a limit of this type a \emph{indeterminate form
  of type \(0/0\)}.

\begin{theorem}[L'Hôpital's Rule]
  Suppose \(f\) and \(g\) are differentiable and \(g'(x)\neq 0\) on an open
  interval \(I\) that contains \(a\) (except possibly at \(a\)). Suppose
  that
  \[
    \lim_{x\to a}f(x)=0,\pm\infty\qquad\qquad\lim_{x\to a} g(x)=0,\pm\infty.
  \]
  Then
  \[
    \lim_{x\to a}\frac{f(x)}{g(x)}=\lim_{x\to a}\frac{f'(x)}{g'(x)}
  \]
  if the limit on the right side exists or is \(\pm\infty\).
\end{theorem}

\newpage
\section{Homework Solutions}
\newpage
\section{Sample Quiz}
\end{document}%

%%% Local Variables:
%%% mode: latex
%%% TeX-master: t
%%% End:
