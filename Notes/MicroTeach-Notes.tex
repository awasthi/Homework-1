\def\auth{Carlos Salinas}%
\def\tight{Micro-teaching Session}%
\def\short{MCSess}%
\def\due{October 3, 2016}%
\def\class{mathematics}%
\def\subject{calculus ii, recitation}
\def\email{salinac@purdue.edu}%

\documentclass[article,oneside]{memoir}
\usepackage{geometry}

%% Base customization and variables
\usepackage{preamble}%

%% Footnote style
\renewcommand*{\thefootnote}{\fnsymbol{footnote}}%

%% Hyperref setup
\hypersetup{%
  breaklinks,%
  colorlinks=true,%
  linkcolor=black,%
  citecolor=black,%
  filecolor=black,%
  menucolor=black,%
  runcolor=black,%
  urlcolor=black,%
  pdftitle={\short},%
  pdfauthor={\auth},%
  pdfkeywords={\subject},%
  pdfsubject={\class},%
  % pageanchor={false},%
  unicode%
}

%% Graphics path as it says
\graphicspath{{figures/}}%

\begin{document}%

\author{\textti{\auth}}%
\title{\textti{\tight}}%
\date{\textti{\due}}%

\thispagestyle{empty}
\frontmatter
\maketitle%
\tableofcontents*%
\newpage

\mainmatter
\chapter{Script}
This is my script for the \emph{Micro-teaching recitation presentation} on
Monday, October 3, 2016. I have attached a sample 15-minute quiz at the end
the document.

\section{L'Hôpital's rule}
Today we go over some of your
\textsf{Web\color{Red!85!black}{\textbf{Assign}}} problems to show you how
to use l'Hôpital's rule to evaluate the limits of quotients \(f/g\) and
products \(fg\).

The problems we will be discussing in today's recitations are problems 2,
3, 4, 7, 8, 9, and 10. But first, a vote. (Draw a table on the chalkboard
\[
  \left.
    \begin{tabular}{|c|c|c|c|}
  \hline
  Problem&Votes&
  Problem&Votes\\
  \hline
  2&&
  3&\\
  4&&
  7&\\
  8&&
  9&\\
  10&&&\\
  \hline
  \end{tabular}
  \;
  \right)
\]
Raise your hand if you want to
see a detailed solution to problem 2 [pause], problem 3, etc.

\section{Exercises}
\begin{problem*}[\textsf{Web\color{Red!85!black}{\textbf{Assign}}}, \# 2]
  Find the limit. Use l'Hôpital's rule if appropriate. If there is a more
  elementary method, consider using it.
  \[
    \lim_{x\to 0}\frac{\sin 2x}{\sin 3x}.
  \]
\end{problem*}
\begin{solution}
  First, let's look at the limit of the numerator and the limit of the
  denominator, individually. For the numerator, we have
  \[
    \lim_{x\to 0}\sin 2x=0
  \]
  and, similarly, for the denominator
  \[
    \lim_{x\to 0}\sin 3x=0.
  \]
  As you may remember for class, this is a limit of the type \(0/0\) and a
  prime candidate for l'Hôpital's rule.

  Remember that l'Hôpital's rule says that the limit of a quotient
  \(f/g\) is the limit of the quotient of their derivatives \(f'/g'\),
  i.e.,
  \[
    \lim_{x\to 0}\frac{\sin 2x}{\sin 3x}=
    \lim_{x\to 0}\frac{2\cos 2x }{3\cos 3x}.
  \]
  Now, the limit of the \(\cos\) in the numerator and denominator, as
  \(x\to 0\), is \(1\), so
  \[
    \lim_{x\to 0}\frac{\sin 2x}{\sin 3x}=%
    \frac{2}{3}
    \left[%
      \frac{\lim_{x\to 0}\cos 2x}{\lim_{x\to 0}\cos 3x}%
    \right]
    =\frac{3}{2}.
  \]
  easy, right?

  Let's have a look at the next problem.
\end{solution}

\begin{problem*}[\textsf{Web\color{Red!85!black}{\textbf{Assign}}}, \# 3]
  Find the limit. Use l'Hôpital's rule if appropriate. If there is a more
  elementary method, consider using it.
  \[
    \lim_{x\to 0}\frac{e^{7x}-1-7x}{x^2}.
  \]
\end{problem*}
\begin{solution}
  For this problem we have
  \[
    \lim_{x\to 0}\frac{e^{7x}-1-7x}{x^2}.
  \]
  Note that as \(x\to 0\), the denominator goes to \(0\). Thus, by
  l'Hôpital's rule
  \begin{align*}
    \lim_{x\to 0}\frac{e^{7x}-1-7x}{x^2}
    &=\lim_{x\to 0}\frac{7e^{7x}-7}{x},
      \intertext{but here the denominator still goes to \(0\), so we use
      l'Hôpital's rule again}
    &=\lim_{x\to 0}\frac{49e^{7x}}{1}\\
    &=49.
  \end{align*}
\end{solution}

\begin{problem*}[\textsf{Web\color{Red!85!black}{\textbf{Assign}}}, \# 4]
  Find the limit. Use l'Hôpital's rule if appropriate. If there is a more
  elementary method, consider using it.
  \[
    \lim_{x\to\infty}\frac{\bigl( \ln(x) \bigr)^2}{5x}.
  \]
\end{problem*}
\begin{solution}
  Both the numerator and denominator go to \(\infty\). This is a limit of
  the type \(\infty/\infty\). Here, we have
  \begin{align*}
    \lim_{x\to\infty}\frac{\bigl( \ln(x) \bigr)^2}{5x}
    &=\lim_{x\to\infty}\frac{2(\ln x)(1/x)}{5}\\
    &=\lim_{x\to\infty}\frac{2\ln x}{5x},
      \intertext{using l'Hôpital's rule again, we have}
    &=\lim_{x\to\infty}\frac{2(1/x)}{5}\\
    &=\lim_{x\to\infty}\frac{2}{5x}\\
    &=0.
  \end{align*}
\end{solution}

\begin{problem*}[\textsf{Web\color{Red!85!black}{\textbf{Assign}}}, \# 7]
  Find the limit. Use l'Hôpital's rule if appropriate. If there is a more
  elementary method, consider using it.
  \[
    \lim_{x\to\infty} x\tan(5/x).
  \]
\end{problem*}
\begin{solution}
  Here we do something you may not be completely familiar with, we do what
  is called a \emph{change of variables}. Setting \(u\defeq 1/x\) we see
  that as \(x\to\infty\), \(u\to 0\) so we can turn the problem
  \[
    \lim_{x\to\infty} x\tan(5/x)
  \]
  into the equivalent problem
  \[
    \lim_{u\to 0}\frac{\tan(5u)}{u}.
  \]
  You see this, right?

  Now, by l'Hôpital's rule
  \begin{align*}
    \lim_{x\to\infty} x\tan(5/x)
    &=\lim_{u\to 0}\frac{\tan(5u)}{u}\\
    &=\lim_{u\to 0}\frac{5\sec^2 u}{1}\\
    &=\lim_{u\to 0}\frac{5}{\cos^2 u}\\
    &=5.
  \end{align*}
\end{solution}

\begin{problem*}[\textsf{Web\color{Red!85!black}{\textbf{Assign}}}, \# 8]
  Find the limit. Use l'Hôpital's rule if appropriate. If there is a more
  elementary method, consider using it.
  \[
    \lim_{x\to 0}(\csc x-\cot x).
  \]
\end{problem*}
\begin{solution}
  Let's write \(\csc(x)-\cot(x)\) under a single quotient
  \begin{align*}
    \csc x-\cot x
    &=\frac{1}{\sin x}-\frac{1}{\tan x}\\
    &=\frac{1}{\sin x}-\frac{\cos x}{\sin x}\\
    &=\frac{1-\cos x}{\sin x}.
  \end{align*}
  Now we can start looking at the limit.

  By l'Hôpital's rule, we have
  \begin{align*}
    \lim_{x\to 0}(\csc x-\cot x)
    &=\lim_{x\to 0}\left[\frac{1-\cos x}{\sin x}\right]\\
    &=\lim_{x\to 0}\frac{\sin x}{\cos x}\\
    &=\lim_{x\to 0}\tan x\\
    &=0.
  \end{align*}
\end{solution}

\begin{problem*}[\textsf{Web\color{Red!85!black}{\textbf{Assign}}}, \# 9]
  Find the limit. Use l'Hôpital's rule if appropriate. If there is a more
  elementary method, consider using it.
  \[
    \lim_{x\to 0}(1-8x)^{1/x}.
  \]
\end{problem*}
\begin{solution}
  For this problem, we can again use the \emph{change of variables}
  \(u=(1/x)\) and solve the problem
  \[
    \lim_{u\to\infty}\left(1-\frac{8}{u}\right)^{u}
  \]
  You may have seen this limit before in yours study of sequences, if you
  have, you will immediately recognize the limit of this function as
  \(e^{-8}\).

  If you don't, that's alright; we'll provide some details. Suppose for a
  moment that the limit of this function is \(L\). Then, using log rules,
  we have
  \begin{align*}
    \frac{\ln L}{\ln\bigl((u-8)/u\bigr)}&=u\\
    \ln L&=u\ln\left(\frac{u-8}{u}\right).
  \end{align*}
  Now, let's take the limit
  \begin{align*}
    \lim_{u\to\infty}u\ln\left(\frac{u-8}{u}\right)
    &=\lim_{u\to\infty}\frac{\ln\bigl((u-8)/u\bigr)}{1/u},
    \intertext{which, by l'Hôpital's rule, becomes}
    &=\lim_{u\to\infty}\frac{\bigl((u-u+8)/u^2\bigr)\bigl(u/(u-8)\bigr)}{-1/u^2}\\
    &=\lim_{u\to\infty}\frac{-8u}{u-8}
      \intertext{and again}
    &=\lim_{u\to\infty}-8.\\
  \end{align*}

  Thus, the log of the limit is \(-8\), i.e.,
  \begin{align*}
    \ln L&=-8
    \intertext{so}
    L&=e^{-8}.
  \end{align*}
\end{solution}

\begin{problem*}[\textsf{Web\color{Red!85!black}{\textbf{Assign}}}, \# 10]
  Find the limit. Use l'Hôpital's rule if appropriate. If there is a more
  elementary method, consider using it.
  \[
    \lim_{x\to\infty}x^{8/x}.
  \]
\end{problem*}
\begin{solution}
  We use the same approach. Let \(u\defeq 1/x\). Then,
  \[
    L=\lim_{x\to\infty} x^{8/x}=\lim_{u\to 0}\left(\frac{1}{u}\right)^{8u},
  \]
  if it exists.

  Thus, taking the natural log of both sides
  \begin{align*}
    \frac{\ln L}{\ln (1/u)}&=8u\\
    \ln L&=8u\ln(1/u).
  \end{align*}

  Now,
  \begin{align*}
    \lim_{u\to 0} u\ln(1/u)
    &=\lim_{u\to 0}\frac{\ln(1/u)}{(1/u)}
      \intertext{which, by l'Hôpital's rule, becomes}
    &=\lim_{u\to 0}\frac{u(-1/u^2)}{-1/u^2}\\
    &=\lim_{u\to 0}u\\
    &=0.
  \end{align*}

  Thus,
  \begin{align*}
    \ln L&=0\\
    L&=e^0=1.
  \end{align*}
\end{solution}

\newpage
\section{Sample Quiz}
You have \textbf{15 minutes} to complete this quiz. You may work in groups,
but you are not allowed to use any other resources.
\begin{problem*}[\textsc{a}]
  Evaluate the limit
  \[
    \lim_{t\to\infty}\frac{\ln t}{t^2}.
  \]
\end{problem*}
\vspace{80pt}
\begin{problem*}[\textsc{b}]
   Evaluate the limit
  \[
    \lim_{x\to 4}\frac{x^2-16}{x-4}.
  \]
\end{problem*}
\vspace{80pt}
\begin{problem*}[\textsc{c}]
  Evaluate the limit
  \[
    \lim_{x\to 2}\frac{x^3-7x^2+10x}{x^2+x-6}.
  \]
\end{problem*}
\end{document}

%%% Local Variables:
%%% mode: latex
%%% TeX-master: t
%%% End:
