\def\auth{Carlos Salinas}%
\def\tight{Micro-teaching Session}%
\def\short{MCSess}%
\def\due{October 3, 2016}%
\def\class{mathematics}%
\def\subject{calculus ii, recitation}
\def\email{salinac@purdue.edu}%

\documentclass[article,oneside]{memoir}
\usepackage{geometry}

%% Base customization and variables
\usepackage{preamble}%

%% Footnote style
\renewcommand*{\thefootnote}{\fnsymbol{footnote}}%

%% Hyperref setup
\hypersetup{%
  breaklinks,%
  colorlinks=true,%
  linkcolor=black,%
  citecolor=black,%
  filecolor=black,%
  menucolor=black,%
  runcolor=black,%
  urlcolor=black,%
  pdftitle={\short},%
  pdfauthor={\auth},%
  pdfkeywords={\subject},%
  pdfsubject={\class},%
  % pageanchor={false},%
  unicode%
}

%% Graphics path as it says
\graphicspath{{figures/}}%

\begin{document}%

\author{\textti{\auth}}%
\title{\textti{\tight}}%
\date{\textti{\due}}%

\thispagestyle{empty}
\frontmatter
\maketitle%
\tableofcontents*%
\newpage

\mainmatter
\chapter{Script}
This is my script for the \emph{Micro-teaching recitation presentation} on
Monday, October 3, 2016. I have attached a sample 15-minute quiz at the end
the document.

\section{L'Hôpital's rule}
Students! This week, you have learned about l'Hôpital's rule and how to use
it to evaluate limits of (a) quotients \(f/g\), and limits of (b) products
\(fg\) where in the first case we may have


\begin{problem*}[WebAssign, \# 2]
  Find the limit. Use l'Hôpital's rule if appropriate. If there is a more
  elementary method, consider using it.
  \[
    \lim_{x\to 0}\frac{\sin 2x}{\sin 3x}.
  \]
\end{problem*}
\begin{solution}
\end{solution}

\begin{problem*}[WebAssign, \# 3]
  Find the limit. Use l'Hôpital's rule if appropriate. If there is a more
  elementary method, consider using it.
  \[
    \lim_{x\to 0}\frac{e^{7x}-1-7x}{x^2}.
  \]
\end{problem*}
\begin{solution}
\end{solution}

\begin{problem*}[WebAssign, \# 4]
  Find the limit. Use l'Hôpital's rule if appropriate. If there is a more
  elementary method, consider using it.
  \[
    \lim_{x\to\infty}\frac{\bigl( \ln(x) \bigr)^2}{5x}.
  \]
\end{problem*}
\begin{solution}
\end{solution}

\begin{problem*}[WebAssign, \# 7]
  Find the limit. Use l'Hôpital's rule if appropriate. If there is a more
  elementary method, consider using it.
  \[
    \lim_{x\to\infty} x\tan(5/x).
  \]
\end{problem*}
\begin{solution}
\end{solution}

\begin{problem*}[WebAssign, \# 8]
  Find the limit. Use l'Hôpital's rule if appropriate. If there is a more
  elementary method, consider using it.
  \[
    \lim_{x\to 0}\bigl(\csc(x)-\cot(x)\bigr).
  \]
\end{problem*}
\begin{solution}
\end{solution}

\begin{problem*}[WebAssign, \# 9]
  Find the limit. Use l'Hôpital's rule if appropriate. If there is a more
  elementary method, consider using it.
  \[
    \lim_{x\to 0}(1-8x)^{1/x}.
  \]
\end{problem*}
\begin{solution}
\end{solution}

\begin{problem*}[WebAssign, \# 10]
  Find the limit. Use l'Hôpital's rule if appropriate. If there is a more
  elementary method, consider using it.
  \[
    \lim_{x\to\infty}x^{8/x}.
  \]
\end{problem*}
\begin{solution}
\end{solution}

\newpage
\section{Sample Quiz}

\end{document}

%%% Local Variables:
%%% mode: latex
%%% TeX-master: t
%%% End:
