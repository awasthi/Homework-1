\def\auth{Carlos Salinas}%
\def\tight{Micro-teaching Session}%
\def\short{MCSess}%
\def\due{October 3, 2016}%
\def\class{mathematics}%
\def\subject{calculus ii, recitation}
\def\email{salinac@purdue.edu}%

\documentclass[article,oneside]{memoir}
\usepackage{geometry}

%% Base customization and variables
\usepackage{preamble}%

%% Footnote style
\renewcommand*{\thefootnote}{\fnsymbol{footnote}}%

%% Hyperref setup
\hypersetup{%
  breaklinks,%
  colorlinks=true,%
  linkcolor=black,%
  citecolor=black,%
  filecolor=black,%
  menucolor=black,%
  runcolor=black,%
  urlcolor=black,%
  pdftitle={\short},%
  pdfauthor={\auth},%
  pdfkeywords={\subject},%
  pdfsubject={\class},%
  % pageanchor={false},%
  unicode%
}

%% Graphics path as it says
\graphicspath{{figures/}}%

\begin{document}%

\author{\textti{\auth}}%
\title{\textti{\tight}}%
\date{\textti{\due}}%

\thispagestyle{empty}
\frontmatter
\maketitle%
\tableofcontents*%
\newpage

\mainmatter
\chapter{Script}
This is my script for the \emph{Micro-teaching recitation presentation} on
Monday, October 3, 2016. I have attached a sample 15-minute quiz at the end
the document.

\section{L'Hôpital's rule}
Today we go over some of your
\textsf{Web\color{Red!85!black}{\textbf{Assign}}} problems to show you how
to use l'Hôpital's rule to evaluate the limits of quotients \(f/g\) and
products \(fg\).

The problems we will be discussing in today's recitations are problems 2,
3, 4, 7, 8, 9, and 10. But first, a vote. (Draw a table on the chalkboard
\[
  \left.
    \begin{tabular}{|c|c|c|c|}
  \hline
  Problem&Votes&
  Problem&Votes\\
  \hline
  2&&
  3&\\
  4&&
  7&\\
  8&&
  9&\\
  10&&&\\
  \hline
  \end{tabular}
  \;
  \right)
\]
Raise your hand if you want to
see a detailed solution to problem 2 [pause], problem 3, etc.

\section{Exercises}
\begin{problem*}[\textsf{Web\color{Red!85!black}{\textbf{Assign}}}, \# 2]
  Find the limit. Use l'Hôpital's rule if appropriate. If there is a more
  elementary method, consider using it.
  \[
    \lim_{x\to 0}\frac{\sin 2x}{\sin 3x}.
  \]
\end{problem*}
\begin{solution}
  First, let's look at the limit of the numerator and the limit of the
  denominator, individually. For the numerator, we have
  \[
    \lim_{x\to 0}\sin 2x=0
  \]
  and, similarly, for the denominator
  \[
    \lim_{x\to 0}\sin 3x=0.
  \]
  As you may remember for class, this is a limit of the type \(0/0\) and a
  prime candidate for l'Hôpital's rule.

  Remember that l'Hôpital's rule says that the limit of a quotient
  \(f/g\) is the limit of the quotient of their derivatives \(f'/g'\),
  i.e.,
  \[
    \lim_{x\to 0}\frac{\sin 2x}{\sin 3x}=
    \lim_{x\to 0}\frac{2\cos 2x }{3\cos 3x}.
  \]
  Now, the limit of the \(\cos\) in the numerator and denominator, as
  \(x\to 0\), is \(1\), so
  \[
    \lim_{x\to 0}\frac{\sin 2x}{\sin 3x}=%
    \frac{2}{3}
    \left[%
      \frac{\lim_{x\to 0}\cos 2x}{\lim_{x\to 0}\cos 3x}%
    \right]
    =\frac{3}{2}.
  \]
  easy, right?

  Let's have a look at the next problem.
\end{solution}

\begin{problem*}[\textsf{Web\color{Red!85!black}{\textbf{Assign}}}, \# 3]
  Find the limit. Use l'Hôpital's rule if appropriate. If there is a more
  elementary method, consider using it.
  \[
    \lim_{x\to 0}\frac{e^{7x}-1-7x}{x^2}.
  \]
\end{problem*}
\begin{solution}
\end{solution}

\begin{problem*}[\textsf{Web\color{Red!85!black}{\textbf{Assign}}}, \# 4]
  Find the limit. Use l'Hôpital's rule if appropriate. If there is a more
  elementary method, consider using it.
  \[
    \lim_{x\to\infty}\frac{\bigl( \ln(x) \bigr)^2}{5x}.
  \]
\end{problem*}
\begin{solution}
\end{solution}

\begin{problem*}[\textsf{Web\color{Red!85!black}{\textbf{Assign}}}, \# 7]
  Find the limit. Use l'Hôpital's rule if appropriate. If there is a more
  elementary method, consider using it.
  \[
    \lim_{x\to\infty} x\tan(5/x).
  \]
\end{problem*}
\begin{solution}
\end{solution}

\begin{problem*}[\textsf{Web\color{Red!85!black}{\textbf{Assign}}}, \# 8]
  Find the limit. Use l'Hôpital's rule if appropriate. If there is a more
  elementary method, consider using it.
  \[
    \lim_{x\to 0}\bigl(\csc(x)-\cot(x)\bigr).
  \]
\end{problem*}
\begin{solution}
\end{solution}

\begin{problem*}[\textsf{Web\color{Red!85!black}{\textbf{Assign}}}, \# 9]
  Find the limit. Use l'Hôpital's rule if appropriate. If there is a more
  elementary method, consider using it.
  \[
    \lim_{x\to 0}(1-8x)^{1/x}.
  \]
\end{problem*}
\begin{solution}
\end{solution}

\begin{problem*}[\textsf{Web\color{Red!85!black}{\textbf{Assign}}}, \# 10]
  Find the limit. Use l'Hôpital's rule if appropriate. If there is a more
  elementary method, consider using it.
  \[
    \lim_{x\to\infty}x^{8/x}.
  \]
\end{problem*}
\begin{solution}
\end{solution}

\newpage
\section{Sample Quiz}

\end{document}

%%% Local Variables:
%%% mode: latex
%%% TeX-master: t
%%% End:
