\chapter{McMullen's Complex Analysis Notes}
\section{Basic Complex Analysis}
\subsection{Some Notation}
The complex numbers will be denoted $\bbC$. We let $\Delta$, $\bbH$, and
$\widehat\bbC$ denote the unit disk $|z|<1$, the upper half-plane $\Im z>1$
and the Riemann sphere $\bbC\cup\{\infty\}$. We write $S^1(r)$ for the
circle $|z|=r$ and $S^1$ for the unit circle, each oriented
counter-clockwise. We also set $\Delta^*=\Delta\minus\{0\}$ and
$\bbC^*=\bbC\minus\{0\}$.
\subsection{Algebra and complex numbers}
The complex numbers are formally defined as the field $\bbC=\bbR[i]$, where
$i^2=-1$. They are represented in the Euclidean plane by
$z=(x,y)=x+iy$. There are two square-roots of $-1$ in $\bbC$; the number
$i$ is the one with positive imaginary part.

An important role is played by the Galois involution $z\mapsto\bar z$. We
define $|z|^2=N(z)=z\bar z=x^2+y^2$. Compatibility of $|z|$ with the
Euclidean metric justifies the identification of $\bbC$ and $\bbR^2$. We
also see that $\bbC$ is a field as $1/z=\bar z/|z|$.

It is also convenient to describe complex numbers by polar coordinates
\[
z=[r,\theta]=r(\cos\theta+i\sin\theta).
\]
Here $r=|z|$ and $\theta=\arg z$ in $\bbR/2\pi\bbZ$. (The multivaluedness
of $\arg z$ requires care but is also the  source of powerful results like
Cauchy's integral formula.) We then have
\[
\left[r_1,\theta_1\right]\left[r_2,\theta_2\right]=
\left[r_1r_2,\theta_1+\theta_2\right].
\]
In particular, the linear maps $f(z)=az+b$, $a\neq 0$, of $\bbC$ to itself,
preserve angles and orientations.

This formula should be proved geometrically; in fact, it is a consequence
of the formula $|ab|=|a||b|$ and similar triangles. It can then be used to
derive the addition formulas for sine and cosine.

\subsection{Algebraic closure}
A critical feature of the complex numbers is that they are
\emph{algebraically closed}, i.e., every polynomial has a root.

Classically, the complex numbers were introduced in the course of solving
real cubic equations. Starting with $x^3+ax+b=0$ one can make a Tschirnhaus
transformation so $a=0$. This is done by introducing a new variable,
$y=cx^2+d$ such that $\sum y_i=\sum {y_i}^2=0$; even when $a$ and $b$ are
real, it may be necessary to choose $c$ complex (the discriminant of the
equation for $c$ is $27b^2+4a^3$). It is negative when the cubic has only
one real root; this can be checked by looking at the product of of the
values of the cubic at its $\min$ and $\max$.

\subsection{Polynomials and rational functions}
Using addition and multiplication we obtain naturally the polynomial
functions $f(z)=\sum_0^n a_nz^2\colon\bbC\to\bbC$. The ring of polynomials
$\bbC[z]$ is an integral domain and a unique factorization domain, since
$\bbC$ is a field. Indeed, since $\bbC$ is algebraically closed, every
polynomial factors into linear terms.

It is useful to add the allowed value $\infty$ to obtain the Riemann sphere
$\widehat\bbC=\bbC\cup\{\infty\}$. Then the rational functions determine
rational maps $f\colon\bbC\to\bbC$. The rational functions $\bbC(z)$ are
the same as the field of fractions for the domain $\bbC[z]$. We set
$f(z)=\infty$ if $q(z)=0$; these points are called the \emph{poles} of $f$.

\subsection{Analysis functions}
Let $U$ be an open set in $\bbC$ and $f\colon U\to\bbC$ a function. We say
that $f$ is \emph{analytic} if
\[
f'(z)=\lim_{t\to\infty}\frac{f(z+)-f(z)}{t}
\]
exists for all $z\in U$. It is important that $t$ approaches $0$ through
arbitrary values of $\bbC$. Remarkably, this condition implies that $f\in
C^\infty$.

%%% Local Variables:
%%% mode: latex
%%% TeX-master: "../Conrads-Notes"
%%% End:
