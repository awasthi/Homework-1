\chapter{Hatcher Algebraic Topology Notes}
\section{The Fundamental Group}
\subsection{The van Kampen Theorem}
Suppose a space $X$ is decomposed as the union of a collection of
pathconnected open subsets $A_\alpha$, each of which contains the
basepoint $x_0\in X$. By the remarks of the preceding paragraph, the
homomorphism $j_\alpha\colon\pi_1(A_\alpha)\to\pi_1(X)$ induced by the
inclusions $A_\alpha\hookrightarrow X$ extend to a homomorphism
$\Phi\colon\ast_\alpha\pi_1(A_\alpha)\to\pi_1(X)$. The van Kapmen theorem
will say that $\Phi$ is very often surjective, but we can expect $\Phi$ to
have a nontrivial kernel in general. For if
$i_{\alpha\beta}\colon\pi_1(A_\alpha\cap B_\beta)\to\pi_1(A_\alpha)$ is the
homomorphism induced by the inclusion $A_\alpha\cap A_\beta\hookrightarrow
A_\alpha$, then $j_\alpha\circ i_{\alpha\beta}=j_\beta\circ
i_{\alpha\beta}$, both these compositions being of the form
$i_{\alpha\beta}(\omega)i_{\alpha\beta}(\omega)^{-1}$ for
$\omega\in\pi_1(A_\alpha\cap A_\beta)$. Van Kampen's theorem asserts that
under fairly broad hypotheses this gives a full description of $\Phi$
\begin{theorem}[Hatcher, 1.20, p.\@ 43]
If $X$ is the union of path-connected open sets $A_\alpha$ each containing
the basepoint $x_0\in X$ and if each intersection $A_\alpha\cap A_\beta$ is
path-connected, then the homomorphism
$\Phi\colon\ast_\alpha\pi_1(A_\alpha)\to\pi_1(X)$ is surjective. If in
addition each intersection $A_\alpha\cap A_\beta\cap A_\gamma$ is
path-connected, then the kernel of $\Phi$ is the normal subgroup $N$
generated by all elements of the form
$i_{\alpha\beta}(\omega)i_{\beta\alpha}(\omega)^{-1}$  for
$\omega\in\pi_1(A_\alpha\cap A_\beta)$, and hence $\Phi$ induces an
isomorphism $\pi_1(X)\cong\ast_\alpha\pi_1(A_\alpha)/N$.
\end{theorem}
\begin{example}[1.21: Wedge Sums]
In chapter 0 of Hatcher, we defined the wedge sum $\bigvee_\alpha X_\alpha$
of a collection of spaces $X_\alpha$ with basepoints $x_\alpha\in X_\alpha$
to be the quotient space of the disjoint union $\bigsqcup_\alpha X_\alpha$
in which all the basepoints $x_\alpha$ are identified to a single point. If
each $x_\alpha$ is a deformation retract of an open neighborhood $U_\alpha$
in $X_\alpha$, then $X_\alpha$ is a deformation retract of its open
neighborhood $A_\alpha=X_\alpha\vee\bigvee_{\beta\neq\alpha}U_\beta$. The
intersection of two or more distinct $A_\alpha$'s is $\bigvee_\alpha
U_\alpha$, which deformation retracts to a point. Van Kampen's theorem then
implies that
$\Phi\colon\ast_\alpha\pi_1(X_\alpha)\to\pi_1\left(\bigvee_\alpha
  X_\alpha\right)$ is an isomorphism.

Thus for a wedge sum $\bigvee_\alpha S^1_\alpha$ of circles,
$\pi_1\left(\bigvee_\alpha S^1_\alpha\right)$ is a free group, the free
product of copies of $\bfZ$, one for each circle $S^1_\alpha$. In
particular, $\pi_1\left(S^1\vee S^1\right)$ is the free group $\bfZ*\bfZ$,
as in the example at the beginning of this section.
\end{example}

\begin{example}[Hatcher 1.22]

\end{example}

\section{Homology}
\subsection{Simplicial and Singular Homology}
\subsubsection[Delta-complexes]{$\Delta$-complexes}
The idea of the $\Delta$-complex generalizes the construction of a
topological space via the quotient of some triangularization of a polygon
in $\bfR^n$. The $n$-dimensional analogue of the triangle is called the
\emph{$n$-simplex}. This is the smallest convex set in a Euclidean space
$\bfR^m$ containing $n+1$ points $v_0,...,v_n$ that do not lie in a less
than $n$ dimensional hyperplane. An equivalent condition is that the
difference vectors $v_1-v_0,...,v_n-v_0$ are linearly independent. The
points $v_i$ are the \emph{vertices} of the simplex, and the simplex itself
is denoted $[v_0,...,v_n]$. For example, there is a standard $n$-simplex
\[
\Delta^n=\left\{\,(t_0,...,t_n)\in\bfR^n\;\middle|\;\text{$\sum_it_i$
and $t_i\geq 0$ for all $i$}\,\right\}
\]
whose vertices are the unit vectors along the coordinate axes. For the
purposes of homology, it is important that we keep track of the ordering on
the vertices $v_i$, so an `$n$-simplex' will always mean an `$n$-simplex
with an ordering on its vertices.' As a consequence, there is a natural
ordering on the edges $[v_i,v_j]$ according to increasing
subscripts.\footnote{I'm not sure what Hatcher means here, unless he is
  choosing the natural ordering on the indices $I\subset\bfN$, i.e., the
  ordering $1<2<\cdots$} Specifying the ordering of the vertices also
determines a canonical linear homeomorphism from the standard $n$-simplex
$\Delta^n$ onto any other $n$-simplex $[v_0,...,v_n]$, preserving the order
of the vertices, namely, $(t_0,...,t_n)\mapsto\sum_it_iv_i$. The
coefficients $t_i$ are \emph{barycentric coordinates} of the point
$\sum_it_iv_i$ in $[v_0,...,v_i]$.

If we delete one of the $n+1$ vertices of the $n$-simplex $[v_0,...,v_n]$,
then the remaining $n$-vertices span an $(n-1)$-simplex called a
\emph{face} of $[v_0,...,v_n]$. We adopt the following convention
\begin{quotation}
The vertices of a face, or of any complex spanned by a subset of the
vertices, will always be ordered according to their order in the larger
simplex.
\end{quotation}
The union of the faces of $\Delta^n$ is the \emph{boundary} of $\Delta^n$,
written $\\partial\Delta^n$. The \emph{open simplex}
$\left.\Delta^n\right.^\circ$ is $\Delta^n\minus\partial\Delta^n$.

A $\Delta$-complex structure on a space $X$ is a collection of maps
$\sigma_\alpha\colon\Delta^n\to X$ with $n$-depending on the index of
$\alpha$ such that:
\begin{enumerate}[label=(\roman*)]
\item The restriction
  $\sigma_\alpha\restriction\left.\Delta^n\right.^\circ$ is
  injective, and each point of $X$ is in the image of exactly one such
  restriction.
\item Each restriction of $\sigma_\alpha$ to a face of $\Delta^n$ is one of
  the maps $\sigma_\beta\colon\Delta^{n-1}\to X$. Here we are
  identifying the face of $\Delta^n$ with $\Delta^{n-1}$ by the canonical
  linear homeomorphism between them that preserves the ordering of the
  vertices.
\item A set $A\subset X$is open iff ${\sigma_\alpha}^{-1}(A)$ is open in
  $\Delta^n$ for each $\sigma_\alpha$.
\end{enumerate}
Among other things, this last condition rules out trivialities like
regarding all of the points in $X$ as individual vertices.

A consequence of (iii) is that $X$ can be built as a quotient space of a
collection of disjoint simplices $\Delta_\alpha^n$, one for each
$\sigma_\alpha\colon\Delta^n\to X$, the quotient space obtained by
identifying each face of $\Delta_\alpha^n$ with the $\Delta_\beta^{n-1}$
corresponding to the restriction $\sigma_\beta$ of $\sigma_\alpha$ to the
face in question, as in (ii). One can think of building the quotient space
inductively: starting with a discrete set of vertices, then attaching edges
to these to produce a graph, the attaching $2$-simplices to the graph, and
so on. From this viewpoint we see that the data specifying a
$\Delta$-complex can be described in a purely combinatorial way as
collections of $n$-simplices $\Delta_\alpha^n$ for each $n$ together with
functions associating to each face of each $n$-simplex $\Delta_\alpha^n$ an
$(n-1)$-simplex $\Delta_\beta^{n-1}$.

More generally, $\Delta$-complexes can be built from collections of
disjoint simplices by identifying varies subsimplices spanned by a subsets
of the vertices, where the identifications are performed using the
canonical linear homeomorphism that preserve the ordering of the
vertices.

Thinking of a $\Delta$-complex $X$ as the quotient space of a collection of
disjoint simplices, it's not hard to see that $X$ must be a Hausdorff
space. Indeed, if $x,y\in X$ they lie in the image of the same simplex
$\Img\sigma_\alpha$ we may take their preimage ${\sigma_\alpha}^{-1}(x)$
and ${\sigma_\alpha}^{-1}(y)$ and find disjoint neighborhoods $U_x$ and
$U_y$ containing these subsets of $\Delta^k$. Then $\sigma_\alpha(U_x)$ and
$\sigma_\alpha(U_y)$ are disjoint and contain $x$ and $y$.


\section{Cohomology}

%%% Local Variables:
%%% mode: latex
%%% TeX-master: "../Sp16-Notes"
%%% End:
