\def\auth{Carlos Salinas}
\def\tight{MA553: Qual Preparation}
\def\short{MA553 Qual Prep}
\def\class{MA553}
\def\subject{abstract algebra}
\def\email{salinac@purdue.edu}

\documentclass[10pt,showtrims,twoside]{article}
% \usepackage{geometry}
\usepackage[dvipsnames]{xcolor}%
\usepackage[%
breaklinks,%
colorlinks=true,%
linkcolor=black,%
citecolor=black,%
filecolor=black,%
menucolor=black,%
runcolor=black,%
urlcolor=black,%
pdftitle={\short},%
pdfauthor={\auth},%
pdfkeywords={\subject},%
pdfsubject={\class},%
pageanchor={false}%
]{hyperref}
% \usepackage{natbib}

% TOC depth
% \setsecnumdepth{subsection}

%% Math and variables all bundled up
\usepackage{carlos-variables}

%% Misc
\usepackage{microtype}
\usepackage[inline]{enumitem}
\usepackage{cleveref}
\usepackage{graphicx}
\graphicspath{{figures/}}
\usepackage{mleftright}
\mleftright

\begin{document}

%% Footnote style
\renewcommand*{\thefootnote}{\fnsymbol{footnote}}

%% Counters
% \counterwithout{exercise}{chapter}
% \numberwithin{equation}{subsection}
% \counterwithout{equation}{chapter}

\thispagestyle{empty}
\author{\href{mailto:\email}{\auth}}
\title{\tight}
\date{\today}

% \frontmatter
\maketitle
\tableofcontents

%% Ulrich homework
% \mainmatter
\section{MA 553 Spring 2016}
% \thispagestyle{empty}
This is material from the course MA 533 as it was taught in the spring of
2016.%
\bigskip
\subsection{Homework}
Most of the homework is Ulrich original (or as original as elementary
exercises in abstract algebra can be). However, an excellent resource and
one that I will often quote on these solutions is \cite{hungerford}. Other
resources include \cite{dummit-foote} and (to a lesser extent)
\cite{herstein}. I may also cite Milne's \emph{Group Theory}, \emph{Field
  Theory}, and \emph{Commutative Algebra: A Primer} notes, respectively,
\cite{milneGT}, \cite{milneFT}, and (no reference for the last). Unless
otherwise stated, whenever we quote a result, e.g., Theorem 1.1, it is
understood to come from Hungerford's \emph{Algebra}.

Throughout these notes

\begin{tabular}{cl}
  $\bbR$   & is the set of real numbers\\
  $\bbC$   & is the set of complex numbers\\
  $\bbQ$   & is the set of rational numbers\\
  $\bbF_q$ & is the finite field of order $q=p^n$ for some prime $p$\\
  $\bbZ$   & is the set of the integers\\
  $\bbN$   & is the set of the natural numbers $1,2,\dotsc$\\
  $k$   & is used to denote the base field with
             characteristic $\Char k$\\
  $K,E,L$& is used to denote field extensions over the base field
                    $k$\\
  $Z_n$    & is the cyclic group of order $n$ not necessarily equal
             (but isomorphic) to $\bbZ/p\bbZ$\\
  $S_n$    & is the symmetric group on $\{1,\dotsc,n\}$\\
  $A_n$    & is the alternating group on $\{1,\dotsc,n\}$\\
  $D_n$    & is the dihedral group of order $n$\\
  $A\smallsetminus B$ & is the set difference of $A$ and $B$, that is, the
                        complement of $A\cap B$ in $A$\\
  $X\simeq Y$ & means $X$ and $Y$ are isomorphic as groups, rings,
                $R$-modules, or fields
\end{tabular}

\newpage

\subsubsection{Homework 1}
\setcounter{exercise}{0}
\begin{problem}
  Let $G$ be a group, $a\in G$ an element of finite order $m$, and $n$ a
  positive integer. Prove that
  \[
    |a^n|=\frac{m}{(m,n)}.
  \]
\end{problem}
\begin{solution}
  Let $\ell$ denote the order of $a^n$. Then $\ell$ is the minimal power of
  $a^n$ such that ${(a^n)}^\ell=e$. Now, observe that
  \begin{equation}
    \label{eq:1:1}
    \begin{aligned}
      {(a^n)}^{m/(m,n)}
      &=a^{nm/(m,n)}\\
      &=a^{mn/(m,n)}\\
      &={(a^m)}^{n/(m,n)}\\
      &=e^{n/(m,n)}\\
      &=e.
    \end{aligned}
  \end{equation}
  Thus $\ell\leq m/(m,n)$.

  On the other hand, by Theorem 3.4 (iv) from \cite[Ch.\@ I \S 3.3, p.\@
  35]{hungerford} since ${(a^n)}^\ell=a^{n\ell}=e$ and the order of $a$ is
  $m$, $m\mid n\ell$ or, equivalently, $mk=n\ell$ for some
  $k\in\bbZ^+$. Now, since $(m,n)\mid m$ and $(m,n)\mid n$, we can
  represent $m$ and $n$ as the products $(m,n)m'$ and $(m,n)n'$,
  respectively. Now, note that $m'=m/(n,m)$ so we must show that
  $m'\leq\ell$. Putting all of this together, we have $mk$
  \begin{equation}
    \label{eq:1:2}
    mk=(m,n)m'k=(m,n)n'\ell=n\ell
  \end{equation}
  so
  \begin{equation}
    \label{eq:1:3}
    m'k=n'\ell.
  \end{equation}
  Thus $m'\mid n'\ell$ so either $m'\mid n'$ or $m'\mid\ell$. But since we
  factored the $(m,n)$ from $m$ and $n$, it follows that $(m',n')=1$ so
  $m'\mid \ell$. Therefore $m'\leq\ell$ and equality holds, that is,
  $\ell=m/(m,n)$.
\end{solution}

\begin{problem}
  Let $G$ be a group, and let $a$, $b$ be elements of finite order $m$, $n$
  respectively. Show that if $ba=ab$ and
  $\langle a\rangle\cap\langle b\rangle=\{e\}$, then $|ab|=mn/(m,n)$.
\end{problem}
\begin{solution}
  Let $\ell$ denote the order of $ab$. Now, playing around with powers of
  $ab$, we have
  \begin{equation}
    \label{eq:1:4}
    \begin{aligned}
      (ab)^{n}
      &=a^nb^n\\
      &=a^n\\
      &\neq e
    \end{aligned}
  \end{equation}
  since the order of $a$ is $m$ and $n<m$. Thus, by Problem 1,
  $|a^n|=m/(m,n)$ so $|ab|=mn/(m,n)$.
\end{solution}

\begin{problem}
  Let $G$ be a group and $H$, $K$ normal subgroups with $H\cap
  K=\{e\}$. Show that
  \begin{enumerate}[label=(\alph*),noitemsep]
  \item $hk=kh$ for every $h\in H$, $k\in K$.
  \item $HK$ is a subgroup of $G$ with $HK\simeq H\times K$.
  \end{enumerate}
\end{problem}
\begin{solution}
  (a) Suppose that $H$ and $K$ are normal in $G$. Then, for every $g\in G$,
  $gh=hg$ and $gk=kg$ for any $h\in H$, $k\in K$. In particular, since
  $H\subset G$, $h\in G$ so $hk=kh$.
  \\\\
  (b) Consider the subset $HK$ of $G$ consisting of all products $hk$ where
  $h\in H$, $k\in K$. First, we show that $HK$ is closed under
  multiplication: Pick $h_1k_1,h_2k_2\in HK$ then
  $h_1k_1h_2k_2=h_1(k_1k_2)h_2=h_1h_2(k_1k_2)$ is in $HK$ since $h_1h_2\in
  H$, $k_1k_2\in K$. Moreover, since $e\in H$ and $e\in K$, $ee=e\in
  HK$. Lastly, given $hk\in HK$,
  $hkh^{-1}k^{-1}=(hkh^{-1})k^{-1}=kk^{-1}=e$ so $HK$ is closed under
  taking inverses. Thus, $HK$ is a subgroup of $G$.

  To see that $HK\simeq H\times K$, consider the map
  $\varphi\colon HK\to (HK/K)\times(HK/H)$ given by
  $\varphi(hk)\coloneq(\pi_K(h),\pi_H(k))$ where $\pi_H\colon HK\to HK/H$
  and $\pi_K\colon HK\to HK/K$ are quotient maps. By the first (or second)
  isomorphism theorem, $H\simeq HK/H$ and $K\simeq HK/H$ so $HK\simeq
  H\times K$.
\end{solution}

\begin{problem}
  Show that $A_4$ has no subgroup of order $6$ (although $6\mid 12=|A_4|$).
\end{problem}
\begin{solution}
  Suppose that $6$ has a subgroup of order $6$ say $H$. Then, by Cauchy's
  theorem, $H$ contains an element $h$ of order $2$, $3$ or $6$. If the
  order of $h$ is $6$, $H$ must be cyclic and hence normal in $A_4$. But
  $A_4$ is simple. If $h$ has order $2$, then the subgroup generated by $h$
  is normal in $G$ which, as we previously mentioned, is
  impossible. Lastly, if $h$ has order $3$ then $h$ must be a product of
  disjoint $3$ cycles.
\end{solution}

%%% Local Variables:
%%% mode: latex
%%% TeX-master: "../MA553-Quals"
%%% End:

\setcounter{exercise}{0}
\subsubsection{Homework 2}
\setcounter{exercise}{0}
\setcounter{equation}{0}

\begin{problem}
  Let $G$ be the group of order $2^n\cdot 3$, $n\geq 2$. Show that $G$ has
  a normal $2$-subgroup $\neq\left\{e\right\}$.
\end{problem}
\begin{solution}
  Suppose that $\card G=2^n\cdot 3$. By the first Sylow theorem, $G$
  contains a $2$-Sylow subgroup, i.e., a subgroup $P$ of order
  $\card P=2^3$; this is, by Corollary 5.3, a $2$-subgroup. Now, by
  Corollary 5.8 (iii), it suffices to show that $P$ is the only $2$-Sylow
  subgroup. By The third Sylow theorem, the number of $2$-Sylow subgroups
  $n_2$ is $n_2\equiv 1\mod 2$ so either $n_2=1$ or $n_2=3$.

  Suppose that $n_2=3$. The
\end{solution}

\begin{problem}
  Let $G$ be a group of order $p^2q$, $p$ and $q$ primes. Show that the
  Sylow $p$-Sylow subgroup or the $q$-Sylow subgroup of $G$ is normal in
  $G$.
\end{problem}
\begin{solution}
\end{solution}

\begin{problem}
  Let $G$ be a subgroup of order $pqr$, $p<q<r$ primes. Show that the
  $r$-Sylow subgroup of $G$ is normal in $G$.
\end{problem}
\begin{solution}
\end{solution}

\begin{problem}
  Let $G$ be a group of order $n$ and let $\varphi\colon G\to S_n$ be given
  by the action of $G$ on $G$ via translation.
  \begin{enumerate}[label=(\alph*),noitemsep]
  \item For $a\in G$ determine the number and the lengths of the disjoint
    cycles of the permutation $\varphi(a)$.
  \item Show that $\varphi(G)\nsubset A_n$ if and only if $n$ is even and
    $G$ has a cyclic $2$-Sylow subgroup.
  \item If $n=2m$, $m$ odd, show that $G$ has a subgroup of index $2$.
  \end{enumerate}
\end{problem}
\begin{solution}
\end{solution}

\begin{problem}
  Show that the only simple groups $\neq\left\{e\right\}$ of order $<60$
  are the groups of prime order.
\end{problem}
\begin{solution}
\end{solution}

%%% Local Variables:
%%% mode: latex
%%% TeX-master: "../MA553-Quals"
%%% End:

\setcounter{exercise}{0}
\include{ulrich/553-hw-3}
\setcounter{exercise}{0}
\subsection{Homework 4}
\begin{problem}
Let $p$ be a prime and let $G$ be a nonAbelian group of order $p^3$. Show
that $G'=Z(G)$.
\end{problem}
\begin{proof}
\end{proof}

\begin{problem}
Let $p$ be an odd prime and let $G$ be a nonAbelian group of order $p^3$
having an element of order $p^2$. Show that there exists an element
$b\notin\langle a \rangle$ of order $p$.
\end{problem}
\begin{proof}
\end{proof}

\begin{problem}
Let $p$ be an odd prime. Determine all groups of order $p^3$.
\end{problem}
\begin{proof}
\end{proof}

\begin{problem}
Show that $(S_n)'=A_n$.
\end{problem}
\begin{proof}
\end{proof}

\begin{problem}
Show that every group of order $<60$ is solvable.
\end{problem}
\begin{proof}
\end{proof}

\begin{problem}
Show that every group of order $60$ that is simple (or not solvable) is
isomorphic to $A_5$.
\end{problem}
\begin{proof}
\end{proof}

%%% Local Variables:
%%% mode: latex
%%% TeX-master: "../MA553-Quals"
%%% End:

\setcounter{exercise}{0}
\subsubsection{Homework 5}
\setcounter{exercise}{0}
\setcounter{equation}{0}

\begin{problem}
  Find all composition series and the composition factors of $D_6$.
\end{problem}
\begin{solution}
\end{solution}

\begin{problem}
  Let $T$ be the subgroup of $\GL(n,\bbR)$ consisting of all upper triangular
  invertible matrices. Show that $T$ is solvable.
\end{problem}
\begin{solution}
\end{solution}

\begin{problem}
  Let $p\in\bbZ$ be a prime number. Show:
  \begin{enumerate}[label=(\alph*),noitemsep]
  \item $(p-1)!\equiv -1\mod{p}$.
  \item If $p\equiv 1\mod{4}$ then $x^2\equiv-1\mod{p}$ for some
    $x\in\bbZ$.
  \end{enumerate}
\end{problem}
\begin{solution}
\end{solution}

\begin{problem}
  \hfill
  \begin{enumerate}[label=(\alph*),noitemsep]
  \item Show that the following are equivalent for an odd prime number
    $p\in\bbZ$:
    \begin{enumerate}[label=(\roman*),noitemsep]
    \item $p\equiv 1\mod 4$.
    \item $p=a^2+b^2$ for some $a$, $b$ in $\bbZ$.
    \item $p$ is not prime in $\bbZ[i]$.
    \end{enumerate}
  \item Determine all prime ideals of $\bbZ[i]$.
  \end{enumerate}
\end{problem}
\begin{solution}
\end{solution}

%%% Local Variables:
%%% mode: latex
%%% TeX-master: "../MA553-Quals"
%%% End:

\setcounter{exercise}{0}
\subsubsection{Homework 6}
\setcounter{exercise}{0}
\setcounter{equation}{0}

\begin{problem}
  Let \(R\) be a domain. Show that \(R\) is a u.f.d.\@ if and only if every
  nonzero nonunit in \(R\) is a product of irreducible elements and the
  intersection of any two principal ideals is again principal.
\end{problem}
\begin{solution}
\end{solution}

\begin{problem}
  Let \(R\) be a p.i.d.\@ and \(\frakp\) a prime ideal of \(R[X]\). Show
  that \(\frakp\) is principal or \(\frakp=(a,f)\) for some \(a\in R\) and
  some monic polynomial \(f\in R[X]\).
\end{problem}
\begin{solution}
\end{solution}

\begin{problem}
  Let \(k\) be a field and \(n\geq 1\). Show that
  \(Z^n+Y^3+X^2\in k(X,Y)[Z]\) is irreducible.
\end{problem}
\begin{solution}
\end{solution}

\begin{problem}
  Let \(k\) be a field of characteristic zero and \(n\geq 1\), \(m\geq
  2\). Show that \({X_1}^{\cramped{n}}+\dotsb+{X_m}^{\cramped{n}}-1\in
  k[X_1,\dotsc,X_m]\) is
  irreducible.
\end{problem}
\begin{solution}
\end{solution}

\begin{problem}
  Show that \(X^{3^n}+2\in\bbQ(\rmi)[X]\) is irreducible.
\end{problem}
\begin{solution}
\end{solution}

%%% Local Variables:
%%% mode: latex
%%% TeX-master: "../MA553-Quals"
%%% End:

\setcounter{exercise}{0}
\subsection{Homework 7}
\begin{problem}
  Let $\bbk\subset\bbK$ and $\bbk\subset\bbL$ be finite field extensions
  contained in some field. Show that:
  \begin{enumerate}[label=(\alph*),noitemsep]
  \item $[\bbK\bbL:\bbL]\leq [\bbK:\bbk]$.
  \item $[\bbK\bbL:\bbk]\leq[\bbK:\bbk][\bbL:\bbk]$.
  \item $\bbK\cap\bbL=\bbk$ if equality holds in (b).
  \end{enumerate}
\end{problem}
\begin{proof}
\end{proof}

\begin{problem}
  Let $\bbk$ be a field of characteristic $\neq 2$ and $a,b$ elements of
  $\bbk$ so that $a$, $b$, $ab$ are not squares in $\bbk$. Show that
  $\left[\bbk{\bigl(\sqrt{a},\sqrt{b}\bigr)}:\bbk\right]=4$.
\end{problem}
\begin{proof}
\end{proof}

\begin{problem}
  Let $R$ be a UFD, but not a field, and write $\bbK\coloneq\Quot(R)$. Show
  that $[\bar\bbK:\bbk]=\infty$.
\end{problem}
\begin{proof}
\end{proof}

\begin{problem}
  Let $\bbk\in\bbK$ be an algebraic field extension. Show that every
  $\bbk$-homomorphism $\delta\colon\bbK\to\bbK$ is an isomorphism.
\end{problem}
\begin{proof}
\end{proof}

\begin{problem}
  Let $\bbK$ be the splitting field of $X^6-4$ over $\bbQ$. Determine
  $\bbK$ and $[\bbK:\bbQ]$.
\end{problem}
\begin{proof}
\end{proof}

%%% Local Variables:
%%% mode: latex
%%% TeX-master: "../MA553-Quals"
%%% End:

\setcounter{exercise}{0}
\subsection{Homework 8}

%%% Local Variables:
%%% mode: latex
%%% TeX-master: "../MA553-Quals"
%%% End:

\setcounter{exercise}{0}
\subsection{Homework 9}
\begin{problem}
  Let $k\subset K$ be a finite extension of fields of characteristic
  $p>0$. Show that if $p\nmid [K:k]$, then $k\subset K$ is separable.
\end{problem}
\begin{proof}
\end{proof}

\begin{problem}
  Let $k\subset K$ be an algebraic extension of fields of characteristic
  $p>0$, let $L$ be an algebraically closed field containing $K$, and let
  $\delta\colon k\to L$ be an embedding. Show that $k\subset K$ is purely
  inseparable if and only if there exists exactly one embedding
  $\tau\colon K\to L$ extending $\delta$.
\end{problem}
\begin{proof}
\end{proof}

\begin{problem}
  Let $k\subset K=k(\alpha,\beta)$ be an algebraic extension of fields of
  characteristic $p>0$, where $\alpha$ is separable over $k$ and $\beta$ is
  purely inseparable over $k$. Show that $K=k(\alpha+\beta)$.
\end{problem}
\begin{proof}
\end{proof}

\begin{problem}
  Let $f(X)\in\bbF_q[X]$ be irreducible. Show that $f(X)\mid X^{q^n}-X$ if
  and only if $\deg f(X)\mid n$.
\end{problem}
\begin{proof}
\end{proof}

\begin{problem}
  Show that $\Aut_{\bbF_q}(\bar\bbF_q)$ is an infinite Abelian group which
  is torsionfree (i.e., $\delta^n=\id$ implies $\delta=\id$ or $n=0$).
\end{problem}
\begin{proof}
\end{proof}

\begin{problem}
  Show that in a finite field, every element can be written as a sum of two
  perfect squares.
\end{problem}
\begin{proof}
\end{proof}

%%% Local Variables:
%%% mode: latex
%%% TeX-master: "../MA553-Quals"
%%% End:

\setcounter{exercise}{0}
\subsection{Homework 10}
\begin{problem}
  Let $\bbk\subset\bbK\coloneq\bbk(\alpha)$ be a simple field extension,
  let $G\coloneq\{\delta_1,\dotsc,\delta_n\}$ be a finite subgroup of
  $\Aut_\bbk(\bbK)$, and write
  $f(X)\coloneq\prod_{i=1}^n(X-\delta_i(\alpha))=\sum_{i=0}^na_iX^i$. Show
  that $f(X)$ is the minimal polynomial of $\alpha$ over $\bbK^2$ and that
  $\bbK^G=\bbk(a_0,\dotsc,a_{n-1})$.
\end{problem}
\begin{proof}
\end{proof}

\begin{problem}
  Let $\bbk$ be a field, $\bbk(X)$ the field of rational functions, and
  $u\in \bbk(X)\smallsetminus\bbk$. Write $u\coloneq f/g$ with $f$ and $g$
  relatively prime in $\bbk[X]$. Show that
  $[\bbk(X):\bbk(u)]=\max\{\deg f,\deg g\}$.
\end{problem}
\begin{proof}
\end{proof}

\begin{problem}
  Let $\bbk$ be a field and $\bbK\coloneq\bbk(X)$ the field of rational
  functions. Show that for every $\delta\in\Aut_\bbk(\bbK)$,
  $\delta(X)\coloneq (aX+b)/(cX+d)$ for some $a$, $b$, $c$, $d$ in $\bbk$
  with $ad-bc\neq 0$, and that conversely, every such rational functions
  uniquely determines an automorphism $\delta\in\Aut_\bbk(\bbK)$.
\end{problem}
\begin{proof}
\end{proof}

\begin{problem}
With the notion of the previous problem let $\delta\in\Aut_\bbk(\bbK)$ and
$G\coloneq\langle \delta \rangle$.
\begin{enumerate}[label=(\alph*),noitemsep]
\item Assume $\delta(X)=1/(1-X)$. Show that $|G|=3$ and determine $\bbK^G$.
\item Assume $\Char\bbk=0$ and $\delta(X)=X+1$. Show that $G$ is infinite and
  determine $\bbK^G$.
\end{enumerate}
\end{problem}
\begin{proof}
\end{proof}

\begin{problem}
  Let $\bbk\subset\bbK$ be a finite Galois extension with
  $G\coloneq\Gal(\bbK/\bbk)$, let $\bbL$ be a subfield of $\bbK$ containing
  $\bbk$ with $H\coloneq\Gal(\bbK/\bbL)$, and let $\bbL'$ be the compositum
  in $\bbK$ of the fields $\delta(\bbL)$, $\delta\in G$. Show that:
\begin{enumerate}[label=(\alph*),noitemsep]
\item $\bbL'$ is the unique smallest subfield of $\bbK$ that contains
  $\bbL$ and is Galois over $\bbk$.
\item $\Gal(\bbK/\bbL')=\bigcap_{\delta\in G}\delta H\delta^{-1}$.
\end{enumerate}
\end{problem}
\begin{proof}
\end{proof}

%%% Local Variables:
%%% mode: latex
%%% TeX-master: "../MA553-Quals"
%%% End:

\setcounter{exercise}{0}
\subsection{Homework 11}

%%% Local Variables:
%%% mode: latex
%%% TeX-master: "../MA553-Quals"
%%% End:

\setcounter{exercise}{0}
\subsubsection{Homework 12}
\setcounter{exercise}{0}
\setcounter{equation}{0}

\begin{problem}
  Prove that the resolvent cubic \(X^4+aX^2+bX+c\) is given by
  \(X^3-aX^2-4cX+4ac-b^2\).
\end{problem}
\begin{solution}
\end{solution}

\begin{problem}
  Show that the general polynomial \(g(Y)=Y^n+u_1Y^{n-1}+\dotsb+u_n\) is
  irreducible in \(k(u_1,\dotsc,u_n)[Y]\).
\end{problem}
\begin{solution}
\end{solution}

\begin{problem}
  Let \(k\) be a field.
  \begin{enumerate}[label=(\alph*)]
  \item Compute the discriminant \(Y^3-Y\in k[Y]\) and \(Y^3-1\in k[Y]\).
  \item Show that the discriminant of the polynomial
    \((Y-X_1)(Y-X_2)(Y-X_3)\) over \(k(X_1,X_2,X_3)\) is of the form
    \[
      \lambda_1{s_1}^{\cramped{4}}+
      \lambda_2{s_1}^{\cramped{4}}s_2+
      \lambda_3{s_1}^{\cramped{3}}s_3+
      \lambda_4{s_1}^{\cramped{2}}{s_2}^{\cramped{2}}+
      \lambda_5s_1s_2s_3+\lambda_6{s_2}^{\cramped{3}}+
      \lambda_7{s_3}^{\cramped{2}}
    \]
    with \(\lambda_i\in k\).
  \item From (b) and (a) conclude that the discriminant
    \(Y^3+aY+b\in k[Y]\) is \(-4a^3-27b^2\).
\end{enumerate}
\end{problem}
\begin{solution}
\end{solution}

\begin{problem}
  Let \(\Phi_n(X)\) be the \(n\)th cyclotomic polynomial over \(\bbQ\).
  \begin{enumerate}[label=(\alph*)]
  \item Let \(n={p_1}^{\cramped{r_1}}\dotsm{p_s}^{\cramped{r_s}}\) with \(p_i\) distinct prime
    numbers and \(r_i>0\). Show that
    \(\Phi(X)=\Phi_{p_1\dotsm p_s}\bigl(X^{{p_1}^{\cramped{r_1-1}}
      \dotsm{p_s}^{\cramped{r_s-1}}}\bigr)\).
  \item For a prime number \(p\) with \(p\nmid n\) show that
    \(\Phi_{pn}(X)=\Phi_n(X^p)/\Phi_n(X)\).
  \end{enumerate}
\end{problem}
\begin{solution}
\end{solution}

%%% Local Variables:
%%% mode: latex
%%% TeX-master: "../MA553-Quals"
%%% End:

\setcounter{exercise}{0}
\subsection{Homework 13}
\begin{problem}
Let $n\geq 3$ and $\rho$ a primitive $n$th root of unity over $\bbQ$. Show
that $[\bbQ(\rho+\rho^{-1}):\bbQ]=\varphi(n)/2$.
\end{problem}
\begin{proof}
\end{proof}

\begin{problem}
Let $\rho$ be a primitive $n$th root of unity over $\bbQ$. Determine all
$n$ so that $\bbQ\subset\bbQ(\rho)$ is cyclic.
\end{problem}
\begin{proof}
\end{proof}

\begin{problem}
  Let $\bbk\subset\bbK$ be an extension of finite fields. Show that
  $\Norm_\bbk^\bbK$ and $\Tr_\bbk^\bbK$ are surjective maps from $\bbK$ to
  $\bbk$.
\end{problem}
\begin{proof}
\end{proof}

\begin{problem}
  Let $f(X)\in\bbk[X]$ be a separable polynomial of degree $n\geq 3$ with
  Galois group isomorphic to $S_n$, and let $\alpha\in\bar\bbk$ be a root
  of $f(X)$.
  \begin{enumerate}[label=(\alph*),noitemsep]
  \item Show that $f(X)$ is irreducible.
  \item Show that $\Aut_\bbk(\bbk(\alpha))=\{\Id\}$.
  \item Show that $\alpha^n\notin\bbk$ if $n\geq 4$.
  \end{enumerate}
\end{problem}
\begin{proof}
\end{proof}

\begin{problem}
  Let $\bbk\subset\bbK$ be a Galois extension.
\begin{enumerate}[label=(\alph*),noitemsep]
\item For $\bbk\subset\bbL\subset\bbK$ show that $\Gal(\bbK/\bbL)$ is
  solvable if $\Gal(\bbK/\bbk)$ is solvable.
\item For $\bbk\subset\bbL\subset\bbK$ with $\bbk\subset\bbL$ normal show
  that $\Gal(\bbL/\bbk)$ and $\Gal(\bbK/\bbL)$ are solvable if and only if
  $\Gal(\bbK/\bbk)$ is solvable.
\item For $\bbk\subset\bbL$ with $\bbK$ and $\bbL$ in a common field show
  that $\Gal(\bbK\bbL/\bbL)$ is solvable if $\Gal(\bbK/\bbk)$ is solvable.
\end{enumerate}
\end{problem}
\begin{proof}
\end{proof}

%%% Local Variables:
%%% mode: latex
%%% TeX-master: "../MA553-Quals"
%%% End:


%% Ulrich exams

%% Ulrich quals

%% Misc quals

%% References
% \backmatter
\bibliographystyle{alpha}
\bibliography{alg-bib}
% \printindex
\end{document}

%%% Local Variables:
%%% mode: latex
%%% TeX-master: t
%%% End:
