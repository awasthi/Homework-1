\def\documentauthor{Carlos Salinas}
\def\documenttitle{MA553: Qual Preparation}
\def\shorttitle{MA553 Qual Prep}
\def\coursename{MA553}
\def\documentsubject{abstract algebra}
\def\authoremail{salinac@purdue.edu}

\documentclass[10pt,showtrims,twoside]{memoir}
\usepackage{geometry}
\usepackage[dvipsnames]{xcolor}
\usepackage[
    breaklinks,
    colorlinks=true,
    linkcolor=black,
    citecolor=black,
    filecolor=black,
    menucolor=black,
    runcolor=black,
    urlcolor=black,
    pdftitle={\shorttitle},
    pdfauthor={\documentauthor},
    pdfkeywords={\documentsubject},
    pdfsubject={\coursename}
    ]{hyperref}
% \usepackage{natbib}

% TOC depth
\setsecnumdepth{subsection}

%% Math
\usepackage{amsmath}
\usepackage{amsfonts}
\usepackage{amssymb}
\usepackage{amsthm}
% \usepackage{mathtools}
% \usepackage{lmodern}
% \usepackage{eucal}

%% Language
% \usepackage[LAE,LFE,T2A,T1]{fontenc}
% \usepackage[utf8]{inputenc}
% \usepackage[farsi,french,german,spanish,russian,english]{babel}
% \babeltags{fr=french,
%            de=german,
%            en=english,
%            es=spanish,
%            pa=farsi,
%            ru=russian
%            }
% \def\spanishoptions{mexico}

% \selectlanguage{english}

% \newcommand{\textfa}[1]{\beginR\textpa{#1}\endR}

% \usepackage{CJKutf8}
% \newcommand{\textkr}[1]{\begin{CJK}{UTF8}{mj}#1\end{CJK}}
% \newcommand{\textjp}[1]{\begin{CJK}{UTF8}{min}#1\end{CJK}}
% \newcommand{\textzh}[1]{\begin{CJK}{UTF8}{bsmi}#1\end{CJK}}

%% Misc
\usepackage{graphicx}
\graphicspath{{figures/}}

\usepackage{microtype}
\usepackage{lineno}
\usepackage{multicol}
\usepackage[inline]{enumitem}
\usepackage{listings}
% \usepackage{mleftright}
% \mleftright
\usepackage{carlos-variables}

%% Unicode math
\usepackage{unicode-math}

% \setmainfont[Ligatures=TeX]{Latin Modern Roman}
% \setsansfont{Latin Modern Sans}
% \setsansfont{Latin Modern Mono}
% \setmathfont[BoldFont={latinmodern-math.otf}]{latinmodern-math.otf}

%% Theorems and definitions
\theoremstyle{plain}
\newtheorem{theorem}{Theorem}
\newtheorem{proposition}[theorem]{Proposition}
\newtheorem{corollary}[theorem]{Corollary}
\newtheorem{claim}[theorem]{Claim}
\newtheorem{lemma}[theorem]{Lemma}
\newtheorem{axiom}[theorem]{Axiom}

\newtheorem*{corollary*}{Corollary}
\newtheorem*{claim*}{Claim}
\newtheorem*{lemma*}{Lemma}
\newtheorem*{proposition*}{Proposition}
\newtheorem*{theorem*}{Theorem}

\theoremstyle{definition}
\newtheorem{definition}{Definition}
\newtheorem{example}{Examples}
\newtheorem{examples}[example]{Example}
\newtheorem{exercise}{Exercise}[subsection]
\newtheorem{problem}[exercise]{Problem}

\newtheorem*{example*}{Example}
\newtheorem*{exercise*}{Exercise}
\newtheorem*{problem*}{Problem}

\begin{document}

%% Footnote style
\renewcommand*{\thefootnote}{\fnsymbol{footnote}}

%% Counters
\counterwithout{exercise}{chapter}
\numberwithin{equation}{subsection}
\counterwithout{equation}{chapter}

\thispagestyle{empty}
\author{\href{mailto:\authoremail}{\documentauthor}}
\title{\documenttitle}
\date{\today}

\frontmatter
\maketitle
\tableofcontents

%% Ulrich homework
\mainmatter
\chapter{MA 553 Spring 2016}
\thispagestyle{empty}
This is material from the course MA 533 as it was taught in the spring of
2016.
\bigskip
\section{Homework}
Most of the homework is Ulrich original (or as original as elementary
exercises in abstract algebra can be). However, an excellent resource and
one that I will often quote on these solutions is \cite{hungerford}. Other
resources include \cite{dummit-foote} and (to a lesser extent)
\cite{herstein}. I may also cite Milne's \emph{Group Theory}, \emph{Field
  Theory}, and \emph{Commutative Algebra: A Primer} notes, respectively,
\cite{milneGT}, \cite{milneFT}, and (no reference for the last).

\begin{tabular}{cl}
  $\bbR$   & is the set of real numbers\\
  $\bbC$   & is the set of complex numbers\\
  $\bbQ$   & is the set of rational numbers\\
  $\bbF_q$ & is the finite field of order $q=p^n$ for some prime $p$\\
  $\bbZ$   & is the set of the integers\\
  $\bbN$   & is the set of the natural numbers $1,2,\dotsc$\\
  $C_n$    & is the cyclic group of order $n$ not necessarily equal
             (but isomorphic) to $\bbZ/p\bbZ$\\
  $S_n$    & is the symmetric group on $\{1,\dotsc,n\}$\\
  $A_n$    & is the alternating group on $\{1,\dotsc,n\}$\\
  $D_n$    & is the dihedral group of order $n$\\
  $A\smallsetminus B$ & is the set difference of $A$ and $B$, that is, the
                        complement of $A\cap B$ in $A$\\
  $X\simeq Y$ & means $X$ and $Y$ are isomorphic as groups, rings,
                $R$-modules, or fields
\end{tabular}

\newpage

\subsection{Homework 1}
\begin{problem}
  Let $G$ be a group, $a\in G$ an element of finite order $m$, and $n$ a
  positive integer. Prove that
  \[
    |a^n|=\frac{m}{\gcd(m,n)}.
  \]
\end{problem}
\begin{proof}
  Without loss of generality, we may assume $n<m$; otherwise, by the
  fundamental theorem of arithmetic, there exist $q$ and $r$ with $r<m$
  such that $n=qm+r$ so $a^n=a^{qm+r}=a^{qm}a^r=a^r$.
\end{proof}

\begin{problem}
  Let $G$ be a group, and let $a$, $b$ be elements of finite order $m$, $n$
  respectively. Show that if $ba=ab$ and
  $\langle a\rangle\cap\langle b\rangle=\{e\}$, then $|ab|=\lcm(m,n)$.
\end{problem}
\begin{proof}
\end{proof}

\begin{problem}
  Let $G$ be a group and $H$, $K$ normal subgroups with $H\cap
  K=\{e\}$. Show that
  \begin{enumerate}[label=(\alph*),noitemsep]
  \item $hk=kh$ for every $h\in H$, $k\in K$.
  \item $HK$ is a subgroup of $G$ with $HK\simeq H\times K$.
  \end{enumerate}
\end{problem}
\begin{proof}
\end{proof}

\begin{problem}
  Show that $A_4$ has no subgroup of order $6$ (although $6\mid 12=|A_4|$).
\end{problem}
\begin{proof}
\end{proof}

%%% Local Variables:
%%% mode: latex
%%% TeX-master: "../MA553-Quals"
%%% End:

\subsubsection{Homework 2}
\setcounter{exercise}{0}
\setcounter{equation}{0}

\begin{problem}
  Let \(G\) be the group of order \(2^n\cdot 3\), \(n\geq 2\). Show that
  \(G\) has a normal \(2\)-subgroup \(\neq\left\{e\right\}\).
\end{problem}
\begin{solution}
  Suppose that \(\card G=2^n\cdot 3\). By the first Sylow theorem, \(G\)
  contains a \(2\)-Sylow subgroup, i.e., a subgroup \(P\) of order
  \(\card P=2^3\); this is, by Corollary 5.3, a \(2\)-subgroup. Now, by
  Corollary 5.8 (iii), it suffices to show that \(P\) is the only
  \(2\)-Sylow subgroup. By The third Sylow theorem, the number of
  \(2\)-Sylow subgroups \(n_2\) is \(n_2\equiv 1\mod 2\) so either
  \(n_2=1\) or \(n_2=3\).

  Suppose that \(n_2=3\). The
\end{solution}

\begin{problem}
  Let \(G\) be a group of order \(p^2q\), \(p\) and \(q\) primes. Show that
  the Sylow \(p\)-Sylow subgroup or the \(q\)-Sylow subgroup of \(G\) is
  normal in \(G\).
\end{problem}
\begin{solution}
\end{solution}

\begin{problem}
  Let \(G\) be a subgroup of order \(pqr\), \(p<q<r\) primes. Show that the
  \(r\)-Sylow subgroup of \(G\) is normal in \(G\).
\end{problem}
\begin{solution}
\end{solution}

\begin{problem}
  Let \(G\) be a group of order \(n\) and let \(\varphi\colon G\to S_n\) be
  given by the action of \(G\) on \(G\) via translation.
  \begin{enumerate}[label=(\alph*),noitemsep]
  \item For \(a\in G\) determine the number and the lengths of the disjoint
    cycles of the permutation \(\varphi(a)\).
  \item Show that \(\varphi(G)\nsubset A_n\) if and only if \(n\) is even
    and \(G\) has a cyclic \(2\)-Sylow subgroup.
  \item If \(n=2m\), \(m\) odd, show that \(G\) has a subgroup of index
    \(2\).
  \end{enumerate}
\end{problem}
\begin{solution}
\end{solution}

\begin{problem}
  Show that the only simple groups \(\neq\left\{e\right\}\) of order
  \(<60\) are the groups of prime order.
\end{problem}
\begin{solution}
\end{solution}

%%% Local Variables:
%%% mode: latex
%%% TeX-master: "../MA553-Quals"
%%% End:

\subsubsection{Homework 3}
\setcounter{exercise}{0}
\setcounter{equation}{0}

\begin{problem}
  Let \(G\) be a finite group, \(p\) a prime number, \(N\) the
  intersubsection of all \(p\)-Sylow subgroups of \(G\). Show that \(N\) is
  a normal \(p\)-subgroup of \(G\) and that every normal \(p\)-subgroup of
  \(G\) is contained in \(N\).
\end{problem}
\begin{solution}
\end{solution}

\begin{problem}
  Let \(G\) be a group of order \(231\) and let \(H\) be an \(11\)-Sylow
  subgroup of \(G\). Show that \(H\subseteq Z(G)\).
\end{problem}
\begin{solution}
\end{solution}

\begin{problem}
  Let \(G=\left\{e,a_1,a_2,a_3\right\}\) be a non-cyclic group of order
  \(4\) and define \(\varphi\colon S_3\to\Aut(G)\) by
  \(\varphi(\sigma)(e)=e\) and \(\varphi(\sigma)(a_1)=a_{\sigma(i)}\). Show
  that \(\varphi\) is well-defined and an isomorphism of groups.
\end{problem}
\begin{solution}
\end{solution}

\begin{problem}
  Determine all groups of order \(18\).
\end{problem}
\begin{solution}
\end{solution}

%%% Local Variables:
%%% mode: latex
%%% TeX-master: "../MA553-Quals"
%%% End:

\subsubsection{Homework 4}
\begin{problem}
  Let $p$ be a prime and let $G$ be a nonAbelian group of order $p^3$. Show
  that $G'=Z(G)$.
\end{problem}
\begin{solution}
\end{solution}

\begin{problem}
  Let $p$ be an odd prime and let $G$ be a nonAbelian group of order $p^3$
  having an element of order $p^2$. Show that there exists an element
  $b\notin\langle a \rangle$ of order $p$.
\end{problem}
\begin{solution}
\end{solution}

\begin{problem}
  Let $p$ be an odd prime. Determine all groups of order $p^3$.
\end{problem}
\begin{solution}
\end{solution}

\begin{problem}
  Show that $(S_n)'=A_n$.
\end{problem}
\begin{solution}
\end{solution}

\begin{problem}
  Show that every group of order $<60$ is solvable.
\end{problem}
\begin{solution}
\end{solution}

\begin{problem}
  Show that every group of order $60$ that is simple (or not solvable) is
  isomorphic to $A_5$.
\end{problem}
\begin{solution}
\end{solution}

%%% Local Variables:
%%% mode: latex
%%% TeX-master: "../MA553-Quals"
%%% End:

\subsubsection{Homework 5}
\setcounter{exercise}{0}
\setcounter{equation}{0}

\begin{problem}
  Find all composition series and the composition factors of $D_6$.
\end{problem}
\begin{solution}
\end{solution}

\begin{problem}
  Let $T$ be the subgroup of $\GL(n,\bbR)$ consisting of all upper triangular
  invertible matrices. Show that $T$ is solvable.
\end{problem}
\begin{solution}
\end{solution}

\begin{problem}
  Let $p\in\bbZ$ be a prime number. Show:
  \begin{enumerate}[label=(\alph*),noitemsep]
  \item $(p-1)!\equiv -1\mod{p}$.
  \item If $p\equiv 1\mod{4}$ then $x^2\equiv-1\mod{p}$ for some
    $x\in\bbZ$.
  \end{enumerate}
\end{problem}
\begin{solution}
\end{solution}

\begin{problem}
  \begin{enumerate}[label=(\alph*),noitemsep]
  \item Show that the following are equivalent for an odd prime number
    $p\in\bbZ$:
    \begin{enumerate}[label=(\roman*),noitemsep]
    \item $p\equiv 1\mod 4$.
    \item $p=a^2+b^2$ for some $a$, $b$ in $\bbZ$.
    \item $p$ is not prime in $\bbZ[i]$.
    \end{enumerate}
  \item Determine all prime ideals of $\bbZ[i]$.
  \end{enumerate}
\end{problem}
\begin{solution}
\end{solution}

%%% Local Variables:
%%% mode: latex
%%% TeX-master: "../MA553-Quals"
%%% End:

\subsection{Homework 6}
\begin{problem}
Let $R$ be a domain. Show that $R$ is a UFD if and only if every nonzero
nonunit in $R$ is a product of irreducible elements and the intersection of
any two principal ideals is again principal.
\end{problem}
\begin{proof}
\end{proof}

\begin{problem}
Let $R$ be a PID and $\frakp$ a prime ideal of $R[X]$. Show that $\frakp$
is principal or $p=(a,f)$ for some $a\in R$ and some monic polynomial $f\in
R[X]$.
\end{problem}
\begin{proof}
\end{proof}

\begin{problem}
Let $k$ be a field and $n\geq 1$. Show that $Z^n+Y^3+X^2\in k(X,Y)[Z]$ is
irreducible.
\end{problem}
\begin{proof}
\end{proof}

\begin{problem}
Let $k$ be a field of characteristic zero and $n\geq 1$, $m\geq 2$. Show
that ${X_1}^n+\dotsb+{X_m}^n-1\in k[X_1,\dotsc,X_m]$ is irreducible.
\end{problem}
\begin{proof}
\end{proof}

\begin{problem}
Show that $X^{3^n}+2\in\bfQ(i)[X]$ is irreducible.
\end{problem}
\begin{proof}
\end{proof}

%%% Local Variables:
%%% mode: latex
%%% TeX-master: "../MA553-Quals"
%%% End:

\subsection{Homework 7}
\begin{problem}
Let $k\subset K$ and $k\subset L$ be finite field extensions contained in
some field. Show that:
\begin{enumerate}[label=(\alph*)]
\item $[KL:L]\leq [K:k]$.
\item $[KL:k]\leq [K:k][L:k]$.
\item $K\cap L=k$ if equality holds in (b).
\end{enumerate}
\end{problem}
\begin{proof}
\end{proof}

\begin{problem}
Let $k$ be a field of characteristic $\neq 2$ and $a,b$ elements of $k$ so
that $a$, $b$, $ab$ are not squares in $k$. Show that
$\bigl[k{(\sqrt{a},\sqrt{b})}:k\bigr]=4$.
\end{problem}
\begin{proof}
\end{proof}

\begin{problem}
Let $R$ be a UFD, but not a field, and write $K\coloneqq\Quot(R)$. Show that $[\bar
K:k]=\infty$.
\end{problem}
\begin{proof}
\end{proof}

\begin{problem}
Let $k\in K$ be an algebraic field extension. Show that every
$k$-homomorphism $\delta\colon K\to K$ is an isomorphism.
\end{problem}
\begin{proof}
\end{proof}

\begin{problem}
Let $K$ be the splitting field of $x^6-4$ over $\bfQ$. Determine $K$ and
$[K:\bfQ]$.
\end{problem}
\begin{proof}
\end{proof}

%%% Local Variables:
%%% mode: latex
%%% TeX-master: "../MA553-Quals"
%%% End:

\subsubsection{Homework 8}
\setcounter{exercise}{0}
\setcounter{equation}{0}

\begin{problem}
  Let \(k\) be a field, \(f\in k[X]\) is a polynomial of degree
  \(n\geq 1\), and \(K\) the splitting field of \(f\) over \(k\). Show that
  \([K:k]\mid n!\).
\end{problem}
\begin{solution}
\end{solution}

\begin{problem}
  Let \(k\) be a field and \(n\geq 0\). Define a map
  \(\Delta_n\colon k[X]\to k[X]\) by
  \(\Delta_n\bigl(\sum a_iX^i\bigr)=\sum a_i\binom{i}{n}X^{i-n}\). Show:
\begin{enumerate}[label=(\alph*),noitemsep]
\item \(\Delta_n\) is \(k\)-linear, and for \(f\), \(g\) in \(k[X]\),
  \(\Delta_n(fg)=\sum_{j=0}^n\Delta_j(f)\Delta_{n-j}(g)\);
\item \(f^{(n)}=n!\Delta_n(f)\);
\item \(f(X+a)=\sum\Delta_n(f)(a)X^n\), where \(a\in k\);
\item \(a\in k\) is a root of \(f\) of multiplicity \(n\) if and only if
  \(\Delta_i(f)(a)=0\) for \(0\leq i\leq n-1\) and
  \(\Delta_n(f)(a)\neq 0\).
\end{enumerate}
\end{problem}
\begin{solution}
\end{solution}

\begin{problem}
  Let \(k\subseteq K\) be a finite filed extension. Show that \(k\) is
  perfect if and only if \(K\) is perfect.
\end{problem}
\begin{solution}
\end{solution}

\begin{problem}
  Let \(K\) be the splitting field of \(X^p-X-1\) over
  \(k=\bbZ/p\bbZ\). Show that \(k\subseteq K\) is normal, separable, of
  degree \(p\).
\end{problem}
\begin{solution}
\end{solution}

\begin{problem}
  Let \(k\) be a field of characteristic \(p>0\), and \(k(X,Y)\) the field
  of rational functions in two variables.
\begin{enumerate}[label=(\alph*),noitemsep]
\item Show that \(\left[k(X,Y):k(X^p,Y^p)\right]=p^2\).
\item Show that the extension \(k(X^p,Y^p)\subseteq k(X,Y)\) is not simple.
\item Find infinitely many distinct fields \(L\) with
  \(k(X^p,Y^p)\subseteq L\subseteq k(X,Y)\).
\end{enumerate}
\end{problem}
\begin{solution}
\end{solution}

%%% Local Variables:
%%% mode: latex
%%% TeX-master: "../MA553-Quals"
%%% End:

\subsection{Homework 9}
\begin{problem}
  Let $k\subset K$ be a finite extension of fields of characteristic
  $p>0$. Show that if $p\nmid[K:k]$, then $k\subset K$ is separable.
\end{problem}
\begin{solution}
\end{solution}

\begin{problem}
  Let $k\subset K$ be an algebraic extension of fields of characteristic
  $p>0$, let $L$ be an algebraically closed field containing $K$, and let
  $\delta\colon k\to L$ be an embedding. Show that $k\subset K$ is purely
  inseparable if and only if there exists exactly one embedding
  $\tau\colon K\to L$ extending $\delta$.
\end{problem}
\begin{solution}
\end{solution}

\begin{problem}
  Let $k\subset K=k(\alpha,\beta)$ be an algebraic extension of fields of
  characteristic $p>0$, where $\alpha$ is separable over $k$ and $\beta$ is
  purely inseparable over $k$. Show that $K=k(\alpha+\beta)$.
\end{problem}
\begin{solution}
\end{solution}

\begin{problem}
  Let $f(X)\in\bbF_q[X]$ be irreducible. Show that $f(X)\mid X^{q^n}-X$ if
  and only if $\deg f(X)\mid n$.
\end{problem}
\begin{solution}
\end{solution}

\begin{problem}
  Show that $\Aut_{\bbF_q}(\bar\bbF_q)$ is an infinite Abelian group which
  is torsionfree (i.e., $\delta^n=\id$ implies $\delta=\id$ or $n=0$).
\end{problem}
\begin{solution}
\end{solution}

\begin{problem}
  Show that in a finite field, every element can be written as a sum of two
  perfect squares.
\end{problem}
\begin{solution}
\end{solution}

%%% Local Variables:
%%% mode: latex
%%% TeX-master: "../MA553-Quals"
%%% End:

\chapter{Homework 10}

%%% Local Variables:
%%% mode: latex
%%% TeX-master: "../MA553-Quals"
%%% End:

\subsection{Homework 11}
\begin{problem}
  Show that every algebraic extension of a finite field is Galois and
  Abelian.
\end{problem}
\begin{proof}
\end{proof}

\begin{problem}
  Let $k$ be a field of characteristic $\neq 2$ and $f(X)\ink[X]$ a
  cubic whose discriminant is a square. Show that $f$ is either irreducible
  or a product of linear polynomials in $k[X]$.
\end{problem}
\begin{proof}
\end{proof}

\begin{problem}
  Let $k$ be a field of characteristic $\neq 2$, and let
  $f(X)\coloneq X^4+aX^2+b\ink[X]$ be irreducible with Galois group
  $G$. Show:
  \begin{enumerate}[label=(\roman*),noitemsep]
  \item If $b$ is a square in $k$, then $G=H$.
  \item If $b$ is not a square in $k$, but $b(a^2-4b)$ is, then
    $G\simeq C_4$.
  \item If neither $b$ nor $b(a^2-4b)$ is a square in $k$, then
    $G\simeq D_4$.
  \end{enumerate}
\end{problem}
\begin{proof}
\end{proof}

\begin{problem}
  Determine the Galois group of:
  \begin{enumerate}[label=(\alph*),noitemsep]
  \item $X^4-5$ over $\bbQ$, over $\bbQ(\sqrt{5})$, over $\bbQ(\sqrt{-5})$;
  \item $X^3-10$ over $\bbQ$;
  \item $X^4-4X^2+5$ over $\bbQ$;
  \item $X^4+3X^3+3X-2$ over $\bbQ$;
  \item $X^4+2X^2+X+3$ over $\bbQ$.
  \end{enumerate}
\end{problem}
\begin{proof}
\end{proof}

\begin{problem}
  Let $K$ be the splitting field of $X^4-X^2-1$ over $\bbQ$. Determine
  all intermediate fields $L$, $\bbQ\subset L\subset K$. Which of
  these are Galois over $\bbQ$?
\end{problem}
\begin{proof}
\end{proof}

%%% Local Variables:
%%% mode: latex
%%% TeX-master: "../MA553-Quals"
%%% End:

\subsection{Homework 12}
\begin{problem}
Prove that the resolvent cubic $X^4+aX^2+bX+c$ is given by
$X^3-aX^2-4cX+4ac-b^2$.
\end{problem}
\begin{proof}
\end{proof}

\begin{problem}
Show that the general polynomial $g(Y)\coloneqq Y^n+u_1Y^{n-1}+\dotsb+u_n$
is irreducible in $k(u_1,\dotsc,u_n)[Y]$.
\end{problem}
\begin{proof}
\end{proof}

\begin{problem}
Let $k$ be a field.
\begin{enumerate}[label=(\alph*),noitemsep]
\item compute the discriminant $Y^3-Y\in k[Y]$ and $Y^3-1\in k[Y]$.
\item Show that the discriminant of the polynomial $(Y-X_1)(Y-X_2)(Y-X_3)$
  over $k(X_1,X_2,X_3)$ is of the form
  \[
    \lambda_1{s_1}^4+\lambda_2{s_1}^4s_2+\lambda_3{s_1}^3s_3+\lambda_4{s_1}^2{s_2}^2+\lambda_5s_1s_2s_3+\lambda_6{s_2}^3+\lambda_7{s_3}^2
  \]
  with $\lambda_i\in k$.
\item From (b) and (a) conclude that the discriminant $Y^3+aY+b\in k[Y]$ is
  $-4a^3-27b^2$.
\end{enumerate}
\end{problem}
\begin{proof}
\end{proof}

\begin{problem}
Let $\Phi_n(X)$ be the $n$th cyclotomic polynomial over $\bbQ$.
\begin{enumerate}[label=(\alph*),noitemsep]
\item Let $n={p_1}^{r_1}\dotsm{p_s}^{r_s}$ with $p_i$ distinct prime
  numbers and $r_i>0$. Show that $\Phi(X)=\Phi_{p_1\dotsm
    p_s}(X^{{p_1}^{r_1-1}\dotsm{p_s}^{r_s-1}})$.
\item For a prime number $p$ with $p\nmid n$ show that
  $\Phi_{pn}(X)=\Phi_n(X^p)/\Phi_n(X)$.
\end{enumerate}
\end{problem}
\begin{proof}
\end{proof}

%%% Local Variables:
%%% mode: latex
%%% TeX-master: "../MA553-Quals"
%%% End:

\subsection{Homework 13}
\begin{problem}
Let $n\geq 3$ and $\rho$ a primitive $n$th root of unity over $\bbQ$. Show
that $[\bbQ(\rho+\rho^{-1}):\bbQ]=\varphi(n)/2$.
\end{problem}
\begin{proof}
\end{proof}

\begin{problem}
Let $\rho$ be a primitive $n$th root of unity over $\bbQ$. Determine all
$n$ so that $\bbQ\subset\bbQ(\rho)$ is cyclic.
\end{problem}
\begin{proof}
\end{proof}

\begin{problem}
  Let $k\subset K$ be an extension of finite fields. Show that
  $\Norm_k^K$ and $\Tr_k^K$ are surjective maps from $K$ to
  $k$.
\end{problem}
\begin{proof}
\end{proof}

\begin{problem}
  Let $f(X)\in k[X]$ be a separable polynomial of degree $n\geq 3$ with
  Galois group isomorphic to $S_n$, and let $\alpha\in\bar k$ be a root
  of $f(X)$.
  \begin{enumerate}[label=(\alph*),noitemsep]
  \item Show that $f(X)$ is irreducible.
  \item Show that $\Aut_k(k(\alpha))=\{\id\}$.
  \item Show that $\alpha^n\notin k$ if $n\geq 4$.
  \end{enumerate}
\end{problem}
\begin{proof}
\end{proof}

\begin{problem}
  Let $k\subset K$ be a Galois extension.
\begin{enumerate}[label=(\alph*),noitemsep]
\item For $k\subset L\subset K$ show that $\Gal(K/L)$ is
  solvable if $\Gal(K/k)$ is solvable.
\item For $k\subset L\subset K$ with $k\subset L$ normal show
  that $\Gal(L/k)$ and $\Gal(K/L)$ are solvable if and only if
  $\Gal(K/k)$ is solvable.
\item For $k\subset L$ with $K$ and $L$ in a common field show
  that $\Gal(KL/L)$ is solvable if $\Gal(K/k)$ is solvable.
\end{enumerate}
\end{problem}
\begin{proof}
\end{proof}

%%% Local Variables:
%%% mode: latex
%%% TeX-master: "../MA553-Quals"
%%% End:


%% Ulrich exams

%% Ulrich quals

%% Misc quals

%% References
\backmatter
\bibliographystyle{alpha}
\bibliography{alg-bib}
% \printindex
\end{document}

%%% Local Variables:
%%% mode: latex
%%% TeX-master: t
%%% End:
