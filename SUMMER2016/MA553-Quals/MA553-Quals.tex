\def\documentauthor{Carlos Salinas}
\def\documenttitle{MA553: Qual Preparation}
% \def\hwnum{1}
\def\shorttitle{MA553 Qual Prep}
\def\coursename{MA553}
\def\documentsubject{abstract algebra}
\def\authoremail{salinac@purdue.edu}

\documentclass[10pt,showtrims,twoside]{memoir}
\usepackage{geometry}
\usepackage[dvipsnames]{xcolor}
\usepackage[
    breaklinks,
    bookmarks=true,
    colorlinks=true,
    pageanchor=false,
    linkcolor=black,
    citecolor=black,
    filecolor=black,
    menucolor=black,
    runcolor=black,
    urlcolor=black,
    % linkcolor=violet!85!black,
    % citecolor=YellowOrange!85!black,
    % urlcolor=Aquamarine!85!black,
    hyperindex=false,
    hyperfootnotes=true,
    pdftitle={\shorttitle},
    pdfauthor={\documentauthor},
    pdfkeywords={\documentsubject},
    pdfsubject={\coursename}
    ]{hyperref}
\usepackage{natbib}
\usepackage[toc,acronym,section=section]{glossaries}


%% Math
\usepackage{amsmath}
\usepackage{amsthm}
\usepackage{amssymb}
\usepackage{mathtools}
% \usepackage{eucal}
% \usepackage{mathrsfs}
% \usepackage[nointegrals]{wasysym}

% %% Language
% \usepackage{cmap}
% \usepackage[LAE,LFE,T2A,T1]{fontenc}
% \usepackage[utf8]{inputenc}
% \usepackage[farsi,french,german,spanish,russian,english]{babel}
% \babeltags{fr=french,
%            de=german,
%            en=english,
%            es=spanish,
%            pa=farsi,
%            ru=russian
%            }
% \def\spanishoptions{mexico}

% \selectlanguage{english}

% \newcommand{\textfa}[1]{\beginR\textpa{#1}\endR}

% \usepackage{CJKutf8}
% \newcommand{\textkr}[1]{\begin{CJK}{UTF8}{mj}#1\end{CJK}}
% \newcommand{\textjp}[1]{\begin{CJK}{UTF8}{min}#1\end{CJK}}
% \newcommand{\textzh}[1]{\begin{CJK}{UTF8}{bsmi}#1\end{CJK}}

% %% Misc
\usepackage{graphicx}
\graphicspath{{figures/}}

\usepackage{microtype}
\usepackage{lineno}
\usepackage{multicol}
\usepackage[inline]{enumitem}
\usepackage{listings}
\usepackage{mleftright}
\mleftright
\usepackage{carlos-variables}

%% Unicode math and Polyglossia
\usepackage{unicode-math}
% \usepackage{unicode-minionmath}

% \setmainfont{CMU Serif}
% \setsansfont{CMU Sans Serif}
% \setmonofont{CMU Typewriter Text}
\setmainfont[Ligatures=TeX]{Latin Modern Roman}
\setsansfont{Latin Modern Sans}
\setsansfont{Latin Modern Mono}
\setmathfont{Latin Modern Math}

% \setmainfont[Ligatures=TeX]{Libertinus Serif}
% \setsansfont{Libertinus Sans}
% \setmonofont{Libertinus Mono}
% \setmathfont{MinionMath-Regular.otf}
% \setmathfont[range={\mathfrak}]{latinmodern-math.otf}
% \setmathfont[range={\mathcal}]{latinmodern-math.otf}
% \setmathfont[range={\mathscr}]{latinmodern-math.otf}
% \setmathfont[range={}]{MinionMath-Regular.otf}

\usepackage{polyglossia}
\usepackage[english]{selnolig}

% \newfontfamily\cyrillicfont[Script=Cyrillic]{Libertinus Serif}
% \newfontfamily\cyrillicfontsf[Script=Cyrillic]{Libertinus Sans}

% \newfontfamily\farsifont[Script=Arabic,
%                          Scale=MatchUppercase]{Amiri}

\setmainlanguage[variant=american]{english}
% \setotherlanguage{farsi}
\setotherlanguage{french}
\setotherlanguage[spelling=new,latesthyphen,babelshorthands]{german}
\setotherlanguage{spanish}
\setotherlanguage[spelling=modern,babelshorthands]{russian}

% \makeatletter
% \@Latintrue
% \makeatother

% \usepackage{xeCJK}
% \usepackage[overlap]{ruby}
% \renewcommand\rubysep{-0.2ex}
% \setCJKmainfont[BoldFont=IPAGothic]{IPAMincho}

% \xeCJKDeclareSubCJKBlock{Kana}{"3040 -> "309F, "30A0 -> "30FF, "31F0 -> "31FF, "1B000 -> "1B0FF}
% \xeCJKDeclareSubCJKBlock{Hangul}{"1100 -> "11FF, "3130 -> "318F, "A960 -> "A97F, "AC00 -> "D7AF, "D7B0 -> "D7FF}

% \setCJKmainfont{HanaMinA}
% \setCJKmainfont[Kana]{HanaMinA}
% \setCJKmainfont[Hangul]{NanumMyeongjo}
% \setCJKsansfont[Hangul]{NanumGothic}

% \usepackage{luatexja-fontspec}
% \setmainjfont{IPAMincho}
% \setsansjfont{IPAGothic}

%% Theorems and definitions
%% remove parentheses
% \makeatletter
% \def\thmhead@plain#1#2#3{%
%   \thmname{#1}\thmnumber{\@ifnotempty{#1}{ }\@upn{#2}}%
%   \thmnote{ {\the\thm@notefont#3}}}
% \let\thmhead\thmhead@plain
% \makeatother

\theoremstyle{plain}
\newtheorem{theorem}{Theorem}
\newtheorem{proposition}[theorem]{Proposition}
\newtheorem{corollary}[theorem]{Corollary}
\newtheorem{claim}[theorem]{Claim}
\newtheorem{lemma}[theorem]{Lemma}
\newtheorem{axiom}[theorem]{Axiom}

\newtheorem*{corollary*}{Corollary}
\newtheorem*{claim*}{Claim}
\newtheorem*{lemma*}{Lemma}
\newtheorem*{proposition*}{Proposition}
\newtheorem*{theorem*}{Theorem}

\theoremstyle{definition}
\newtheorem{definition}{Definition}
\newtheorem{example}{Examples}
\newtheorem{examples}[example]{Example}
\newtheorem{exercise}{Exercise}[subsection]
\newtheorem{problem}[exercise]{Problem}

\newtheorem*{example*}{Example}
\newtheorem*{exercise*}{Exercise}
\newtheorem*{problem*}{Problem}
\makeindex

\begin{document}
%% Footnotes
\renewcommand*{\thefootnote}{\fnsymbol{footnote}}

%% Counters
\setsecnumdepth{subsection}
\counterwithout{exercise}{chapter}
\numberwithin{equation}{subsection}
\counterwithout{equation}{chapter}

%% Header-footer
\chapterstyle{veelo}

%% Redefine the QED symbol
% \renewcommand\qedsymbol{\ensuremath{\null\hfill\QED}}

\thispagestyle{empty}
\author{\href{mailto:\authoremail}{\documentauthor}}
\title{\documenttitle}
\date{\today}

\frontmatter
\maketitle
\tableofcontents

%% Notes
% \chapter{Course Notes}
\thispagestyle{empty}

These notes roughly correspond to the three sections (by the same name) on
groups, rings and fields from Dummit and Foote's \emph{Abstract Algebra}
\cite{dummit-foote:abstract-algebra}. I also make nominal use of Herstein's
\emph{Topics in Abstract Algebra} \cite{herstein:topics-in-algebra}

\section{Group Theory}


%%% Local Variables:
%%% mode: latex
%%% TeX-master: "../MA553-HW-Current"
%%% End:

% \section{Rings}
\begin{problem}
Let $R$ be a commutative ring with $1\neq 0$ and let $\frakp$ be a
prime ideal of $R$. Let $I$ and $J$ be ideals of $R$ such that $I\cap
J\subset\frakp$, prove that either $I\subset P$ or $J\subset P$.
\end{problem}
\begin{proof}
Without loss of generality, suppose that $I\nsubset J$. We show that
$J\subset\frakp$. Let $x\in I$. Then $x\notin\frakp$. But for any $y\in J$,
$xy\in I\cap J$. Thus, $xy\in\frakp$. Since $\frakp$ is prime, $x\in\frakp$
or $y\in\frakp$. But $x\notin\frakp$ hence, $y\in\frakp$. This is true for
any $y\in J$. Thus, $J\subset\frakp$.
\end{proof}
\begin{problem}
Prove that a finite integral domain is a field.
\end{problem}
\begin{proof}
Let $a\in R$ be a nonzero element. Define the map $\varphi_a\colon R\to R$
by $\varphi_a(x)\coloneqq ax$. Then $\varphi_a$ defines a group
homomorphism on $R$ viewed as an additive Abelian group: Let $x,y\in R$
then
\begin{align*}
\varphi_a(x+y)&=a(x+y)\\
&=ax+ay\\
&=\varphi_a(x)+\varphi_a(y).
\end{align*}
Now, let $x\in\ker\varphi$. Then $\varphi_a(x)=ax=0$. Since $R$ is a domain
and $a\neq 0$, $x=0$. Thus, $\varphi$ is injective. Since $R$ is finite and
$\varphi_a\colon R\to R$ is injective, $\varphi_a$ is surjective (by the
pigeonhole principle). Thus, there exists an element $b\in R$ such that
$\varphi_a(b)=ab=1$. Thus, $a$ is a unit. Since $\varphi_a$ chosen
arbitrarily, it follows that every nonzero element $a\in R$ is a
unit. Thus, $R$ is a field.
\end{proof}

\begin{problem}
An element $x$ of a ring $R$ is called nilpotent if some power of $x$ is
zero. Prove that if $x$ is nilpotent, then $1+x$ is a unit in $R$.
\end{problem}
\begin{proof}
First we will prove the following:
\begin{lemma}
If $x$ is nilpotent, then $-x$ is nilpotent.
\end{lemma}
\begin{proof}
\renewcommand\qedsymbol{$\clubsuit$}
Suppose that $x$ is nilpotent. Then $x^n=0$ for some positive integer
$n$. Then
\[
(-x)^n=(-1)^n\cdot x^n=(-1)^n\cdot 0=0.
\]
Thus, $-x$ is nilpotent.
\end{proof}
Now, since $x$ is nilpotent, by the preceding lemma, $-x$ is
nilpotent. Thus
\[
(-x)^n-1=(-x-1)((-x)^{n-1}+\cdots+1).
\]
Since $x^n=0$, we have
\[
-1=((-x)-1)((-x)^{n-1}+\cdots+1)
\]
or
\[
1=(1+x)((-x)^{n-1}+\cdots+1).
\]
Thus, $1+x$ is a unit.
\end{proof}

\begin{problem}
Let $R$ be a nonzero commutative ring with $1$. Show that if $I$ is an
ideal of $R$ such that $1+a$ is a unit in $R$ for all $a\in I$, then $I$ is
contained in every maximal ideal of $R$.
\end{problem}
\begin{proof}
Seeking a contradiction, assume otherwise. Then there exists a maximal
ideal $\frakm$ such that $\frakm\nsupset I$, i.e., for some $a\in I$,
$a\notin\frakm$. Consider the ideal generated by $(a)$. Since $a\in I$,
$(a)\neq R$ since $I$ is a proper ideal of $R$, in particular, since $a$ is
a nonunit. Consider the ideal $\frakm+(a)$. Since $a\notin\frakm$,
$\frakm\subset\frakm+(a)$. But since $\frakm$ is maximal, it follows that
$\frakm+(a)=R$. Hence, there exists an element $m\in\frakm$ such that
$m+ra=1$ for some $r\in r$. Then we have $m=1-ra$. Since $-r\in R$ and
$a\in I$, we have $-ra\in I$ so $m=1+(-ra)$ is a unit thus,
$\frakm=R$. This contradicts that $\frakm$ is a maximal ideals. Thus, $I$
is contained in every maximal ideal of $R$.
\end{proof}

\begin{problem}
Let $R$ be an integral domain and $F$ be its field of fractions. Let
$\frakp$ be a prime ideal in $R$ and
\[
R_\frakp\coloneqq
\left\{\,\tfrac{a}{b}\;\middle|\;a,b\in R,\,b\notin\frakp\,\right\}\subset F.
\]
Show that $R_\frakp$ has a unique maximal ideal.
\end{problem}
\begin{proof}
We will show that
\[
\frakp
R_\frakp\coloneqq\left\{\,\tfrac{a}{b}\;\middle|\;a\in\frakp,\,b\notin\frakp\,\right\}
\]
is the unique maximal ideal of $R$. We will show that $a/b\in R_\frakp$ is
a unit if and only if $a/b\notin\frakp R_\frakp$.

$\implies$ Suppose that $a/b$ is a unit. Then there exists an element
$a'/b'$ such that
\[
\left(\frac{a}{b}\right)\left(\frac{c}{d}\right)=\frac{ac}{bd}=\frac{1}{1}.
\]
That is, there exists an element $s\in R\minus\frakp$ such that
$s(ac-bd)=0$. Since $R$ is an integral domain, $s\neq 0$ so $ac-bd=0$
implies $ac=bd$. Since $b,d\notin\frakp$, $bd\notin\frakp$ (since $\frakp$
is prime) and, in particular, $ac\notin\frakp$ so $a/b\notin\frakp
R_\frakp$.

$\impliedby$ Conversely, suppose that $a/b\notin\frakp R_\frakp$. Then
$a\notin\frakp$. Thus, $b/a\in R_\frakp$ and
\[
\left(\frac{a}{b}\right)\left(\frac{b}{a}\right)=\frac{ab}{ba}=\frac{1}{1}.
\]
Thus, $a/b$ is a unit in $R_\frakp$.

Now, since $\frakp R_\frakp$ does not contain any units, it is a proper
ideal of $R_\frakp$. Morevore, for every $a/b\notin\frakp R_\frakp$,
$\frakp R_\frakp +(a/b)=R_\frakp$ so $\frak R_\frakp$ is a maximal ideal,
i.e., is not contained in any proper ideal of $R_\frakp$. Any other ideal
must contain a unit or is strictly contained in $\frakp R_\frakp$. Thus,
$\frakp R_\frakp$ is the unique maximal ideal of $R_\frakp$.
\end{proof}

\begin{problem}
Let $m$ and $n$ be relatively prime integers. Show that there is an
isomorphism $Z_{mn}^\times\cong Z_m^\times\times Z_n^\times$.
\end{problem}
\begin{proof}
Suppose $m$ and $n$ are relatively prime. Then $(m)+(n)=\bbZ$, i.e., $(m)$
and $(n)$ are comaximal. By the Chinese remainder theorem there is a ring
isomorphism
\[
Z_{mn}\cong Z_m\times Z_n.
\]
which gives an isomorphism of the group of units
\[
Z_{mn}^\times\cong\left(Z_m\times Z_n\right)^\times.
\]
Thus, it suffices to show that $\left(Z_m\times
  Z_n\right)^\times=Z_m^\times\times Z_m^\times$.

Suppose $(a,b)\in\left(Z_m\times Z_n\right)^\times$. Then $(a,b)$ is a unit
in $Z_m\times Z_n$, i.e., there exists $(c,d)$ such that
$(a,b)(c,d)=(1,1)$. But $(a,b)(c,d)=(1,1)$ if and only if $ac=1$ and
$bd=1$. Thus, $a\in Z_m^\times$ and $b\in Z_n^\times$ so $(a,b)\in
Z_m^\times\times Z_n^\times$. Conversely, if $(a,b)\in Z_m^\times\times
Z_n^\times$ then $a$ is a unit in $Z_m$ and $b$ is a unit in $Z_n$. Thus,
there exists elements $c\in Z_m$ and $d\in Z_n$ such that $ac=1$ and
$bd=1$ so $(a,b)(c,d)=(ac,bd)=(1,1)$. Thus, $(a,b)\in\left(Z_m\times
  Z_n\right)^\times$.
\end{proof}

\begin{problem}
Show that if $x$ is non-nilpotent in $R$ then a maximal ideal $\frakp$ of
$R$, which does not contain $x^n$ for $n=1,2,...$, is prime.
\end{problem}
\begin{proof}
I think what the professor had in mind was to prove this: ``Show that if
$x$ is non-nilpotent in $R$ then the ideal $\frakp$, which is maximal with
respect to not containing $x^n$ for any $n\in\bbZ$, is prime.''

This looks like a standard commutative algebra problem. Let
$S\coloneqq\left\{\,x^k\;\middle|\;k\geq 1\,\right\}$, i.e., the
multiplicative set generated by $x$ and suppose that $\frakp$ is an ideal
maximal with respect to $\frakp\cap S=\emptyset$. Seeking a contradiction
suppose $a,b\in R$ with $ab\in\frakp$ but $a,b\notin\frakp$. Then, the
ideals $\frakp+(a)$ and $\frakp+(b)$ contain $\frakp$ and therefore must
contain a power of $x$, say $x^m$ and $x^n$, respectively. Thus, we have
\[
x^mx^n=x^{m+n}\in(\frakp+(a))(\frakp+(b))\subset\frakp+(ab)\subset\frakp.
\]
But $\frakp$ is maximal with respect to not containing any power of
$x$. This is a contradiction. Thus, we must have $a\in\frakp$ or
$b\in\frakp$ which implies $\frakp$ is prime.
\end{proof}

\begin{problem}
Let $\bbQ$ be the field of rational numbers and
$D=\left\{\,a+b\sqrt{2}\;\middle|\;a,b\in\bbQ\,\right\}$.
\begin{enumerate}[label=(\alph*)]
\item Show that $D$ is a principal ideal domain.
\item Show that $\sqrt{3}$ is not an element of $D$.
\end{enumerate}
\end{problem}
\begin{proof}
(a) We prove the following stronger result (which is, incidentally, easier
to prove than what we are asked to prove): $D$ is a field (in fact, it is
the extension $\bbQ(\sqrt{2})$). Let $a+b\sqrt{2}\in D$ be a nonzero
element. To show that $a+b\sqrt{2}$ is a unit, it suffices to find an
inverse for it. Hence, we have
\[
\frac{1}{a+b\sqrt{2}}=\frac{a-b\sqrt{2}}{a^2-2b^2}=\frac{a}{a^2-2b^2}-\frac{b}{a^2-2b^2}\sqrt{2}.
\]
Note that $a^2-2b^2\neq 0$ if and only if $a^2=2b^2$, but this implies that
$a=\sqrt{2}b$ which is impossible since $\sqrt{2}\notin\bbQ$ so that the
above is indeed in $D$. Now, we have
\begin{align*}
(a+b\sqrt{2})\left(\frac{a}{a^2-2b^2}-\frac{b}{a^2-2b^2}\sqrt{2}\right)
&=\frac{1}{a^2-2b^2}\left(a^2+ab\sqrt{2}-2b^2+-ba\sqrt{2}\right)\\
&=\frac{a^2-2b^2}{a^2-2b^2}\\
&=1.
\end{align*}
Thus, $D$ is a field.
\\\\
(b) We shall proceed by contradiction. Suppose that $\sqrt{3}\in D$. Then
\[
\sqrt{3}=a+b\sqrt{2}
\]
for some $a,b\in\bbQ$. Squaring both sides, we have
\begin{align*}
3&=a^2+2b^2+2ab\sqrt{2}\\
3-a^2-2b^2&=2ab\sqrt{2}\\
\sqrt{2}&=\frac{3-a^2-2b^2}{2ab}.
\end{align*}
This implies that $\sqrt{2}\in\bbQ$, which is a contradiction.
\end{proof}

\begin{problem}
Show that if $p$ is a prime such that $p\equiv 1\pmod{4}$, then $x^2+1$ is
not irreducible in $\bbF_p[x]$.
\end{problem}
\begin{proof}
Since $p\equiv 1\pmod{4}$, $p=a^2+b^2$ for some integers $a$ and $b$. It
follows that $b\neq 0\pmod{p}$ or else $a=\sqrt{p}$ or $a^2+b^2>p$, a
contradiction. Thus $b$ is a unit in $\bbF_p$. We claim that $ab^{-1}$ is a
root of $x^2+1$. First note that
\[
(ab^{-1})^2+1=a^2b^{-2}+1.
\]
Since $a^2+b^2\equiv 0\pmod{p}$, it follows that $b^{-2}(a^2+b^2)\equiv
0\pmod{p}$, but $b^{-2}(a^2+b^2)=a^2b^{-2}+1$. Thus, $a^2b^{-2}+1=0$ in
$\bbF_p$. Thus, $a^2b^{-2}+1=0$ in $\bbF_p$ so $x^2+1$ has a root in
$\bbF_p[x]$ and hence, is reducible.
\end{proof}

\begin{problem}
Show that if $p$ is a prime such that $p\equiv 3\pmod{4}$, then $x^2+1$ is
irreducible in $\bbF_p[x]$.
\end{problem}
\begin{proof}
Note that $p-1\equiv 2\pmod{4}$. In particular, we see that $4\nmid p-1$
for all primes $p$ satisfying the conditions above. Now, consider
multiplicative subgroup $(\bbF_p[x])^\times\cong Z_{p-1}$ of
$\bbF_p[x]$, this is a cyclic group of order $p-1$. If $F_p^\times$ had an
element of order $4$ then, by Lagrange's theorem, $4\mid p-1$. But this is
false. Now suppose there exists $a\in\bbF_p$ such that $a^2=-1$. Then
$a^4=(-1)^2=1$. It follows that $a\neq 1$ and $a^3\neq 1$, so $a$ is an
element of order $4$ in $\bbF_p^\times$. Thus, $x^2$ does not have a root
in $\bbF_p[x]$. Since $x^2+1$ is of degree $2$, it follows that $x^2+1$ is
irreducible in $\bbF_p[x]$ for $p\equiv 3\pmod{4}$.
\end{proof}

\begin{problem}
Find a simpler description for each of the following rings:
\begin{enumerate}
\item $\bbZ[x]/(x^2-3,2x+4)$;
\item $\bbZ[i]/(2+i)$ $(i^2=-1)$.
\end{enumerate}
\end{problem}
\begin{proof}
\end{proof}

\begin{problem}
Show that $\bbZ[\sqrt{-13}]$ is not a principal ideal domain.
\end{problem}
\begin{proof}
It suffices to exhibit an ideal that is not generated by a single
element. To that end, consider the ideal generated by
\end{proof}

\begin{problem}
Let $D$ be a principal ideal domain. Prove that every nonzero prime ideal
of $D$ is a maximal ideal.
\end{problem}
\begin{proof}
\end{proof}

\begin{problem}
Prove or disprove that a nonzero prime ideal $P$ of a principal ideal
domain $R$ is a maximal ideal.
\end{problem}
\begin{proof}
\end{proof}

\begin{problem}
Consider the polynomial $f(x)=x^4+1$.
\begin{enumerate}[label=(\alph*)]
\item Use the Eisenstein Criterion to show that $f(x)$ is irreducible in
  $\bbZ[x]$.
\item Prove that $f(x)$ is reducible in $\bbF_p[x]$ for every prime
  $p$.
\end{enumerate}
\end{problem}
\begin{proof}
\end{proof}

\begin{problem}
Assume that $f(x)$ and $g(x)$ are polynomials in $\bbQ[x]$ and that
$f(x)g(x)\in\bbZ[x]$. Prove that the product of any coefficient of $f(x)$
with any coefficient of $g(x)$ is an integer.
\end{problem}
\begin{proof}
\end{proof}

\begin{problem}
Let $k$ be a field, $x,y$, indeterminates. Let $f(x)$ and $g(x)$ be
relatively prime polynomials in $k[x]$. Show that in the polynomial ring
$k(y)[x]$, $f(x)-yg(x)$ is irreducible.
\end{problem}
\begin{proof}
\end{proof}

%%% Local Variables:
%%% mode: latex
%%% TeX-master: "../MA553-Quals"
%%% End:

% \section{Fields}
\begin{problem}
Let $F$ be a field with prime characteristic $\ch(F)=p$. Let $L/F$ be a
finite extension such that $p$ does not divide $[L:F]$. Show that $L/F$ is
a separable extension.
\begin{proof}
Seeking a contradiction, suppose that $L/F$ is not separable. The there
exists an element $\alpha\in L$ such that its minimal polynomial
$m_{\alpha,F}(X)$ is not separable, i.e., $m_{\alpha,F}$ has a multiple
root. But recall that an irreducible polynomial $g(X)$ is separable if
$\deg(D(g))=\deg(g)-1$. Thus, we must have
$\deg\left(D(m_{\alpha,F})\right)<\deg\left(m_{\alpha,F}\right)-1$ (since
for any polynomial $f$, $\deg(D(f))\leq \deg(f)-1$). But since $\Ch(F)=p$,
this is true only if $p\mid\deg(m_{\alpha,F})$. For suppose not. Then
$m_{\alpha,F}(X)=X^n+a_{n-1}X^{n-1}+\cdots+a_0$ and
\[
D(m_{\alpha,F})=nX^{n-1}+\text{some terms of lower degree.}
\]
so that $\deg(D(m_{\alpha,F}))=n-1=\deg(m_{\alpha,F})-1$. Hence, we have
$p\mid[F(\alpha):L]$ and by the tower theorem,
\[
[L:F]=[L:F(\alpha)][F(\alpha):L]
\]
implies that $p\mid[L:F]$. This is a contradiction. Thus, $L/F$ is
separable.
\end{proof}
\end{problem}

\begin{problem}
Let $\zeta_5$ be a primitive $5$-th root of unity, and denote
$\theta=\zeta_5+\zeta_5^{-1}$ as an element of the cyclotomic field
$\bbQ(\zeta_5)$. Show that the minimal polynomial of $\theta$ over $\bbQ$
is $m_{\theta,\bbQ}(X)=X^2+X-1$.
\end{problem}
\begin{proof}
Via some algebra, \textbf{:\textasciicircum)}, we have
\begin{align*}
(\zeta_5+{\zeta_5}^{-1})^2+(\zeta_5+{\zeta_5}^{-1})-1
&={\zeta_5}^2+2+{\zeta_5}^{-2}+\zeta_5+{\zeta_5}^{-1}-1,
\shortintertext{but since ${\zeta_p}^{-k}={\zeta_p}^{p-k}$ we have}
&={\zeta_5}^2+2+{\zeta_5}^3+\zeta_5+{\zeta_5}^4-1\\
&={\zeta_5}^4+{\zeta_5}^3+{\zeta_5}^2+\zeta_5+1\\
&=0.
\end{align*}
Thus, $m_{\theta,\bbQ}$ satisfies $\theta$. This implies that the
minimal polynomial of $\theta$ divides $m_{\theta,\bbQ}$. Therefore, to
show that the minimal polynomial of $\theta$ is in fact $m_{\theta,\bbQ}$
we must show that $m_{\theta,\bbQ}$ is irreducible.

To see that $m_{\theta,\bbQ}$ is irreducible we employ Eisetnstein's
criterion. Consider the shifted polynomial
\[
m_{\theta,\bbQ}(X+2)=(X+2)^2+(X+2)-1=X^2+4X+4+X+2-1=X^2+5X+5.
\]
By Eisenstein's criterion, $5\mid 5$ and $5\mid 5X$, but $5^2\nmid
5$. Thus, $m_{\theta,\bbQ}(X+2)$ is irreducible so $m_{\theta,\bbQ}(X)$ is
irreducible. Therefore, the minimal polynomial of $\theta$ is
$m_{\theta,\bbQ}$.

Now, since $\bbQ$ is characteristic $0$, $\Gal(\bbQ(\zeta_5)/\bbQ)\cong
(\bbZ/(5))^\times\cong Z_4$. Since $Z_4$ has a unique subgroup of order
$2$, by he fundamental theorem of Galois theory, $\bbQ(\theta)$ is the only
extension of degree $2$ under $\bbQ(\zeta_5)$. Similarly, $\bbQ$ is the
only other proper subfield since the only other subgroup of $Z_4$ is the
trivial subgroup.
\end{proof}

\begin{problem}
Prove or disprove the following: If $f(x),g(x)\in\bbQ[x]$ are irreducible
polynomials that have the same splitting field, then $\deg f=\deg g$.
\end{problem}
\begin{proof}
This is false. Consider the polynomial $f(X)=X^3-2$. The splitting field of
this polynomial is $\bbQ\left(\sqrt[3]{2},\zeta_3\right)$. However, by the
primitive element theorem, there exists
$\alpha\in\bbQ(\sqrt[3]{2},\zeta_3)$ such that
$\bbQ(\sqrt[3]{2},\zeta_3)=\bbQ(\alpha)$ and the
$\deg(m_{\alpha,\bbQ})=\left[\bbQ\left(\sqrt[3]{2},\zeta_3\right):\bbQ\right]=6$.
\end{proof}

\begin{problem}
Prove or disprove that every finite algebraic extension field of
$\bbF_{p^n}$ is Galois.
\end{problem}
\begin{proof}
The adjective \emph{algebraic} is redundant in the above for every finite
extension is necessarily algebraic.
\\\\
Let $F$ be a finite extension of $\bbF_{p^n}$. Then $F$ must be a finite
field of characteristic $p$ since $\bbF_p\subset\bbF_{p^n}\subset F$. By
the uniqueness theorem for finite fields, $F\cong\bbF_{p^m}$ for some
positive integer $m$. Hence, $\bbF_{p^m}/\bbF_p$ is Galois, being the
splitting field of the separable polynomial $X^{p^m}-X$.

By the fundamental theorem of Galois theory, since $F$ is Galois over
$\bbF_p$, $F$ is Galois over any subfield containing $\bbF_p$. Thus,
$F/\bbF_p$ is Galois.
\end{proof}

\begin{problem}
If $[K:\bbF_p]$ divides $[L:\bbF_p]$, does it follow that $K$ is isomorphic
to a subfield of $L$?
\end{problem}
\begin{proof}
Yes. Put $n\coloneqq\left[K:\bbF_p\right]$, $m\coloneqq[L:\bbF_p]$, and
suppose $n\mid m$. By the fundamental theorem for finite fields,
$K\cong\bbF_{p^n}$ and $L\cong\bbF_{p^m}$. Now, $\Gal(L/\bbF_p)\cong Z_m$
(generated by the Frobenius automorphism). Since $n\mid m$, $Z_m$ has a
subgroup of order $Z_{m/n}$. Thus, by the fundamental theorem of Galois
theory, $L$ has a subfield $E$ such that
\[
\left[E:\bbF_p\right]=\left[Z_m:Z_{m/n}\right]=m/(m/n)=n.
\]
Thus, by the fundamental theorem for finite fields, $E\cong\bbF_{p^n}\cong
K$.
\end{proof}

\begin{problem}
Let $\bbF_p$ be a finite field whose cardinality $p$ is prime. Fix a
positive integer $n$ which is not divisible by $p$, and let $\zeta_n$ be a
primitive $n$th root of unity. Show that
$\left[\bbF_p(\zeta_n):\bbF_p\right]=a$ is the least positive integer such
that $p^a\equiv 1\pmod{n}$. (\emph{Hint:} the Galois group of the extension
of $\bbF_p$ is generated by the Frobenius automorphism.)
\end{problem}
\begin{proof}
By the fundamental theorem of finitely fields,
$G\coloneqq\Gal(\bbF_p(\zeta_n)/\bbF_p)=\langle\sigma\rangle$ where $\sigma$
is the Frobenius automorphism. Since
$\left[\bbF_p(\zeta_n):\bbF_p\right]=a$, the order of $\sigma$ is
$a$. Since $\zeta_n$ generates $\bbF_p(\zeta_n)$, by the fundamental
theorem of Galois theory, the identity automorphism is the only
automorphism in $G$ which fixes $\zeta_n$: If $b$ is a positive integer
with $b<a$, then ${\zeta_n}^{p^b}=\sigma^b(\zeta_n)\neq\zeta_n$. Hence,
$p^b\nequiv 1\pmod{n}$.

Since $\sigma_a=\id_{\bbF_{p}(\zeta_n)}$, we have that
$\sigma_a(\zeta_n)=\zeta_n$. But $\sigma_a(\zeta_n)={\zeta_n}^{p^a}$. Hence
${\zeta_n}^{p^n}=\zeta_n$. Since the $n$th roots of unity form a cyclic
multiplicative group generated by $\zeta_n$ of order $n$, it follows from
${\zeta_n}^{p^a}=\zeta_n$ that $p^a\equiv 1\pmod{n}$.
\end{proof}

\begin{problem}
Fix a prime $p$, and consider the polynomial $f(x)=x^p-x-1$. Let
$\bbF_p(f)$ be the splitting field of $f(x)$ over $\bbF_p$. Let
$a\in\bbF_p(f)$ be a root of $f$. Show that $a\mapsto a+1$ defines an
automorphism of $\bbF_p(f)$. Show that
$\Gal(\bbF_p(f)/\bbF_p)\cong\bbZ_p$. Prove that $f(x)$ is irreducible in
$\bbZ[x]$. $\bbF_p(f)/\bbF_p$ is called an Artin--Schreier Extension.
\end{problem}
\begin{proof}
Since $\bbF_p$ is of characteristic $p$, the \emph{freshman's dream holds},
i.e.,
\[
(a+1)^p-(a+1)-1=a^p+1^p-a-1-1=a^p-a-1=0.
\]
Thus, $a+1$ is a root of $f$. Note that if $a\in\bbF_p$, then $0$ is a root
of $f$ since $a+1,a+2,\cdots a+(p-a)=0$ would be the roots of this
polynomial. But $f(0)=0^p-0-1=-1\neq 0$. Thus, $a\notin\bbF_p$.

Now, we note that $\bbF_p(a)=\bbF_p(a+1)$: $1,a\in\bbF_p(a)$ so
$a+1\in\bbF_p(a)$ and $a,-1\in\bbF_p(a+1)$ so
$(a+1)-1=a\in\bbF_p(a+1)$. Thus,
\[
\bbF(a)=\bbF(a+1)=\bbF(a+2)=\cdots=\bbF(a+p-1).
\]
Since all of $a,a+1,...,a+p-1$ are roots of $f$, and all of these fields
are equal, $\bbF_p(a)=\bbF_p(f)$, i.e., $\bbF_p(a)$ is the splitting field
of $f$. Hence, any map $a\mapsto a+i$, for $0\leq i\leq p-1$, determines an
automorphism of $\bbF_p(f)$. Note that $a\mapsto a+i$ is just $i-1$
applications of the map $a\mapsto a+1$. hence,
$\Gal\left(\bbF_p(f)/\bbF_p\right)$ is cyclic generated by $a\mapsto
a+1$. Moreover, this is a group of order $p$ since $a+p=a$ but $a+i\neq a$
for all $1\leq i\leq p-1$. Thus, $\Gal\left(\bbF_p(f)/\bbF_p\right)\cong
Z_p$.

Since $f$ is a monic polynomial of degree
$p=\left[\bbF_p(a):\bbF_p\right]$, with $a$ as root, it follows that
$f(X)=m_{\alpha,\bbF_p}(X)$.

Hence, $f$ is irreducible in $\bbF_p[X]$.

Since $\bbZ$ is an integral domain, $f$ is a nonconstant monic polynomial
in $\bbZ[X]$ and $(p)$ is a proper ideol of $\bbZ$, and $\bar f=f$ is
irreducible in $\bbF_p[X]\cong(\bbZ/(p))[X]$, if $f$ is irreducible in
$\bbZ[X]$.
\end{proof}

\begin{problem}
Let $x$ and $y$ be indeterminates over the field $\bbF_2$. Prove that there
exists infinitely many subfields of $L=\bbF_2(x,y)$ that contain the field
$K=\bbF_2(x^2,y^2)$.
\end{problem}
\begin{proof}
\end{proof}

\begin{problem}
Let $K/F$ be an algebraic field extension. If $K=F(a)$ for some $a\in K$,
prove that there are only finitely many subfields of $K$ that contain $F$.
\end{problem}
\begin{proof}
\end{proof}

\begin{problem}
Let $p$ be a prime integer. Recall that a field extension $K/F$ is called a
$p$-extension if $K/F$ is Galois and $[K:F]$ is a power of $p$. If $K/F$
and $L/K$ are $p$-extensions, prove that the Galois closure of $L/F$ is a
$p$-extension.
\end{problem}
\begin{proof}
\end{proof}

\begin{problem}
Give an example where $K/F$ and $L/K$ are $p$-extensions, but $L/F$ is not
Galois.
\end{problem}
\begin{proof}
\end{proof}

\begin{problem}
Let $L/\bbQ$ be the splitting field of the polynomial $x^6-2\in\bbQ[x]$.
\begin{enumerate}[label=(\alph*)]
\item If $a$ is one root of $x^6-2$, draw the subfield lattice of the
  extension $\bbQ(a)$ over $\bbQ$.
% \begin{proof}[Subfield lattice]
% Alright. Let's crank it out! Let $f(x)=x^6-2$. The
% splitting field of this polynomial is just
% $L=\bbQ(\sqrt[6]{2},\zeta_6)$ with index
% $[L:\bbQ]=6\cdot\varphi(6)=6\cdot 2=12$. First, we'll
% calculate the Galois group of this extension. To that end,
% it suffices to look at the automorphisms on the generators
% of $L$.

% Clearly
% \[\Gal(L/\bbQ)=\left<\,
% \sigma,\tau\;\middle|\;
% \sigma^6=\tau^2=1,\,\tau\sigma=\sigma^5\tau\, \right>,\]
% where
% \begin{align*}
% \sigma
% &\colon
% \begin{cases}
% \sqrt[6]{2}&\longmapsto\zeta_6\sqrt[6]{2},\\
% \zeta_6&\longmapsto\zeta_6,
% \end{cases},
% &\tau
% &\colon
% \begin{cases}
% \sqrt[6]{2}&\longmapsto\sqrt[6]{2},\\
% \zeta_6&\longmapsto\zeta_6^5.
% \end{cases}
% \end{align*}
% Clearly $\sigma^6=\tau^2=1$. What is less trivial is
% showing $\sigma\tau=\tau\sigma^5$. Observe
% \begin{align*}
% \sigma^5
% &\colon
% \begin{cases}
% \sqrt[6]{2}&\longmapsto\zeta_6^5\sqrt[6]{2},\\
% \zeta_6&\longmapsto\zeta_6,
% \end{cases},\\
% \sigma\tau
% &\colon
% \begin{cases}
% \sqrt[6]{2}&\longmapsto\zeta_6\sqrt[6]{2},\\
% \zeta_6&\longmapsto\zeta_6^5,
% \end{cases},
% &\tau\sigma^5
% &\colon
% \begin{cases}
% \sqrt[6]{2}&\longmapsto(\zeta_6^5)^5\sqrt[6]{2}
% =\zeta_6\sqrt[6]{2},\\
% \zeta_6&\longmapsto\zeta_6^5.
% \end{cases}
% \end{align*}
% Thus $\Gal(L/\bbQ)\cong D_{12}$. From here, we simply use
% the Fundamental Theorem of Galois Theory and observe the
% correspondence between subfields of $L$ and subgroups of
% $D_{12}$. (If only I knew the subgroup lattice of
% $D_{12}$).
% \end{proof}
\item Give generators for each subfield $K$ of $L$ for which
  $[K:\bbQ]=2$. How many are there?
% \begin{proof}[Solution]
% There is at least one and it corresponds to the subgroup
% $\langle \sigma \rangle\leq D_{12}$ whose index
% $[D_{12}:\langle \sigma \rangle]=2$. Therefore, the only
% subfield is $K=\bbQ(\zeta_6)=\bbQ(\sqrt{-3})$ (a degree $2$
% extension over $\bbQ$).at
% \end{proof}
\item Give generators for each subfield $K$ of $L$ for which
  $[K:\bbQ]=3$. How many are there?
\item Give generators for each subfield $K$ of $L$ for which
  $[K:\bbQ]=4$. How many are there?
\item How many subfields $K$ of $L$ have index $[L:K]=2$?
% \begin{proof}[Solution]
% This is also has at least one such subfield corresponding
% to the subgroup $\langle \tau \rangle\leq D_{12}$. The
% field is $\bbQ(\sqrt[6]{2})$. The extension to $L$ is
% certainly degree $2$.
% \end{proof}
\end{enumerate}
\end{problem}

\begin{problem}
Give an example of a field $F$ having characteristic $p>0$ and irreducible
monic polynomial $f(x)\in F[x]$ that has a multiple root.
\begin{proof}

\end{proof}
\end{problem}

\begin{problem}
Let $f$ be an irreducible polynomial of degree $k$ over $\bbF_p$. Find the
splitting field of $f$ and its Galois group.
\end{problem}
\begin{proof}
\end{proof}

\begin{problem}
Let $n$ be a positive integer and $d$ a positive integer that divides
$n$. Suppose $a\in\bbR$ is a root of the polynomial
$x^n-2\in\bbQ[x]$. Prove that there is precisely one subfield $F$ of
$\bbQ(a)$ with $[F:\bbQ]=d$.
\end{problem}
\begin{proof}
\end{proof}

\begin{problem}
Let $a=\sqrt[3]{5-\sqrt{7}}$.
\begin{enumerate}[label=(\alph*)]
\item Find the minimal polynomial of $a$, and the conjugates of $a$.
\item Determine the Galois closure of $F$ of $\bbQ(a)$.
\item Show that $F/\bbQ$ is an extension by radicals.
\item Conclude that $\Gal(F/\bbQ)$ is solvable.
\end{enumerate}
\end{problem}
\begin{proof}
\end{proof}

\begin{problem}
Let $F$ be a field of characteristic $p>0$. Fix an element $c$ in
$F$. Prove that $f(x)=x^p-c$ is irreducible in $F[x]$ if and only if $f(x)$
has no roots in $F$.
\end{problem}
\begin{proof}
\end{proof}

\begin{problem}
Determine the Galois group of the splitting field over $\bbQ$ and all its
subfields for
\begin{enumerate}[label=(\alph*)]
\item $f(x)=x^3-2$
\item $f(x)=x^4+2$
\item $f(x)=x^4+4$
\item $f(x)=x^4+4x+2$
\end{enumerate}
\end{problem}
\begin{proof}
\end{proof}

\begin{problem}
Show that $\sqrt{2}\notin\bbQ(\sqrt[3]{2},\zeta_3)$, where
$\zeta_3^2+\zeta_3+1=0$.
\end{problem}
\begin{proof}
\end{proof}

\begin{problem}
Let $L/F$ be a Galois extension of degree $[L:F]=2p$, where $p$ is aan odd
prime.
\begin{enumerate}[label=(\alph*)]
\item Show that hhere exits a unique queadratic subfield $E$, i.e.,
  $F\subseteq E\subseteq L$ and $[E:F]=2$.
\item Does there exist a unique subfield $K$ of index $2$, i.e.,
  $F\subseteq E\subseteq L$ and $[E:F]=2$.
\end{enumerate}
\end{problem}
\begin{proof}
\end{proof}

\begin{problem}
Let $L/F$ be a Galois extension of degree $[L:F]=p^2$ for some prime
$p$. Let $K$ be a subfield satisfying $F\subset K\subset L$. Must $K/F$ be
a normal extension?
\end{problem}
\begin{proof}
\end{proof}

\begin{problem}
Let $L/F$ be the Galois closure of he separable algebraic field extension
$F(\theta)/F$. Let $p$ be a prime that divides $[L:F]$. Prove that there
exists a subfield $K$ of $L$ such that $[L:K]=p$ and $L=K(\theta)$.
\end{problem}
\begin{proof}
Since $p$ divides $[L:K]$, $[L:K]=pn$ for some positive integer
$n$.
\end{proof}
\begin{problem}
Suppose $L/\bbQ$ is a finite field extension with $[L:\bbQ]=4$. Is it
possible that there exist precisely two subfields $K_1$ and $K_2$ of $L$
for which $[L:K_i]=2$? Justify your answer.
\end{problem}
\begin{proof}
\end{proof}

%%% Local Variables:
%%% mode: latex
%%% TeX-master: "../MA553-Quals"
%%% End:


%% Ulrich homework
\mainmatter
\chapter{MA 553 Spring 2016}
\thispagestyle{empty}
This is material from the course MA 533 as it was taught in the spring of
2016.
\bigskip
\section{Homework}
Most of the homework is Ulrich original (or as original as elementary
exercises in abstract algebra can be). However, an excellent resource and
one that I will often quote on these solutions is \cite{hungerford}. Other
resources include \cite{dummit-foote} and (to a lesser extent)
\cite{herstein}. I may also cite Milne's \emph{Group Theory}, \emph{Field
  Theory}, and \emph{Commutative Algebra: A Primer} notes, respectively,
\cite{milneGT}, \cite{milneFT}, and (no reference for the last).

\begin{tabular}{cl}
  $\bbR$   & is the set of real numbers\\
  $\bbC$   & is the set of complex numbers\\
  $\bbQ$   & is the set of rational numbers\\
  $\bbF_q$ & is the finite field of order $q=p^n$ for some prime $p$\\
  $\bbZ$   & is the set of the integers\\
  $\bbN$   & is the set of the natural numbers $1,2,\dotsc$\\
  $C_n$    & is the cyclic group of order $n$ not necessarily equal
             (but isomorphic) to $\bbZ/p\bbZ$\\
  $S_n$    & is the symmetric group on $\{1,\dotsc,n\}$\\
  $A_n$    & is the alternating group on $\{1,\dotsc,n\}$\\
  $D_n$    & is the dihedral group of order $n$\\
  $A\smallsetminus B$ & is the set difference of $A$ and $B$, that is, the
                        complement of $A\cap B$ in $A$\\
  $X\simeq Y$ & means $X$ and $Y$ are isomorphic as groups, rings,
                $R$-modules, or fields
\end{tabular}

\newpage

\subsection{Homework 1}
\begin{problem}
  Let $G$ be a group, $a\in G$ an element of finite order $m$, and $n$ a
  positive integer. Prove that
  \[
    |a^n|=\frac{m}{\gcd(m,n)}.
  \]
\end{problem}
\begin{proof}
  Without loss of generality, we may assume $n<m$; otherwise, by the
  fundamental theorem of arithmetic, there exist $q$ and $r$ with $r<m$
  such that $n=qm+r$ so $a^n=a^{qm+r}=a^{qm}a^r=a^r$.
\end{proof}

\begin{problem}
  Let $G$ be a group, and let $a$, $b$ be elements of finite order $m$, $n$
  respectively. Show that if $ba=ab$ and
  $\langle a\rangle\cap\langle b\rangle=\{e\}$, then $|ab|=\lcm(m,n)$.
\end{problem}
\begin{proof}
\end{proof}

\begin{problem}
  Let $G$ be a group and $H$, $K$ normal subgroups with $H\cap
  K=\{e\}$. Show that
  \begin{enumerate}[label=(\alph*),noitemsep]
  \item $hk=kh$ for every $h\in H$, $k\in K$.
  \item $HK$ is a subgroup of $G$ with $HK\simeq H\times K$.
  \end{enumerate}
\end{problem}
\begin{proof}
\end{proof}

\begin{problem}
  Show that $A_4$ has no subgroup of order $6$ (although $6\mid 12=|A_4|$).
\end{problem}
\begin{proof}
\end{proof}

%%% Local Variables:
%%% mode: latex
%%% TeX-master: "../MA553-Quals"
%%% End:

\subsubsection{Homework 2}
\setcounter{exercise}{0}
\setcounter{equation}{0}

\begin{problem}
  Let \(G\) be the group of order \(2^n\cdot 3\), \(n\geq 2\). Show that
  \(G\) has a normal \(2\)-subgroup \(\neq\left\{e\right\}\).
\end{problem}
\begin{solution}
  Suppose that \(\card G=2^n\cdot 3\). By the first Sylow theorem, \(G\)
  contains a \(2\)-Sylow subgroup, i.e., a subgroup \(P\) of order
  \(\card P=2^3\); this is, by Corollary 5.3, a \(2\)-subgroup. Now, by
  Corollary 5.8 (iii), it suffices to show that \(P\) is the only
  \(2\)-Sylow subgroup. By The third Sylow theorem, the number of
  \(2\)-Sylow subgroups \(n_2\) is \(n_2\equiv 1\mod 2\) so either
  \(n_2=1\) or \(n_2=3\).

  Suppose that \(n_2=3\). The
\end{solution}

\begin{problem}
  Let \(G\) be a group of order \(p^2q\), \(p\) and \(q\) primes. Show that
  the Sylow \(p\)-Sylow subgroup or the \(q\)-Sylow subgroup of \(G\) is
  normal in \(G\).
\end{problem}
\begin{solution}
\end{solution}

\begin{problem}
  Let \(G\) be a subgroup of order \(pqr\), \(p<q<r\) primes. Show that the
  \(r\)-Sylow subgroup of \(G\) is normal in \(G\).
\end{problem}
\begin{solution}
\end{solution}

\begin{problem}
  Let \(G\) be a group of order \(n\) and let \(\varphi\colon G\to S_n\) be
  given by the action of \(G\) on \(G\) via translation.
  \begin{enumerate}[label=(\alph*),noitemsep]
  \item For \(a\in G\) determine the number and the lengths of the disjoint
    cycles of the permutation \(\varphi(a)\).
  \item Show that \(\varphi(G)\nsubset A_n\) if and only if \(n\) is even
    and \(G\) has a cyclic \(2\)-Sylow subgroup.
  \item If \(n=2m\), \(m\) odd, show that \(G\) has a subgroup of index
    \(2\).
  \end{enumerate}
\end{problem}
\begin{solution}
\end{solution}

\begin{problem}
  Show that the only simple groups \(\neq\left\{e\right\}\) of order
  \(<60\) are the groups of prime order.
\end{problem}
\begin{solution}
\end{solution}

%%% Local Variables:
%%% mode: latex
%%% TeX-master: "../MA553-Quals"
%%% End:

\subsubsection{Homework 3}
\setcounter{exercise}{0}
\setcounter{equation}{0}

\begin{problem}
  Let \(G\) be a finite group, \(p\) a prime number, \(N\) the
  intersubsection of all \(p\)-Sylow subgroups of \(G\). Show that \(N\) is
  a normal \(p\)-subgroup of \(G\) and that every normal \(p\)-subgroup of
  \(G\) is contained in \(N\).
\end{problem}
\begin{solution}
\end{solution}

\begin{problem}
  Let \(G\) be a group of order \(231\) and let \(H\) be an \(11\)-Sylow
  subgroup of \(G\). Show that \(H\subseteq Z(G)\).
\end{problem}
\begin{solution}
\end{solution}

\begin{problem}
  Let \(G=\left\{e,a_1,a_2,a_3\right\}\) be a non-cyclic group of order
  \(4\) and define \(\varphi\colon S_3\to\Aut(G)\) by
  \(\varphi(\sigma)(e)=e\) and \(\varphi(\sigma)(a_1)=a_{\sigma(i)}\). Show
  that \(\varphi\) is well-defined and an isomorphism of groups.
\end{problem}
\begin{solution}
\end{solution}

\begin{problem}
  Determine all groups of order \(18\).
\end{problem}
\begin{solution}
\end{solution}

%%% Local Variables:
%%% mode: latex
%%% TeX-master: "../MA553-Quals"
%%% End:

\subsubsection{Homework 4}
\begin{problem}
  Let $p$ be a prime and let $G$ be a nonAbelian group of order $p^3$. Show
  that $G'=Z(G)$.
\end{problem}
\begin{solution}
\end{solution}

\begin{problem}
  Let $p$ be an odd prime and let $G$ be a nonAbelian group of order $p^3$
  having an element of order $p^2$. Show that there exists an element
  $b\notin\langle a \rangle$ of order $p$.
\end{problem}
\begin{solution}
\end{solution}

\begin{problem}
  Let $p$ be an odd prime. Determine all groups of order $p^3$.
\end{problem}
\begin{solution}
\end{solution}

\begin{problem}
  Show that $(S_n)'=A_n$.
\end{problem}
\begin{solution}
\end{solution}

\begin{problem}
  Show that every group of order $<60$ is solvable.
\end{problem}
\begin{solution}
\end{solution}

\begin{problem}
  Show that every group of order $60$ that is simple (or not solvable) is
  isomorphic to $A_5$.
\end{problem}
\begin{solution}
\end{solution}

%%% Local Variables:
%%% mode: latex
%%% TeX-master: "../MA553-Quals"
%%% End:

\subsubsection{Homework 5}
\setcounter{exercise}{0}
\setcounter{equation}{0}

\begin{problem}
  Find all composition series and the composition factors of $D_6$.
\end{problem}
\begin{solution}
\end{solution}

\begin{problem}
  Let $T$ be the subgroup of $\GL(n,\bbR)$ consisting of all upper triangular
  invertible matrices. Show that $T$ is solvable.
\end{problem}
\begin{solution}
\end{solution}

\begin{problem}
  Let $p\in\bbZ$ be a prime number. Show:
  \begin{enumerate}[label=(\alph*),noitemsep]
  \item $(p-1)!\equiv -1\mod{p}$.
  \item If $p\equiv 1\mod{4}$ then $x^2\equiv-1\mod{p}$ for some
    $x\in\bbZ$.
  \end{enumerate}
\end{problem}
\begin{solution}
\end{solution}

\begin{problem}
  \begin{enumerate}[label=(\alph*),noitemsep]
  \item Show that the following are equivalent for an odd prime number
    $p\in\bbZ$:
    \begin{enumerate}[label=(\roman*),noitemsep]
    \item $p\equiv 1\mod 4$.
    \item $p=a^2+b^2$ for some $a$, $b$ in $\bbZ$.
    \item $p$ is not prime in $\bbZ[i]$.
    \end{enumerate}
  \item Determine all prime ideals of $\bbZ[i]$.
  \end{enumerate}
\end{problem}
\begin{solution}
\end{solution}

%%% Local Variables:
%%% mode: latex
%%% TeX-master: "../MA553-Quals"
%%% End:

\subsection{Homework 6}
\begin{problem}
Let $R$ be a domain. Show that $R$ is a UFD if and only if every nonzero
nonunit in $R$ is a product of irreducible elements and the intersection of
any two principal ideals is again principal.
\end{problem}
\begin{proof}
\end{proof}

\begin{problem}
Let $R$ be a PID and $\frakp$ a prime ideal of $R[X]$. Show that $\frakp$
is principal or $p=(a,f)$ for some $a\in R$ and some monic polynomial $f\in
R[X]$.
\end{problem}
\begin{proof}
\end{proof}

\begin{problem}
Let $k$ be a field and $n\geq 1$. Show that $Z^n+Y^3+X^2\in k(X,Y)[Z]$ is
irreducible.
\end{problem}
\begin{proof}
\end{proof}

\begin{problem}
Let $k$ be a field of characteristic zero and $n\geq 1$, $m\geq 2$. Show
that ${X_1}^n+\dotsb+{X_m}^n-1\in k[X_1,\dotsc,X_m]$ is irreducible.
\end{problem}
\begin{proof}
\end{proof}

\begin{problem}
Show that $X^{3^n}+2\in\bfQ(i)[X]$ is irreducible.
\end{problem}
\begin{proof}
\end{proof}

%%% Local Variables:
%%% mode: latex
%%% TeX-master: "../MA553-Quals"
%%% End:

\subsection{Homework 7}
\begin{problem}
Let $k\subset K$ and $k\subset L$ be finite field extensions contained in
some field. Show that:
\begin{enumerate}[label=(\alph*)]
\item $[KL:L]\leq [K:k]$.
\item $[KL:k]\leq [K:k][L:k]$.
\item $K\cap L=k$ if equality holds in (b).
\end{enumerate}
\end{problem}
\begin{proof}
\end{proof}

\begin{problem}
Let $k$ be a field of characteristic $\neq 2$ and $a,b$ elements of $k$ so
that $a$, $b$, $ab$ are not squares in $k$. Show that
$\bigl[k{(\sqrt{a},\sqrt{b})}:k\bigr]=4$.
\end{problem}
\begin{proof}
\end{proof}

\begin{problem}
Let $R$ be a UFD, but not a field, and write $K\coloneqq\Quot(R)$. Show that $[\bar
K:k]=\infty$.
\end{problem}
\begin{proof}
\end{proof}

\begin{problem}
Let $k\in K$ be an algebraic field extension. Show that every
$k$-homomorphism $\delta\colon K\to K$ is an isomorphism.
\end{problem}
\begin{proof}
\end{proof}

\begin{problem}
Let $K$ be the splitting field of $x^6-4$ over $\bfQ$. Determine $K$ and
$[K:\bfQ]$.
\end{problem}
\begin{proof}
\end{proof}

%%% Local Variables:
%%% mode: latex
%%% TeX-master: "../MA553-Quals"
%%% End:

\subsubsection{Homework 8}
\setcounter{exercise}{0}
\setcounter{equation}{0}

\begin{problem}
  Let \(k\) be a field, \(f\in k[X]\) is a polynomial of degree
  \(n\geq 1\), and \(K\) the splitting field of \(f\) over \(k\). Show that
  \([K:k]\mid n!\).
\end{problem}
\begin{solution}
\end{solution}

\begin{problem}
  Let \(k\) be a field and \(n\geq 0\). Define a map
  \(\Delta_n\colon k[X]\to k[X]\) by
  \(\Delta_n\bigl(\sum a_iX^i\bigr)=\sum a_i\binom{i}{n}X^{i-n}\). Show:
\begin{enumerate}[label=(\alph*),noitemsep]
\item \(\Delta_n\) is \(k\)-linear, and for \(f\), \(g\) in \(k[X]\),
  \(\Delta_n(fg)=\sum_{j=0}^n\Delta_j(f)\Delta_{n-j}(g)\);
\item \(f^{(n)}=n!\Delta_n(f)\);
\item \(f(X+a)=\sum\Delta_n(f)(a)X^n\), where \(a\in k\);
\item \(a\in k\) is a root of \(f\) of multiplicity \(n\) if and only if
  \(\Delta_i(f)(a)=0\) for \(0\leq i\leq n-1\) and
  \(\Delta_n(f)(a)\neq 0\).
\end{enumerate}
\end{problem}
\begin{solution}
\end{solution}

\begin{problem}
  Let \(k\subseteq K\) be a finite filed extension. Show that \(k\) is
  perfect if and only if \(K\) is perfect.
\end{problem}
\begin{solution}
\end{solution}

\begin{problem}
  Let \(K\) be the splitting field of \(X^p-X-1\) over
  \(k=\bbZ/p\bbZ\). Show that \(k\subseteq K\) is normal, separable, of
  degree \(p\).
\end{problem}
\begin{solution}
\end{solution}

\begin{problem}
  Let \(k\) be a field of characteristic \(p>0\), and \(k(X,Y)\) the field
  of rational functions in two variables.
\begin{enumerate}[label=(\alph*),noitemsep]
\item Show that \(\left[k(X,Y):k(X^p,Y^p)\right]=p^2\).
\item Show that the extension \(k(X^p,Y^p)\subseteq k(X,Y)\) is not simple.
\item Find infinitely many distinct fields \(L\) with
  \(k(X^p,Y^p)\subseteq L\subseteq k(X,Y)\).
\end{enumerate}
\end{problem}
\begin{solution}
\end{solution}

%%% Local Variables:
%%% mode: latex
%%% TeX-master: "../MA553-Quals"
%%% End:

\subsection{Homework 9}
\begin{problem}
  Let $k\subset K$ be a finite extension of fields of characteristic
  $p>0$. Show that if $p\nmid[K:k]$, then $k\subset K$ is separable.
\end{problem}
\begin{solution}
\end{solution}

\begin{problem}
  Let $k\subset K$ be an algebraic extension of fields of characteristic
  $p>0$, let $L$ be an algebraically closed field containing $K$, and let
  $\delta\colon k\to L$ be an embedding. Show that $k\subset K$ is purely
  inseparable if and only if there exists exactly one embedding
  $\tau\colon K\to L$ extending $\delta$.
\end{problem}
\begin{solution}
\end{solution}

\begin{problem}
  Let $k\subset K=k(\alpha,\beta)$ be an algebraic extension of fields of
  characteristic $p>0$, where $\alpha$ is separable over $k$ and $\beta$ is
  purely inseparable over $k$. Show that $K=k(\alpha+\beta)$.
\end{problem}
\begin{solution}
\end{solution}

\begin{problem}
  Let $f(X)\in\bbF_q[X]$ be irreducible. Show that $f(X)\mid X^{q^n}-X$ if
  and only if $\deg f(X)\mid n$.
\end{problem}
\begin{solution}
\end{solution}

\begin{problem}
  Show that $\Aut_{\bbF_q}(\bar\bbF_q)$ is an infinite Abelian group which
  is torsionfree (i.e., $\delta^n=\id$ implies $\delta=\id$ or $n=0$).
\end{problem}
\begin{solution}
\end{solution}

\begin{problem}
  Show that in a finite field, every element can be written as a sum of two
  perfect squares.
\end{problem}
\begin{solution}
\end{solution}

%%% Local Variables:
%%% mode: latex
%%% TeX-master: "../MA553-Quals"
%%% End:

\chapter{Homework 10}

%%% Local Variables:
%%% mode: latex
%%% TeX-master: "../MA553-Quals"
%%% End:

\subsection{Homework 11}
\begin{problem}
  Show that every algebraic extension of a finite field is Galois and
  Abelian.
\end{problem}
\begin{proof}
\end{proof}

\begin{problem}
  Let $k$ be a field of characteristic $\neq 2$ and $f(X)\ink[X]$ a
  cubic whose discriminant is a square. Show that $f$ is either irreducible
  or a product of linear polynomials in $k[X]$.
\end{problem}
\begin{proof}
\end{proof}

\begin{problem}
  Let $k$ be a field of characteristic $\neq 2$, and let
  $f(X)\coloneq X^4+aX^2+b\ink[X]$ be irreducible with Galois group
  $G$. Show:
  \begin{enumerate}[label=(\roman*),noitemsep]
  \item If $b$ is a square in $k$, then $G=H$.
  \item If $b$ is not a square in $k$, but $b(a^2-4b)$ is, then
    $G\simeq C_4$.
  \item If neither $b$ nor $b(a^2-4b)$ is a square in $k$, then
    $G\simeq D_4$.
  \end{enumerate}
\end{problem}
\begin{proof}
\end{proof}

\begin{problem}
  Determine the Galois group of:
  \begin{enumerate}[label=(\alph*),noitemsep]
  \item $X^4-5$ over $\bbQ$, over $\bbQ(\sqrt{5})$, over $\bbQ(\sqrt{-5})$;
  \item $X^3-10$ over $\bbQ$;
  \item $X^4-4X^2+5$ over $\bbQ$;
  \item $X^4+3X^3+3X-2$ over $\bbQ$;
  \item $X^4+2X^2+X+3$ over $\bbQ$.
  \end{enumerate}
\end{problem}
\begin{proof}
\end{proof}

\begin{problem}
  Let $K$ be the splitting field of $X^4-X^2-1$ over $\bbQ$. Determine
  all intermediate fields $L$, $\bbQ\subset L\subset K$. Which of
  these are Galois over $\bbQ$?
\end{problem}
\begin{proof}
\end{proof}

%%% Local Variables:
%%% mode: latex
%%% TeX-master: "../MA553-Quals"
%%% End:

\subsection{Homework 12}
\begin{problem}
Prove that the resolvent cubic $X^4+aX^2+bX+c$ is given by
$X^3-aX^2-4cX+4ac-b^2$.
\end{problem}
\begin{proof}
\end{proof}

\begin{problem}
Show that the general polynomial $g(Y)\coloneqq Y^n+u_1Y^{n-1}+\dotsb+u_n$
is irreducible in $k(u_1,\dotsc,u_n)[Y]$.
\end{problem}
\begin{proof}
\end{proof}

\begin{problem}
Let $k$ be a field.
\begin{enumerate}[label=(\alph*),noitemsep]
\item compute the discriminant $Y^3-Y\in k[Y]$ and $Y^3-1\in k[Y]$.
\item Show that the discriminant of the polynomial $(Y-X_1)(Y-X_2)(Y-X_3)$
  over $k(X_1,X_2,X_3)$ is of the form
  \[
    \lambda_1{s_1}^4+\lambda_2{s_1}^4s_2+\lambda_3{s_1}^3s_3+\lambda_4{s_1}^2{s_2}^2+\lambda_5s_1s_2s_3+\lambda_6{s_2}^3+\lambda_7{s_3}^2
  \]
  with $\lambda_i\in k$.
\item From (b) and (a) conclude that the discriminant $Y^3+aY+b\in k[Y]$ is
  $-4a^3-27b^2$.
\end{enumerate}
\end{problem}
\begin{proof}
\end{proof}

\begin{problem}
Let $\Phi_n(X)$ be the $n$th cyclotomic polynomial over $\bbQ$.
\begin{enumerate}[label=(\alph*),noitemsep]
\item Let $n={p_1}^{r_1}\dotsm{p_s}^{r_s}$ with $p_i$ distinct prime
  numbers and $r_i>0$. Show that $\Phi(X)=\Phi_{p_1\dotsm
    p_s}(X^{{p_1}^{r_1-1}\dotsm{p_s}^{r_s-1}})$.
\item For a prime number $p$ with $p\nmid n$ show that
  $\Phi_{pn}(X)=\Phi_n(X^p)/\Phi_n(X)$.
\end{enumerate}
\end{problem}
\begin{proof}
\end{proof}

%%% Local Variables:
%%% mode: latex
%%% TeX-master: "../MA553-Quals"
%%% End:

\subsection{Homework 13}
\begin{problem}
Let $n\geq 3$ and $\rho$ a primitive $n$th root of unity over $\bbQ$. Show
that $[\bbQ(\rho+\rho^{-1}):\bbQ]=\varphi(n)/2$.
\end{problem}
\begin{proof}
\end{proof}

\begin{problem}
Let $\rho$ be a primitive $n$th root of unity over $\bbQ$. Determine all
$n$ so that $\bbQ\subset\bbQ(\rho)$ is cyclic.
\end{problem}
\begin{proof}
\end{proof}

\begin{problem}
  Let $k\subset K$ be an extension of finite fields. Show that
  $\Norm_k^K$ and $\Tr_k^K$ are surjective maps from $K$ to
  $k$.
\end{problem}
\begin{proof}
\end{proof}

\begin{problem}
  Let $f(X)\in k[X]$ be a separable polynomial of degree $n\geq 3$ with
  Galois group isomorphic to $S_n$, and let $\alpha\in\bar k$ be a root
  of $f(X)$.
  \begin{enumerate}[label=(\alph*),noitemsep]
  \item Show that $f(X)$ is irreducible.
  \item Show that $\Aut_k(k(\alpha))=\{\id\}$.
  \item Show that $\alpha^n\notin k$ if $n\geq 4$.
  \end{enumerate}
\end{problem}
\begin{proof}
\end{proof}

\begin{problem}
  Let $k\subset K$ be a Galois extension.
\begin{enumerate}[label=(\alph*),noitemsep]
\item For $k\subset L\subset K$ show that $\Gal(K/L)$ is
  solvable if $\Gal(K/k)$ is solvable.
\item For $k\subset L\subset K$ with $k\subset L$ normal show
  that $\Gal(L/k)$ and $\Gal(K/L)$ are solvable if and only if
  $\Gal(K/k)$ is solvable.
\item For $k\subset L$ with $K$ and $L$ in a common field show
  that $\Gal(KL/L)$ is solvable if $\Gal(K/k)$ is solvable.
\end{enumerate}
\end{problem}
\begin{proof}
\end{proof}

%%% Local Variables:
%%% mode: latex
%%% TeX-master: "../MA553-Quals"
%%% End:


%% Ulrich exams

%% Ulrich quals

%% Misc quals

%% References
\backmatter
\bibliographystyle{plain}
\bibliography{alg-bib}
\printindex
\end{document}

%%% Local Variables:
%%% mode: latex
%%% TeX-master: t
%%% End:
