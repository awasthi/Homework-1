\subsection{Ulrich: Summer 2006}
\setcounter{exercise}{0}
\setcounter{equation}{0}

\begin{problem}
  Let \(G\) be a group of order \(2n\), where \(n\) is odd. Show that \(G\)
  as a subgroup of index \(2\). (\emph{Hint}: embed \(G\) into
  \(S_{2n}\)).
\end{problem}
\begin{solution}

\end{solution}

\begin{problem}
  Let \(G\) be a group of odd order and let \(H\) be a normal subgroup of
  order \(5\). Show that \(H\) is in the center of \(G\).
\end{problem}
\begin{solution}
\end{solution}

\begin{problem}
  Show that up to isomorphism, there are at most two groups of order
  \(147\) having an element of order \(49\).
\end{problem}
\begin{solution}
\end{solution}

\begin{problem}
  Let \(R\) be a principal ideal domain and \(\frakm\) a maximal ideal of
  the polynomial ring \(R[X]\) with \(\frakm\cap R\neq \{0\}\). Show that
  \(\frakm=(p,f)\) for some prime element \(p\) of \(R\) and some monic
  irreducible polynomial \(f\) in \(R[X]\).
\end{problem}
\begin{solution}
\end{solution}

\begin{problem}
  Let \(k\subseteq K\) be a normal extension of fields of characteristic
  \(p>0\) with \(G=\Aut_k(K)\). Show that the extension \(k\subseteq K^G\)
  is purely inseparable.
\end{problem}
\begin{solution}
  Take \(\alpha\in K^G\). Then, since \(\alpha\in K\) and \(K\) is a normal
  extension, \(\alpha\) is the root of some polynomial \(f\in k[X]\). Then
  \(f\) factors completely over \(K\) and \(G\) acts transitively on the
  roots of \(f\). But \(\sigma\) fixes every element in \(K^G\) so \(a\) is
  the only root of \(f\). Thus, \(f=(X-\alpha)^n\) for some \(n\in\bbN\)
  and we have \(K^G\) is purely inseparable.
\end{solution}

\begin{problem}
  Let \(k\subseteq K_1\) and \(k\subseteq K_2\) be finite a Galois extension
  contained in a common field, and write \(K=K_1K_2\).
  \begin{enumerate}[label=(\alph*)]
  \item Show that the extension \(k\subseteq K\) is finite Galois.
  \item Show that the Galois group \(\Gal(K/k)\) is isomorphic to the
    subgroup
    \(H=\bigl\{\,(\sigma,\tau):\sigma\restrict{K_1\cap
      K_2}=\tau\restrict{K_1\cap K_2}\,\bigr\}\) of
    \(\Gal(K_1/k)\times\Gal(K_2/k)\).
  \end{enumerate}
\end{problem}
\begin{solution}
\end{solution}

\begin{problem}
  Let \(p\) be a prime number, \(\zeta\in\bbC\) a primitive \(p\)th root of
  unity and \(K=\bbQ(\zeta)\). Determine those \(p\) for which \(K\) has a
  unique maximal power subfield \(k\subsetneq K\).
\end{problem}
\begin{solution}
\end{solution}

%%% Local Variables:
%%% mode: latex
%%% TeX-master: "../MA553-Quals"
%%% End:
