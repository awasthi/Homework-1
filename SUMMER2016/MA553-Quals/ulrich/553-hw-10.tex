\subsubsection{Homework 10}
\setcounter{exercise}{0}
\setcounter{equation}{0}

\begin{problem}
  Let \(k\subset K=k(\alpha)\) be a simple field extension, let
  \(G=\{\delta_1,\dotsc,\delta_n\}\) be a finite subgroup of \(\Aut_k(K)\),
  and write
  \(f(X)=\prod_{i=1}^n(X-\delta_i(\alpha))=\sum_{i=0}^na_iX^i\). Show that
  \(f(X)\) is the minimal polynomial of \(\alpha\) over \(K^2\) and that
  \(K^G=k(a_0,\dotsc,a_{n-1})\).
\end{problem}
\begin{solution}
\end{solution}

\begin{problem}
  Let \(k\) be a field, \(k(X)\) the field of rational functions, and
  \(u\in k(X)\setminus k\). Write \(u= f/g\) with \(f\) and \(g\)
  relatively prime in \(k[X]\). Show that
  \([k(X):k(u)]=\max\{{\deg f},{\deg g}\}\).
\end{problem}
\begin{solution}
\end{solution}

\begin{problem}
  Let \(k\) be a field and \(K= k(X)\) the field of rational
  functions. Show that for every \(\delta\in\Aut_k(K)\),
  \(\delta(X)= (aX+b)/(cX+d)\) for some \(a\), \(b\), \(c\), \(d\) in \(k\)
  with \(ad-bc\neq 0\), and that conversely, every such rational functions
  uniquely determines an automorphism \(\delta\in\Aut_k(K)\).
\end{problem}
\begin{solution}
\end{solution}

\begin{problem}
  With the notion of the previous problem let \(\delta\in\Aut_k(K)\) and
  \(G=\langle \delta \rangle\).
  \begin{enumerate}[label=(\alph*),noitemsep]
  \item Assume \(\delta(X)=1/(1-X)\). Show that \(|G|=3\) and determine
    \(K^G\).
  \item Assume \(\Ch k=0\) and \(\delta(X)=X+1\). Show that \(G\) is
    infinite and determine \(K^G\).
  \end{enumerate}
\end{problem}
\begin{solution}
\end{solution}

\begin{problem}
  Let \(k\subset K\) be a finite Galois extension with \(G=\Gal(K/k)\), let
  \(L\) be a subfield of \(K\) containing \(k\) with \(H=\Gal(K/L)\), and
  let \(L'\) be the compositum in \(K\) of the fields \(\delta(L)\),
  \(\delta\in G\). Show that:
  \begin{enumerate}[label=(\alph*),noitemsep]
  \item \(L'\) is the unique smallest subfield of \(K\) that contains \(L\)
    and is Galois over \(k\).
  \item \(\Gal(K/L')=\bigcap_{\delta\in G}\delta H\delta^{-1}\).
\end{enumerate}
\end{problem}
\begin{solution}
\end{solution}

%%% Local Variables:
%%% mode: latex
%%% TeX-master: "../MA553-Quals"
%%% End:
