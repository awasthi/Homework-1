\chapter{MA 553 Spring 2016}
\thispagestyle{empty}
This is material from the course MA 533 as it was taught in the spring of
2016.
\bigskip
\section{Homework}
Most of the homework is Ulrich original (or as original as elementary
exercises in abstract algebra can be). However, an excellent resource and
one that I will often quote on these solutions is \cite{hungerford}. Other
resources include \cite{dummit-foote} and (to a lesser extent)
\cite{herstein}. I may also cite Milne's \emph{Group Theory}, \emph{Field
  Theory}, and \emph{Commutative Algebra: A Primer} notes, respectively,
\cite{milneGT}, \cite{milneFT}, and (no reference for the last).


Throughout these notes

\begin{tabular}{cl}
  $\bbR$   & is the set of real numbers\\
  $\bbC$   & is the set of complex numbers\\
  $\bbQ$   & is the set of rational numbers\\
  $\bbF_q$ & is the finite field of order $q=p^n$ for some prime $p$\\
  $\bbZ$   & is the set of the integers\\
  $\bbN$   & is the set of the natural numbers $1,2,\dotsc$\\
  $k$   & is used to denote the base field with
             characteristic $\Char k$\\
  $K,E,L$& is used to denote field extensions over the base field
                    $k$\\
  $C_n$    & is the cyclic group of order $n$ not necessarily equal
             (but isomorphic) to $\bbZ/p\bbZ$\\
  $S_n$    & is the symmetric group on $\{1,\dotsc,n\}$\\
  $A_n$    & is the alternating group on $\{1,\dotsc,n\}$\\
  $D_n$    & is the dihedral group of order $n$\\
  $A\smallsetminus B$ & is the set difference of $A$ and $B$, that is, the
                        complement of $A\cap B$ in $A$\\
  $X\simeq Y$ & means $X$ and $Y$ are isomorphic as groups, rings,
                $R$-modules, or fields
\end{tabular}

\newpage

\subsection{Homework 1}
\begin{problem}
  Let $G$ be a group, $a\in G$ an element of finite order $m$, and $n$ a
  positive integer. Prove that
  \[
    |a^n|=\frac{m}{\gcd(m,n)}.
  \]
\end{problem}
\begin{proof}
  Let $\ell$ denote the order of $a^n$. Then $\ell$ is the minimal power of
  $a^n$ such that ${(a^n)}^\ell=e$. Now, observe that
  \begin{equation}
    \label{eq:1:1}
    \begin{aligned}
      {(a^n)}^{m/\gcd(m,n)}
      &=a^{nm/\gcd(m,n)}\\
      &=a^{mn/\gcd(m,n)}\\
      &={(a^m)}^{n/\gcd(m,n)}\\
      &=e^{n/\gcd(m,n)}\\
      &=e.
    \end{aligned}
  \end{equation}
  Thus $\ell\leq m$.

  By Theorem 3.4(iv) from \cite[Ch.\@ 3, p.\@ 35]{hungerford}, $g^$
\end{proof}

\begin{problem}
  Let $G$ be a group, and let $a$, $b$ be elements of finite order $m$, $n$
  respectively. Show that if $ba=ab$ and
  $\langle a\rangle\cap\langle b\rangle=\{e\}$, then $|ab|=\lcm(m,n)$.
\end{problem}
\begin{proof}
\end{proof}

\begin{problem}
  Let $G$ be a group and $H$, $K$ normal subgroups with $H\cap
  K=\{e\}$. Show that
  \begin{enumerate}[label=(\alph*),noitemsep]
  \item $hk=kh$ for every $h\in H$, $k\in K$.
  \item $HK$ is a subgroup of $G$ with $HK\simeq H\times K$.
  \end{enumerate}
\end{problem}
\begin{proof}
\end{proof}

\begin{problem}
  Show that $A_4$ has no subgroup of order $6$ (although $6\mid 12=|A_4|$).
\end{problem}
\begin{proof}
\end{proof}

%%% Local Variables:
%%% mode: latex
%%% TeX-master: "../MA553-Quals"
%%% End:
