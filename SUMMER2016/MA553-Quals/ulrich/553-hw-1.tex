\section{MA 553 Spring 2016}
% \thispagestyle{empty}
This is material from the course MA 533 as it was taught in the spring of
2016.%
\bigskip
\subsection{Homework}
Most of the homework is Ulrich original (or as original as elementary
exercises in abstract algebra can be). However, an excellent resource and
one that I will often quote on these solutions is \cite{hungerford}. Other
resources include \cite{dummit-foote} and (to a lesser extent)
\cite{herstein}. I may also cite Milne's \emph{Group Theory}, \emph{Field
  Theory}, and \emph{Commutative Algebra: A Primer} notes, respectively,
\cite{milneGT}, \cite{milneFT}, and (no reference for the last). Unless
otherwise stated, whenever we quote a result, e.g., Theorem 1.1, it is
understood to come from Hungerford's \emph{Algebra}.

Throughout these notes

\begin{tabular}{cl}
  \(\bbR\)   & is the set of real numbers\\
  \(\bbC\)   & is the set of complex numbers\\
  \(\bbQ\)   & is the set of rational numbers\\
  \(\bbF_q\) & is the finite field of order \(q=p^n\) for some prime \(p\)\\
  \(\bbZ\)   & is the set of the integers\\
  \(\bbN\)   & is the set of the natural numbers \(1,2,\dotsc\)\\
  \(k\)   & is used to denote the base field with
            characteristic \(\Ch k\)\\
  \(K,E,L\)& is used to denote field extensions over the base field
             \(k\)\\
  \(Z_n\)    & is the cyclic group of order \(n\) not necessarily equal
               (but isomorphic) to \(\bbZ/p\bbZ\)\\
  \(S_n\)    & is the symmetric group on \(\{1,\dotsc,n\}\)\\
  \(A_n\)    & is the alternating group on \(\{1,\dotsc,n\}\)\\
  \(D_n\)    & is the dihedral group of order \(n\)\\
  \(A\setminus B\) & is the set difference of \(A\) and \(B\), that is, the
                     complement of \(A\cap B\) in \(A\)\\
  \(X\cong Y\) & means \(X\) and \(Y\) are isomorphic as groups, rings,
                 \(R\)-modules, or fields
\end{tabular}

\newpage

\subsubsection{Homework 1}
\setcounter{exercise}{0}
\setcounter{equation}{0}

\begin{problem}
  Let \(G\) be a group, \(a\in G\) an element of finite order \(m\), and
  \(n\) a positive integer. Prove that
  \[
    \ord(a^n)=\frac{m}{(m,n)}.
  \]
\end{problem}
\begin{solution}
  Let \(\ell\) denote the order of \(a^n\). Then \(\ell\) is the minimal
  power of \(a^n\) such that \({(a^n)}^\ell=e\). Now, observe that
    \begin{align*}
      {(a^n)}^{m/(m,n)}
      &=a^{nm/(m,n)}\\
      &=a^{mn/(m,n)}\\
      &={(a^m)}^{n/(m,n)}\\
      &=e^{n/(m,n)}\\
      &=e.
    \end{align*}
    Thus \(\ell\leq m/(m,n)\).

    On the other hand, by Theorem 3.4 (iv) since
    \({(a^n)}^\ell=a^{n\ell}=e\) and the order of \(a\) is \(m\),
    \(m\mid n\ell\) or, equivalently, \(mk=n\ell\) for some
    \(k\in\bbZ^+\). Now, since \((m,n)\mid m\) and \((m,n)\mid n\), we can
    represent \(m\) and \(n\) as the products \((m,n)m'\) and \((m,n)n'\),
    respectively. Now, note that \(m'=m/(n,m)\) so we must show that
    \(m'\leq\ell\). Putting all of this together, we have \(mk\)
  \[
    mk=(m,n)m'k=(m,n)n'\ell=n\ell
  \]
  so
  \[
    m'k=n'\ell.
  \]
  Thus \(m'\mid n'\ell\) so either \(m'\mid n'\) or \(m'\mid\ell\). But
  since we factored the \((m,n)\) from \(m\) and \(n\), it follows that
  \((m',n')=1\) so \(m'\mid \ell\). Therefore \(m'\leq\ell\) and equality
  holds, that is, \(\ell=m/(m,n)\).
\end{solution}

\begin{problem}
  Let \(G\) be a group, and let \(a\), \(b\) be elements of finite order
  \(m\), \(n\) respectively. Show that if \(ba=ab\) and
  \(\langle a\rangle\cap\langle b\rangle=\{e\}\), then
  \(\ord(ab)=mn/(m,n)\).
\end{problem}
\begin{solution}
  Let \(\ell\) denote the order of \(ab\). Now, playing around with powers
  of \(ab\), we have
  \begin{align*}
    (ab)^{n}
    &=a^nb^n\\
    &=a^n\\
    &\neq e
  \end{align*}
  since the order of \(a\) is \(m\) and \(n<m\). Thus, by Problem 1,
  \(\ord(a^n)=m/(m,n)\) so \(\ord(ab)=mn/(m,n)\).
\end{solution}

\begin{problem}
  Let \(G\) be a group and \(H\), \(K\) normal subgroups with
  \(H\cap K=\{e\}\). Show that
  \begin{enumerate}[label=(\alph*),noitemsep]
  \item \(hk=kh\) for every \(h\in H\), \(k\in K\).
  \item \(HK\) is a subgroup of \(G\) with \(HK\cong H\times K\).
  \end{enumerate}
\end{problem}
\begin{solution}
  (a) Suppose that \(H\) and \(K\) are normal in \(G\). Then, for every
  \(g\in G\), \(gh=hg\) and \(gk=kg\) for any \(h\in H\), \(k\in K\). In
  particular, since \(H\subseteq G\), \(h\in G\) so \(hk=kh\).
  \\\\
  (b) Consider the subset \(HK\) of \(G\) consisting of all products \(hk\)
  where \(h\in H\), \(k\in K\). First, we show that \(HK\) is closed under
  multiplication: Pick \(h_1k_1,h_2k_2\in HK\) then
  \(h_1k_1h_2k_2=h_1(k_1k_2)h_2=h_1h_2(k_1k_2)\) is in \(HK\) since
  \(h_1h_2\in H\), \(k_1k_2\in K\). Moreover, since \(e\in H\) and
  \(e\in K\), \(ee=e\in HK\). Lastly, given \(hk\in HK\),
  \(hkh^{-1}k^{-1}=(hkh^{-1})k^{-1}=kk^{-1}=e\) so \(HK\) is closed under
  taking inverses. Thus, \(HK\) is a subgroup of \(G\).

  To see that \(HK\cong H\times K\), consider the map
  \(\varphi\colon HK\to (HK/K)\times(HK/H)\) given by
  \(\varphi(hk)=(\pi_K(h),\pi_H(k))\) where \(\pi_H\colon HK\to HK/H\) and
  \(\pi_K\colon HK\to HK/K\) are quotient maps. By the first (or second)
  isomorphism theorem, \(H\cong HK/H\) and \(K\cong HK/H\) so
  \(HK\cong H\times K\).
\end{solution}

\begin{problem}
  Show that \(A_4\) has no subgroup of order \(6\) (although
  \(6\mid 12=\card A_4\).
\end{problem}
\begin{solution}
  We proceed by contradiction. Suppose that \(A_4\) has a subgroup of order
  \(6\), call it \(H\). Then, we claim that \(H\) must contain all elements
  \(\sigma^2\) where \(\sigma\in A\).
  \begin{subproof}[Proof of claim]
    Since \(\card H=6\), \((A_4:H)=2\) which implies that \(H\) is must be
    a normal subgroup of \(A_4\). Now, consider the collection of \(G/H\)
    of right-cosets of \(H\) in \(G\). By Theorem 5.4, \(G/H\) is a group
    with order \(\card(G/H)=2\) so either \(\bar\sigma=\bar e\) or
    \({\bar\sigma}^2=\bar e\). Thus, \(\sigma^2\in H\).
  \end{subproof}
  Thus, \(H\) must contain all of the squares in \(A_4\). However, counting
  all of the elements in \(A_4\) and squaring them
  \begin{align*}
    (1)^2&=(1)
    &(1\,2\,3)^2&=(1\,3\,2)
    \\
    (1\,3\,2)^2&=(1\,2\,3)
    &(1\,2\,4)^2&=(1\,4\,2)\\
    (1\,4\,2)^2&=(1\,2\,4)
    &(1\,3\,4)^2&=(1\,4\,3)\\
    (1\,4\,3)^2&=(1\,3\,4)
    &(2\,3\,4)^2&=(2\,3\,4)\\
    (2\,4\,3)^2&=(2\,4\,3)
    &((1\,2)(3\,4))^2
             &=(1)\\
    ((1\,3)(2\,4))^2&=(1)&
    ((1\,4)(2\,3))^2&=(1)
  \end{align*}
  we see that there are a total of \(9\) squares (\(8\) nontrivial ones)
  which exceeds the order of \(H\). This is a contradiction therefore,
  \(G\) has no subgroup of order \(6\).
\end{solution}

%%% Local Variables:
%%% mode: latex
%%% TeX-master: "../MA553-Quals"
%%% End:
