\subsubsection{Homework 9}
\setcounter{exercise}{0}
\setcounter{equation}{0}

\begin{problem}
  Let \(k\subseteq K\) be a finite extension of fields of characteristic
  \(p>0\). Show that if \(p\nmid[K:k]\), then \(k\subseteq K\) is
  separable.
\end{problem}
\begin{solution}
\end{solution}

\begin{problem}
  Let \(k\subseteq K\) be an algebraic extension of fields of
  characteristic \(p>0\), let \(L\) be an algebraically closed field
  containing \(K\), and let \(\delta\colon k\to L\) be an embedding. Show
  that \(k\subseteq K\) is purely inseparable if and only if there exists
  exactly one embedding \(\tau\colon K\to L\) extending \(\delta\).
\end{problem}
\begin{solution}
\end{solution}

\begin{problem}
  Let \(k\subseteq K=k(\alpha,\beta)\) be an algebraic extension of fields
  of characteristic \(p>0\), where \(\alpha\) is separable over \(k\) and
  \(\beta\) is purely inseparable over \(k\). Show that
  \(K=k(\alpha+\beta)\).
\end{problem}
\begin{solution}
\end{solution}

\begin{problem}
  Let \(f(X)\in\bbF_q[X]\) be irreducible. Show that \(f(X)\mid X^{q^n}-X\)
  if and only if \(\deg f(X)\mid n\).
\end{problem}
\begin{solution}
\end{solution}

\begin{problem}
  Show that \(\Aut_{\bbF_q}(\bar\bbF_q)\) is an infinite Abelian group
  which is torsionfree (i.e., \(\delta^n=\id\) implies \(\delta=\id\) or
  \(n=0\).
\end{problem}
\begin{solution}
\end{solution}

\begin{problem}
  Show that in a finite field, every element can be written as a sum of two
  perfect squares.
\end{problem}
\begin{solution}
\end{solution}

%%% Local Variables:
%%% mode: latex
%%% TeX-master: "../MA553-Quals"
%%% End:
