\section{Ulrich}
\subsection{Ulrich: Winter 2002}
\setcounter{exercise}{0}
\setcounter{equation}{0}

\begin{problem}
  Let $G$ be a group and $H$ a subgroup of finite index. Show that there
  exists a normal subgroup $N$ of $G$ of finite index with $N\subset H$.
\end{problem}
\begin{solution}
  Let $n=[G:H]$ and $X=\left\{\,H,g_1H,\ldots,g_{n-1}H\,\right\}$ the set
  of left-cosets of $H$ in $G$ with representatives
  $g_0=e,g_1,\ldots,g_{n-1}$. Let $G$ act on $X$ by left multiplication,
  i.e., $g\mapsto g g_iH$; this is indeed an action since
  $e(g_iH)=eg_iH=g_iH$ for all $g_iH\in X$ and for $k_1,k_2\in G$
  $k_2(k_1g_iH)=k_2k_1g_iH=(k_2k_1)g_iH$. By Cayley's theorem, this induces
  a homomorphism $\varphi\colon G\to S_n$. Note that the action is not
  necessarily faithful. However, by the first isomorphism theorem, the
  kernel of $\varphi$, $N=\Ker\varphi$, is a normal subgroup of $G$ with
  index $[G:N]\leq |S_n|=n!$ and $N\subset H$ since $g\in N$ if and only if
  $gg_iH=g_iH$ which, in particular, implies that $gH=H$. Thus, $N\subset
  H$ and $[G:N]<\infty$.
\end{solution}

\begin{problem}
  Show that every group of order $992$ ($=32\cdot 31$) is solvable.
\end{problem}
\begin{solution}
  Suppose $G$ is a group with order $|G|=992=2^5\cdot 3$. By Sylow's
  theorem, the number of $2$-Sylow subgroups in $G$ is either $1$ or
  $3$. If the number of $2$-Sylow subgroups is $1$, then $P\lhd G$ and the
  quotient $G/P$ has order $[G:P]=3$, hence, is cyclic. Moreover, since $P$
  is a $p$-group, it is solvable. Since $P$ and $G/P$ are solvable, $G$ is
  solvable.

  Now, suppose the number of $2$-Sylow subgroups is $3$. Let
  $\Syl_2(G)=\{P,P_1,P_2\}$. Then, by Sylow's theorem, the three
  $2$-Sylow subgroups are conjugate, i.e., there exists $g_1,g_2\in G$ such
  that $P_1=g_1P{g_1}^{-1}$ and $P_2=g_2P{g_2}^{-1}$. Thus, $G$ acts on the
  set $\Syl_2(P)$ by conjugation. This actions defines a (not necessarily
  injective) homomorphism $\varphi\colon G\to S_3$. Now, we ask: What is
  the kernel of this homomorphism? By the first isomorphism theorem, we
  know that the index of the kernel in $G$ divides the order of $S_3$,
  i.e., $[G:{\Ker\varphi}]\mid 6$. Since $|G|<\infty$ implies that the
  order of the kernel is one of the following values
  \[
    |{\Ker\varphi}|=2^4,2^4\cdot 3,2^5,2^5\cdot 3.
  \]
  Now, $|{\Ker\varphi}|\neq 2^5\cdot 3$ since we know at least one
  automorphism, namely conjugation by $g_1$, which sends $P\mapsto
  P_1$. Thus, the order of the kernel is either $2^4$, $2^4\cdot 3$ or
  $2^5$. If the $|{\Ker\varphi}|=2^4$ or $2^5$, we are done for similar
  reasons to the argument we gave in the previous paragraph, namely, that
  $\Ker\varphi\lhd G$ and $G/\Ker\varphi$ is solvable (for
  $|{\Ker\varphi}|=2^4$, the quotient $G/{\Ker\varphi}$ has order $6$ so is
  isomorphic to one of two groups, $S_3$ or $Z_6$, both of which are
  solvable).

  Suppose $\Ker\varphi$ has order $2^4\cdot 3$. Then the number of
  $3$-Sylow subgroups is either $1$, $4$ or $16$. If this number is $1$, we
  are done as $Q\in \Syl_3({\Ker\varphi})$ is a normal subgroup and the
  quotient is a $p$-group. Suppose the number of $3$-Sylow subgroups is
  $16$. Then there are $16\cdot 2=32$ elements of order $3$ in
  $\Ker\varphi$.
\end{solution}

\begin{problem}
  Let $G$ be a group of order $56$ with a normal $2$-Sylow subgroup $Q$,
  and let $P$ be a $7$-Sylow subgroup of $G$. Show that either $G\cong
  P\times Q$ or $Q\cong\bbZ/(2)\times\bbZ/(2)\times\bbZ/(2)$.

  [\emph{Hint}: $P$ acts on $Q\setminus\{e\}$ via conjugation. Show
  that this action is either trivial or transitive.]
\end{problem}
\begin{solution}
  First, note that, by the fundamental theorem of arithmetic, the order of
  $G$ can be broken down into $56=2^3 \cdot 7$. Suppose $G$ has a normal
  $2$-Sylow subgroup $Q$ and let $P\in\Syl_3(G)$. Then
  $|\Syl_3(G)|=1,4$. If $|{\Syl_3(G)}|=1$, then $P$ is the unique $3$-Sylow
  subgroup of $G$, hence it is normal. Thus, $|P||Q|=|G|$ and $PQ=G$ since,
  if $g\in Q\cap G$, then $|g|=3$, but $2\mid |g|$ so $g=e$. Thus, $G\cong
  P\times Q$.

  Now, suppose $|{\Syl_3(G)}|=4$. Then $G$ contains $4$ $3$-Sylow subgroups
  which, by Sylow's theorem, are conjugate, i.e., there exists
  $g_1,g_2,g_3\in G$ such that
  $\Syl_p(G)=\left\{P,g_1P{g_1}^{-1},g_2P{g_2}^{-1},g_3P{g_3}^{-1}\right\}$. Let
  $P$ act on $Q$ by conjugation. Then
\end{solution}

\begin{problem}
  Let $R$ be a commutative ring and $\Rad(R)$ the intersection of all
  maximal ideals of $R$.
  \begin{enumerate}[label=(\alph*),noitemsep]
  \item Let $a\in R$. Show that $a\in\Rad(R)$ if and only if $1+ab$ is a
    unit for every $b\in R$.
  \item Let $R$ be a domain and $R[X]$ the polynomial ring over
    $R$. Deduce that $\Rad(R[X])=0$.
  \end{enumerate}
\end{problem}
\begin{solution}
\end{solution}

\begin{problem}
  Let $R$ be a unique factorization domain and $\frakp$ a prime ideal of $R[X]$
  with $\frakp\cap R=0$.
  \begin{enumerate}[label=(\alph*),noitemsep]
  \item Let $n$ be the smallest possible degree of a nonzero polynomial in
    $\frakp$. Show that $\frakp$ contains a primitive polynomial $f$ of
    degree $n$.
  \item Show that $\frakp$ is the principal ideal generated by $f$.
  \end{enumerate}
\end{problem}
\begin{solution}
\end{solution}

\begin{problem}
  Let $k$ be a field of characteristic zero. assume that every polynomial
  in $k[X]$ of odd degree and every polynomial in $k[X]$ of degree two has
  a root in $k$. Show that $k$ is algebraically closed.
\end{problem}
\begin{solution}
\end{solution}

\begin{problem}
  Let $k\subset K$ be a finite Galois extension with Galois group
  $\Gal(K/k)$, let $L$ be a field with $ k\subset L\subset K$, and set
  $H=\left\{\,\sigma\in\Gal(K/k):\sigma(L)=L\,\right\}$.
  \begin{enumerate}[label=(\alph*),noitemsep]
  \item Show that $H$ is the normalizer of $\Gal(K/L)$ in $\Gal(K/k)$.
  \item Describe the group $H/{\Gal(K/L)}$ as an automorphism group.
  \end{enumerate}
\end{problem}
\begin{solution}
\end{solution}

%%% Local Variables:
%%% mode: latex
%%% TeX-master: "../MA553-Quals"
%%% End:
