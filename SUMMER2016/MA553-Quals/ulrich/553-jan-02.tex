\section{Ulrich}
\subsection{Ulrich: Winter 2002}
\setcounter{exercise}{0}
\setcounter{equation}{0}

\begin{problem}
  Let $G$ be a group and $H$ a subgroup of finite index. Show that there
  exists a normal subgroup $N$ of $G$ of finite index with $N\subset H$.
\end{problem}
\begin{solution}
\end{solution}

\begin{problem}
  Show that every group of order $992$ ($=2^5\cdot 31$) is solvable.
\end{problem}
\begin{solution}
\end{solution}

\begin{problem}
  Let $G$ be a group of order $56$ with a normal $2$-Sylow subgroup $Q$,
  and let $P$ be a $7$-Sylow subgroup of $G$. Show that either $G\simeq
  P\times Q$ or $Q\simeq\bbZ/(2)\times\bbZ/(2)\times\bbZ/(2)$.

  [\emph{Hint}: $P$ acts on $Q\setminus\{e\}$ via conjugation. Show
  that this action is either trivial or transitive.]
\end{problem}
\begin{solution}
\end{solution}

\begin{problem}
  Let $R$ be a commutative ring and $\Rad(R)$ the intersection of all
  maximal ideals of $R$.
  \begin{enumerate}[label=(\alph*),noitemsep]
  \item Let $a\in R$. Show that $a\in\Rad(R)$ if and only if $1+ab$ is a
    unit for every $b\in R$.
  \item Let $R$ be a domain and $R[X]$ the polynomial ring over
    $R$. Deduce that $\Rad(R[X])=0$.
  \end{enumerate}
\end{problem}
\begin{solution}
\end{solution}

\begin{problem}
  Let $R$ be a unique factorization domain and $P$ a prime ideal of $R[X]$
  with $P\cap R=0$.
  \begin{enumerate}[label=(\alph*),noitemsep]
  \item Let $n$ be the smallest possible degree of a nonzero polynomial in
    $P$. Show that $P$ contains a primitive polynomial $f$ of degree $n$.
  \item Show that $P$ is the principal ideal generated by $f$.
  \end{enumerate}
\end{problem}
\begin{solution}
\end{solution}

\begin{problem}
  Let $k$ be a field of characteristic zero. assume that every polynomial
  in $k[X]$ of odd degree and every polynomial in $k[X]$ of degree two has
  a root in $k$. Show that $k$ is algebraically closed.
\end{problem}
\begin{solution}
\end{solution}

\begin{problem}
  Let $k\subset K$ be a finite Galois extension with Galois group
  $\Gal(K/k)$, let $L$ be a field with $ k\subset L\subset K$, and set
  $H=\left\{\,\sigma\in\Gal(K/k):\sigma(L)=L\,\right\}$.
  \begin{enumerate}[label=(\alph*),noitemsep]
  \item Show that $H$ is the normalizer of $\Gal(K/L)$ in $\Gal(K/k)$.
  \item Describe the group $H/{\Gal(K/L)}$ as an automorphism group.
  \end{enumerate}
\end{problem}
\begin{solution}
\end{solution}

%%% Local Variables:
%%% mode: latex
%%% TeX-master: "../MA553-Quals"
%%% End:
