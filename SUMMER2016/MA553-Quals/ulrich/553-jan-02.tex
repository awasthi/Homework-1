\section{Ulrich}
\subsection{Ulrich: Winter 2002}
\setcounter{exercise}{0}
\setcounter{equation}{0}

\begin{problem}
  Let \(G\) be a group and \(H\) a subgroup of finite index. Show that
  there exists a normal subgroup \(N\) of \(G\) of finite index with
  \(N\subseteq H\).
\end{problem}
\begin{solution}
  Suppose \(N<G\) with \(n=[G:N]<\infty\). Let \(G\) act on \(H\) by
  translation. This action gives a homomorphism \(\varphi\colon G\to
  S_n\). Then, by the first isomorphism theorem
  \({[G:{\Ker\varphi}]}\mid{\card S_n=n!}\). Thus, \(\Ker\varphi\) is a
  normal subgroup of \(G\) with finite index.
\end{solution}

\begin{problem}
  Show that every group of order \(992\) (\(=32\cdot 31)\) is solvable.
\end{problem}
\begin{solution}
  Suppose \(\card G=992=32\cdot 31=2^5\cdot 31\). By Sylow's theorem, \(G\)
  has \(1\) or \(32\) \(31\)-Sylow subgroups. In the former case, this
  implies that there is a unique \(31\)-Sylow subgroup \(P\) and therefore
  \(P\trileq G\). Moreover, since \(\card(G/P)=2^3\), \(G/P\) is solvable
  since it is a \(p\)-group. Thus, both \(G/P\) and \(P\) are solvable
  (the latter since it is Abelian), which implies that \(G\) is solvable.

  On the other hand, if \(G\) contains \(32\) \(31\)-Sylow subgroups, then
  there are exactly \(32\cdot 31-32\cdot 30=32\) elements not of order
  \(31\). This implies that there is exactly one \(2\)-Sylow subgroup \(Q\)
  in \(G\). Again, since \(\card G/Q=31\), \(G/Q\) is solvable and \(Q\) is
  solvable since it is a \(p\)-group. Thus, \(G\) is solvable.

  In every case, \(G\) we see that is solvable.
\end{solution}

\begin{problem}
  Let \(G\) be a group of order \(56\) with a normal \(2\)-Sylow subgroup
  \(Q\), and let \(P\) be a \(7\)-Sylow subgroup of \(G\). Show that either
  \(G\cong P\times Q\) or \(Q\cong\bbZ/(2)\times\bbZ/(2)\times\bbZ/(2)\).
  [\emph{Hint}: \(P\) acts on \(Q\setminus\{e\}\) via conjugation. Show
  that this action is either trivial or transitive.]
\end{problem}
\begin{solution}
  Suppose \(G\) is a group of order \(56=2^3\cdot 7\) with a normal
  \(2\)-Sylow subgroup \(Q\) and let \(P\in\Syl_7(G)\).
\end{solution}

\begin{problem}
  Let \(R\) be a commutative ring and \(\Rad(R)\) the intersection of all
  maximal ideals of \(R\).
  \begin{enumerate}[label=(\alph*)]
  \item Let \(a\in R\). Show that \(a\in\Rad(R)\) if and only if \(1+ab\)
    is a unit for every \(b\in R\).
  \item Let \(R\) be a domain and \(R[X]\) the polynomial ring over
    \(R\). Deduce that \(\Rad(R[X])=0\).
  \end{enumerate}
\end{problem}
\begin{solution}
\end{solution}

\begin{problem}
  Let \(R\) be a unique factorization domain and \(\frakp\) a prime ideal
  of \(R[X]\) with \(\frakp\cap R=0\).
  \begin{enumerate}[label=(\alph*)]
  \item Let \(n\) be the smallest possible degree of a nonzero polynomial
    in \(\frakp\). Show that \(\frakp\) contains a primitive polynomial
    \(f\) of degree \(n\).
  \item Show that \(\frakp\) is the principal ideal generated by \(f\).
  \end{enumerate}
\end{problem}
\begin{solution}
\end{solution}

\begin{problem}
  Let \(k\) be a field of characteristic zero. Assume that every polynomial
  in \(k[X]\) of odd degree and every polynomial in \(k[X]\) of degree two
  has a root in \(k\). Show that \(k\) is algebraically closed.
\end{problem}
\begin{solution}
\end{solution}

\begin{problem}
  Let \(k\subseteq K\) be a finite Galois extension with Galois group
  \(\Gal(K/k)\), let \(L\) be a field with \( k\subseteq L\subseteq K\),
  and set \(H=\left\{\,\sigma\in\Gal(K/k):\sigma(L)=L\,\right\}\).
  \begin{enumerate}[label=(\alph*)]
  \item Show that \(H\) is the normalizer of \(\Gal(K/L)\) in
    \(\Gal(K/k)\).
  \item Describe the group \(H/{\Gal(K/L)}\) as an automorphism group.
  \end{enumerate}
\end{problem}
\begin{solution}
\end{solution}

%%% Local Variables:
%%% mode: latex
%%% TeX-master: "../MA553-Quals"
%%% End:
