\subsubsection{Homework 2}
\setcounter{exercise}{0}
\setcounter{equation}{0}

\begin{problem}
  Let \(G\) be the group of order \(2^n\cdot 3\), \(n\geq 2\). Show that
  \(G\) has a normal \(2\)-subgroup \(\neq\left\{e\right\}\).
\end{problem}
\begin{solution}
  Suppose \(\card G=2^n\cdot 3\). By Sylow's theorem, \(G\) contains a
  \(2\)-Sylow subgroup \(P\) of order \(\card P=2^n\). If \(P\) is the
  unique \(2\)-Sylow subgroup in \(G\), \(P\trileq G\).

  Otherwise, Sylow's theorem implies that \(\card({\Syl_2(G)})\) must
  divide \(3\) and, since \(3\) is prime, must in fact equal \(3\). Then,
  each \(Q\in\Syl_2(G)\) is conjugate to \(P\). Enumerate the set
  \(\Syl_2(G)=\left\{P,P',P''\right\}\) and let \(G\) act on \(\Syl_2(G)\)
  by conjugation. This action gives rise to a homomorphism
  \(\varphi\colon G\to S_3\) given by the permutation representation of the
  action. This action is nontrivial since there exists elements
  \(g_1,g_2\in G\) such that \(P'=g_1P{g_1}^{-1}\) and
  \(P''=g_2P{g_2}^{-1}\) (which correspond to the permutations \((1\,2)\)
  and \((1\,3)\)). By the first isomorphism theorem,
  \(\Ker\varphi\trileq G\) and \((G:{\Ker\varphi})\mid \card S_3=6\). But
  we observed that the image of \(G\) in \(S_3\) contains at least \(3\)
  permutations: \((1\,2)\), \((1\,3)\) and
  \((1\,2)(1\,3)=(1\,3\,2)\). Thus, \((G:{\Ker\varphi})=3\) or \(6\). In
  either case, \(\Ker\varphi\) is a \(2\)-subgroup of \(G\).
\end{solution}

\begin{problem}
  Let \(G\) be a group of order \(p^2q\), \(p\) and \(q\) primes. Show that
  the \(p\)-Sylow subgroup or the \(q\)-Sylow subgroup of \(G\) is normal
  in \(G\).
\end{problem}
\begin{solution}
  Suppose \(\card G=p^2q\). Assuming \(p<q\) there are \(1\) or \(p^2\)
  \(q\)-Sylow subgroups. If there is \(1\) \(q\)-Sylow subgroup \(Q\) then
  \(Q\trileq G\). Otherwise, there are \(p^2\) \(q\)-Sylow subgroups in
  \(G\) and, counting the total number of elements of order \(q\), there
  are \(p^2(q-1)=p^2q-p^2\) remaining elements in \(G\) which leaves just
  enough room for \(1\) \(p\)-Sylow subgroup \(P\) which implies that
  \(P\trileq G\). Otherwise, \(p>q\) and we must be one \(1\) \(p\)-Sylow
  subgroup \(P\) in \(G\) which implies \(P\trileq G\). In each case, we
  either have a normal \(p\)-Sylow subgroup or a normal \(q\)-Sylow
  subgroup.
\end{solution}

\begin{problem}
  Let \(G\) be a subgroup of order \(pqr\), \(p<q<r\) primes. Show that the
  \(r\)-Sylow subgroup of \(G\) is normal in \(G\).
\end{problem}
\begin{solution}
  By Sylow's theorem, we have \(1\) or \(pq\) \(r\)-Sylow subgroup in
  \(G\). In the former case, there is a unique \(r\)-Sylow subgroup \(R\)
  which implies \(R\trileq G\). In the latter case, there are \(pq\)
  \(r\)-Sylow subgroups in \(G\) and that implies that we have
  \(pq(r-1)=pqr-pq\) elements of order \(r\). That leaves room for exactly
  \(pq\) elements that do not have order \(r\). Now we ask, what are the
  possible number of \(p\)- and \(q\)-Sylow subgroups? At minimum, we have
  \(1\) \(p\)- and \(1\) \(q\)-Sylow subgroups. This yields a total of
  \begin{align*}
    (p-1)+(q-1)+1&=p+q-1\\
                 &<pq
  \end{align*}
  which flows under the total number of elements to complete the size of
  the group. What is the next smallest possible number of \(p\)- and
  \(q\)-Sylow subgroups is \(r\). In this case, we have
  \begin{align*}
    r(p-1)+r(q-1)+1&=rp-r+rq-r+1\\
                   &=r(p+q-2)+1\\
                   &>pq
  \end{align*}
  since \(r>p\) and \(p+q-2>2p-2>p\). Thus, we cannot have \(pq\)
  \(r\)-Sylow subgroups in \(G\). It follows that there is only \(1\)
  \(r\)-Sylow subgroup \(R\) in \(G\) and so \(R\trileq G\).
\end{solution}

\begin{problem}
  Let \(G\) be a group of order \(n\) and let \(\varphi\colon G\to S_n\) be
  given by the action of \(G\) on \(G\) via translation.
  \begin{enumerate}[label=(\alph*),noitemsep]
  \item For \(a\in G\) determine the number and the lengths of the disjoint
    cycles of the permutation \(\varphi(a)\).
  \item Show that \(\varphi(G)\nsubseteq A_n\) if and only if \(n\) is even
    and \(G\) has a cyclic \(2\)-Sylow subgroup.
  \item If \(n=2m\), \(m\) odd, show that \(G\) has a subgroup of index
    \(2\).
  \end{enumerate}
\end{problem}
\begin{solution}
  For (a), let \(\{g_0=e,g_1,\dotsc,g_{n-1}\}\) be an enumeration of
  \(G\). Fix \(a=g_k\) in \(G\) for some \(0\leq k\leq n-1\). Then the
  action of \(G\) on itself by translation gives a homomorphism
  \(\varphi\colon G\to S_n\) which sends
  \(\left\{g_0,g_1,\dotsc,g_n\right\}\) to the set
  \(\left\{ag_0,ag_1,\dotsc,ag_n\right\}\). If \(a\) is nontrivial, the
  latter set equals \(G\) so has no fixed point. This implies that every
  nontrivial \(a\) in \(G\) corresponds to an \(n\)-cycle in \(S_n\). I
  don't know what he's talking about so I am just moving on.

  For (b),
\end{solution}

\begin{problem}
  Show that the only simple groups \(\neq\left\{e\right\}\) of order
  \(<60\) are the groups of prime order.
\end{problem}
\begin{solution}
  First, let us list all of the possible orders of groups with order less
  than \(60\), these orders are
  \begin{center}
    \begin{tabular}{ccccc}
      \(4\)&\(6\)&\(8\)&\(9\)&\(10\)\\
      \(12\)&\(14\)&\(15\)&\(16\)&\(18\)\\
      \(20\)&\(21\)&\(22\)&\(24\)&\(25\)\\
      \(26\)&\(27\)&\(28\)&\(30\)&\(32\)\\
      \(33\)&\(34\)&\(35\)&\(36\)&\(38\)\\
      \(39\)&\(40\)&\(42\)&\(44\)&\(45\)\\
      \(46\)&\(48\)&\(49\)&\(50\)&\(51\)\\
      \(52\)&\(54\)&\(55\)&\(56\)&\(58\)
    \end{tabular}.
  \end{center}
  These integers fall into one of the following categories
  \(n=p^2,pq,p^3,p^2q,pqr,p^4,p^3q,p^2q^2,p^5,p^4q\); here they are by type
  \begin{center}
    \begin{tabular}{cccccccccc}
      \(p^2\)&\(pq\)&\(p^3\)&\(p^2q\)&\(pqr\)&\(p^4\)&\(p^3q\)&\(p^2q^2\)&\(p^5\)&\(p^4q\)\\
      \hline\\
      \(4\)&\(6\)&\(8\)&\(12\)&\(30\)&\(16\)&\(24\)&\(36\)&\(32\)&\(48\)\\
      \(9\)&\(10\)&\(27\)&\(18\)&\(42\)&&\(40\)\\
      \(25\)&\(14\)&&\(20\)&&&\(56\)\\
      \(49\)&\(15\)&&\(28\)\\
      &\(21\)&&\(44\)\\
      &\(22\)&&\(45\)\\
      &\(26\)&&\(50\)\\
      &\(33\)&&\(52\)\\
      &\(34\)\\
      &\(35\)\\
      &\(38\)\\
      &\(39\)\\
      &\(46\)\\
      &\(51\)\\
      &\(54\)\\
      &\(55\)\\
      &\(58\)\\
    \end{tabular}.
  \end{center}
  All \(p\)-groups have a nontrivial center, so groups of orders
  corresponding to the the \(p^2\), \(p^3\), \(p^4\) and \(p^5\) columns
  are not simple. Similarly, groups of order \(pq\) are not simple and we
  have just shown that groups of order \(p^2q\) and \(pqr\) are not simple.

  Now we cover the following cases:
  \begin{quote}
    \begin{claim*}
      \hfill
      \begin{enumerate}[label=\textnormal{(\alph*)},noitemsep]
      \item If \(\card G=p^nq\) for \(n\geq 2\), \(G\) contains a
        nontrivial normal subgroup.
      \item If \(\card G=p^2q^2\), \(G\) contains a nontrivial normal
        subgroup.
      \end{enumerate}
    \end{claim*}
  \end{quote}
  \begin{subproof}[Proof of claim]
    For (a), consider the \(p\)-Sylow subgroup \(P\) of \(G\).
  \end{subproof}
\end{solution}

%%% Local Variables:
%%% mode: latex
%%% TeX-master: "../MA553-Quals"
%%% End:
