\subsubsection{Homework 2}
\setcounter{exercise}{0}
\setcounter{equation}{0}

\begin{problem}
  Let \(G\) be the group of order \(2^n\cdot 3\), \(n\geq 2\). Show that
  \(G\) has a normal \(2\)-subgroup \(\neq\left\{e\right\}\).
\end{problem}
\begin{solution}
  Suppose \(\card G=2^n\cdot 3\). By Sylow's theorem, \(G\) contains a
  \(2\)-Sylow subgroup \(P\) of order \(\card P=2^n\). If \(P\) is the
  unique \(2\)-Sylow subgroup in \(G\), \(P\trileq G\).

  Otherwise, Sylow's theorem implies that \(\card({\Syl_2(G)})\) must
  divide \(3\) and, since \(3\) is prime, must in fact equal \(3\). Then,
  each \(Q\in\Syl_2(G)\) is conjugate to \(P\). Enumerate the set
  \(\Syl_2(G)=\left\{P,P',P''\right\}\) and let \(G\) act on \(\Syl_2(G)\)
  by conjugation. This action gives rise to a homomorphism
  \(\varphi\colon G\to S_3\) given by the permutation representation of the
  action. This action is nontrivial since there exists elements
  \(g_1,g_2\in G\) such that \(P'=g_1P{g_1}^{-1}\) and
  \(P''=g_2P{g_2}^{-1}\) (which correspond to the permutations \((1\,2)\)
  and \((1\,3)\)). By the first isomorphism theorem,
  \(\Ker\varphi\trileq G\) and \([G:{\Ker\varphi}]\mid \card S_3=6\). But
  we observed that the image of \(G\) in \(S_3\) contains at least \(3\)
  permutations: \((1\,2)\), \((1\,3)\) and
  \((1\,2)(1\,3)=(1\,3\,2)\). Thus, \([G:{\Ker\varphi}]=3\) or \(6\). In
  either case, \(\Ker\varphi\) is a \(2\)-subgroup of \(G\).
\end{solution}

\begin{problem}
  Let \(G\) be a group of order \(p^2q\), \(p\) and \(q\) primes. Show that
  the \(p\)-Sylow subgroup or the \(q\)-Sylow subgroup of \(G\) is normal
  in \(G\).
\end{problem}
\begin{solution}
  % Suppose \(\card G=p^2q\). We may assume, without loss of generality, that
  % \(p<q\). If the number of \(q\)-Sylow subgroups of \(G\) is \(p^2\), then
  % by counting all the distinct elements of order \(q\), we see that
  % \(p^2(q-1)=p^2q-p^2\) there is room for exactly one \(p\)-Sylow
  % subgroup. Since this is the only \(p\)-Sylow subgroup of \(G\), it must
  % be normal.

  % On the other hand, if \(G\) contains \(q\) \(p\)-Sylow subgroups,
\end{solution}

\begin{problem}
  Let \(G\) be a subgroup of order \(pqr\), \(p<q<r\) primes. Show that the
  \(r\)-Sylow subgroup of \(G\) is normal in \(G\).
\end{problem}
\begin{solution}
\end{solution}

\begin{problem}
  Let \(G\) be a group of order \(n\) and let \(\varphi\colon G\to S_n\) be
  given by the action of \(G\) on \(G\) via translation.
  \begin{enumerate}[label=(\alph*),noitemsep]
  \item For \(a\in G\) determine the number and the lengths of the disjoint
    cycles of the permutation \(\varphi(a)\).
  \item Show that \(\varphi(G)\nsubset A_n\) if and only if \(n\) is even
    and \(G\) has a cyclic \(2\)-Sylow subgroup.
  \item If \(n=2m\), \(m\) odd, show that \(G\) has a subgroup of index
    \(2\).
  \end{enumerate}
\end{problem}
\begin{solution}
\end{solution}

\begin{problem}
  Show that the only simple groups \(\neq\left\{e\right\}\) of order
  \(<60\) are the groups of prime order.
\end{problem}
\begin{solution}
\end{solution}

%%% Local Variables:
%%% mode: latex
%%% TeX-master: "../MA553-Quals"
%%% End:
