\section{Field Theory and Galois Theory}
Notes taken from Keith Conrad's blurbs.

\subsection{Roots and irreducibles}
This handout discusses relationships between roots of irreducible
polynomials and field extensions.
\subsubsection{Roots in larger fields}
For most fields $K$, there are polynomials in $K[X]$ without a root in
$K$. Consider $X^2+1$ in $\bfR[X]$ or $X^3-2$ in $\bfF_7[X]$. If we are
willing to enlarge the field. The following is due to Kronecker.
\begin{theorem}
  Let $K$ be a field and $f(X)$ be a nonconstant polynomial in
  $K[X]$. There exists a field extension of $K$ containing a root of
  $f(X)$.
\end{theorem}
\begin{proof}
  It suffices to prove the theorem when $f(X)=\pi(X)$ is irreducible.

  Set $F=K[t]/(\pi(t))$ where $t$ is an indeterminate. Since $\pi(t)$ is
  irreducible in $K[t]$, $F$ is a field. Inside of $F$ we have $K$ as a
  subfield: the congruence classes represented by constants. There is also
  a root of $\pi(X)$ in $F$, namely the class of $t$. Indeed, writing $\bar
  t$ for the congruence class of $t$ in $F$, the congruence $\pi(t)\equiv
  0\mod \pi(t)$ becomes the equation $\pi(\bar t)=0$ in $F$.
\end{proof}

\begin{corollary}
  Let $K$ be a field and $f(X)=c_mX^m+\cdots+c_0$ a polynomial in $K[X]$
  with degree $m\geq 1$. There is a field $L\supset K$ such that in $L[X]$
  \[
    F(X)=c_m(X-\alpha_1)\cdots(X-\alpha_m).
  \]
\end{corollary}
\begin{proof}
  We induct on the degree $m$. The case $m=1$ is clear, using $L=K$. By
  Theorem 2.1, there is a field $F\supset K$ such that that $f(X)$ has a
  root in $F$, say $\alpha$. Then in $F[X]$,
  \[
    f(X)=(X-\alpha_1)g(X),
  \]
  where $\deg g(X)=m-1$. The leading coefficient of $g(X)$ is also $c_m$.

  Since $g(X)$ has smaller degree than $f(X)$, by induction on the degree
  there is a field $L\supset F$ (so $L\supset K$) such that $g(X)$
  decomposes into linear factors in $L[X]$, so we get the desired
  factorization of $f(X)$ in $L[X]$.
\end{proof}

\begin{corollary}
  Let $f(X)$ and $g(X)$ be nonconstant in $K[X]$. They are relatively prime
  in $K[X]$ if and only if they do not have a common root in any extension
  field of $K$.
\end{corollary}
\begin{proof}
  Assume $f(X)$ and $g(X)$ are relatively prime in $K[X]$. Then we can
  write
  \[
    f(X)u(X)+g(X)v(X)=1
  \]
  for some $u(X)$ and $v(X)$ in $K[X]$. If there were an $\alpha$ in a
  field extension of $K$ which is a common root of $f(X)$ and $g(X)$, then
  substituting $\alpha$ for $X$ in the above polynomial identity makes the
  left side $0$ while the right side is $1$. This is a contradiction, so
  $f(X)$ and $g(X)$ have no common root in any field extension of $K$.

  Now assume $f(X)$ and $g(X)$ are not relatively prime in $K[X]$. Say,
  $h(X)\in K[X]$ is a (nonconstant) common factor. There is a field
  extension of $K$ in which $h(X)$ has a root and this root will be a
  common root of $f(X)$ and $g(X)$.
\end{proof}

\subsubsection[Divisibility and roots in KX]{Divisibility and roots in
  $K[X]$}
There is an important connection between roots of a polynomial and
divisibility by \emph{linear} polynomials. For $f(X)\in K[X]$ and
$\alpha\in K$, $f(\alpha)=0$ $\iff$ $(X-\alpha)\mid f(X)$. The next result
is an analogue for divisibility by higher degree polynomials in $K[X]$,
provided they are irreducible. (All linear polynomials are irreducible.)

\begin{theorem}
  Let $\pi(X)$ be an irreducible in $K[X]$ and let $\alpha$ be a root of
  $\pi(X)$ in some larger field. For $h(X)$ in $K[X]$, $h(\alpha)=0$ $\iff$
  $\pi(X)\mid h(X)$ in $K[X]$.
\end{theorem}
\begin{proof}
  If $h(X)=\pi(X)g(X)$, then $h(\alpha)=\pi(\alpha)g(\alpha)=0$.

  Now assume $h(\alpha)=0$. Then $h(X)$ and $\pi(X)$ have a common root, so
  by Corollary 2.4 they have a common factor in $K[X]$. Since $\pi(X)$ is
  irreducible, this means $\pi(X)\mid h(X)$ in $K[X]$. To see this argument
  more directly, suppose $h(\alpha)=0$ and $\pi(X)$ does not divide
  $h(X)$. Then (because $\pi$ is irreducible) the polynomials $\pi(X)$ and
  $h(Xx)$ are relatively prime in $K[X]$ so we can write
  \[
    \pi(X)u(X)+h(X)v(X)=1
  \]
  for some $u(X),v(X)\in K[X]$. Substitute $\alpha$ for $X$ and the left
  side vanishes. The right side is $1$ so we have a contradiction.
\end{proof}

\begin{theorem}
  Let $K$ be a field and $L$ be a larger field. For $f(X)$ and $g(X)$in
  $K[X]$, $f(X)\mid g(X)$ in $K[X]$ if and only if $f(X)\mid g(X)$ in
  $L[X]$.
\end{theorem}
\begin{proof}
  It is clear that divisibility inf $K[X]$ implies divisibility in larger
  $L[X]$. Conversely suppose $f(X)\mid g(X)$ in $L[X]$. Then
  \[
    g(X)=f(X)h(X)
  \]
  for some $h(X)\in L[X]$. By the division algorithm in $K[X]$,
  \[
    g(X)=f(X)q(X)+r(X)
  \]
  where $q(X)$ and $r(X)$ are in $K[X]$ and $r(X)=0$ or $\deg r<\deg
  f$. Comparing these two formulas for $g(X)$, the uniqueness of the
  division algorithm in $L[X]$ implies $q(X)=h(X)$ and $r(X)=0$. Therefore
  $g(X)=f(X)q(X)$, so $f(X)\mid g(X)$ in $L[X]$.
\end{proof}

\subsection[Raising to the pth power in characteristic p]{Raising to the $p$th power in characteristic $p$}
\begin{lemma}
  Let $A$ be a commutative ring with prime characteristic. Pick any $a$ and
  $b$ in $A$.
  \begin{enumerate}[label=\textnormal{(\alph*)}]
  \item $(a+b)^p=a^p+b^p$.
  \item When $A$ is a domain, $a^p=b^p$ $\implies$ $a=b$.
  \end{enumerate}
\end{lemma}
\begin{proof}
  (a) By the binomial theorem,
  \[
    (a+b)^p=a^p+\sum_{k=1}^{p-1}\binom{p}{k}a^{p-k}b^k+b^p.
  \]
  For $1\leq k\leq p-1$, the integer $\binom{p}{k}$ is a multiple of $p$,
  so the intermediate terms are $0$ in $A$.

  (b) Now assume $A$ is a domain and $a^p=b^p$. Then
  $0=a^p-b^p=(a-b)^p$. (Note $(-1)^p=-1$ for $p\neq 2$, and also for $p=2$
  since $2=0$ $\implies$ $-1=1$ in $A$.) Since $A$ is a domain, $a-b=0$ so
  $a=b$.
\end{proof}

\begin{lemma}
  Let $F$ be a field containing $\bfF_p$. For $c\in F$, $c\in\bfF_p$ $\iff$
  $c^p=c$.
\end{lemma}
\begin{proof}
  Every element $c$ of $\bfF_p$ satisfies the equation $c^p=c$. Conversely,
  solutions to this equation are the roots of $X^p-X$, which has at most
  $p$ roots. The elements of $\bfF_p$already fulfill this upper bound, so
  there are no further roots in characteristic $p$.
\end{proof}

\begin{theorem}
  For any $f(X)\in\bfF_p[X]$, $f(X)^p=f(X^{p^r})=f(X^{p^r})$ for $r\geq
  0$. If $F$ is a field of characteristic $p$ other than $\bfF_p$, this is
  not always true in $F[X]$.
\end{theorem}
\begin{proof}
  Writing
  \[
    f(X)=c_mX^m+c_{m-1}X^{m-1}+\cdots+c_1X+c_0,
  \]
  Lemma 4.1a with $A=\bfF_p[X]$ gives
  \begin{align*}
    f(X)^p
    &=\left( c_mX^m+c_{m-1}X^{m-1}+\cdots+c_1X+c_0 \right)^p\\
    &={c_m}^pX^{mp}+{c_{m-1}}^pX^{p(m-1)}+\cdots+{c_1}^pX^p+c_0^p\\
    &=c_m{(X^p)}^m+c_{m-1}{(X^p)}^{m-1}+\cdots+c_1X^p+c_0,
  \end{align*}
  since $c^p=c$ for any $c\in\bfF_p$. The last expression is
  $f(X^p)$. Applying this result $r$ times, we find
  $f(X)^{p^r}=f(X^{p^r})$.
\end{proof}

Let $f(X)\in\bfF_p[X]$ be nonconstant, with degree $m$. Let
$L\supset\bfF_p$ be a field over which $f(X)$ decomposes into linear
factors, i.e., (2.1) holds. It is possible to that some roots of $f(X)$ are
multiple roots. As long as that does not happen, the following corollary
says something about the $p$th powers of the roots.

\begin{corollary}
  When $f(X)\in\bfF_p[X]$ has distinct roots, raising all roots of $f(X)$
  to the $p$th power permutes the roots
  \[
    \left\{{\alpha_1}^p,\ldots,{\alpha_m}^p\right\}=\left\{{\alpha_1},\ldots,{\alpha_m}\right\}.
  \]
\end{corollary}
\begin{proof}
  Let $S=\{\alpha_1,\ldots,\alpha_m\}$. Since $f(X)^p=f(X^p)$ by Theorem
  4.3, the $p$th power of each root of $f(X)$ is again a root of
  $f(X)$. Therefore raising to the $p$th power defines a function
  $\varphi\colon S\to S$. By Lemma 4.1b, $\varphi$ takes different values
  on different elements of $S$. Since $S$ is a finite set, $\varphi$ must
  assume each element of $S$ as a value (in the language of set theory, a
  one-to-one function from a finite set to itself is onto), so $\varphi$ is
  a permutation of $S$.
\end{proof}

\subsection[Roots of irreducibles in FpX]{Roots of irreducibles in
  $\bfF_p[X]$}
\begin{lemma}
  For $h(X)$ in $\bfF_p[X]$ with degree $m$, $\bfF_p[X]/(h(X))$ has size
  $p^m$.
\end{lemma}
\begin{proof}
  By the division algorithm in $\bfF_p[X]$, every congruence class modulo
\end{proof}

%%% Local Variables:
%%% mode: latex
%%% TeX-master: "../MA553-Quals"
%%% End:
