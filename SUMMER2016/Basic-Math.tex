\def\documentauthor{Carlos Salinas}
\def\documenttitle{Basic Mathematics}
% \def\hwnum{1}
\def\shorttitle{Basics}
\def\coursename{MA571}
\def\documentsubject{abstract algebra, algebraic geometry, algebraic
  topology, commutative algebra, differential geometry, lie groups,
  functional analysis, real analysis}
\def\authoremail{salinac@purdue.edu}

\documentclass[article,10pt]{memoir}
\usepackage{geometry}
\usepackage[dvipsnames]{xcolor}
\usepackage[
    breaklinks,
    bookmarks=true,
    colorlinks=true,
    pageanchor=false,
    linkcolor=black,
    citecolor=black,
    filecolor=black,
    menucolor=black,
    runcolor=black,
    urlcolor=black,
    hyperindex=false,
    hyperfootnotes=true,
    pdftitle={\shorttitle},
    pdfauthor={\documentauthor},
    pdfkeywords={\documentsubject},
    pdfsubject={\coursename}
    ]{hyperref}
\usepackage{natbib}

%% Math
\usepackage{amsmath}
\usepackage{amsthm}
\usepackage{amssymb}
\usepackage{mathtools}
\usepackage{eucal}
\usepackage{mathrsfs}
\usepackage[nointegrals]{wasysym}

%% Language
\usepackage{cmap}
\usepackage[LAE,LFE,T2A,T1]{fontenc}
\usepackage[utf8]{inputenc}
\usepackage[farsi,french,german,spanish,russian,english]{babel}
\babeltags{fr=french,
           de=german,
           en=english,
           es=spanish,
           pa=farsi,
           ru=russian
           }
\def\spanishoptions{mexico}

\selectlanguage{english}

\newcommand{\textfa}[1]{\beginR\textpa{#1}\endR}

\usepackage{CJKutf8}
\newcommand{\textkr}[1]{\begin{CJK}{UTF8}{mj}#1\end{CJK}}
\newcommand{\textjp}[1]{\begin{CJK}{UTF8}{min}#1\end{CJK}}
\newcommand{\textzh}[1]{\begin{CJK}{UTF8}{bsmi}#1\end{CJK}}

%% Misc
\usepackage{graphicx}
\graphicspath{{figures/}}

\usepackage{microtype}
\usepackage{lineno}
\usepackage{multicol}
\usepackage[inline]{enumitem}
\usepackage{listings}
\usepackage{mleftright}
\mleftright
\usepackage{carlos-variables}

% %% Unicode math and Polyglossia
% \usepackage{unicode-math}
% \usepackage{unicode-minionmath}

% \setmainfont[Ligatures=TeX]{Libertinus Serif}
% \setsansfont{Libertinus Sans}
% \setmonofont{Libertinus Mono}
% \setmathfont{MinionMath-Regular.otf}
% \setmathfont[range={\mathfrak}]{latinmodern-math.otf}
% \setmathfont[range={\mathcal}]{latinmodern-math.otf}
% \setmathfont[range={\mathscr}]{latinmodern-math.otf}
% \setmathfont[range={}]{MinionMath-Regular.otf}

% \usepackage{polyglossia}

% \newfontfamily\cyrillicfont[Script=Cyrillic]{Libertinus Serif}
% \newfontfamily\cyrillicfontsf[Script=Cyrillic]{Libertinus Sans}

% \newfontfamily\farsifont[Script=Arabic,
%                          Scale=MatchUppercase]{Amiri}

% \setmainlanguage[variant=american]{english}
% \setotherlanguage{farsi}
% \setotherlanguage{french}
% \setotherlanguage[spelling=new,latesthyphen,babelshorthands]{german}
% \setotherlanguage{spanish}
% \setotherlanguage[spelling=modern,babelshorthands]{russian}

% \makeatletter
% \@Latintrue
% \makeatother

% \usepackage{xeCJK}
% \usepackage[overlap]{ruby}
% \renewcommand\rubysep{-0.2ex}
% \xeCJKDeclareSubCJKBlock{Kana}{"3040 -> "309F, "30A0 -> "30FF, "31F0 -> "31FF, "1B000 -> "1B0FF}
% \xeCJKDeclareSubCJKBlock{Hangul}{"1100 -> "11FF, "3130 -> "318F, "A960 -> "A97F, "AC00 -> "D7AF, "D7B0 -> "D7FF}

% \setCJKmainfont{HanaMinA}
% \setCJKmainfont[Kana]{HanaMinA}
% \setCJKmainfont[Hangul]{NanumMyeongjo}
% \setCJKsansfont[Hangul]{NanumGothic}

% %% Theorems and definitions
%% remove parentheses
% \makeatletter
% \def\thmhead@plain#1#2#3{%
%   \thmname{#1}\thmnumber{\@ifnotempty{#1}{ }\@upn{#2}}%
%   \thmnote{ {\the\thm@notefont#3}}}
% \let\thmhead\thmhead@plain
% \makeatother

\theoremstyle{plain}
\newtheorem{theorem}{Theorem}
\newtheorem{proposition}[theorem]{Proposition}
\newtheorem{corollary}[theorem]{Corollary}
\newtheorem{claim}[theorem]{Claim}
\newtheorem{lemma}[theorem]{Lemma}
\newtheorem{axiom}[theorem]{Axiom}

\newtheorem*{corollary*}{Corollary}
\newtheorem*{claim*}{Claim}
\newtheorem*{lemma*}{Lemma}
\newtheorem*{proposition*}{Proposition}
\newtheorem*{theorem*}{Theorem}

\theoremstyle{definition}
\newtheorem{definition}{Definition}
\newtheorem{example}{Examples}
\newtheorem{examples}[example]{Example}
\newtheorem{exercise}{Exercise}[chapter]
\newtheorem{problem}[exercise]{Problem}

\newtheorem*{example*}{Example}
\newtheorem*{exercise*}{Exercise}
\newtheorem*{problem*}{Problem}

\begin{document}
%% Footnotes
\renewcommand*{\thefootnote}{\fnsymbol{footnote}}

%% Counters
\counterwithout{exercise}{chapter}
\numberwithin{equation}{section}
\counterwithout{equation}{chapter}

%% Redefine the QED symbol
% \renewcommand\qedsymbol{\ensuremath{\null\hfill\QED}}

\chapterstyle{veelo}
\pagestyle{ruled}
\author{\href{mailto:\authoremail}{\documentauthor}}
\title{\documenttitle}
\date{\today}
\maketitle
\tableofcontents
\chapter{Abstract Algebra}
The main text we will be following for this material is
\cite{hungerford}. However, proofs and examples will be lifted from
\cite{dummit-foote} and \cite{herstein} when possible.

We will split this chapter into four sections: group theory, ring
theory, module theory, and field and Galois theory.

\section{Group Theory}
We begin with the definition of a group. A \emph{group}


%%% Local Variables:
%%% mode: latex
%%% TeX-master: "../Basic-Math"
%%% End:

\chapter{Real Analysis}

%%% Local Variables:
%%% mode: latex
%%% TeX-master: "../Basic-Math"
%%% End:

% \chapter{McMullen's Complex Analysis Notes}
\section{Basic Complex Analysis}
\subsection{Some Notation}
The complex numbers will be denoted $\bbC$. We let $\Delta$, $\bbH$, and
$\widehat\bbC$ denote the unit disk $|z|<1$, the upper half-plane $\Im z>1$
and the Riemann sphere $\bbC\cup\{\infty\}$. We write $S^1(r)$ for the
circle $|z|=r$ and $S^1$ for the unit circle, each oriented
counter-clockwise. We also set $\Delta^*=\Delta\minus\{0\}$ and
$\bbC^*=\bbC\minus\{0\}$.
\subsection{Algebra and complex numbers}
The complex numbers are formally defined as the field $\bbC=\bbR[i]$, where
$i^2=-1$. They are represented in the Euclidean plane by
$z=(x,y)=x+iy$. There are two square-roots of $-1$ in $\bbC$; the number
$i$ is the one with positive imaginary part.

An important role is played by the Galois involution $z\mapsto\bar z$. We
define $|z|^2=N(z)=z\bar z=x^2+y^2$. Compatibility of $|z|$ with the
Euclidean metric justifies the identification of $\bbC$ and $\bbR^2$. We
also see that $\bbC$ is a field as $1/z=\bar z/|z|$.

It is also convenient to describe complex numbers by polar coordinates
\[
z=[r,\theta]=r(\cos\theta+i\sin\theta).
\]
Here $r=|z|$ and $\theta=\arg z$ in $\bbR/2\pi\bbZ$. (The multivaluedness
of $\arg z$ requires care but is also the  source of powerful results like
Cauchy's integral formula.) We then have
\[
\left[r_1,\theta_1\right]\left[r_2,\theta_2\right]=
\left[r_1r_2,\theta_1+\theta_2\right].
\]
In particular, the linear maps $f(z)=az+b$, $a\neq 0$, of $\bbC$ to itself,
preserve angles and orientations.

This formula should be proved geometrically; in fact, it is a consequence
of the formula $|ab|=|a||b|$ and similar triangles. It can then be used to
derive the addition formulas for sine and cosine.

\subsection{Algebraic closure}
A critical feature of the complex numbers is that they are
\emph{algebraically closed}, i.e., every polynomial has a root.

Classically, the complex numbers were introduced in the course of solving
real cubic equations. Starting with $x^3+ax+b=0$ one can make a Tschirnhaus
transformation so $a=0$. This is done by introducing a new variable,
$y=cx^2+d$ such that $\sum y_i=\sum {y_i}^2=0$; even when $a$ and $b$ are
real, it may be necessary to choose $c$ complex (the discriminant of the
equation for $c$ is $27b^2+4a^3$). It is negative when the cubic has only
one real root; this can be checked by looking at the product of of the
values of the cubic at its $\min$ and $\max$.

\subsection{Polynomials and rational functions}
Using addition and multiplication we obtain naturally the polynomial
functions $f(z)=\sum_0^n a_nz^2\colon\bbC\to\bbC$. The ring of polynomials
$\bbC[z]$ is an integral domain and a unique factorization domain, since
$\bbC$ is a field. Indeed, since $\bbC$ is algebraically closed, every
polynomial factors into linear terms.

It is useful to add the allowed value $\infty$ to obtain the Riemann sphere
$\widehat\bbC=\bbC\cup\{\infty\}$. Then the rational functions determine
rational maps $f\colon\bbC\to\bbC$. The rational functions $\bbC(z)$ are
the same as the field of fractions for the domain $\bbC[z]$. We set
$f(z)=\infty$ if $q(z)=0$; these points are called the \emph{poles} of $f$.

\subsection{Analysis functions}
Let $U$ be an open set in $\bbC$ and $f\colon U\to\bbC$ a function. We say
that $f$ is \emph{analytic} if
\[
f'(z)=\lim_{t\to\infty}\frac{f(z+)-f(z)}{t}
\]
exists for all $z\in U$. It is important that $t$ approaches $0$ through
arbitrary values of $\bbC$. Remarkably, this condition implies that $f\in
C^\infty$.

%%% Local Variables:
%%% mode: latex
%%% TeX-master: "../Conrads-Notes"
%%% End:

\chapter{McMullen's Topology Notes}

%%% Local Variables:
%%% mode: latex
%%% TeX-master: "../Conrads-Notes"
%%% End:

\chapter{Measure Theory}

%%% Local Variables:
%%% mode: latex
%%% TeX-master: "../Basic-Math"
%%% End:

% \chapter{Functional Analysis}

%%% Local Variables:
%%% mode: latex
%%% TeX-master: "../Basic-Math"
%%% End:

\chapter{Commutative Algebra}

%%% Local Variables:
%%% mode: latex
%%% TeX-master: "../Basic-Math"
%%% End:

\chapter{Algebraic Topology}

%%% Local Variables:
%%% mode: latex
%%% TeX-master: "../Basic-Math"
%%% End:

\chapter{Differential Geometry}

%%% Local Variables:
%%% mode: latex
%%% TeX-master: "../Basic-Math"
%%% End:

\chapter{Algebraic Geometry}
We will talk about some algebraic geometry and then promptly move to linear
algebraic groups.

Here is the preliminary material
\section{Some Algebraic Geometry}
Let $k$ be an algebraically closed field and put $V\coloneqq k^n$. The
elements of the polynomial algebra $S=k[T_1,\dotsc,T_n]$ (abbreviated to
$k[\bfT]$) can be viewed as $k$-valued functions on $V$. We say that $v\in
V$ is a \emph{zero} of $f\in k[\bfT]$ if $f(v)=0$ and that $v$ is a zero of
an ideal $I$ of $S$ if $f(v)=0$ for all $f\in I$. We denote by $\calV(I)$
the set of zeros of the ideal $I$. If $X$ is any subset of $V$, let
$\calI(X)\subset s$ be the deal of the $f\in S$ with $f(v)=0$ for all $v\in
X$.

Recall that the \emph{radical} or nilradical $\sqrt{I}$ of the ideal $I$ is
the idal of the $f\in S$ with $f^n\in I$ for some integer $n\geq 1$. A
\emph{radical ideal} is one that coincides

%%% Local Variables:
%%% mode: latex
%%% TeX-master: "../Basic-Math"
%%% End:

\bibliographystyle{plain}
\bibliography{basics}
\end{document}

%%% Local Variables:
%%% mode: latex
%%% TeX-master: t
%%% End:
