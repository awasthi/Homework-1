\begin{questions}
  \question Let $\calP_2(\bbR)$ be the set of all polynomials of degree
  less than or equal to $2$ with coefficients in $\bbR$, i.e., if
  $p(t)=at^2+bt+c$ is a polynomial in $\calP_2(\bbR)$, then $a,b,c\in\bbR$.
  \begin{parts}
    \part[4] Show that the set $\calP_2(\bbR)$ is closed under addition and
    multiplication by scalars. What is a \emph{zero} for this set?
    \begin{solution}
      Take $p(t)=a_1t^2+b_1t+c_1$, $q(t)=a_2t^2+b_2t+c_2$ in
      $\calP_2(\bbR)$ and $c\in\bbR$, then
      \begin{align*}
        p(t)+q(t)
        &=a_1+t^2+b_1t+c_1+a_2t^2+b_2t+c_2
        &c(pt)
        &=c(a_1t^2+b_1t+c_1)
        \\
        &=(a_1+a_2)t^2+(b_1+b_2)t+(c_1+c_2)&
        &=ca_1+cb_1t+cc_1.
      \end{align*}
      More generally, we can show that $\calP_2(\bbR)$ satisfies all $8$ of
      the vector space axioms; but they are all trivial calculations that
      come down to basically these two facts that $\calP_2(\bbR)$ is closed
      under addition and multiplication by scalars.
    \end{solution}
    \part[4] The set $\calP_2(\bbR)$ is in fact a vector space. Find a
    basis for $\calP_2(\bbR)$.%
    \begin{solution}
      The basis I was looking for was $\{\,1,t,t^2\,\}$. If your basis had
      three linearly independent elements, that should be enough.
    \end{solution}
    \part[12] Define an inner product $\langle \blank,\blank
    \rangle\colon\calP_2(\bbR)\times\calP_2(\bbR)\to\bbR$ by
    \[
      \langle p(t),q(t) \rangle\longmapsto \int_0^1 p(t)q(t)\diff t.
    \]
    Find a polynomial $q\in\calP_2(\bbR)$ such that
    $\langle p,q \rangle=p(1/2)$ for every $p\in\calP_2(\bbR)$.
    [\textsc{Hint}: You should start by looking at the basis you found in
    part (b). If you chose a nice basis $t^2$ should be in your basis. Now
    for a general $q(t)=at^2+bt+c\in\calP_2(\bbR)$ we have
    \[
      \langle t^2,p(t) \rangle=\int_0^1t^2 q(t)\diff t
      =\left(\frac{1}{2}\right)^2=\frac{1}{4}.
    \]
    Can you come up with enough equations to solve for the unknowns $a$,
    $b$, $c$?]
    \begin{solution}
      Let $p(t)=at^2+bt+c$. Using the basis $\{\,1,t,t^2\,\}$ we have
      \begin{align*}
        \int_0^1 at^2+bt+c\diff t
        &=\frac{a}{3}+\frac{b}{2}+c\\
        &=1\\
        \int_0^1 t(at^2+bt+c)\diff t
        &=\int_0^1 at^3+bt^2+ct\diff t\\
        &=\frac{a}{4}+\frac{b}{3}+\frac{c}{2}\\
        &=\frac{1}{2}\\
        \int_0^1 t^2(at^2+bt+c)\diff t
        &=\int_0^1 at^4+bt^3+ct^2\diff t\\
        &=\frac{a}{5}+\frac{b}{4}+\frac{c}{3}\\
        &=\frac{1}{4}
      \end{align*}
      Now, we can write the system above as
      \[
        A=\begin{bmatrix}[ccc|c]
          1/3&1/2&1&1\\
          1/4&1/3&1/2&1/2\\
          1/5&1/4&1/3&1/4
        \end{bmatrix}.
      \]
      Putting $A$ in row reduced echelon form, we have
      \[
        A_{\text{rref}}
        =
        \begin{bmatrix}[ccc|c]
          1&0&0&-15\\
          0&1&0&15\\
          0&0&1&-3/2
        \end{bmatrix}.
      \]
      Thus, the polynomial $q(t)=-15t^2+15t-3/2$.
    \end{solution}
  \end{parts}
\end{questions}

%%% Local Variables:
%%% mode: latex
%%% TeX-master: "MA265-Quiz"
%%% End:
