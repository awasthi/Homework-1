\begin{questions}
  % \question
  % \begin{parts}
  %   \part[2] What does it mean for a collection of vectors
  %   $\{\bfv_1,\ldots,\bfv_n\}$ to be linearly dependent?
  %   \\\\
  %   \part[2] What is a basis?
  %   \\\\
  %   \part[2] Suppose $\{\bfv_1,\ldots,\bfv_n\}$ is a basis for $\bbR^n$, why
  %   isn't $\{\bfv_1,\ldots,\bfv_n,\bfv_{n+1}\}$ a basis for $\bbR^n$?
  % \end{parts}
  \question For the following problems write \textbf{T} for true,
  \textbf{F} for false. You do not need to justify your answers.
  \begin{parts}
    \part[3] For all $m\times n$ matrices $A$ and $B$, $\nullity(A+B)=\nullity
    A+\nullity B$.
    \\
    \part[3] For all $n\times n$ matrices $A$ and $B$, $\nullity(AB)=(\nullity
    A)(\nullity B)$.
    \\
    \part[3] For all $n\times n$ matrices $A$ and $B$, where $A$ is an
    elementary matrix, $\nullity(AB)=\nullity B$.
    \\
    \part[3] If $\bfx_p$ is a solution to the system $A\bfx=\bfb$, then
    $\bfy+\bfx_p$ is also a solution to $A\bfx=\bfb$ for any
    $\bfy\in\Nullspace A$.
  \end{parts}

  \bigskip\bigskip
  \bigskip\bigskip
  \bigskip\bigskip
  \bigskip\bigskip
  \question[8] Prove that if $\bfu$, $\bfv$ and $\bfw$ are in $\bbR^3$ and
  $\bfu$ is orthogonal to both $\bfv$ and $\bfw$, then $\bfu$ is orthogonal
  to every vector in $\Span\{\bfv,\bfw\}$.

  [\emph{Hint}: What does it mean for a vector $\bfx$ to be in
  $\Span\{\bfv,\bfw\}$ and what does it mean for two vectors to be
  orthogonal?]
\end{questions}

%%% Local Variables:
%%% mode: latex
%%% TeX-master: "MA265-Quiz"
%%% End:
