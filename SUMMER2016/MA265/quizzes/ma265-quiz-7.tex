\begin{questions}
  \question For the following problems write \textbf{T} for true,
  \textbf{F} for false. You do not need to justify your answers.
  \begin{parts}
    \part[3] For all $m\times n$ matrices $A$ and $B$, $\nullity(A+B)=\nullity
    A+\nullity B$.
    \part[3] For all $n\times n$ matrices $A$ and $B$, $\nullity(AB)=(\nullity
    A)(\nullity B)$.
    \part[3] For all $n\times n$ matrices $A$ and $B$, where $A$ is an
    elementary matrix, $\nullity(AB)=\nullity B$.
    \part[3] If $\bfx_p$ is a solution to the system $A\bfx=\bfb$, then
    $\bfy+\bfx_p$ is also a solution to $A\bfx=\bfb$ for any
    $\bfy\in\Nullspace A$.
  \end{parts}
  \begin{solution}
    The answers for part (a), (b), (c) and (d) are \textbf{F}, \textbf{F},
    \textbf{T} and \textbf{T} respectively. For part (c) and (d) you should
    refer to Kolman and Hill (particularly Ch.\@ 4.7 on \emph{Homogeneous
      Systems}).

    To see that (a) is false consider the matrices
    \[
      A=\begin{bmatrix}
        1&0\\0&1
      \end{bmatrix}\quad\text{and}\quad
      B=\begin{bmatrix}
        -1&0\\0&-1
      \end{bmatrix}.
    \]
    The nullity of $A$ and $B$ is both $0$, but
    \[
      A+B=\begin{bmatrix}0&0\\0&0\end{bmatrix}
    \]
    which has nullity $2$ and $0+0$ is by no means equal to $2$.

    To see that (b) is false, consider the matrices
    \[
      A=\begin{bmatrix}%
        1&0\\0&0
      \end{bmatrix}\quad\text{and}\quad%
      B=\begin{bmatrix}%
        1&0\\0&1
      \end{bmatrix}.
    \]
    Then nullity of $A$ is $1$ whereas the nullity of $B$ is $0$, but
    \[
      AB=\begin{bmatrix}%
        1&0\\0&0
      \end{bmatrix}
    \]
    which has nullity $1$. Again, $0\cdot 1$ is not equal to $1$.

  \end{solution}
  \question[8] Prove that if $\bfu$, $\bfv$ and $\bfw$ are in $\bbR^3$ and
  $\bfu$ is orthogonal to both $\bfv$ and $\bfw$, then $\bfu$ is orthogonal
  to every vector in $\Span\{\bfv,\bfw\}$.

  [\emph{Hint}: What does it mean for a vector $\bfx$ to be in
  $\Span\{\bfv,\bfw\}$ and what does it mean for two vectors to be
  orthogonal?]
  \begin{solution}
    Starting from the top. We know that $\bfu\cdot\bfv=0$ and
    $\bfu\cdot\bfw=0$. Now, what does it mean for $\bfx$ to be in
    $\Span\{\bfv,\bfw\}$? It means that there exists scalars
    $a_1,a_2\in\bbR$ (both can possibly be $0$) such that
    $\bfx=a_1\bfv+a_2\bfw$. Thus
    \begin{align*}
      \bfu\cdot\bfx&=\bfu\cdot(a_1\bfv+a_2\bfw)\\
                   &=\bfu\cdot (a_1\bfv)+\bfu\cdot(a_2\bfw)\\
                   &=a_1(\bfu\cdot\bfv)+a_2(\bfu\cdot\bfw)\\
                   &=0+0\\
                   &=0.
    \end{align*}
    Thus, $\bfu$ is orthogonal to $\bfx$. Since the choice of $\bfx$ was
    arbitrary, we conclude that $\bfu$ is orthogonal to every vector in
    $\Span\{\bfv,\bfw\}$.
  \end{solution}
\end{questions}

%%% Local Variables:
%%% mode: latex
%%% TeX-master: "MA265-Quiz"
%%% End:
