\begin{questions}
  \question[10] Using the method of elimination, determine if the following
  system of equations has a solution, no solution or many solutions:
  \[
    \label{eq:A}
    \tag{$\star$}
    \begin{aligned}
      x_1+2x_2-3x_3&=-4\\
      2x_1+x_2-3x_3&=4.
    \end{aligned}
  \]
  \begin{solution}
    By looking at the number of free variables vs.\@ equations, we can
    already tell that the system will either have no solutions or many
    solutions (such a system is said to be
    \emph{\href{https://en.wikipedia.org/wiki/Underdetermined_system}{underdetermined}}).

    Take the first equation and subtract it from the second equation to get
    \begin{align*}
      x_1+2x_2-3x_3&=-4\\
      x_1-x_2&=8.
    \end{align*}
    Then $x_1=8+x_2$. Now, substitute this into the first equation
    \[
      (8+x_2)+2x_2-3x_3=-4
    \]
    so $3x_2=-4-8+3x_3$ or $x_2=-4+x_3$. Thus
    \[
      \label{eq:B}
      \tag{$\vardiamondsuit$}
      \begin{aligned}
        x_1&=8+x_2&
        x_2&=-4+x_3.\\
        &=8-4+x_3\\
        &=4+x_3\\
      \end{aligned}
    \]
    Therefore, the system has \textbf{many solutions}; you give me a value
    of $x_3$ and, using the equations in \eqref{eq:B}, I can find values
    for $x_1$ and $x_2$ that solve the system \eqref{eq:A}.
  \end{solution}
  \question[5] Given the matrices
  \[
    A=
    \begin{bmatrix}
      1&-1\\
      -1&1
    \end{bmatrix}
    \qquad
    B=
    \begin{bmatrix}
      1&1\\
      0&1
    \end{bmatrix}
  \]
  find
  \[
    2A+3B^\intercal.
  \]
  \begin{solution}
    The calculation is straightforward
    \begin{align*}
      2A+3B^\intercal
      &=
        2\begin{bmatrix}
          1&-1\\
          -1&1
        \end{bmatrix}
       +3\begin{bmatrix}
         1&1\\
         0&1
       \end{bmatrix}^\intercal\\%
      &=
        \begin{bmatrix}
          2&-2\\
          -2&2
        \end{bmatrix}
      +3\begin{bmatrix}
        1&0\\
        1&1
      \end{bmatrix}\\
      &=\begin{bmatrix}
          2&-2\\
          -2&2
        \end{bmatrix}
      +\begin{bmatrix}
        3&0\\
        3&3
      \end{bmatrix}\\
      &=\begin{bmatrix}
        2+3&-2+0\\
        -2+3&2+3
      \end{bmatrix}\\
      &=\begin{bmatrix}
        5&-2\\
        1&5
      \end{bmatrix}.
    \end{align*}
  \end{solution}
\end{questions}

%%% Local Variables:
%%% mode: latex
%%% TeX-master: "MA265-Quiz"
%%% End:
