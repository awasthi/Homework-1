\newcommand{\tf}[1][{}]{%
\fillin[#1][0.25in]%
}

\begin{questions}
  \question The following are \textbf{T}rue/\textbf{F}alse
  questions. You do not have to justify your answers. The symbols $A$, $B$
  and $C$ represent $m\times n$ matrices and $\bfx=(x_1,\dotsc,x_n)$ and
  $\bfb=(b_1,\dotsc,b_m)$ be $1\times n$ and $1\times m$ vectors,
  respectively.
  \\
  \begin{parts}
    \part[5] \tf[T] If $A$ and $B$ are invertible $n\times n$ matrices
    then $AB$ is invertible.
    \part[5] \tf[T] Suppose $A$ and $B$ are $n\times n$ matrices such that
    $AB=I$ then $BA=I$.
    \part[5] \tf[F] Suppose $AB=BA$ then $B$ must equal $A^{-1}$.
    \part[5] \tf[T] Suppose $AB=AC$ then $B$ must equal $C$.
    \part[5] \tf[T] Suppose $AA^\rmT=I$ then $\det(A)=\pm 1$.
  \end{parts}
  \newpage
  \question[10] Suppose $A$ is a skew-symmetric matrix $3\times 3$ matrix ,
  i.e., $A^\rmT=-A$. Which of the following is \textbf{not} true:
  \\
  \begin{choices}
    \choice $
    \left[
      \begin{smallmatrix}
        0&2&-1\\
        -2&0&-4\\
        1&4&0
      \end{smallmatrix}
    \right]$ is skew-symmetric%
    \choice $\det(A^\rmT)=\pm 1$%
    \choice $\det(A)=0$%
    \choice All of the above are true
  \end{choices}
  \newpage
  \question[10] Suppose $A$ is an idempotent matrix, i.e., $A^2=A$. Which
  of the following is true:
  \\
  \begin{choices}
    \choice $A$ must be the identity matrix $I$
    \choice ${(A^\rmT)}^2=A$
    \choice $\det(A)=0$
    \choice All of the above are true
    \choice None of the above are true
  \end{choices}
  \newpage
  \question[10] Consider the matrix
  \[
    A\coloneq
    \begin{bmatrix}
      0&0&2&0\\
      2&-1&0&0\\
      1&-1&0&0\\
      -1&1&0&1
    \end{bmatrix}.
  \]
  Which of the following statements is true:
  \\
  \begin{choices}
    \choice The columns are linearly independent
    \choice The matrix is not invertible
    \choice The matrix has determinant $-2$
    \choice None of the above
  \end{choices}
  \newpage
  \question[10] Consider the following system of equations
  \begin{align*}
    x_1+x_3&=5\\
    x_1-x_2-x_3&=6\\
    x_2+x_3&=7.
  \end{align*}
  The above system of linear equations is:
  \\
  \begin{choices}
    \choice Inconsistent
    \choice Consistent with a unique solution
    \choice Consistent with infinitely many solutions
    \choice None of the above
  \end{choices}
  \newpage
  \question Set
  \[
    A\coloneq
    \begin{bmatrix}
      1&2\\1&1
    \end{bmatrix}
    \quad\text{and}\quad
    B\coloneq
    \begin{bmatrix}
      3&5\\
      -1&2
    \end{bmatrix}.
  \]
  \\
  \begin{parts}
    \part[5] Compute $A^{-1}$.
    \part[5] Compute $ABA^{-1}$.
    \part[5] Compute $\det (ABA^{-1})$.
    \part[10] What is the relationship between $\det(B)$ and
    $\det(ABA^{-1})$ and why?
  \end{parts}
  % \newpage
  % \question[15] Recall that an $m\times n$ matrix $B$ is equivalent to an
  % $m\times n$ matrix $A$ if there exist nonsingular matrices $P$ and $Q$
  % such that
  % \[
  %   B=PAQ.
  % \]
  % Show that
  % \\
  % \begin{parts}
  %   \part[5] Every matrix $A$ is equivalent to itself.
  %   \part[5] If $B$ is equivalent to $A$, then $A$ is equivalent to $B$.
  %   \part[5] If $C$ is equivalent to $B$ and $B$ is equivanent to $A$, then
  %   $C$ is equivalent to $A$.
  % \end{parts}
\end{questions}

% \question[10] For which rational numbers $\lambda$ does the homogeneous
% system
% \begin{align*}
%   x+(\lambda-3)y&=0\\
%   (\lambda-3)x+y&=0
% \end{align*}
% have a nontrivial solution?
% \newpage

%%% Local Variables:
%%% mode: latex
%%% TeX-master: "MA265-Quiz"
%%% End:
