\begin{questions}
  \question Which of the following are \emph{not} a basis for the vector
  space of all symmetric $2\times 2$ matrices? Why? [\textsc{Hint}: Recall
  that a symmetric matrix must satisfy $A=A^\intercal$.]
  \begin{choices}
    \choice%
    $\displaystyle\left\{\,%
      \begin{bmatrix}1&0\\0&0\end{bmatrix},%
      \begin{bmatrix}0&0\\0&1\end{bmatrix},
      \begin{bmatrix}1&0\\0&0\end{bmatrix}%
      \,\right\}$.%
    \CorrectChoice%
    $\displaystyle\left\{\,%
      \begin{bmatrix}1&1\\1&0\end{bmatrix},
      \begin{bmatrix}1&2\\2&-3\end{bmatrix},
      \begin{bmatrix}0&1\\1&1\end{bmatrix}%
      \,\right\}$.%
    \CorrectChoice%
    $\displaystyle\left\{\,
      \begin{bmatrix}1&0\\0&0\end{bmatrix},
      \begin{bmatrix}-1&-1\\-1&-1\end{bmatrix},%
      \begin{bmatrix}1&0\\0&1\end{bmatrix}%
      \,\right\}$.%
  \end{choices}
  \begin{solution}
    The correct choices are marked in \hilight{\textbf{red}}. Here is the
    rationale that accompanies it. We know that a symmetric matrix has the
    property that $A=A^\intercal$. That is, if
    \[
      A=\begin{bmatrix} a&b\\c&d
      \end{bmatrix}
    \]
    then
    \[
      A=
      \begin{bmatrix}
        a&b\\c&d
      \end{bmatrix}
      =
      \begin{bmatrix}
        a&c\\b&d
      \end{bmatrix}
      =
      \begin{bmatrix}
        a&b\\c&d
      \end{bmatrix}^{\intercal} =A^{\intercal}.
    \]
    This forces $b=c$. Then we can replace our original matrix $A$ by one
    that looks like
    \[
      A=\begin{bmatrix}
        a&b\\
        b&d
      \end{bmatrix}= a\begin{bmatrix}
        1&0\\
        0&0
      \end{bmatrix}+ b\begin{bmatrix}
        0&1\\
        1&0
      \end{bmatrix}+ d\begin{bmatrix}
        0&0\\
        0&1
      \end{bmatrix}
    \]
    since we know $b=c$ must be true for any $2\times 2$ symmetric matrix.
    Thus, one basis for the set of all $2\times 2$ symmetric matrices is
    the set
    \[
      \label{eq:A}
      \tag{$\star$} \left\{
        \begin{bmatrix}1&0\\0&0\end{bmatrix},
        \begin{bmatrix}0&1\\1&0\end{bmatrix},
        \begin{bmatrix}0&0\\0&1\end{bmatrix} \right\}.
    \]
    The rest of the problem comes down to taking the set of vectors for
    options A, B and C and trying to reduce them to the set in
    \eqref{eq:A}.
  \end{solution}

  \question Which of the following are \emph{not} a basis for $\bbR^3$?
  Why?
  \begin{choices}
    \CorrectChoice%
    $\displaystyle\left\{\,
      \begin{bmatrix}1\\2\\3\end{bmatrix},
      \begin{bmatrix}3\\2\\1\end{bmatrix},
      \begin{bmatrix}0\\0\\1\end{bmatrix}%
      \,\right\}$.%
    \choice%
    $\displaystyle\left\{\,
      \begin{bmatrix}1\\2\\3\end{bmatrix},
      \begin{bmatrix}3\\2\\1\end{bmatrix}%
      \,\right\}$.%
    \CorrectChoice%
    $\displaystyle\left\{\,
      \begin{bmatrix}0\\2\\-1\end{bmatrix},
      \begin{bmatrix}1\\1\\1\end{bmatrix},
      \begin{bmatrix}2\\5\\0\end{bmatrix}%
      \,\right\}$.%
  \end{choices}
  \begin{solution}
    The correct choices are marked in \hilight{\textbf{red}}.

    Since $\dim\bbR^3=3$ we know that B can't possible be a basis for
    $\bbR^3$ so we immediately disqualify it.

    So that leaves A and B.

    For A, let
    \[
      A=\begin{bmatrix}%
        1&2&3\\
        3&2&1\\
        0&0&1
      \end{bmatrix}
    \]
    and find $A_{\text{rref}}$ (which can be done very quickly). Then
    \[
      A_{\text{rref}}=%
      \begin{bmatrix}%
        1&0&0\\
        0&1&0\\
        0&0&1
      \end{bmatrix}
    \]
    is full rank. Thus, the set in A is a basis for $\bbR^3$.

    For B we follow the same procedure. Let
    \[
      A=\begin{bmatrix}%
        0&2&-1\\
        1&1&1\\
        2&5&0
      \end{bmatrix}.
    \]
    Then, doing some real quick row operations gets you to
    \[
      A_{\text{rref}}=%
      \begin{bmatrix}%
        1&0&0\\
        0&1&0\\
        0&0&1
      \end{bmatrix}
    \]
    which is full rank. Thus, the set in B is also a basis for $\bbR^3$.
  \end{solution}
\end{questions}

%%% Local Variables:
%%% mode: latex
%%% TeX-master: "MA265-Quiz"
%%% End:
