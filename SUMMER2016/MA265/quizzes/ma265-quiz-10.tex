\begin{questions}
  \question Let $T\colon\bbR^3\to\bbR^3$ be a linear map that sends
  \begin{align*}
    T(1,0,0)&=(3,2,4)\\
    T(0,1,0)&=(2,0,2)\\
    T(0,0,1)&=(4,2,3).
  \end{align*}
  \begin{parts}
    \part[4] Find the value of $T(2,1,-1)$.
    \begin{solution}
      Since $T$ is a linear map, we know that
      $T(\bfv+\bfw)=T(\bfv)+T(\bfw)$ and $T(c\bfv)=cT(\bfv)$ so
      \begin{align*}
        T(2,1,-1)&=T((2,0,0)+(0,1,0)+(0,0,-1))\\
                 &=T(2,0,0)+T(0,1,0)+T(0,0,-1)\\
                 &=2T(1,0,0)+T(0,1,0)-T(0,0,1)\\
                 &=2(3,2,4)+(2,0,2)-(4,2,3)\\
                 &=(6,4,8)+(2,0,2)+(-4,-2,-3)\\
                 &=(6+2-4,4+0-2,8+2-3)\\
                 &=(4,2,7).
      \end{align*}
    \end{solution}
    \part[6] Find the matrix representation of $T$ with respect to the
    standard basis on $\bbR^3$.
    \begin{solution}
      \sloppy Using the standard basis on $\bbR^3$ which, by the way, is
      the set $\left\{(1,0,0),(0,1,0),(0,0,1)\right\}$, for a general
      vector
      \[
        \bfx=x_1(1,0,0)+x_2(0,1,0)+x_3(0,0,1),
      \]
      we have
      \[
        T(\bfx)=(3x_1+2x_2+4x_3,2x_1+0x_2+2x_3,4x_1+2x_2+3x_3).
      \]
      Thus,
      \begin{align*}
        \begin{bmatrix}
          A_{11}&A_{12}&A_{13}\\
          A_{21}&A_{22}&A_{23}\\
          A_{31}&A_{32}&A_{33}
        \end{bmatrix}
        \begin{bmatrix}
          x_1\\
          x_2\\
          x_3
        \end{bmatrix}&=
        \begin{bmatrix}
          A_{11}x_1+A_{12}x_2+A_{13}x_3\\
          A_{21}x_1+A_{22}x_2+A_{23}x_3\\
          A_{31}x_1+A_{32}x_2+A_{33}x_3
        \end{bmatrix}\\
        &=
        \begin{bmatrix}
          3x_1+2x_2+4x_3\\2x_1+0x_2+2x_3\\4x_1+2x_2+3x_3
        \end{bmatrix}.
      \end{align*}
      This tells us that the matrix must be
      \[
        A=\begin{bmatrix}
          3&2&4\\
          2&0&2\\
          4&2&3
        \end{bmatrix}.
      \]
    \end{solution}
    \part[10] Using the matrix representation of $T$, find the
    characteristic polynomial. \textbf{You do not have to simplify it.}
    \begin{solution}
      To find the minimal polynomial of $T$, we find
      \begin{align*}
        \det(A-\lambda I)
        &=\det\begin{bmatrix}
          3-\lambda&2&4\\
          2&0-\lambda&2\\
          4&2&3-\lambda
        \end{bmatrix}\\
        &=(3-\lambda)\det\begin{bmatrix}0-\lambda&2\\2&3-\lambda\end{bmatrix}%
                                                        -2\det\begin{bmatrix}2&2\\4&3-\lambda\end{bmatrix}\\
        &\phantom{{}={}}+4\det\begin{bmatrix}2&0-\lambda\\4&2\end{bmatrix}\\
        &=(3-\lambda)(\lambda^2-3\lambda-4)-2(6-2\lambda-8)+4(4+4\lambda)\\
        &=(3\lambda^2-9\lambda-12-\lambda^3+3\lambda^2+4\lambda)\\
        &\phantom{{}={}}+(-12+4\lambda+16)\\
        &\phantom{{}={}}+(16+16\lambda)\\
        &=-\lambda^3+6\lambda^2+15\lambda+8.
      \end{align*}
      Using this equation, we can find the eigenvalues of $T$. One thing
      you can do is first try to find a root of
      $\lambda^3-6\lambda^2-15\lambda-8$. As it turns out, $\lambda=-1$ is
      a root since
      \[
        (-1)^3-6(-1)^2-15(-1)-8=-1-6+15-8=-15+15=0.
      \]
      So, using long division, we can factor $(\lambda+1)$ from
      $\lambda^3-6\lambda^2-15\lambda-8$ and so
      \[
        \lambda^3-6\lambda^2-15\lambda-8=(\lambda+1)(\lambda^2-7\lambda-8).
      \]
      And you can do it again since $(-1)^2-7(-1)-8=1+7-8=0$ so
      \[
        \lambda^3-6\lambda^2-15\lambda-8=(\lambda+1)^2(\lambda-8).
      \]
      Thus, the eigenvalues are $-1$ and $8$.
    \end{solution}
  \end{parts}
\end{questions}

%%% Local Variables:
%%% mode: latex
%%% TeX-master: "MA265-Quiz"
%%% End:
