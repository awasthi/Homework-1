\begin{questions}
  \question%
  Consider the matrix
  \[
    \label{eq:A}
    \tag{$\bigstar$}
    A=\begin{bmatrix}
      1&1&2\\
      1&2&1\\
      2&3&3\\
      0&-1&-1
      \end{bmatrix}.
  \]
  \begin{parts}
    \part[12] Recall that the \textbf{nullspace} of an $m\times n$ matrix
    $A$ is the set of vectors $\bfx$ in $\bbR^m$ such that
    $A\bfx=\mathbf{0}$. This subset spans a subspace of $\bbR^m$. Give a
    description of the nullspace of the matrix \eqref{eq:A} by writing down
    basis for the nullspace.

    [\textsc{Hint}: You should begin by putting the matrix in rref.]
    \begin{solution}
      We know from Kolman and Hill that elementary row operations do not
      change the nullspace of a matrix. Therefore, our first step should be
      to find the row-reduced echelon form $A$; call it
      $A_{\text{rref}}$. After doing some calculations to the side, we get
      that
      \[
        A_{\text{rref}}=
        \begin{bmatrix}
          1&0&0\\
          0&1&0\\
          0&0&1\\
          0&0&0
        \end{bmatrix}.
      \]
      Now, $\bfx=(x_1,x_2,x_3,x_4)$ is in the nullspace of
      $A_{\text{rref}}$ if $A_{\text{rref}}\bfx=(0,0,0,0)$. When does this
      happen? Well
      \[
        A_{\text{rref}}\bfx=%
        \begin{bmatrix}
          1&0&0\\
          0&1&0\\
          0&0&1\\
          0&0&0
        \end{bmatrix}
        \begin{bmatrix}
          x_1\\x_2\\x_3\\x_4
        \end{bmatrix}=
        \begin{bmatrix}
          x_1\\x_2\\x_3\\0
        \end{bmatrix}
        =\begin{bmatrix}
          0\\0\\0\\0
        \end{bmatrix}
      \]
      so $x_1=x_2=x_3=0$. This forces $\bfx$ to be $(0,0,0,0)$ so the
      nullspace of $A_{\text{rref}}$ (which is the same as the nullspace of
      $A$) is $\{\mathbf{0}\}$, hence, it has no basis.
    \end{solution}
    \part[8] The \textbf{range} or \textbf{columnspace} of an $m\times n$
    matrix $A$ is the set of vectors $\bfy$ in $\bbR^n$ that are, in some
    sense, ``hit'' by vectors $\bfx$ in $\bbR^n$ by the matrix $A$, i.e.,
    $\bfy=A\bfx$ for some $\bfx$. Using your calculations from above (the
    hint), write down a basis for the range of \eqref{eq:A}.
    \begin{solution}
      Using the row reduced echelon of the matrix above, we see that
      \[
        A_{\text{rref}}\bfx=%
        \begin{bmatrix}
          1&0&0\\
          0&1&0\\
          0&0&1\\
          0&0&0
        \end{bmatrix}
        \begin{bmatrix}
          x_1\\x_2\\x_3\\x_4
        \end{bmatrix}=
        \begin{bmatrix}
          x_1\\x_2\\x_3\\0
        \end{bmatrix}=
        x_1\begin{bmatrix}
          1\\0\\0\\0
        \end{bmatrix}%
        +
        x_2
        \begin{bmatrix}
          0\\1\\0\\0
        \end{bmatrix}+
        x_3
        \begin{bmatrix}
          0\\0\\1\\0
        \end{bmatrix}.
      \]
      Hence, the set $\left\{
        \left[\begin{smallmatrix}
          1\\0\\0\\0
        \end{smallmatrix}\right],
      \left[\begin{smallmatrix}
          0\\1\\0\\0
        \end{smallmatrix}\right],
        \left[\begin{smallmatrix}
          0\\0\\1\\0
        \end{smallmatrix}\right]
      \right\}$ is a basis for the range. \textbf{Note} this is not a basis
      for $\bbR^4$ which is a $4$-dimensional vector space.
    \end{solution}
  \end{parts}
\end{questions}

%%% Local Variables:
%%% mode: latex
%%% TeX-master: "MA265-Quiz"
%%% End:
