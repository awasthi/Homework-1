\begin{questions}
  %% Question 1
  \question[12] List all the properties a set $V$ must
  satisfy in order to be a vector space.
  \\\\
  (\emph{Hint:} there are eight of them.)
  \begin{solution}
    The axioms that a \emph{real} vector space (i.e., a vector space with
    scalar in $\bbR$, because you can have \emph{complex} vector spaces
    with scalars in the complex or \emph{imaginary} numbers $\bbC$) must
    satisfy are
    \begin{enumerate}
    \item associativity of addition:
      $\bfu+\bfv+\bfw=\bfu+(\bfv+\bfw)=(\bfu+\bfv)+\bfw$;
    \item commutativity of addition: $\bfv+\bfw=\bfw+\bfu$;
    \item additivie identity: there exists some element (called a
      \emph{zero}) $\mathbf{0}\in V$ such that for any $\bfv\in V$,
      $\bfv+\mathbf{0}=\mathbf{0}+\bfv=\bfv$.
    \item additive inverse: for every $\bfv\in V$ there exist an inverse
      element $\bfw\in V$ such that $\bfv+\bfw=\mathbf{0}$. \textbf{Note}
      the vector $\bfw$ is not always $-\bfv$, it just so happens that in
      the linear algebra we cover it will almost always be $-\bfv$;
    \item compatibility of scalar multiplication: if $a,b\in\bbR$,
      $a(b\bfv)=(ab)\bfv$ for all $\bfv\in V$;
    \item multiplicative identity: there exist element $1\in\bbR$ such that
      $1\bfv=\bfv$ for all $\bfv\in V$;
    \item distributivity with respect to vector addition: if $a\in\bbR$,
      $\bfv,\bfw\in V$ then $a(\bfv+\bfw)=a\bfv+a\bfw$;
    \item distributivity with respect to scalar addition: if $a,b\in\bbR$,
      $\bfv\in V$ then $(a+b)\bfv=a\bfv+a\bfw$.
    \end{enumerate}
  \end{solution}
  \question Which of the following subsets $W$ of $\bbR^3$ are subspaces?
  \begin{parts}
    \part[2]
    $\displaystyle W=
    \left\{\,\left[\begin{smallmatrix}x\\y\\z\end{smallmatrix}\right]:x\leq y\leq
      z\,\right\}$.
    \part[2]
    $\displaystyle W=
    \left\{\,\left[\begin{smallmatrix}x\\y\\z\end{smallmatrix}\right]:x+y+z=0\,\right\}$.
    \part[2]
    $\displaystyle W=
    \left\{\,\left[\begin{smallmatrix}x\\y\\z\end{smallmatrix}\right]:x^2+y^2+z^2=1\,\right\}$.
    \part[2]
    $\displaystyle W=
    \left\{\,\left[\begin{smallmatrix}x+2y+3z\\z\\0\end{smallmatrix}\right]:x,y,z\in\bbR\,\right\}$.
  \end{parts}
  (By now you should have a feel of what a vector spaces is so you do not
  need to check all of the conditions; but for those that are not
  subspaces, give me a reason, e.g., the set is not closed under addition,
  multiplication by scalars, etc.)
  \begin{solution}
    The set in (a) is not subspace and this is easy to see because
    $[-1,0,1]$ is in the set, but $-1[-1,0,1]=[1,0,-1]$ is not since
    $1\nleq 0\nleq -1$.

    The set in (b) is a vector space because, first, it is nonempty since
    it contains $[0,0,0]$ and if $c\in\bbR$,
    $[x_1,y_1,z_1],[x_2,y_2,z_2]\in W$ then
    \begin{align*}
      c\begin{bmatrix}
        x_1\\
        y_1\\
        z_1
      \end{bmatrix}
      &=\begin{bmatrix}
        cx_1\\
        cy_1\\
        cz_1
      \end{bmatrix},&
      \begin{bmatrix}
        x_1\\
        y_1\\
        z_1
      \end{bmatrix}+
      \begin{bmatrix}
        x_2\\
        y_2\\
        z_2
      \end{bmatrix}
     &=\begin{bmatrix}
        x_1+x_2\\
        y_1+y_2\\
        z_1+z_2
      \end{bmatrix}
    \end{align*}
    and both satisfy
    \begin{align*}
      cx_1+cy_1+cz_1
      &=c(x_1+y_1+z_1)&
      x_1+x_2+y_1+y_2+z_1+z_2
      &=(x_1+y_1+z_1)+(x_2+y_2+z_2)\\
      &=0&&=0.
    \end{align*}

    The set in (c) is the sphere. It is clearly not a subspace since
    $[1,0,0],[0,1,0]\in W$, but $[1,0,0]+[0,1,0]=[1,1,0]$ has norm
    $1^2+1^2+0^2=2\neq 0$ hence $[1,1,0]\notin W$.

    Lastly, the set in (d) is a subspace because, first, it is nonempty
    since $[0,0,0]\in W$ and if $c\in\bbR$, $[x_1+2y_1+3z_1,z_1,0],[x_2+2y_2+3z_2,z_2,0]\in
    W$ then
    \begin{align*}
      c\begin{bmatrix}
        x_1+2y_1+3z_1\\
        z_1\\
        0
      \end{bmatrix}
      &=\begin{bmatrix}
        c(x_1+2y_1+3z_1)\\
        cz_1\\
        c\cdot 0
      \end{bmatrix}\\
      &=\begin{bmatrix}
        cx_1+2cy_1+3cz_1\\
        cz_1\\
        0
      \end{bmatrix}\label{eq:A}\tag{$\clubsuit$}\\
            \begin{bmatrix}
        x_1+2y_1+3z_1\\
        z_1\\
        0
      \end{bmatrix}+
      \begin{bmatrix}
        x_2+2y_2+3z_2\\
        z_2\\
        0
      \end{bmatrix}
     &=\begin{bmatrix}
       x_1+2y_1+3z_1+x_2+2y_2+3z_2\\
       z_1+z_2\\
       0
     \end{bmatrix}\\
      &=\begin{bmatrix}
        (x_1+x_2)+2(y_1+y_2)+3(z_1+z_2)\\
        (z_1+z_2)\\
        0
      \end{bmatrix}\label{eq:B}\tag{$\spadesuit$}
    \end{align*}
    Putting $x=cx_1,y=cy_1,z=cz_1$ for the entries in \eqref{eq:A}, we see
    that the vector in \eqref{eq:A} indeed belongs to $W$ since it has the
    shape $[x+2y+3z,z,0]$. Putting $x=x_1+x_2,y=y_1+y_2,z=z_1+z_2$ we see
    that the vector in \eqref{eq:B} also belongs to $W$. Thus, $W$ is a
    subspace.
  \end{solution}

\end{questions}

%%% Local Variables:
%%% mode: latex
%%% TeX-master: "MA265-Quiz"
%%% End:
