\def\class{MA 265}
\def\examnum{1}
\def\term{Summer 2016}
\def\examdate{\today}
\def\auth{Carlos Salinas}
\def\email{salinac@purdue.edu}
\def\subject{linear algebra}
\def\tight{MA 265: Quiz \examnum}
% \def\timelimit{1:00 hr}

\documentclass[11pt,letterpaper,addpoints]{exam}%
% \usepackage[top=1in, bottom=1.5in, left=1in, right=1in]{geometry}%

%% Base customization and variables
\usepackage[libertine]{clos-vars}

%% Formatting

%% Footnote style
\renewcommand*{\thefootnote}{\fnsymbol{footnote}}

%% Hyperref setup
\hypersetup{%
  breaklinks,%
  colorlinks=true,%
  linkcolor=black,%
  citecolor=black,%
  filecolor=black,%
  menucolor=black,%
  runcolor=black,%
  urlcolor=black,%
  pdftitle={\tight},%
  pdfauthor={\auth},%
  pdfkeywords={\subject},%
  pdfsubject={\class},%
  pageanchor={false}%
}

%% Graphics path as it says
\graphicspath{{figures/}}

%% Commands
\DeclareMathOperator{\Nullspace}{Nullspace}
\DeclareMathOperator{\Rowspace}{Rowspace}

\begin{document}
% %% Footnote style
% \renewcommand*{\thefootnote}{\fnsymbol{footnote}}

% % These commands set up the running header on the top of the exam pages
% \pagestyle{head}
% \firstpageheader{}{}{}
% \runningheader{\class}{\examnum\ - Page \thepage\ of \numpages}{\examdate}
% \runningheadrule

% \begin{flushright}
% \begin{tabular}{p{2.8in} r l}
% \textbf{\class} & \textbf{Name (Print):} & \makebox[2in]{\hrulefill}\\
% \textbf{\term} \\% &Instructor:& Tatsunari Watanabe\\
% \textbf{Midterm \examnum} \\% &TA:& Carlos Salinas\\
% \textbf{\examdate} &&\\
% \textbf{Time Limit: \timelimit} &
% \end{tabular}\\
% \end{flushright}
% \rule[1ex]{\textwidth}{.1pt}

% This exam contains \numpages\ pages (including this cover page) and
% \numquestions\ problems.  Check to see if any pages are missing.  Enter
% all requested information on the top of this page, and put your initials
% on the top of every page, in case the pages become separated.\\

% You may \textit{not} use your books, notes, or any calculator on this exam.\\

% You are required to show your work on each problem on this exam.  The following rules apply:\\

% \begin{minipage}[t]{3.7in}
% \vspace{0pt}
% \begin{itemize}

% \item \textbf{If you use a ``fundamental theorem'' you must indicate this} and explain
% why the theorem may be applied.

% \item \textbf{Organize your work}, in a reasonably neat and coherent way, in
% the space provided. Work scattered all over the page without a clear ordering will
% receive very little credit.

% \item \textbf{Mysterious or unsupported answers will not receive full
% credit}.  A correct answer, unsupported by calculations, explanation,
% or algebraic work will receive no credit; an incorrect answer supported
% by substantially correct calculations and explanations might still receive
% partial credit.


% \item If you need more space, use the back of the pages; clearly indicate when you have done this.
% \end{itemize}

% Do not write in the table to the right.
% \end{minipage}
% \hfill
% \begin{minipage}[t]{2.3in}
% \vspace{0pt}
% %\cellwidth{3em}
% \gradetablestretch{2}
% \vqword{Problem}
% \addpoints % required here by exam.cls, even though questions haven't started yet.
% \gradetable[v]%[pages]  % Use [pages] to have grading table by page instead of question

% \end{minipage}
% \newpage % End of cover page
\noindent Instructor: Tatsunari Watanabe
\hfill \textbf{Name:\underline{\phantom{My name way too long}}.}\\
\noindent TA: \href{mailto:\email}{\auth}
\\\\
\begin{center}
  {\Large \textbf{MA 26500-215 \quad Quiz \examnum}}

  \bigskip

  \today
\end{center}

\printanswers
\CorrectChoiceEmphasis{\color{Red!85!black}\bfseries}

\input{ma265-quiz-\examnum}
\end{document}

%%% Local Variables:
%%% mode: latex
%%% TeX-master: t
%%% End:
