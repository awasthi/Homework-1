\begin{questions}
  \question[6] Find the least squares solution $\bar\bfx$ of the system
  $A\bar{\bfx}=\bar{\bfb}$ where
  \[
    A=
    \begin{bmatrix}
      0&1\\
      0&0\\
      -1&0
    \end{bmatrix},
    \qquad
    \bar{\bfb}=
    \begin{bmatrix}
      2\\1\\3
    \end{bmatrix}.
  \]
  %
  \begin{solution}
    First, compute all the necessary matrices and vectors
    \begin{align*}
      A^\rmT A
      &=\begin{bmatrix}
        0&1\\
        0&0\\
        -1&0
      \end{bmatrix}^\rmT
      \begin{bmatrix}
        0&1\\
        0&0\\
        -1&0
      \end{bmatrix}
      \\
      &=\begin{bmatrix}
        0&0&-1\\
        1&0&0
      \end{bmatrix}
      \begin{bmatrix}
        0&1\\
        0&0\\
        -1&0
      \end{bmatrix}
      \\
      &=\begin{bmatrix}
        1&0\\
        0&1
      \end{bmatrix}\\
      A^\rmT\bar\bfb
      &=\begin{bmatrix}
        1&1&0\\
        0&0&-1
        \end{bmatrix}
       \begin{bmatrix}
         2\\1\\3
       \end{bmatrix}\\
      &=\begin{bmatrix}
        3\\-2
        \end{bmatrix}.
    \end{align*}
    Then, to find $\bar\bfx$ we compute
    \begin{align*}
      \bar\bfx
      &={(A^\rmT A)}^{-1}(A^\rmT\bar\bfb)\\
      &=
      \begin{bmatrix}
        1&0\\
        0&1\\
      \end{bmatrix}^{-1}
      \begin{bmatrix}
        3\\-2
      \end{bmatrix}\\
      &=\begin{bmatrix}
        3\\-2
        \end{bmatrix}.
    \end{align*}
  \end{solution}
  \question[4] Suppose that $A$ and $B$ are conjugate matrices. Show that
  if $\lambda$ is an eigenvalue of $A$ then it is an eigenvalue of $B$.
  %
  \begin{solution}
    Suppose that $\lambda$ is an eigenvalue of $A$ and that $A$ is
    conjugate to $B$. Then, $\lambda$ is an eigenvalue of $A$ means that
    there exists a vector (the associated eigenvector) $\bfx$ such that
    $A\bfx=\lambda\bfx$; while $A$ is conjugate to $B$ means that there
    exists an invertible matrix $P$ such that $A=PBP^{-1}$. Thus,
    \[
      PBP^{-1}\bfx=A\bfx=\lambda\bfx
    \]
    so
    \begin{align*}
      PBP^{-1}\bfx
      &=\lambda\bfx\\
      BP^{-1}\bfx&=P^{-1}\lambda\bfx\\
      &=\lambda P^{-1}\bfx\\
      \intertext{now let $\bfy=P^{-1}\bfx$ and we have}
      B\bfy&=\lambda\bfy.
    \end{align*}
    So $\lambda$ is an eigenvalue of $B$ with associated eigenvector
    $\bfy=P^{-1}\bfx$.
  \end{solution}
  \question[8] Suppose that $P$ is an idempotent matrix, i.e., $P^2=P$. Show
  that the only possible eigenvalues for $P$ are $\lambda=0$ and
  $\lambda=1$.
  \begin{solution}
    Suppose that $P$ is an idempotent matrix and $\lambda$ is an eigenvalue
    of $P$. Then $P\bfx=\lambda\bfx$ for some eigenvector $\bfx\neq
    \mathbf{0}$. Now, since we have
    \[
      P^2\bfx=P\bfx
    \]
    then
    \begin{align*}
      P^2\bfx&=P(P\bfx)\\
      P\bfx&=\lambda P\bfx\\
      \lambda\bfx&=\lambda^2\bfx.
    \end{align*}
    Thus, $\lambda^2=\lambda$ so
    $\lambda^2-\lambda=\lambda(\lambda-1)=0$. Thus, $\lambda=0$ or
    $\lambda=1$.
  \end{solution}
\end{questions}
%%% Local Variables:
%%% mode: latex
%%% TeX-master: "MA265-Quiz"
%%% End:
