\def\documentauthor{Carlos Salinas}
\def\documenttitle{MA571: Qual Problems}
% \def\hwnum{1}
\def\shorttitle{MA571 Quals}
\def\coursename{MA571}
\def\documentsubject{point-set topology}
\def\authoremail{salinac@purdue.edu}

\documentclass[10pt,showtrims,twoside]{memoir}
\usepackage{geometry}
\usepackage[dvipsnames]{xcolor}
\usepackage[
    breaklinks,
    bookmarks=true,
    colorlinks=true,
    pageanchor=false,
    linkcolor=black,
    citecolor=black,
    filecolor=black,
    menucolor=black,
    runcolor=black,
    urlcolor=black,
    % linkcolor=violet!85!black,
    % citecolor=YellowOrange!85!black,
    % urlcolor=Aquamarine!85!black,
    hyperindex=false,
    hyperfootnotes=true,
    pdftitle={\shorttitle},
    pdfauthor={\documentauthor},
    pdfkeywords={\documentsubject},
    pdfsubject={\coursename}
    ]{hyperref}
\usepackage{natbib}
\usepackage[toc,acronym,section=section]{glossaries}

%% Math
\usepackage{amsmath}
\usepackage{amsthm}
\usepackage{amssymb}
\usepackage{mathtools}
% \usepackage{eucal}
% \usepackage{mathrsfs}
% \usepackage[nointegrals]{wasysym}

% %% Language
% \usepackage{cmap}
% \usepackage[LAE,LFE,T2A,T1]{fontenc}
% \usepackage[utf8]{inputenc}
% \usepackage[farsi,french,german,spanish,russian,english]{babel}
% \babeltags{fr=french,
%            de=german,
%            en=english,
%            es=spanish,
%            pa=farsi,
%            ru=russian
%            }
% \def\spanishoptions{mexico}

% \selectlanguage{english}

% \newcommand{\textfa}[1]{\beginR\textpa{#1}\endR}

% \usepackage{CJKutf8}
% \newcommand{\textkr}[1]{\begin{CJK}{UTF8}{mj}#1\end{CJK}}
% \newcommand{\textjp}[1]{\begin{CJK}{UTF8}{min}#1\end{CJK}}
% \newcommand{\textzh}[1]{\begin{CJK}{UTF8}{bsmi}#1\end{CJK}}

%% Misc
\usepackage{graphicx}
\graphicspath{{figures/}}

\usepackage{microtype}
\usepackage{lineno}
\usepackage{multicol}
\usepackage[inline]{enumitem}
\usepackage{listings}
\usepackage{mleftright}
\mleftright
\usepackage{carlos-variables}

%% Unicode math and Polyglossia
\usepackage{unicode-math}
% \usepackage{unicode-minionmath}

% \setmainfont{CMU Serif}
% \setsansfont{CMU Sans Serif}
% \setmonofont{CMU Typewriter Text}
\setmainfont[Ligatures=TeX]{Latin Modern Roman}
\setsansfont{Latin Modern Sans}
\setsansfont{Latin Modern Mono}
\setmathfont{Latin Modern Math}

% \setmainfont[Ligatures=TeX]{Libertinus Serif}
% \setsansfont{Libertinus Sans}
% \setmonofont{Libertinus Mono}
% \setmathfont{Minion Math}
% \setmathfont[range={\mathfrak}]{XITS Math}
% \setmathfont[range={\mathcal,\mathbfcal},StylisticSet=1]{XITS Math}
% \setmathfont[range=\mathscr]{XITS Math}
% % \setmathfont[range={\mathfrak}]{latinmodern-math.otf}
% % \setmathfont[range={\mathcal}]{latinmodern-math.otf}
% % \setmathfont[range={\mathscr}]{latinmodern-math.otf}
% \setmathfont[range={}]{Minion Math}

\usepackage{polyglossia}
\usepackage[english]{selnolig}

\newfontfamily\cyrillicfont[Script=Cyrillic]{CMU Serif}
\newfontfamily\cyrillicfontsf[Script=Cyrillic]{CMU Sans Serif}
\newfontfamily\cyrillicfonttt[Script=Cyrillic]{CMU Typewriter Text}

% \newfontfamily\cyrillicfont[Script=Cyrillic]{Libertinus Serif}
% \newfontfamily\cyrillicfontsf[Script=Cyrillic]{Libertinus Sans}

% \newfontfamily\farsifont[Script=Arabic,
%                          Scale=MatchUppercase]{Amiri}

\setmainlanguage[variant=american]{english}
% \setotherlanguage{farsi}
\setotherlanguage{french}
\setotherlanguage[spelling=new,latesthyphen,babelshorthands]{german}
\setotherlanguage{spanish}
\setotherlanguage[spelling=modern,babelshorthands]{russian}

% \makeatletter
% \@Latintrue
% \makeatother

% \usepackage{xeCJK}
% \usepackage[overlap]{ruby}
% \renewcommand\rubysep{-0.2ex}
% \setCJKmainfont[BoldFont=IPAGothic]{IPAMincho}

% \xeCJKDeclareSubCJKBlock{Kana}{"3040 -> "309F, "30A0 -> "30FF, "31F0 -> "31FF, "1B000 -> "1B0FF}
% \xeCJKDeclareSubCJKBlock{Hangul}{"1100 -> "11FF, "3130 -> "318F, "A960 -> "A97F, "AC00 -> "D7AF, "D7B0 -> "D7FF}

% \setCJKmainfont{HanaMinA}
% \setCJKmainfont[Kana]{HanaMinA}
% \setCJKmainfont[Hangul]{NanumMyeongjo}
% \setCJKsansfont[Hangul]{NanumGothic}

% \usepackage{luatexja-fontspec}
% \setmainjfont{IPAMincho}
% \setsansjfont{IPAGothic}

% %% Theorems and definitions
%% remove parentheses
% \makeatletter
% \def\thmhead@plain#1#2#3{%
%   \thmname{#1}\thmnumber{\@ifnotempty{#1}{ }\@upn{#2}}%
%   \thmnote{ {\the\thm@notefont#3}}}
% \let\thmhead\thmhead@plain
% \makeatother

\theoremstyle{plain}
\newtheorem{theorem}{Theorem}
\newtheorem{proposition}[theorem]{Proposition}
\newtheorem{corollary}[theorem]{Corollary}
\newtheorem{claim}[theorem]{Claim}
\newtheorem{lemma}[theorem]{Lemma}
\newtheorem{axiom}[theorem]{Axiom}

\newtheorem*{corollary*}{Corollary}
\newtheorem*{claim*}{Claim}
\newtheorem*{lemma*}{Lemma}
\newtheorem*{proposition*}{Proposition}
\newtheorem*{theorem*}{Theorem}

\theoremstyle{definition}
\newtheorem{definition}{Definition}
\newtheorem{example}{Examples}
\newtheorem{examples}[example]{Example}
\newtheorem{exercise}{Exercise}[subsection]
\newtheorem{problem}[exercise]{Problem}

\newtheorem*{example*}{Example}
\newtheorem*{exercise*}{Exercise}
\newtheorem*{problem*}{Problem}

\makeindex

\begin{document}
%% Footnotes
\renewcommand*{\thefootnote}{\fnsymbol{footnote}}

%% Counters
\setsecnumdepth{subsection}
\counterwithout{exercise}{chapter}
\numberwithin{equation}{subsection}
\counterwithout{equation}{chapter}

%% Header-footer
\chapterstyle{veelo}

%% Redefine the QED symbol
% \renewcommand\qedsymbol{\ensuremath{\null\hfill\QED}}

\thispagestyle{empty}
\author{\href{mailto:\authoremail}{\documentauthor}}
\title{\documenttitle}
\date{\today}

\frontmatter
\maketitle
\tableofcontents

%% Munkres notes

%% McClure Homework
\mainmatter
\chapter{MA 571 Fall 2015}
\thispagestyle{empty}
\bigskip
\section{Homework}
This is material from the course MA 533 as taught in the fall of 2015.
Most of the homework is from \cite{munkres} with a few exercises
(especially those involving the quotient topology and manifolds) written by
McClure.
\subsection{Homework 1}

%%% Local Variables:
%%% mode: latex
%%% TeX-master: "../MA571-Quals"
%%% End:

\subsection{Homework 2}

%%% Local Variables:
%%% mode: latex
%%% TeX-master: "../MA571-Quals"
%%% End:

\subsubsection{Homework 3}
\setcounter{exercise}{0}


%%% Local Variables:
%%% mode: latex
%%% TeX-master: "../MA571-Quals"
%%% End:

\subsubsection{Homework 4}
\setcounter{exercise}{0}


%%% Local Variables:
%%% mode: latex
%%% TeX-master: "../MA571-Quals"
%%% End:

\subsubsection{Homework 5}

%%% Local Variables:
%%% mode: latex
%%% TeX-master: "../MA571-Quals"
%%% End:

\subsection{Homework 6}

%%% Local Variables:
%%% mode: latex
%%% TeX-master: "../MA571-Quals"
%%% End:

\subsection{Homework 7}

%%% Local Variables:
%%% mode: latex
%%% TeX-master: "../MA571-Quals"
%%% End:

\subsubsection{Homework 8}
\setcounter{exercise}{0}


%%% Local Variables:
%%% mode: latex
%%% TeX-master: "../MA571-Quals"
%%% End:

\subsubsection{Homework 9}
\setcounter{exercise}{0}


%%% Local Variables:
%%% mode: latex
%%% TeX-master: "../MA571-Quals"
%%% End:

\include{mcclure/571-hw-10}
\subsubsection{Homework 11}
\setcounter{exercise}{0}


%%% Local Variables:
%%% mode: latex
%%% TeX-master: "../MA571-Quals"
%%% End:

\subsubsection{Homework 12}

%%% Local Variables:
%%% mode: latex
%%% TeX-master: "../MA571-Quals"
%%% End:

% \subsubsection{Homework 13}

%%% Local Variables:
%%% mode: latex
%%% TeX-master: "../MA571-Quals"
%%% End:


%% McClure Quals
\section{Past Qualifying Examinations}
\subsection{MA 571: Qualifying Exam, August 2014}
\setcounter{exercise}{0}
\begin{problem}
  Let $X$ be a topological space, let $A$ be a subset of $X$, and let $U$
  be an open subset of $X$. Prove that
  $U\cap \bar A\subseteq\overline{U\cap A}$.
\end{problem}
\begin{solution}
  Let $x\in U\cap\bar A$. Then $x\in U$ and $x\in\bar A$. This means that,
  since $U$ is open, by Lemma C there exist an open neighborhood $V$ of $x$
  such that $V\subseteq U$. Moreover, since $x\in\bar A$,
  $V'\cap A\neq\emptyset$ for every open neighborhood $V'$ of $x$. In
  particular, $V\cap A\neq\emptyset$. Thus, we have $V\cap U\neq\emptyset$
  and $V\cap A\neq\emptyset$ so $V\cap(U\cap A)\neq\emptyset$.
\end{solution}

\begin{problem}
Let $X$ be the following subspace of $\bbR^2$:
\[
((0,1]\times[0,1])\cup([2,3)\times[0,1]).
\]
Let $\sim$ be the equivalence relation on $X$ with $(1,t)\sim(2,t)$ (that
is $(s,t)\sim(s',t')\iff(s,t)=(s',t')$ or $t=t'$ and $\{s,s'\}=\{1,2\}$;
you do \emph{not} have to prove that this is an equivalence
relation). Prove that $X/{\sim}$ is homeomorphic to
$(0,2)\times[0,1]$. (\emph{Hint}: construct maps in both directions).
\end{problem}
\begin{solution}
We shall proceed by the hint. Let $q\colon X\to X/{\sim}$ denote the
quotinet map. Then, for $(x,y)\in X$, we define the map

We shall proceed by the hint. Let $q\colon X\to X/{\sim}$ denote the
quotient map. Then, for $x\in X$, we define the map
\[
h(s,t)=
\begin{cases}
(s,t)&\text{if $(s,t)\in(0,1]\times[0,1]$}\\
(s-1,t)&\text{if $(s,t)\in(2,3]\times[0,1])$}
\end{cases}
\]
from $X\to(0,2)\times[0,1]$.

By the UMP of the quotient space (Theorem Q.3), if we can show that $h$ is
continuous and preserves the equivalence relation, the induced map on the
quotient space, $h'\colon X/{\sim}\to (0,2)\times[0,1]$ will be
continuous. To that end, we will use the pasting lemma. First, note that
$(0,1]\times[0,1]$ and $[2,3)\times[0,1]$ are closed subsets of $X$ since
$(0,1]\times[0,1]$ is the complement of $((1,\infty)\times (-2,2))\cap X$
which is open in $X$ (since $X$ inherits its topology from $\bbR^2$),
similarly, $[2,3)\times[0,1]$ is closed in $X$ since it is the complement
of $((-\infty,2)\times(-2,2))\cap X$ which is open in $X$ for the same
reasons. It is clear that the maps $x\mapsto x$ and $x\mapsto x-1$ are
continuous onto their image, since the latter is nothing more than the
inclusion map and the former is nothing more than subtraction, which is
continuous by Theorem 21.5. Thus, by the pasting lemma, $h$ is continuous.

Now we show that $h$ does in fact preserve the equivalence
relation. Suppose $(s,t)\sim(s',t')$. Then either $(s,t)=(s',t')$ or $t=t'$
and $s,s'\in\{1,2\}$. In the former case, we have $h(s,t)=h(s',t')$
(whether $(s,t),(s',t')\in(0,1]\times[0,1]$ or its complement). In the
latter case, we may, without loss of generality, assume that $(s,t)=(1,t)$
and $(s',t')=(2,t)$. Then $h(s,t)=(1,t)=(2-1,t)=h(s',t')$. Thus, by Theorem
Q.3, the induced map $h'\colon X/{\sim}\to(0,2)\times[0,1]$ is
continuous. Moreover, the map is bijective with inverse
\[
(h')^{-1}=
\begin{cases}
[s,t]&\text{if $x\in (0,1]$}\\
[s+1,t]&\text{if $x\in [1,2)$}
\end{cases}.
\]
This is clearly an inverse as
\[
h'\circ (h')^{-1}=\id_{X/{\sim}}
\]
and
\[
(h')^{-1}\circ h'=\id_{(0,2)\times[0,1]}.
\]
Thus, by Theorem 26.6, $h'$ is a homeomorphism.
\end{solution}

\begin{problem}
Prove that there is an equivalence relation $\sim$ on the interval $[0,1]$
such that $[0,1]/{\sim}$ is homeomorphic to $[0,1]\times[0,1]$. As part of
your solution \emph{explain} how you are using one or more properties of the
quotient topology.
\end{problem}
\begin{solution}
First, it suffices to find a continuous surjective map $f\colon[0,1]\to
[0,1]\times[0,1]$ and quotient out by the preimage of every point
$x\in[0,1]\times[0,1]$. These maps are hard to describe in general, but
they exists (take for example a space-filling curve). Next, note that if
$C$ is a closed subset of $[0,1]$ then it is compact so $f(C)$ is
compact. But since $[0,1]\times[0,1]$ is compact Hausdorff, then
$f(C)\subseteq[0,1]\times[0,1]$ will be closed. It follows by that $f$ will
be a Munkres quotient map, so by Theorem Q.4, $f'\colon
[0,1]/{\sim}\to[0,1]\times[0,1]$ is a homeomorphism for some equivalence
relation $\sim$ on $[0,1]$.
\end{solution}

\begin{problem}
Let $D$ be the closed unit disk in $\bbR^2$, that is, the set
\[
\left\{\,(x,y):x^2+y^2\leq 1\,\right\}.
\]
Let $E$ be the open unit disk
\[
\left\{\,(x,y):x^2+y^2<1\,\right\}.
\]
Let $X$ be the one-point compactification of $E$, and let $f\colon D\to X$
be the map defined by
\[
f(x,y)=
\begin{cases}
(x,y)&\text{if $x^2+y^2<1$}\\
\infty&\text{if $x^2+y^2=1$.}
\end{cases}
\]
Prove that $f$ is continuous.
\end{problem}
\begin{solution}
By the section on the one-point-compactification, it suffices to check two
cases of open sets (1) all sets $U$ open in $E$, and (2) all sets of the
form $U=X\smallsetminus C$ containing the point at infinity, $\infty$, where $C$ is
compact. In the first case, it is clear that $f$ is continuous since it is
just the inclusion map and is in fact bijective on $E$. For the second
case, suppose that $U$ is a neighborhood of $\infty$. Then $Y-U$ is a
compact subset of $E$, hence closed since $X$ is a compact Hausdorff
space. But since $f$ is bijective, continuous on $E$, then $f^{-1}(X-U)$ is
a closed subset of $E$. Thus, by theorem 18.2, $f$ is continuous.
\end{solution}

\begin{problem}
Let $X$ and $Y$ be homotopy-equivalent topological spaces. Suppose that $X$
is path-connected. Prove that $Y$ is path-connected.
\end{problem}
\begin{solution}
% Suppose that $X$ is homotopy-equivalent to $Y$. Then there exists a
% continuous maps $f\colon X\to Y$ and $g\colon Y\to X$ such that $g\circ
% f\simeq\id_X$ and $g\circ f\simeq\id_Y$. Now, since $X$ is path-connected,
% its image is path-connected (as we will show shortly) thus, it suffices to
% show that for any point $y\in Y$, there exists a path $p\colon I\to Y$ from
% $p(0)=y$ to $p(1)\in f(X)$. Let $y\in Y\smallsetminus f(X)$.
First we prove the following important result:
\begin{lemma}
\label{lem:path-connected-image}
Path-connectedness is a topological property, i.e., if $X$ is
path-connected and $f\colon X\to Y$ is a continuous map then, $f(X)$ is
path connected.
\end{lemma}
\begin{solution}
\renewcommand\qedsymbol{$\clubsuit$}
Since $X$ is path-connected, for any pair of points $x,x'\in X$ there
exists a continuous map $p\colon [0,1]\to X$ such that $p(0)=x$ and
$p(1)=x'$. Since composition of continuous maps is continuous, $f\circ
p\colon[0,1]\to Y$ is a path from $f(x)$ to $f(x')$. Since this property
holds for any $y\in f(X)$, it follows that $f(X)$ is path-connected.
\end{solution}
Now, suppose that $X$ is homotopy-equivalent to $Y$. Then there exists
continuous maps $f\colon X\to Y$ and $g\colon Y\to X$ such that $g\circ
f\simeq\id_X$ and $f\circ g\simeq\id_Y$. Now, since $X$ is path-connected,
by Lemma (\ref{lem:path-connected-image}) we have $f(X)$ is
path-connected. Thus, it suffices to show that for every point $y\in Y$
there exists a path $p\colon [0,1]\to Y$ from $y$ to some point $y'\in
f(X)$. Now, since $f\circ g\simeq\id_Y$, there exists a homotopy, say
$H\colon Y\times[0,1]\to Y$ such that $H(s,0)=f\circ g(s)$ and
$H(s,1)=s$. Consider the evaluation $H_y= H(y,t)\circ H(y,t)$ where
the map $(y,t)\colon [0,1]\to Y\times[0,1]$ is the imbedding of $[0,1]$ at
$y$ (which is continuous by Theorem 18.4) thus, $H_y$ is
continuous. Moreover, $H_y(0)=f\circ g(y)\in f(Y)$ and
$H_y(1)=\id_Y(y)=y$ so $H_y$ is a path from $y$ to a point $f\circ g(y)$ in
$f(X)$. Since we can do this for any point $y\in Y$, it follows, since
path-connectedness is an equivalence relation, that $Y$ is path-connected.
\end{solution}
\begin{problem}
Let $a$ and $b$ denote the points $(-1,0)$ and $(1,0)$ in $\bbR^2$. Let
$x_0$ denote the origin $(0,0)$. Use the Seifert--van Kampen theorem to
calculate $\pi_1\left(\bbR^2\smallsetminus\{a,b\},x_0\right)$. You may not use any other
method.
\end{problem}
\begin{solution}
We'll use Theorem 70.2's version of the Seifert--van Kampen theorem. Define
\[
U=\left(-\infty,\tfrac{1}{2}\right)\times\bbR
\quad\text{and}\quad
V=\left(-\tfrac{1}{2},\infty\right)\times\bbR.
\]
Then $U\cap V=(-1/2,1/2)\times\bbR$ is clearly path-connected since it is a
convex set. Moreover, note that $U\simeq\bbR^2\smallsetminus\{x_0\}$ and
$V\simeq\bbR^2\smallsetminus\{x_0\}$ (in the case of $U$, first consider the
homeomorphism $(x,y)\mapsto(x+1,y)$ which sends $a$ to $(0,0)$ and then the
homotopy $(x,y)\mapsto\tfrac{1}{t}(x,ys)$).

Once we have established the above, since the fundamental group of a space
is invariant under homotopy-equivalence,
$\pi_1(U,x_0)\cong\pi_1\left(\bbR^2\smallsetminus\{x_0\},y_0\right)\cong\bbZ$ for some
arbitrary $y_0\neq x_0$ and similarly $\pi_1(V,x_0)\cong\bbZ$. Thus, by the
classical version of the Seifert--van Kampen theorem
\[
\pi_1\left(\bbR^2\smallsetminus\{a,b\},x_0\right)\cong
\frac{\bbZ*\bbZ}{N}
\]
where $N$ is the least normal subgroup
\end{solution}

\begin{problem}
Let $p\colon E\to B$ be a covering map with $B$ locally connected, and let
$x\in B$. Prove that $x$ has a neighborhood $W$ with the following
property: for every connected component $C$ of $p^{-1}(W)$, the map
$p\colon C\to W$ is a homeomorphism.
\end{problem}
\begin{solution}
Let $U$ be an evenly covered neighborhood of $x$. Then
$p^{-1}(U)=\bigcup_\alpha V_\alpha$ where the $V_\alpha$ are open in $E$
and  $V_\alpha\cap V_\beta=\emptyset$ whenever $\alpha\neq\beta$. For any
$\alpha$, let $C$ be a connected component of $p^{-1}(U)$ containing
$p^{-1}(x)\cap V_\alpha$ (the latter is a one point set since
$\left.p\right|_{V_\alpha}$ is a bijection). Then $C\subseteq V_\alpha$ for
at most one such $\alpha$ for otherwise $C\cap V_\beta\neq\emptyset$ for
some $\beta\neq\alpha$, so $C\cap V_\beta$ and $C\cap V_\alpha$ form a
separation of (note that $C\smallsetminus(C\cap V_\beta)=C\cap V_\alpha$ and
vice-versa thus, $C\cap V_\beta$ and $C\cap V_\alpha$ are open and closed
in the subspace topology on $C$, conversely) by Lemma 23.1.

Thus, $p(C)\subseteq U$ is connected by Theorem 23.5. Moreover, since
$V_\alpha\supseteq C$ is homeomorphic to $U$ by the restriction
$\left.p\right|_{V_\alpha}$, $p(C)$ is a connected component of $U$ as the
following lemma shows
\begin{lemma}
Suppose $C$ is a connected component of $X$ and $h\colon X\to Y$ is a
homeomorphism. Then $h(C)$ is a connected component of $Y$.
\end{lemma}
\begin{solution}[Solution of lemma]
\renewcommand\qedsymbol{$\clubsuit$}
Let $C$ be a connected component of $X$. By theorem 23.5, $h(C)$ is a
connected subset of $Y$, moreover, is open. By Theorem 25.1, $h(C)$ is
contained in a connected component of $Y$, say $D$. Hence, we must show
that $D\subseteq h(C)$. Now, since $h$ is a homeomorphism, $h^{-1}(D)$ is a
connected subset of $X$, by Theorem 23.5, so is contained in only one
component of $X$. But $h^{-1}(D)\cap C\neq\emptyset$ so $h^{-1}(D)\subseteq
C$. Thus, since $h$ is a set-bijection, $D\subseteq h(C)$.
\end{solution}
so by Theorem 25.3, $p(C)$ is open in $B$ since $B$ is locally
connected. Thus, the restriction $\left.p\right|_{C}$ is a homeomorphism
onto its image $W= p(C)$, by Lemma A, which is a neighborhood of $x$.
\end{solution}

%%% Local Variables:
%%% mode: latex
%%% TeX-master: "../MA571-Quals"
%%% End:

\include{mcclure/571-jan-14}
\subsection{McClure: Winter 2012}
\setcounter{exercise}{0}

\begin{problem}
  Let \(X\) be a topological space. Recall that a subset of \(X\) is
  \emph{dense} if its closure is \(X\). Prove that the intersection of two
  dense open sets is dense.
\end{problem}
\begin{solution}
  Suppose \(U\) and \(V\) are open dense subsets of \(X\). We will show
  that \(U\cap V\) is dense in \(X\), i.e., \(\overline{U\cap V}=X\). To
  that end, we will show that for any point \(x\in X\), for any
  neighborhood \(W\) of \(x\), \(W\cap(U\cap V)\neq\emptyset\). Therefore,
  let \(x\in X\). Let \(W\) be a neighborhood of \(x\). Then, since \(U\)
  is dense in \(X\), \(W\cap U\neq\emptyset\). Let \(y\in W\cap U\). Then,
  since \(U\) and \(V\) are open, \(U\cap V\) is open so \(U\cap V\) is a
  neighborhood of \(y\). Moreover, since \(V\) is dense in \(X\),
  \((W\cap U)\cap V\neq\emptyset\). Now, since intersection is associative,
  \((W\cap U)\cap V=W\cap(U\cap V)\neq\emptyset\). Thus,
  \(x\in\overline{U\cap V}\) and we have \(\overline{U\cap V}=X\) as
  desired.
\end{solution}

\begin{problem}
  Let \(X\) be a set with two elements \(\{a,b\}\). Give \(X\) the
  \emph{indiscrete} topology. Give \(X\times\bbR\) the product
  topology. Let \(A\subset X\times\bbR\) be
  \((\{a\}\times[0,1])\cup(\{b\}\times(0,1))\). Prove that \(A\) is
  compact.

  You may use the fact that a set is compact if every covering by
  \emph{basic} open sets has a finite subcovering.
\end{problem}
\begin{solution}
  Let \(\calU\) be an open cover of \(A\) by basic open sets. Then each
  \(U\in\calU\) is of the form \(\{a,b\}\times V\) where \(V\) is an open
  subset of \(\bbR\). Then, the \(V\)'s, i.e., \(\pi_2(U)\) where
  \(\pi_2\colon X\times\bbR\to\bbR\) is an open map by previous work, form
  open cover of \([0,1]\) (since \(\bigcup_{U\in\calU} U\supset A\), we
  must have \(\bigcup_{U\in\calU}\pi_2(U)\supset[0,1]\). Now, since
  \([0,1]\) is compact in \(X\) there is a finite collection of the
  \(V\)'s, say \(\left\{V_1,\dotsc,V_n\right\}\), that cover
  \([0,1]\). Call \(U_i\) the element of \(\calU\) such that
  \(\pi_2(U_i)=V_i\). Then the \(U_i\)'s form a finite subcover of
  \(A\). Thus, \(A\) is compact.
\end{solution}

\begin{problem}
  Let \(B^2\) be the disk
  \[
    \left\{\,(x,y)\in\bbR^2:x^2+y^2\leq 1\,\right\}.
  \]
  Let \(S^1\) be the circle
  \[
    \left\{\,(x,y)\in\bbR^2:x^2+y^2=1\,\right\}.
  \]
  Prove that there is an equivalence relation \(\sim\) such that \(B^2\) is
  homeomorphic to \((S^1\times[0,1])/{\sim}\). As port of your solution
  explain how you are using one or more properties of the quotient
  topology.
\end{problem}
\begin{solution}
  Such an equivalence relation is called the cone of \(S^1\). We define it
  as follows, let \((x,y,z),(x',y',z')\in S^1\times[0,1]\) then we say
  \((x,y,z)\sim(x',y',z')\) if and only if \((x,y)=(x',y')\) or
  \(z=z'=0\). We shall take it on faith that \(\sim\) is in fact an
  equivalence relation (we may return to this if time permits).

  By the UMP of the quotient space, we need to find a continuous surjection
  \(f\colon S^1\times[0,1]\to B^2\) that preserves the equivalence relation
  \(\sim\). So consider the map \(f(x,y,z)=(zx,zy)\). This map is
  continuous by Theorem 18.4 since \(\pi_1\circ f(x,y,z)=zx\) is
  multiplication on \(\bbR\) and similarly for \(\pi_2\circ
  f(x,y,z)\). Moreover, this map preserves the equivalence relation: let
  \((x,y,z)\sim(x',y',z')\) then \((x,y,z)=(x',y',z')\) in which case
  \[
    f(x,y,z)=(zx,zy)=(z'x',z'y')=f(x',y',z')
  \]
  or \(z=z'=0\) so
  \[
    f(x,y,0)=(0\cdot x,0\cdot y)=(0,0)=(0\cdot x',0\cdot y')=f(x',y',0).
  \]
  In either case, we have \(f(x,y,z)=f(x',y',z')\). Thus, by the UMP of the
  quotient space, the induced map
  \(f'\colon (S^1\times[0,1])/{\sim}\to B^2\) is continuous.

  Now, since \(S^1\times[0,1]\) is closed and bounded, by Heine--Borel,
  \(S^1\times[0,1]\) is a compact subset of \(\bbR^3\). Therefore,
  \((S^1\times[0,1])/{\sim}\) is compact. Since \(B^2\subset\bbR^2\) is
  Hausdorff, it suffices to show, by Theorem 26.6, that \(f\) is bijective.

  It is eassy to see that \(f\) is surjective since for any point
  \((x,y)\neq(0,0)\) in \(B^2\), \(\sqrt{x^2+y^2}\leq 1\) so letting
  \(z=\sqrt{x^2+y^2}\), \(x'=x/\sqrt{x^2+y^2}\), and
  \(y'=y/\sqrt{x^2+y^2}\) we have
  \[
    f(x',y',z)=\sqrt{x^2+y^2} \left(\frac{x}{\sqrt{x^2+y^2}},
      \frac{y}{\sqrt{x^2+y^2}}\right)=(x,y).
  \]
  And, trivially, if \((x,y)=0\), we have \(\varphi(x,y,0)=0\) for any
  \((x,y)\in S^1\).

  To see that it is injective, merely note that, by the definition of
  \(f\), \(f(x,y,z)=f(x',y',z')\) if and only if \((x,y,z)=(x',y',z')\) or
  \(z=z'=0\) which precisely means that \((x,y,z)\sim(x',y',z')\). Thus,
  \(f\) is injective.

  It follows that \((S^1\times[0,1])/{\sim}\approx B^2\).
\end{solution}

\begin{problem}
  Let \(X\) be a set with \(2\) elements \(\{a,b\}\). Give \(X\) the
  \emph{discrete} topology. Let \(Y\) be any topological space. Recall that
  \(\scrC(X,Y)\) denotes the set of continuous functions from \(X\) to
  \(Y\), with the compact-open topology. Prove that \(\scrC(X,Y)\) is
  homeomorphic to \(Y\times Y\) (with the product topology).
\end{problem}
\begin{solution}
  Consider the map \(F\colon\scrC(X,Y)\to Y\times Y\) given by
  \(F(f)=(f(a),f(b))\). This map is continuous by Theorem 18.4, since
  \(\pi_1(F)\) and \(\pi_2(F)\) are, respectively, the evaluation of \(f\)
  at \(a\) and the evaluation of \(f\) and \(b\), both of which are
  continuous because under the compact-open topology. This map is clearly
  surjective since for any \((y_1,y_2)\in Y\times Y\) we may define the
  function \(f(a)= y_1\) and \(f(b)= y_2\) which is continuous sinced \(X\)
  has the discrete topology. Moreover, \(F\) is injective since if
  \((f(a),f(b))=(g(a),g(b))\) then \(f(x)=g(x)\) for all \(x\in X\) hence,
  \(f=g\). Therefore, to show that \(F\) is a homeomorphism, it suffices to
  show that \(F\) is an open map.

  Now it suffices to find a continuous inverse. For any
  \((y_1,y_2)\in Y\times Y\), define the map
  \(g\colon Y\times Y\to\scrC(X,Y)\).
  \[
    g(y_1,y_2)= f (y)=\begin{cases}
      a&\text{if \(y=y_1\)}\\
      b&\text{if \(y=y_2\)}.
    \end{cases}
  \]

\end{solution}

\begin{problem}
  Let \(X\) and \(Y\) be homotopy-equivalent topological spaces. Suppose
  that \(X\) is path-connected. Prove that \(Y\) is path-connected.
\end{problem}
\begin{solution}
\end{solution}

\begin{problem}
  Suppose that \(X\) is a wedge of two circles: that is, \(X\) is a
  Hausdorff space which is a union of two subspaces \(A_1\) and \(A_2\)
  such that \(A_1\) and \(A_2\) are each homeomorphic to \(S^1\) and
  \(A_1\cap A_2\) is a single point \(p\).

  Use the Seifert--van Kampen theorem to calculate \(\pi_1(X,p)\). You
  should state what deformation retractions you are using, but you do not
  have to give formulas for them.
\end{problem}
\begin{solution}
\end{solution}

\begin{problem}
  Let \(p\colon E\to B\) be a covering map. Let \(A\) be a connected space
  and let \(a\in A\). Prove that if two continuous functions
  \(\alpha,\beta\colon A\to E\) have a property that \(\alpha(a)=\beta(a)\)
  and \(p\circ\alpha=p\circ\beta\) then \(\alpha=\beta\).

  For partial credit, you may assume that \(p\) is the standard covering
  map from \(\bbR\) to \(S^1\).
\end{problem}
\begin{solution}
\end{solution}

Here's an extra problem I felt like doing since I thought it might be on
the exam:
\begin{problem*}
  \begin{theorem*}[Munkres, Theorem 18.4]
    Let \(f\colon A\to X\times Y\) be given by the equation
    \(f(a)=(f_1(a),f_2(a))\). Then \(f\) is continuous if and only if
    \(f_1\colon A\to X\) and \(f_2\colon A\to Y\) are continuous.
  \end{theorem*}
\end{problem*}
\begin{solution}
  Let \(\pi_1\colon X\times Y\to X\) and \(\pi_2\colon X\times Y\to Y\) be
  projections onto the 1st and 2nd factors, respectively. These maps are
  continuous and open by previous work. Now, for every \(a\in A\) we have
  \[
    \pi_1(f(a))=f_1(a)\qquad\text{and}\qquad \pi_2(f(a))=f_2(a).
  \]
  Therefore, if \(f\) is continuous, then \(f_1\) and \(f_2\) are the
  composites of the continuous functions above therefore, are continuous.

  Conversely, suppose that \(f_1\) and \(f_2\) are continuous. By Lemma C,
  it suffices to show that for each basic open set
  \(U\times V\subset X\times Y\), the preimage \(f^{-1}(U\times V)\) is
  open. But \(a\in f^{-1}(U\times V)\) if and only if
  \(f(a)\in U\times V\), if and only if \(f_1(a)\in U\) and
  \(f_2(a)\in V\). Thus, \(f^{-1}(U\times V)=f^{-1}(U)\cap f^{-1}(V)\)
  which is open in \(A\) since \(U\) is open in \(X\) and \(V\) is open in
  \(Y\) and \(f_1,f_2\) are continuous.
\end{solution}

%%% Local Variables:
%%% mode: latex
%%% TeX-master: "../MA571-Quals"
%%% End:

\subsection{McClure: Winter 2011}
\setcounter{exercise}{0}

\begin{problem}
  Let $A$ be a subset of a topological space $X$ and let $B$ be a subset of
  $A$. Prove that $\bar A\setminus\bar B\subset\overline{A\setminus B}$.
\end{problem}
\begin{solution}
\end{solution}

\begin{problem}
  Let $G$ be a topological group (thas is, a group with a topology for
  which the group operations are continuous) and let $H$ be a subgroup of
  $G$. Suppose that $G$ is connected, that $H$ is a normal subgroup of $G$,
  and that the subspace topology on $H$ is discrete. Prove that
  $g h=hg$ for every $g\in G$, $h\in G$.
\end{problem}
\begin{solution}
\end{solution}

\begin{problem}
  Let $X$ be the space with two points and the discrete topology. Let
  $Y=\prod_{n=1}^\infty X$, with the product topology. What are the
  connected components of $Y$? Prove that your answer is correct.
\end{problem}
\begin{solution}
\end{solution}

\begin{problem}
  Let $X$ and $Y$ be topological spaces. Let $x_0\in X$ and let $C$ be a
  compact subset of $Y$. Let $N$ be an open set in $X\times Y$ containing
  $\left\{x_0\right\}\times C$. Prove that there is an open set $U$
  containing $x_0$ and an open set $V$ containing $C$ such that
  $U\times V\subset N$.
\end{problem}
\begin{solution}
\end{solution}

\begin{problem}
  Let $X$ and $Y$ be homotopy-equivalent topological spaces. Suppose that
  $X$ is connected. Prove that $Y$ is connected.
\end{problem}
\begin{solution}
\end{solution}

\begin{problem}
  Let $p\colon E\to B$ be a covering map. Let $e_0\in E$ and $b_0\in B$
  with $p(e_0)=b_0$. Let $Y$ be simply connected (in particular, $Y$ is
  path-connected). Let $y_0\in Y$. Let $f\colon Y\to B$ be continuous, with
  $f(y_0)=b_0$.

  Prove that the following function $g\colon Y\to E$ is well-defined: Given
  $y\in Y$, choose a path $\gamma$ from $y_0$ to $y$; let $\beta$ be the
  lift of $f\circ\gamma$ to $E$ starting at $e_0$; now define
  $g(y)=\beta(1)$.

  You may use the fact (without proving it) that the lift of a path
  homotopy is again a path homotopy.
\end{problem}
\begin{solution}
\end{solution}

\begin{problem}
  Let $S^2$ be the $2$-sphere, that is, the following subspace of $\bbR^3$:
  \[
    \left\{\,(x,y,z)\in\bbR^3:x^2+y^2+z^2=1\,\right\}.
  \]
  Let $x_0$ be the point $(0,0,1)\in S^2$.

  Use the Seifert--van Kampen theorem to prove that $\pi_1(S^2,x_0)$ is the
  trivial group. You may use either of the two versions of the Seifert--van
  Kampen theorem given in Munkres's book. You will not get credit for any
  other method.
\end{problem}
\begin{solution}
\end{solution}

%%% Local Variables:
%%% mode: latex
%%% TeX-master: "../MA571-Quals"
%%% End:


%% Kaufmann quals
% \chapter{MA571 (Midterm 2015)}
\begin{problem}
Prove that a function to a product space is continuous if and only if its
components are.
\end{problem}
\begin{proof}
\end{proof}

\begin{problem}
Prove that a subspace is closed if and only if it contains all of its limit
points.
\end{problem}
\begin{proof}
\end{proof}

\begin{problem}
Prove that the projection maps for a product are open maps.
\end{problem}
\begin{proof}
\end{proof}

\begin{problem}
Prove that $\partial A=\emptyset$ if and only if $A$ is open and closed.
\end{problem}
\begin{proof}
\end{proof}

\begin{problem}
Prove that a metric space satisfies the 1st countability axiom.
\end{problem}
\begin{proof}
\end{proof}

\begin{problem}
Prove that $\bfR^\omega$ is not metrizable in the box topology.
\end{problem}
\begin{proof}
\end{proof}

\begin{problem}
Show that the diagonal map is not continuous in the box topology, but it is
in the product topology.
\end{problem}
\begin{proof}
\end{proof}

\begin{problem}
Prove the sequence lemma.
\end{problem}
\begin{proof}
\end{proof}

\begin{problem}
Give an example of a surjective map of spaces that is not a quotient map.
\end{problem}
\begin{proof}
\end{proof}

\begin{problem}
Prove that if $f_n$ is a sequence of functions $X\to\bfR$ considered as
elements of $X^{\bfR}$ with the product topology, then $f_n\to f$ if and
only if for each $x\in X$ the sequence $f_n(x)$ converges to the point
$f_n(x)$.
\end{problem}
\begin{proof}
\end{proof}

\begin{problem}
Prove that if $f_n$ is a sequence of functions $X\to\bfR$ considered as
elements of $X^{\bfR}$ with the topology induced by the uniform metric
$\bar\rho$, then $f_n\to f$ if and only if the sequence of functions
$f_n$ converges uniformly to the point $f$. (Recall that $f_n\colon X\to
Y$, with $Y$ a metric space, uniformly converges to $f$ if for any
$\varepsilon>0$ there exists an integer $N$ such that for all $n>N$ and
$x\in D$, $d_y(f_n(x),f(x))<\varepsilon$.)
\end{problem}
\begin{proof}
\end{proof}

\begin{problem}
Give an example of a surjective map of spaces that is not a quotient map.
\end{problem}
\begin{proof}
\end{proof}

\begin{problem}
\end{problem}
\begin{proof}
\end{proof}

\begin{problem}
\end{problem}
\begin{proof}
\end{proof}

\begin{problem}
\end{problem}
\begin{proof}
\end{proof}

\begin{problem}
\end{problem}
\begin{proof}
\end{proof}

\begin{problem}
\end{problem}
\begin{proof}
\end{proof}

\begin{problem}
\end{problem}
\begin{proof}
\end{proof}

\begin{problem}
\end{problem}
\begin{proof}
\end{proof}

\begin{problem}
\end{problem}
\begin{proof}
\end{proof}

\begin{problem}
\end{problem}
\begin{proof}
\end{proof}

\begin{problem}
\end{problem}
\begin{proof}
\end{proof}


%%% Local Variables:
%%% mode: latex
%%% TeX-master: "../MA571-Quals"
%%% End:

% \chapter{MA571 (Final 2015)}
\begin{problem}
\end{problem}
\begin{proof}
\end{proof}

\begin{problem}
\end{problem}
\begin{proof}
\end{proof}

\begin{problem}
\end{problem}
\begin{proof}
\end{proof}

\begin{problem}
\end{problem}
\begin{proof}
\end{proof}

\begin{problem}
\end{problem}
\begin{proof}
\end{proof}

\begin{problem}
\end{problem}
\begin{proof}
\end{proof}

\begin{problem}
\end{problem}
\begin{proof}
\end{proof}

\begin{problem}
\end{problem}
\begin{proof}
\end{proof}

\begin{problem}
\end{problem}
\begin{proof}
\end{proof}

\begin{problem}
\end{problem}
\begin{proof}
\end{proof}

\begin{problem}
\end{problem}
\begin{proof}
\end{proof}

\begin{problem}
\end{problem}
\begin{proof}
\end{proof}


%%% Local Variables:
%%% mode: latex
%%% TeX-master: "../MA571-Quals"
%%% End:

\backmatter
\bibliographystyle{plain}
\bibliography{top-bib}
\printindex
\end{document}

%%% Local Variables:
%%% mode: latex
%%% TeX-master: t
%%% End:
