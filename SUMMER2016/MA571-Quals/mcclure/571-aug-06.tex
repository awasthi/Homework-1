\section{McClure}
\subsection{McClure: Summer 2006}
\setcounter{exercise}{0}
\setcounter{equation}{0}

\begin{problem}
  Let $X$ be a connected space and let $f\colon X\to Y$ be a function which
  is continuous and onto. Prove that $Y$ is connected. (This is a theorem
  in Munkres---prove it from the definitions).
\end{problem}
\begin{solution}
  Let $C,D$ be a separation of $Y$. We will show that either
  $C=\emptyset,D=Y$ or $C=Y,D=\emptyset$. Since $f$ is a continuous map,
  $f^{-1}(C)$ and $f^{-1}(D)$ are open subsets of $X$. Moreover,
  \[
    X=f^{-1}(Y)=f^{-1}(C\cup D)=f^{-1}(C)\cup f^{-1}(D).
  \]
\end{solution}

\begin{problem}
  Let $X$ be the Cartesian product $\prod_{i=1}^\infty\bbR$ with the
  \emph{box topology} (recall that a basis for this topology consists of
  all sets of the form $\prod_{i=1}^\infty U_i$, where $U_i$ is open in
  $\bbR$). Let $f\colon\bbR\to X$ be the function which takes $t$ to
  $(t,t,\ldots)$. Prove that $f$ is continuous.
\end{problem}
\begin{solution}
\end{solution}

\begin{problem}
  Let $Y$ be a topological space. Let $X$ be a set and let $f\colon X\to Y$
  be a function. Give $X$ the topology in which the open sets are the empty
  set and the sets $f^{-1}(V)$ with $V$ open in $Y$ (you do not have to
  verify that this is a topology). Let $a\in X$ and let $B$ be a closed set
  in $X$ not containing $a$. Prove that $f(a)$ is not in the closure of
  $f(B)$.
\end{problem}
\begin{solution}
\end{solution}

For the next two problems, let $P$ be the Cartesian product
$\prod_{i=1}^\infty\{0,1\}$ with the usual Cartesian product
topology. (Note that $\{0,1\}$ is a set with two points, it is not an
interval.)
\begin{problem}
  Prove that every function from the Cantor set $C$ to $P$ which is
  one-to-one, onto and continuous is a homeomorphism.
\end{problem}
\begin{solution}
\end{solution}

\begin{problem}
  \hfil
  \begin{enumerate}[label=(\alph*),noitemsep]
  \item Give a clear and specific description of a function from the Cantor
    set to $P$ which is one-to-one and onto. You do not have to prove that
    your function is one-to-one and onto.
  \item Prove that the function you described in part (a) is
    continuous. (If it isn't continuous, go back and choose a different
    function that is).
  \end{enumerate}
\end{problem}
\begin{solution}
\end{solution}

\begin{problem}
  Let $X$ and $Y$ be topological spaces, let $x_0\in X$, $y_0\in Y$ and let
  $f\colon X\to Y$ be a continuous function which takes $x_0$ to $y_0$.

  Is the following statement true? If $f$ is one-to-one then
  $f_*\colon\pi(X,x_0)\to \pi_1(Y,y_0)$ is one-to-one. Prove or give a
  counterexample (and if you give a counterexample, justify it). You may
  use anything in Munkre's book.
\end{problem}
\begin{solution}
\end{solution}

\begin{problem}
  Let $S^2$ be the $2$-sphere, that is, the following subspace of $\bbR^3$:
  the set
  \[
    \left\{\,(x,y,z)\in\bbR^3:x^2+y^2+z^2=1\,\right\}.
  \]
  Let $x_0$ be the point $(0,0,1)$ of $S^2$.


  Use the Seifert--van Kampen theorem to prove that $\pi_1(S^2,x_0)$ is the
  trivial group. You may use either of the two versions of the Seifert--van
  Kampen theorem given in Munkre's book. You will not get credit for any
  other method.
\end{problem}

\begin{solution}
\end{solution}

%%% Local Variables:
%%% mode: latex
%%% TeX-master: "../MA571-Quals"
%%% End:
