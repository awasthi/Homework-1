\subsection{McClure: Winter 2008}
\setcounter{exercise}{0}

\begin{problem}
  Let \(X\) be a topological space with a countable basis. Prove that every
  open cover of \(X\) has a countable subcover.
\end{problem}
\begin{solution}
  Suppose \(\calB\) is a countable basis for \(X\). Let
  \(\{B_n:n\in\bbN\}\) be an enumeration of \(\calB\) and let
  \(\calU=\{U_\alpha:\alpha\in A\}\) an open cover of \(X\). To every
  \(B_n\) in \(\calB\) associate an element \(U_n\) such that
  \(B_n\subseteq U_n\) if such an \(n\) exists; otherwise, associate to
  \(B_n\) the empty set \(\emptyset\). We claim that the subcollection
  \(\calU'=\{U_n:n\in\bbN\}\) is a countable subcover of \(\calU\).

  The countability of \(\calU'\) follows from the countability of
  \(\calB\). Let \(f\colon\calB\to\calU\) be the function which assigns
  \(B_n\) to \(U_n\). Then, by definition, \(\calU'=f(\calB)\) and,
  restricting the codomain of \(f\) to \(\calU'\), we have
  \(f'\colon\calB\to\calU'\) is surjective. Since \(\calB\) is countable,
  there exists a surjection \(g\colon\bbN\to\calB\). Then, the map
  \(f'\circ g\colon\bbN\to\calU'\) is surjective. It follows that
  \(\calU'\) is at most countable.

  It remains to show that \(\calU'\) covers \(X\). Suppose \(x\notin U_n\)
  for any \(n\in\bbN\). Then, since \(\calU\) is a cover of \(X\), there
  exists an open set \(U_\alpha\in\calU\) such that \(x\in
  U_\alpha\). Moreover, since \(\calB\) is a basis for \(X\), and
  \(U_\alpha\) is open, there exists a basis element \(B_m\in\calB\)
  containing \(x\) contained in \(U_\alpha\). It follows that there is some
  nonempty open set \(U_m\) in the cover \(\calU\) containing \(B_m\) which
  implies that \(f(B_m)\neq\emptyset\). This contradicts the assumption
  that no \(x\notin U_n\) for any \(n\in\bbN\). Thus, \(\calU'\) covers
  \(X\).

  It follows that every cover \(\calU\) of \(X\) admits a finite subcover
  \(\calU'\) of \(\calU\).
\end{solution}

\begin{problem}
  Let \(X\) be a compact space and suppose there is a finite family of
  continuous functions \(f_i\colon X\to\bbR\), \(i=1,\dotsc,n\), with the
  following property: given \(x\neq y\) in \(X\) there is an \(i\) such
  that \(f_i(x)\neq f_i(y)\). Prove that \(X\) is homeomorphic to a
  subspace of \(\bbR^n\).
\end{problem}
\begin{solution}
  Consider the map \(f\colon X\to\bbR^n\) given by
  \[
    f(x)=\bigl(f_1(x),\dotsc,f_n(x)\bigr).
  \]
  By a theorem from Munkres regarding continuous bijective maps between
  compact and Hausdorff spaces, it suffices to show that the restriction
  \(f'\colon X\to f(X)\) of the codomain of \(f\) to its image \(f(X)\) is
  a bijection.

  First, from Munkres, we know that the restriction of the codomain of a
  continuous function to a subspace is again continuous. Therefore,
  \(f'\colon X\to f(X)\) is continuous.

  Moreover, since \(\bbR^n\) is Hausdorff and \(f(X)\subseteq\bbR^n\) is a
  subspace, \(f(X)\) is Hausdorff.

  Lastly, we show that \(f'\colon X\to f(X)\) is bijective. It is clear
  that \(f'\) is surjective since we have restricted the codomain of \(f\)
  to its image. To see that \(f'\) is injective let \(x,y\in X\) and
  suppose that \(f'(x)=f'(y)\). Then
  \[
    \bigl(f_1(x),\dotsc,f(x)\bigr)= \bigl(f_1(y),\dotsc,f(y)\bigr).
  \]
  This implies that \(x=y\) for otherwise, there is a \(x\neq y\) in \(X\)
  such that \(f_i(x)=f_i(y)\) for all \(1\leq i\leq n\); which contradicts
  our assumption.

  It follows that \(f'\colon X\to f(X)\) is a continuous bijection from a
  compact to a Hausdorff space and thus, is a homeomorphism.
\end{solution}

\begin{problem}
  Let \(X\) be any topological space and let \(Y\) be a Hausdorff
  space. Let \(f,g\colon X\to Y\) be continuous functions. Prove that the
  set \(\left\{\,x\in X:f(x)=g(x)\,\right\}\) is closed.
\end{problem}
\begin{solution}
  Recall that a topological space $Y$ is Hausdorff if and only if the
  diagonal $\Delta=\left\{\,(y,y):y\in Y\,\right\}$ is closed in
  $Y\times Y$ with the product topology. Define a map \(F\colon X\to
  Y\times Y\) by
  \[
    F(x)=\bigl(f(x),g(x)\bigr).
  \]
  Then, by the u.m.p\@ of the product topology, \(F\) is continuous because
  \[
    (\pi_1\circ F)(x)=f(x),\quad(\pi_2\circ F)(x)=g(x)
  \]
  are continuous; where \(\pi_1,\pi_2\colon Y\times Y\to Y\) are the
  standard projections \((y_1,y_2)\mapsto y_1\) and \((y_1,y_2)\mapsto
  y_2\), respectively. We claim that
  \[
    \left\{\,x\in X:f(x)=g(x)\,\right\}=F^{-1}(\Delta).
  \]

  Let \(x\in\left\{\,x\in X:f(x)=g(x)\,\right\}\) then \(f(x)=g(x)\) so
  \(F(x)\in\Delta\). Thus, \(x\in F^{-1}(\Delta)\). On the other hand, if
  \(x\in F^{-1}(\Delta)\) then \(F(x)=(f(x),g(x))=(y,y)\) for some \(y\in
  Y\). Thus, \(f(x)=g(x)\) so \(x\in\left\{\,x\in X:f(x)=g(x)\,\right\}\).

  It follows that since
  \[
    \left\{\,x\in X:f(x)=g(x)\,\right\}=F^{-1}(\Delta),
  \]
  \(F\) is continuous and \(\Delta\) is closed in \(Y\times Y\), the set
  \[
    \left\{\,x\in X:f(x)=g(x)\,\right\}
  \]
  is closed in \(X\).
\end{solution}

\begin{problem}
  Let \(X\) be the two-point set \(\{0,1\}\) with the discrete
  topology. Let \(Y\) be a countable product of copies of \(X\); thus an
  element of \(Y\) is a sequence of \(0\)s and \(1\)s.

  For each \(n\geq 1\), let \(y_n\in Y\) be the element
  \((1,\dotsc,1,0,\dotsc)\), with \(n\) \(1\)s at the beginning and all
  other entries \(0\). Let \(y\in Y\) be the element with all \(1\)s. Prove
  that the set \({\{y_n\}}_{n\in\bbN}\cup\{y\}\) is closed. Give a clear
  explanation. Do not use a metric.
\end{problem}
\begin{solution}
  Let \(A={\{y_n\}}_{n\in\bbN}\cup\{y\}\). Assuming \(Y\) is given the
  discrete topology, we must show that for every \(x\in A\), for every
  neighborhood \(U\) of \(x\), \(U\cap A\neq\emptyset\). By one of Prof.\@
  McClure's lemmas, it suffices to show that this holds for basic open
  sets.

  We break the proof into two cases (1) \(x\neq y\) and (2) \(x=y\).

  For (1), let \(U=\prod_{n\in\bbN} U_n\) be a basic neighborhood of
  \(x\). Since \(x\neq y\), the first \(N\) entries of \(x\) consists of
  all \(1\). Therefore, the first \(N\) \(U_n\) must contain \(\{1\}\) and
  the last \(U_n\), \(n\geq N\), must contain \(\{0\}\). Since a \(U\) is a
  basic open, \(U_n=X\) for infinitely many \(U_n\). Let $N'$ be the
  smallest integer such that \(U_n=X\) for all \(n\geq N'\). Then \(U\)
  contains the element \((1,\dotsc,1,0,\dotsc)\) with \(N'\) \(1\) at the
  beginning. This is an element of \(A\). Thus, \(U\cap A\neq\emptyset\).

  For (2), let \(U=\prod_{n\in\bbN} U_n\) be a basic open set containing
  \(x=y\). Let \(N\) be the smallest integer such that \(U_n=X\) for all
  \(n\geq N\). Then \(U\) contains the element \((1,\dotsc,1,0,\dotsc)\)
  with \(N\) \(1\) in the beginning. This is an element of \(A\), thus
  \(U\cap A\neq\emptyset\).

  Since the latter arguments holds for any basic neighborhood \(U\) for any
  \(x\in A\), it follows that \(A\) is closed.
\end{solution}

\begin{problem}
  Let \(X\) be a connected space. Let \(\calU\) be an open covering of
  \(X\) and let \(U\) be a nonempty set in \(\calU\). Say that a set \(V\)
  in \(\calU\) is \emph{reachable from \(U\)} if there is a sequence
  \[
    U=U_1,U_2,\dotsc,U_n=V
  \]
  of sets in \(\calU\) such that \(U_i\cap U_{i+1}\neq\emptyset\) for each
  \(i\) from \(1\) to \(n-1\). Prove that every nonempty \(V\) in \(\calU\)
  is reachable from \(U\).
\end{problem}
\begin{solution}
  Fix \(U\) in the cover \(\calU\). Seeking a contradiction, suppose \(U'\)
  is not reachable from \(U\). Let
  \begin{align*}
    \calA&=\left\{\,V\in\calU:\text{\(V\) is reachable from
           \(U\)}\,\right\},\\
    \calB&=\left\{\,V\in\calU:\text{\(V\) is not reachable from
    \(U\)}\,\right\}.
  \end{align*}
  Then, \(\calA\) is nonempty since it contains \(U\) and \(\calB\) is
  nonempty since it contains \(U'\).

  We claim that \(C=\bigcup_{V\in\calA}V\) and \(D=\bigcup_{V\in\calB}V\)
  form a separation of \(X\). First, \(C\) and \(D\) are open since they
  are the (arbitrary) union of open sets. Next, we must show that
  \(C\cup D=X\). Let \(x\in X\). Then \(x\in V\) for some \(V\in\calU\)
  since \(\calU\) covers \(X\). Thus, either \(V\) is reachable from \(U\)
  or \(V\) is not reachable from \(U\). In the former case, \(x\in C\) and
  in the latter case \(x\in D\). Thus, \(C\cup D\supseteq X\) and
  \(C\cup D=X\); it is clear that both \(C,D\subseteq X\) so the union
  \(C\cup D\subseteq X\).

  Thus, \(C,D\) forms a separation of \(X\). This contradicts the
  assumption that \(X\) is connected. Therefore, it must be the case that
  every set \(V\) is reachable from \(U\) for any \(U\in\calU\).
\end{solution}

\begin{problem}
  Let \(p\colon E\to B\) be a covering map. Suppose that points are closed
  in \(B\). Let \(A\subseteq E\) be compact. Prove that for every
  \(b\in B\) the set \(A\cap p^{-1}\{b\}\) is finite.
\end{problem}
\begin{solution}
  Fix a point \(b\in B\). Then the singleton set \(\{b\}\) is closed in
  \(B\) and since \(p\colon E\to B\) is a covering map, it is continuous so
  \(p^{-1}\{b\}\) is closed in \(A\). By some theorem from Munkres dealing
  with the subspace topology, \(A\cap p^{-1}\{b\}\) is a closed subset of
  \(A\) in the subspace topology. Thus, \(A\cap p^{-1}\{b\}\) is compact in
  \(A\). Now, since \(p\) is a covering map, there exists an evenly covered
  neighborhood \(V\) of \(b\), i.e., a neighborhood such that \(p^{-1}(V)\)
  is the disjoint union of open sets \(\bigsqcup_{\alpha\in A} U_\alpha\)
  such that the restriction \(p_\alpha=p\restrict{U_\alpha}\) yields a
  homeomorphism \(U_\alpha\approx_{p_\alpha} V\). Since
  \(p^{-1}\{b\}\subset p^{-1}(V)\), the \(\bigsqcup_{\alpha\in A}U_\alpha\)
  form an open cover of \(A\cap p^{-1}\{b\}\). Since \(A\cap p^{-1}\{b\}\)
  is compact, only finitely many of the \(U_\alpha\), say
  \(\{U_1,\dotsc,U_N\}\), cover \(A\cap p^{-1}\{b\}\). Thus,
  \(A\cap p^{-1}\{b\}\subseteq\bigcup_{n=1}^N U_n\). Since each
  \(U_n\approx V\), each \(U_n\cap p^{-1}\{b\}\) consists of a single
  point. Thus, the number of points in \(A\cap p^{-1}\{b\}\) is less than
  or equal to \(N\).
\end{solution}

\begin{problem}
  Let \(p\colon E\to B\) be a covering map. Let \(Y\) be a path-connected
  space and let \(y_0\) be a point in \(Y\). Let \(h,k\colon Y\to E\) be
  continuous functions with \(h(y_0)=k(y_0)\) and \(p\circ h=p\circ
  k\). Prove that \(h\) and \(k\) are the same function.
\end{problem}
\begin{solution}
  % Since the image of a path-connected space under a continuous map is again
  % path-connected and \(h(y_0)=k(y_0)\), \(h(Y)\) and \(k(Y)\) both lie in
  % the same path-connected component. Now, suppose that \(h\neq k\). Then
  % there exists \(y_1\in Y\) such that \(h(y_1)\neq k(y_1)\). Let
  % \(f_1,f_2\colon I\to E\) be a path from \(h(y_0)\) to \(h(y_1)\) and
  % \(k(y_0)\) to \(k(y_1)\), respectively. Then, since
  % \(p\circ h=p\circ k\), \(p\circ f_1\) and \(p\circ f_2\) are paths in
  % \(B\) that both begin at \(p(h(y_0))\) and end at \(p(h(y_1))\). By the
  % path-lifting property, \(p\circ f_1\) and \(p\circ f_2\) lift uniquely to
  % paths \(\tilde f_1\) and \(\tilde f_2\) in \(E\) with
  % \(\tilde f_1(1)=\tilde f_2(1)=h(y_0)\). It follows that
  Since the image of a path-connected space under a continuous map is again
  path-connected and \(h(y_0)=k(y_0)\), \(h(Y)\) and \(k(Y)\) both lie in
  the same path component of \(E\). Now, suppose that \(h\neq k\). Then
  there exist \(y_1\in Y\) such that \(h(y_1)\neq k(y_1)\).
\end{solution}

%%% Local Variables:
%%% mode: latex
%%% TeX-master: "../MA571-Quals"
%%% End:
