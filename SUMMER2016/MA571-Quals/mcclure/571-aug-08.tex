\subsection{McClure: Summer 2008}
\setcounter{exercise}{0}
\setcounter{equation}{0}

\begin{problem}
  Let \(X\) and \(Y\) be topological spaces and let \(f\colon X\to Y\) be a
  continuous function. Prove that
  \[
    f(\bar A)\subseteq\overline{f(A)}
  \]
  for all subsets \(A\) of \(X\).
\end{problem}
\begin{solution}
  This is another characterization of continuity. Let \(x\in\bar A\). Then
  for every neighborhood \(U\) of \(x\), \(U\cap A\neq\emptyset\). In
  particular, if \(U\) is an neighborhood of \(f(x)\), \(f^{-1}(V)\)
  is a neighborhood of \(x\) so \(f^{-1}(V)\cap A\neq\emptyset\). Thus,
  \(V\cap f(A)\neq\emptyset\) so \(f(x)\in\overline{f(A)}\). It follows
  that \(f(\bar A)\subseteq\overline{f(A)}\).
\end{solution}

\begin{problem}
  Suppose that \(X\) is connected and every point of \(X\) has a
  path-connected open neighborhood. Prove that \(X\) is path-connected.
\end{problem}
\begin{solution}
  Let \(x,y\in X\) and suppose there does not exists a path from \(x\) to
  \(y\). Hence, the sets
  \begin{align*}
    C&=\bigl\{\,y\in X:\text{there exists a path \(p\colon I\to X\)
       from \(x\) to \(y\)}\,\bigr\}\\
    D&=\bigl\{\,y\in X:\text{there does not exists a path \(p\colon I\to X\)
       from \(x\) to \(y\)}\,\bigr\}
  \end{align*}
  are nonempty. We claim that both \(C\) and \(D\) form a separation of
  \(X\). First we must show that these sets are open and that their union
  \(X\). The latter is immediate as every point in \(X\) is either
  connected to \(x\) by a path or it is not, i.e., \(C=X\setminus D\), and
  at least one of these sets is nonempty, namely, \(C\). To see that \(C\)
  is open let \(y\in C\). Then there exists a path-connected open
  neighborhood \(U\) of \(y\). If \(U\cap D\neq\emptyset\) and \(z\) is a
  point in that intersection, there exists a path \(q\colon I\to X\) from
  \(y\) to \(z\). But there exists a path \(p\colon I\to X\) from \(x\) to
  \(y\) so \(q*p\colon I\to Xs\) is a path from \(x\) to \(z\), which is a
  contradiction. Thus, \(U\subset C\) and \(C\) must be open. Consequently,
  a similar argument tells us that \(C\) is closed since if \(y\in\bar C\),
  \(U\cap C\neq\emptyset\) for every neighborhood of \(y\). In particular,
  for that special path-connected neighborhood \(U\) of \(y\), there is
  some \(z\in C\cap U\). If \(y\in D\), then there exist a path
  \(p\colon I\to X\) from \(x\) to \(y\), namely, the composite of the path
  from \(x\) to \(z\) and from \(z\) to \(y\). This yields a contradiction
  so \(y\in C\). Thus, \(C\) is closed so \(X\setminus D\) is open. It
  follows that \(C\) and \(D\) form a separation of \(X\). But \(X\) is
  connected
\end{solution}

\begin{problem}
  Let \(X\) be a topological space and let \(f,g\colon X\to[0,1]\) be a
  continuous function. Suppose that \(X\) is connected and \(f\) is
  onto. Prove that there must be a point \(x\in X\) with \(f(x)=g(x)\).
\end{problem}
\begin{solution}
  Seeking a contradiction, suppose \(f(x)\neq g(x)\) for any \(x\in
  X\). Consider the map \(h\colon X\to I\times I\) given by
  \(h(x)=\bigl(f(x),g(x)\bigr)\). This map is continuous by the u.m.p.\@ of
  the product topology since each projection
  \(\pi_1\circ h=f,\pi_2\circ h=g\) is continuous. Since \(f(x)\neq g(x)\),
  \(h(x)\subseteq I\times I\setminus\Delta\) for otherwise, there is a
  point \(x\in X\) such that \(h(x)=\bigl(f(x),g(x)\bigr)\in\Delta\) which
  implies \(f(x)=g(x)\). Now, we note that the set
  \((I\times I)\setminus\Delta\) consists of two connected components,
  \begin{align*}
    C&=\bigl\{\,x\in X:f(x)>g(x)\,\bigr\}\\
    D&=\bigl\{\,x\in X:f(x)<g(x)\,\bigr\}.
  \end{align*}
  Thus, since \(X\) is connected, \(h(X)\subseteq C\) or \(h(X)\subseteq
  D\). In the former case, since \(f\) is onto, \(f(x)=0\) for some \(x\in
  X\) so \(g(x)<0\), which contradicts the assumption that the image of
  \(X\) under \(g\) lies in \(I\). Similarly, if \(h(X)\subseteq D\), then
  since \(f\) is onto, \(f(x)=1\) for some \(x\in X\), which implies
  \(g(x)>1\). Again, this is nonsense. Therefore, the image of \(X\) under
  \(h\) must contain a point in the diagonal \(\Delta\), i.e.,
  \(f(x)=g(x)\) for some \(x\in X\).
\end{solution}

\begin{problem}
  Let \(X\) be the two-point set \(\{0,1\}\) with the discrete
  topology. Let \(Y\) be a countable product of copies of \(X\); thus an
  element of \(Y\) is a sequence of \(0\)s and \(1\)s.  Let \(A\) be the
  subset of \(Y\) consisting of sequences with only a finite number of
  \(1\)s. Is A closed?  Prove or disprove.
\end{problem}
\begin{solution}
  We shall disprove this by showing that the sequence consisting of
  infinitely many \(1\)s is in the closure of this set. It suffices to show
  that for every basic neighborhood \(U\) of \(\mathbf{1}\),
  \(U\cap A\neq\emptyset\). Let \(U\) be a basic open neighborhood of
  \(\mathbf{1}\). Then \(U=\prod_{i\in\bbN} U_i\) where only finitely many
  of then \(U_i\neq \{0,1\}\). Let \(U_N\) be the last such \(U_i\). Then,
  for \(1\leq i\leq N\), \(U_i\supseteq\{1\}\) otherwise \(\mathbf{1}\) is
  not in the set \(U\). Let \(\bfx\) be the sequence consisting of \(N\)
  \(1\)s and \(0\)s for the remainder of the sequence. Then \(\bfx\in A\)
  since it contains finitely many \(1\)s and \(\bfx\in U\) since the first
  \(N\) terms of the sequence \(\bfx\) are in one of the \(U_i\) and the
  last terms in the sequence are in \(\{0,1\}\). Since this can be done for
  any basic neighborhood of \(\mathbf{1}\), it follows that \(\mathbf{1}\)
  is contained in the closure of \(A\). In particular, \(A\) is not
  closed.
\end{solution}

\begin{problem}
  Prove the Tube Lemma: given topological spaces \(X\) and \(Y\) with \(Y\)
  compact, a point \(x_0\in X\), and an open set \(N\) of \(X\times Y\)
  containing \(\{x_0\}\times Y\), prove that there is an open set \(W\) of
  \(X\) containing \(x_0\) with \(W\times Y\subseteq N\).
\end{problem}
\begin{solution}
  Let \(f\colon Y\to X\times Y\) be the map \(f(y)=(x_0,y)\) for a fixed
  \(x_0\in X\). This map is continuous since the projections
  \(\pi_1\circ f=x_0\) and \(\pi_2\circ f=\id_Y\) are continuous. Thus,
  since the image of a compact space under a continuous map is compact, the
  set
  \[
    f(Y)=\{x_0\}\times Y
  \]
  is compact. For each point \((x_0,y)\in \{x_0\}\times Y\), take an
  neighborhood \(V_y\). Then the collection \(\{V_y:y\in Y\}\) forms an
  open cover of \(\{x_0\}\times Y\) since the sets \(V_y\) are open and for
  any \((x_0,y)\in Y\), \((x_0,y)\in V_y\) so
  \[
    \bigcup_{y\in Y}V_y\supseteq \{x_0\}\times Y.
  \]
  Now, since \(\{x_0\}\times Y\) is compact, there exists a finite
  subcover, say \(\{V_1,\dotsc,V_N\}\) of \(\{V_y:y\in Y\}\) that covers
  \(\{x_0\}\times Y\).
\end{solution}

\begin{problem}
  Let \(X\) and \(Y\) be topological spaces and let \(f\colon X\to Y\) be a
  continuous function. Let \(x_0\in X\) and let \(y_0=f(x_0)\). Find an
  example in which \(f\) is onto but
  \(f_*\colon\pi_1(X,x_0)\to\pi_1(Y,y_0)\) is not onto. Prove that your
  example really has this property. You may use any fact from Munkres. Let
  \(X\) and \(Y\) be topological spaces and let \(f\colon X\to Y\) be a
  continuous function. Let \(x_0\in X\) and let \(y_0=f(x_0)\). Find an
  example in which f is onto but \(f_*\colon\pi_1(X,x_0)\to\pi_1(Y,y_0)\)
  is not onto. Prove that your example really has this property. You may
  use any fact from Munkres.
\end{problem}
\begin{solution}
\end{solution}

\begin{problem}
  Prove that every continuous map from \(S^2\) to \(S^1\) is homotopic to a
  constant map (hint: use covering spaces). You may use any fact from
  Munkres.
\end{problem}
\begin{solution}
\end{solution}

%%% Local Variables:
%%% mode: latex
%%% TeX-master: "../MA571-Quals"
%%% End:
