\subsection{McClure: Winter 2011}
\setcounter{exercise}{0}

\begin{problem}
  Let \(A\) be a subset of a topological space \(X\) and let \(B\) be a
  subset of \(A\). Prove that
  \(\bar A\setminus\bar B\subseteq\overline{A\setminus B}\).
\end{problem}
\begin{solution}
  Let \(x\in\bar A\setminus\bar B\), then \(x\) is in the closure of \(A\),
  but not in the closure of \(B\), i.e., there exists a neighborhood \(U\)
  of \(x\) such that \(U\cap B=\emptyset\) and \(U\cap A\neq\emptyset\). In
  particular, the latter shows that \(A\setminus B\neq\emptyset\) and so
  \(U\cap A\setminus B\neq\emptyset\). This is true for every neighborhood
  of \(x\in\bar A\setminus\bar B\). Thus, \(x\in\overline{A\setminus B}\).
\end{solution}

\begin{problem}
  Let \(G\) be a topological group (that is, a group with a topology for
  which the group operations are continuous) and let \(H\) be a subgroup of
  \(G\). Suppose that \(G\) is connected, that \(H\) is a normal subgroup
  of \(G\), and that the subspace topology on \(H\) is discrete. Prove that
  \(gh=hg\) for every \(g\in G\), \(h\in G\).
\end{problem}
\begin{solution}
  Fix \(h\in H\) and consider the map \(f_h \colon H\to H\) given by
  \(f_h(g)=ghg^{-1}\). Since \(H\) is normal in \(G\), \(ghg^{-1}\in H\)
  and since multiplication is continuous in \(G\) and \(f_h\) is the
  composition \(g\mapsto gh\mapsto ghg^{-1}\), \(f_h\) is continuous. Now,
  since \(H\) has the discrete topology, it is totally disconnected, i.e.,
  singleton sets \(\{h\}\) are the only connected components of \(H\) and
  since \(G\) is connected, \(f_h(G)=\{h'\}\) for some \(h'\in H\). Since
  \(ehe^{-1}=h\), \(h'=h\). Since we can do this for any \(h\in H\),  it
  follows that \(gh=hg\) for all \(g\in G\), \(h\in H\).
\end{solution}

\begin{problem}
  Let \(X\) be the space with two points and the discrete topology. Let
  \(Y=\prod_{n=1}^\infty X\), with the product topology. What are the
  connected components of \(Y\)? Prove that your answer is correct.
\end{problem}
\begin{solution}
  Let \(X=\{a,b\}\). The connected components of \(Y\) are precisely the
  singleton sets, i.e., the sets consisting of a single sequence \(a\)s and
  \(b\)s. First, note that if \(C\) is a component of \(Y\), then
  \(\pi_n(C)\) is either contained in \(\{a\}\) or \(\{b\}\) since
  \(\{a\}\) and \(\{b\}\) are the components of \(X\). Suppose, without
  loss of generality, that \(\pi_n(C)=\{a\}\). Then \(C\) must consist of
  those sequences of \(a\) and \(b\) which have \(a\) as their \(n\)th
  term. Proceeding in this fashion, we see that \(C\) must be a sequence of
  \(a\), \(b\) and not a collection of these.
\end{solution}

\begin{problem}
  Let \(X\) and \(Y\) be topological spaces. Let \(x_0\in X\) and let \(C\)
  be a compact subset of \(Y\). Let \(N\) be an open set in \(X\times Y\)
  containing \(\left\{x_0\right\}\times C\). Prove that there is an open
  set \(U\) containing \(x_0\) and an open set \(V\) containing \(C\) such
  that \(U\times V\subseteq N\).
\end{problem}
\begin{solution}
  Same as the proof of the tube lemma.
\end{solution}

\begin{problem}
  Let \(X\) and \(Y\) be homotopy-equivalent topological spaces. Suppose
  that \(X\) is connected. Prove that \(Y\) is connected.
\end{problem}
\begin{solution}
  Suppose that \(X\simeq Y\). Then there exists a continuous maps
  \(f\colon X\to Y\) and \(g\colon Y\to X\) such that the composition
  \(g\circ f\colon X\to X\) is homotopic to \(\id_X\) and
  \(f\circ g\colon Y\to Y\) is homotopic to \(\id_Y\). Seeking a
  contradiction, suppose \(C\), \(D\) is a separation of \(Y\). Then
  \(f^{-1}(C)\) and \(f^{-1}(D)\) are open and disjoint in \(X\); that
  these sets are open follows from continuity, that they are disjoint:
  suppose not, then \(x\in f^{-1}(C)\cap f^{-1}(D)\) so \(f(x)\in C\cap
  D\), but \(C,D\) is a separation of \(Y\) (in particular, \(C\cap
  D=\emptyset\)). It follows that either \(f^{-1}(C)\) or \(f^{-1}(D)\) is
  empty. Assume the latter. Then \(f\circ g(Y)\subseteq V\). But since
  \(f\circ g\simeq \id_Y\) there exists a homotopy \(H\colon I\times Y\to
  Y\) such that \(H(y,0)=f\circ g(y)\) and \(H(y,1)=y\). Fix \(y\in Y\),
  then the map \(p_y=H(y,\cdot)\) is a path from \(y\) to \(f\circ
  g(y)\). It follows that \(y\in V\). Thus, \(V=Y\) and \(U=\emptyset\) so
  \(Y\) is connected.
\end{solution}

\begin{problem}
  Let \(p\colon E\to B\) be a covering map. Let \(e_0\in E\) and
  \(b_0\in B\) with \(p(e_0)=b_0\). Let \(Y\) be simply connected (in
  particular, \(Y\) is path-connected). Let \(y_0\in Y\). Let
  \(f\colon Y\to B\) be continuous, with \(f(y_0)=b_0\). Prove that the
  following function \(g\colon Y\to E\) is well-defined: Given \(y\in Y\),
  choose a path \(\gamma\) from \(y_0\) to \(y\); let \(\beta\) be the lift
  of \(f\circ\gamma\) to \(E\) starting at \(e_0\); now define
  \(g(y)=\beta(1)\). You may use the fact (without proving it) that the
  lift of a path homotopy is again a path homotopy.
\end{problem}
\begin{solution}
  Let \(\gamma_1\) and \(\gamma_2\) be two paths beginning at \(y_0\) and
  ending at \(y\). Now, lift \(f\circ\gamma_1\) to \(\beta_1\) beginning at
  \(e_0\) and ending at \(e_1=\beta_1(1)\) and lift \(f\circ\bar \gamma_2\)
  to \(\beta_2\) beginning at \(e_1\) and ending at
  \(e_2=\beta_2(1)\). Then \(\beta_1*\beta_2\) is a lifting of the loop
  \(f\circ(\gamma_1*\bar\gamma_2)\). But by hypothesis
  \[
    \pi_1\bigl(\pi_1(Y,y_0)\bigr)\subseteq p_*\bigl(\pi_1(E,e_0)\bigr)
  \]
  so \(\bigl[f\circ(\gamma_1*\bar \gamma_2)\bigr]\) is in
  \(p_*\bigl(\pi_1(E,e_0)\bigr)\). By Theorem 54.6, \(\beta_1*\beta_2\) is
  a loop in \(E\) so we must have \(e_2=e_0\). Thus,  lifting of
  \(f\circ\gamma_1\) and \(f\circ\gamma_2\) must begin and end at the same
  point so \(g\) is well defined.
\end{solution}

\begin{problem}
  Let \(S^2\) be the \(2\)-sphere, that is, the following subspace of
  \(\bbR^3\):
  \[
    \left\{\,(x,y,z)\in\bbR^3:x^2+y^2+z^2=1\,\right\}.
  \]
  Let \(x_0\) be the point \((0,0,1)\in S^2\).

  Use the Seifert--van Kampen theorem to prove that \(\pi_1(S^2,x_0)\) is
  the trivial group. You may use either of the two versions of the
  Seifert--van Kampen theorem given in Munkres's book. You will not get
  credit for any other method.
\end{problem}
\begin{solution}
\end{solution}

%%% Local Variables:
%%% mode: latex
%%% TeX-master: "../MA571-Quals"
%%% End:
