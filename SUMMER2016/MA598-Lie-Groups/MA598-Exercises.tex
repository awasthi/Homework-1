\def\documentauthor{Carlos Salinas}
\def\documenttitle{Exercises in Basic Mathematics}
% \def\hwnum{1}
\def\shorttitle{Basic Mathematics}
\def\coursename{MA544}
\def\documentsubject{algebra, linear algebra, differential geometry, lie groups}
\def\authoremail{salinac@purdue.edu}

\documentclass[article,10pt]{memoir}
\usepackage{geometry}
\usepackage[dvipsnames]{xcolor}
\usepackage[
    breaklinks,
    bookmarks=true,
    colorlinks=true,
    pageanchor=false,
    linkcolor=black,
    citecolor=black,
    filecolor=black,
    menucolor=black,
    runcolor=black,
    urlcolor=black,
    % linkcolor=violet!85!black,
    % citecolor=YellowOrange!85!black,
    % urlcolor=Aquamarine!85!black,
    hyperindex=false,
    hyperfootnotes=true,
    pdftitle={\shorttitle},
    pdfauthor={\documentauthor},
    pdfkeywords={\documentsubject},
    pdfsubject={\coursename}
    ]{hyperref}
\usepackage{natbib}

%% Math
\usepackage{amsmath}
\usepackage{amsfonts}
\usepackage{amssymb}
\usepackage{amsthm}
\usepackage{mathtools}
% \usepackage{eucal}
% \usepackage{mathrsfs}
% \usepackage[nointegrals]{wasysym}

%% Language
\usepackage{cmap}
\usepackage[LAE,LFE,T2A,T1]{fontenc}
\usepackage[utf8]{inputenc}
\usepackage[farsi,french,german,spanish,russian,english]{babel}
\babeltags{fr=french,
           de=german,
           en=english,
           es=spanish,
           pa=farsi,
           ru=russian
           }
\def\spanishoptions{mexico}

\selectlanguage{english}

\newcommand{\textfa}[1]{\beginR\textpa{#1}\endR}

\usepackage{CJKutf8}
\newcommand{\textkr}[1]{\begin{CJK}{UTF8}{mj}#1\end{CJK}}
\newcommand{\textjp}[1]{\begin{CJK}{UTF8}{min}#1\end{CJK}}
\newcommand{\textzh}[1]{\begin{CJK}{UTF8}{bsmi}#1\end{CJK}}

%% Misc
\usepackage{graphicx}
\graphicspath{{figures/}}

\usepackage{microtype}
\usepackage{lineno}
\usepackage{multicol}
\usepackage[inline]{enumitem}
\usepackage{listings}
\usepackage{mleftright}
\mleftright
\usepackage{carlos-variables}

% %% Unicode math and Polyglossia
% \usepackage{unicode-math}
% \usepackage{unicode-minionmath}

% % \setmainfont[Ligatures=TeX]{Minion Pro}
% \setmainfont[Ligatures=TeX]{Libertinus Serif}
% \setsansfont{Libertinus Sans}
% \setmonofont{Libertinus Mono}

% \setmathfont{MinionMath-Regular.otf}
% \setmathfont[range={\mathfrak}]{latinmodern-math.otf}
% \setmathfont[range={\mathcal}]{latinmodern-math.otf}
% \setmathfont[range={\mathscr}]{latinmodern-math.otf}
% \setmathfont[range={}]{MinionMath-Regular.otf}

% \usepackage{polyglossia}

% \newfontfamily\cyrillicfont[Script=Cyrillic]{Libertinus Serif}
% \newfontfamily\cyrillicfontsf[Script=Cyrillic]{Libertinus Sans}

% \newfontfamily\farsifont[Script=Arabic,
%                          Scale=MatchUppercase]{Amiri}

% % \newfontfamily\farsifont[Script=Arabic,
% %                          Scale=MatchUppercase]{Adobe Arabic}
% % \newfontfamily\farsifontsf[Script=Arabic,
% %                          Scale=MatchUppercase]{Myriad Arabic}

% \setmainlanguage[variant=american]{english}
% \setotherlanguage{farsi}
% \setotherlanguage{french}
% \setotherlanguage[spelling=new,latesthyphen,babelshorthands]{german}
% \setotherlanguage{spanish}
% \setotherlanguage[spelling=modern,babelshorthands]{russian}

% \makeatletter
% \@Latintrue
% \makeatother

% \usepackage{xeCJK}
% \usepackage[overlap]{ruby}
% \renewcommand\rubysep{-0.2ex}
% \xeCJKDeclareSubCJKBlock{Kana}{"3040 -> "309F, "30A0 -> "30FF, "31F0 -> "31FF, "1B000 -> "1B0FF}
% \xeCJKDeclareSubCJKBlock{Hangul}{"1100 -> "11FF, "3130 -> "318F, "A960 -> "A97F, "AC00 -> "D7AF, "D7B0 -> "D7FF}

% \setCJKmainfont{HanaMinA}
% \setCJKmainfont[Kana]{HanaMinA}
% \setCJKmainfont[Hangul]{NanumMyeongjo}
% \setCJKsansfont[Hangul]{NanumGothic}

% % \setCJKmainfont{Adobe Ming Std}
% % \setCJKmainfont{Adobe Ming Std}
% % \setCJKmainfont[Kana]{Kozuka Mincho Pr6N}
% % \setCJKmainfont[Hangul]{Adobe Myungjo Std}
% % \setCJKsansfont[Hangul]{Adobe Gothic Std}

%% Theorems and definitions
%% remove parentheses
% \makeatletter
% \def\thmhead@plain#1#2#3{%
%   \thmname{#1}\thmnumber{\@ifnotempty{#1}{ }\@upn{#2}}%
%   \thmnote{ {\the\thm@notefont#3}}}
% \let\thmhead\thmhead@plain
% \makeatother

\theoremstyle{plain}
\newtheorem{theorem}{Theorem}
\newtheorem{proposition}[theorem]{Proposition}
\newtheorem{corollary}[theorem]{Corollary}
\newtheorem{claim}[theorem]{Claim}
\newtheorem{lemma}[theorem]{Lemma}
\newtheorem{axiom}[theorem]{Axiom}

\newtheorem*{corollary*}{Corollary}
\newtheorem*{claim*}{Claim}
\newtheorem*{lemma*}{Lemma}
\newtheorem*{proposition*}{Proposition}
\newtheorem*{theorem*}{Theorem}

\theoremstyle{definition}
\newtheorem{definition}{Definition}
\newtheorem{example}{Examples}
\newtheorem{examples}[example]{Example}
\newtheorem{exercise}{Exercise}[section]
\newtheorem{problem}[exercise]{Problem}

\newtheorem*{example*}{Example}
\newtheorem*{exercise*}{Exercise}
\newtheorem*{problem*}{Problem}

\begin{document}
%% Footnote style
\renewcommand*{\thefootnote}{\fnsymbol{footnote}}

%% Counters
\counterwithout{exercise}{chapter}
\numberwithin{equation}{section}
\counterwithout{equation}{chapter}

%% Redefine the QED symbol
% \renewcommand\qedsymbol{\ensuremath{\null\hfill\QED}}

\chapterstyle{veelo}
\pagestyle{ruled}
\author{\href{mailto:\authoremail}{\documentauthor}}
\title{\documenttitle}
\date{\today}
\maketitle
\tableofcontents

%% Notes
\chapter{Differential Geometry Exercises}
\section{The Matrix Exponential; Some Matrix Lie Groups}
\subsection{The Exponential Map}
\subsection{The Lie Groups $\GL(n,\bbR)$, $\SL(n,\bbR)$, $\oalg(n)$,
  $\SO(n)$, the Lie Algebras $\glalg(n,\bbR)$, $\slalg(n,\bbR)$,
  $\oalg(n)$, $\soalg(n)$, and the Exponential Map}
The notation $\slalg(n,\bbR)$ and $\soalg(n)$ is rather strange and
deserves some explanation. The groups $\GL(n,\bbR)$, $\SL(n,\bbR)$,
$\oalg(n)$, and $\SO(n)$ are more than just groups. They are also
topological groups, which means that they are topological spaces (viewed as
subspaces of $\bbR^{n^2}$) and that the multiplication and the inverse
operations are smooth. In fact, they are smooth real manifolds. Such
objects are called \emph{Lie groups}. The real vector spaces $\slalg(n)$
and $\soalg(n)$ are what are called \emph{Lie algebras}. However, we have
not defined the algebra structure on $\slalg(n,\bbR)$ and $\soalg(n)$
yet. The algebra structure is called the \emph{Lie bracket}, which is
defined as
\[
[A,B]=AB-BA.
\]

Lie algebras are associated with Lie groups. What is going on is that the
Lie algebra of a Lie group is its tangent space at the identity, i.e., the
space of all tangent vectors at the identity. In some sense, the Lie
algebra achieves a linearization of the Lie group. The exponential map is a
map from the Lie algebra to the Lie group, for example,
\[
\exp\colon\soalg(n)\longrightarrow\SO(n)
\]
and
\[
\exp\colon\slalg(n,\bbR)\longrightarrow\SL(n,\bbR).
\]
The exponential map often allows a parametrization of the Lie group
elements by simpler objects, the Lie algebra elements.

One might ask, ``What happened to the Lie algebras $\glalg(n,\bbR)$ and
$\oalg(n)$ associated with the Lie group $\GL(n,\bbR)$ and $\oalg(n)$?'' We
will see later that $\glalg(n,\bbR)$ is the set of all real $n\times n$
matrices and that $\oalg(n)=\soalg(n)$.

The properties of the exponential map play an important role in the study
of Lie groups. For example, it is clear that the map
\[
\exp\colon\glalg(n,\bbR)\longrightarrow\GL(n,\bbR)
\]
is well-defined, but since $\det(e^A)=e^{\tr A}$, every matrix of the form
$e^A$ has a positive determinant and $\exp$ is not surjective.

%%% Local Variables:
%%% mode: latex
%%% TeX-master: "../MA598-Exercises"
%%% End:


%% Bibliography
% \bibliographystyle{plain}
% \bibliography{lie-bib}
\end{document}

%%% Local Variables:
%%% mode: latex
%%% TeX-master: t
%%% End:
