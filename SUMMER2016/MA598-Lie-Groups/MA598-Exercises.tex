\def\documentauthor{Carlos Salinas}
\def\documenttitle{Exercises in Basic Mathematics}
% \def\hwnum{1}
\def\shorttitle{Basic Mathematics}
\def\coursename{MA544}
\def\documentsubject{algebra, linear algebra, differential geometry, lie groups}
\def\authoremail{salinac@purdue.edu}

\documentclass[article,10pt]{memoir}
\usepackage{geometry}
\usepackage[dvipsnames]{xcolor}
\usepackage[
    breaklinks,
    bookmarks=true,
    colorlinks=true,
    pageanchor=false,
    linkcolor=black,
    citecolor=black,
    filecolor=black,
    menucolor=black,
    runcolor=black,
    urlcolor=black,
    % linkcolor=violet!85!black,
    % citecolor=YellowOrange!85!black,
    % urlcolor=Aquamarine!85!black,
    hyperindex=false,
    hyperfootnotes=true,
    pdftitle={\shorttitle},
    pdfauthor={\documentauthor},
    pdfkeywords={\documentsubject},
    pdfsubject={\coursename}
    ]{hyperref}
\usepackage{natbib}

%% Math
\usepackage{amsmath}
\usepackage{amsfonts}
\usepackage{amssymb}
\usepackage{amsthm}
\usepackage{mathtools}
% \usepackage{eucal}
% \usepackage{mathrsfs}
% \usepackage[nointegrals]{wasysym}

%% Language
% \usepackage{cmap}
% \usepackage[LAE,LFE,T2A,T1]{fontenc}
% \usepackage[utf8]{inputenc}
% \usepackage[farsi,french,german,spanish,russian,english]{babel}
% \babeltags{fr=french,
%            de=german,
%            en=english,
%            es=spanish,
%            pa=farsi,
%            ru=russian
%            }
% \def\spanishoptions{mexico}

% \selectlanguage{english}

% \newcommand{\textfa}[1]{\beginR\textpa{#1}\endR}

% \usepackage{CJKutf8}
% \newcommand{\textkr}[1]{\begin{CJK}{UTF8}{mj}#1\end{CJK}}
% \newcommand{\textjp}[1]{\begin{CJK}{UTF8}{min}#1\end{CJK}}
% \newcommand{\textzh}[1]{\begin{CJK}{UTF8}{bsmi}#1\end{CJK}}

%% Misc
\usepackage{graphicx}
\graphicspath{{figures/}}

\usepackage{microtype}
\usepackage{lineno}
\usepackage{multicol}
\usepackage[inline]{enumitem}
\usepackage{listings}
\usepackage{mleftright}
\mleftright
\usepackage{carlos-variables}

%% Unicode math and Polyglossia
\usepackage{unicode-math}
% \usepackage{unicode-minionmath}

% \setmainfont{CMU Serif}
% \setsansfont{CMU Sans Serif}
% \setmonofont{CMU Typewriter Text}
\setmainfont[Ligatures=TeX]{Latin Modern Roman}
\setsansfont{Latin Modern Sans}
\setsansfont{Latin Modern Mono}
% \setmathfont{Latin Modern Math}

% \setmainfont[Ligatures=TeX]{Libertinus Serif}
% \setsansfont{Libertinus Sans}
% \setmonofont{Libertinus Mono}
% \setmathfont{Minion Math}
% \setmathfont[range={\mathfrak}]{XITS Math}
% \setmathfont[range={\mathcal,\mathbfcal},StylisticSet=1]{XITS Math}
% \setmathfont[range=\mathscr]{XITS Math}
% % \setmathfont[range={\mathfrak}]{latinmodern-math.otf}
% % \setmathfont[range={\mathcal}]{latinmodern-math.otf}
% % \setmathfont[range={\mathscr}]{latinmodern-math.otf}
% \setmathfont[range={}]{Minion Math}

\usepackage{polyglossia}
\usepackage[english]{selnolig}

\newfontfamily\cyrillicfont[Script=Cyrillic]{CMU Serif}
\newfontfamily\cyrillicfontsf[Script=Cyrillic]{CMU Sans Serif}
\newfontfamily\cyrillicfonttt[Script=Cyrillic]{CMU Typewriter Text}

% \newfontfamily\cyrillicfont[Script=Cyrillic]{Libertinus Serif}
% \newfontfamily\cyrillicfontsf[Script=Cyrillic]{Libertinus Sans}

% \newfontfamily\farsifont[Script=Arabic,
%                          Scale=MatchUppercase]{Amiri}

\setmainlanguage[variant=american]{english}
% \setotherlanguage{farsi}
\setotherlanguage{french}
\setotherlanguage[spelling=new,latesthyphen,babelshorthands]{german}
\setotherlanguage{spanish}
\setotherlanguage[spelling=modern,babelshorthands]{russian}

% \makeatletter
% \@Latintrue
% \makeatother

\usepackage{xeCJK}
\usepackage[overlap]{ruby}
\renewcommand\rubysep{-0.2ex}
\setCJKmainfont[BoldFont=IPAGothic]{IPAMincho}

% \xeCJKDeclareSubCJKBlock{Kana}{"3040 -> "309F, "30A0 -> "30FF, "31F0 -> "31FF, "1B000 -> "1B0FF}
% \xeCJKDeclareSubCJKBlock{Hangul}{"1100 -> "11FF, "3130 -> "318F, "A960 -> "A97F, "AC00 -> "D7AF, "D7B0 -> "D7FF}

% \setCJKmainfont{HanaMinA}
% \setCJKmainfont[Kana]{HanaMinA}
% \setCJKmainfont[Hangul]{NanumMyeongjo}
% \setCJKsansfont[Hangul]{NanumGothic}

% % \setCJKmainfont{Adobe Ming Std}
% % \setCJKmainfont{Adobe Ming Std}
% % \setCJKmainfont[Kana]{Kozuka Mincho Pr6N}
% % \setCJKmainfont[Hangul]{Adobe Myungjo Std}
% % \setCJKsansfont[Hangul]{Adobe Gothic Std}


%% Theorems and definitions
%% remove parentheses
% \makeatletter
% \def\thmhead@plain#1#2#3{%
%   \thmname{#1}\thmnumber{\@ifnotempty{#1}{ }\@upn{#2}}%
%   \thmnote{ {\the\thm@notefont#3}}}
% \let\thmhead\thmhead@plain
% \makeatother

\theoremstyle{plain}
\newtheorem{theorem}{Theorem}
\newtheorem{proposition}[theorem]{Proposition}
\newtheorem{corollary}[theorem]{Corollary}
\newtheorem{claim}[theorem]{Claim}
\newtheorem{lemma}[theorem]{Lemma}
\newtheorem{axiom}[theorem]{Axiom}

\newtheorem*{corollary*}{Corollary}
\newtheorem*{claim*}{Claim}
\newtheorem*{lemma*}{Lemma}
\newtheorem*{proposition*}{Proposition}
\newtheorem*{theorem*}{Theorem}

\theoremstyle{definition}
\newtheorem{definition}{Definition}
\newtheorem{example}{Examples}
\newtheorem{examples}[example]{Example}
\newtheorem{exercise}{Exercise}[section]
\newtheorem{problem}[exercise]{Problem}

\newtheorem*{example*}{Example}
\newtheorem*{exercise*}{Exercise}
\newtheorem*{problem*}{Problem}

\begin{document}
%% Footnote style
\renewcommand*{\thefootnote}{\fnsymbol{footnote}}

%% Counters
\counterwithout{exercise}{chapter}
\numberwithin{equation}{section}
\counterwithout{equation}{chapter}

%% Redefine the QED symbol
% \renewcommand\qedsymbol{\ensuremath{\null\hfill\QED}}

\chapterstyle{veelo}
\pagestyle{ruled}
\author{\href{mailto:\authoremail}{\documentauthor}}
\title{\documenttitle}
\date{\today}
\maketitle
\tableofcontents

%% Notes
%% Basic Mathematics
\chapter{Basic Mathematics Exercises}

%%% Local Variables:
%%% mode: latex
%%% TeX-master: "../MA598-Exercises"
%%% End:


%% Abstract Algebra
\chapter{Algebra Exercises}

%%% Local Variables:
%%% mode: latex
%%% TeX-master: "../MA598-Exercises"
%%% End:


%% Algebraic Geometry
\chapter{Algebraic Geometry Exercises}
\section{Elementary Algebraic Geometry}
% \begin{example}
% Consider the equation
% \[
% x^2+y^2=1.
% \]
% Over $\bfR$, a good picture of the solution is a circle.

% Over $\bfC$, it is a $2$-sphere without two points. This can be seen as
% follows. By stereographic projection from the North pole onto the
% equatorial plane, the complex plane $\bfC$ is in bijection with the sphere
% $S^2$ with the North pole $N$ removed. The equation
% \[
% x^2+y^2=1
% \]
% can be written as
% \[
% (x+iy)(x-iy)=1,
% \]
% an by letting $w\coloneqq x+iy$ and $z\coloneqq x-iy$, we see that
% it is equivalent to
% \[
% wz=1.
% \]
% Clearly, every $w\neq 0$ determines a unique $z$, and thus, the solution
% set is indeed $S^2\smallsetminus\{N,S\}$.
% \end{example}
% \begin{example}
% Note that $\bar k$ is the solution set corresponding to the empty ideal in
% $k[x]$. Similarly ${\bar k}^n$ is the solution set corresponding to the
% empty ideal in $k[x_1,\dotsc,x_n]$. We also denote ${\bar k}^n$ by
% $\bfA^n({\bar k})$, and call it \emph{the points of affine $n$-space over
%   $\bar k$}.

% Can we view $\bfA^1({\bar k})\smallsetminus\{0\}$ as the solution set of
% equations? Yes indeed. Let $\frakU$ be the ideal, $(zw-1)$, generated by
% the polynomial $zw-1\in k[z,w]$. The solutions of
% \[
% zw-1=0
% \]
% are in bijection with the set of all $z\in\bfA^1(\bar
% k)\smallsetminus\{0\}$.
% \end{example}
% \begin{example}
% We will prove later on that $\bfA^2(\bar k)\smallsetminus\{0\}$ is not the
% solution set of any set of equations. On the other hand, $\bfA^2(\bar
% k)\smallsetminus\{0\}$ is ``locally'' an algebraic solution set. it is
% possible to cover $\bfA^2(\bar k)\smallsetminus\{0\}$ with two ``affine
% patches.'' Indeed,
% \begin{enumerate}[label=(\alph*),noitemsep]
% \item Consider $k[x,y,z]$ and the equation
% \[
% xz=1.
% \]
% \item Consider $k[x,y,t]$ and the equation
% \[
% yt=1.
% \]
% \end{enumerate}
% The solution set, $\dagger_a$, of $xz=1$ is in bijection with
% \[
% \left\{\,(x,y)\in\bfA^2(\bar k):x\neq 0\,\right\}
% \]
% and the solution set, $\dagger_b$, of $yt=1$ is in bijection with
% \[
% \left\{\,(x,y)\in\bfA^2(\bar k):y\neq 0\,\right\},
% \]
% and thus,
% \[
% \dagger_a\cup\dagger_b=\bfA^2(\bar k)\smallsetminus\{0\}.
% \]
% \end{example}

% Generally speaking, to ``do geometry,''  we need
% \begin{enumerate}[label=(\alph*),noitemsep]
% \item A topological space.
% \item A notion of locally standard objects. For example, in the case of
%   real manifolds, a ball in $\bfR^n$. In the case of complex manifold, a
%   ball in $\bfC^n$. In the case of algebraic varieties, something defined
%   by a system of the form (1.8).
% \item Some set of functions on the space (perhaps locally defined). For
%   example, in the real case, $C^k$-functions, or smooth functions, or
%   analytic functions. In the complex case, holomorphic functions.
% \item Maps between objects defined by (1), (2), and (3).
% \end{enumerate}

% Another theme in algebraic geometry is that of a classifying space (or
% moduli space). Assume that we have some geometric algebraic object
% $X$. This object $X$ is at least a topological space.

% \textsc{Question}: Given $X$, with some topological structure, ``classify''
% all the algebraic structures it carries, compatible with the underlying
% topological structure.

% \begin{example}
% Consider the elliptic curve of equation
% \[
% y^2=ax^3+bx+c,\qquad
% a,b,c\in\bfC,
% \]
% where the right-hand side has distinct roots. Geometrically, this is a
% one-holed torus with one point missing. If we compactify, we obtain the
% usual torus.
% \begin{enumerate}[label=\textsc{Problem \arabic*.}]
% \item What are the algebraic structures (up to some suitable notion of
%   isomorphism) carried by the torus?

%   The collection
% \item
% \end{enumerate}
% \end{example}
\section{Affine Geometry (first level of abstraction), Zariski Topology}
\begin{definition}
Given any ideal $\fraka\subset k[X_1,\dotsc,X_q]$, define $V_k(\fraka)$
by
\[
V_k(\fraka)\coloneqq\left\{\,\bfx\in\bfA^q:\text{for every
$f\in\fraka$, $f(\bfx)=0$}\,\right\}.
\]
We call $V_k(\fraka)$ the \emph{set of $\Omega$-points of the affine
  $k$-variety determined by $\fraka$.} With a slight abuse of language, we
call $V_k(\fraka)$ the \emph{affine $k$-variety determined by $\fraka$.}
Similarly, given by any ideal $\fraka\subset\bar k[X_1,\dotsc,X_q]$,
defined by $V_{\bar k}(\fraka)$ by
\[
V_{\bar k}(\fraka)\coloneqq\left\{\,\bfx\in \bfA^q:
\text{for every $f\in\fraka$, $f(\bfx)=0$}\,\right\}.
\]
We call $V_{\bar k}(\fraka)$ the \emph{set of $\Omega$-points of the
  (geometric) affine $\bar k$-variety determined by $\fraka$,} or for
short, the \emph{(geometric) affine variety determined by $\fraka$.}
\end{definition}

To ease the notation, we usually drop the subscript $k$ or $\bar k$ and
simply write $V$.

If $A$ is a (commutative) ring (with unit $1$), recall that the
\emph{radical}, $\sqrt{\frakb}$, of an ideal, $\fraka\subset A$, is defined
by
\[
\sqrt{\fraka}\coloneqq\left\{\,a\in A:\text{there exists $n\geq 1$,
    $a^n\in\fraka$}\,\right\}.
\]
A \emph{radical ideal} is an ideal, $\fraka$, such that
$\fraka=\sqrt{\fraka}$.

The following properties are easily verified. We state them for $V_k$, but
they also hold for $V_{\bar k}$:
\begin{align*}
&V(0)=\text{$\bfA^n$, $V(A)=\emptyset$}\\
&V(\fraka\cap\frakb)=V(\fraka\frakb)=V(\fraka)\cup V(\frakb)\\
&\fraka\subset\text{$\frakb$ implies that $V(\frakb)\subset V(\frakb)$}\\
&V\left({\textstyle\sum_\alpha\fraka_\alpha}\right)
={\textstyle \bigcap_\alpha V(\fraka_\alpha)}\\
&V(\sqrt{\fraka})=V(\fraka)
\end{align*}


%%% Local Variables:
%%% mode: latex
%%% TeX-master: "../MA598-Exercises"
%%% End:


%% Differential Geometry
\chapter{Differential Geometry Exercises}
\section{The Matrix Exponential; Some Matrix Lie Groups}
\subsection{The Exponential Map}
\subsection{The Lie Groups $\GL(n,\bbR)$, $\SL(n,\bbR)$, $\oalg(n)$,
  $\SO(n)$, the Lie Algebras $\glalg(n,\bbR)$, $\slalg(n,\bbR)$,
  $\oalg(n)$, $\soalg(n)$, and the Exponential Map}
The notation $\slalg(n,\bbR)$ and $\soalg(n)$ is rather strange and
deserves some explanation. The groups $\GL(n,\bbR)$, $\SL(n,\bbR)$,
$\oalg(n)$, and $\SO(n)$ are more than just groups. They are also
topological groups, which means that they are topological spaces (viewed as
subspaces of $\bbR^{n^2}$) and that the multiplication and the inverse
operations are smooth. In fact, they are smooth real manifolds. Such
objects are called \emph{Lie groups}. The real vector spaces $\slalg(n)$
and $\soalg(n)$ are what are called \emph{Lie algebras}. However, we have
not defined the algebra structure on $\slalg(n,\bbR)$ and $\soalg(n)$
yet. The algebra structure is called the \emph{Lie bracket}, which is
defined as
\[
[A,B]=AB-BA.
\]

Lie algebras are associated with Lie groups. What is going on is that the
Lie algebra of a Lie group is its tangent space at the identity, i.e., the
space of all tangent vectors at the identity. In some sense, the Lie
algebra achieves a linearization of the Lie group. The exponential map is a
map from the Lie algebra to the Lie group, for example,
\[
\exp\colon\soalg(n)\longrightarrow\SO(n)
\]
and
\[
\exp\colon\slalg(n,\bbR)\longrightarrow\SL(n,\bbR).
\]
The exponential map often allows a parametrization of the Lie group
elements by simpler objects, the Lie algebra elements.

One might ask, ``What happened to the Lie algebras $\glalg(n,\bbR)$ and
$\oalg(n)$ associated with the Lie group $\GL(n,\bbR)$ and $\oalg(n)$?'' We
will see later that $\glalg(n,\bbR)$ is the set of all real $n\times n$
matrices and that $\oalg(n)=\soalg(n)$.

The properties of the exponential map play an important role in the study
of Lie groups. For example, it is clear that the map
\[
\exp\colon\glalg(n,\bbR)\longrightarrow\GL(n,\bbR)
\]
is well-defined, but since $\det(e^A)=e^{\tr A}$, every matrix of the form
$e^A$ has a positive determinant and $\exp$ is not surjective.

%%% Local Variables:
%%% mode: latex
%%% TeX-master: "../MA598-Exercises"
%%% End:


%% Bibliography
% \bibliographystyle{plain}
% \bibliography{lie-bib}
\end{document}

%%% Local Variables:
%%% mode: latex
%%% TeX-master: t
%%% End:
