\chapter{Introduction}
In any algebra textbook, the study of group theory is usually concerned
about the theory of finite (or at least finitely generated)
groups. However, most groups which appear as groups of symmetries of
various geometric objects are not finite: for example, the group
$\SO_3\bbR$ of all rotations of three-dimensional space is not finite and
is not even finitely generated. Thus, much of the material learned in a
basic algebra course does not apply here; for exampel, it is not clear,
whether, say, the set of all morphisms between such groups can be
explicitly described.

The theory of Lie groups answers these questions by replacing the notion of
a finitely generated group by that of a Lie group --- a group which at the
same time is a finite dimensional manifold. It turns out that in many ways
such groups can be described and studied as easily as finitely generated
groups --- or even easier. The key role is played by the notion of a Lie
algebra, the tangent space to $G$ at the identity. It turns out that the
group operation on $G$ defines a certain skew-symmetric bilinear form on
$\frakg\coloneq T_1G$; axiomatizing the properties of this operation gives
a definition of the Lie algebra.

The fundamental result of the theory of Lie groups is that many properties
of Lie groups are completely determined by the properties of corresponding
Lie algebras. For example, the set of morphisms between two (connected and
simply connected) Lie groups is the same as the set of morphisms between
the corresponding Lie algebras; thus, describing them is essentially
reduced to a linear algebra problem.

Similarly, Lie algebras also provide a key to the study of the structure of
Lie groups and their representations. In particular, this allows one to get
a complete classification of Lie groups (semisimple and more generally,
reductive Lie groups; this includes all compact Lie groups and all
classical Lie groups such as $\SO_n\bbR$) in terms of relatively simple
geometric objects so-called root systems. This result is considered by many
mathematicians to be one of the most beautiful achievements in all
mathematics.

To conclude this introduction, we will give  a simple example which shows
that Lie groups naturally appear as groups of symmetries of various
objects --- and how one can use the theory of Lie grousp and Lie algebras
to make use of these symmetries.

Let $S^2\subset\bbR^3$ be the unit sphere. Define the Laplace operator
$\Delta\colon C^\infty(S^2)\to C^\infty(S^2)$ by
$\Delta_{\text{sph}}\,f\coloneq\left.\Delta(\tilde f)\right|_{S^2}$, where
$\tilde f$ is the result of extending $f$ to $\bbR^3\setminus\{(0,0,0)\}$
(constant along each ray), and $\Delta$ is the usual Laplace operator in
$\bbR^3$. It is easy to see that $\Delta_{\text{sph}}$ is a second order
differential operator on the sphere; one can write explicit formulas for it
in the spherical coordinates, but they are not particularly nice.

For many applications, it is important to know that the eigenvalues and
eigenfunctions of $\Delta_{\text{sph}}$. In particular, this problem arises
in quantum mechanics: the eigenvalues are related to to the energy levels
of a hydrogen atom in quantum mechanical description. Unfortunately, trying
to find the eigenfunctions by brute force gives a second-order differential
equation which is very difficult to solve.

However, it is easy to notice that this problem has some symmetry ---
namely, the group $\SO_3\bbR$ acting on the sphere by rotations. How can
one us this symmetry?

If we had just one symmetry, given by some rotation $R\colon S^2\to S^2$,
we could consider its action on the space of complex-valued functions
$C^\infty(S^2,\bbC)$. If we could diagonalize this operator, this would
help us study $\Delta_{\text{sph}}$: it is a general result of linear
algebra that if $A$ and $B$ are two linear operators, and $A$ is
diagonalizable, then $B$ must preserve eigenspaces for $A$. Applying this
to the pair $R$, $\Delta_{text{sph}}$, we get that $\Delta_{\text{sph}}$
preserves eigenspaces for $R$, so we can diagonalize $\Delta_{\text{sph}}$
independently in each of the eigenspaces.

HOwever, this will not solve the problem: for each individual rotation $R$,
the eigenspace will still be too large (in fact, infinite-dimensional), so
diagonalizing $\Delta_{\text{sph}}$ in each of them is not very easy
either. This is not surprising: after all, we only used one of many
symmetries. Can we use all of the rotations in $\SO_3\bbR$ simultaneously?

This however presents two problems
\begin{itemize}
\item $\SO_3\bbR$ is not a finitely generated group, so apparently we will
  need to use infinitely (in fact, uncountably) many different symmetries
  and diagonalize each of them.
\item $\SO_3\bbR$ is not commutative, so different operators from
  $\SO_3\bbR$ cannot be diagonalized simultaneously.
\end{itemize}

The goal of the theory of Lie groups is to give tools to deal with these
(and similar) problems. IN short, the answer to the first problem is that
$\SO_3\bbR$ is in a certain sense finitely generated --- namely, it is
generated by three generators, infinitesimal rotations around the $x$-,
$y$-, $z$-axes.

The answer to the second problem is that instead of decomposing the
$C^\infty(S^2,\bbC)$ into adirect sum of common eigenspaces for operators
$R\in\SO_3\bbR$, we need to decompose it into irreducible representations
of $\SO_3\bbR$. In order to do this, we need to develop the theory of
representations of $\SO_3\bbR$. WE will do this an complete the analysis of
this example in a couple of sections.


%%% Local Variables:
%%% mode: latex
%%% TeX-master: "../MA598-Lie-Groups"
%%% End:
