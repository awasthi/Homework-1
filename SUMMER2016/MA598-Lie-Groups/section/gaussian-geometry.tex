\chapter{Gaussian Geometry}
To begin, we will review some Gaussian geometry.

Let the $n$-dimensional real vector space $\bbR^n$ be equipped with its
standard Euclidean scalar product
$\langle\cdot,\cdot\rangle\colon\bbR^n\times\bbR^n\to\bbR$. This is
given by
\[
\langle \bfx,\bfy \rangle\coloneqq x_1y_1+\dotsb+x_ny_n
\]
and induces the norm $|\cdot|\colon\bbR^n\to\bbR_0^+$ on $\bbR^n$ is given
by
\[
|\bfx|=\sqrt{{x_1}^2+\dotsb+{x_n}^2}.
\]

\begin{definition}
A map $\Phi\colon\bbR^n\to\bbR^n$ is said to be a Euclidean motion of
$\bbR^n$ if it is given by $\Phi(\bfx)\coloneqq A\bfx+\bfb$ where
$\bfb\in\bbR^n$ and $A\in\Orth(n)$ where
\[
\Orth(n)\coloneqq\left\{\,X\in\bbR^{n\times n}:X^\rmT X=I\,\right\}.
\]
A Euclidean motion $\Phi$ is said to be rigid or orientation preserving if
$A\in\SO(n)$ where
\[
\SO(n)\coloneqq\left\{\,X\in\Orth(n):{\det X=1}\,\right\}.
\]
\end{definition}

\begin{definition}
A parametericzed curve in $\bbR^n$ is a $C^1$-map $\gamma\colon I\to\bbR^n$
from an open interval $I$ on the real line $\bbR$. The image $\gamma(I)$ in
$\bbR^n$ is the corresponding geometric curve. We say that the map
$\gamma\colon I\to\bbR^n$ parametrizes $\gamma(I)$. The derivative
$\dot\gamma(t)$ is called the tangent of $\gamma$ at the point $\gamma(t)$ and
\[
L(\gamma)\coloneqq\int_I|\dot\gamma(t)|\diff t\leq\infty
\]
is the arclength of $\gamma$. The curve $\gamma$ is said to be regular if
$\gamma'(t)\neq 0$ for all $t\in I$.
\end{definition}

\begin{example}
For example, if $\bfp$ and $\bfq$ are two distinct points in $\bbR^n$, then
$\gamma\colon\bbR\to\bbR^n$ with
\[
\gamma(t)\coloneqq (1-t)\bfp+t\bfq
\]
parametrizes the straight line through $\bfp=\gamma(0)$ and
$\bfq=\gamma(1)$.
\end{example}

\begin{example}
If $r\in\bbR^+$ and $\bfp\in\bbR^n$ then $\gamma\colon\bbR\to\bbR^2$ with
\[
\gamma(t)\coloneqq\bfp+r(\cos t,\sin t)
\]
parametrizes a circle with center $\bfp$ and radius $r$. The arclength of
the curve $\gamma|_{(0,2\pi)}$ is
\[
L(\gamma;0,2\pi)=\int_0^{2\pi}|\dot\gamma(t)|\diff t=2\pi r.
\]
\end{example}

\begin{definition}
A regular curve $\gamma\colon(a,b)\to\bbR^n$ is said to parametrize
$\gamma(a,b)$ by arclength if $\dot\gamma(s)=1$ for all $s\in(a,b)$, i.e.,
the tangent $\dot\gamma(s)$ are elements of the unit sphere $S^{n-1}$ in
$\bbR^n$.
\end{definition}

\begin{theorem}
Let $\gamma\colon I\to\bbR^n$ be a regular curve in $\bbR^n$. Then the
image $\gamma(I)$ of $\gamma$ can be parametrized by arclength.
\end{theorem}

\begin{proof}
Define the arclength function $\sigma\colon (a,b)\to\bbR^+$ by
\[
\sigma(t)\coloneqq\int_a^t|\dot\gamma(u)|\diff u
\]
Then $\dot\sigma(t)=|\dot\gamma(t)|>0$ so $\sigma$ is strictly increasing
and
\[
\sigma(a,b)=(0,L(\gamma)).
\]
Let $\tau\colon(0,L(\gamma))\to(a,b)$ be the inverse of $\sigma$ such that
$\sigma(\tau(s))=s$ for all $s\in(0,L(\gamma))$. By differentiating we get
\[
\frac{d}{ds}(\sigma(\tau(s)))=\dot\sigma(\tau(s))\dot\tau(s)=1.
\]
\end{proof}

%%% Local Variables:
%%% mode: latex
%%% TeX-master: "../MA598-Lie-Groups"
%%% End:
