\section{Lie Groups}
This week we will be talking about the first chapter of Knapp's `Lie
Groups Beyond an Introduction.' We will begin with basic definitions.

\subsection{Definitions and examples}
Let $\bbk$ be a field. An \emph{algebra $\frakg$} is a vector space over
$\bbk$ with a bilinear product $[X,Y]$. Additionally, $\frakg$ is said to
be a \emph{Lie algebra} if the product satisfies
\begin{enumerate}[label=(\alph*),noitemsep]
\item $[X,X]=0$ for all $X\in\frakg$ (and hence, is antisymmetric, i.e.,
  $[X,Y]=-[Y,X]$) and
\item the \emph{Jacobi identity} $[[X,Y],Z]+[[Y,Z],X]+[[Z,X],Y]=0$.
\end{enumerate}

For any algebra $\frakg$, we get a linear map
$\ad\colon\frakg\to\End_\bbk\frakg$ given by
\[
  (\ad X)(Y)=[X,Y].
\]
The fact that the image is in $\End_\bbk\frakg$ follows from the linearity
of the bracket in the second variable, and the fact that $\ad$ is linear,
follows from the linearity of the bracket in the first variable.

Suppose that (a) holds in the definition of Lie algebra. Then (b) holds if
and only if
\[
  [Z,[X,Y]]=[X,[Z,Y]]+[[Z,X],Y],
\]
which holds if and only if
\[
  (\ad Z)[X,Y]=[X,(\ad Z)(Y)]+[(\ad Z)(X),Y].
\]
Any $D$ in $\End_\bbk\frakg$ for which
\[
D[X,Y]=[X,DY]+[DX,Y]
\]
is called a \emph{derivation}. We have just seen that in a Lie algebra,
every $\ad X$ is a derivation. Conversely, if (a) holds and if every $\ad
X$ for $X\in\frakg$ is a derivation, then $\frakg$ is a Lie algebra.

Now, let us make some definitions concerning the Lie algebra. A
\emph{homomorphism} of Lie algebras is a linear map
$\varphi\colon\frakg\to\frakh$ such that
\[
  \varphi([X,Y]_\frakg)=[\varphi(X),\varphi(Y)]_\frakh
\]
for all $X$ and $Y$ in $\frakg$. If $\fraka$ and $\frakb$ are subsets of
$\frakg$, we write
\[
  [\fraka,\frakb]\coloneq\Span\left\{\,[X,Y]:X\in\fraka,Y\in\frakb\,\right\}.
\]
A \emph{subalgebra} or \emph{Lie subalgebra} $\frakh$ of $\frakg$ is a
subspace satisfying $[\frakh,\frakh]\subset\frakh$; $\frakh$ itself is a
Lie algebra with the product $[X,Y]$ restricted to $\frakh$. An \emph{ideal
$\frakh$} in $\frakg$ is a subspace satisfying
$[\frakh,\frakg]\subset\frakh$; $\frakh$ is automatically a subalgebra. The
Lie algebra $\frakg$ is said to be \emph{Abelian} if $[\frakg,\frakg]=0$; a
vector space with all brackets defined to be zero is automatically an
Abelian Lie algebra.

\begin{examples}
  \begin{enumerate}[label=(\arabic*)]
  \item Let $U$ be an open set in $\bbR^n$. A \emph{smooth vector field} on
    $U$ is any operator on smooth functions on $U$ of the form
    $X=\sum_{i=1}^na_i(x)\partial/\partial x_i$ with all $a_i(x)$ in
    $C^\infty(U)$. The real vector space $\frakg$ over all smooth vector
    fields on $U$ becomes a Lie algebra if the bracket is defined by
    $[X,Y]\coloneq XY-YX$. The skew-symmetry and the Jacobi identity follow
    from the next example applied to the associative algebra of all
    operators generated by all smooth vector fields.
  \item Let $\frakg$ be an associative algebra. Then $\frakg$ becomes a Lie
    algebra under $[X,Y]\coloneq XY-YX$. Certainly, $[X,X]=0$. For the
    Jacobi identity, we have
    \begin{align*}
      [[X,Y],Z]+[[Y,Z],X]+[[Z,X],Y]
      &\begin{aligned}
        =&[X,Y]Z-Z[X,Y]\\
        &+[Y,Z]X-X[Y,Z]\\
        &+[Z,X]Y-Y[Z,X]\\
      \end{aligned}\\
      &\begin{aligned}
        =&(XY-YX)Z-Z(XY-YX)\\
        &+(YZ-ZY)X-X(YZ-ZY)\\
        &+(ZX-XZ)Y-Y(ZX-XZ)\\
      \end{aligned}\\
      &=
    \end{align*}
  \item Let $\frakg\coloneq\frakgl(n,\bbk)$ denote the associative algebra
    of all $n$-by-$n$ matrices with entries in the field $\bbk$, and define
    a bracket product by $[X,Y]\coloneq XY-YX$. Then $\frakg$ becomes a Lie
    algebra. The special case of $\frakgl(n,\bbk)$ arises when $V$ is the
    vector space $\bbk^n$ of all $n$-dimensional column vectors over
    $\bbk$.
  \item Example 1 generalizes to any smooth manifold $M$. The vector space
    of all smooth vector fields on $M$ becomes a real Lie algebra if the
    bracket is defined by $[X,Y]\coloneq XY-YX$.
  \item (Review of the \emph{Lie algebra of a Lie group}) Let $G$ be a Lie
    group. If $f\colon G\to\bbR$ is a smooth function and if $g$ is in $G$,
    let $f_g$ be the left translate $f_g(x)=f(gx)$.
  \end{enumerate}
\end{examples}

%%% Local Variables:
%%% mode: latex
%%% TeX-master: "../MA598-Lie-Groups"
%%% End:
