\section{Lie Algebras and Lie Groups}
The first part of this chapter treats Lie algebras, beginning with
definitions and many examples. The notions of solvable, nilpotent, radical,
semisimple, and simple are introduced, and these notions are followed by a
discussion of the effect of a change of the underlying field.

The idea of a semidirect products begins the development of the main
structural theorems for real Lie algebras---the iterated construction of
all solvable Lie algebras form derivations and semidirect products, Lie's
theorem for solvable Lie algebras, Engel's theorem in connection with
nilpotent Lie algebras, and Cartan's criteria for solvability and
semisimplicity in terms of the Killing form. From Cartan's criterion for
semisimplicity, it follows that semisimple Lie algebras are direct sums of
simple Lie algebras.

Cartan's criterion for semisimplicity is used also to provide a long list
of classical examples of semisimple Lie algebras. Some of these examples
are defined in terms of quaternion matrices. QUaternion matrices of size
$n\times n$ may be related to complex matrices of size $2n\times 2n$.

The treatment of Lie algebras concluded with a study of the
finite-dimensional complex-linear representation of $\fraksl(2,\bbC)$.


%%% Local Variables:
%%% mode: latex
%%% TeX-master: "../MA598-Lie-Groups"
%%% End:
