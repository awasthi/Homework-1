\chapter{Algebraic Geometry Exercises}
\section{Elementary Algebraic Geometry}
\section{Affine Geometry (first level of abstraction), Zariski Topology}
\begin{definition}
Given any ideal $\fraka\subset k[X_1,\dotsc,X_q]$, define $\calV_k(\fraka)$
by
\[
\calV_k(\fraka)\coloneqq\left\{\,\bfx\in\bbA^q:\text{for every
$f\in\fraka$, $f(\bfx)=0$}\,\right\}.
\]
We call $\calV_k(\fraka)$ the \emph{set of $\Omega$-points of the affine
  $k$-variety determined by $\fraka$.} With a slight abuse of language, we
call $\calV_k(\fraka)$ the \emph{affine $k$-variety determined by $\fraka$.}
Similarly, given by any ideal $\fraka\subset\bar k[X_1,\dotsc,X_q]$,
defined by $\calV_{\bar k}(\fraka)$ by
\[
\calV_{\bar k}(\fraka)\coloneqq\left\{\,\bfx\in \bbA^q:
\text{for every $f\in\fraka$, $f(\bfx)=0$}\,\right\}.
\]
We call $\calV_{\bar k}(\fraka)$ the \emph{set of $\Omega$-points of the
  (geometric) affine $\bar k$-variety determined by $\fraka$,} or for
short, the \emph{(geometric) affine variety determined by $\fraka$.}
\end{definition}

To ease the notation, we usually drop the subscript $k$ or $\bar k$ and
simply write $\calV$.

If $A$ is a (commutative) ring (with unit $1$), recall that the
\emph{radical}, $\sqrt{\frakb}$, of an ideal, $\fraka\subset A$, is defined
by
\[
\sqrt{\fraka}\coloneqq\left\{\,a\in A:\text{there exists $n\geq 1$,
    $a^n\in\fraka$}\,\right\}.
\]
A \emph{radical ideal} is an ideal, $\fraka$, such that
$\fraka=\sqrt{\fraka}$.

The following properties are easily verified. We state them for $\calV_k$, but
they also hold for $\calV_{\bar k}$:
\begin{align*}
&\text{$\calV(0)=\bbA^n$, $\calV(A)=\emptyset$}\\
&\calV(\fraka\cap\frakb)=\calV(\fraka\frakb)=\calV(\fraka)\cup\calV(\frakb)\\
&\text{$\fraka\subset\frakb$ implies that $\calV(\frakb)\subset\calV(\frakb)$}\\
&\calV\left({\textstyle\sum_\alpha\fraka_\alpha}\right)
={\textstyle \bigcap_\alpha\calV(\fraka_\alpha)}\\
&\calV(\sqrt{\fraka})=\calV(\fraka)
\end{align*}
From the relations above, it follows that the sets $\calV(\fraka)$ can be taken
as closed subsets of $\bbA^q$, and we obtain a topology on $\bbA^q$. This
is the \emph{$k$-topology on $\bbA^q$.} If we consider ideals in $\bar
k[X_1,\dotsc,X_q]$ (i.e., sets of the form $\calV_{\bar k}(\fraka)$), we obtain
the \emph{Zariski topology on $\bbA^q$.}

The Zariski topology is not necessarily Hausdorff (except when $\calV(\fraka)$
consits of a finite set of points.)

Let us see that $\bbA^q$ is not Hausdorff in the Zariski topology. Let
$P,Q\in\bbA^q$, with $P\neq Q$. The line $\overleftrightarrow{PQ}$ is
isomorphic to $\bbA^1$. Thus, it is enough to show that $\bbA^1$ is not
Hausdorff. Consider any ideal $\fraka\subset\bar k[X]$. Then, $\fraka$ is a
principal ideal, and thus
\[
\fraka=(f)
\]
for some polynomial $f$, which shows that $\calV(\fraka)=\calV(f)$ is a finite
set. As a consequence, the closed sets of $\bbA^1$ (other than $\bbA^1$)
are finite. Then, the union of two closed sets (distinct from $\bbA^1$) is
also finite, and thus distinct from $\bbA^1$.

The topology on $\bbA^q$ is not the product topology on
$\prod_{i=1}^q\bbA^1$.

For example, when $n=2$, the closed set in $\bbA^1\times\bbA^1$ are those
sets consisting of finitely many horizontal and vertical lines, and
intersections of such sets. However
\[
X^2+Y^2-1=0
\]
defines a closed set in $\bbA^2$ not of the previous form.

To go backwards from subsets of $\bbA^q$ to ideals, we make the following
definition.
\begin{definition}
 Given any subset $S\subset\bbA^q$, define $\calI_k(S)$ and $\calI_{\bar
   k}(S)$ by
\[
\calI_k(S)\coloneqq\left\{\,f\in k[X_1,\dotsc,X_q]:
\text{for every $s\in S$, $f(s)=0$}\,\right\}
\]
and
\[
\calI_{\bar k}(S)\coloneqq\left\{\,f\in\bar k[X_1,\dotsc,X_q]:
\text{for every $s\in S$, $f(s)=0$}\,\right\}
\]
\end{definition}

The following properties are easily shown (following our conventions, they
are stated for $\calI_k$, but they are easily shown for $\calI_{\bar k}$).



%%% Local Variables:
%%% mode: latex
%%% TeX-master: "../MA598-Exercises"
%%% End:
