\chapter{Algebraic Geometry Exercises}
\section{Elementary Algebraic Geometry}
\begin{example}
Consider the equation
\[
x^2+y^2=1.
\]
Over $\bfR$, a good picture of the solution is a circle.

Over $\bfC$, it is a $2$-sphere without two points. This can be seen as
follows. By stereographic projection from the North pole onto the
equatorial plane, the complex plane $\bfC$ is in bijection with the sphere
$S^2$ with the North pole $N$ removed. The equation
\[
x^2+y^2=1
\]
can be written as
\[
(x+iy)(x-iy)=1,
\]
an by letting $w\coloneqq x+iy$ and $z\coloneqq x-iy$, we see that
it is equivalent to
\[
wz=1.
\]
Clearly, every $w\neq 0$ determines a unique $z$, and thus, the solution
set is indeed $S^2\smallsetminus\{N,S\}$.
\end{example}

%%% Local Variables:
%%% mode: latex
%%% TeX-master: "../MA598-Exercises"
%%% End:
