\chapter{Algebraic Geometry Exercises}
\section{Elementary Algebraic Geometry}
% \begin{example}
% Consider the equation
% \[
% x^2+y^2=1.
% \]
% Over $\bfR$, a good picture of the solution is a circle.

% Over $\bfC$, it is a $2$-sphere without two points. This can be seen as
% follows. By stereographic projection from the North pole onto the
% equatorial plane, the complex plane $\bfC$ is in bijection with the sphere
% $S^2$ with the North pole $N$ removed. The equation
% \[
% x^2+y^2=1
% \]
% can be written as
% \[
% (x+iy)(x-iy)=1,
% \]
% an by letting $w\coloneqq x+iy$ and $z\coloneqq x-iy$, we see that
% it is equivalent to
% \[
% wz=1.
% \]
% Clearly, every $w\neq 0$ determines a unique $z$, and thus, the solution
% set is indeed $S^2\smallsetminus\{N,S\}$.
% \end{example}
% \begin{example}
% Note that $\bar k$ is the solution set corresponding to the empty ideal in
% $k[x]$. Similarly ${\bar k}^n$ is the solution set corresponding to the
% empty ideal in $k[x_1,\dotsc,x_n]$. We also denote ${\bar k}^n$ by
% $\bfA^n({\bar k})$, and call it \emph{the points of affine $n$-space over
%   $\bar k$}.

% Can we view $\bfA^1({\bar k})\smallsetminus\{0\}$ as the solution set of
% equations? Yes indeed. Let $\frakU$ be the ideal, $(zw-1)$, generated by
% the polynomial $zw-1\in k[z,w]$. The solutions of
% \[
% zw-1=0
% \]
% are in bijection with the set of all $z\in\bfA^1(\bar
% k)\smallsetminus\{0\}$.
% \end{example}
% \begin{example}
% We will prove later on that $\bfA^2(\bar k)\smallsetminus\{0\}$ is not the
% solution set of any set of equations. On the other hand, $\bfA^2(\bar
% k)\smallsetminus\{0\}$ is ``locally'' an algebraic solution set. it is
% possible to cover $\bfA^2(\bar k)\smallsetminus\{0\}$ with two ``affine
% patches.'' Indeed,
% \begin{enumerate}[label=(\alph*),noitemsep]
% \item Consider $k[x,y,z]$ and the equation
% \[
% xz=1.
% \]
% \item Consider $k[x,y,t]$ and the equation
% \[
% yt=1.
% \]
% \end{enumerate}
% The solution set, $\dagger_a$, of $xz=1$ is in bijection with
% \[
% \left\{\,(x,y)\in\bfA^2(\bar k):x\neq 0\,\right\}
% \]
% and the solution set, $\dagger_b$, of $yt=1$ is in bijection with
% \[
% \left\{\,(x,y)\in\bfA^2(\bar k):y\neq 0\,\right\},
% \]
% and thus,
% \[
% \dagger_a\cup\dagger_b=\bfA^2(\bar k)\smallsetminus\{0\}.
% \]
% \end{example}

% Generally speaking, to ``do geometry,''  we need
% \begin{enumerate}[label=(\alph*),noitemsep]
% \item A topological space.
% \item A notion of locally standard objects. For example, in the case of
%   real manifolds, a ball in $\bfR^n$. In the case of complex manifold, a
%   ball in $\bfC^n$. In the case of algebraic varieties, something defined
%   by a system of the form (1.8).
% \item Some set of functions on the space (perhaps locally defined). For
%   example, in the real case, $C^k$-functions, or smooth functions, or
%   analytic functions. In the complex case, holomorphic functions.
% \item Maps between objects defined by (1), (2), and (3).
% \end{enumerate}

% Another theme in algebraic geometry is that of a classifying space (or
% moduli space). Assume that we have some geometric algebraic object
% $X$. This object $X$ is at least a topological space.

% \textsc{Question}: Given $X$, with some topological structure, ``classify''
% all the algebraic structures it carries, compatible with the underlying
% topological structure.

% \begin{example}
% Consider the elliptic curve of equation
% \[
% y^2=ax^3+bx+c,\qquad
% a,b,c\in\bfC,
% \]
% where the right-hand side has distinct roots. Geometrically, this is a
% one-holed torus with one point missing. If we compactify, we obtain the
% usual torus.
% \begin{enumerate}[label=\textsc{Problem \arabic*.}]
% \item What are the algebraic structures (up to some suitable notion of
%   isomorphism) carried by the torus?

%   The collection
% \item
% \end{enumerate}
% \end{example}
\section{Affine Geometry (first level of abstraction), Zariski Topology}
\begin{definition}
Given any ideal $\fraka\subset k[X_1,\dotsc,X_q]$, define $\calV_k(\fraka)$
by
\[
\calV_k(\fraka)\coloneqq\left\{\,\bfx\in\bfA^q:\text{for every
$f\in\fraka$, $f(\bfx)=0$}\,\right\}.
\]
We call $\calV_k(\fraka)$ the \emph{set of $\Omega$-points of the affine
  $k$-variety determined by $\fraka$.} With a slight abuse of language, we
call $\calV_k(\fraka)$ the \emph{affine $k$-variety determined by $\fraka$.}
Similarly, given by any ideal $\fraka\subset\bar k[X_1,\dotsc,X_q]$,
defined by $\calV_{\bar k}(\fraka)$ by
\[
\calV_{\bar k}(\fraka)\coloneqq\left\{\,\bfx\in \bfA^q:
\text{for every $f\in\fraka$, $f(\bfx)=0$}\,\right\}.
\]
We call $\calV_{\bar k}(\fraka)$ the \emph{set of $\Omega$-points of the
  (geometric) affine $\bar k$-variety determined by $\fraka$,} or for
short, the \emph{(geometric) affine variety determined by $\fraka$.}
\end{definition}

To ease the notation, we usually drop the subscript $k$ or $\bar k$ and
simply write $\calV$.

If $A$ is a (commutative) ring (with unit $1$), recall that the
\emph{radical}, $\sqrt{\frakb}$, of an ideal, $\fraka\subset A$, is defined
by
\[
\sqrt{\fraka}\coloneqq\left\{\,a\in A:\text{there exists $n\geq 1$,
    $a^n\in\fraka$}\,\right\}.
\]
A \emph{radical ideal} is an ideal, $\fraka$, such that
$\fraka=\sqrt{\fraka}$.

The following properties are easily verified. We state them for $\calV_k$, but
they also hold for $\calV_{\bar k}$:
\begin{align*}
&\text{$\calV(0)=\bfA^n$, $\calV(A)=\emptyset$}\\
&\calV(\fraka\cap\frakb)=\calV(\fraka\frakb)=\calV(\fraka)\cup\calV(\frakb)\\
&\text{$\fraka\subset\frakb$ implies that $\calV(\frakb)\subset\calV(\frakb)$}\\
&\calV\left({\textstyle\sum_\alpha\fraka_\alpha}\right)
={\textstyle \bigcap_\alpha\calV(\fraka_\alpha)}\\
&\calV(\sqrt{\fraka})=\calV(\fraka)
\end{align*}
From the relations above, it follows that the sets $\calV(\fraka)$ can be taken
as closed subsets of $\bfA^q$, and we obtain a topology on $\bfA^q$. This
is the \emph{$k$-topology on $\bfA^q$.} If we consider ideals in $\bar
k[X_1,\dotsc,X_q]$ (i.e., sets of the form $\calV_{\bar k}(\fraka)$), we obtain
the \emph{Zariski topology on $\bfA^q$.}

The Zariski topology is not necessarily Hausdorff (except when $\calV(\fraka)$
consits of a finite set of points.)

Let us see that $\bfA^q$ is not Hausdorff in the Zariski topology. Let
$P,Q\in\bfA^q$, with $P\neq Q$. The line $\overleftrightarrow{PQ}$ is
isomorphic to $\bfA^1$. Thus, it is enough to show that $\bfA^1$ is not
Hausdorff. Consider any ideal $\fraka\subset\bar k[X]$. Then, $\fraka$ is a
principal ideal, and thus
\[
\fraka=(f)
\]
for some polynomial $f$, which shows that $\calV(\fraka)=\calV(f)$ is a finite
set. As a consequence, the closed sets of $\bfA^1$ (other than $\bfA^1$)
are finite. Then, the union of two closed sets (distinct from $\bfA^1$) is
also finite, and thus distinct from $\bfA^1$.

The topology on $\bfA^q$ is not the product topology on
$\prod_{i=1}^q\bfA^1$.

For example, when $n=2$, the closed set in $\bfA^1\times\bfA^1$ are those
sets consisting of finitely many horizontal and vertical lines, and
intersections of such sets. However
\[
X^2+Y^2-1=0
\]
defines a closed set in $\bfA^2$ not of the previous form.

To go backwards from subsets of $\bfA^q$ to ideals, we make the following
definition.
\begin{definition}
 Given any subset $S\subset\bfA^q$, define $\calI_k(S)$ and $\calI_{\bar
   k}(S)$ by
\[
\calI_k(S)\coloneqq\left\{\,f\in k[X_1,\dotsc,X_q]:
\text{for every $s\in S$, $f(s)=0$}\,\right\}
\]
and
\[
\calI_{\bar k}(S)\coloneqq\left\{\,f\in\bar k[X_1,\dotsc,X_q]:
\text{for every $s\in S$, $f(s)=0$}\,\right\}
\]
\end{definition}

%%% Local Variables:
%%% mode: latex
%%% TeX-master: "../MA598-Exercises"
%%% End:
