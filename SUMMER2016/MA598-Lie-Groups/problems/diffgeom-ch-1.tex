\chapter{Differential Geometry Exercises}
\section{The Matrix Exponential; Some Matrix Lie Groups}
\subsection{The Exponential Map}
\subsection{The Lie Groups $\GL(n,\bbR)$, $\SL(n,\bbR)$, $\oalg(n)$,
  $\SO(n)$, the Lie Algebras $\glalg(n,\bbR)$, $\slalg(n,\bbR)$,
  $\oalg(n)$, $\soalg(n)$, and the Exponential Map}
The notation $\slalg(n,\bbR)$ and $\soalg(n)$ is rather strange and
deserves some explanation. The groups $\GL(n,\bbR)$, $\SL(n,\bbR)$,
$\oalg(n)$, and $\SO(n)$ are more than just groups. They are also
topological groups, which means that they are topological spaces (viewed as
subspaces of $\bbR^{n^2}$) and that the multiplication and the inverse
operations are smooth. In fact, they are smooth real manifolds. Such
objects are called \emph{Lie groups}. The real vector spaces $\slalg(n)$
and $\soalg(n)$ are what are called \emph{Lie algebras}. However, we have
not defined the algebra structure on $\slalg(n,\bbR)$ and $\soalg(n)$
yet. The algebra structure is called the \emph{Lie bracket}, which is
defined as
\[
[A,B]=AB-BA.
\]

Lie algebras are associated with Lie groups. What is going on is that the
Lie algebra of a Lie group is its tangent space at the identity, i.e., the
space of all tangent vectors at the identity. In some sense, the Lie
algebra achieves a linearization of the Lie group. The exponential map is a
map from the Lie algebra to the Lie group, for example,
\[
\exp\colon\soalg(n)\longrightarrow\SO(n)
\]
and
\[
\exp\colon\slalg(n,\bbR)\longrightarrow\SL(n,\bbR).
\]
The exponential map often allows a parametrization of the Lie group
elements by simpler objects, the Lie algebra elements.

One might ask, ``What happened to the Lie algebras $\glalg(n,\bbR)$ and
$\oalg(n)$ associated with the Lie group $\GL(n,\bbR)$ and $\oalg(n)$?'' We
will see later that $\glalg(n,\bbR)$ is the set of all real $n\times n$
matrices and that $\oalg(n)=\soalg(n)$.

The properties of the exponential map play an important role in the study
of Lie groups. For example, it is clear that the map
\[
\exp\colon\glalg(n,\bbR)\longrightarrow\GL(n,\bbR)
\]
is well-defined, but since $\det(e^A)=e^{\tr A}$, every matrix of the form
$e^A$ has a positive determinant and $\exp$ is not surjective.

%%% Local Variables:
%%% mode: latex
%%% TeX-master: "../MA598-Exercises"
%%% End:
