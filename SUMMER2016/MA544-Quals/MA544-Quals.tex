\def\auth{Carlos Salinas}
\def\tight{MA544: Qual Preparation}
\def\short{MA544 Qual Prep}
\def\class{MA544}
\def\subject{measure theory}
\def\email{salinac@purdue.edu}

\documentclass[10pt,showtrims,twoside]{memoir}
% \usepackage{geometry}
\usepackage[dvipsnames]{xcolor}
\usepackage[
    breaklinks,
    colorlinks=true,
    linkcolor=black,
    citecolor=black,
    filecolor=black,
    menucolor=black,
    runcolor=black,
    urlcolor=black,
    pdftitle={\short},
    pdfauthor={\auth},
    pdfkeywords={\subject},
    pdfsubject={\class},
    pageanchor={false}
    ]{hyperref}
% \usepackage{natbib}

% TOC depth
\setsecnumdepth{subsection}

%% Math
\usepackage{amsmath}
\usepackage{amsfonts}
\usepackage{amssymb}
\usepackage{amsthm}
\usepackage{mathtools}
% \usepackage{lmodern}
\usepackage{eucal}

%% Language
\usepackage[LAE,LFE,T2A,T1]{fontenc}
\usepackage[utf8]{inputenc}
\usepackage[farsi,french,german,spanish,russian,english]{babel}
\babeltags{fr=french,
           de=german,
           en=english,
           es=spanish,
           pa=farsi,
           ru=russian
           }
\def\spanishoptions{mexico}

\selectlanguage{english}

\newcommand{\textfa}[1]{\beginR\textpa{#1}\endR}

\usepackage{CJKutf8}
\newcommand{\textkr}[1]{\begin{CJK}{UTF8}{mj}#1\end{CJK}}
\newcommand{\textjp}[1]{\begin{CJK}{UTF8}{min}#1\end{CJK}}
\newcommand{\textzh}[1]{\begin{CJK}{UTF8}{bsmi}#1\end{CJK}}

%% Misc
\usepackage{graphicx}
\graphicspath{{figures/}}

\usepackage[babel=true]{microtype}
\usepackage{lineno}
\usepackage{multicol}
\usepackage[inline]{enumitem}
\usepackage{listings}
\usepackage{mleftright}
\mleftright
\usepackage{carlos-variables}

%% Unicode math
% \usepackage{unicode-math}

% \setmainfont[Ligatures=TeX]{Latin Modern Roman}
% \setsansfont{Latin Modern Sans}
% \setsansfont{Latin Modern Mono}
% \setmathfont{Latin Modern Math}
% \setmathfont[BoldFont={latinmodern-math.otf}]{latinmodern-math.otf}

% \setmainfont[Ligatures=TeX]{Minion Pro}
% \setmathfont{Minion Math}

%% Theorems and definitions
\theoremstyle{plain}
\newtheorem{theorem}{Theorem}
\newtheorem{proposition}[theorem]{Proposition}
\newtheorem{corollary}[theorem]{Corollary}
\newtheorem{claim}[theorem]{Claim}
\newtheorem{lemma}[theorem]{Lemma}
\newtheorem{axiom}[theorem]{Axiom}

\newtheorem*{corollary*}{Corollary}
\newtheorem*{claim*}{Claim}
\newtheorem*{lemma*}{Lemma}
\newtheorem*{proposition*}{Proposition}
\newtheorem*{theorem*}{Theorem}

\theoremstyle{definition}
\newtheorem{definition}{Definition}
\newtheorem{example}{Examples}
\newtheorem{examples}[example]{Example}
\newtheorem{exercise}{Exercise}[subsection]
\newtheorem{problem}[exercise]{Problem}

\newtheorem*{example*}{Example}
\newtheorem*{exercise*}{Exercise}
\newtheorem*{problem*}{Problem}

\begin{document}

%% Footnote style
\renewcommand*{\thefootnote}{\fnsymbol{footnote}}

%% Counters
\counterwithout{exercise}{chapter}
\numberwithin{equation}{subsection}
\counterwithout{equation}{chapter}

\thispagestyle{empty}
\author{\href{mailto:\email}{\auth}}
\title{\tight}
\date{\today}

\frontmatter
\maketitle
\tableofcontents

%% Danielli homework
\mainmatter
% MA 54400 Homework assignment - Spring 2016
% Problems in parentheses are relative to material not yet covered in class.
% #1 - Due Jan. 20 Read Sections 2.1, 2.3; Chapter 2: # 1, 2, 3, 11, 13.
% #2 - Due Jan. 25 Read Section 3.1; Problem #1: Show that the boundary of
% any interval has outer measure zero; Problem #2: Show that a set
% consisting of a single point has outer measure zero.
% #3 - Due Feb. 1 Read Sections 3.2-4.  Chapter 3: # 5, 7, 8, 9, 10.
% #4 - Due Feb. 8 Read Section 3.5. Chapter 3: # 12, 13, 15.
% #5 - Due Feb. 15 Read Sections 3.6, 4.1 Chapter 3: # 14, 16, 18; Chapter
% 4: # 1, 2.
% Practice Problems for Midterm 1 - Solutions
% #6 - Due Feb. 22 Read Sections 4.2-3. Chapter 4: # 4, 7, 8.
% #7 - Due Mar. 2 Read Sections 5.1-2. Chapter 4: #9, 11, 15, 18 (ignore
% the second part concerning convergence in measure); Chapter 5: # 1, 3.
% #8 - Due Mar. 7 Read Section 5.3, 6.1. Chapter 5: # 2, 4, 6, 7,
% 21. Chapter 6: # 10, 11.
% #9 - Due Mar. 21 Read Section 6.2-3. Chapter 5: # 20; 6: # 1, 3, 4, 6, 7,
% 9.
% Practice Problems for Midterm 2 (You can ignore #4) – Solutions
% #10 – Due Apr. 11 Read Section 7.1-6. Chapter 7: # 1, 2, 12, 13, 6, 8, 9
% #11 – Due Apr. 18 Read Section 5.4, Sections 8.1-4. Chapter 7: # 11, 15;
% Chapter 5: # 8, 11, 12, 17; Prove Theorem 8.3.
% #12 – Due Apr. 25 Chapter 8: # 2, 4, 5, 6, 9, 11, 16, 18.
% Final Exam on Tuesday, May 3, 8-10 am, UNIV 117
% Practice Problems for the Final Exam - Solutions


% Interesting symbols
% $\mbfalpha,\mbfvartheta,\mbfvarphi,\mbfvarkappa,\mbfvarepsilon,\mbfvarpi$
\chapter{MA 544 Spring 2016}
\thispagestyle{empty} This is material from the course MA 544 as it was
taught in the spring of 2016.  \bigskip
\section{Homework}
These exercises were assigned from Wheeden and Zygmund's \emph{Measure and
  Integral}, therefore, most of the theorems I reference will be from
\cite{wheeden-zygmund}. Other resources include \cite{folland} and
\cite{royden}. For more elementary results, I cite \cite{rudin-1}. Unless
otherwise stated, whenever we quote a result, e.g., Theorem 1.1, it is
understood to come from Wheeden and Zygmund's \emph{Measure and Integral}.

\bigskip

Throughout these notes

\begin{tabular}{cl}
  $\bbR$ & is the set of real numbers\\
  $\bbR^+$ & is the set of positive real numbers, that is, $x\in\bbR$ with
             $x\geq 0$\\
  $\bbC$ & is the set of complex numbers\\
  $\bbQ$ & is the set of rational numbers\\
  $\bbZ$ & is the set of the integers\\
  $\bbZ^+$ & is the set of positive integers, that is, $x\in\bbZ$ with
             $x\geq 0$\\
  $\bbN$ & is the set of the natural numbers $1,2,\ldots$\\
  $A\smallsetminus B$ & is the set difference of $A$ and $B$, that is, the
                        complement of $A\cap B$ in $A$\\
  $m^*(E)$ & the outer measure of $E$\\
  $m_*(E)$ & the inner measure of $E$\\
  $m(E)$ & the Lebesgue measure of $E$\\
  $\|\blank\|$ & the standard Euclidean norm on $\bbR^n$\\
  $f\asymp g$ & means $f$ is asymptotically equivalent to $g$, that is,
                $\lim_{x\to\infty} g(x)/f(x)=1$\\
\end{tabular}

\newpage

\subsection{Homework 1}
\begin{problem}[Wheeden \& Zygmund Ch.\@ 2, Ex.\@ 1]
  Let $f(x)=x\sin(1/x)$ for $0<x\leq 1$ and $f(0)=0$. Show that $f$ is
  bounded and continuous on $[0,1]$, but that $V[f;0,1]=+\infty$.
\end{problem}
\begin{solution}
  Let $f$ equal $x\sin(1/x)$. We will show that $f$ is bounded and
  continuous on $[0,1]$, but that it is not of bounded variation on
  $[0,1]$.

  First we will show that $f$ is bounded. Note that both $|x|$ and
  $|\sin(1/x)|$ are bounded by $1$ on the interval $[0,1]$. Since
  $|f|=|x||\sin(1/x)|$, it follows that $|f|\leq 1$ on $[0,1]$. Thus, $f$
  is bounded on $[0,1]$.

  Next we show that $f$ is continuous. It is easy to show that $f$ is
  continuous on the subinterval $(0,1]$ since both $|x|$ and $\sin(1/x)$
  are continuous on that interval and we know that the product of
  continuous functions is continuous. To see that $f$ is continuous at $0$
  we must show that $f(x^+)=f(0)$; that is, the limit of $f$ as $x$
  approaches $0$ from the right is $f(0)$ which by definition is $0$. To
  this end, it suffices to take a (monotonically decreasing) sequence
  $x_n\searrow 0$ and show that the limit of the sequence
  ${\{f(x_n)\}}_{n=1}^\infty$ is $0$. Let $\varepsilon>0$ be given then,
  since $x_n$ converges to $0$ there exists an index $N$ such that
  $|0-x_n|<\varepsilon$ whenever $n\geq N$. Since $|f(x_n)|\leq |x_n|$ on
  $[0,1]$, the following inequality holds
  \begin{align*}
    |0-f(x_n)|
    &=\left|0-x_n\sin(1/x_n)\right|\\
    &\leq |x_n|\\
    &<\varepsilon.
  \end{align*}
  Thus, $f$ is continuous at $0$ and it converges to $0$.

  Despite the nice properties that $f$ seemingly possesses, $f$ is not
  b.v.\@ on $[0,1]$. To show that $f$ is not b.v.\@ on $[0,1]$ we must show
  that for any positive real number $M$ there exists some partition
  $\Gamma=\{\,x_0<x_1<\cdots<x_n\,\}$ of $[0,1]$ such that the sum
  associated to $\Gamma$
  \[
    \sum_{i=1}^n|f(x_i)-f(x_{i-1})|>M.
  \]
  Let $N$ be the smallest integer greater than $M$ and let $n$ be the
  smallest integer greater than or equal to $N/2$. Then the partition
  $\Gamma=\{\,x_0=1<x_1<\cdots<x_{n+1}=1\,\}$ where
  $x_i=2/((3+(n-i))\pi)$\marginremark{I took care to choose $x_i$ such that
    $x_{i}<x_{i+1}$ in the partition.} for $1\leq i\leq N$. Then we have
  the inequality
  \begin{align*}
    S_\Gamma
    &=\sum_{i=1}^{n+1}|f(x_i)-f(x_{i-1})|\\
    &=\sum_{i=2}^n
      |f(x_i)-f(x_{i-1})|+|f(x_{n+1})-f(x_n)|+|f(x_0)-f(x_1)|\\
    &=N+|f(x_{n+1})-f(x_n)|+|f(x_0)-f(x_1)|\\
    &>M.
  \end{align*}
  Thus, $f$ is not b.v.\@ on $[0,1]$.
\end{solution}

\begin{problem}[Wheeden \& Zygmund Ch.\@ 2, Ex.\@ 2]
  Prove theorem (2.1).
\end{problem}
\begin{solution}
  Recall the statement of Theorem 2.1:
  \begin{quote}
    \begin{enumerate}[label=(\alph*),noitemsep]
    \item If $f$ is of bounded variation on $[a,b]$, then $f$ is bounded on
      $[a,b]$.
    \item Let $f$ and $g$ be of bounded variation on $[a,b]$. Then $cf$
      (for any real constant $c$), $f+g$, and $fg$ are of bounded variation
      on $[a,b]$. Moreover, $f/g$ is of bounded variation on $[a,b]$ if
      there exists an $\varepsilon>0$ such that $|g(x)|\geq\varepsilon$ for
      $x\in[a,b]$.
    \end{enumerate}
  \end{quote}
  \noindent%
  We shall prove these in alphabetical order:

  For part (a) we shall proceed by contradiction. First, without loss of
  generality, we may assume that $f(a)=0$ since the function the variation
  of $g(x)=f(x)-f(a)$ is equal to the variation of $f$ and $g(a)=0$.
  Suppose that $f$ is b.v.\@ on $[a,b]$ with variation $V=V[f;a,b]$, but
  that $f$ is unbounded on $[a,b]$; that is, given a positive real number
  $M$ there exists a point $x$ in $[a,b]$ such that $|f(x)|>M$. In
  particular, there exists $x\in[a,b]$ such that $|f(x)|>V$. Hence, for any
  $x\in[a,b]$ by the triangle inequality we have
  \begin{align*}
    V&<|f(x)|\\
     &=|f(x)-f(a)+f(a)|\\
     &\leq|f(x)-f(a)|+|f(a)|\\
     &\leq V.
  \end{align*}
  This is a contradiction. Therefore, it must be the case that if $f$ is
  b.v.\@ on $[a,b]$ then $f$ is bounded on $[a,b]$.

  We break part (b) into three sections. Suppose $f$ and $g$ are b.v.\@ on
  $[a,b]$ with variation $V$ and $V'$, respectively.  We will show that (i)
  $cf$; (ii) $f+g$; and (iii) $fg$ are b.v.\@ on $[a,b]$. Moreover, we show
  that (iv) $f/g$ is b.v.\@ on $[a,b]$ if there exists $\varepsilon>0$ such
  that $|g(x)|\geq\varepsilon$ for all $x\in[a,b]$.

  For part (i) above let $c$ be a real number. Given a partition
  $\Gamma=\{\,x_0<x_1<\cdots<x_n\,\}$ of $[a,b]$, we have
  \begin{align*}
    S_\Gamma
    &=\sum_{i=1}^n |cf(x_i)-cf(x_{i-1})|\\
    &=\sum_{i=1}^n |c||f(x_i)-cf(x_{i-1})|\\
    &=|c|\sum_{i=1}^n|f(x_i)-f(x_{i-1})|\\
    &\leq |c| V
  \end{align*}
  since $V$ is the supremum of the sums of the form
  $\sum_{i=1}^m|f(x_i)-f(x_{i-1})|$ over all partitions of $[a,b]$. Thus,
  $V[cf;a,b]\leq |c|V$ so $cf$ is b.v.\@ on $[a,b]$.

  For part (ii) given a partition $\Gamma=\{\,x_0<x_1<\cdots<x_n\,\}$ of
  the interval $[a,b]$, by the triangle inequality we have
  \begin{align*}
    S_\Gamma
    &=\sum_{i=1}^n\left|(f(x_i)+g(x_i))-(f(x_{i-1})+g(x_{i-1}))\right|\\
    &=\sum_{i=1}^n\left|(f(x_i)-f(x_{i-1}))+(g(x_i)-g(x_{i-1}))\right|\\
    &\leq\sum_{i=1}^n|f(x_i)-f(x_{i-1})|+\sum_{i=1}^n|g(x_i)-g(x_{i-1})|\\
    &\leq V+V'.
  \end{align*}
  Thus, $f+g$ is b.v.\@ on $[a,b]$

  For part (iii) since $f$ and $g$ are b.v.\@ on $[a,b]$ by part (a) $f$
  and $g$ are bounded on $[a,b]$ by, say, $M$ and $N$, respectively. Now,
  given a partition $\Gamma=\{\,x_0<x_1<\cdots<x_n\,\}$ of $[a,b]$, by the
  triangle inequality we have
  \begin{align*}
    S_{\Gamma}
    &=\sum_{i=1}^n\left|f(x_i)g(x_i)-f(x_{i-1})g(x_{i-1})\right|\\
    &\begin{aligned} =\sum_{i=1}^n
      \left|f(x_i)g(x_i)\right.{}&{}\left.-f(x_{i-1})g(x_{i-1})\right.\\
      &\left.+f(x_i)g(x_{i-1})-f(x_i)g(x_{i-1})\right|
    \end{aligned}\\
    &\begin{aligned} =\sum_{i=1}^n
      \left|(f(x_i)g(x_i)\right.{}&{}\left.-f(x_i)g(x_{i-1}))\right.\\
      &\left.-(f(x_{i-1})g(x_{i-1})-f(x_i)g(x_{i-1}))\right|
    \end{aligned}\\
    &\begin{aligned}
      \leq\sum_{i=1}^n|f(x_i)g(x_i){}&{}-f(x_i)g(x_{i-1})|\\
      &+\sum_{i=1}^n|f(x_{i-1})g(x_{i-1})-f(x_i)g(x_{i-1})|
    \end{aligned}\\
    &=\sum_{i=1}^n|f(x_i)||g(x_i)-g(x_{i-1})|+\sum_{i=1}^n|g(x_{i-1})||f(x_i)-f(x_{i-1})|\\
    &=\sum_{i=1}^n
      M|g(x_i)-g(x_{i-1})|+\sum_{i=1}^n N|f(x_i)-f(x_{i-1})|\\
    &\leq MV'+NV.
  \end{align*}
  Thus, $fg$ is b.v.\@ on $[a,b]$.

  Finally, for part (iv) suppose there exists $\varepsilon>0$ such that
  $|g(x)|\geq\varepsilon$ for all $x\in[a,b]$. Then, given a partition
  $\Gamma=\{\,x_0<x_1<\cdots<x_n\,\}$ of $[a,b]$, largely by the triangle
  inequality, we have
  \begin{align*}
    S_\Gamma
    &=\sum_{i=1}^n |f(x_i)/g(x_i)-f(x_{i-1})/g(x_{i-1})|\\
    &=\sum_{i=1}^n\left|\frac{f(x_i)g(x_{i-1})-
      f(x_{i-1})g(x_i)}{g(x_i)g(x_{i-1})}\right|\\
    &\leq\frac{1}{\varepsilon^2}\sum_{i=1}^n|f(x_i)g(x_{i-1})-f(x_{i-1})g(x_i)|\\
    &
      \begin{aligned}
        =\frac{1}{\varepsilon^2}\sum_{i=1}^n |f(x_i)g(x_{i-1}){}&{}-f(x_{i-1})g(x_{i-1})\\
        &-(f(x_{i-1})g(x_i)-f(x_{i-1})g(x_{i-1}))|
      \end{aligned}\\
    &\leq
      \frac{1}{\varepsilon^2}\sum_{i=1}^n|g(x_{i-1})||f(x_i)-f(x_{i-1})|
      +\frac{1}{\varepsilon^2}\sum_{i=1}^n|f(x_{i-1})||g(x_i)-g(x_{i-1})|\\
    &=\frac{1}{\varepsilon^2}\sum_{i=1}^nM_g|f(x_i)-f(x_{i})|
      +\frac{1}{\varepsilon^2}\sum_{i=1}^nM_f|g(x_i)-g(x_i)|\\
    &=\frac{1}{\varepsilon^2}M_g\sum_{i=1}^n|f(x_i)-f(x_{i})|
      +\frac{1}{\varepsilon^2}M_f\sum_{i=1}^n|g(x_i)-g(x_i)|\\
    &\leq\frac{1}{\varepsilon^2}(NV+MV')
  \end{align*}
  where, as above, $f$ is bounded by $M$ and $g$ is bounded by $N$. Thus,
  $f/g$ is b.v.\@ on $[a,b]$.
  \\\\
  This concludes the proof of Theorem 2.1.
\end{solution}

\begin{problem}[Wheeden \& Zygmund Ch.\@ 2, Ex.\@ 3]
  If $[a',b']$ is a subinterval of $[a,b]$ show that $P[a',b']\leq P[a,b]$
  and $N[a',b']\leq N[a,b]$.
\end{problem}
\begin{solution}
  We will prove this by digging in to the definition of $N$ and $P$.
  Recall that given a partition $\Gamma=\{\,x_0<x_1<\cdots<x_n\,\}$ of the
  interval $[a,b]$, $P$ and $N$ are defined to be the supremum over the sum
  of the positive and, respectively, the sum negative terms of $S_\Gamma$;
  that is, $P$ and $N$ are the supremum over every partition $\Gamma$ of
  $[a,b]$ of
  \[
    P_\Gamma=\sum_{i=1}^n\left[f(x_i)-f(x_{i-1})\right]^+
    \quad\text{and}\quad
    N_\Gamma=\sum_{i=1}^n\left[f(x_i)-f(x_{i-1})\right]^-.
  \]

  Let $f\colon[a,b]\to\bbR$ be a function of bounded variation on $[a,b]$
  and let $[a',b']$ be a subinterval of $[a,b]$. Without loss of
  generality, we may assume that $[a',b']$ is strictly contained in
  $[a,b]$; that is, $a'\neq a$ and $b'\neq b$. We aim to show that
  $P[a',b']\leq P[a,b]$ and $N[a',b']\leq N[a,b]$. Since the argument for
  $N$ is similar to that of $P$, we will omit it here for the sake of
  brevity. Now, consider the closure of the complement of $[a',b']$ in
  $[a,b]$, $\overline{[a,b]\smallsetminus [a',b']}=[a,a']\cup[b',b]$. Since
  $[a,a']$, $[a',b']$ and $[b',b]$ are close intervals we may take
  partitions\marginremark{On second thought, this is not the correct way to
    go, but the argument is almost correct. You need to start with a
    partition $\Gamma$ of $[a,b]$ and refined it so that some subset
    $\Gamma'$ of $\Gamma$ is a partition of $[a',b']$ and argue that
    because $P[a,b]$ is a supremum and for any partition $\Gamma$,
    $P_{\Gamma'}$ is in the sum $P_\Gamma$, it follows that
    $P[a',b']\leq P[a,b]$.}
  \begin{align*}
    \Gamma_a&=\{\,x_0<x_1\cdots<x_\ell\,\},\\
    \Gamma_{ab}&=\{\,x_\ell<x_{\ell+1}<\cdots<x_m\,\}
    \shortintertext{and}
    \Gamma_b&=\{\,x_m<x_{m+1}<\cdots<x_n\,\}
  \end{align*}

  of $[a,a']$, $[a',b']$ and $[b',b]$, respectively and extend this to a
  partition
  \[
    \Gamma=\{\,x_0<x_1<\cdots<x_\ell<x_{\ell+1}\cdots<x_m<x_{m+1}\cdots<x_n\,\}
  \]
  of $[a,b]$. Then, by the definition of $N$ we have the string of
  inequalities
  \begin{align*}
    P_\Gamma&=\sum_{i=1}^n[f(x_i)-f(x_{i-1})]^+\\
            &\begin{aligned}
              =\sum_{i=1}^\ell&[f(x_i)-f(x_{i-1})]^+\\
              &+\sum_{i=\ell+1}^m[f(x_i)-f(x_{i-1})]^+\\
              &+\sum_{i=m+1}^n[f(x_i)-f(x_{i-1})]^+
            \end{aligned}\\
            &=P_{\Gamma_{ab}}+P_{\Gamma_a}+P_{\Gamma_b}\\
            &\leq P[a,b].
  \end{align*}
  Taking the supremum on the left, we have
  \[
    P[a,a']+P[a',b']+P[b',a']\leq P[a,b].
  \]
  Since $P$ is strictly positive, it must be the case that
  $P[a',b']\leq P[a,b]$.
\end{solution}

\begin{problem}[Wheeden \& Zygmund Ch.\@ 2, Ex.\@ 11]
  Show that $\int_a^bf\diff\varphi$ exists if and only if given
  $\varepsilon>0$ there exists $\delta>0$ such that
  $\left|R_\Gamma-R_{\Gamma'}\right|<\varepsilon$ if
  $|\Gamma|,|\Gamma'|<\delta$.
\end{problem}
\begin{solution}
  One direction is straightforward. Namely $\impliedby$: suppose that given
  $\varepsilon>0$ there exists $\delta>0$ such that
  $|R_\Gamma-R_{\Gamma'}|<\varepsilon$ whenever $|\Gamma|$ and $|\Gamma'|$
  are less than $\delta$. Let ${\{\Gamma_n\}}_{n=1}^\infty$ be a decreasing
  sequence of partitions (by which we mean $\Gamma_n\subset\Gamma_{n+1}$)
  of $[a,b]$ such that $|\Gamma_n|\to 0$. Then, by convergence, there
  exists $N\in\bbN$ such that $n\geq N$ implies $|\Gamma_n|<\delta$. Then,
  for $n,m\geq N$, we have
  \begin{align*}
    |R_{\Gamma_n}-R_{\Gamma_m}|<\varepsilon.
  \end{align*}
  Thus, by the Cauchy criterion for convergence, the sequence
  ${\{R_{\Gamma_n}\}}_{n=0}^\infty$ converges and its limit is by
  definition the Riemann--Stieltjes integral $\int_a^b f\diff\varphi$.

  On the other hand $\implies$: suppose that $I=\int_a^b f\diff\varphi$
  exists. Then given $\varepsilon>0$ there exists $\delta>0$ such that
  $|I-R_\Gamma|<\varepsilon/2$ whenever $|\Gamma|<\delta$. Let $\Gamma$ and
  $\Gamma'$ be two partitions of $[a,b]$ with norm
  $|\Gamma|,|\Gamma'|<\delta$. Then we have
  \begin{align*}
    |R_\Gamma-R_{\Gamma'}|
    &=|R_\Gamma-I-(R_{\Gamma'}-I)|\\
    &\leq |R_\Gamma-I|+|R_{\Gamma'}-I|\\
    &<\frac{\varepsilon}{2}+\frac{\varepsilon}{2}\\
    &=\varepsilon.
  \end{align*}
  Thus, $I$ satisfies the Cauchy condition.
\end{solution}

\begin{problem}[Wheeden \& Zygmund Ch.\@ 2, Ex.\@ 13]
  Prove theorem (2.16).
\end{problem}
\begin{solution}
  Recall the statement of Theorem 2.16:
  \begin{quote}
    \begin{enumerate}[label=(\roman*),noitemsep]
    \item If $\int_a^b f\diff\varphi$ exists, then so do
      $\int_a^bcf\diff\varphi$ and $\int_a^b f\diff(c\varphi)$ for any
      constant $c$, and
      \[
        \int_a^bcf\diff\varphi=\int_a^bf\diff(c\varphi)=c\int_a^bf\diff\varphi.
      \]
    \item If $\int_a^b f_1\diff\varphi$ and $\int_a^bf_2\diff\varphi$ both
      exist, so does $\int_a^b\left(f_1+f_2\right)\diff\varphi$, and
      \[
        \int_a^b\left(f_1+f_2\right)\diff\varphi=\int_a^bf_1\diff\varphi+\int_a^bf_2\diff\varphi.
      \]
    \item If $\int_a^bf\diff\varphi_1$ and $\int_a^bf\diff\varphi_2$ both
      exist, so does $\int_a^bf\diff\left(\varphi_1+\varphi_2\right)$, and
      \[
        \int_a^bf\diff\left(\varphi_1+\varphi_2\right)=\int_a^bf\diff\varphi_1+\int_a^bf\diff\varphi_2.
      \]
    \end{enumerate}
  \end{quote}
  \noindent%
  We prove this in (Roman) numerical order.


  For (i) suppose that $I=\int_a^bf\diff\varphi$ exists. Then, given
  $\varepsilon>0$, there exists $\delta>0$ such that
  $|I-R_\Gamma|<\varepsilon/|c|$ whenever $\Gamma$ is a partition of
  $[a,b]$ with $|\Gamma|<\delta$. We claim that
  $\int_a^b cf\diff\varphi=|c|I$. Let $\Gamma=\{,x_0<x_1<\cdots<x_n\,\}$ be
  a partition $[a,b]$ with $|\Gamma|<\delta$. Then the Riemann sums
  $\R_\Gamma'$ of the pair $(cf,\varphi)$ associated to the partition
  $\Gamma$ give us the chain of inequalities
  \begin{align*}
    ||c|I-R_\Gamma'|
    &=\left| |c|I-
      \sum_{i=1}^n cf(\xi_i)[\varphi(x_i)-\varphi(x_{i-1})]
      \right|\\
    &=|c|\left|
      \sum_{i=1}^n cf(\xi_i)[\varphi(x_i)-\varphi(x_{i-1})]
      \right|\\
    &=|c||I-R_\Gamma'|\\
    &<|c|\left[\frac{\varepsilon}{|c|}\right]\\
    &=\varepsilon.
  \end{align*}
\end{solution}

%%% Local Variables:
%%% mode: latex
%%% TeX-master: "../MA544-Quals"
%%% End:

\subsection{Homework 2}
\begin{problem}
  Show that the boundary of any interval has outer measure zero.
\end{problem}
\begin{solution}
  Let $I\coloneq\prod_{i=1}^n I_i$ be a closed interval in $\bbR^n$ and let
  $J$ be the boundary of $I$. We must show that given $\varepsilon>0$ there
  exists a countable collection of intervals $\{I_n\}_{n\in J}$ covering
  $J$ such that
  \[
    \sum_{n\in J}\vol(I_n)<\varepsilon.
  \]
  First, note that we can write $J$ as the union $\bigcup_{i=1}^n J_i$
  where
  \[
    J_i\coloneq%
    % \underbrace{
      [a_1,b_1]\times\cdots\times\{a_i\}\times\cdots\times[a_n,b_n]%
    % }_{J_i^1}
    \cup
    [a_1,b_1]\times\cdots\times\{b_i\}\times\cdots\times[a_n,b_n].
  \]
  Since the countable union of null sets has measure zero, it suffices to
  show that the set
  \[
    [a_1,b_1]\times\cdots\times[a_{n-1},b_{n-1}]\times\{a_n\}%
  \]
  has measure zero. Consider the collection $\{I_\varepsilon\}$ consisting
  of the single interval
  \[
    I_\varepsilon\coloneq [a_1,b_1]\times\cdots\times[a_{n-1},b_{n-1}]
    \times\left[a_n-\frac{\varepsilon}{2\prod_{i=1}^{n-1}(b_i-a_i)},
      a_n+\frac{\varepsilon}{2\prod_{i=1}^{n-1}(b_i-a_i)}\right].
  \]
  It is clear that $I_\varepsilon\supset J$. Now, computing the volume of
  this interval, we have
  \begin{align*}
    \vol(I_\varepsilon)%
    &=\prod_{i=1}^{n-1}(b_i-a_i)%
    \left[%
    a_n+\frac{\varepsilon}{2\prod_{i=1}^{n-1}(b_i-a_i)}%
    -\left(%
    a_n -\frac{\varepsilon}{2\prod_{i=1}^{n-1}(b_i-a_i)} \right)
    \right]\\
    &=\left[\prod_{i=1}^{n-1}(b_i-a_i)\right]%
      \frac{\varepsilon}{\prod_{i=1}^{n-1}(b_i-a_i)}\\
    &=\varepsilon.
  \end{align*}
  Thus, $J$ has measure zero.
\end{solution}

\begin{problem}
  Show that a set consisting of a single point has outer measure zero.
\end{problem}
\begin{solution}
  Let $\{a\}$ be the set consisting of a single point $a\in\bbR$. Then we
  must show that given $\varepsilon>0$ there exists a countable collection
  of intervals $\{I_n\}$ such that
  \[
    \sum_{n\in J} m(I_n)<\varepsilon.
  \]
  Consider the collection $\{I_\varepsilon\}$ consisting of the single
  interval
  \[
    I_\varepsilon\coloneq
    \left[a-\frac{\varepsilon}{2},a+\frac{\varepsilon}{2}\right].
  \]
  It is clear that $\{a\}\subset I_\varepsilon$. Moreover,
  \begin{align*}
    \vol(I_\varepsilon)
    &=a+\frac{\varepsilon}{2}-\left(a-\frac{1}{\varepsilon}\right)\\
    &=\varepsilon.
  \end{align*}
  Thus, $\{a\}$ has measure zero.
\end{solution}

%%% Local Variables:
%%% mode: latex
%%% TeX-master: "../MA544-Quals"
%%% End:

\subsection{Homework 3}
\begin{problem}[Wheeden \& Zygmund Ch.\@ 3, Ex.\@ 5]
  Construct a subset of $[0,1]$ in the same manner as the Cantor set,
  except that at the $k$th stage each interval removed has length
  $\delta 3^{-k}$, $0<\delta<1$. Show that the resulting set is perfect,
  has measure $1-\delta$, and contains no interval.
\end{problem}
\begin{solution}
  \underline{The construction}: pick a $\delta\in(0,1)$ and remove the
  open interval $(\delta/3,1-\delta/3)$ from $[0,1]$ which, in the first
  step of the construction, will yield the sets
  \[
    C_{1,1}\coloneq\left[0,\frac{\delta}{3}\right]\quad\text{and}\quad
    C_{1,2}\coloneq\left[1-\frac{\delta}{3},1\right].
  \]
  In the next step of the construction, we will be removing the interval
  $(\delta/9,\delta/3-\delta/9)\cup(1-\delta/3+\delta/9,1-\delta/9)$ from
  the union $C_1\coloneq C_{1,1}\cup C_{1,2}$ and so on. We can state this
  more cleanly: at the $n$th stage in the construction of our set
  \[
    C_n\coloneq%
    C_{n-1}\setminus\bigcup_{i=1}^{n-1}%
    \left(\frac{i\delta}{3^n},\frac{(i+1)\delta}{3^n}\right)\cup%
    \left(\frac{3^n-(i+1)\delta}{3^n},\frac{3^n-i\delta}{3^n}\right).
  \]
  (Probably; I'll check it later after I have some more time.)  Thus, at
  the $n$th stage in the construction $C_n$ is the union of $2^n$ disjoint
  intervals of length $\delta 3^{-n}$. Taking the intersection of these
  sets
  \[
    C\coloneq\bigcap_{i\in\bbN} C_i
  \]
  we claim that the set $C$ is perfect, has measure $\delta-1$ and contains
  no intervals. \textcolor{Red}{**Forget this. For some reason, I have
    forgotten the construction, so what I made was a Cantor set with
    measure zero, not what was asked for. We'll proceed as if we
    constructed what was asked of us.**}
  \\\\
  \underline{$C$ is perfect}: To prove that $C$ is perfect we must show
  that $C$ is closed and dense in itself. $C$ is closed because it is it he
  (arbitrary) intersection of closed intervals. To see that $C$ is dense in
  itself we must show that given any $\varepsilon>0$ for any $x\in C$, the
  open ball $B(x,\varepsilon)$ contains another point, call it $y$, in
  $C$. But first, we prove the following claim: At the end of each stage
  $C_n$ in the construction of $C$, \textcolor{Red}{the set of all
    endpoints of intervals in $C_n$ is a subset of $C$}. (We will not prove
  this here as it is tedious and I have already proved this before; and
  after all these notes are mostly for myself.) Now, since $x\in C$,
  $x\in C_n$ for every $n\in\bbN$. In particular, for sufficiently large
  $N$, $\delta 3^{-N}<\varepsilon$ and $x$ is in one of the closed interval
  that makes up $C_N$. Pick an endpoint $y\neq x$ in that interval. Then
  $y\in C$ and $y\in B(x,\varepsilon)$. Thus, $C$ is dense in itself.
  \\\\
  \underline{$C$ has measure $1-\delta$}: For this we simply compute the
  length of $C$ by evaluating the limit of sequence of lengths, which is
  justified by Theorem 3.26 since $C_{n+1}\subset C_n$ and $C_n$ is
  measurable for all $n\in\bbN$, hence
  \begin{align*}
    \lim_{n\to\infty}m^*(C_n)
    &=\lim_{n\to\infty}
      \left[1-\sum_{i=1}^n\delta\left(\frac{2^i}{3^{i+1}}\right)\right]\\
    &=\lim_{n\to\infty}
      \left[1-\sum_{i=1}^n\frac{\delta}{3}\left(\frac{2}{3}\right)^i\right]\\
    &=1-\frac{\delta}{3}3\\
    &=1-\delta
  \end{align*}
  as desired.
  \\\\
  \underline{$C$ contains no intervals}: Seeking a contradiction, we will
  assume that $I$ is an interval of positive measure contained in
  $C$. Since $I\coloneq[a,b]$ is a connected subset of $\bbR$ and
  $I\subset C$, then $I$ must be contained in some interval $I_n$ in the
  $n$th step of the construction of $C$ for every $n\in\bbN$. However, for
  sufficiently large $N$, $m(I_N)<m(I)=b-a$. Thus, $C$ contains no
  intervals.
\end{solution}

\begin{problem}[Wheeden \& Zygmund Ch.\@ 3, Ex.\@ 7]
  Prove (3.15).
\end{problem}
\begin{solution}
  Here is the statement of the lemma:
  \begin{quote}
    \emph{If ${\{I_k\}}_{k=1}^{N}$ is a finite collection of nonoverlapping
      intervals, then $\bigcup_{k=1}^NI_k$ is measurable and
      $m\left(\bigcup_{k=1}^NI_k\right)=\sum_{k=1}^Nm(I_k)$.}
  \end{quote}
  By Theorem 3.12, the union $\bigcup_{n=1}^N I_n$ is measurable. Now we
  show that $m\left(\bigcup_{n=1}^NI_n\right)=\sum_{n=1}^Nm(I_n)$. At least
  one inequality is straightforward, by Theorem 3.12 we have
  \[
    m\left(\bigcup_{k=1}^N I_k\right)\leq\sum_{n=1}^Nm(I_n).
  \]
  To see the reverse note that $m(I_n^\circ)=m(I_n)$ and
  $I_n^\circ\subset I_n$ so by Theorem 3.3
  \begin{align*}
    m\left(\bigcup_{n=1}^NI_n\right)
    &\geq m\left(\bigcup_{n=1}^NI_n^\circ\right)\\
    &=\sum_{n=1}^Nm(I_n^\circ)\\
    &=\sum_{n=1}^Nm(I_n).
  \end{align*}
  Thus, $\bigcup_{n=1}^N I_n$ is measurable and
  $m\left(\bigcup_{n=1}^NI_n\right)=\sum_{n=1}^Nm(I_n)$.
\end{solution}

\begin{problem}[Wheeden \& Zygmund Ch.\@ 3, Ex.\@ 8]
  Show that the Borel algebra $\calB$ in $\bbR^n$ is the smallest
  $\sigma$-algebra containing the closed sets in $\bbR^n$.
\end{problem}
\begin{solution}
  Since $\calB$ is the smallest $\sigma$-algebra containing all of the open
  sets of $\bbR^n$, it contains all of the closed sets of $\bbR^n$. Now,
  suppose that $\calB'$ is another $\sigma$-algebra containing the closed
  sets in $\bbR^n$. Then, $\calB'\subset\calB$ since $\calB$ contains all
  of the closed sets in $\bbR^n$. However, since $\calB'$ is a
  $\sigma$-algebra, it contains all of the open sets in $\bbR^n$, so
  $\calB'\subset\calB$ since $\calB$ is the smallest $\sigma$-algebra
  containing the open sets in $\bbR^n$. Thus, $\calB'=\calB$.
\end{solution}

\begin{problem}[Wheeden \& Zygmund Ch.\@ 3, Ex.\@ 9]
  If ${\{E_k\}}_{k=1}^\infty$ is a sequence of sets with
  $\sum m^*(E_k)<+\infty$, show that $\limsup E_k$ (and also $\liminf E_k$)
  has measure zero.
\end{problem}
\begin{solution}
  Suppose ${\{E_n\}}_{n=1}^\infty$ is a sequence of sets with
  $\sum_{n=1}^\infty m^*(E_n)<\infty$. Then, since the sum
  $\sum_{n=1}^\infty m^*(E_n)$ converges, given $\varepsilon>0$ there
  exist $N\in\bbN$ such that $n\geq N$ implies
  $\sum_{j=n}^\infty m^*(E_j)<\varepsilon$. But what does this say about
  the $\varlimsup_{n\to\infty} E_n$? Well, recall that
  \[
    \varlimsup_{n\to\infty}
    E_n\coloneq\bigcap_{j=1}^\infty\bigcup_{k=j}^\infty E_k.
  \]
  Define $F_n\coloneq\bigcup_{j=n}^\infty E_j$. Then $F_n\supset F_{n+1}$
  and $F_n\searrow\varlimsup_{n\to\infty} E_n$. Moreover, by Theorem 3.12
  $m^*(F_1)\leq\sum_{n=1}^\infty m^*(E_n)<\infty$ so by Theorem 3.26,
  $m(\varlimsup_{n\to\infty} E_n)=\lim_{n\to\infty}m(F_n)=0$ since the sum
  $\sum_{n=1}^\infty m(E_n)$ converges.

  Setting $F_n\coloneq\bigcap_{j=n}^\infty E_j$ yields the same for
  $\varliminf_{n\to\infty} E_n$.
\end{solution}

\begin{problem}[Wheeden \& Zygmund Ch.\@ 3, Ex.\@ 10]
  If $E_1$ and $E_2$ are measurable, show that
  $m(E_1\cup E_2)+m(E_1\cap E_2)=m(E_1)+m(E_2)$.
\end{problem}
\begin{solution}
\end{solution}

%%% Local Variables:
%%% mode: latex
%%% TeX-master: "../MA544-Quals"
%%% End:

\subsubsection{Homework 4}
\setcounter{exercise}{0}
\setcounter{equation}{0}

% #4 - Due Feb. 8 Read Sections 3.5, 4.1. Chapter 3: # 12, 13, 14, 15 (16,
% 18), (Chapter 4: # 1, 2

\begin{problem}[Wheeden \& Zygmund Ch.\@ 3, Ex.\@ 12]
  If $E_1$ and $E_2$ are measurable sets in $\bbR^1$, show $E_1\times E_2$
  is a measurable subset of $\bbR^2$ and $m(E_1\times
  E_2)=m(E_1)m(E_2)$. (Interpret $0\cdot\infty$ as $0$.) [\emph{Hint:} Use
  a characterization of measurability.]
\end{problem}
\begin{solution}
  The proof of this result is rather long and we shall omit it for now as I
  gain nothing from retracing my steps on this one.
\end{solution}

\begin{problem}[Wheeden \& Zygmund Ch.\@ 3, Ex.\@ 13]
  Motivated by (3.7), define the \emph{inner measure} of $E$ by
  $m_*(E)=\sup m(F)$, where the supremum is taken over all closed subsets
  $F$ of $E$. Show that
  \begin{enumerate}[label=(\roman*),noitemsep]
  \item $m_*(E)\leq m^*(E)$, and
  \item if $m^*(E)<\infty$, then $E$ is measurable if and only if
    $m_*(E)=m^*(E)$.
  \end{enumerate} [Use (3.22).]
\end{problem}
\begin{solution}
  First we show part (i). If $m^*(E)=\infty$, the inequality holds
  trivially. Suppose that $m^*(E)<\infty$. Then, since $F$ is closed, it is
  measurable and $m(F)=m^*(F)$. Moreover, $F\subseteq E$ so by the
  monotonicity of the outer measure,
  \[
    m(F)=m^*(F)<m^*(E).
  \]
  Taking the supremum over all $F$ on the left, we have
  \[
    m_*(E)=\sup_{F\subseteq E}m(F)<m^*(E)
  \]
  as we set out to show.

  Next we show part (ii). Let $E\subseteq\bbR^n$ with
  $m^*(E)<\infty$. $\implies$ Suppose that $E$ is measurable. Then, by
  Lemma 3.22, there exists a closed set $F\subseteq E$ such that
  $m^*(E\setminus F)<\varepsilon$. Since closed sets are measurable,
  by Corollary 3.31, we have
  \[
    m^*(E\setminus F)=m(E)-m(F)<\varepsilon
  \]
  so
  \[
    m(E)<m(F)+\varepsilon.
  \]
  Letting $\varepsilon$ go to $0$, we have
  \[
    m(E)\leq m(F);
  \]
  and taking the supremum on the right
  \[
    m(E)\leq m_*(E).
  \]
  But, by part (i), $m_*(E)\leq m^*(E)=m(E)$. Thus, $m_*(E)=m^*(E)$ as was
  to be shown.

  $\impliedby$ On the other hand, suppose that $m_*(E)=m^*(E)$. Then, given
  $\varepsilon>0$ there exists an open set $G$ containing $E$ and a closed
  set $F$ contained in $E$ such that
  \begin{align*}
    m(G)-m^*(E)&<\frac{\varepsilon}{2}\\
    m_*(E)-m(F)&<\frac{\varepsilon}{2}.
  \end{align*}
  Then
  \begin{align*}
    m^*(E\setminus F)
    &<m^*(G\setminus F)\\
    &=m^*(G)-m^*(G\cap F)\\
    &=m^*(G)-m^*(F)\\
    &<\frac{\varepsilon}{2}+m^*(E)-\left(m^*(E)-\frac{\varepsilon}{2}\right)\\
    &=\varepsilon.
  \end{align*}
  Thus, by Lemma 3.22, $E$ is measurable.
\end{solution}

\begin{problem}[Wheeden \& Zygmund Ch.\@ 3, Ex.\@ 15]
  If $E$ is measurable and $A$ is any subset of $E$, show that
  $m(E)=m_*(A)+m^*(E\setminus A)$. (See Exercise 13 for the definition
  of $m_*(A)$.)
\end{problem}
\begin{solution}
  Suppose $A\subseteq E$. If $A$ is measurable, by Problem 2, the outer and
  inner measure of $A$ agree; symbolically, we have
  $m(A)=m^*(A)=m_*(A)$. Thus, we have
  \[
    m^*(E\setminus A)=m^*(E)-m^*(A)=m^*(E)-m_*(A).
  \]

  If $A$ is not measurable and $m(E)<\infty$, then we must have
  $m^*(A),m^*(E\setminus A)<\infty$ by the monotonicity of the outer
  measure; since both $A$ an $E\setminus A$ are subsets of $E$. Hence, we
  may, without any ambiguity, subtract the quantity $m^*(E\setminus A)$
  from $m(E)$ and we have
  \begin{align*}
    m(E)-m^*(E\setminus A)
    &=m(E)
      -\inf\left\{\,m(G):\text{$E\setminus A\subseteq G$ and $G$ is
      open}\,\right\}\\
    &=m(E)
      -\inf\left\{\,m(G):\text{$E\setminus A\subseteq G\subseteq E$ and $G$ is
      open}\,\right\}\\
    &=
  \end{align*}
\end{solution}

%%% Local Variables:
%%% mode: latex
%%% TeX-master: "../MA544-Quals"
%%% End:

\subsubsection{Homework 5}
\setcounter{exercise}{0}
\setcounter{equation}{0}

% #5 - Due Feb. 15 Read Sections 3.6, 4.1 Chapter 3: # 14, 16, 18, Chapter 4: # 1, 2.
\begin{problem}[Wheeden \& Zygmund Ch.\@ 3, Ex.\@ 14]
  Show that the conclusion of part (ii) of Exercise 13 is false if
  \(m^*(E)=\infty\).
\end{problem}
\begin{solution}
  Part (ii) of Exercise 13 is part (ii) of Problem 2 from the last section
  (Homework 4). In that problem we showed that if the outer measure of
  \(E\) is finite, then \(E\) is measurable if and only if its outer and
  inner measure agree. Here we construct a counter example to this when the
  outer measure of \(E\) is \(\infty\); that is, we show that there exists
  a set \(E\) with \(m^*(E)=\infty\) such that \(m^*(E)\neq m_*(E)\). So,
  which set shall it be? Since we are unoriginal, we will pull an example
  from Wheeden and Zygmund itself.

  Let \(V\subseteq[0,1]\) be Vitali's unmeasurable (Theorem 3.38) and
  consider the union \(E=V\cup(2,\infty)\). It is clear that the inner and
  outer measure of \(E\) are both \(\infty\). However, \(E\) itself must be
  unmeasurable for otherwise \(E\cap [0,1]=V\) is measurable.
\end{solution}

\begin{problem}[Wheeden \& Zygmund Ch.\@ 3, Ex.\@ 16]
  Prove (3.34).
\end{problem}
\begin{solution}
  We must prove Equation 3.34; that is, if \(P\) is a parallellepiped
  \[
    m(P)=\vol(P).
  \]

  We may, without loss of generality, assume that one of the vertices of
  \(P\) is \(\mathbf{0}\). Let \(\{\bfe_1,\dotsc,\bfe_n\}\) be a set of
  vectors such that
  \[
    P=\left\{\,x\in\bbR^n: \text{\(x=\sum\nolimits_{k=1}^n t_k\bfe_k\),
        \(0\leq t_k\leq 1\)}\,\right\}.
  \]
  By definition, the measure of \(P\) is
  \[
    m(P)=\inf_{\calS}\left[\sum_{I_n\in\calS}\vol(I_n)\right]
  \]
  where \(\calS\) is a cover of \(P\) by intervals. Take the set of
  \begin{quote}
    \begin{remarks*}
      Literally nobody cares about this problem. I don't remember how to do
      it, but it must have been painful if I can't figure it out now, even.
    \end{remarks*}
  \end{quote}
\end{solution}

\begin{problem}[Wheeden \& Zygmund Ch.\@ 3, Ex.\@ 18]
  Prove that outer measure is \emph{translation invariant}; that is, if
  \(E_h=\left\{\,x+h:x\in E\,\right\}\) is the translate of \(E\) by \(h\),
  \(h\in\bbR^n\), show that \(m^*(E_h)=m^*(E)\). If \(E\) is measurable,
  show that \(E_h\) is also measurable. [This fact was used in proving
  (3.37).]
\end{problem}
\begin{solution}
  Let \(E\subseteq\bbR^n\) and \(h\in\bbR^n\) and define the set \(E_h\) to
  be the set \(E_h=\left\{\,x+h:x\in E\,\right\}\). We will show that the
  outer measure of \(E\) is preserved under such translations. But first,
  let us point out that \(E_h\) is nothing more than the image of \(E\)
  under the linear transformation \(T\colon\bbR^n\to\bbR^n\) given by
  \(x\mapsto x+h\). By Theorem 3.35, such a map preserves measurability of
  sets and for any measurable set \(E'\subseteq\bbR^n\),
  \(m(T(E'))=(\det T)m(E')=m(E')\) (since \(\det T=1\). Now, by Theorem
  3.6, for every \(\varepsilon>0\), there exist an open set
  \(G\supseteq E\) such that \(m^*(G)\leq m^*(E)+\varepsilon\). Consider
  the image of \(G\) under \(T\), \(T(G)\) is an open set containing
  \(E_h\) so \(m^*(G)\geq m^*(E)\) and
  \[
    m^*(T(G))=m^*(G)<m^*(E)+\varepsilon.
  \]
  Letting \(\varepsilon\to 0\), we achieve the inequality
  \[
    m^*(E_h)\leq m^*(E).
  \]

  To get the other inequality, take the map \(T^{-1}\colon\bbR^n\to\bbR^n\)
  which takes \(x\mapsto x-h\); this sends \(E_h\) to \(E\) and the same
  argument shows that
  \[
    m^*(E)\leq m^*(E).
  \]
  Thus, we have \(m^*(E)=m^*(E_h)\), as was to be shown.
\end{solution}

\begin{problem}[Wheeden \& Zygmund Ch.\@ 4, Ex.\@ 1]
  Prove corollary (4.2) and theorem (4.8)
\end{problem}
\begin{solution}
  The corollary and theorem in question are:
  \begin{quote}
    \emph{If \(f\) is measurable, then \(\left\{\,f>-\infty\,\right\}\),
      \(\left\{\,f<+\infty\,\right\}\), \(\left\{\,f=+\infty\,\right\}\),
      \(\left\{\,a\leq f\leq b\,\right\}\), \(\left\{\,f=a\,\right\}\),
      etc., are all measurable. Moreover \(f\) is measurable if and only if
      \(\left\{\,a<f<+\infty\,\right\}\) is measurable for every finite
      \(a\).}
  \end{quote}
  and
  \begin{quote}
    \emph{If \(f\) is measurable and \(\lambda\) is any real number, then
      \(f+\lambda\) and \(\lambda f\) are measurable.}
  \end{quote}

  Their proofs are quite simple. For the corollary: Suppose
  \(f\colon E\to\bbR\) is a measurable function. By Theorem 4.1, \(f\) is
  measurable if and only if for every finite \(\alpha\in\bbR\), the sets
  \begin{align*}
    &\left\{\,x\in E: f(x)\geq\alpha\,\right\}\\
    &\left\{\,x\in E:f(x)<\alpha\,\right\}\\
    &\left\{\,x\in E:f(x)\leq\alpha\,\right\}
  \end{align*}
  are measurable. Since measurable sets form a \(\sigma\)-algebra on
  \(\bbR^n\), we know that the countable union and intersection of
  measurable sets is measurable. Thus,
  \begin{align*}
    \left\{\,x\in E:f(x)>-\infty\,\right\}
    &=\bigcup_{\alpha\in\bbZ}\left\{\,x\in E:f(x)>\alpha\,\right\}\\
    \left\{\,x\in E:f(x)=\infty\,\right\}
    &=\bigcap_{n=1}^\infty\left\{\,x\in E:f(x)>n\,\right\}\\
    \left\{\,x\in E:f(x)<\infty\,\right\}
    &=\bigcup_{\alpha\in\bbZ}\left\{\,x\in E:f(x)<\alpha\,\right\}
  \end{align*}
  are easily seen to be measurable.

  Showing that \(\{\,x\in E:f(x)=\alpha\,\}\) and
  \(\left\{\,x\in E:\alpha<f(x)<\beta\,\right\}\) are measurable requires
  some clever (but not too clever) intersection/union of the sets we get
  from Theorem 4.1.

  % Viewing \(\lambda\) as the constant mapping \(g\colon E\to \bbR\) where
  % \(x\mapsto\lambda\) we have that \(\lambda\) is measurable since it is
  % continuous.

  For the theorem: Suppose \(f\) is measurable and \(\lambda\) is a
  constant. By Theorem 4.1, for any finite \(\alpha\in\bbR\) we have
  \[
    \left\{\,x\in E:f(x)>\alpha-\lambda\,\right\}
  \]
  so
  \[
    \left\{\,x\in E:f(x)+\lambda>\alpha\,\right\}
  \]
  is measurable. Thus, \(f+\lambda\) is measurable. Similarly, for
  \(\lambda\neq 0\), taking the set
  \[
    \left\{\,x\in E:f(x)>\alpha/\lambda\,\right\}
    =
    \left\{\,x\in E:\lambda f(x)>\alpha\,\right\}
  \]
  shows that \(\lambda f\) is measurable; otherwise, if \(\lambda=0\),
  \(\lambda f=0\) is constant and hence is continuous which in turn implies
  that it is measurable.
\end{solution}

\begin{problem}[Wheeden \& Zygmund Ch.\@ 4, Ex.\@ 2]
  Let \(f\) be a simple function, taking its distinct values on disjoint
  sets \(E_1,\dotsc,E_N\). Show that \(f\) is measurable if and only if
  \(E_1,\dotsc,E_N\) are measurable.
\end{problem}
\begin{solution}
  \(\implies\) Suppose that \(f\) is measurable. Then, by Corollary 4.2,
  the sets of the form \(\left\{\,f=\alpha_n\,\right\}=E_n\) are
  measurable. So the sets \(E_n\) are measurable.

  \(\impliedby\) On the other hand, suppose that the sets \(E_n\) are
  measurable. Then, \(\chi_{E_n}\) is measurable so by Theorem 4.8, \(f\)
  is measurable since it is the sum
  \[
    f=\sum_{n=1}^N \alpha_{E_n}.
  \]
\end{solution}

%%% Local Variables:
%%% mode: latex
%%% TeX-master: "../MA544-Quals"
%%% End:

\subsubsection{Homework 6}
\setcounter{exercise}{0}
% #6 - Due Feb. 22 Read Sections 4.2-3. Chapter 4: # 4, 7, 8.
\begin{problem}[Wheeden \& Zygmund Ch.\@ 4, Ex.\@ 4]
  Let $f$ be defined and measurable in $\bbR^n$. If $T$ is a nonsingular
  linear transformation of $\bbR^n$, show that $f(T(x))$ is measurable. [If
  $E_1=\left\{\,x: f(x)>a\,\right\}$ and
  $E_2=\left\{\,x:f(T(x))>a\,\right\}$, show $E_2=T^{-1}(E_1)$.]
\end{problem}
\begin{solution}
  % Let $f$ be a measurable function and $T$ a nonsingular linear
  % transformation. Then we show that for any finite $\alpha\in\bbR$, for
  % $E_1=\left\{\,x:f(x)>\alpha\,\right\}$ and
  % $E_2=\left\{\,x:f(T(x))>\alpha\,\right\}$, we have $E_2=T^{-1}(E_1)$
  % which, by Theorem 3.35, shows that $E_2$ is measurable and hence, the
  % composition $f\circ T$ is measurable.
  Let $f\colon\bbR^n\to\bbR$ be a measurable function and
  $T\colon\bbR^n\to\bbR^n$ be a linear transformation. Then, we show that
  the composition $f\circ T$ is measurable. Fix a finite $\alpha\in\bbR$
  and let
  \begin{align*}
    E_1&=\left\{\,x:f(x)>\alpha\,\right\}\\
    E_2&=\left\{\,x:f(T(x))>\alpha\,\right\}.
  \end{align*}
  Then, by Theorem 3.35, it suffices to show that $E_2=T^{-1}(E_1)$ since
  $T^{-1}$ is a nonsingular linear transformation so it sends measurable
  sets to measurable sets. But this equality is obvious: Suppose
  $x\in E_2$; then $f(T(x))>\alpha$ so, because $T$ is nonsingular and
  therefore bijective, clearly $x\in T^{-1}(E_1)$ so $E_2\subset
  T^{-1}(E_1)$. One the other hand, if $x\in T^{-1}(E_1)$ then $x$ is a
  point in $E$ such that $f(T(x))>\alpha$ so $x\in E_2$. Thus,
  $E_2=T^{-1}(E_1)$ and consequently, $f\circ T$ is a measurable function.
\end{solution}

\begin{problem}[Wheeden \& Zygmund Ch.\@ 4, Ex.\@ 7]
  Let $f$ be usc and less that $\infty$ on a compact set $E$. Show that $f$
  is bounded above on $E$. Show also that $f$ assumes its maximum on $E$,
  i.e., that there exists $x_0\in E$ such that $f(x_0)\geq f(x)$ for all
  $x\in E$.
\end{problem}
\begin{solution}
  First we show that $f$ is bounded. Suppose that $f$ is u.s.c.\@ on
  $E$. Then, by Theorem 4.14 (i), sets of the form
  $\left\{\,x\in E:f(x)<\alpha\,\right\}$ are relatively open. Let
  $\calG={\{G_\alpha\}}_{\alpha\in\bbZ}$ where
  $G_a=\left\{\,x\in E:f(x)<\alpha\,\right\}$. Then $\calG$ forms an open
  cover of $E$ and since $E$ is compact there exists a finite subset
  ${\{G_{\alpha_n}\}}_{n=1}^N$ for some finite subset
  $\{\,\alpha_1,\ldots,\alpha_N\,\}$ of $\bbZ$. Let
  $\alpha=\max\left\{\,\alpha_1,\ldots,\alpha_N\,\right\}$. Then,
  $f(x)<\alpha$ for all $x\in E$ so $f$ is bounded above by $\alpha$.

  Next, we show that $f$ in fact assumes its maximum (locally) on $E$ by
  using only topological properties of $f$. Since sets of the form
  $\left\{\,x\in E:f(x)\geq\alpha\,\right\}$ are relatively closed, by
  Theorem 4.14 (i), for fixed $x\in E$ the sets
  $F_x=\left\{\,y\in E:f(y)\geq f(x)\,\right\}$ are relatively
  closed. Consider the collection ${\{F_x\}}_{x\in E}$ of closed subsets of
  $E$. First, note that each of these sets is nonempty since
  $f(x)\geq f(x)$ so $x\in F_x$ for every $x\in E$. Now, let
  ${\{x_n\}}_{n=1}^N\subset E$ and consider the collection
  ${\{F_{x_n}\}}_{n=1}^N$. Then $\bigcap_{n=1}^N F_{x_n}\neq\emptyset$
  since for $x$ the point in $\{\,x_1,\ldots,x_N\,\}$ such that
  $f(x)=\min\left\{\,f(x_1),\ldots,f(x_N)\,\right\}$, $x\in F_{x_n}$ for
  all $1\leq n\leq N$. Thus, by the finite intersection property, the
  intersection $F=\bigcap_{x\in E}F_x$ is nonempty. Let
  $y\in\bigcap_{x\in E} F_x$, then $f(y)\geq f(x)$ for all $x\in E$ so $f$
  achieves its maximum (locally) on $E$.
\end{solution}

\begin{problem}[Wheeden \& Zygmund Ch.\@ 4, Ex.\@ 8]
  \hfill
  \begin{enumerate}[label=(\alph*),noitemsep]
  \item Let $f$ and $g$ be two functions which are u.s.c.\@ at $x_0$. Show
    that $f+g$ is u.s.c.\@ at $x_0$. Is $f-g$ u.s.c.\@ at $x_0$? When is
    $fg$ u.s.c.\@ at $x_0$?
  \item If $\left\{f_k\right\}$ is a sequence of functions are u.s.c.\@ at
    $x_0$, show that $\inf f_k(x)$ is u.s.c.\@ at $x_0$.
  \item If $\left\{f_k\right\}$ is a sequence of functions which are
    u.s.c.\@ at $x_0$ and which converge uniformly near $x_0$, show that
    $\lim f_k$ is u.s.c.\@ at $x_0$.
  \end{enumerate}
\end{problem}
\begin{solution}
  We prove these in alphabetical order (a) $\to$ (b) $\to$ (c).

  For (a), suppose that $f$ and $g$ are u.s.c.\@ at $x_0$. Then given
  $M>f(x_0),g(x_0)$ there exists $\delta_1,\delta_2>0$ such that
  $f(x),g(x)<M/2$ for all $|x_1-x_0|<\delta_1,|x_2-x_0|<\delta_2$,
  respectively. Let $\delta$ be the minimum of
  $\{\,\delta_1,\delta_2\,\}$. Then for any $x$ such that $|x-x_0|<\delta$,
  we have
  \begin{align*}
    |f(x)+g(x)-(f(x_0)+g(x_0))|
    &=|(f(x)-f(x_0))+(g(x)-g(x_0))|\\
    &\leq |(f(x)-f(x_0))|+|(g(x)-g(x_0))|\\
    &<\frac{M}{2}+\frac{M}{2}\\
    &=M.
  \end{align*}
  Thus, $f+g$ is u.s.c.

  For that second little part of (a), the one that asks ``Is $f-g$ u.s.c.\@
  at $x_0$?'' we provide a counter example.\marginremark{In fact, we could
    have predicted this, since if $g$ is u.s.c.\@ at $x_0$, $-g$ is
    l.s.c.\@ at $x_0$.} In fact, the following is enough of a
  counterexample: Take $f=0$ (which is continuous everywhere) and $g$ any
  function that is u.s.c.\@, but not continuous, at $x_0$ then $f-g=-g$ is
  l.s.c.\@ at $x_0$. Another counterexample is provided by the equations
  $u_1$ and $u_2$ from Ch.\@ 4 of Wheeden and Zygmund: Fix an $x_0\in\bbR$
  and define
  \begin{align*}
    u_1(x)&=\begin{cases}
      0&\text{if $x<x_0$,}\\
      1&\text{if $x\geq x_0$,}
    \end{cases}
    &
    u_2(x)&=\begin{cases}
      0&\text{if $x\leq x_0$,}\\
      1&\text{if $x>x_0$.}
    \end{cases}
  \end{align*}
  Then
  \[
    u_1(x)-u_2(x)=
    \begin{cases}
      0&\text{if $x\leq x_0$,}\\
      1&\text{if $x>x_0$.}
    \end{cases}
  \]
  is not u.s.c.\@ at $x_0$ since being u.s.c.\@ at $x_0$ implies that for
  $1/2>f(x_0)=0$ there exists $\delta>0$ such that $f(x)<1/2$ for all
  $x\in (x_0-\delta,x_0+\delta)$. But for any $x'>x_0$ in
  $(x_0-\delta,x+\delta)$, $u(x')=1>1/2$ which contradicts the assumption
  that $u$ is u.s.c.\@ at $x_0$.

  For that last bit of part (a), we claim that a sufficient condition for
  the product of two u.s.c.\@ maps $fg$ to be u.s.c.\@ is that
  $(fg)(x)\geq 0$ for all $x\in E$. If this condition is satisfied, then
  for any $x\to x_0$, $x\in E$, we have
  \[
    \limsup_{x\to x_0} (fg)(x)\leq%
    \Bigl(\limsup_{x\to x_0}f(x)\Bigr)%
    \Bigl(\limsup_{x\to x_0}g(x)\Bigr)%
    \leq (fg)(x_0).
  \]
  Hence, the product $fg$ is u.s.c.\@ at $x_0$.

  For (b), suppose $\{f_n\}_{n=1}^\infty$ is a sequence of functions that
  are u.s.c.\@ at $x_0$. Let $f=\inf f_n$. Then for every $\varepsilon>0$,
  by the definition of the infimum, $f(x_0)+\varepsilon>f_n(x_0)$ for some
  $n\in\bbN$.
\end{solution}

%%% Local Variables:
%%% mode: latex
%%% TeX-master: "../MA544-Quals"
%%% End:

\subsubsection{Homework 7}
\setcounter{exercise}{0}
\setcounter{equation}{0}

% #7 - Due Feb. 29 Read Sections 5.1-2. Chapter 4: #9, 11, 15, 18 (ignore
% the second part concerning convergence in measure);
\begin{problem}[Wheeden \& Zygmund Ch.\@ 4, Ex.\@ 9]
  \hfill
  \begin{enumerate}[label=(\alph*),noitemsep]
  \item Show that the limit of a decreasing (increasing) sequence of
    functions u.s.c.\@ (l.s.c.) at $x_0$ is u.s.c.\@ (l.s.c.) at $x_0$. In
    particular, the limit of a decreasing (increasing) sequence of
    functions continuous at $x_0$ is u.s.c.\@ (l.s.c.) at $x_0$.
  \item Let $f$ be u.s.c.\@ and less than $\infty$ on $[a,b]$. Show that there
    exists continuous $f_k$ on $[a,b]$ such that $f_k\downarrow f$.
  \end{enumerate}
\end{problem}
\begin{solution}
  For part (a) we may as well assume that $f\geq 0$ for all $x$. Let
  $\{f_n\}$, $n\in\bbN$, be a sequence of decreasing functions with limit
  $f$ which are u.s.c.\@ at $x_0$. Then, for every $n\in\bbN$, for every
  sequence $x\to x_0$,
  \[
    \limsup_{x\to x_0}f_n(x)\leq f_n(x_0).
  \]
  Now, we claim that $f(x)\leq f_n(x)$ for every $x$ and every $n\in\bbN$.
  \begin{subproof}[Proof of claim]
    Suppose $f(x)>f_{N_1}(x)$ for some $x$, $N_1\in\bbN$. Then there exists
    a real number $\varepsilon>0$ such that $0<\varepsilon<|f(x)-f_n(x)|$
    (we may, for example, take $\varepsilon$ to be in $\bbQ$ which is dense
    in $\bbR$). Then, since $f_n\downarrow f$, there exists an index
    $N_1\in\bbN$ such that
    \[
      |f(x)-f_n(x)|<\varepsilon.
    \]
    However, since the sequence $f_n$ decreases to $f$, for
    $n\geq\max\{N_1,N_2\}$, $f_n(x)\leq f_{N_1}(x)$ so
    \[
      |f(x)-f_n(x)|>|f(x)-f_{N_1}(x)|>\varepsilon.
    \]
    This is a contradiction.
  \end{subproof}
  Having established this, for every sequence $x\to x_0$, we have
  \[
    \limsup_{x\to x_0} f(x)\leq \limsup_{x\to x_0} f_n(x)\leq f_n(x_0).
  \]
  Letting $n\to\infty$,
  \[
    \limsup_{x\to x_0} f(x)\leq \lim_{n\to\infty}f_n(x_0)=f(x_0).
  \]

  For part (b) suppose $f\colon [a,b]\to\bbR$ is u.s.c.\@ on $[a,b]$ and
  $f(x)<\infty$ for all $x\in [a,b]$. For a fixed $x\in[a,b]$, $f$ is
  u.s.c.\@ at $x$ if for every $\varepsilon>0$, there exists a neighborhood
  $B(x,\delta)$ such that $f(y)<f(x)+\varepsilon$. Now, let
  $\varepsilon=1/n$. Then, for each $x\in [a,b]$, there exists a
  neighborhood $B(x,\delta_x)$ such that $f(y)<f(x)+\varepsilon$ for
  $y\in B(x,\delta_x)$.

  The following
  \href{http://math.stackexchange.com/questions/462534/recognizing-uppersemicontinuous-function-as-a-pointwise-decreasing-limit}{post}
  on the Mathematics StackExchange contains a solution to part (b) of this
  problem.

  First, we claim that $f(x)\neq\infty$ for any $x\in[a,b]$, it must be
  bounded.
  \begin{subproof}[Proof of claim]
    By Theorem 4.14 (a), sets of the form $\{\,x\in[a,b]:f(x)<a\,\}$ is
    relatively open for all finite $a$. Define
    \[
      E_n=\left\{\,x\in [a,b]:f(x)<n\,\right\}.
    \]
    Then, the collection $\calE=\{E_n\}$, $n\in\bbN$, is an open cover of
    $[a,b]$. Since $[a,b]$ is compact, there exists a finite subcover
    $\{E_{n_1},\ldots,E_{n_m}\}$ of $\calE$. Letting
    $M=\max\{n_1,\ldots,n_m\}$, we have $f<M$ for all $x\in[a,b]$. Thus,
    $f$ is bounded on $[a,b]$.
  \end{subproof}\noindent
  Now that we have established that $f$ is bounded on $[a,b]$ by, say, $M$
  then $\sup_{x\in[a,b]} f\leq M$. Define
  \[
    f_n(x)=\sup_{y\in[a,b]}\bigl[f(y)-n|x-y|\bigr].
  \]
  We claim that this family of functions $\{f_n\}$, $n\in\bbN$, is
  continuous and that $f_n\to f$. To see that $f$ is continuous, we observe
  that this family of functions is in fact Lipschitz continuous
  \begin{align*}
    |f_n(x)-f_n(y)|
    &=\left|\sup_{z\in[a,b]}\bigl[ f(z)-n|x-z|
      \bigr]-\sup_{z\in[a,b]}\bigl[ f(z)-n|y-z| \bigr]\right|\\
    &\leq\left|\sup_{z\in[a,b]}\bigl[ f(z)-n|x-z|-f(z)-n|y-z| \bigr]\right|\\
    &=\left|\sup_{z\in[a,b]}\bigl[-n|x-z|-n|y-z|\bigr]\right|\\
    &=\left|\sup_{z\in[a,b]}\bigl[-n|x-y+(y-z)|-n|y-z|\bigr]\right|\\
    &\leq\left|\sup_{z\in[a,b]}\bigl[-n|x-y|-2n|y-z|\bigr]\right|\\
    &=n|x-y|.
  \end{align*}
  Thus, $f_n$ is Lipschitz and in particular, it is continuous.

  To see that $f_n\to f$ pointwise, let $\varepsilon>0$ be given then we
  must show that there exists some index $N$ such that $n\geq N$ implies
  \[
    |f(x)-f_n(x)|<\varepsilon.
  \]
  Expanding the equation above, we see that
  \[
    |f(x)-f_n(x)|=
    \left|
      f(x)-\sup_{y\in[a,b]}\bigl[f(y)-n|x-y|\bigr].
    \right|
  \]
\end{solution}

\begin{problem}[Wheeden \& Zygmund Ch.\@ 4, Ex.\@ 11]
  Let $f$ be defined on $\bbR^n$ and let $B(x)$ denote the open ball
  $\left\{\,y:|x-y|<r\,\right\}$ with center $x$ and fixed radius $r$. Show
  that the function $g(x)=\sup\left\{\,f(y):y\in B(x)\,\right\}$ is
  l.s.c.\@ and the function $h(x)=\inf\left\{\,f(y):y\in B(x)\,\right\}$ is
  u.s.c.\@ on $\bbR^n$. Is the same true for the closed ball
  $\left\{\,y:|x-y|\leq r\,\right\}$?
\end{problem}
\begin{solution}
  Note that, by properties of the infimum/supremum for any set of real
  numbers $S\subset\bbR$,
  \[
    \sup S=-\inf (-S)
  \]
  where $-S=\left\{\,-s:s\in S\,\right\}$. Thus,
  \begin{align*}
    g(x)
    &=-\inf\left\{\,-f(y):y\in B(x,r)\,\right\}\\
    &=\sup\left\{\,f(y):y\in B(x,r)\, \right\}.
  \end{align*}
  Letting $f'=-f$, it suffices to show that $g'(x)=\inf\{\,f'(y):y\in
  B(x,r)\,\}$ is u.s.c.\@ since for any u.s.c.\@ function $f$, $-f$ is
  l.s.c. Therefore, we show that $h$ is u.s.c.

  To see that $h$ is u.s.c., let $M>h(x_0)$. Then we must show that there
  exists a neighborhood $B(x_0,\delta)$ such that $M>h(x)$ for every
  $x\in B(x_0,\delta)$. Since $h(x_0)$ is the infimum of $f(x)$ over all
  $x\in B(x_0,r)$, given $\varepsilon>0$ there exists $x\in B(x_0,r)$ such
  that $f(x)<h(x_0)+\varepsilon<M$. Define $\delta=(r-|x-y|)/2$. Then we
  claim that for any $x\in B(x_0,\delta)$,
  \[
    g(x)<M.
  \]
  \begin{subproof}[Proof of claim]
    Let $x\in B(x_0,\delta)$. Then $y\in B(x_0,\delta)$ since
    \begin{align*}
      |x-y|&=|x-x_0-(y-x_0)|\\
           &\leq |x-x_0|+|y-x_0|\\
           &=(r-|y-x_0|)/2+|y-x_0|\\
           &=r/2+|y-x_0|/2\\
           &<r.
    \end{align*}
    Thus,
    \[
      g(x)\leq f(y)<g(x_0)+\varepsilon<M.
    \]
  \end{subproof}
  It follows that $g$ is u.s.c.
\end{solution}

\begin{problem}[Wheeden \& Zygmund Ch.\@ 4, Ex.\@ 15]
  Let $\left\{f_k\right\}$ be a sequence of measurable functions defined on
  a measurable set $E$ with $m(E)<\infty$. If $|f_k(x)|\leq M_x<\infty$ for
  all $k$ for each $x\in E$, show that given $\varepsilon>0$, there is
  closed $F\subseteq E$ and finite $M$ such that
  $m(E\setminus F)<\varepsilon$ and $|f_k(x)|\leq M$ for all $x\in F$.
\end{problem}
\begin{solution}
  Set $f=\sup_{n\in\bbN} |f_n|$; then, $f$ is measurable since it is the
  supremum of measurable functions $|f_n|$. By Lusin's theorem, there
  exists a closed subset $F'$ of $E$ with $m(E\setminus F')<\varepsilon$
  and a continuous function $g\colon E\to\bbR$ such that
  $g\restrict{F'}=f\restrict{F'}$. Now, consider the set
\end{solution}

\begin{problem}[Wheeden \& Zygmund Ch.\@ 4, Ex.\@ 18]
  If $f$ is measurable on $E$, define
  $\omega_f(a)=m\{\,f>a\,\}$ for $-\infty<a<\infty$. If
  $f_k\uparrow f$, show that $\omega_{f_k}\uparrow\omega_f$. If $f_k\to f$,
  show that $\omega_{f_k}\to\omega_f$ at each point of continuity of
  $\omega_f$. [For the second part, show that if $f_k\to f$, then
  $\limsup_{k\to\infty}\omega_{f_k}(a)\leq\omega_f(a-\varepsilon)$ and
  $\liminf_{k\to\infty}\omega_{f_k}(a)\geq\omega_f(a+\varepsilon)$ for
  every $\varepsilon>0$.]
\end{problem}
\begin{solution}

\end{solution}

% (Chapter 5: # 1, 2, 3, 4).
\begin{problem}[Wheeden \& Zygmund Ch.\@ 5, Ex.\@ 1]
  If $f$ is a simple measurable function (not necessarily positive) taking
  values $a_j$ on $E_j$, $j=1,\ldots,N$, show that
  $\int_E f=\sum_{j=1}^N a_jm(E_j)$. [Use (5.24)].
\end{problem}
\begin{solution}
\end{solution}

\begin{problem}[Wheeden \& Zygmund Ch.\@ 5, Ex.\@ 3]
  Let $\left\{f_k\right\}$ be a sequence of nonnegative measurable
  functions defined on $E$. If $f_k\to f$ and $f_k\leq f$ a.e.\@ on $E$,
  show that $\int_E f_k\to\int_E f$.
\end{problem}
\begin{solution}
\end{solution}

%%% Local Variables:
%%% mode: latex
%%% TeX-master: "../MA544-Quals"
%%% End:

\subsection{Homework 8}
\begin{problem}[Wheeden \& Zygmund Ch.\@ 5, Ex.\@ 2]
  Show that the conclusion of (5.32) are not true without the assumption
  that $\varphi\in L(E)$. [In part (ii), for example, take
  $f_k=\chi_{(k,\infty)}$.]
\end{problem}
\begin{solution}
\end{solution}

\begin{problem}[Wheeden \& Zygmund Ch.\@ 5, Ex.\@ 4]
  If $f\in L(0,1)$, show that $x^kf(x)\in L(0,1)$ for $k=1,2,...$, and
  $\int_0^1 x^kf(x)\diff x\to 0$.
\end{problem}
\begin{solution}
\end{solution}

\begin{problem}[Wheeden \& Zygmund Ch.\@ 5, Ex.\@ 6]
  Let $f(x,y)$, $0\leq x,y\leq 1$, satisfy the following conditions: for
  each $x$, $f(x,y)$ is an integrable function of $y$, and
  $\partial f(x,y)/\partial x$ is a bounded function of $(x,y)$. Show that
  $\partial f(x,y)/\partial x$ is a measurable function of $y$ for each $x$
  and
  \[
    \frac{\rmd}{\rmd x}\int_0^1f(x,y)\diff
    y=\int_0^1\frac{\partial}{\partial x}f(x,y)\diff y.
  \]
\end{problem}
\begin{solution}
\end{solution}

\begin{problem}[Wheeden \& Zygmund Ch.\@ 5, Ex.\@ 7]
  Give an example of an $f$ that is not integrable, but whose improper
  Riemann integral exists and is finite.
\end{problem}
\begin{solution}
\end{solution}

\begin{problem}[Wheeden \& Zygmund Ch.\@ 5, Ex.\@ 21]
  If $\int_A f=0$ for every measurable subset A of a measurable set $E$,
  show that $f=0$ a.e.\@ in $E$.
\end{problem}
\begin{solution}
\end{solution}

\begin{problem}[Wheeden \& Zygmund Ch.\@ 6, Ex.\@ 10]
  Let $V_n$ be the volume of the unit ball in $\bbR^n$. Show by using
  Fubini's theorem that
  \[
    V_n=2V_{n-1}\int_0^1\left(1-t^2\right)^{(n-1)/2}\diff t.
  \]
  (We also observe that by setting $w=t^2$, the integral is a multiple of a
  classical $\beta$-function and so can be expressed in terms of the
  $\Gamma$-function: $\Gamma(s)=\int_0^\infty e^{-t}t^{s-1}\diff t$,
  $s>0$.)
\end{problem}
\begin{solution}
\end{solution}

\begin{problem}[Wheeden \& Zygmund Ch.\@ 6, Ex.\@ 11]
  Use Fubini's theorem to prove that
  \[
    \int_{\bbR^n}e^{-|\bfx|^2}\diff\bfx=\pi^{n/2}.
  \]
  (For $n=1$, write
  $\left(\int_{-\infty}^\infty e^{-x^2}\diff
    x\right)^2=\int_{-\infty}^\infty\int_{-\infty}^\infty e^{-x^2-y^2}\diff
  xdy$ and use polar. For $n>1$, use the formula
  $e^{-|\bfx|^2}=e^{-{x_1}^2}\cdots e^{-{x_n}^2}$ and Fubini's theorem to
  reduce the case $n=1$.)
\end{problem}
\begin{solution}
\end{solution}

%%% Local Variables:
%%% mode: latex
%%% TeX-master: "../MA544-Quals"
%%% End:

\subsection{Homework 9}
\begin{problem}[Wheeden \& Zygmund Ch.\@ 6, Ex.\@ 1]
\begin{enumerate}[label=(\alph*),noitemsep]
\item Let $E$ be a measurable subset of $\bbR^2$ such that for almost every
  $x\in\bbR$, $\left\{\,y:(x,y)\in E\,\right\}$ has
  $\bbR$-measure zero. Show that $E$ has measure zero and that for almost
  every $y\in\bbR$, $\left\{\,x:(x,y)\in E\,\right\}$ has
  measure zero.
\item Let $f(x,y)$ be nonnegative and measurable in $\bbR^2$. Suppose that
  for almost every $x\in\bbR$, $f(x,y)$ is finite for almost every
  $y$. Show that for almost $y\in\bbR$, $f(x,y)$ is finite for almost
  every $x$.
\end{enumerate}
\end{problem}
\begin{solution}
\end{solution}

\begin{problem}[Wheeden \& Zygmund Ch.\@ 6, Ex.\@ 3]
Let $f$ be measurable and finite a.e.\@ on $[0,1]$. If $f(x)-f(y)$ is
integrable over the square $0\leq x\leq 1$, $0\leq y\leq 1$, show that
$f\in L[0,1]$.
\end{problem}
\begin{solution}
\end{solution}

\begin{problem}[Wheeden \& Zygmund Ch.\@ 6, Ex.\@ 4]
Let $f$ be measurable and periodic with period $1$: $f(t+1)=f(t)$. Suppose
there is a finite $c$ such that
\[
\int_0^1|f(a+t)-f(b+t)|\diff t\leq c
\]
for all $a$ and $b$. Show that $f\in L[0,1]$. (Set $a=x$, $b=-x$, integrate
with respect to $x$, and make the change of variables $\xi=x+t$,
$\eta=-x+t$.)
\end{problem}
\begin{solution}
\end{solution}

\begin{problem}[Wheeden \& Zygmund Ch.\@ 6, Ex.\@ 6]
For $f\in L(\bbR)$, define the \emph{Fourier transform $\hat f$} of $f$
by
\[
\hat f(x)=\frac{1}{2\pi}\int_{-\infty}^\infty f(t)e^{-ixt}\diff t
\]
for $x\in\bbR$. (For complex-valued function $F=F_0+iF_1$ whose real and
imaginary parts $F_0$ and $F_1$ are integrable, we define $\int F=\int
F_0+i\int F_1$.) Show that if $f$ and $g$ belong to $L(\bbR)$, then
\[
\widehat{(f*g)}(x)=2\pi\hat f(x)\hat g(x).
\]
\end{problem}
\begin{solution}
\end{solution}

\begin{problem}[Wheeden \& Zygmund Ch.\@ 6, Ex.\@ 7]
Let $F$ be a closed subset of $\bbR$ and let $\delta(x)=\delta(x,F)$ be
the corresponding distance function. If $\lambda>0$ and $f$ is nonnegative
and integrable over the complement of $F$, prove that the function
\[
\int_{\bbR}\frac{\delta^\lambda(y)f(y)}{|x-y|^{1+\lambda}}\diff
t
\]
is integrable over $F$ and so is finite a.e.\@ in $F$. (In case
$f=\chi_{(a,b)}$, this reduces to Theorem 6.17.)
\end{problem}
\begin{solution}
\end{solution}

\begin{problem}[Wheeden \& Zygmund Ch.\@ 6, Ex.\@ 9]
\begin{enumerate}[label=(\alph*)]
\item Show that $M_\lambda(x;F)=+\infty$ if $x\notin F$, $\lambda>0$.
\item Let $F=[c,d]$ be a closed subinterval of a bounded open interval
  $(a,b)\subset\bbR$, and let $M_\alpha$ be the corresponding
  Marcinkiewicz integral, $\lambda>0$. Show that $M_\lambda$ is finite for
  every $x\in(c,d)$ and that $M_\lambda(c)=M_\lambda(d)=\infty$. Show also
  that $\int M_\lambda\leq\lambda^{-1}|G|$, where $G=(a,b)-[c,d]$.
\end{enumerate}
\end{problem}
\begin{solution}
\end{solution}

%%% Local Variables:
%%% mode: latex
%%% TeX-master: "../MA544-Quals"
%%% End:

\subsection{Homework 10}
\begin{problem}[Wheeden \& Zygmund Ch.\@ 7, Ex.\@ 1]
Let $f$ be measurable in $\bbR^n$ and different from zero in some set of
positive measure. Show that there is a positive constant $c$ such that
$f^*(\bfx)\geq c\|\bfx\|^{-n}$ for $\|\bfx\|\geq 1$.
\end{problem}
\begin{proof}
\end{proof}

\begin{problem}[Wheeden \& Zygmund Ch.\@ 7, Ex.\@ 2]
Let $\varphi(\bfx),\bfx\in\bbR^n$, be a bounded measurable function such
that $\varphi(\bfx)=0$ for $\|\bfx\|\geq 1$ and $\int\varphi=1$. For
$\varepsilon>0$, let
$\varphi_\varepsilon(\bfx)=\varepsilon^{-n}\varphi(\bfx/\varepsilon)$. ($\varphi_\varepsilon$
is called an \emph{approximation to the identity}.) If $f\in L(\bbR^n)$,
show that
\[
\lim_{\varepsilon\to 0}(f*\varphi_\varepsilon)(\bfx)=f(\bfx)
\]
in the Lebesgue set of $f$. (Note that $\int\varphi_\varepsilon=1$,
$\varepsilon>0$, so that
\[
(f*\varphi_\varepsilon)(\bfx)-f(\bfx)=\int\left[f(\bfx-\bfy)-f(\bfx)\right]\varphi_\varepsilon(\bfy)\diff\bfy.
\]
Use Theorem 7.16.)
\end{problem}
\begin{proof}
\end{proof}

\begin{problem}[Wheeden \& Zygmund Ch.\@ 7, Ex.\@ 6]
Show that if $\alpha>0$, then $x^\alpha$ is absolutely continuous on every
bounded subinterval of $[0,\infty)$.
\end{problem}
\begin{proof}
\end{proof}

\begin{problem}[Wheeden \& Zygmund Ch.\@ 7, Ex.\@ 8]
Prove the following converse of Theorem 7.31: If $f$ is of bounded
variation on $[a,b]$, and if the function $V(x)=V[a,x]$ is absolutely
continuous on $[a,b]$, then $f$ is absolutely continuous on $[a,b]$.
\end{problem}
\begin{proof}
\end{proof}

\begin{problem}[Wheeden \& Zygmund Ch.\@ 7, Ex.\@ 9]
If $f$ is of bounded variation on $[a,b]$, show that
\[
\int_a^b|f'|\leq V[a,b].
\]
Show that if equality holds in this inequality, then $f$ is absolutely
continuous on $[a,b]$. (For the second part, use Theorems 2.2(ii) and 7.24
to show that $V(x)$ is absolutely continuous and then use the result of
Exercise 8).
\end{problem}
\begin{proof}
\end{proof}

\begin{problem}[Wheeden \& Zygmund Ch.\@ 7, Ex.\@ 12]
Use Jensen's inequality to prove that if $a,b\geq 0$, $p,q>1$,
$(1/p)+(1/q)=1$, then
\[
ab\leq\frac{a^p}{p}+\frac{b^q}{q}.
\]
More generally, show that
\[
a_1\dotsm a_N=\sum_{j=1}^N\frac{{a_j}^{p_j}}{p_j},
\]
where $a_j\geq 0$, $p_j>1$, $\sum_{j=1}^N(1/p_j)=1$. (Write
$a_j=e^{x_j/p_j}$ and use the convexity of $e^x$).
\end{problem}
\begin{proof}
\end{proof}

\begin{problem}[Wheeden \& Zygmund Ch.\@ 7, Ex.\@ 13]
Prove Theorem 7.36.
\end{problem}
\begin{proof}
Recall the statement of Theorem 7.36
\begin{quote}
\begin{enumerate}[label=\textnormal{(\roman*)}]
\item If $\varphi_1$ and $\varphi_2$ are convex in $(a,b)$, then
  $\varphi_1+\varphi_2$ is convex in $(a,b)$.
\item If $\varphi$ is convex in $(a,b)$ and $c$ is a positive constant,
  then $c\varphi$ is convex in $(a,b)$.
\item If $\varphi_k$, $k=1,2,\dotsc$, are convex in $(a,b)$ and
  $\varphi_k\to\varphi$ in $(a,b)$, then $\varphi$ is convex in $(a,b)$.
\end{enumerate}
\end{quote}
\end{proof}

%%% Local Variables:
%%% mode: latex
%%% TeX-master: "../MA544-Quals"
%%% End:

\subsection{Homework 11}

%%% Local Variables:
%%% mode: latex
%%% TeX-master: "../MA544-Quals"
%%% End:

\subsubsection{Homework 12}
\setcounter{exercise}{0}
\setcounter{equation}{0}

\begin{problem}[Wheeden \& Zygmund Ch.\@ 8, Ex.\@ 2]
  Prove the converse of Hölder's inequality for $p=1$ and $\infty$. Show
  also that for $1\leq p\leq\infty$, a real-valued measurable $f$ belongs
  to $L^p(E)$ if $fg\in L^1(E)$ for every $g\in L^{p'}(E)$,
  $1/p+1/p'=1$. The negation is also of interest: if $f\in L^p(E)$ then
  there exists $g\in L^{p'}(E)$ such that $fg\notin L^1(E)$. (To verify the
  negation, construct $g$ of the form $\sum a_kg_k$ satisfying
  $\int_E fg_k\to\infty$.)
\end{problem}
\begin{solution}
\end{solution}

\begin{problem}[Wheeden \& Zygmund Ch.\@ 8, Ex.\@ 3]
  Prove Theorems 8.12 and 8.13. Show that Minkowski’s inequality for series
  fails when $p<1$.
\end{problem}
\begin{solution}
\end{solution}

\begin{problem}[Wheeden \& Zygmund Ch.\@ 8, Ex.\@ 4]
  Let $f$ and $g$ be real-valued and not identically $0$ (i.e., neither
  function equals $0$ a.e.), and let $1<p<\infty$. Prove that equality
  holds in the inequality $|\int fg|\leq\|f\|_p\|g\|_{p'}$ if and only if
  $fg$ has constant sign a.e.\@ and $|f|^p$ is a multiple of $|g|^{p'}$
  a.e.
  \\\\
  If $\|f+g\|_p=\|f\|_p+\|g\|_{p}$ and $g\neq 0$ in Minkowski's inequality,
  show that $f$ is a multiple of $g$.
  \\\\
  Find analogues of these results for the spaces $\ell^p$.
\end{problem}
\begin{solution}
\end{solution}

\begin{problem}[Wheeden \& Zygmund Ch.\@ 8, Ex.\@ 5]
  For $0<p\leq\infty$ and $0<|E|<\infty$, define
  \[
    N_p[f]=\left(\frac{1}{E}\int_E|f|^p\right)^{1/p},
  \]
  where $N_\infty[f]$ means $\|f\|_\infty$. Prove that if $p_1<p_2$, then
  $N_{p_1}[f]\leq N_{p_2}[f]$. Prove also that if $1\leq p\leq \infty$,
  then $N_p[f+g]\leq N_p[f]+N_p[g]$,
  $(1/|E|)\int_E|fg|\leq N_p[f]N_{p'}[g]$, $1/p+1/p'=1$, and
  $\lim_{p\to\infty} N_p[f]=\|f\|_\infty$. Thus, $N_p$ behaves like
  $\|\cdot\|_p$ but has the advantage of being monotone in $p$. Recall
  Exercise 28 of Chapter 5.
\end{problem}
\begin{solution}
\end{solution}

\begin{problem}[Wheeden \& Zygmund Ch.\@ 8, Ex.\@ 6]
  \hfill
  \begin{enumerate}[label=(\alph*),noitemsep]
  \item Let $1\leq p_i$, $r\leq\infty$ and $\sum_{i=1}^k1/p_i=1/r$. Prove
    the following generalization of Hölder's inequality:
    \[
      \|f_1\dotsm f_k\|_r\leq\|f_1\|_{p_1}\dotsm\|f_k\|_{p_k}.
    \]
  \item Let $1\leq p<r<q\leq\infty$ and define $\theta\in(0,1)$ by
    $1/r=\theta/p+(1-\theta)/q$. Prove the interpolation estimate
    \[
      \|f\|_r\leq{\|f\|_p}^\theta{\|f\|_q}^{1-\theta}.
    \]
    In particular, if $A=\max\left\{\|f\|_p,\|f\|_q\right\}$, then
    $\|f\|_r\leq A$.
\end{enumerate}
\end{problem}
\begin{solution}
\end{solution}

\begin{problem}[Wheeden \& Zygmund Ch.\@ 8, Ex.\@ 9]
  If $f$ is real-valued and measurable on $E$, $|E|>0$, define its
  essential infimum on $E$ by
  \[
    \essinf f=\sup\left\{\,\alpha:|\{\,x\in
      E:f(x)<\alpha\,\}|=0\,\right\}.
  \]
  If $f\geq 0$, show that $\essinf_E f=(\esssup 1/f)^{-1}$.
\end{problem}
\begin{solution}
\end{solution}

\begin{problem}[Wheeden \& Zygmund Ch.\@ 8, Ex.\@ 11]
  If $f_k\to f$ in $L^p$, $1\leq p<\infty$, $g_k\to g$ pointwise, and
  $\|g_k\|_\infty<M$ for all $k$, prove that $f_kg_k\to fg$ in $L^p$.
\end{problem}
\begin{solution}
\end{solution}

%%% Local Variables:
%%% mode: latex
%%% TeX-master: "../MA544-Quals"
%%% End:


\section{Danielli}
\subsection{Danielli: Practice Exams Spring 2016}
\setcounter{exercise}{0}
\setcounter{equation}{0}

\subsubsection{Exam 1 Practice}
\begin{problem}
  Let $E\subset\bbR^n$ be a measurable set, $r\in\bbR$ and define the set
  $rE=\left\{\,rx : x\in E\,\right\}$. Prove that $rE$ is measurable, and
  that $|rE|=|r|^n|E|$.
\end{problem}
\begin{solution}
  Define a map a linear map $T\colon\bbR^n\to\bbR^n$ by $T(x)=rx$. Since a
  the image of a measurable set $E$ under linear map is measurable and
  $m(T(E))=|{\det T}|m(E)=|r|^nm(E)$, it suffices to show that $T(E)=rE$.

  Let $y\in T(E)$ then $y=rx$ for some $x\in E$. Thus, $y\in rE$. Let $y\in
  rE$. Then, $y=rx=T(x)$ for some $x\in E$. Thus, $y\in T(E)$. It follows
  that $m(rE)=|r|^nm(E)$.
\end{solution}

\begin{problem}
  Let $\{E_k\}$, $k\in\bbN$ be a collection of measurable sets. Define the
  set
  \[
    \liminf_{k\to\infty} E_k
    =\bigcup_{k=1}^\infty\left(\bigcap_{n=k}^\infty E_n\right).
  \]
  Show that
  \[
    m\left(\liminf_{k\to\infty}
      E_k\right)\leq\liminf_{k\to\infty}m(E_k).
  \]
\end{problem}
% Following the style of \cite[Ch.\@ 1, p.\@ 2]{wheeden-zygmund},
  % particularly, the sets defined after the introduction of equation (1.1),
  % set
  % \begin{equation}
  %   \label{eq:prep:1:2}
  %   V_k=\bigcap_{\ell=k}^\infty E_\ell.
  % \end{equation}
  % Note that the collection of sets $\{V_k\}$ forms an increasing sequence,
  % that is, if $x\in V_k$ then, by \eqref{eq:prep:1:2}, $x $ is in the
  % intersection $E_k\cap\bigl(\bigcap_{\ell=k+1}E_\ell\bigr)$, but, by
  % \eqref{eq:prep:1:2}, $\bigcap_{\ell=k+1}E_\ell=V_{k+1}$ thus, $x $ is
  % in $V_{k+1}$ so $V_{k+1}\supset V_k$. Hence, we have
  % $V_k\nearrow\liminf E_k$.

  % Now, consider the sequence $\{|V_k|\}$ formed by the Lebesgue measure of
  % the $V_k$. By Theorem 3.26 from \cite[Ch.\@ 3, p.\@ 51]{wheeden-zygmund},
  % since $V_k\nearrow\liminf E_k$,
  % \begin{equation}
  %   \label{eq:prep:1:3}
  %   \lim_{k\to\infty}|V_k|=
  %   \lim_{k\to\infty}\left|\bigcap_{\ell=k}^\infty E_\ell\right|=
  %   \left|\liminf_{k\to\infty} E_k\right|.
  % \end{equation}
  % But note that, by the monotonicity of the Lebesgue measure, we have
  % \begin{equation}
  %   \label{eq:prep:1:4}
  %   \left|\bigcap_{\ell=k}^\infty E_\ell\right|\leq |E_k|,
  % \end{equation}
  % so, by properties of the $\liminf$, in particular, by Theorem 19(v) from
  % \cite[Ch.\@ 1, p.\@ 23]{royden}, we have
  % \begin{equation}
  %   \label{eq:prep:1:5}
  %   \limsup_{k\to\infty}|V_k|\leq\liminf_{k\to\infty}|E_k|.
  % \end{equation}
  % Hence, by \eqref{eq:prep:1:3} and Proposition 19 (iv), since the sequence
  % $\{|V_k|\}$ converges and is equal to the measure of $\liminf E_k$, by
  % \eqref{eq:prep:1:5}, we have
  % \begin{equation}
  %   \label{eq:prep:1:6}
  %   \left|\liminf_{k\to\infty} E_k\right|\leq\liminf_{k\to\infty}|E_k|
  % \end{equation}
  % as was to be shown.
\begin{solution}
  Here's a quick and dirty way of proving this: let $\chi_{E_n}$ be the
  characteristic function of $E_n$. Then, by Fatou's lemma,
  \begin{equation}
    \label{eq:ex-1-prep:fatou}
    \int\liminf_{n\to\infty}\chi_{E_n}(x)\diff x
    \leq\liminf_{n\to\infty}\int\chi_{E_n}(x)\diff x.
  \end{equation}
  By definition of the characteristic function, it is easy to see that the
  right hand-side of the Equation \eqref{eq:ex-1-prep:fatou} is
  \[
    \liminf_{k\to\infty}m(E_k).
  \]
  But what about the left-hand side of \eqref{eq:ex-1-prep:fatou}? We claim
  that
  \[
    \liminf_{n\to\infty}\chi_{E_n}=\chi_{E}
  \]
  where $E=\liminf_{n\to\infty} E_n$.
  \begin{quote}
    \begin{proof}[Proof of claim]
      Let $x\in E$. We must show that
      $\liminf_{n\to\infty}\chi_{E_n}(x)=1$. By definition
      \[
        \liminf_{n\to\infty}\chi_{E_n}=%
        \lim_{n\to\infty}\left[\inf_{k\geq n}\chi_{E_k}\right].
      \]
      Now
    \end{proof}
  \end{quote}

  Define
  \[
    V_n=\bigcap_{k=n}^\infty E_k.
  \]
  Note that ${\{V_n\}}_{n=1}^\infty$ forms an increasing sequence of
\end{solution}

\begin{problem}
  Consider the function
  \[
    F(x)=
    \begin{cases}
      |B(\mathbf{0},x)|&x>0\\
      0&x=0
    \end{cases}.
  \]
  Here $B(\mathbf{0},r)=\left\{\, y \in\bbR^n:| y |<r\,\right\}$. Prove
  that $F$ is monotonic increasing and continuous.
\end{problem}
\begin{solution}
  Define the linear map $T\colon[0,\infty)\times\bbR^n\to\bbR^n$ by
  $T(r) x = rx $. We claim that $B(\mathbf{0},r)=T(r,B(\mathbf{0},1))$. To
  reduce notation, set $B_1= B(\mathbf{0},1)$ and $B_r= B(\mathbf{0},r)$.
  \begin{quote}
    \begin{proof}[Proof of claim]
      Let $x\in B_r$. Then $|x |<r$ so $|x |/r<1$. Thus, $|x |/r\in B_1$ so
      it is in the image of $B_1$ under the map $T(r,\cdot)$.

      On the other hand, suppose $x\in T(r,B_1)$. Then $x =r y $ for some
      $ y \in B_1$. Then, since $| y |<1$, $|x |=r| y |<r$ so $x\in B_r$.
    \end{proof}
  \end{quote}

  From the claim, we see that $F(x)=|T(x,B(\mathbf{0},1))|$ which, by
  Problem 1, is nothing more that the polynomial $|B_1|x^n$. It is clear,
  from this equivalence, that $F$ is monotonically increasing: Take
  $x,y\in[0,\infty)$ such that $x<y$, then $x^n<y^n$ so
  \begin{equation}
    \label{eq:prep:1:7}
    F(x)=|B_1|x^n<|B_1|y^n=F(y).
  \end{equation}
  Thus, $F$ is monotonically increasing.

  In the argument above, since $F(x)=|B_1|x^n$ is a polynomial in
  $[0,\infty)$ (and polynomials are continuous on $\bbR$) $F$ is continuous
  on $[0,\infty)$.
\end{solution}

\begin{problem}
  Let $f\colon\bbR\to\bbR$ be a function. Let $C$ be the set of all points
  at which $f$ is continuous. Show that $C$ is a set of type $G_\delta$.
\end{problem}
\begin{solution}
  (Without much motivation) let us consider the collection of sets
  $\{E_k\}$ defined by
  \begin{equation}
    \label{eq:prep:1:8}
    E_k=\left\{\,x\in\bbR:
      \text{there exists $\delta>0$ such that $y,z\in B(x,\delta)$ implies $\left|f(y)-f(z)\right|<\frac{1}{k}$}\,\right\}.
  \end{equation}
  We claim that $C=\bigcap_{k=1}^\infty E_k$ and that each $E_k$ is open.
  \begin{solution}[Proof of claim]
    First, we demonstrate equality. $\subset$: Suppose $x\in C$. Then, by
    the definition of continuity, for every $\varepsilon>0$, there exists a
    $\delta>0$ such that $y\in B(x,\delta)$ implies
    $|f(x)-f(y)|<\delta$. In particular, for every $k$, there exists
    $\delta>0$ such that for $y\in B(x,\delta)$ the inequality
    $|f(x)-f(y)|<1/k$ holds. Thus, $x$ is in $\bigcap_{k=1}^\infty E_k$.

    $\supset$: On the other hand, suppose that
    $x\in\bigcap_{k=1}^\infty E_k$. Then, given $\varepsilon>0$, by the
    Archimedean property, there exists a positive integer $N$ such that
    $1/N<\varepsilon$. Then, since $x\in\bigcap_{k=1}^\infty E_k$,
    $x\in E_N$ so
    \begin{equation}
      \label{eq:prep:1:9}
      |f(x)-f(y)|<\frac{1}{N}<\varepsilon.
    \end{equation}
    Thus, $x$ is in $C$ and $C=\bigcap_{k=1}^\infty E_k$.

    All that remains to be shown is that the $E_k$ are open. But this is
    clear by the way we defined $E_k$ in \eqref{eq:prep:1:8}: Let
    $x\in E_k$, then there exists $\delta>0$ such that for any
    $y,z\in B(x,\delta)$, $|f(y)-f(z)|<1/k$; Let $x'\in B(x,\delta)$ and
    set $\delta'=\min\{|(x+\delta)-x'|,|(x-\delta)-x|\}$. Then, since
    $B(x',\delta')\subset B(x,\delta)$, for every $y,z\in B(x',\delta')$,
    we have $|f(y)-f(z)|<1/k$. Hence, $x'\in E_k$ for any
    $x'\in B(x,\delta)$ so $B(x,\delta)\subset E_k$.
  \end{solution}
  Since $C$ can be expressed as the countable intersection of open sets
  $E_k$, it follows that $C$ is a $G_\delta$ set.
\end{solution}
\begin{problem}
  Let $f\colon\bbR\to\bbR$ be a function. Is it true that if the sets
  $\left\{\,f=r\,\right\}$ are measurable for all $r\in\bbR$, then $f$ is
  measurable?
\end{problem}
\begin{solution}
  If $\left\{\,f=r\,\right\}$ are measurable for all $r\in\bbR$, it is not
  necessarily the case that $f$ is measurable. Consider the following
  construction: Let $E\subset(0,1)$ be an unmeasurable set.\footnote{It's
    construction does not concern us. The interested reader such direct
    their refer to Theorem 3.38 from \cite[Ch.\@ 3, p.\@
    57-58]{wheeden-zygmund} or Theorem 17 from \cite[Ch.\@ 2\S 7, p.\@
    48]{royden}.} Define a map $f\colon\bbR\to\bbR$ by
  \begin{equation}
    \label{eq:prep:1:11}
    f(x)=
    \begin{cases}
      x&\text{if $x\in\bbR\setminus((0,1)\setminus E)$},\\
      x+1&\text{if $x\in (0,1)\setminus E$.}
    \end{cases}
  \end{equation}
  By the definition, it is clear that $\left\{\,f=r\,\right\}$ is
  measurable and $\left|\left\{\,f=r\,\right\}\right|=0$ since
  $\{\,f=r\,\}$ contains at most two elements. However, the set
  $\left\{\,0<f<1\,\right\}=E$ is not measurable. Thus, $f$ is not
  measurable.
\end{solution}

\begin{problem}
  Let $\left\{f_k\right\}_{k=1}^\infty$ be a sequence of measurable
  functions on $\bbR$. Prove that the set
  $\left\{\,x:\text{$\lim_{k\to\infty} f_k(x)$ exists}\,\right\}$ is
  measurable.
\end{problem}
\begin{solution}
  By Theorem 4.12 from \cite[Ch.\@ 4, p.\@ 67]{wheeden-zygmund},
  $\liminf_{k\to\infty}f_k$ and $\limsup_{k\to\infty}f_k$ are
  measurable. By Theorem 4.7 from \cite[Ch.\@ 4, p.\@ 66]{wheeden-zygmund}
  \begin{equation}
    \label{eq:prep:1:12}
    \left\{\,\liminf_{k\to\infty} f_k<\limsup_{k\to\infty} f_k\,\right\}
  \end{equation}
  is measurable. Since
  \begin{equation}
    \label{eq:prep:1:13}
    \left\{\,\text{$\lim_{k\to\infty}f_k$ exists}\,\right\}=
    \left\{\,{\limsup_{k\to\infty}f_k=\liminf_{k\to\infty}f_k}\,\right\}=
    \bbR\setminus
    \left\{\,{\liminf_{k\to\infty} f_k<\limsup_{k\to\infty} f_k}\,\right\},
  \end{equation}
  by Theorem 3.17 from \cite[Ch.\@ 3, p.\@ 48]{wheeden-zygmund}, the set
  $\left\{\,\text{$\lim_{k\to\infty}f_k$ exists}\,\right\}$ is measurable.
  % In a fashion similar to that of Problem 4, consider the set collection
  % of sets $\{E_k\}$ given by
  % \begin{equation}
  %   \label{eq:prep:1:11}
  %       %   E_k= \left\{\, x\in\bbR:\text{there exists $N$ such that $m,n\geq N$
  %     implies $\left|f_n(x)-f_m(x)\right|<\frac{1}{k}$} \,\right\}.
  % \end{equation}
  % You can show that the $E_k$ are open and that
  % $\left\{\,x:\text{$\lim_{x\to\infty}f_k(x)$
  %   exists}\,\right\}=\bigcap_{k=1}^\infty E_k$. Then, since open sets
  % are measurable and, by Theorem 3.18 from \cite[Ch.\@ 3, p.\@
  % 48]{wheeden-zygmund}, the countable intersection of measurable sets is
  % measurable, $\left\{\,x:\text{$\lim_{x\to\infty}f_k(x)$
  %   exists}\,\right\}$ is measuable.
\end{solution}
\begin{problem}
  A real valued function $f$ on an interval $[a,b]$ is said to be
  \emph{absolutely continuous} if for every $\varepsilon>0$, there exists a
  $\delta>0$ such that for every finite disjoint collection
  $\left\{(a_k,b_k)\right\}_{k=1}^N$ of open intervals in $(a,b)$
  satisfying $\sum_{k=1}^Nb_k-a_k<\delta$, one has
  $\sum_{k=1}^N\left|f(b_k)-f(a_k)\right|<\varepsilon$. Show that an
  absolutely continuous function on $[a,b]$ is of bounded variation on
  $[a,b]$.
\end{problem}
\begin{solution}
  Suppose $f$ is absolutely continuous on $[a,b]$. Let $\varepsilon=
  1$. Then, there exists $\delta>0$ such that for every finite disjoint
  collection $\left\{(a_k,b_k)\right\}_{k=1}^N$ of open intervals in
  $(a,b)$ satisfying $\sum_{k=1}^Nb_k-a_k<\delta$, one has
  $\sum_{k=1}^N\left|f(b_k)-f(a_k)\right|<1$. Let
  $N=\lceil(b-a)/\delta\rceil$, that is, $N$ is the smallest integer
  greater than $(b-a)/\delta$, and consider the partition $\Gamma=\{x_k\}$
  where $x_k= a+k(b-a)/N$, for $k=0,\dotsc,N$. Then
  $x_k-x_{k-1}<(b-a)/N<\delta$ so, by Theorem 2.2(i) from \cite[Ch.\@ 2,
  p.\@ 19]{wheeden-zygmund}, we have $V[f;x_{k-1},x_k]<1$ for
  $k=0,\dotsc,N$. In follows by Theorem 2.2(ii) that
  \begin{equation}
    \label{eq:prep:1:14}
    V[f;a,b]=\sum_{k=1}^N V[f;x_{k-1},x_k]<N.
  \end{equation}
  Thus, $f$ is b.v.\@ on $[a,b]$.
\end{solution}

\begin{problem}
  Let $f$ be a continuous function from $[a,b]$ into $\bbR$. Let
  $\chi_{\{c\}}$ be the characteristic function of a singleton
  $\left\{c\right\}$, that is, $\chi_{\{c\}}(x)=0$ if $x\neq c$ and
  $\chi_{\{c\}}(c)=1$. Show that
  \[
    \int_a^b f d \chi_{\{c\}}=
    \begin{cases}
      0&\text{if $c\in(a,b)$,}\\
      -f(a)&\text{if $c=a$,}\\
      f(b)&\text{if $c=b$.}
    \end{cases}
  \]
\end{problem}
\begin{solution}
  The result follows quite easily from more sophisticated measure theoretic
  arguments. At this point, however, such language has not been discussed
  so we shall prove this using nothing but the definition of the
  Riemann--Stieltjes integral and properties thereof.

  Let us consider each case $c\in(a,b)$, $c=a$, and $c=b$ separately.

  Recall that the given a partition $\Gamma=\{x_0,\dotsc,x_m\}$ of $[a,b]$,
  the Riemann--Stieltjes sum of $f$ with respect to $\varphi$ is
  \begin{equation}
    \label{eq:prep:1:15}
    R_\Gamma=\sum_{k=1}^mf(\xi_k)[\varphi(x_k)-\varphi(x_{k-1})].
  \end{equation}
  The Riemann--Stieltjes integral is defined as the limit
  \begin{equation}
    \label{eq:prep:1:16}
    \int_a^b f\diff\varphi=\lim_{|\Gamma|\to 0} R_\Gamma
  \end{equation}
  if it exists.

  Suppose $c\in(a,b)$. Then, for any partition $\Gamma$ of $[a,b]$, either
  $c\in\Gamma$ or $c\notin\Gamma$. In the latter case, $R_\Gamma=0$. In the
  former case $c$ is one of the $x_k$, say $c=x_\ell$ for $0<\ell<m$. Then
  \begin{equation}
    \label{eq:prep:1:17}
    \begin{aligned}
      R_\Gamma&=\sum_{k=1}^mf(\xi_k)[\chi_{\{c\}}(x_k)-\chi_{\{c\}}(x_{k-1})]\\
      &=0+\dotsb+0+f(\xi_{\ell-1})-f(\xi_\ell)+0+\dotsb+0\\
      &=f(\xi_{\ell-1})-f(\xi_\ell).
    \end{aligned}
  \end{equation}
  Since $f$ is continuous, given $\varepsilon>0$ there exists $\delta>0$
  such that $|\xi_\ell-\xi_{\ell-1}|<\delta$ implies
  $|f(\xi_{\ell})-f(\xi_{\ell-1})|<\varepsilon$. It follows that the
  quantity in \eqref{eq:prep:1:17} approaches $0$ as $|\Gamma|$ approaches
  $0$. Therefore, $\int_a^b f\diff\chi_{\{c\}}=0$.

  Suppose $c=a$. Then, since any partition $\Gamma$ of $[a,b]$ must contain
  the point $a$, we have
  \begin{equation}
    \label{eq:prep:1:18}
    \begin{aligned}
      R_\Gamma
      &=\sum_{k=1}^mf(\chi_k)[\chi_{\{c\}}(x_k)-\chi_{\{c\}}(x_{k-1})]\\
      &
      \begin{aligned}
        =f(\xi_1)[\chi_{\{c\}}(x_1)-\chi_{\{c\}}(x_0)]&
        +f(\xi_2)[\chi_{\{c\}}(x_2)-\chi_{\{c\}}(x_1)]\\
        &+\dotsb+f(\xi_m)[\chi_{\{c\}}(x_m)-\chi_{\{c\}}(x_{m-1})]
      \end{aligned}\\
      &=-f(\xi_1)+0+\dotsb+0\\
      &=-f(\xi_1)
    \end{aligned}
  \end{equation}
  Taking the limit as $|\Gamma|\to 0$, $\xi_1\to a$ so, by continuity of
  $f$, $f(\xi_1)\to f(a)$. Thus, $\int_a^b f\diff\chi_{\{c\}}=-f(a)$.

  A similar argument to the one above shows that, if $c=b$, the
  Riemann--Stieltjes integral $\int_a^bf\diff\chi_{\{c\}}=f(b)$.
\end{solution}

%%% Local Variables:
%%% mode: latex
%%% TeX-master: "../MA544-Quals"
%%% End:

\subsubsection{Exam 1}
\setcounter{exercise}{0}
\setcounter{equation}{0}

I lost this exam. These are the questions I could recall explicitly. For
the first problem, we were asked to show that the Dichlet function
\(\indicate_\bbQ(x)\) is not Riemann integrable and prove something about
\(\bbQ\). For the second question, we were asked to show that the measure
of countable union of disjoint measurable sets \(\{E_n:n\in\bbN\}\), is
equal to the sum of their individual measures (or something to that
effect).
\begin{problem}
\end{problem}
% \begin{solution}
% \end{solution}

\begin{problem}
\end{problem}
% \begin{solution}
% \end{solution}

\begin{problem}
\hfill
\begin{enumerate}[label=(\roman*),noitemsep]
\item Show that if \(B_r=\left\{\,x\in\bbR^n:|x|<r\,\right\}\), then there
  exists a constant \(C\) such that \(|B_r|=Cr^n\).
  \\\\
  (\emph{Hint}: Think of \(B_r\) as \(\left\{\,rx:x\in B_1\,\right\}\).)
\item Let \(E\subseteq\bbR^n\) be a measurable set and let
  \(\varphi_E\colon\bbR^n\to\bbR\) be defined
  \(\varphi_E(x)=\bigl|E\cap B_{|x|}\bigr|\). Use part (i) to prove that
  \(\varphi_E\) is continuous.
\end{enumerate}
\end{problem}
\begin{solution}
\end{solution}

\begin{problem}
  Assume that \(f\colon[a,b]\to\bbR\) is of bounded variation on
  \([a,b]\). Prove that \(f\) is measurable.
\end{problem}
\begin{solution}
\end{solution}

%%% Local Variables:
%%% mode: latex
%%% TeX-master: "../MA544-Quals"
%%% End:

\subsubsection{Exam 2 Practice Problems}
\setcounter{exercise}{0}
\setcounter{equation}{0}

\begin{problem}
  Define for \( x \in\bbR^n\),
  \[
    f(x)=
    \begin{cases}
      |x|^{-(n+1)}&\text{if \(x\neq\mathbf{0}\),}\\
      0&\text{if \(x=\mathbf{0}\).}
    \end{cases}
  \]
  Prove that \(f\) is integrable outside any ball
  \(B(\mathbf{0},\varepsilon)\), and that there exists a constant \(C>0\)
  such that
  \[
    \int\limits_{\bbR^n\setminus B(\mathbf{0},\varepsilon)}f(x)\diff x
    \leq\frac{C}{\varepsilon}.
  \]
\end{problem}
\begin{solution}
  Danielli gave a wonderful solution to this problem by using spherical
  coordinates to compute the integral. However, she did not justify the use
  of polar coordinates, or even make it clear what exactly the meaning of
  \(\rmd x\) and \(rmd \sigma\) mean in this context. Here, we shed some
  light into this method and prove the polar decomposition of the Lebesgue
  measure.

  First, note that, given \(\varepsilon>0\), \(f(x)\neq 0\) for any
  \(x\in\bbR^n\setminus B_r(\mathbf{0})\). Now, define the map
  \(\Phi\colon\bbR^n\setminus\{\mathbf{0}\}\to(0,\infty)\times S^{n-1}\) by
  the rule \(\Phi(x)=(\|x\|,x/\|x\|)\). This map is smooth with a smooth
  inverse \(\Phi^{-1}(r,y)=ry\) and Jacobian
  \(\partial\Phi(x_1,\dotsc,x_n)/\partial(r,x)=\)
\end{solution}

\begin{problem}
  Let \(\left\{f_k\right\}\) be a sequence of nonnegative measurable
  functions on \(\bbR^n\), and assume that \(f_k\) converges pointwise
  almost everywhere to a function \(f\). If
  \[
    \int_{\bbR^n} f=\lim_{k\to\infty}\int_{\bbR^n} f_k<\infty,
  \]
  show that
  \[
    \int_E f=\lim_{k\to\infty}\int_E f_k
  \]
  for all measurable subsets \(E\) of \(\bbR^n\). Moreover, show that this
  is not necessarily true if
  \(\int_{\bbR^n} f=\lim_{k\to\infty} f_k=\infty\).
\end{problem}
\begin{solution}
\end{solution}

\begin{problem}
  Assume that \(E\) is a measurable set of \(\bbR^n\), with
  \(|E|<\infty\). Prove that a nonnegative function \(f\) defined on \(E\)
  is integrable if and only if
  \[
    \sum_{k=0}^\infty\left|\left\{\, x \in E:f( x )\geq
        k\,\right\}\right|<\infty.
  \]
\end{problem}
\begin{solution}
\end{solution}

\begin{problem}
  Suppose that \(E\) is a measurable subset of \(\bbR^n\), with
  \(|E|<\infty\). If \(f\) and \(g\) are measurable functions on \(E\),
  define
  \[
    \rho(f,g)=\int_E\frac{|f-g|}{1+|f-g|}.
  \]
  Prove that \(\rho(f_k,f)\to 0\) as \(k\to\infty\) if and only if \(f_k\)
  converges to \(f\) as \(k\to\infty\).
\end{problem}
\begin{solution}
\end{solution}

\begin{problem}
  Define the \emph{gamma function} \(\Gamma\colon\bbR^+\to\bbR\) by
  \[
    \Gamma(y)=\int_0^\infty e^{-u}u^{y-1}\diff u,
  \]
  and the \emph{beta function} \(\beta\colon\bbR^+\times\bbR^+\to\bbR\) by
  \[
    \beta(x,y)=\int_0^1 t^{x-1}(1-t)^{y-1}\diff t.
  \]
  \begin{enumerate}[label=(\alph*)]
  \item Prove that the definition of the gamma function is well-posed,
    i.e., the function \(u\mapsto e^{-u}u^{y-1}\) is in \(L(\bbR^+)\) for
    all \(y\in\bbR^+\).
  \item Show that
    \[
      \beta(x,y)=\frac{\Gamma(x)\Gamma(y)}{\Gamma(x+y)}.
    \]
  \end{enumerate}
\end{problem}
\begin{solution}
\end{solution}

\begin{problem}
  Let \(f\in L(\bbR^n)\) and for \(\mathbf{h}\in\bbR^n\) define
  \(f_{\mathbf{h}}\colon\bbR^n\to\bbR\) be
  \(f_{\mathbf{h}}( x )= f( x -\mathbf{h})\). Prove that
  \[
    \lim_{\mathbf{h}\to\mathbf{0}}\int_{\bbR^n}\left|f_{\mathbf{h}}-f\right|=0.
  \]
\end{problem}
\begin{solution}
\end{solution}

\begin{problem}
\begin{enumerate}[label=(\alph*)]
\item If \(f_k,g_k,f,g\in L(\bbR^n)\), \(f_k\to f\) and \(g_k\to g\) a.e.\@
  in \(\bbR^n\), \(|f_k|\leq g_k\) and
  \[
    \int_{\bbR^n}g_k\longrightarrow\int_{\bbR^n}g,
  \]

  prove that
  \[
    \int_{\bbR^n} f_k\longrightarrow\int_{\bbR^n}f.
  \]
\item Using part (a) show that if \(f_k,f\in L(\bbR^n)\) and \(f_k\to f\)
  a.e.\@ in \(\bbR^n\), then
  \[
    \int_{\bbR^n}|f_k-f|\longrightarrow 0\qquad\text{as \(k\to\infty\)}
  \]
  if and only if
  \[
    \int_{\bbR^n}|f_k|\longrightarrow\int_{\bbR^n}|f|\qquad\text{as
      \(k\to\infty\)}.
  \]
\end{enumerate}
\end{problem}
\begin{solution}
\end{solution}

%%% Local Variables:
%%% mode: latex
%%% TeX-master: "../MA544-Quals"
%%% End:

\subsubsection{Exam 2 (2010)}
\setcounter{exercise}{0}
\setcounter{equation}{0}

\begin{problem}
  Suppose \(f\in L^1(\bbR^n)\). Show that for every \(\varepsilon>0\) there
  exists a ball \(B\), centered at the origin, such that
  \[
    \int\limsclap{\bbR^n\setminus B}|f|<\varepsilon.
  \]
  \\\\
  \emph{Hint}: Use the monotone convergence theorem.
\end{problem}
\begin{solution}
  Consider the sequence of functions \(\{f_n(x):n\in\bbN\}\) where
  \[
    f_n(x)=|f(x)|\indicate_{B(\mathbf{0},n)}(x).
  \]
  Then, \(f_n\uparrow |f|\) so by the monotone convergence theorem, given
  \(\varepsilon>0\), there exists an index \(N\in\bbN\) such that
  \(n\geq N\) implies
  \begin{align*}
    \int_{\bbR^n}|f(x)|\diff x-\int_{\bbR^n}f_n(x)\diff x
    &=\int_{\bbR^n}|f(x)|-|f(x)|\indicate_{B(\mathbf{0},n)}\diff x\\
    &=\int\limsclap{\bbR^n\setminus B(\mathbf{0},n)}|f(x)|\diff x\\
    &<\varepsilon.
  \end{align*}
  Let \(B=B(\mathbf{0},N+1)\).
\end{solution}
\begin{problem}
  \hfill
  \begin{enumerate}[label=(\alph*)]
  \item Prove the following generalization of \emph{Chebyshev's
      inequality}: Let \(0<p<\infty\) and \(E\subseteq\bbR^n\) be
    measurable. Assume that \(|f|^p\in L^1(E)\). Then
    \[
      m\bigl\{\,x\in E:f(x)>\alpha\,\bigr\}
      \leq\frac{1}{\alpha^p}\int\limsclap{\left\{\,f>\alpha\,\right\}}f^p,
    \]
    for \(\alpha>0\).
  \item Let \(p\), \(E\), and \(f\) be as in part (a). In addition, assume
    that \(\{f_k\}\) is a sequence such that \(\int_E|f_k-f|^p\to 0\) as
    \(k\to\infty\). Show that \(f_k\to f\) in measure on \(E\).
    \\\\
    Recall that \(f_k\to f\) in measure on \(E\) if and only if for every
    \(\varepsilon>0\)
    \[
      \lim_{k\to\infty}m\bigl\{\,x\in
      E:|f_k(x)-f(x)|>\varepsilon\,\bigr\}=0.
    \]
\end{enumerate}
\end{problem}
\begin{solution}
  Part (a) is almost trivial. Let
  \[
    E_\alpha=\bigl\{\,x\in E:f(x)>\alpha\,\bigr\}.
  \]
  Then, \(|f|^p\geq \alpha^p\) for all \(x\in E_\alpha\). Thus,
  \[
    \int_{E_\alpha}\alpha^p\diff x=\alpha^pm(E_\alpha)\leq\int_{E_\alpha}|f|^p\diff x,
  \]
  as was to be shown.

  Part (b) follows directly from Chebyshev's inequality, as
  \[
    \lim_{n\to\infty}m(E_\varepsilon) <\lim_{n\to\infty}
    \left[\frac{1}{\varepsilon^p}\int_{E_\varepsilon}|f_n(x)-f(x)|^p\diff
      x\right]=
    \frac{1}{\varepsilon^p}\lim_{n\to\infty}\int_{E_\varepsilon}|f_n(x)-f(x)|\diff
    x=0,
  \]
  where \(E_\varepsilon=\bigl\{\,x\in
  E:|f_n(x)-f(x)|>\varepsilon\,\bigr\}\).
\end{solution}

\begin{problem}
  Let \(f\in L^1(\bbR)\), and define
  \[
    F(\xi)=\int_{\bbR} f(x)\cos(2\pi x\xi)\diff x.
  \]
  Prove that \(F\) is continuous and bounded on \(\bbR\).
\end{problem}
\begin{solution}
  It is easy to see that \(F\) is bounded as \(|{\cos(2\pi x\xi)}|<1\) for
  all \(\chi\in\bbR\) so
  \[
    |F(\xi)|= \left| \int_\bbR f(x)\cos(2\pi x\xi)\diff x \right \leq
    \left|\int_\bbR f(x)\diff x\right|\leq \|f\|_1.
  \]

  To see that \(F\) is in fact continuous, note that since \(\cos(2\pi
  x\xi)\) is continuous, as a function of \(\xi\), given \(\varepsilon>0\)
  there exist \(\delta>0\) such that \(|\xi-\chi|<\delta\) implies
  \[
    |{\cos(2\pi x\xi)-\cos(2\pi x\chi)}|<\frac{\varepsilon}{\|f\|_1}.
  \]
  Thus, for \(|\xi-\chi|<\delta\), we have
  \begin{align*}
    |F(\xi)-F(\xi)|
    &=\left|
      \int_\bbR f(x)\cos(2\pi x\xi)\diff x
      -\int_\bbR f(x)\cos(2\pi x\chi)\diff x
      \right|\\
    &\leq
      \int_\bbR\Bigl|f(x)\bigl(\cos(2\pi x\xi)-\cos(2\pi x\chi)\bigr)\Bigr|\diff x
    \\
    &\leq \frac{\varepsilon}{\|f\|_1}\int_\bbR |f(x)|\diff x\\
    &=\varepsilon.
  \end{align*}
  Thus, \(F\) is continuous.
\end{solution}

\begin{problem}
  Use repeated integration techniques to prove that
  \[
    \int_{\bbR^n} \rme^{-|x|^2}\diff x=\pi^{n/2}.
  \]
  \\\\
  \emph{Hint}: Start from the case \(n=1\) by using the polar coordinates
  in
  \[
    \left[\int_{\bbR} \rme^{-x^2}\diff x\right]^2=%
    \left[\int_{\bbR} \rme^{-x^2}\diff x\right]%
    \left[\int_{\bbR} \rme^{-x^2}\diff y\right]%
  \]
\end{problem}
\begin{solution}
\end{solution}

%%% Local Variables:
%%% mode: latex
%%% TeX-master: "../MA544-Quals"
%%% End:

\section{MA 544 - Midterm 2}
\begin{problem}
Assume that $f\in L^1(\bbR^n)$. Show that for every $\varepsilon>0$ there
exists a ball $B$, centered at the origin, such that
\[
\int_{\bbR^n\minus B}|f|<\varepsilon.
\]
\end{problem}
\begin{proof}
Recall that $f\in L^1(\bbR^n)$ if and only if $|f|\in
L^1(\bbR^n)$. Let $B_k\coloneqq B(\mathbf{0},k)$ for $k\in\bbN$ and
$\chi_{B_k}$ be the indicator function associated with $B_k$. Then, the
sequence of maps $\left\{|f_k|\right\}$ defined $f_k\coloneqq f\chi_{B_k}$
converge pointwise to $|f_k|$. Since $|f|\in L^1(\bbR^n)$, by the monotone
convergence theorem, we have
\begin{equation}
\label{eq:monotonicity-2-1}
\int_{\bbR^n} |f_k|=\int_{B_k}|f|\longrightarrow\int_{\bbR^n}|f|.
\end{equation}
But this means, exactly, that for every $\varepsilon>0$ there exists
sufficiently large $N\in\bbN$ such that
\begin{equation}
  \label{eq:desired-inequality-2-1}
\begin{aligned}
\varepsilon&>\left|\int_{\bbR^n}|f_k|-\int_{\bbR^n}|f|\right|\\
&=-\int_{\bbR^n}|f_k|+\int_{\bbR^n}|f|\\
&=-\int_{\bbR^n}|f|+\int_{\bbR^n}|f|\\
&=-\int_{B_k}|f|+\int_{\bbR^n}|f|\\
&=\int_{\bbR^n\minus B_k}|f|
\end{aligned}
\end{equation}
as desired.
\end{proof}

\begin{problem}
Let $f\in L^1(E)$, and let $\{E_j\}$ be a countable collection of pairwise
disjoint measurable subsets of $E$, such that $E=\bigcup_{j=1}^\infty
E_j$. Prove that
\[
\int_E f=\sum_{j=1}^\infty\int_{E_j}f.
\]
\end{problem}
\begin{proof}
First, since the $E_j$'s are pairwise disjoint, by Theorem 3.23, we have
\begin{equation}
\label{eq:disjoint-measure-2-2}
|E|=\sum_{j=1}^\infty|E_j|.
\end{equation}
Let $\chi_{E_j}$ be the characteristic function of the subset $E_j$ of
$E$ and define $f_j\coloneqq f\chi_{E_j}$ for $j\in\bbN$. Note that, since
both $f$ and $\chi_{E_j}$ are measurable on $E$, $f_j$ is
measurable on $E$ and $\sum_{j=1}^\infty f_j=f$. Moreover, since
$E_j\subset E$, by monotonicity of the integral we have
\begin{equation}
\label{eq:monotonicity-2-2}
\int_{E} f=
\int_{E_j} f+\int_{E\minus E_j}f=
\int_E f_j+\int_{E\minus E_j}f.
\end{equation}
Hence, because the $E_j$'s are disjoint $(E\minus E_k)\minus
E_\ell=(E\minus E_\ell)\minus E_k$ so
\begin{equation}
\label{eq:desired-sum-2}
\int_E f=\sum_{j=1}^\infty\int_E f_j=\sum_{j=1}^\infty\int_{E_j}f
\end{equation}
as desired.
\end{proof}

\begin{problem}
Let $\{f_k\}$ be a family in $L^1(E)$ satisfying the following property:
For any $\varepsilon>0$ there exits $\delta>0$ such that $|A|<\delta$
implies
\[
\int_A |f_k|<\varepsilon
\]
for all $k\in\bbN$. Assume $|E|<\infty$, and $f_k(x)\to f(x)$ as
$k\to\infty$ for a.e.\@ $x\in E$. Show that
\[
\lim_{k\to\infty}\int_E f_k=\int_E f.
\]
(\emph{Hint:} Use Egorov's theorem.)
\end{problem}
\begin{proof}

\end{proof}

\begin{problem}
Let $I\coloneqq[0,1]$, $f\in L^1(I)$, and define $g(x)\coloneqq\int_x^1
t^{-1}f(t)\diff t$ for $x\in I$. Prove that $g\in L^1(I)$ and
\[
\int_I g=\int_I f.
\]
\end{problem}
\begin{proof}
\end{proof}

%%% Local Variables:
%%% mode: latex
%%% TeX-master: "../MA544-Quals"
%%% End:

\subsubsection{Final Exam Practice Problems}
\setcounter{exercise}{0}

\begin{problem}
Suppose $f\in L^1(\bbR^n)$ and that $x$ is a point in the Lebesgue set of
$f$. For $r>0$, let
\[
A(r)=\frac{1}{|r|^n}\int_{B(0,r)}|f(\bfx-\bfy)-f(\bfx)|\diff\bfy.
\]
Show that:
\begin{enumerate}[label=(\alph*),noitemsep]
\item $A(r)$ is a continuous function of $r$, and $A(r)\to 0$ as $r\to 0$;
\item there exists a constant $M>0$ such that $A(r)\leq M$ for all $r>0$.
\end{enumerate}
\end{problem}
\begin{solution}
(a) Without loss of generality, we may assume $r<s$. Then, we want to show
that as $r\to s$, the quantity
\[
|A(s)-A(r)|\longrightarrow 0.
\]
Set $F(\bfy)= |f(\bfx-\bfy)-f(\bfx)|$ and consider said quantity
\begin{align*}
|A(s)-A(r)|
&=\left|
\frac{1}{|s|^n}\int_{B_s}F(\bfy)\diff\bfy
-
\frac{1}{|r|^n}\int_{B_r}F(\bfy)\diff\bfy
\right|\\
&=\left|
\frac{1}{|s|^n}\int_{B_s\smallsetminus B_r}F(\bfy)\diff\bfy+
\frac{1}{|s|^n}\int_{B_r}F(\bfy)\diff\bfy-
\frac{1}{|r|^n}\int_{B_r}F(\bfy)\diff\bfy
\right|\\
&=\left|
\frac{1}{|s|^n}\int_{B_s\smallsetminus B_r}F(\bfy)\diff\bfy
+\left(\frac{1}{|s|^n}-\frac{1}{|r|^n}\right)\int_{B_r}F(\bfy)\diff\bfy
\right|\\
&\leq
\underbrace{\frac{1}{|s|^n}\int_{B_s\smallsetminus B_r}F(\bfy)\diff\bfy}_{I_1}
+\underbrace{\left(\frac{1}{|s|^n}-\frac{1}{|r|^n}\right)\int_{B_r}F(\bfy)\diff\bfy}_{I_2}.
\end{align*}
Hence, we must show that the quantities $I_1,I_2\to 0$ as $r\to s$.

To see that $A(r)\to 0$ as $r\to 0$, note that $x$ is a point of the
Lebesgue set of $f$ and that
\[
0=\lim_{B_r\searrow\bfx}\frac{1}{|B_1||r|^n}\int_{B_r}|f(\bfy)-f(\bfx)|\diff\bfy=\frac{1}{|B_1|}\lim_{B_r\searrow\bfx}\frac{1}{|r|^n}\int_{B_r}|f(\bft)-f(\bfx)|\diff\bft=\lim_{r\to
0}A(r).
\]
by making the change of variables $\bft=\bfx-\bfy$.
\\\\
(b)
\end{solution}

\begin{problem}
Let $E\subset\bbR^n$ be a measurable set, $1\leq n<\infty$. Assume
$\{f_k\}$ is a sequence in $L^p(E)$ converging pointwise a.e.\@ on $E$ to a
function $f\in L^p(E)$. Prove that
\[
\|{f_k-f}\|_p\longrightarrow 0
\]
if and only if
\[
\|{f_k}\|_p\longrightarrow\|f\|_p
\]
as $k\to\infty$.
\end{problem}
\begin{solution}
\end{solution}

\begin{problem}
Let $1<p<\infty$, $f\in L^p(E)$, $g\in L^{p'}(E)$.
\begin{enumerate}[label=(\alph*),noitemsep]
\item Prove that $f*g\in C(\bbR^n)$.
\item Does this conclusion continue to be valid when $p=1$ and $p=\infty$?
\end{enumerate}
\end{problem}
\begin{solution}
\end{solution}

\begin{problem}
Let $f\in L(\bbR)$, and let $F(t)=\int_{\bbR}f(x)\cos(tx)dx$.
\begin{enumerate}[label=(\alph*),noitemsep]
\item Prove that $F(t)$ is continuous for $t\in\bbR$.
\item Prove the following
  \href{https://en.wikipedia.org/wiki/Riemann–Lebesgue_lemma}{\emph{Riemann--Lebesgue
      lemma}}:
\[
\lim_{t\to\infty}F(t)=0.
\]
\end{enumerate}
\end{problem}
\begin{solution}
\end{solution}

\begin{problem}
Let $f$ be of bounded variation on $[a,b]$, $-\infty<a<b<\infty$. If
$f=g+h$, with $g$ absolutely continuous and $h$ singular. Show that
\[
\int_a^b\varphi \diff f=\int_a^b\varphi f'dx+\int_a^b\varphi \diff h
\]
for all functions $\varphi$ continuous on $[a,b]$.
\end{problem}
\begin{solution}
\end{solution}

%%% Local Variables:
%%% mode: latex
%%% TeX-master: "../MA544-Quals"
%%% End:

\subsection{Final Exam 2010}
\begin{problem}
Suppose that $f\in L^1(\bbR^n)$, and that $\bfx$ is a point in the Lebesgue
set of $f$. For $r>0$, let
\[
A(r)\coloneqq \frac{1}{r^n}\int_{B_r}|f(\bfx-\bfy)-f(\bfx)|\diff\bfy,
\]
where $B_r\coloneqq B(\mathbf{0},r)$.
\\\\
Show that
\begin{enumerate}[label=(\alph*)]
\item $A(r)$ is a continuous function of $r$, and $A(r)\to 0$ as $r\to 0$.
\item There exists a constant $M>0$ such that $A(r)\leq M$ for all $r>0$.
\end{enumerate}
\end{problem}
\begin{proof}
(a)
\\\\
(b)
\end{proof}

\begin{problem}
Let $E\subset\bbR^n$ be a measurable set, $1\leq p<\infty$. assume that
$\left\{ f_k \right\}$ is a sequence in $L^p(E)$ converging pointwise
a.e.\@ on $E$ to a function $f\in L^p(E)$. Prove that
\[
\|f_k-f\|_p\longrightarrow 0\iff
\|f_k\|_p\longrightarrow\|f\|_p
\]
\emph{Hint}: To prove one of the implications, you can use the following
fact without proving it:
\[
\left|
\frac{a-b}{2}
\right|
\leq
\frac{|a|^p+|b|^p}{2}
\]
for all $a,b\in\bbR$.
\end{problem}
\begin{proof}
\end{proof}

\begin{problem}
Let $0<p<q<r\leq\infty$, $E\subset\bbR^n$ be a measurable set. Show that
each $f\in L^q(E)$ is the sum of a function $g\in L^p(E)$ and a function
$h\in L^r(E)$.
\end{problem}
\begin{proof}
\end{proof}

\begin{problem}
Prove that $f\colon[a,b]\to\bbR$ is Lipschitz continuous if and only if $f$
is absolutely continuous and there exists a constant $M>0$ such that
$|f'|<M$ a.e.\@ on $[a,b]$.
\end{problem}
\begin{proof}
\end{proof}

\begin{problem}
Let $1<p<\infty$, $f\in L^p(\bbR^n)$, $g\in L^{p'}(\bbR^n)$.
\begin{enumerate}[label=(\alph*)]
\item Prove that $f*g\in C(\bbR^n)$.
\item Does this conclusion continue to be valid when $p=1$ or $p=\infty$?.
\end{enumerate}
\end{problem}
\begin{proof}
\end{proof}

%%% Local Variables:
%%% mode: latex
%%% TeX-master: "../MA544-Quals"
%%% End:

\subsection{Final Exam}

%% Local Variables:
%%% mode: latex
%%% TeX-master: "../MA544-Quals"
%%% End:


%%% Quals
%% Danielli
% \section{MA 544 Past Quals}
\subsection{Danielli: Winter 2012}
\setcounter{exercise}{0}
\setcounter{equation}{0}

\begin{problem}
  Let \(f(x,y)\), \(0\leq x,y\leq 1\), satisfy the following conditions:
  for each \(x\), \(f(x,y)\) is an integrable function of \(y\), and
  \(\partial f(x,y)/\partial x\) is a bounded function of \((x,y)\). Prove
  that \(\partial f(x,y)/\partial x\) is a measurable function of \(y\) for
  each \(x\) and
  \[
    \frac{\rmd}{\rmd x}\int_0^1f(x,y)\diff y= \int_0^1\frac{\partial
      f(x,y)}{\partial x}\diff y.
  \]
\end{problem}
\begin{solution}
  The end points can be dealt with separately. Fix a point \(x_0\in(0,1)\)
  and consider the sequence of measurable functions \(\{f_n'\}\) where
  \[
    f_n'(y)=\frac{f(x_0+h_n,y)-f(x_0,y)}{h_n}
  \]
  where \(\{h_n\}\) is a sequence of numbers converging to \(0\). Since
  \(f\) is differentiable as a function of \(x\), the sequence
  \(\{f_n'(x_0,y)\}\) converges to \(\partial f/\partial x(x_0,y)\). Now,
  since \(|\partial f/\partial x(x,y)|\leq M\) for some \(M\in\bbR^+\) for
  all \((x,y)\in[0,1]\times[0,1]\), by the bounded convergence theorem
  \begin{align*}
    \lim_{n\to\infty}\int_0^1 f'_n(y)\diff y
    &=\int_0^1\lim_{n\to\infty} f'_n(y)\diff y\\
    &=\int_0^1 \frac{\partial f(x_0,y)}{\partial x}\diff y.
  \end{align*}
\end{solution}

\begin{problem}
  Let \(f\) be a function of bounded variation on \([a,b]\),
  \(-\infty<a<b<\infty\). If \(f=g+h\), with \(g\) absolutely continuous
  and \(h\) singular, show that
  \[
    \int_a^b\varphi\diff f=\int_a^b\varphi f'\diff x+\int_a^b\varphi\diff
    h.
  \]
  \\\\
  \emph{Hint}: A function \(h\) is said to be singular if \(h'=0\).
\end{problem}
\begin{solution}
\end{solution}

\begin{problem}
  Let \(E\subseteq\bbR\) be a measurable set, and let \(K\) be a measurable
  function on \(E\times E\). Assume that there exists a positive constant
  \(C\) such that
  \[
    \label{eq:jan-12:1}
    \tag{\(\bigstar\)}
    \int_E K(x,y)\diff x\leq C
  \]
  for a.e.\@ \(y\in E\), and
  \[
    \label{eq:jan-12:2}
    \tag{\(\clubsuit\)}
    \int_E K(x,y)\diff y\leq C
  \]
  for a.e.\@ \(x\in E\).

  Let \(1<p<\infty\), \(f\in L^p(E)\), and define
  \[
    T_f(x)=\int_E K(x,y)f(y)\diff y.
  \]
  \begin{enumerate}[label=(\alph*),noitemsep]
  \item Prove that \(T_f\in L^p(E)\) and
    \[
      \label{eq:jan-12:3}
      \tag{\(\spadesuit\)}
      \|T_f\|_p\leq C\|f\|_p.
    \]
  \item Is \eqref{eq:jan-12:3} still valid if \(p=1\) or \(\infty\)? If so,
    are assumptions \eqref{eq:jan-12:1} and \eqref{eq:jan-12:2} needed?
  \end{enumerate}
\end{problem}
\begin{solution}
\end{solution}

\begin{problem}
  Let \(f\) be a nonnegative measurable function on \([0,1]\) satisfying
  \[
    \label{eq:jan-12:4}%
    \tag{\(\blacklozenge\)}%
    m\left\{\,x\in[0,1]:f(x)>\alpha\,\right\}<\frac{1}{1+\alpha^2}
  \]
  for \(\alpha>0\).
  \begin{enumerate}[label=(\alph*),noitemsep]
  \item Determine values of \(p\in[1,\infty)\) for which \(f\in L^p[0,1]\).
  \item If \(p_0\) is the minimum value of \(p\) for which \(p\) may fail
    to be in \(L^p\), give an example of a function which satisfies
    \eqref{eq:jan-12:4}, but which is not in \(L^{p_0}[0,1]\).
\end{enumerate}
\end{problem}
\begin{solution}
\end{solution}

%%% Local Variables:
%%% mode: latex
%%% TeX-master: "../MA544-Quals"
%%% End:

\subsection{Danielli: Summer 2011}
\setcounter{exercise}{0}
\setcounter{equation}{0}

\begin{problem}
  Let \(f\in L^1(\bbR)\), and let
  \(\hat f(x)=\int_{\bbR} f(t)\cos(xt)\diff t\).
  \begin{enumerate}[label=(\alph*),noitemsep]
  \item Prove that \(\hat f(x)\) is continuous for \(x\in\bbR\).
  \item Prove the following \emph{Riemman--Lebesgue lemma}:
    \[
      \lim_{x\to\infty}\hat f(x)=0.
    \]
  \end{enumerate}
  \emph{Hint}: Start by proving the statement for \(f=\indicate_{[a,b]}\).
\end{problem}
\begin{solution}
  For part (a): let \(\varepsilon>0\) be given. Then, since \(\cos(xt)\) is
  continuous there exists \(\delta'>0\) such that \(|x-y|<\delta\) implies
  \[
    |\cos(xt)-\cos(yt)|<\frac{\varepsilon}{\|f\|_1}.
  \]
  Now, let \(\delta=\delta'\). Then we have
  \begin{align*}
    |\hat f(x)-\hat f(y)|
    &=\left|
      \int_\bbR f(t)\cos(xt)\diff t
      -\int_\bbR f(t)\cos(yt)\diff t
      \right|\\
    &\leq\int_\bbR|f(t)||\cos(xt)-\cos(yt)|\diff t\\
    &<\frac{\varepsilon}{\|f\|_1}\int_\bbR|f(t)|\diff t\\
    &=\frac{\varepsilon}{\|f\|_1}\|f\|_1\\
    &=\varepsilon.
  \end{align*}
  Since this can be done for any \(x\in\bbR\), \(\hat f\) is continuous on
  \(\bbR\).

  For part (b): since simple functions are dense in \(L^1(\bbR)\), \(f\)
  there exists a sequence of simple functions \(\{s_n\}\), \(n\in\bbN\),
  such that \(\int_\bbR s_n\to\|f\|_1\). Therefore, it suffices to prove
  the result for characteristic functions. Let \(f=\indicate_{[a,b]}\) and
  consider the limit
  \[
    \lim_{x\to\infty}\hat f(x)=\lim_{x\to\infty}\int_\bbR f(t)\cos(xt)\diff t.
  \]
  Since \(f=\indicate_{[a,b]}\), we have
  \begin{align*}
    \lim_{x\to\infty}\int_\bbR f(t)\cos(xt)\diff t
    &=\lim_{x\to\infty}\int_a^b\cos(xt)\diff t\\
    &=\lim_{x\to\infty}\biggl[\frac{1}{x}(\sin(xa)-\sin(xb))\biggr]\\
    &=\lim_{x\to\infty}\biggl[\frac{\sin(xa)}{x}-\frac{\sin(xb)}{x}\biggr]\\
    &=\biggl[\lim_{x\to\infty}\frac{\sin(xa)}{x}\biggr]
      -\biggl[\lim_{x\to\infty}\frac{\sin(xb)}{x}\biggr]\\
    &=1-1\\
    &=0,
  \end{align*}
  as we set out to show.
\end{solution}

\begin{problem}
  \hfill
  \begin{enumerate}[label=(\alph*),noitemsep]
  \item Suppose that \(f_k,f\in L^2(E)\), with \(E\) a measurable set, and
    that
    \[
      \label{eq:aug-11:1}
      \tag{\(\bigstar\)}
      \int_E f_kg\too\int_E fg
    \]
    as \(k\to\infty\) for all \(g\in L^2(E)\). If, in addition,
    \(\|f_k\|_2\to\|f\|_2\) show that \(f_k\) converges to \(f\) in
    \(L^2\), i.e., that
    \[
      \int_E|f-f_k|^2\too 0
    \]
    as \(k\to\infty\).
  \item Provide an example of a sequence \(f_k\) in \(L^2\) and a function
    \(f\) in \(L^2\) satisfying \eqref{eq:aug-11:1}, but such that \(f_k\)
    does \emph{not} converge to \(f\) in \(L^2\).
  \end{enumerate}
\end{problem}
\begin{solution}
  For part (a): expand the limit
  \begin{gather}
    \label{eq:aug-11:2-a}
    \begin{aligned}
      \lim_{n\to\infty}\int_E|f-f_n|^2\diff x
      &=\lim_{n\to\infty}\biggl[\int_E\bigl(|f|^2-2|ff_n|+|f|_n^2\bigr)\diff x\biggr]\\
      &=\lim_{n\to\infty}\biggl[\|f_n\|_2+\|f\|_2-2\int_E ff_n\diff x\biggr]\\
      &=\lim_{n\to\infty}\|f_n\|_2+\lim_{n\to\infty}\|f\|_2-2\lim_{n\to\infty}\int_E
      ff_n\diff x.
    \end{aligned}
  \end{gather}
  Since
  \[
    \int_Ef_ng\diff x\too \int_E fg\diff x
  \]
  for every \(g\in L^p(E)\),
  \[
    \int_E f_nf\diff x\too \int_Ef^2\diff x={\|f\|_2}^{\!2}.
  \]
  Moreover, \(\|f_n\|_2\to\|f\|_2\) so the limit in \eqref{eq:aug-11:2-a}
  converges to
  \[
    \lim_{n\to\infty}\|f_n\|_2+\lim_{n\to\infty}\|f\|_2-2\lim_{n\to\infty}\int_E
    ff_n\diff x
    =%
    \|f\|_2+\|f\|_2-2\|f\|_2=0
  \]
  as \(n\to\infty\).

  For part (b), consider the sequence \(\{f_n\}\), \(n\in\bbN\), where
  \(f_n(x)=\log(n)\exp(-nx)\). Then, we claim that
  \(f_n\xrightarrow{L^2[0,1]}0\), but that \(f_n\nrightarrow 0\)
  pointwise. To see the former, first note that
  \[
    \begin{aligned}
      \lim_{n\to\infty}\biggl[\int_0^1 f_n(x)\diff x\biggr]
      &=\lim_{n\to\infty}\biggl[\int_0^1\log(n)\exp(-nx)\diff x\biggr]\\
      &=\lim_{n\to\infty}\biggl[\log(n)\exp(-nx)\bigr|_0^1\biggr]\\
      &=\lim_{n\to\infty}\biggl[
      \frac{1}{n}\log(n)-\frac{1}{n}\log(n)\exp(-n)
      \biggr]\\
      &=\lim_{n\to\infty}\left[%
        \biggl(\frac{1-\exp(-n)}{n}\biggr)\log(n)%
      \right]\\
      &=0.
    \end{aligned}
  \]
  However, \(f_n\) does not converge to \(0\) a.e.\@ since, for
  \(x=0\) there exist no \(N\in\bbN\) such that
  \[
    |\log(n)|<1.
  \]
  for all \(n\geq N\).
\end{solution}

\begin{problem}
  A bounded function \(f\) is said to be of bounded variation on \(\bbR\)
  if it is of bounded variation on any finite subinterval \([a,b]\), and
  moreover \(A\defeq\sup_{a,b}V[a,b;f]<\infty\). Here, \(V[a,b;f]\) denotes the
  total variation of \(f\) over the interval \([a,b]\). Show that:
\begin{enumerate}[label=(\alph*),noitemsep]
\item \(\displaystyle\int_{\bbR}|f(x+h)-f(x)|\diff x\leq A|h|\) for all
  \(h\in\bbR\).
  \\\\
  \emph{Hint}: For \(h>0\), write
  \[
    \int_{\bbR} |f(x+h)-f(x)|\diff x=
    \sum_{n=-\infty}^\infty\int_{nh}^{(n+1)h}|f(x+h)-f(x)|\diff x.
  \]
\item
  \(\displaystyle\left|\int_{\bbR} f(x)\varphi'(x)\diff x\right|\leq A\),
  where \(\varphi\) is any function of class \(C^1\), of bounded variation,
  compactly supported, with \(\sup_{x\in\bbR}|\varphi(x)|\leq 1\).
\end{enumerate}
\end{problem}
\begin{solution}
  For part (a), it suffices to consider only positive \(h\) as, making the
  change of variables \(u=x+h\) yields
  \[
    \int_{\bbR} |f(u)-f(u-h)|\diff u=
    \int_{\bbR} |f(u+(-h))-f(u)|\diff u
  \]
  where \(-h\) is positive (and letting \(h=0\), we have a trivial
  inequality). Now, taking the hint, write
  \[
    \int_{\bbR} |f(x+h)-f(x)|\diff x=%
    \sum_{n=-\infty}^\infty\int_{nh}^{(n+1)h}|f(x+h)-f(x)|\diff x.
  \]
  Now, since \(|f((n+1)h)-f(nh)|\) is a sum in the total variation of \(f\)
  on the interval \([nh,(n+1)h]\), \(|f(x+h)-f(x)|\) is bounded by
  \(V[nh,(n+2)h;f]\). Thus, we have
  \begin{align*}
    \int_{\bbR} |f(x+h)-f(x)|\diff x
    &=\sum_{n=-\infty}^\infty\int_{nh}^{(n+1)h}|f(x+h)-f(x)|\diff x\\
    &\leq\sum_{n=-\infty}^\infty\int_{nh}^{(n+1)h}V[nh,(n+2)h;f]\diff x\\
    &=\sum_{n=-\infty}^\infty V[nh,(n+2)h;f]\int_{nh}^{(n+1)h}\diff x\\
    &=\sum_{n=-\infty}^\infty V[nh,(n+2)h;f]|h|\\
    &=2A|h|.
  \end{align*}
  I suspect there is an error here as the most obvious bound we can get is
  \(2A|h|\) and not the stricter \(A|h|\).

  For part (b), \(f\) is absolutely continuous since it is of bounded
  variation and \(\varphi\) is absolutely continuous since it is Lipschitz
  (\(\varphi\) is differentiable on a compact set, thus, by the mean value
  theorem \(|\varphi(x)-\varphi(y)|\leq \varphi'(\xi)|x-y|\) for some
  \(\xi\in\Supp\varphi\)). Assuming \(\Supp\varphi\) has nonempty interior,
  \(\Supp\varphi\) contains a closed interval \(I=[a,b]\) (in fact, it is
  of the form \(\bigcup_{n\in\bbN} I_n\)) and thus, by integration by
  parts, we have
  \begin{align*}
    \int_a^b f\varphi'\diff x%
    &=f(b)\varphi(b)-f(a)\varphi(a)
      -\int_a^b f'\varphi\diff x\\
    &\leq f(b)-f(a)
      -\int_a^b f'\diff x\\
    &=2(f(b)-f(a))\\
    &\leq 2V[a,b;f]
  \end{align*}
  Thus, summing over every
  \[
    \sum_{n=0}^\infty \int_{a_n}^{b_n} f\varphi'\diff x\leq 2|A|.
  \]
\end{solution}

\begin{problem}
  \hfill
  \begin{enumerate}[label=(\alph*),noitemsep]
  \item Prove the \emph{generalized Hölder's inequality}: Assume
    \(1\leq p_j\leq\infty\), \(j=1,\dotsc,n\), with
    \(\sum_{j=1}^n 1/p_j=1/r\leq 1\). If \(E\) is a measurable set and
    \(f_j\in L^{p_j}(E)\) for \(j=1,\dotsc,n\), then
    \(\prod_{j=1}^n f_j\in L^r(E)\) and
    \[
      \|f_1\dotsm f_n\|_r\leq\|f_1\|_{p_1}\dotsm\|f_n\|_{p_n}.
    \]
  \item Use part (a) to show that that if \(1\leq p,q,r\leq\infty\), with
    \(1/p+1/q=1/r+1\), \(f\in L^p(\bbR)\), and \(g\in L^p(\bbR)\), then
    \[
      |(f*g)(x)|^r\leq{\|f\|_p}^{\!r-p}{\|g\|_q}^{\!r-q}\int|f(y)|^p|g(x-y)|^q\diff
      y.
    \]
    \\\\
    (Recall that \((f*g)(x)=\int f(y)g(x-y)\diff y\).)
  \item Prove \emph{Young's convolution theorem}: Assume that \(p\), \(q\),
    \(r\), \(f\), and \(g\) are as in part (b). Then \(f*g\in L^r(\bbR)\)
    and
    \[
      \|f*g\|_r\leq\|f\|_p\|g\|_q.
    \]
  \end{enumerate}
\end{problem}
\begin{solution}
  For (a) we shall proceed by induction on \(n\) the number of measurable
  functions \(f_j\in L^{p_j}(E)\), \(1\leq j\leq n\). The case \(n=2\)
  holds by using Hölder's inequality on the exponents \(r/p+r/q=1\),
  \begin{align*}
    \biggl[\int_E|f_1f_2|^r\biggr]^{1/r}\rmd x
    &=\|{f_1}^{\!r}{f_2}^{\!r}\|_1\\
    &\leq\|{f_1}^{\!r}\|_{p/r}\|{f_2}^{\!r}\|_{q/r}\\
    &=\|f_1\|_p\|f_2\|_q.
  \end{align*}
  Now, suppose this holds for \(n-1\) measurable functions
  \(f_j\in L^{p_j}(E)\), \(1\leq j\leq n-1\). Then for
  \(f_j\in L^{p_j}(E)\) with \(\sum_{j=1}^n 1/p_j=1/r\), we have
  \(r'=\sum_{j=1}^{n-1}1/p_j=1/r-1/p_n\) so by the inductive step
  \[
    \|f_1\dotsm f_{n-1}\|_{r'}\leq%
    \|f_1\|_{p_1}\dotsm\|f_{n-1}\|_{p_{n-1}}
  \]
  hence, \(f_1\dotsm f_{n-1}\in L^{r'}(E)\). Thus,
  \begin{align*}
    \|f_1\dotsm f_{n-1}f_n\|_r
    &\leq \|f_1\dotsm f_{n-1}\|_{r'}\|f_n\|_{p_n}\\
    &\leq\|f_1\|_{p_1}\dotsm\|f_{n-1}\|_{p_{n-1}}\|f_n\|_{p_n},
  \end{align*}
  as we set out to show.

  For part (b), by the generalized Hölder's inequality, we have
  \begin{align*}
    |f*g|
    &=\biggl|\int_\bbR f(y)g(x-y)\diff y\biggr|\\
    &=\biggl|%
      \int_\bbR \bigl(f(y)^{1/p+1/q}g(x-y)^{1/p+1/q}\bigr)^r\rmd y%
      \biggr|\\
    &=
  \end{align*}
\end{solution}

%%% Local Variables:
%%% mode: latex
%%% TeX-master: "../MA544-Quals"
%%% End:

%% Bañuelos
% \subsection{Bañuelos: Summer 2012}
\setcounter{exercise}{0}



%%% Local Variables:
%%% mode: latex
%%% TeX-master: "../MA544-Quals"
%%% End:

% \subsection{Bañuelos: Winter 2013}
\setcounter{exercise}{0}
\setcounter{equation}{0}
\begin{problem}
  \hfill
  \begin{enumerate}[label=(\alph*)]
  \item \hfill
    \begin{enumerate}[label=(\roman*),noitemsep]
    \item Define almost uniform convergence on the measure space
      $(X,\calF,\mu)$.
    \item Let $f_n$ be a sequence of nonnegative measurable functions
      converging almost uniformly to the nonnegative function $f$. Prove
      that $\sqrt{f_n}$ converges almost uniformly to $\sqrt{f}$.
    \end{enumerate}
  \item \hfill
    \begin{enumerate}[label=(\roman*),noitemsep]
    \item Suppose $f_n$ has the property that $\int_X |f_n|\diff\mu\to 0$.
    \item Does it follow that $f_n\to 0$ almost everywhere? Justify your
      answer.
    \item Does it follow that $f_n\to 0$ almost uniformly? Justify your
      answer.
    \end{enumerate}
  \end{enumerate}
\end{problem}
\begin{solution}
\end{solution}

\begin{problem}
  Let $(X,\calF,\mu)$ be a measure space and let $1\leq p\leq\infty$ and
  $q$ be its conjugate exponent. Suppose $f_n\to f$ in $L^p$ and $g_n\to g$
  in $L^q$. Prove that $f_ng_n\to fg$ in $L^1$.
\end{problem}
\begin{solution}
\end{solution}

\begin{problem}
  Let $\{a_k\}$ be a sequence of positive numbers converging to
  infinity. Prove that the following limit exists
  \[
    \lim_{k\to\infty}\int_0^\infty\frac{\exp(-x)\cos
      x}{a_kx^2+(1/a_k)}\diff x
  \]
  and find it. Make sure to justify all steps.
\end{problem}
\begin{solution}
\end{solution}

\begin{problem}
  Let $(X,\calF,\mu)$ be $\sigma$-finite and $f$ be measurable such that
  for all $\lambda>0$
  \[
    \mu\left( \left\{ \,x\in X:|f(x)|>\lambda\, \right\} \right)\leq%
    \frac{20}{\lambda^p}
  \]
  where $1<p<\infty$. Let $q$ be the conjugate exponent of $p$. Prove that
  there is a constant $C$ depending only on $p$ such that
  \[
    \int_E |f(x)|\diff\mu\leq Cm(E)^{1/q},
  \]
  for all measurable sets $E$ with $0<\mu(E)<\infty$. (The inequality holds
  trivially when $\mu(E)=0$ or $\mu(E)=\infty$.)

  [\emph{Hint}: Recall $\int_E |f(x)|\diff\mu=\int_0^\infty?\diff\lambda$
  and ``break it'' at the right place!]
\end{problem}
\begin{solution}
\end{solution}

\begin{problem}
  Suppose $f\colon[0,1]\to\bbR$ is of bounded variation with
  $V(f;0,1)=\alpha$. For any $\beta>\alpha$, set
  \[
    A=\left\{\,x\in(0,1):\limsup_{h\to 0}\frac{|f(x+h)-f(x)|}{|h|}>\beta\,\right\}.
  \]
  Prove that for any $0<p<1$,  $m(A)\leq (\alpha/\beta)^p$, where $m$
  denotes the Lebesgue measure.
\end{problem}
\begin{solution}
\end{solution}

\begin{problem}
  Let $f\in L^1(0,1)$ and for $x\in(0,1)$, define
  \[
    h(x)=\int_x^1\frac{f(t)}{t}\diff t.
  \]
  \begin{enumerate}[label=(\roman*),noitemsep]
  \item Prove that $h$ is continuous on $(0,1)$.
  \item Show that
    \[
      \int_0^1 h(t)\diff t=\int_0^1 f(t)\diff t.
    \]
  \end{enumerate}
\end{problem}
\begin{solution}
\end{solution}

%%% Local Variables:
%%% mode: latex
%%% TeX-master: "../MA544-Quals"
%%% End:

% \section{Bañuelos}
\subsection{Bañuelos: Summer 2000}
\setcounter{exercise}{0}
\setcounter{equation}{0}
\begin{problem}
  Let $(X,\calF,\mu)$ be a measure space and suppose $\{f_n\}$ is a
  sequence of measurable functions with the property that for all
  $n\geq 1$
  \[
    \mu\left(\left\{\,x\in X:|f_n(x)|\geq\lambda\,\right\}\right)\leq C
    \exp(-\lambda^2/n)
  \]
  for all $\lambda>0$. (Here $C$ is a constant independent of $n$.) Let
  $n_k=2^k$. Prove that
  \[
    \limsup_{k\to\infty}\frac{|f_{n_k}|}{\sqrt{n_k\log(\log(n_k))}}\leq
    1\quad\text{a.e.}
  \]
\end{problem}
\begin{solution}
  Suppose ${\{f_n\}}_{n=1}^\infty$ is a sequence of measurable functions
  such that
  \begin{equation}
    \label{ban:aug00-1}%
    \mu\left(\left\{\,x\in X:|f_n(x)|\geq\lambda\,\right\}\right)\leq C
    \exp(-\lambda^2/n)
  \end{equation}
    for all $\lambda$. Now, consider the subsequence
  ${\{f_{2^k}\}}_{k=1}^\infty$ of ${\{f_n\}}_{n=1}^\infty$. We aim to show
  that
  \[
    \limsup_{k\to\infty}\frac{|f_{2^k}|}{\sqrt{2^k\log(\log(2^k))}}\leq 1
  \]
  almost everywhere. To that end, it suffices to show that the set
  \[
    E=%
    \left\{\,%
      x\in
      X:\limsup_{k\to\infty}\frac{|f_{2^k}|}{\sqrt{2^k\log(\log(2^k))}}>1%
      \,%
    \right\}
  \]
  has measure zero. Let $x\in E$ then
  \[
    \limsup_{k\to\infty} \frac{|f_{2^k}(x)|}{\sqrt{2^k\log(\log(2^k))}}>1.
  \]
  This means that there exists some subsequence
  $\{k_m\}_{m=1}^\infty\subset\{k\}_{n=1}^\infty$ such that
  \[
    \lim_{m\to\infty}\frac{|f_{2^{k_m}}(x)|}{\sqrt{2^{k_m}\log(\log(2^{k_m}))}}>1.
  \]
  This means that, for sufficiently large $N$
  \[
    |f_{2^{k_n}}(x)|>\sqrt{2^{k_n}\log(\log(2^{k_n}))}
  \]
  for all $n\geq N$. But by Equation \eqref{ban:aug00-1} we have
  \begin{equation}
    \label{eq:ban00-2}
    \begin{aligned}
      \mu\left(\left\{\,%
          x\in
          X:\frac{|f_{2^{k_n}}(x)|}{\sqrt{2^{k_n}\log(\log(2^{k_n}))}}\geq
          1 \,\right\}\right) &\leq%
      C\exp\left(-\left.
          \left(\sqrt{2^{k_n}\log(\log(2^{k_n}))}\right)^2\right/2^{k_n}%
      \right)\\
      &=C\exp\left(-\left.2^{k_n}\log(\log(2^{k_n}))\right/2^{k_n} \right)\\
      &=C\exp\left(-\log(\log(2^{k_n}))\right)\\
      &=C\exp\left(\log(1/\log(2^{k_n}))\right)\\
      &=\frac{C}{\log(2^{k_n})}.
    \end{aligned}
  \end{equation}
  Letting $n\to\infty$, we see that the measure of the set on the left-hand
  side of Equation \eqref{eq:ban00-2} must go to $0$ so $\mu(E)=0$.
\end{solution}

\begin{problem}
  Let $(X,\calF,\mu)$ be a finite measure space. Let $f_n$ be a sequence of
  measurable functions with $f_1\in L^1(\mu)$ and with the property that
  \[
    \mu\left(\left\{\,x\in X:|f_n(x)|>\lambda\,\right\}\right) \leq
    \mu\left(\left\{\,x\in X:|f_1(x)|>\lambda\,\right\}\right)
  \]
  for all $n$ and all $\lambda>0$. Prove that
  \[
    \lim_{n\to\infty}\frac{1}{n}\int_X\left[\max_{1\leq j\leq
        n}|f_j|\right]\diff\mu=0.
  \]

  [\emph{Hint}: You may use the fact that
  $\|f\|_1=\int_0^\infty\mu\left(\left\{\,|f(x)|>\lambda\,\right\}\right)\diff\lambda$.]
\end{problem}
\begin{solution}
  Define $g_n,h_n\colon\calF\to[0,\infty]$ for $n\in\bbN$ by
  \begin{align*}
    g_n(\lambda)%
    &=\mu\left(\left\{\,x\in X:|f_n(x)|>\lambda\,\right\}\right),%
    &h_n(\lambda)%
    &=\mu\left(\left\{\,x\in X:\max_{1\leq i\leq n}|f_i(x)|>\lambda\,\right\}\right).%
  \end{align*}
  Now, note that, by the monotonicity of $\mu$, we have
  \[
    h_n(\lambda)\leq \sum_{i=1}^n g_n(\lambda)\leq ng_1(\lambda).
  \]
  Thus,
  \[
    \frac{h_n(\lambda)}{n}\leq g_1(\lambda).
  \]
  Since $\|f_1\|_1=\int_0^\infty g_1(\lambda)\diff\lambda$, by Lebesgue's
  dominated convergence theorem, we have
  \begin{align*}
    \lim_{n\to\infty}\frac{1}{n}%
    \int_X\left[\max_{1\leq j\leq n}|f_j|\right]\diff\mu
    &=\lim_{n\to\infty}\int_X\frac{h_n(x)}{n}\diff\mu\\
    &=\int_X\lim_{n\to\infty}\frac{h_n(x)}{n}\diff\mu\\
    &\leq\int_X\lim_{n\to\infty}\frac{\mu(X)}{n}\\
    &=0
  \end{align*}
  as we wanted to show.
\end{solution}

\begin{problem}
  \hfill
  \begin{enumerate}[label=(\roman*)]
  \item Let $(X,\calF,\mu)$ be a finite measure space. Let $\{f_n\}$ be a
    sequence of measurable functions. Prove that $f_n\to f$ is measurable
    if and only if every subsequence $\{f_{n_k}\}$ contains a further
    subsequence $\{f_{n_{k_j}}\}$ that converges a.e.\@ to $f$.
  \item Let $(X,\calF,\mu)$ be a finite measure space. Let
    $F\colon\bbR\to\bbR$ be continuous and $f_n\to f$ in measure. Prove
    that $F(f_n)\to F(f)$ in measure. (You may assume, of course, that
    $f_n$, $F$, $F(f_n)$, and $F(f)$ are all measurable.)
  \end{enumerate}
\end{problem}
\begin{solution}
  Recall that a sequence of measurable functions $\{f_n\}$ converge in
  measure to a limit $f$ if for every $\varepsilon>0$ the limit
  \[
    \lim_{n\to\infty}
    \mu\left(\left\{\,x\in X:|f(x)-f_n(x)|\geq\varepsilon\,\right\}\right)=0.
  \]

  For part (i) $\implies$ suppose that $f_n\to f$ in measure. Then given
  $\varepsilon>0$ and $\delta>0$ there exists $N\in\bbN$ such that $n\geq
  N$ implies
  \[
    \mu\left(\left\{\,%
        x\in X:|f(x)-f_n(x)|\geq\varepsilon%
        \,\right\}\right)%
    <\delta.
  \]
  In particular, given $\varepsilon=k^{-1}$ and $\delta=2^{-k}$, consider
  the countable collection of measurable sets ${\{E_k\}}_{k=1}^\infty$
  given by
  \[
    E_k=\left\{\,x\in X:|f(x)-f_{n_k}(x)|\geq\frac{1}{k}\,\right\},
  \]
  where $n_k\geq N(k)$ (which depends on our choice of $k$) such that
  \[
    \mu(E_k)<\frac{1}{2^k}.
  \]
  Now, by the Borel--Cantelli lemma, since
  \[
    \sum_{k=1}^\infty \mu(E_k)<\sum_{k=1}^\infty 2^{-k}=1<\infty,
  \]
  for almost every $x\in X$, there exists $N_x\in\bbN$ such that $x\notin
  E_k$ for $k\geq N_x$. This means that for $k\geq N_x$, we have
  \[
    |f(x)-f_{n_k}(x)|<\frac{1}{k}.
  \]
  Let $\{f_{n_{k+1}}\}$ be the subsequence of $\{f_{n_k}\}$. Then
  \[
    \lim_{k\to\infty} f_{n_{k+1}}=f
  \]
  as desired.

  $\impliedby$ On the other hand, suppose that every subsequence
  $\{f_{n_k}\}$ of $\{f_n\}$ contains a subsequence $\{f_{n_{k_j}}\}$ that
  converges to $f$. Seeking a contradiction, suppose that given
  $\varepsilon>0$ there exists a subsequence $\{f_{n_k}\}$ of $\{f_n\}$
  such that
  \[
    M=\mu\left(\left\{\,x\in
        X:|f(x)-f_{n_k}(x)|\geq\varepsilon\,\right\}\right)>0.
  \]
  But by assumption there exists a subsequence $\{f_{n_{k_j}}\}$ of
  $\{f_{n_k}\}$ that converges almost everywhere to $f$. We claim that this
  implies that $f_{n_{k_j}}\to f$ in measure.
  \begin{quote}
  \begin{proof}[Proof of claim]
    \renewcommand{\qedsymbol}{$\blacksquare$}
    This is adapted from a proof in Royden, Proposition 3, Ch.\@ 5.

    First note that $f$ is measurable since it is the pointwise limit
    almost everywhere of a sequence of measurable functions. Let
    $\varepsilon,\delta>0$ be given. \hilight{Here is where the assumption
      that $\mu(X)<\infty$ is essential!} By Egorov's theorem, there is a
    measurable subset $E\subset X$ with $\mu(X\setminus E)<\delta$ such
    that $f_n\to f$ uniformly on $E$. Thus, there is an index $N$ such that
    $n\geq N$ implies
    \[
      |f_n(x)-f(x)|<\varepsilon
    \]
    for all $x\in E$. Thus, for $n\geq N$,
    \[
      \left\{\,x\in X:|f(x)-f_n(x)|\geq\varepsilon\,\right\}\subset
      X\setminus E
    \]
    so
    \[
      \mu\left(%
        \left\{\,x\in X:|f(x)-f_n(x)|\geq\varepsilon\,\right\}%
      \right)%
      <\varepsilon.
    \]
    Thus, we have
    \[
      \lim_{n\to\infty}\mu\left(%
        \left\{\,x\in X:|f(x)-f_n(x)|\geq\varepsilon\,\right\}%
      \right)%
      =0,
    \]
    i.e., $f_n\to f$ in measure.
  \end{proof}
  \end{quote}
  Hence, since $f_{n_{k_j}}\to f$ in measure, but $M>0$ we have a
  contradiction.

  For (ii) since $F$ is continuous given $\varepsilon>0$ there exist
  $\delta>0$ such that $|x-x'|<\delta$ implies
  $|F(x)-F(x')|<\varepsilon$. By part (i), $f_n\to f$ in measure if and
  only if every subsequence $\{f_{n_k}\}$ of $\{f_n\}$ contains a
  subsequence $\{f_{n_{k_j}}\}$ that converges to $f$ almost everywhere,
  i.e., given $\delta>0$ there exists an index $N$ such that $n_{k_j}\geq
  N$ implies
  \[
    |f(x)-f_{n_{j_k}}(x)|<\delta
  \]
  for almost every $x\in X$. Thus,
  \[
    \left|F(f(x))-F(f_{n_{j_k}}(x))\right|<\varepsilon
  \]
  and we see that for every subsequence $\{F\circ f_{n_k}\}$ of $\{F\circ
  f_n\}$ we can find a subsequence $\{F\circ f_{n_{j_k}}\}$ that converges
  almost everywhere to $F\circ f$.
\end{solution}

\begin{problem}
  Let $(X,\calF,\mu)$ be a finite measure space and suppose $f\in L^1(\mu)$
  is nonnegative. Suppose $1<p<\infty$ and let $1<q<\infty$ be its
  conjugate exponent, i.e., $1/p+1/q=1$. Suppose $f$ has the property that
  \[
    \int_Ef\diff\mu\leq \mu(E)^{1/q}
  \]
  for all measurable sets $E$. Prove that $f\in L^r(\mu)$ for any
  $1\leq r<p$.

  [\emph{Hint}: Consider $\left\{\,x\in X:2^n\leq f(x)<2^{n+1}\,\right\}$.]
\end{problem}
\begin{solution}
  Since $\mu(X)<\infty$ it suffices to consider all $x\in X$ such that
  $f(x)>1$ for, by properties of the Lebesgue integral, we can split the
  integral of $f$ over $X$ into
  \[
    \int_X f\diff x=\int_{X\setminus E} f\diff x+\int_{E} f\diff x
    \leq\int_{X\setminus E} f\diff x+C
  \]
  for some constant $C$ and where
  $E=\left\{\,x\in X:0\leq f(x)<1\,\right\}$. So the finiteness of
  $\|f\|_1$ depends only on what happens in $X\setminus E$. Following the
  hint, let us partition $X\setminus E$ into a disjoint countable
  collection of measurable sets $\{E_n\}$ where
  \[
    E_n=\left\{\,x\in X:2^n\leq f(x)<2^{n+1}\,\right\}.
  \]

  Okay, by Jensen's inequality, we have
  \[
    \left(\frac{\int_X f\diff x}{\mu(X)}\right)^r
    \leq
    \frac{\int_Xf^r\diff x}{\mu(X)}
  \]
  which gives us ... absolutely nothing.
\end{solution}

\begin{problem}
  Let $f$ be a continuous function on $[-1,1]$. Find
  \[
    \lim_{n\to\infty}\int_{-1/n}^{1/n} f(x)(1-n|x|)\diff x.
  \]
\end{problem}
\begin{solution}
  First, let us make the following substitution $y=nx$, $\rmd y=n\diff x$
  into the integral
  \[
    \int_{-1/n}^{1/n} f(x)(1-n|x|)\diff x
    =\int_{-1}^1nf(y/n)(1-|y|)\diff y
  \]
\end{solution}

\begin{problem}
  Let $(X,\calF,\mu)$ be a measure space and suppose $f\in L^p(\mu)$,
  $1\leq p<\infty$. Suppose $E_n$ is a sequence of measurable sets
  satisfying $\mu(E_n)=1/n$ for all $n$. Prove that
  \[
    \lim_{n\to\infty}\left[n^{p-1}{p}\int_{E_n}|f|\diff\mu\right]=0.
  \]
\end{problem}
\begin{solution}
\end{solution}

\begin{problem}
  Let $(X,\calM,\mu)$ be a measure space and let $\{g_n\}$ be a sequence of
  nonnegative measurable functions with the property that $g_n\in L^1(\mu)$
  for every $n$ and $g_n\to g$ in $L^1(\mu)$. Let $\{f_n\}$ be another
  sequence of nonnegative measurable functions on $(X,\calF,\mu)$.
  \begin{enumerate}[label=(\roman*),noitemsep]
  \item If $f_n\leq g_n$ almost everywhere for every $n$, prove that
    \[
      \limsup_{n\to\infty}\int_X f_n\diff\mu\leq\int_X\limsup_{n\to\infty}f_n\diff\mu.
    \]

    [\emph{Hint}: Start by considering a subsequence $\{f_{n_k}\}$ such
    that
    \[
      \lim_{n_k\to\infty}\int_X
      f_{n_k}\diff\mu=\limsup_{n\to\infty}\int_X f_n\diff\mu
    \]
    and let $\{g_{n_{k_j}}\}$ be a subsequence of $\{g_{n_k}\}$ such
    that $g_{n_{k_j}}\to g$ almost everywhere.]
  \item If $f_n\to f$ almost everywhere and if $f_n\leq g_n$ almost
    everywhere for all $n$, then $\|f_n-f\|_1\to 0$ as $n\to\infty$.
  \end{enumerate}
\end{problem}
\begin{solution}
\end{solution}

\begin{problem}
  Let $f\in L^1(\bbR)$. Consider the function
  \[
    F(x)=\int_\bbR \exp(\rmi xt) f(t)\diff t.
  \]
  \begin{enumerate}[label=(\roman*),noitemsep]
  \item Show that $F\in L^\infty(\bbR)$ and that $F$ is continuous at every
    $x\in\bbR$. Moreover, if $|t|^kf(t)\in L^\infty(\bbR)$ for all $k\geq
    1$, show that $F$ is infinitely differentiable, i.e., $F\in
    C^\infty(\bbR)$.
  \item Suppose $f$ is continuous as well as in $L^1(\bbR)$. Show that
    $\lim_{|x|\to\infty} F(x)=0$.
  \end{enumerate}
  [\emph{Hint}: Using $\exp(-\rmi\pi)=-1$, write
  $F(x)=\left.\left(\int_\bbR(\exp(\rmi xt)-\exp(\rmi xt-\rmi\pi))\right)\right/2$.]
\end{problem}
\begin{solution}
\end{solution}

%%% Local Variables:
%%% mode: latex
%%% TeX-master: "../MA544-Quals"
%%% End:

%% Lempert
% \include{lempert}

%% Glossaries
\backmatter
% \printglossary

%% References
\bibliographystyle{plain}
\bibliography{anal-bib}
% \printindex
\end{document}

%%% Local Variables:
%%% mode: latex
%%% TeX-master: t
%%% End:
