\section{Differentiation}
This portion of the notes corresponds to material covered before the final.

This section deals with questions of differentiability and culminates with
a couple of results tying together the Lebesgue integral with the
derivative à la the familiar fundamental theorem of calculus for Riemann
integrals.

\subsection{The indefinite integral}
If $f$ is a Riemann integrable function on an interval $[a,b]$ of $\bbR$,
then the familiar definition for its
\href{https://en.wikipedia.org/wiki/Antiderivative}{\emph{indefinite
    integral}} is
\begin{equation}
\label{eq:3:indefinite-integral}
\begin{aligned}
F(x)&=\int_a^xf(y)dy,&a\leq x\leq b.
\end{aligned}
\end{equation}
The
\href{https://en.wikipedia.org/wiki/Fundamental_theorem_of_calculus}{\emph{fundamental
    theorem of calculus}} then asserts that $F'=f$ if $f$ is continuous. In
this section, we study the analogue of this result for Lebesgue integrable
functions.

Since we want to generalize our results to $\bbR^n$, first we must find a
suitable notion of indefinite integral for multivariable functions. In two
dimensions we might, for instance, define the indefinite integral $F$ of
$f$ to be
\begin{equation}
\label{eq:3:2-dim-indefinite-integral}
F(x_1,x_2)\coloneqq \int_{a_1}^{x_1}\int_{a_2}^{x_2}f(y_1,y_2)\diff
y_2\diff y_1.
\end{equation}

As it turns out, it is better to abandon the notion that the indefinite
integral be a function of a point an instead let it be a function of
a set. Therefore, given a function $f$, integrable on some measurable
subset $A$ of $\bbR^n$, we define the \emph{indefinite integral of $f$} to
be the function
\begin{equation}
  \label{eq:3:lebesgue-indefinite-integral}
F(E)\coloneqq \int_E f,
\end{equation}
where $E$ is a measurable subset of $A$.

The function $F$ is an example of a
\href{https://en.wikipedia.org/wiki/Set_function}{\emph{set function}}, by
which we mean any real-valued function $F$ defined on a $\sigma$-algebra
$\Sigma$ of measurable sets such that
\begin{enumerate}[label=(\roman*),noitemsep]
\item $F(E)$ is finite for every $E\in\Sigma$.
\item F is
  \href{https://en.wikipedia.org/wiki/Measure_(mathematics)#Properties}{\emph{countably
      additive}}; i.e., if $E$ is the union of disjoint sets
  $E_k\in\Sigma$, $k=1,2,\dotsc$, then
  \begin{equation}
    \label{eq:3:countably-additive-set-function}
    F(E)\sum_{k\in\bbN} F(E_k).
  \end{equation}
\end{enumerate}

\section{$L^p$ Classes}
Let's take a small detour to ch.\@ 5 of
\cite{wheeden-zygmund:measure-and-integral} to talk about $L^p$ spaces.
\subsection{The relation between the Riemann--Stieltjes integral and the
  Lebesgue integral, and the $L^p$ spaces, $0<p<\infty$}
As it turns out, there is a remarkably simple and useful representation of
the Lebesgue integral (over measurable subsets of $\bbR^n$) in terms of the
Riemann--Stiltjes integrals (over measurable subset of $\bbR$). In order to
establish this relationship, we will need to study the function
\begin{equation}
  \label{eq:eq:3:dist-function}
\omega(\alpha)\coloneqq\omega_{f,E}(\alpha)\coloneqq
\left|\left\{\,\bfx\in E:f(\bfx)>\alpha\,\right\}\right|,
\end{equation}
where $f$ is a measurable function on $E$ and $-\infty<\alpha<\infty$. We
call $\omega_{f,E}$ (or simply $\omega$) the
\href{https://en.wikipedia.org/wiki/Probability_density_function#Formal_definition}{\emph{distribution
    function}} \emph{of $f$ on $E$}.

The function $\omega$ is clearly not affected by changing $f$ in a set of
measure zero, and is decreasing. As $\alpha\nearrow\infty$, we have
\[
\left\{\,\bfx\in E:f(\bfx)>\alpha\,\right\}\searrow
\left\{\,\bfx\in E:f(\bfx)=\infty\,\right\}.
\]
hence, assuming that $f$ is finite a.e.\@ in $E$, by Theorem 3.62(ii),
$\lim_{\alpha\to\infty}\omega=0$, unless
$\omega(\alpha)\equiv\infty$. Similarly, we have
$\lim_{\alpha\to-\infty}\omega=|E|$. For now, let us assume that the
measure of $E$ is finite; this will ensure that $\omega$ is bounded.

In the following results, we assume that $f$ is a measurable function that
is finite a.e.\@ in $E$, $|E|<\infty$, and write
\[
\begin{aligned}
\omega(\alpha)&=\omega_{f,E}(\alpha),&\left\{\,f>a\,\right\}&=\left\{\,\bfx\in
  E:f(\bfx)>\alpha\,\right\},
\end{aligned}
\]
etc.
\begin{lemma}[5.38]
If $\alpha<\beta$, then $\left|\left\{\,\alpha\leq
    f\leq\beta\,\right\}\right|=\omega(\alpha)-\omega(\beta)$.
\end{lemma}
\begin{proof}
For $\alpha<\beta$, we have
$\left\{\,f>\beta\,\right\}\subset\left\{\,f>\alpha\,\right\}$ and
$\left\{\,\gamma<f\leq\beta\,\right\}=\left\{
  \,f>\alpha\,\right\}\setminus\left\{ \,f>\beta\,\right\}$. Since
$\left|\left\{\,f>\beta\,\right\}\right|<\infty$, the lemma follows from
Corollary 3.25.
\end{proof}

Given $\alpha$, let
\[
  \begin{aligned}
    \omega(\alpha{+})&\coloneqq\lim_{\varepsilon\searrow
      0}\omega(\alpha+\varepsilon)\quad&
    \omega(\alpha{-})&\coloneqq\lim_{\varepsilon\searrow 0}\omega(\alpha-\varepsilon).
  \end{aligned}
\]
denote the limits of $\omega$ from the right and left at $\alpha$.

\begin{lemma}[5.39]
\hfill
\begin{enumerate}[label=\textnormal{(\alph*)},noitemsep]
\item $\omega(\alpha+)=\omega(\alpha)$; i.e., $\omega$ is continuous from
  the right.
\item $\omega(\alpha-)=\left|\left\{\,f\geq\alpha\,\right\}\right|$.
\end{enumerate}
\end{lemma}

\begin{corollary}[5.40]
\hfill
\begin{enumerate}[label=\textnormal{(\alph*)},noitemsep]
\item
  $\omega(\alpha-)-\omega(\alpha)=\left|\left\{\,f=\alpha\,\right\}\right|$;
  in particular, $\omega$ is continuous at $\alpha$ if and only if
  $\left|\{\,f=\alpha\,\}\right|=0$.
\item $\omega$ is constant in an open interval $(\alpha,\beta)$ if and only
  if $\left|\{\,\alpha<f<\beta\,\}\right|=0$, that is, if and only if $f$
  takes almost no values between $\alpha$ and $\beta$.
\end{enumerate}
\end{corollary}

The rest of this section establishes the relations between the Lebesgue and
Riemann--Stieltjes integrals. As always, we assume $f$ is measurable and
finite a.e.\@ in $E$, $|E|<\infty$ and $\omega=\omega_{E,f}$.

\begin{theorem}[5.41]
If $a\leq f(\bfx)\leq b$ ($a$ and $b$ are finite) for all $\bfx\in E$, then
\[
\int_E f=-\int_a^b\alpha\diff\omega(\alpha).
\]
\end{theorem}
\begin{proof}
The Lebesgue integral on the left-hand side exists since $f$ is bounded and
$|E|<\infty$. The Riemann--Stieltjes integral on the right-hand side exists
by Theorem 2.24. To show that they are equal, let us partition the interval
the interval $[a,b]$ by $a=\alpha_0<\alpha_1<\dotsb<\alpha_k=b$ and let
$E_j=\left\{,\alpha_{j-1}<f\leq\alpha_j\,\right\}$. The $E_j$ are disjoint
and $E=\bigcup_{j=1}^k E_j$. Hence, $\int_E f=\sum_{j=1}^k\int_{E_j}f$ and,
therefore
\[
\sum_{j=1}^\infty \alpha_{j-1}|E_j|\leq\int_E f\leq\sum_{j=1}^k\alpha_j|E_j|.
\]
By Lemma 5.38,
$|E_j|=\omega(\alpha_j)\omega(\alpha_j)=-[\omega(\alpha_j)-\omega(\alpha_{j-1})]$. Hence,
the sums are Riemann--Stieltjes sums for
$-\int_a^b\alpha\diff\omega(\alpha)$. Since the sums must converge to
$-\int_a^b\alpha\diff\omega(\alpha)$ as the norm of the partition tends to
zero, the conclusion follows.
\end{proof}

We can extend the conclusion of Theorem 5.41 to the case when $f$ is not
bounded as follows.
\begin{theorem}[5.42]
Let $f$ be any measurable function on $E$, and let
$E_{ab}\coloneqq\left\{\,\bfx\in E:a<f(\bfx)<b\,\right\}$ ($a$ and $b$
finite). Then,
\[
\int_{E_{ab}}f=-\int_a^b\alpha\diff\omega(\alpha).
\]
\end{theorem}
\begin{proof}[Sketch of proof]
Take $\omega_{ab}(\alpha)\coloneqq\left|\left\{\,\bfx\in
    E_{ab}:f(\bfx)>\alpha\,\right\}\right|$. By Theorem 5.41, we have
\[
\int_{E_{ab}}f=-\int_a^b\alpha\diff\omega_{ab}(\alpha).
\]
Taking the limit of Riemann--Stieltjes sums that approximate the integrals,
it suffices to show that
$\omega_{ab}(\alpha)-\omega_{ab}(\beta)=\omega(\alpha)-\omega(\beta)$. Then
The expression on the right-hand side of the equation above, is seen to be
$\int_a^b\alpha\diff\omega(\alpha)$.
\end{proof}

\begin{theorem}[5.43]
If either $\int_E f$ or $\int_{-\infty}^\infty\alpha\diff\omega(\alpha)$
exist and is finite, then the other exists and is finite, and
\[
\int_E f=-\int_{-\infty}^\infty\alpha\diff\omega(\alpha).
\]
\end{theorem}

Two measurable functions $f$ and $g$ are said to be
\href{https://en.wikipedia.org/wiki/Equidistributed_sequence#Sequences_equidistributed_with_respect_to_an_arbitrary_measure}{\emph{equimeasurable}},
or \emph{equidistributed}, if
\[
\omega_{f,E}(\alpha)=\omega_{g,E}(\alpha)
\]
for all $\alpha$.

We may intuitively think of equimeasurable functions as being
\emph{rearrangements} of each other. For such functions, we have
\[
\begin{aligned}
\left|\{\,a<f\leq b\,\}\right|&=
\left|\{\,a<g\leq b\,\}\right|&
\left|\{\,f=a\,\}\right|&=
\left|\{\,g=a\,\}\right|,
\end{aligned}
\]
etc. We also gave the following immediate corollary of Theorem 5.43.

\begin{corollary}[5.44]
If $f$ and $g$ are equimeasurable on $E$ and $f\in L(E)$, then $g\in
L(E)$ and
\[
\int_E f=\int_E g.
\]
\end{corollary}

The method used to derive Theorem 5.41 through 5.43 illustrates a basic
difference between the Lebesgue and the Riemann integral. The Riemann
integral is defined by a limiting process whose initial step involves
partitioning the domain of $f$. On the other hand, the Lebesgue integral
can be obtained from a process that partitions the \emph{range} of $f$. In
order to define the process more clearly, let $f$ be a nonnegative
measurable function that is finite a.e.\@ in $E$, $|E|<\infty$. Let
$\Gamma=\{\,0=\alpha_0<\alpha_1<\cdots\,\}$ be a partition of the positive
ordinate axis by a countable number of points $\alpha_k\to\infty$, and let
$|\Gamma|=\sup_k(\alpha_{k+1}-\alpha_k)$. Set
$E_k\coloneqq\left\{\,\alpha_k\leq f<\alpha_{k+1}\,\right\}$ and
$Z\coloneqq\left\{\,f=\infty\,\right\}$. Then the $E_k$ are measurable and
disjoint, $|Z|=0$ and $E=\left(\bigcup E_k\right)\cup Z$, so that
$|E|=\sum_k|E_k|$. Let
\[
  \begin{aligned}
    S_\Gamma&\coloneqq\sum_{k\in\bbN}\alpha_k|E_k|,&
    S_\Gamma&\coloneqq\sum_{k\in\bbN}\alpha_{k+1}|E_k|.
  \end{aligned}
\]

\section{$L^p$ Classes}
Let's talk about $L^p$ classes now and some important results about
$L^p$ spaces.
\subsection{Definition of $L^p$}
If $E$ is a measurable subset of $\bbR^n$ and satisfies $0<p<\infty$, then
\href{https://en.wikipedia.org/wiki/Lp_space}{$L^p(E)$} denotes the
collection of measurable $f$ for which $\int_E|f|^p$ is finite, i.e.,
\begin{equation}
  \label{eq:3:lp-space}
L^p(E)\coloneqq\left\{\,f:\int_E |f|^p<\infty\,\right\}
\end{equation}
for $0<p<\infty$. Here, $f$ may be complex-valued, in which case, if
$f=f_1+if_2$ for measurable real-valued $f_1$ and $f_2$, we have
$|f|^2={f_1}^2+{f_2}^2$, so that
\[
|f_1|,|f_2|\leq|f|\leq |f_1|+|f_2|.
\]
It follows that $f\in L^p(E)$ if and only if both $f_1,f_2\in L^p(E)$.

We shall write
\[
\|f\|_{p,E}\coloneqq\left(\int_E|f|^p\right)^{1/p},
\]
for $0<p<\infty$. Thus, $L^p(E)$ is the set of measurable $f$ for which
$\|f\|_{p,E}$ is finite. Whenever it is clear from context, we will omit
$E$ in $L^p(E)$ and $\|f\|_{p,E}$, and instead write $L^p$ and
$\|f\|_p$. Also note that $L=L^1$.

In order to define $L^\infty(E)$, let $f$ be real-valued and measurable on
a set $E$ of positive measure. Define the
\href{https://en.wikipedia.org/wiki/Essential_supremum_and_essential_infimum}{\emph{essential
  supremum}} of $f$ on $E$ to be
\begin{equation}
  \label{eq:3:essential-supremum}
\esssup_E f\coloneqq\inf\left\{\,\alpha:\left|\left\{\,\bfx\in
      E:f(\bfx)>\alpha\,\right\}\right|=0\,\right\}.
\end{equation}
In words, this the essential supremum of $f$ is the least upper bound of
$f$ outside of a set of measure zero. It can be restated as such: $\esssup
f$ is the smallest number $M$, $-\infty\leq M\leq\infty$, such that
$f(\bfx)\leq M$ almost everywhere in $E$.

In the definition of $\esssup f$, we have made the explicit assumption that
the measure of $E$ is nonzero. Otherwise, $\esssup f=-\infty$ which can
result in awkward or incorrect statements of results involving $L^p$
spaces. Therefore, we shall adopt the convention that $\esssup f=0$ if
$|E|=0$.

A real or complex-valued measurable $f$ is said to be \emph{essentially
  bounded}, or simply \emph{bounded} almost everywhere on $E$ if $\esssup
|f|$ is finite. The class of all functions that are essentially bounded on
$E$ is denoted by $L^\infty(E)$. Clearly, $f\in L^\infty(E)$ if and only if
its real and imaginary parts belong to $L^\infty(E)$. We shall use the
notation $\|f\|_\infty$ synonymously with $\esssup f$.

The following theorem gives some good motivation for the use the notation
$\|f\|_\infty$, at least in the case $|E|<\infty$.
\begin{theorem}[8.1]
If $|E|<\infty$, then $\|f\|_\infty=\lim_{p\to\infty}\|f\|_p$.
\end{theorem}
\begin{proof}[Sketch of proof]
We may assume that $|E|>0$, for otherwise we have a trivial statement,
i.e., $\|f\|_p=0$ for all $p$ and by convention $\|f\|_\infty=0$ so clearly
$\|f\|_p\to\|f\|_\infty$ as $p\to\infty$. Set $M\coloneqq\|f\|_\infty$. If $M'<M$,
\end{proof}

%%% Local Variables:
%%% mode: latex
%%% TeX-master: "../MA544-Quals"
%%% End:
