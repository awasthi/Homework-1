\subsubsection{Homework 4}
\setcounter{exercise}{0}
% #4 - Due Feb. 8 Read Sections 3.5, 4.1. Chapter 3: # 12, 13, 14, 15 (16,
% 18), (Chapter 4: # 1, 2

\begin{problem}[Wheeden \& Zygmund Ch.\@ 3, Ex.\@ 12]
  If $E_1$ and $E_2$ are measurable sets in $\bbR^1$, show $E_1\times E_2$
  is a measurable subset of $\bbR^2$ and $m(E_1\times
  E_2)=m(E_1)m(E_2)$. (Interpret $0\cdot\infty$ as $0$.) [\emph{Hint:} Use
  a characterization of measurability.]
\end{problem}
\begin{solution}
  The proof of this result is rather long and we shall omit it for now as I
  gain nothing from retracing my steps on this one.
\end{solution}

\begin{problem}[Wheeden \& Zygmund Ch.\@ 3, Ex.\@ 13]
  Motivated by (3.7), define the \emph{inner measure} of $E$ by
  $m_*(E)=\sup m(F)$, where the supremum is taken over all closed subsets
  $F$ of $E$. Show that
  \begin{enumerate}[label=(\roman*),noitemsep]
  \item $m_*(E)\leq m^*(E)$, and
  \item if $m^*(E)<\infty$, then $E$ is measurable if and only if
    $m_*(E)=m^*(E)$.
  \end{enumerate} [Use (3.22).]
\end{problem}
\begin{solution}
  First we show part (i). If $m^*(E)=\infty$, the inequality holds
  trivially. Suppose that $m^*(E)<\infty$. Then, since $F$ is closed, it is
  measurable and $m(F)=m^*(F)$. Moreover, $F\subset E$ so by the
  monotonicity of the outer measure,
  \[
    m(F)=m^*(F)<m^*(E).
  \]
  Taking the supremum over all $F$ on the left, we have
  \[
    m_*(E)=\sup_{F\subset E}m(F)<m^*(E)
  \]
  as we set out to show.

  Next we show part (ii). Let $E\subset\bbR^n$ with
  $m^*(E)<\infty$. $\implies$ Suppose that $E$ is measurable. Then, by
  Lemma 3.22, there exists a closed set $F\subset E$ such that
  $m^*(E\smallsetminus F)<\varepsilon$. Since closed sets are measurable,
  by Corollary 3.31, we have
  \[
    m^*(E\smallsetminus F)=m(E)-m(F)<\varepsilon
  \]
  so
  \[
    m(E)<m(F)+\varepsilon.
  \]
  Letting $\varepsilon$ go to $0$, we have
  \[
    m(E)\leq m(F);
  \]
  and taking the supremum on the right
  \[
    m(E)\leq m_*(E).
  \]
  But, by part (i), $m_*(E)\leq m^*(E)=m(E)$. Thus, $m_*(E)=m^*(E)$ as was
  to be shown.

  $\impliedby$ On the other hand, suppose that $m_*(E)=m^*(E)$. Then, given
  $\varepsilon>0$ there exists an open set $G$ containing $E$ and a closed
  set $F$ contained in $E$ such that\marginremark{These are the definitions
  of}
  \begin{align*}
    m(G)-m^*(E)&<\frac{\varepsilon}{2}\\
    m_*(E)-m(F)&<\frac{\varepsilon}{2}.
  \end{align*}
  Then
  \begin{align*}
    m^*(E\smallsetminus F)
    &<m^*(G\smallsetminus F)\\
    &=m^*(G)-m^*(G\cap F)\\
    &=m^*(G)-m^*(F)\\
    &<\frac{\varepsilon}{2}+m^*(E)-\left(m^*(E)-\frac{\varepsilon}{2}\right)\\
    &=\varepsilon.
  \end{align*}
  Thus, by Lemma 3.22, $E$ is measurable.
\end{solution}

\begin{problem}[Wheeden \& Zygmund Ch.\@ 3, Ex.\@ 15]
  If $E$ is measurable and $A$ is any subset of $E$, show that
  $m(E)=m_*(A)+m^*(E\smallsetminus A)$. (See Exercise 13 for the definition
  of $m_*(A)$.)
\end{problem}
\begin{solution}
\end{solution}

%%% Local Variables:
%%% mode: latex
%%% TeX-master: "../MA544-Quals"
%%% End:
