\subsection{Homework 3}
\begin{problem}[Wheeden \& Zygmund Ch.\@ 3, Ex.\@ 5]
  Construct a subset of $[0,1]$ in the same manner as the Cantor set,
  except that at the $k$th stage each interval removed has length
  $\delta 3^{-k}$, $0<\delta<1$. Show that the resulting set is perfect,
  has measure $1-\delta$, and contains no interval.
\end{problem}
\begin{proof}
  \underline{The construction}: pick a $\delta\in(0,1)$ and remove the
  open interval $(\delta/3,1-\delta/3)$ from $[0,1]$ which, in the first
  step of the construction, will yield the sets
  \[
    C_{1,1}\coloneq\left[0,\frac{\delta}{3}\right]\quad\text{and}\quad
    C_{1,2}\coloneq\left[1-\frac{\delta}{3},1\right].
  \]
  In the next step of the construction, we will be removing the interval
  $(\delta/9,\delta/3-\delta/9)\cup(1-\delta/3+\delta/9,1-\delta/9)$ from
  the union $C_1\coloneq C_{1,1}\cup C_{1,2}$ and so on. We can state this
  more cleanly: at the $n$th stage in the construction of our set
  \[
    C_n\coloneq%
    C_{n-1}\setminus\bigcup_{i=1}^{n-1}%
    \left(\frac{i\delta}{3^n},\frac{(i+1)\delta}{3^n}\right)\cup%
    \left(\frac{3^n-(i+1)\delta}{3^n},\frac{3^n-i\delta}{3^n}\right).
  \]
  (Probably; I'll check it later after I have some more time.)  Thus, at
  the $n$th stage in the construction $C_n$ is the union of $2^n$ disjoint
  intervals of length $\delta 3^{-n}$. Taking the intersection of these
  sets
  \[
    C\coloneq\bigcap_{i\in\bbN} C_i
  \]
  we claim that the set $C$ is perfect, has measure $\delta-1$ and contains
  no intervals. \textcolor{Red}{**Forget this. For some reason, I have
    forgotten the construction, so what I made was a Cantor set with
    measure zero, not what was asked for. We'll proceed as if we
    constructed what was asked of us.**}
  \\\\
  \underline{$C$ is perfect}: To prove that $C$ is perfect we must show
  that $C$ is closed and dense in itself. $C$ is closed because it is it he
  (arbitrary) intersection of closed intervals. To see that $C$ is dense in
  itself we must show that given any $\varepsilon>0$ for any $x\in C$, the
  open ball $B(x,\varepsilon)$ contains another point, call it $y$, in
  $C$. But first, we prove the following claim: At the end of each stage
  $C_n$ in the construction of $C$, \textcolor{Red}{the set of all
    endpoints of intervals in $C_n$ is a subset of $C$}. (We will not prove
  this here as it is tedious and I have already proved this before; and
  after all these notes are mostly for myself.) Now, since $x\in C$,
  $x\in C_n$ for every $n\in\bbN$. In particular, for sufficiently large
  $N$, $\delta 3^{-N}<\varepsilon$ and $x$ is in one of the closed interval
  that makes up $C_N$. Pick an endpoint $y\neq x$ in that interval. Then
  $y\in C$ and $y\in B(x,\varepsilon)$. Thus, $C$ is dense in itself.
  \\\\
  \underline{$C$ has measure $1-\delta$}: For this we simply compute the
  length of $C$ by evaluating the limit of sequence of lengths, which is
  justified by Theorem 3.26 since $C_{n+1}\subset C_n$ and $C_n$ is
  measurable for all $n\in\bbN$, hence
  \begin{align*}
    \lim_{n\to\infty}m^*(C_n)
    &=\lim_{n\to\infty}
      \left[1-\sum_{i=1}^n\delta\left(\frac{2^i}{3^{i+1}}\right)\right]\\
    &=\lim_{n\to\infty}
      \left[1-\sum_{i=1}^n\frac{\delta}{3}\left(\frac{2}{3}\right)^i\right]\\
    &=1-\frac{\delta}{3}3\\
    &=1-\delta
  \end{align*}
  as desired.
  \\\\
  \underline{$C$ contains no intervals}: Seeking a contradiction, we will
  assume that $I$ is an interval of positive measure contained in
  $C$. Since $I\coloneq[a,b]$ is a connected subset of $\bbR$ and
  $I\subset C$, then $I$ must be contained in some interval $I_n$ in the
  $n$th step of the construction of $C$ for every $n\in\bbN$. However, for
  sufficiently large $N$, $m(I_N)<m(I)=b-a$. Thus, $C$ contains no
  intervals.
\end{proof}

\begin{problem}[Wheeden \& Zygmund Ch.\@ 3, Ex.\@ 7]
  Prove (3.15).
\end{problem}
\begin{proof}
  Here is the statement of the lemma:
  \begin{quote}
    \emph{If ${\{I_k\}}_{k=1}^{N}$ is a finite collection of nonoverlapping
      intervals, then $\bigcup_{k=1}^NI_k$ is measurable and
      $m\left(\bigcup_{k=1}^NI_k\right)=\sum_{k=1}^Nm(I_k)$.}
  \end{quote}
  By Theorem 3.12, the union $\bigcup_{n=1}^N I_n$ is measurable. Now we
  show that $m\left(\bigcup_{n=1}^NI_n\right)=\sum_{n=1}^Nm(I_n)$. At least
  one inequality is straightforward, by Theorem 3.12 we have
  \[
    m\left(\bigcup_{k=1}^N I_k\right)\leq\sum_{n=1}^Nm(I_n).
  \]
  To see the reverse note that $m(I_n^\circ)=m(I_n)$ and
  $I_n^\circ\subset I_n$ so by Theorem 3.3
  \begin{align*}
    m\left(\bigcup_{n=1}^NI_n\right)
    &\geq m\left(\bigcup_{n=1}^NI_n^\circ\right)\\
    &=\sum_{n=1}^Nm(I_n^\circ)\\
    &=\sum_{n=1}^Nm(I_n).
  \end{align*}
  Thus, $\bigcup_{n=1}^N I_n$ is measurable and
  $m\left(\bigcup_{n=1}^NI_n\right)=\sum_{n=1}^Nm(I_n)$.
\end{proof}

\begin{problem}[Wheeden \& Zygmund Ch.\@ 3, Ex.\@ 8]
  Show that the Borel algebra $\calB$ in $\bbR^n$ is the smallest
  $\sigma$-algebra containing the closed sets in $\bbR^n$.
\end{problem}
\begin{proof}
  Since $\calB$ is the smallest $\sigma$-algebra containing all of the open
  sets of $\bbR^n$, it contains all of the closed sets of $\bbR^n$. Now,
  suppose that $\calB'$ is another $\sigma$-algebra containing the closed
  sets in $\bbR^n$. Then, $\calB'\subset\calB$ since $\calB$ contains all
  of the closed sets in $\bbR^n$. However, since $\calB'$ is a
  $\sigma$-algebra, it contains all of the open sets in $\bbR^n$, so
  $\calB'\subset\calB$ since $\calB$ is the smallest $\sigma$-algebra
  containing the open sets in $\bbR^n$. Thus, $\calB'=\calB$.
\end{proof}

\begin{problem}[Wheeden \& Zygmund Ch.\@ 3, Ex.\@ 9]
  If ${\{E_k\}}_{k=1}^\infty$ is a sequence of sets with
  $\sum m^*(E_k)<+\infty$, show that $\limsup E_k$ (and also $\liminf E_k$)
  has measure zero.
\end{problem}
\begin{proof}
  Suppose ${\{E_n\}}_{n=1}^\infty$ is a sequence of sets with
  $\sum_{n=1}^\infty m^*(E_n)<\infty$. Then, since the sum
  $\sum_{n=1}^\infty m^*(E_n)$ converges, given $\varepsilon>0$ there
  exist $N\in\bbN$ such that $n\geq N$ implies
  $\sum_{j=n}^\infty m^*(E_j)<\varepsilon$. But what does this say about
  the $\varlimsup_{n\to\infty} E_n$? Well, recall that
  \[
    \varlimsup_{n\to\infty}
    E_n\coloneq\bigcap_{j=1}^\infty\bigcup_{k=j}^\infty E_k.
  \]
  Define $F_n\coloneq\bigcup_{j=n}^\infty E_j$. Then $F_n\supset F_{n+1}$
  and $F_n\searrow\varlimsup_{n\to\infty} E_n$. Moreover, by Theorem 3.12
  $m^*(F_1)\leq\sum_{n=1}^\infty m^*(E_n)<\infty$ so by Theorem 3.26,
  $m(\varlimsup_{n\to\infty} E_n)=\lim_{n\to\infty}m(F_n)=0$ since the sum
  $\sum_{n=1}^\infty m(E_n)$ converges.

  Setting $F_n\coloneq\bigcap_{j=n}^\infty E_j$ yields the same for
  $\varliminf_{n\to\infty} E_n$.
\end{proof}

\begin{problem}[Wheeden \& Zygmund Ch.\@ 3, Ex.\@ 10]
  If $E_1$ and $E_2$ are measurable, show that
  $m(E_1\cup E_2)+m(E_1\cap E_2)=m(E_1)+m(E_2)$.
\end{problem}
\begin{proof}
\end{proof}

%%% Local Variables:
%%% mode: latex
%%% TeX-master: "../MA544-Quals"
%%% End:
