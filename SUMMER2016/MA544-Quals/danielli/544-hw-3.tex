\subsubsection{Homework 3}
\setcounter{exercise}{0}
\setcounter{equation}{0}

\begin{problem}[Wheeden \& Zygmund Ch.\@ 3, Ex.\@ 5]
  Construct a subset of $[0,1]$ in the same manner as the Cantor set,
  except that at the $k$th stage each interval removed has length
  $\delta 3^{-k}$, $0<\delta<1$. Show that the resulting set is perfect,
  has measure $1-\delta$, and contains no interval.
\end{problem}
\begin{solution}
  We construct the prescribed subset as follows: take the open interval
  $(1/2-\delta/6,1/2+\delta/6)$ and remove it from the closed interval
  $[0,1]$ the result is a union of two disjoint closed intervals
  \[
    \begin{aligned}
    E_{1,1}&=\left[0,\frac{1}{2}-\frac{1}{6}\delta\right],&
    E_{1,2}&=\left[\frac{1}{2}+\frac{1}{6}\delta,1\right],
    \end{aligned}
  \]
  whose union we call $E_1$; this marks the first step in the construction
  of this Cantor-like set. Next, we remove the set
  \[
    \left(\frac{1}{4}-\frac{5}{36}\delta,\frac{1}{4}+\frac{1}{36}\delta\right)
    \cup
    \left(\frac{3}{4}+\frac{\delta}{36},\frac{3}{4}+\frac{5}{36}\delta\right)
  \]
  from the set $E_1$ which yields $E_2$ the union of the four closed
  intervals
  \[
    \begin{aligned}
      E_{2,1}&=\left[0,\frac{1}{4}-\frac{5}{36}\delta\right],&
      E_{2,2}&=\left[\frac{1}{4}+\frac{1}{36}\delta,
        \frac{1}{2}-\frac{1}{6}\delta\right],\\
      E_{2,3}&=\left[\frac{1}{2}+\frac{1}{6}\delta,
        \frac{3}{4}+\frac{\delta}{36}\right],&
      E_{2,4}&=\left[\frac{3}{4}+\frac{5}{36}\delta,1\right].
    \end{aligned}
  \]
  In the $n$th step of the construction, we remove an open interval of
  length $3^{-n}\delta$ from the center of each interval $E_{n-1,i}$
  yielding $E_n$ which is the union of $2^n$ intervals $E_{n,i}$ of length
  $2^{-n}-\delta 2^{-n}\sum_{i=1}^n2^{i-1}3^{-i}$. Let $E$ be the intersection
  $\bigcap_{i=1}^\infty E_i$. This concludes our construction.

  Next we show that $E$ is perfect, has measure $1-\delta$ and contains no
  interval.

  To see that $E$ is perfect, we must show that $E$ is closed and that
  and dense in itself. The set $E$ is closed because it is the (arbitrary)
  intersection of closed intervals. To see that $E$ is dense in itself, we
  must show that for every $\varepsilon>0$, for every $x\in E$, the
  intersection $(B(x,\varepsilon)\cap E)\setminus\{x\}$ is
  nonempty. Let $\varepsilon>0$ and $x\in E$ be given. Then, since
  $x\in E$, $x\in E_n$ for every $n$. Thus, $x$ is in some closed interval
  $E_{n,i}\subseteq E_n$. Let $N$ be the smallest integer such that the
  length of $E_{N,i}=[a,b]$ is less that $\varepsilon$. Then, $a,b\in E$
  and $a,b\in B(x,\varepsilon)$ and $x$ is cannot be equal to both $a$ and
  $b$. Thus, $(E\cap B(x,\varepsilon))\setminus\{x\}\neq\emptyset$. It
  follows that $E$ is a perfect set.

  To see that the measure of $E$ is $1-\delta$ by Theorem 3.26 (ii) since
  $m(E_1)=1-\delta/3<\infty$ and $E_n\searrow E$ we have
  \begin{align*}
    m(E)
    &=m\left(\bigcap_{i=1}^\infty E_i\right)\\
    &=\lim_{n\to\infty}\sum_{i=1}^n m(E_i)\\
    &=\lim_{n\to\infty}\sum_{i=1}^{2^n}
      \left[
      \frac{1}{2^n}-\frac{\delta}{2^n}\sum_{i=1}^n\frac{2^{i-1}}{3^i}
      \right]\\
    &=\lim_{n\to\infty}\left[1-\delta\sum_{i=1}^n\frac{2^{i-1}}{3^i}\right]\\
    &=\lim_{n\to\infty}\left[1-\frac{\delta}{3}\sum_{i=1}^n
      \left(\frac{2}{3}\right)^{i-1}\right]
    \shortintertext{letting $j=i-1$, we can rewrite the series above as the
      geometric series}
    &=1-\frac{\delta}{3}\lim_{n\to\infty}\sum_{j=0}^n\left(\frac{2}{3}\right)^j\\
    &=1-\delta,
  \end{align*}
  as desired.

  Lastly, we must show that $E$ contains no interval. Seeking a
  contradiction, suppose that $E$ contains an interval $I=[a,b]$ of length
  $b-a$. Then, since $I\subseteq E$, $I\subseteq E_n$ for all $n$ so, since $I$
  is connected, it must be contained in one of the $E_{n,i}$ for all
  $n$. Let $N$ be the smallest integer such that $m(E_{N,i})<b-a$ and
  $E_{N,i}=[c,d]$ contains $I$. Then, since $I\subseteq E_{N,i}$, both $a$
  and $b$ are points in $I$, $|b-a|\leq |d-c|=m(E_{N,i})$. This is a
  contradiction. Thus, it must be the case that $E$ contains no interval.
\end{solution}

\begin{problem}[Wheeden \& Zygmund Ch.\@ 3, Ex.\@ 7]
  Prove (3.15).
\end{problem}
\begin{solution}
  Here is the statement of the lemma:
  \begin{quote}
    \emph{If ${\{I_k\}}_{k=1}^{N}$ is a finite collection of nonoverlapping
      intervals, then $\bigcup_{k=1}^NI_k$ is measurable and
      $m\left(\bigcup_{k=1}^NI_k\right)=\sum_{k=1}^Nm(I_k)$.}
  \end{quote}
  By Theorem 3.12, the union $\bigcup_{n=1}^N I_n$ is measurable. Hence, it
  remains to show that
  $m\left(\bigcup_{n=1}^NI_n\right)=\sum_{n=1}^Nm(I_n)$.

  We take the approach of extending the argument provided in Theorem
  3.2. As in Theorem 3.2, we note that, since ${\{I_n\}}_{n=1}^N$ covers
  the union $\bigcup_{n=1}^N I_n$, then
  \[
    m\left(\bigcup_{n=1}^N I_n\right)\leq \sigma%
    \left(\bigcup_{n=1}^N I_n\right)=\sum_{n=1}^N m(I_n).
  \]

  On the other hand, note that $I_n$ is the union
  $I_n^\circ\cup\partial I_n$ of its interior and its boundary. In the
  previous homework, we showed that the boundary of an interval has measure
  zero. Hence, we have
  \[
    m(I_n^\circ)\leq m(I_n)\leq m(I_n^\circ)+m(\partial I_n)=m(I_n^\circ)
  \]
  so $m(I_n)=m(I_n^\circ)$. Now, note that
  \[
    m\left(\bigcup_{n=1}^N I_n^\circ\right)=%
    \sum_{n=1}^N m(I_n^\circ)=%
    \sum_{n=1}^N m(I_n).
  \]
  Hence, we have
  \begin{align*}
    \sum_{n=1}^N m(I_n)
    &=m\left(\bigcup_{n=1}^N I_n^\circ\right)\\
    &\leq m\left(\bigcup_{n=1}^N I_n\right)\\
    &\leq \sum_{n=1}^N m(I_n).
  \end{align*}
  Thus, equality $m\left(\bigcup_{n=1}^NI_n\right)=\sum_{n=1}^Nm(I_n)$
  holds.
\end{solution}

\begin{problem}[Wheeden \& Zygmund Ch.\@ 3, Ex.\@ 8]
  Show that the Borel algebra $\calB$ in $\bbR^n$ is the smallest
  $\sigma$-algebra containing the closed sets in $\bbR^n$.
\end{problem}
\begin{solution}
  Since $\calB$ is the smallest $\sigma$-algebra containing all of the open
  sets of $\bbR^n$, it contains all of the closed sets of $\bbR^n$. Now,
  suppose that $\calB'$ is another $\sigma$-algebra containing the closed
  sets in $\bbR^n$. Then, $\calB'\subseteq\calB$ since $\calB$ contains all
  of the closed sets in $\bbR^n$. However, since $\calB'$ is a
  $\sigma$-algebra, it contains all of the open sets in $\bbR^n$, so
  $\calB'\subseteq\calB$ since $\calB$ is the smallest $\sigma$-algebra
  containing the open sets in $\bbR^n$. Thus, $\calB'=\calB$.
\end{solution}

\begin{problem}[Wheeden \& Zygmund Ch.\@ 3, Ex.\@ 9]
  If ${\{E_k\}}_{k=1}^\infty$ is a sequence of sets with
  $\sum m^*(E_k)<\infty$, show that $\limsup E_k$ (and also $\liminf E_k$)
  has measure zero.
\end{problem}
\begin{solution}
  First, since ${\{E_n\}}_{n=1}^\infty$ is a sequence of sets with
  \[
    \sum_{i=1}^\infty m^*(E_i)<\infty
  \]
  for every $\varepsilon>0$ there exists $N\in\bbN$ such that $n\geq N$
  implies
  \[
    \sum_{i=n}^\infty m^*(E_i)<\varepsilon.
  \]
  Let's put this aside for now.

  Define $E=\limsup_{n\to\infty} E_n$ and $E_n'=\bigcup_{i=n}^\infty
  E_n$. It is easy to see that ${\{E_n'\}}_{n=1}^\infty$ is a decreasing
  sequence of sets whose intersection $\bigcap_{n=1}^\infty E_n$ is the
  limit supremum $E$. By the monotonicity of the outer measure, we have
  \[
    m^*(E)\leq m^*(E_n')
  \]
  for all $n\in\bbN$. On the other hand,
  \[
    m^*(E_n')\leq \sum_{i=n}^\infty m^*(E_i)<\varepsilon
  \]
  for every $\varepsilon$. Letting $\varepsilon$ go to $0$ we have
  $m^*(E)=0$.

  Lastly, we note that $E'=\liminf_{n\to\infty} E_n$ is a subset of
  $\limsup_{n\to\infty} E_n$, so that $m^*(E')=0$.
\end{solution}

\begin{problem}[Wheeden \& Zygmund Ch.\@ 3, Ex.\@ 10]
  If $E_1$ and $E_2$ are measurable, show that
  $m(E_1\cup E_2)+m(E_1\cap E_2)=m(E_1)+m(E_2)$.
\end{problem}
\begin{solution}
  We may, without loss of generality, assume that $m(E_1),m(E_2)<\infty$
  for otherwise there is nothing to show as equality holds trivially.

  Now, by Carathéodory's theorem we have the following characterization of
  measurability: a set $E$ is measurable if and only if for every set $A$
  we have
  \[
    m^*(A)=m^*(A\cap E)+m^*(A\setminus E).
  \]
  Therefore, the following equalities hold
  \begin{align*}
    m(E_1)&=m(E_1\cap E_2)+m(E_1\setminus E_2)\\
    m(E_2)&=m(E_1\cap E_2)+m(E_2\setminus E_1).
  \end{align*}
  Moreover, from elementary set theory we have
  \[
    (E_1\cup E_2)\setminus E_2=E_1\setminus (E_1\cap E_2),
  \]
  $E_1\subseteq E_1\cup E_2$ and $E_1\cap E_2\subseteq E_1$ so
  \[
    m(E_1\cup E_2)+m(E_1\cap E_2)=m(E_1)+m(E_2)
  \]
  as desired.
\end{solution}

%%% Local Variables:
%%% mode: latex
%%% TeX-master: "../MA544-Quals"
%%% End:
