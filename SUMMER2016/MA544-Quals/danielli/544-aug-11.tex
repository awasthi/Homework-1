\subsection{Danielli: Summer 2011}
\setcounter{exercise}{0}
\setcounter{equation}{0}

\begin{problem}
  Let \(f\in L^1(\bbR)\), and let
  \(\hat f(x)=\int_{\bbR} f(t)\cos(xt)\diff t\).
  \begin{enumerate}[label=(\alph*),noitemsep]
  \item Prove that \(\hat f(x)\) is continuous for \(x\in\bbR\).
  \item Prove the following \emph{Riemman--Lebesgue lemma}:
    \[
      \lim_{x\to\infty}\hat f(x)=0.
    \]
  \end{enumerate}
  \emph{Hint}: Start by proving the statement for \(f=\indicate_{[a,b]}\).
\end{problem}
\begin{solution}
  For part (a): let \(\varepsilon>0\) be given. Then, since \(\cos(xt)\) is
  continuous there exists \(\delta'>0\) such that \(|x-y|<\delta\) implies
  \[
    |\cos(xt)-\cos(yt)|<\frac{\varepsilon}{\|f\|_1}.
  \]
  Now, let \(\delta=\delta'\). Then we have
  \begin{align*}
    |\hat f(x)-\hat f(y)|
    &=\left|
      \int_\bbR f(t)\cos(xt)\diff t
      -\int_\bbR f(t)\cos(yt)\diff t
      \right|\\
    &\leq\int_\bbR|f(t)||\cos(xt)-\cos(yt)|\diff t\\
    &<\frac{\varepsilon}{\|f\|_1}\int_\bbR|f(t)|\diff t\\
    &=\frac{\varepsilon}{\|f\|_1}\|f\|_1\\
    &=\varepsilon.
  \end{align*}
  Since this can be done for any \(x\in\bbR\), \(\hat f\) is continuous on
  \(\bbR\).

  For part (b): since simple functions are dense in \(L^1(\bbR)\), \(f\)
  there exists a sequence of simple functions \(\{s_n\}\), \(n\in\bbN\),
  such that \(\int_\bbR s_n\to\|f\|_1\). Therefore, it suffices to prove
  the result for characteristic functions. Let \(f=\indicate_{[a,b]}\) and
  consider the limit
  \[
    \lim_{x\to\infty}\hat f(x)=\lim_{x\to\infty}\int_\bbR f(t)\cos(xt)\diff t.
  \]
  Since \(f=\indicate_{[a,b]}\), we have
  \begin{align*}
    \lim_{x\to\infty}\int_\bbR f(t)\cos(xt)\diff t
    &=\lim_{x\to\infty}\int_a^b\cos(xt)\diff t\\
    &=\lim_{x\to\infty}\biggl[\frac{1}{x}(\sin(xa)-\sin(xb))\biggr]\\
    &=\lim_{x\to\infty}\biggl[\frac{\sin(xa)}{x}-\frac{\sin(xb)}{x}\biggr]\\
    &=\biggl[\lim_{x\to\infty}\frac{\sin(xa)}{x}\biggr]
      -\biggl[\lim_{x\to\infty}\frac{\sin(xb)}{x}\biggr]\\
    &=1-1\\
    &=0,
  \end{align*}
  as we set out to show.
\end{solution}

\begin{problem}
  \hfill
  \begin{enumerate}[label=(\alph*),noitemsep]
  \item Suppose that \(f_k,f\in L^2(E)\), with \(E\) a measurable set, and
    that
    \[
      \label{eq:aug-11:1}
      \tag{\(\bigstar\)}
      \int_E f_kg\too\int_E fg
    \]
    as \(k\to\infty\) for all \(g\in L^2(E)\). If, in addition,
    \(\|f_k\|_2\to\|f\|_2\) show that \(f_k\) converges to \(f\) in
    \(L^2\), i.e., that
    \[
      \int_E|f-f_k|^2\too 0
    \]
    as \(k\to\infty\).
  \item Provide an example of a sequence \(f_k\) in \(L^2\) and a function
    \(f\) in \(L^2\) satisfying \eqref{eq:aug-11:1}, but such that \(f_k\)
    does \emph{not} converge to \(f\) in \(L^2\).
  \end{enumerate}
\end{problem}
\begin{solution}
  For part (a): expand the limit
  \begin{gather}
    \label{eq:aug-11:2-a}
    \begin{aligned}
      \lim_{n\to\infty}\int_E|f-f_n|^2\diff x
      &=\lim_{n\to\infty}\biggl[\int_E\bigl(|f|^2-2|ff_n|+|f|_n^2\bigr)\diff x\biggr]\\
      &=\lim_{n\to\infty}\biggl[\|f_n\|_2+\|f\|_2-2\int_E ff_n\diff x\biggr]\\
      &=\lim_{n\to\infty}\|f_n\|_2+\lim_{n\to\infty}\|f\|_2-2\lim_{n\to\infty}\int_E
      ff_n\diff x.
    \end{aligned}
  \end{gather}
  Since
  \[
    \int_Ef_ng\diff x\too \int_E fg\diff x
  \]
  for every \(g\in L^p(E)\),
  \[
    \int_E f_nf\diff x\too \int_Ef^2\diff x={\|f\|_2}^{\!2}.
  \]
  Moreover, \(\|f_n\|_2\to\|f\|_2\) so the limit in \eqref{eq:aug-11:2-a}
  converges to
  \[
    \lim_{n\to\infty}\|f_n\|_2+\lim_{n\to\infty}\|f\|_2-2\lim_{n\to\infty}\int_E
    ff_n\diff x
    =%
    \|f\|_2+\|f\|_2-2\|f\|_2=0
  \]
  as \(n\to\infty\).

  For part (b), consider the sequence \(\{f_n\}\), \(n\in\bbN\), where
  \(f_n(x)=\log(n)\exp(-nx)\). Then, we claim that
  \(f_n\xrightarrow{L^2[0,1]}0\), but that \(f_n\nrightarrow 0\)
  pointwise. To see the former, first note that
  \[
    \begin{aligned}
      \lim_{n\to\infty}\biggl[\int_0^1 f_n(x)\diff x\biggr]
      &=\lim_{n\to\infty}\biggl[\int_0^1\log(n)\exp(-nx)\diff x\biggr]\\
      &=\lim_{n\to\infty}\biggl[\log(n)\exp(-nx)\bigr|_0^1\biggr]\\
      &=\lim_{n\to\infty}\biggl[
      \frac{1}{n}\log(n)-\frac{1}{n}\log(n)\exp(-n)
      \biggr]\\
      &=\lim_{n\to\infty}\left[%
        \biggl(\frac{1-\exp(-n)}{n}\biggr)\log(n)%
      \right]\\
      &=0.
    \end{aligned}
  \]
  However, \(f_n\) does not converge to \(0\) a.e.
\end{solution}

\begin{problem}
  A bounded function \(f\) is said to be of bounded variation on \(\bbR\)
  if it is of bounded variation on any finite subinterval \([a,b]\), and
  moreover \(A\defeq\sup_{a,b}V[a,b;f]<\infty\). Here, \(V[a,b;f]\) denotes the
  total variation of \(f\) over the interval \([a,b]\). Show that:
\begin{enumerate}[label=(\alph*),noitemsep]
\item \(\displaystyle\int_{\bbR}|f(x+h)-f(x)|\diff x\leq A|h|\) for all
  \(h\in\bbR\).
  \\\\
  \emph{Hint}: For \(h>0\), write
  \[
    \int_{\bbR} |f(x+h)-f(x)|\diff x=
    \sum_{n=-\infty}^\infty\int_{nh}^{(n+1)h}|f(x+h)-f(x)|\diff x.
  \]
\item
  \(\displaystyle\left|\int_{\bbR} f(x)\varphi'(x)\diff x\right|\leq A\),
  where \(\varphi\) is any function of class \(C^1\), of bounded variation,
  compactly supported, with \(\sup_{x\in\bbR}|\varphi(x)|\leq 1\).
\end{enumerate}
\end{problem}
\begin{solution}
\end{solution}

\begin{problem}
  \hfill
  \begin{enumerate}[label=(\alph*),noitemsep]
  \item Prove the \emph{generalized Hölder's inequality}: Assume
    \(1\leq p\leq\infty\), \(j=1,\dotsc,n\), with
    \(\sum_{j=1}^\infty 1/p_j=1/r\leq 1\). If \(E\) is a measurable set and
    \(f_j\in L^{p_j}(E)\) for \(j=1,\dotsc,n\), then
    \(\prod_{j=1}^n f_j\in L^r(E)\) and
    \[
      \|f_1\dotsm f_n\|_r\leq\|f_1\|_{p_1}\dotsm\|f_n\|_{p_n}.
    \]
  \item Use part (a) to show that that if \(1\leq p,q,r\leq\infty\), with
    \(1/p+1/q=1/r+1\), \(f\in L^p(\bbR)\), and \(g\in L^p(\bbR)\), then
    \[
      |(f*g)(x)|^r\leq{\|f\|_p}^{\!r-p}{\|g\|_q}^{\!r-q}\int|f(y)|^p|g(x-y)|^q\diff
      y.
    \]
    \\\\
    (Recall that \((f*g)(x)=\int f(y)g(x-y)\diff y\).)
  \item Prove \emph{Young's convolution theorem}: Assume that \(p\), \(q\),
    \(r\), \(f\), and \(g\) are as in part (b). Then \(f*g\in L^r(\bbR)\)
    and
    \[
      \|f*g\|_r\leq\|f\|_p\|g\|_q.
    \]
  \end{enumerate}
\end{problem}
\begin{solution}
\end{solution}

%%% Local Variables:
%%% mode: latex
%%% TeX-master: "../MA544-Quals"
%%% End:
