\subsection{Danielli: Summer 2011}
\setcounter{exercise}{0}
\setcounter{equation}{0}

\begin{problem}
  Let \(f\in L^1(\bfR)\), and let \(F(t)=\int_{\bfR} f(x)\cos(tx)\diff x\).
\begin{enumerate}[label=(\alph*),noitemsep]
\item Prove that \(F(t)\) is continuous for \(t\in\bfR\).
\item Prove the following \emph{Riemman--Lebesgue lemma}:
\[
\lim_{t\to\infty}F(t)=0.
\]
\end{enumerate}
\emph{Hint}: Start by proving the statement for \(f=\chi_{[a,b]}\).
\end{problem}
\begin{solution}

\end{solution}

\begin{problem}
\begin{enumerate}[label=(\alph*),noitemsep]
\item Suppose that \(f_k,f\in L^2(E)\), with \(E\) a measurable set, and
  that
\begin{equation}
\label{eq:aug-11:1}
\int_E f_kg\longrightarrow\int_E fg
\end{equation}
as \(k\to\infty\) for all \(g\in L^2(E)\). If, in addition,
\(\|f_k\|_2\to\|f\|_2\) show that \(f_k\) converges to \(f\) in \(L^2\),
i.e., that
\[
\int_E|f-f_k|^2\longrightarrow 0
\]
as \(k\to\infty\).
\item Provide an example of a sequence \(f_k\) in \(L^2\) and a function
  \(f\) in \(L^2\) satisfying \eqref{eq:aug-11:1}, but such that \(f_k\)
  does \emph{not} converge to \(f\) in \(L^2\).
\end{enumerate}
\end{problem}
\begin{solution}
\end{solution}

\begin{problem}
  A bounded function \(f\) is said to be of bounded variation on \(\bfR\)
  if it is of bounded variation on any finite subinterval \([a,b]\), and
  moreover \(A=\sup_{a,b}V[a,b;f]<\infty\). Here, \(V[a,b;f]\) denotes the
  total variation of \(f\) over the interval \([a,b]\). Show that:
\begin{enumerate}[label=(\alph*),noitemsep]
\item \(\displaystyle\int_{\bfR}|f(x+h)-f(x)|\diff x\leq A|h|\) for all
  \(h\in\bfR\).
  \\\\
  \emph{Hint}: For \(h>0\), write
\[
\int_{\bfR} |f(x+h)-f(x)|\diff x=
\sum_{n=-\infty}^\infty\int_{nh}^{(n+1)h}|f(x+h)-f(x)|\diff x.
\]
\item
  \(\displaystyle\left|\int_{\bfR} f(x)\varphi'(x)\diff x\right|\leq A\),
  where \(\varphi\) is any function of class \(C^1\), of bounded variation,
  compactly supported, with \(\sup_{x\in\bfR}|\varphi(x)|\leq 1\).
\end{enumerate}
\end{problem}
\begin{solution}
\end{solution}

\begin{problem}
\begin{enumerate}[label=(\alph*),noitemsep]
\item Prove the \emph{generalized Hölder's inequality}: Assume
  \(1\leq p\leq\infty\), \(j=1,\dotsc,n\), with
  \(\sum_{j=1}^\infty 1/p_j=1/r\leq 1\). If \(E\) is a measurable set and
  \(f_j\in L^{p_j}(E)\) for \(j=1,\dotsc,n\), then
  \(\prod_{j=1}^n f_j\in L^r(E)\) and
    \[
      \|f_1\dotsm f_n\|_r\leq\|f_1\|_{p_1}\dotsm\|f_n\|_{p_n}.
    \]
  \item Use part (a) to show that that if \(1\leq p,q,r\leq\infty\), with
    \(1/p+1/q=1/r+1\), \(f\in L^p(\bfR)\), and \(g\in L^p(\bfR)\), then
\[
|(f*g)(x)|\leq{\|f\|_p}^{r-p}{\|g\|_q}^{r-q}\int|f(y)|^p|g(x-y)|^q\diff y.
\]
\\\\
(Recall that \((f*g)(x)=\int f(y)g(x-y)\diff y\).)
\item Prove \emph{Young's convolution theorem}: Assume that \(p\), \(q\),
  \(r\), \(f\), and \(g\) are as in part (b). Then \(f*g\in L^r(\bfR)\) and
  \[
    \|f*g\|_r\leq\|f\|_p\|g\|_q.
  \]
\end{enumerate}
\end{problem}
\begin{solution}
\end{solution}

%%% Local Variables:
%%% mode: latex
%%% TeX-master: "../MA544-Quals"
%%% End:
