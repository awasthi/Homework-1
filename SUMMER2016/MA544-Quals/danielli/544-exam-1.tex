\subsubsection{Exam 1}
\setcounter{exercise}{0}
\setcounter{equation}{0}

I lost this exam. These are the questions I could recall explicitly. For
the first problem, we were asked to show that the Dichlet function
\(\indicate_\bbQ(x)\) is not Riemann integrable and prove something about
\(\bbQ\). For the second question, we were asked to show that the measure
of countable union of disjoint measurable sets \(\{E_n:n\in\bbN\}\), is
equal to the sum of their individual measures (or something to that
effect).
\begin{problem}
\end{problem}
% \begin{solution}
% \end{solution}

\begin{problem}
\end{problem}
% \begin{solution}
% \end{solution}

\begin{problem}
\hfill
\begin{enumerate}[label=(\roman*)]
\item Show that if \(B_r=\left\{\,x\in\bbR^n:|x|<r\,\right\}\), then there
  exists a constant \(C\) such that \(|B_r|=Cr^n\). (\emph{Hint}: Think of
  \(B_r\) as \(\left\{\,rx:x\in B_1\,\right\}\).)
\item Let \(E\subseteq\bbR^n\) be a measurable set and let
  \(\varphi_E\colon\bbR^n\to\bbR\) be defined
  \(\varphi_E(x)=m(E\cap B_{|x|})\). Use part (i) to prove that
  \(\varphi_E\) is continuous.
\end{enumerate}
\end{problem}
\begin{solution}
  For part (i), as in the practice problems, define the linear map
  \(T\colon\bbR^n\to\bbR^n\) by \(T(x)=rx\). Note that this map is
  Lipschitz so the image of a measurable set \(E\) under \(T\) is
  measurable and \(m(T(E))=|{\det T}|m(E)=|r|^nm(E)\). It is not too
  difficult to see that
  \[
    T(B_1)=B_r
  \]
  as sets, so \(m(B_r)=|r|^nm(B_1)\). Now, let \(C=m(B_1)\).

  For part (ii), note that for any \(|x|,|y|\in\bbR\), by part (i), we have
  \begin{align*}
    |\varphi_E(x)-\varphi_E(y)|
    &\leq\bigr|C|x|-C|y|\bigl|\\
    &=C\bigl||x|-|y|\bigr|.
  \end{align*}
  In particular, given \(\varepsilon>0\), there exists \(\delta>0\) such
  that
  \[
    \bigl||x|-|y|\bigr|<\frac{\varepsilon}{C}.
  \]
  Thus,
  \begin{align*}
    |\varphi_E(x)-\varphi_E(y)|
    &<C\left(\frac{\varepsilon}{C}\right)\\
    &=\varepsilon
  \end{align*}
  and \(\varphi_E\) is continuous. Check this \(\emptyset\) out
\end{solution}

\begin{problem}
  Assume that \(f\colon[a,b]\to\bbR\) is of bounded variation on
  \([a,b]\). Prove that \(f\) is measurable.
\end{problem}
\begin{solution}
  Suppose that \(f\colon[a,b]\to\bbR\) is b.v.\@ on \([a,b]\). Then \(f\),
  by Jordan's theorem, \(f=g-h\) where \(g\) and \(h\) are monotone
  increasing functions. Since monotone functions are a.e.\@ continuous,
  \(g\) and \(h\) are measurable functions. Thus, \(f\) is measurable.
\end{solution}

%%% Local Variables:
%%% mode: latex
%%% TeX-master: "../MA544-Quals"
%%% End:
