\subsubsection{Exam 1}
\setcounter{exercise}{0}
\setcounter{equation}{0}

I lost this exam. These are the questions I could recall explicitly. For
the first problem, we were asked to show that the Dichlet function
\(\indicate_\bbQ(x)\) is not Riemann integrable and prove something about
\(\bbQ\). For the second question, we were asked to show that the measure
of countable union of disjoint measurable sets \(\{E_n:n\in\bbN\}\), is
equal to the sum of their individual measures (or something to that
effect).
\begin{problem}
\end{problem}
% \begin{solution}
% \end{solution}

\begin{problem}
\end{problem}
% \begin{solution}
% \end{solution}

\begin{problem}
\hfill
\begin{enumerate}[label=(\roman*),noitemsep]
\item Show that if \(B_r=\left\{\,x\in\bbR^n:|x|<r\,\right\}\), then there
  exists a constant \(C\) such that \(|B_r|=Cr^n\).
  \\\\
  (\emph{Hint}: Think of \(B_r\) as \(\left\{\,rx:x\in B_1\,\right\}\).)
\item Let \(E\subseteq\bbR^n\) be a measurable set and let
  \(\varphi_E\colon\bbR^n\to\bbR\) be defined
  \(\varphi_E(x)=m(E\cap B_{|x|})\). Use part (i) to prove that
  \(\varphi_E\) is continuous.
\end{enumerate}
\end{problem}
\begin{solution}
  For part (i), as in the practice problems, define the linear map
  \(T\colon\bbR^n\to\bbR^n\) by \(T(x)=rx\). Note that this map is
  Lipschitz so the image of a measurable set \(E\) under \(T\) is
  measurable and \(m(T(E))=|{\det T}|m(E)=|r|^nm(E)\). It is not too
  difficult to see that
  \[
    T(B_1)=B_r
  \]
  as sets, so \(m(B_r)=|r|^nm(B_1)\). Now, let \(C=m(B_1)\).

  For part (ii), note that we can write the map \(\varphi_E\) as
\end{solution}

\begin{problem}
  Assume that \(f\colon[a,b]\to\bbR\) is of bounded variation on
  \([a,b]\). Prove that \(f\) is measurable.
\end{problem}
\begin{solution}
\end{solution}

%%% Local Variables:
%%% mode: latex
%%% TeX-master: "../MA544-Quals"
%%% End:
