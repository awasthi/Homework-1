\subsection{Homework 10}
\begin{problem}[Wheeden \& Zygmund Ch.\@ 7, Ex.\@ 1]
Let $f$ be measurable in $\bbR^n$ and different from zero in some set of
positive measure. Show that there is a positive constant $c$ such that
$f^*(\bfx)\geq c\|\bfx\|^{-n}$ for $\|\bfx\|\geq 1$.
\end{problem}
\begin{solution}
\end{solution}

\begin{problem}[Wheeden \& Zygmund Ch.\@ 7, Ex.\@ 2]
Let $\varphi(\bfx),\bfx\in\bbR^n$, be a bounded measurable function such
that $\varphi(\bfx)=0$ for $\|\bfx\|\geq 1$ and $\int\varphi=1$. For
$\varepsilon>0$, let
$\varphi_\varepsilon(\bfx)=\varepsilon^{-n}\varphi(\bfx/\varepsilon)$. ($\varphi_\varepsilon$
is called an \emph{approximation to the identity}.) If $f\in L(\bbR^n)$,
show that
\[
\lim_{\varepsilon\to 0}(f*\varphi_\varepsilon)(\bfx)=f(\bfx)
\]
in the Lebesgue set of $f$. (Note that $\int\varphi_\varepsilon=1$,
$\varepsilon>0$, so that
\[
(f*\varphi_\varepsilon)(\bfx)-f(\bfx)=\int\left[f(\bfx-\bfy)-f(\bfx)\right]\varphi_\varepsilon(\bfy)\diff\bfy.
\]
Use Theorem 7.16.)
\end{problem}
\begin{solution}
\end{solution}

\begin{problem}[Wheeden \& Zygmund Ch.\@ 7, Ex.\@ 6]
Show that if $\alpha>0$, then $x^\alpha$ is absolutely continuous on every
bounded subinterval of $[0,\infty)$.
\end{problem}
\begin{solution}
\end{solution}

\begin{problem}[Wheeden \& Zygmund Ch.\@ 7, Ex.\@ 8]
Prove the following converse of Theorem 7.31: If $f$ is of bounded
variation on $[a,b]$, and if the function $V(x)=V[a,x]$ is absolutely
continuous on $[a,b]$, then $f$ is absolutely continuous on $[a,b]$.
\end{problem}
\begin{solution}
\end{solution}

\begin{problem}[Wheeden \& Zygmund Ch.\@ 7, Ex.\@ 9]
If $f$ is of bounded variation on $[a,b]$, show that
\[
\int_a^b|f'|\leq V[a,b].
\]
Show that if equality holds in this inequality, then $f$ is absolutely
continuous on $[a,b]$. (For the second part, use Theorems 2.2(ii) and 7.24
to show that $V(x)$ is absolutely continuous and then use the result of
Exercise 8).
\end{problem}
\begin{solution}
\end{solution}

\begin{problem}[Wheeden \& Zygmund Ch.\@ 7, Ex.\@ 12]
Use Jensen's inequality to prove that if $a,b\geq 0$, $p,q>1$,
$(1/p)+(1/q)=1$, then
\[
ab\leq\frac{a^p}{p}+\frac{b^q}{q}.
\]
More generally, show that
\[
a_1\dotsm a_N=\sum_{j=1}^N\frac{{a_j}^{p_j}}{p_j},
\]
where $a_j\geq 0$, $p_j>1$, $\sum_{j=1}^N(1/p_j)=1$. (Write
$a_j=e^{x_j/p_j}$ and use the convexity of $e^x$).
\end{problem}
\begin{solution}
\end{solution}

\begin{problem}[Wheeden \& Zygmund Ch.\@ 7, Ex.\@ 13]
Prove Theorem 7.36.
\end{problem}
\begin{solution}
Recall the statement of Theorem 7.36
\begin{quote}
\begin{enumerate}[label=\textnormal{(\roman*)},noitemsep]
\item If $\varphi_1$ and $\varphi_2$ are convex in $(a,b)$, then
  $\varphi_1+\varphi_2$ is convex in $(a,b)$.
\item If $\varphi$ is convex in $(a,b)$ and $c$ is a positive constant,
  then $c\varphi$ is convex in $(a,b)$.
\item If $\varphi_k$, $k=1,2,\dotsc$, are convex in $(a,b)$ and
  $\varphi_k\to\varphi$ in $(a,b)$, then $\varphi$ is convex in $(a,b)$.
\end{enumerate}
\end{quote}
\end{solution}

%%% Local Variables:
%%% mode: latex
%%% TeX-master: "../MA544-Quals"
%%% End:
