\subsection{Homework 9}
\begin{problem}[Wheeden \& Zygmund Ch.\@ 6, Ex.\@ 1]
\begin{enumerate}[label=(\alph*)]
\item Let $E$ be a measurable subset of $\bbR^2$ such that for almost every
  $x\in\bbR$, $\left\{\,y:(x,y)\in E\,\right\}$ has
  $\bbR$-measure zero. Show that $E$ has measure zero and that for almost
  every $y\in\bbR$, $\left\{\,x:(x,y)\in E\,\right\}$ has
  measure zero.
\item Let $f(x,y)$ be nonnegative and measurable in $\bbR^2$. Suppose that
  for almost every $x\in\bbR$, $f(x,y)$ is finite for almost every
  $y$. Show that for almost $y\in\bbR$, $f(x,y)$ is finite for almost
  every $x$.
\end{enumerate}
\end{problem}
\begin{proof}
\begin{russian}
Привет друзья, меня зовут Карлос.
\end{russian}
\end{proof}

\begin{problem}[Wheeden \& Zygmund Ch.\@ 6, Ex.\@ 3]
Let $f$ be measurable and finite a.e.\@ on $[0,1]$. If $f(x)-f(y)$ is
integrable over the square $0\leq x\leq 1$, $0\leq y\leq 1$, show that
$f\in L[0,1]$.
\end{problem}
\begin{proof}
\end{proof}

\begin{problem}[Wheeden \& Zygmund Ch.\@ 6, Ex.\@ 4]
Let $f$ be measurable and periodic with period $1$: $f(t+1)=f(t)$. Suppose
there is a finite $c$ such that
\[
\int_0^1|f(a+t)-f(b+t)|\diff t\leq c
\]
for all $a$ and $b$. Show that $f\in L[0,1]$. (Set $a=x$, $b=-x$, integrate
with respect to $x$, and make the change of variables $\xi=x+t$,
$\eta=-x+t$.)
\end{problem}
\begin{proof}
\end{proof}

\begin{problem}[Wheeden \& Zygmund Ch.\@ 6, Ex.\@ 6]
For $f\in L(\bbR)$, define the \emph{Fourier transform $\hat f$} of $f$
by
\[
\hat f(x)=\frac{1}{2\pi}\int_{-\infty}^\infty f(t)e^{-ixt}\diff t
\]
for $x\in\bbR$. (For complex-valued function $F=F_0+iF_1$ whose real and
imaginary parts $F_0$ and $F_1$ are integrable, we define $\int F=\int
F_0+i\int F_1$.) Show that if $f$ and $g$ belong to $L(\bbR)$, then
\[
\widehat{(f*g)}(x)=2\pi\hat f(x)\hat g(x).
\]
\end{problem}
\begin{proof}
\end{proof}

\begin{problem}[Wheeden \& Zygmund Ch.\@ 6, Ex.\@ 7]
Let $F$ be a closed subset of $\bbR$ and let $\delta(x)=\delta(x,F)$ be
the corresponding distance function. If $\lambda>0$ and $f$ is nonnegative
and integrable over the complement of $F$, prove that the function
\[
\int_{\bbR}\frac{\delta^\lambda(y)f(y)}{|x-y|^{1+\lambda}}\diff
t
\]
is integrable over $F$ and so is finite a.e.\@ in $F$. (In case
$f=\chi_{(a,b)}$, this reduces to Theorem 6.17.)
\end{problem}
\begin{proof}
\end{proof}

\begin{problem}[Wheeden \& Zygmund Ch.\@ 6, Ex.\@ 9]
\begin{enumerate}[label=(\alph*)]
\item Show that $M_\lambda(x;F)=+\infty$ if $x\notin F$, $\lambda>0$.
\item Let $F=[c,d]$ be a closed subinterval of a bounded open interval
  $(a,b)\subset\bbR$, and let $M_\alpha$ be the corresponding
  Marcinkiewicz integral, $\lambda>0$. Show that $M_\lambda$ is finite for
  every $x\in(c,d)$ and that $M_\lambda(c)=M_\lambda(d)=\infty$. Show also
  that $\int M_\lambda\leq\lambda^{-1}|G|$, where $G=(a,b)-[c,d]$.
\end{enumerate}
\end{problem}
\begin{proof}
\end{proof}

%%% Local Variables:
%%% mode: latex
%%% TeX-master: "../MA544-Quals"
%%% End:
