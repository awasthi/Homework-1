\subsubsection{Exam 2 (2010)}
\setcounter{exercise}{0}
\setcounter{equation}{0}

\begin{problem}
  Suppose \(f\in L^1(\bbR^n)\). Show that for every \(\varepsilon>0\) there
  exists a ball \(B\), centered at the origin, such that
  \[
    \int\limsclap{\bbR^n\setminus B}|f|<\varepsilon.
  \]
  \\\\
  \emph{Hint}: Use the monotone convergence theorem.
\end{problem}
\begin{solution}
  Consider the sequence of functions \(\{f_n(x):n\in\bbN\}\) where
  \[
    f_n(x)=|f(x)|\indicate_{B(\mathbf{0},n)}(x).
  \]
  Then, \(f_n\uparrow |f|\) so by the monotone convergence theorem, given
  \(\varepsilon>0\), there exists an index \(N\in\bbN\) such that
  \(n\geq N\) implies
  \begin{align*}
    \int_{\bbR^n}|f(x)|\diff x-\int_{\bbR^n}f_n(x)\diff x
    &=\int_{\bbR^n}|f(x)|-|f(x)|\indicate_{B(\mathbf{0},n)}\diff x\\
    &=\int\limsclap{\bbR^n\setminus B(\mathbf{0},n)}|f(x)|\diff x\\
    &<\varepsilon.
  \end{align*}
  Let \(B=B(\mathbf{0},N+1)\).
\end{solution}
\begin{problem}
  \hfill
  \begin{enumerate}[label=(\alph*)]
  \item Prove the following generalization of \emph{Chebyshev's
      inequality}: Let \(0<p<\infty\) and \(E\subseteq\bbR^n\) be
    measurable. assume that \(|f|^p\in L^1(E)\). Then
    \[
      m\bigl\{\,x\in E:f(x)>\alpha\,\bigr\}
      \leq\frac{1}{\alpha^p}\int_{\left\{\,f>\alpha\,\right\}}f^p,
    \]
    for \(\alpha>0\).
  \item Let \(p\), \(E\), and \(f\) be as in part (a). In addition, assume
    that \(\{f_k\}\) is a sequence such that \(\int_E|f_k-f|^p\to 0\) as
    \(k\to\infty\). Show that \(f_k\to f\) in measure on \(E\).
    \\\\
    Recall that \(f_k\to f\) in measure on \(E\) if and only if for every
    \(\varepsilon>0\)
    \[
      \lim_{k\to\infty}m\bigl\{\,x\in
      E:|f_k(x)-f(x)|>\varepsilon\,\bigr\}=0.
    \]
\end{enumerate}
\end{problem}
\begin{solution}
\end{solution}

\begin{problem}
  Let \(f\in L^1(\bbR)\), and define
  \[
    F(\xi)=\int_{\bbR} f(x)\cos(2\pi x\xi)\diff x.
  \]
  Prove that \(F\) is continuous and bounded on \(\bbR\).
\end{problem}
\begin{solution}
\end{solution}

\begin{problem}
  Use repeated integration techniques to prove that
  \[
    \int_{\bbR^n} \rme^{-|x|^2}\diff x=\pi^{n/2}.
  \]
  \\\\
  \emph{Hint}: Start from the case \(n=1\) by using the polar coordinates
  in
  \[
    \left[\int_{\bbR} e^{-x^2}\diff x\right]^2=%
    \left[\int_{\bbR} e^{-x^2}\diff x\right]%
    \left[\int_{\bbR} e^{-x^2}\diff y\right]%
  \]
\end{problem}
\begin{solution}
\end{solution}

\begin{problem}
\end{problem}
\begin{solution}
\end{solution}

%%% Local Variables:
%%% mode: latex
%%% TeX-master: "../MA544-Quals"
%%% End:
