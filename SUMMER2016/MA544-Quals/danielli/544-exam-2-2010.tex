\subsubsection{Exam 2 (2010)}
\setcounter{exercise}{0}
\setcounter{equation}{0}

\begin{problem}
  Suppose \(f\in L^1(\bbR^n)\). Show that for every \(\varepsilon>0\) there
  exists a ball \(B\), centered at the origin, such that
  \[
    \int\limsclap{\bbR^n\setminus B}|f|<\varepsilon.
  \]
  \\\\
  \emph{Hint}: Use the monotone convergence theorem.
\end{problem}
\begin{solution}
  Consider the sequence of functions \(\{f_n(x):n\in\bbN\}\) where
  \[
    f_n(x)=|f(x)|\indicate_{B(\mathbf{0},n)}(x).
  \]
  Then, \(f_n\uparrow |f|\) so by the monotone convergence theorem, given
  \(\varepsilon>0\), there exists an index \(N\in\bbN\) such that
  \(n\geq N\) implies
  \begin{align*}
    \int_{\bbR^n}|f(x)|\diff x-\int_{\bbR^n}f_n(x)\diff x
    &=\int_{\bbR^n}|f(x)|-|f(x)|\indicate_{B(\mathbf{0},n)}\diff x\\
    &=\int\limsclap{\bbR^n\setminus B(\mathbf{0},n)}|f(x)|\diff x\\
    &<\varepsilon.
  \end{align*}
  Let \(B=B(\mathbf{0},N+1)\).
\end{solution}
\begin{problem}
  \hfill
  \begin{enumerate}[label=(\alph*)]
  \item Prove the following generalization of \emph{Chebyshev's
      inequality}: Let \(0<p<\infty\) and \(E\subseteq\bbR^n\) be
    measurable. Assume that \(|f|^p\in L^1(E)\). Then
    \[
      m\bigl\{\,x\in E:f(x)>\alpha\,\bigr\}
      \leq\frac{1}{\alpha^p}\int\limsclap{\left\{\,f>\alpha\,\right\}}f^p,
    \]
    for \(\alpha>0\).
  \item Let \(p\), \(E\), and \(f\) be as in part (a). In addition, assume
    that \(\{f_k\}\) is a sequence such that \(\int_E|f_k-f|^p\to 0\) as
    \(k\to\infty\). Show that \(f_k\to f\) in measure on \(E\).
    \\\\
    Recall that \(f_k\to f\) in measure on \(E\) if and only if for every
    \(\varepsilon>0\)
    \[
      \lim_{k\to\infty}m\bigl\{\,x\in
      E:|f_k(x)-f(x)|>\varepsilon\,\bigr\}=0.
    \]
\end{enumerate}
\end{problem}
\begin{solution}
  Part (a) is almost trivial. Let
  \[
    E_\alpha=\bigl\{\,x\in E:f(x)>\alpha\,\bigr\}.
  \]
  Then, \(|f|^p\geq \alpha^p\) for all \(x\in E_\alpha\). Thus,
  \[
    \int_{E_\alpha}\alpha^p\diff x=\alpha^pm(E_\alpha)\leq\int_{E_\alpha}|f|^p\diff x,
  \]
  as was to be shown.

  Part (b) follows directly from Chebyshev's inequality, as
  \[
    \lim_{n\to\infty}m(E_\varepsilon) <\lim_{n\to\infty}
    \left[\frac{1}{\varepsilon^p}\int_{E_\varepsilon}|f_n(x)-f(x)|^p\diff
      x\right]=
    \frac{1}{\varepsilon^p}\lim_{n\to\infty}\int_{E_\varepsilon}|f_n(x)-f(x)|\diff
    x=0,
  \]
  where \(E_\varepsilon=\bigl\{\,x\in
  E:|f_n(x)-f(x)|>\varepsilon\,\bigr\}\).
\end{solution}

\begin{problem}
  Let \(f\in L^1(\bbR)\), and define
  \[
    F(\xi)=\int_{\bbR} f(x)\cos(2\pi x\xi)\diff x.
  \]
  Prove that \(F\) is continuous and bounded on \(\bbR\).
\end{problem}
\begin{solution}
  It is easy to see that \(F\) is bounded as \(|{\cos(2\pi x\xi)}|<1\) for
  all \(\chi\in\bbR\) so
  \[
    |F(\xi)|= \left| \int_\bbR f(x)\cos(2\pi x\xi)\diff x \right \leq
    \left|\int_\bbR f(x)\diff x\right|\leq \|f\|_1.
  \]

  To see that \(F\) is in fact continuous, note that since \(\cos(2\pi
  x\xi)\) is continuous, as a function of \(\xi\), given \(\varepsilon>0\)
  there exist \(\delta>0\) such that \(|\xi-\chi|<\delta\) implies
  \[
    |{\cos(2\pi x\xi)-\cos(2\pi x\chi)}|<\frac{\varepsilon}{\|f\|_1}.
  \]
  Thus, for \(|\xi-\chi|<\delta\), we have
  \begin{align*}
    |F(\xi)-F(\xi)|
    &=\left|
      \int_\bbR f(x)\cos(2\pi x\xi)\diff x
      -\int_\bbR f(x)\cos(2\pi x\chi)\diff x
      \right|\\
    &\leq
      \int_\bbR\Bigl|f(x)\bigl(\cos(2\pi x\xi)-\cos(2\pi x\chi)\bigr)\Bigr|\diff x
    \\
    &\leq \frac{\varepsilon}{\|f\|_1}\int_\bbR |f(x)|\diff x\\
    &=\varepsilon.
  \end{align*}
  Thus, \(F\) is continuous.
\end{solution}

\begin{problem}
  Use repeated integration techniques to prove that
  \[
    \int_{\bbR^n} \rme^{-|x|^2}\diff x=\pi^{n/2}.
  \]
  \\\\
  \emph{Hint}: Start from the case \(n=1\) by using the polar coordinates
  in
  \[
    \left[\int_{\bbR} \rme^{-x^2}\diff x\right]^2=%
    \left[\int_{\bbR} \rme^{-x^2}\diff x\right]%
    \left[\int_{\bbR} \rme^{-x^2}\diff y\right]%
  \]
\end{problem}
\begin{solution}
  We shall proceed by induction. For the case \(n=1\), note that
  \begin{align*}
    \left[\int_{\bbR} \rme^{-x^2}\diff x\right]^2
    &=\left[\int_{\bbR} \rme^{-x^2}\diff x\right]%
      \left[\int_{\bbR} \rme^{-x^2}\diff y\right]\\
    &=\int\limsclap{\bbR\times\bbR}\rme^{-x^2+y^2}\diff x\rmd y.
  \end{align*}
  Making a smooth change of variables via polar coordinates, we have
  \[
    \left[\int_{\bbR} \rme^{-x^2}\diff x\right]^2
    =\int_0^{2\pi}\int_0^\infty\rme^{-r^2}r\diff r\diff\theta=\pi.
  \]
  Thus,
  \[
    \int_\bbR\rme^{-x^2}\diff x=\sqrt{\pi}.
  \]

  Now, suppose
  \[
    \int_{\bbR^k}\rme^{\cramped{-{x_1}^2-\dotsb-{x_k}^2}}\diff
    x_1\dotsm\rmd x_k=\sqrt[k]{\pi}
  \]
  for all \(k\leq n-1\). Then, for the case \(n\), we have
  \begin{align*}
    \int_{\bbR^n}\rme^{\cramped{-{x_1}^2\dotsb-{x_{n-1}}^2-{x_n}^2}}\diff x
    &=\int_{\bbR}\rme^{\cramped{-{x_1}^2\dotsb-{x_{n-1}}^2}}\rme^{\cramped{-{x_n}^2}}\diff
      x_1\dotsm\rmd x_{n-1}\rmd x_n
  \end{align*}
\end{solution}

%%% Local Variables:
%%% mode: latex
%%% TeX-master: "../MA544-Quals"
%%% End:
