\subsubsection{Exam 2 Practice Problems}
\setcounter{exercise}{0}
\setcounter{equation}{0}

\begin{problem}
  Define for \( x \in\bbR^n\),
  \[
    f(x)=
    \begin{cases}
      |x|^{-(n+1)}&\text{if \(x\neq\mathbf{0}\),}\\
      0&\text{if \(x=\mathbf{0}\).}
    \end{cases}
  \]
  Prove that \(f\) is integrable outside any ball
  \(B(\mathbf{0},\varepsilon)\), and that there exists a constant \(C>0\)
  such that
  \[
    \int\limits_{\bbR^n\setminus B(\mathbf{0},\varepsilon)}f(x)\diff x
    \leq\frac{C}{\varepsilon}.
  \]
\end{problem}
\begin{solution}
  Danielli gave a wonderful solution to this problem by using spherical
  coordinates to compute the integral. However, she did not justify the use
  of polar coordinates, or even make it clear what exactly the meaning of
  \(\rmd x\) and \(\rmd\sigma\) mean in this context. The justification
  comes from the polar decomposition of the Lebesgue measure (we shall not
  prove this here). Changing the integral in question to polar coordinates,
  we have
  \begin{align*}
    \int\limits_{\mathclap{\bbR^n\setminus B(\mathbf{0},\varepsilon)}}f(x)\diff x
    &=\int\limits_{S^{n-1}(\mathbf{0},r)}\int_\varepsilon^\infty
      \frac{1}{r^{n+1}}\diff r\rmd\sigma
      \intertext{which, by Tonelli's theorem, becomes}
    &=\int_\varepsilon^\infty
      \frac{1}{r^{n+1}}
      \left[\int_{S^{n-1}(\mathbf{0},r)}\rmd\sigma\right]\rmd r
      \intertext{which, by a problem from the homework, becomes}
    &=C_n\int_\varepsilon^\infty r^{n-1}\left[\frac{1}{r^{n+1}}\right]\rmd
      r\\
    &=C_n\int_\varepsilon^\infty\frac{1}{r^2}\diff r\\
    &=\frac{C_n}{\varepsilon},
  \end{align*}
  as desired.
\end{solution}

\begin{problem}
  Let \(\left\{f_k\right\}\) be a sequence of nonnegative measurable
  functions on \(\bbR^n\), and assume that \(f_k\) converges pointwise
  almost everywhere to a function \(f\). If
  \[
    \int_{\bbR^n} f=\lim_{k\to\infty}\int_{\bbR^n} f_k<\infty,
  \]
  show that
  \[
    \int_E f=\lim_{k\to\infty}\int_E f_k
  \]
  for all measurable subsets \(E\) of \(\bbR^n\). Moreover, show that this
  is not necessarily true if
  \(\int_{\bbR^n} f=\lim_{k\to\infty}\int_{\bbR^n} f_k=\infty\).
\end{problem}
\begin{solution}
  Let \(E\) be a measurable subset of \(\bbR^n\). First, note that since
  \(f_n\to f\) a.e.\@ in \(\bbR^n\), we clearly have
  \(f_n\restrict E\to f\restrict E\) and
  \(f_n\restrict{\bbR^n\setminus E}\to f\restrict{\bbR^n\setminus
    E}\). By the monotonicity of the Lebesgue integral,
  \[
    \lim_{n\to\infty}\int_{\bbR^n}f\diff x
    =\lim_{n\to\infty}
    \left[
      \int_E f_n\diff x+
      \int_{\bbR^n\setminus E}f_n\diff x
    \right]
  \]
  and by basic properties of the limit supremum, we have
  % \begin{equation}
  %   \label{eq:ex-2-prep:limsup-props}
  \begin{align*}
    \limsup_{n\to\infty}\int_E f_n\diff x
    +\liminf_{n\to\infty}\int_{\bbR^n\setminus E}f_n\diff x\\
    &\leq%
      \int_{\bbR^n}f\diff x\\
    &=
      \int_E f\diff x
      +\int_{\bbR^n\setminus E}f\diff x\\
    &\leq
    \limsup_{n\to\infty}\int_E f_n\diff x
    +\limsup_{n\to\infty}\int_{\bbR^n\setminus E} f_n\diff x.
  \end{align*}
  % \end{equation}
  by Fatou's lemma, we have
  \begin{align*}
    \int_Ef\diff x%
    &=\int_E\liminf_{n\to\infty} f_n\diff x\\
    &\leq\liminf_{n\to\infty}\int_E f_n\diff x\\
    \int\limits_{\mathclap{\bbR^n\setminus E}}f\diff x
    &=\int\limits_{\mathclap{\bbR^n\setminus E}}\liminf_{n\to\infty} f_n\diff x\\
    &\leq\liminf_{n\to\infty}\int\limits_{\mathclap{\bbR^n\setminus E}} f_n\diff x,
  \end{align*}
  both of which are finite since \(f\in L^1(\bbR^n)\) and
  \(\lim_{n\to\infty}\int_{\bbR^n}f_n\diff x<\infty\); by the monotonicity
  of the Lebesgue integral.

\end{solution}

\begin{problem}
  Assume that \(E\) is a measurable set of \(\bbR^n\), with
  \(|E|<\infty\). Prove that a nonnegative function \(f\) defined on \(E\)
  is integrable if and only if
  \[
    \sum_{k=0}^\infty m\bigl\{\, x \in E:f( x )\geq k\,\bigr\}<\infty.
  \]
\end{problem}
\begin{solution}
\end{solution}

\begin{problem}
  Suppose that \(E\) is a measurable subset of \(\bbR^n\), with
  \(|E|<\infty\). If \(f\) and \(g\) are measurable functions on \(E\),
  define
  \[
    \rho(f,g)=\int_E\frac{|f-g|}{1+|f-g|}.
  \]
  Prove that \(\rho(f_k,f)\to 0\) as \(k\to\infty\) if and only if \(f_k\)
  converges to \(f\) as \(k\to\infty\).
\end{problem}
\begin{solution}
\end{solution}

\begin{problem}
  Define the \emph{gamma function} \(\Gamma\colon\bbR^+\to\bbR\) by
  \[
    \Gamma(y)=\int_0^\infty e^{-u}u^{y-1}\diff u,
  \]
  and the \emph{beta function} \(\beta\colon\bbR^+\times\bbR^+\to\bbR\) by
  \[
    \beta(x,y)=\int_0^1 t^{x-1}(1-t)^{y-1}\diff t.
  \]
  \begin{enumerate}[label=(\alph*)]
  \item Prove that the definition of the gamma function is well-posed,
    i.e., the function \(u\mapsto e^{-u}u^{y-1}\) is in \(L(\bbR^+)\) for
    all \(y\in\bbR^+\).
  \item Show that
    \[
      \beta(x,y)=\frac{\Gamma(x)\Gamma(y)}{\Gamma(x+y)}.
    \]
  \end{enumerate}
\end{problem}
\begin{solution}
\end{solution}

\begin{problem}
  Let \(f\in L(\bbR^n)\) and for \(\mathbf{h}\in\bbR^n\) define
  \(f_{\mathbf{h}}\colon\bbR^n\to\bbR\) be
  \(f_{\mathbf{h}}( x )= f( x -\mathbf{h})\). Prove that
  \[
    \lim_{\mathbf{h}\to\mathbf{0}}\int_{\bbR^n}\left|f_{\mathbf{h}}-f\right|=0.
  \]
\end{problem}
\begin{solution}
\end{solution}

\begin{problem}
\begin{enumerate}[label=(\alph*)]
\item If \(f_k,g_k,f,g\in L(\bbR^n)\), \(f_k\to f\) and \(g_k\to g\) a.e.\@
  in \(\bbR^n\), \(|f_k|\leq g_k\) and
  \[
    \int_{\bbR^n}g_k\longrightarrow\int_{\bbR^n}g,
  \]

  prove that
  \[
    \int_{\bbR^n} f_k\longrightarrow\int_{\bbR^n}f.
  \]
\item Using part (a) show that if \(f_k,f\in L(\bbR^n)\) and \(f_k\to f\)
  a.e.\@ in \(\bbR^n\), then
  \[
    \int_{\bbR^n}|f_k-f|\longrightarrow 0\qquad\text{as \(k\to\infty\)}
  \]
  if and only if
  \[
    \int_{\bbR^n}|f_k|\longrightarrow\int_{\bbR^n}|f|\qquad\text{as
      \(k\to\infty\)}.
  \]
\end{enumerate}
\end{problem}
\begin{solution}
\end{solution}

%%% Local Variables:
%%% mode: latex
%%% TeX-master: "../MA544-Quals"
%%% End:
