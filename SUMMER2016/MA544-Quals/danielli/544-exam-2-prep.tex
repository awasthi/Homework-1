\subsubsection{Exam 2 Practice Problems}
\setcounter{exercise}{0}

\begin{problem}
Define for $\bfx\in\bbR^n$,
\[
f(\bfx)=
\begin{cases}
\left|\bfx\right|^{-(n+1)}&\text{if $\bfx\neq \mathbf{0}$,}\\
0&\text{if $\bfx=\mathbf{0}$.}
\end{cases}
\]
Prove that $f$ is integrable outside any ball $B(\mathbf{0},\varepsilon)$,
and that there exists a constant $C>0$ such that
\[
\int_{\bbR^n\smallsetminus B(\mathbf{0},\varepsilon)}f(\bfx) \diff \bfx\leq\frac{C}{\varepsilon}.
\]
\end{problem}
\begin{proof}
Recall that a real-valued function $f\colon\bbR^n\to\bbR$ is
\href{https://en.wikipedia.org/wiki/Fubini's_theorem#Tonelli.27s_theorem_for_non-negative_functions}{Lebesgueintegrable}
on a subset $E$ of $\bbR^n$ if
\begin{equation}
\label{eq:prep:2:1}
\int_E f(\bfx)\diff\bfx<\infty.
\end{equation}
Let $f$ be as given in the statement of the problem and set
$B_\varepsilon= B(\mathbf{0},\varepsilon)$. Consider
the change of variables to
\href{https://en.wikipedia.org/wiki/N-sphere#Spherical_coordinates}{\emph{hyperspherical
    coordinates}} $(x_1,\dotsc,x_n)\mapsto(r,\Theta)$ where
$\Theta=(\theta_1,\dotsc,\theta_{n-1})$.\footnote{The explicit construction
  of the map $(x_1,\dotsc,x_n)\mapsto(r,\Theta)$ is of no concern to us for
  now. What is important is that it exists.} By Theorem 7.26(iii) from
\cite[Ch.\@ 7, p.\@ 123]{rudin-2}, we have
\begin{equation}
\label{eq:prep:2:2}
\begin{aligned}
\int_{\bbR^n\smallsetminus B_\varepsilon}f(\bfx)\diff\bfx
&=\int_{\bbR^n\smallsetminus B_\varepsilon} f(\bfx)\diff\bfx\\
&=\int_{\bbR^n\smallsetminus
  B_\varepsilon}\frac{1}{|\bfx|^{n+1}}\diff\bfx.\\
&=\int_{S^{n-1}_r}\int_\varepsilon^\infty\frac{1}{|r|^{n+1}} \diff r dV,
\end{aligned}
\end{equation}
where $S_r^{n-1}$ is the $(n-1)$-sphere centered at $\mathbf{0}$ with
radius $r$, that is, the subset $\left\{\,\bfx\in\bbR^n:|\bfx|=1\,\right\}$
of $\bbR^n$ and $dV$ is the
\href{https://en.wikipedia.org/wiki/Volume_element}{\emph{volume element}}
of $S_r^{n-1}$. Since $1/|r|^{n+1}$ is nonnegative, by
\href{https://en.wikipedia.org/wiki/Fubini's_theorem#Tonelli.27s_theorem}{Tonelli's
  theorem} the the iterated integrals in \eqref{eq:prep:2:2} may be
exchange, that is,
\begin{equation}
  \label{eq:prep:2:3}
\int_{S^{n-1}_r}\int_\varepsilon^\infty\frac{1}{|r|^{n+1}} \diff r dV=
\int_\varepsilon^\infty\left(\int_{S_r^{n-1}}1\diff
  V\right)\frac{1}{|r|^{n+1}}\diff r.
\end{equation}

Now, note that from Problem 1 of the review sheet for Exam 1, we have
\begin{equation}
  \label{eq:prep:2:4}
\int_{S_r^{n-1}}1\diff V=|S_r^{n-1}|_{\bbR^{n-1}}=|S^{n-1}|_{\bbR^{n-1}}|r|^{n-1}.
\end{equation}
Set $C=|S^{n-1}|_{\bbR^{n-1}}$. Putting equations
\eqref{eq:prep:2:2}, \eqref{eq:prep:2:3}, and \eqref{eq:prep:2:4} together,
we have
\begin{equation}
\label{eq:prep:2:5}
\begin{aligned}
\int_{\bbR^n\smallsetminus B_\varepsilon}f(\bfx)\diff\bfx
&=\int_\varepsilon^\infty C|r|^{n-1}\frac{1}{|r|^{n+1}}\diff r\\
&=\int_\varepsilon^\infty\frac{C}{|r|^2}\diff r\\
&=\lim_{x\to\infty}\left[-\frac{C}{x}-\left(-\frac{C}{\varepsilon}\right)\right]\\
&=\frac{C}{\varepsilon},
\end{aligned}
\end{equation}
as was to be shown.
\end{proof}

\begin{problem}
Let $\left\{f_k\right\}$ be a sequence of nonnegative measurable functions
on $\bbR^n$, and assume that $f_k$ converges pointwise almost everywhere to
a function $f$. If
\[
\int_{\bbR^n} f=\lim_{k\to\infty}\int_{\bbR^n} f_k<\infty,
\]
show that
\[
\int_E f=\lim_{k\to\infty}\int_E f_k
\]
for all measurable subsets $E$ of $\bbR^n$. Moreover, show that this is not
necessarily true if $\int_{\bbR^n} f=\lim_{k\to\infty} f_k=\infty$.
\end{problem}
\begin{proof}
Let $E\subset\bbR^n$ be a measurable subset of $\bbR^n$. Then, since
$f_k\to f$ pointwise a.e.\@ on $\bbR^n$, then $f_k\to f$ pointwise a.e.\@ on
$E$ and $\bbR^n\smallsetminus E$. To prove that the limit of the sequence
of integrals $\left\{\int_Ef_k\right\}$ exist and is equal to $\int_E f$,
it suffices to prove that
\begin{equation}
  \label{eq:prep:2:6}
\int_E
f\leq\liminf_{k\to\infty}\int_Ef_k\leq\limsup_{k\to\infty}\int_Ef_k\leq\int_E f.
\end{equation}

The lower bound in \eqref{eq:prep:2:6} follows from an application of
Fatou's lemma:
\begin{equation}
  \label{eq:prep:2:7}
\int_E f=\int_E\liminf_{k\to\infty} f\leq\liminf_{k\to\infty}\int_E f_k.
\end{equation}
Also by Fatou's lemma, we have
\begin{equation}
  \label{eq:prep:2:8}
\int_{\bbR^n\smallsetminus E}f=\int_{\bbR^n\smallsetminus
  E}\liminf_{k\to\infty}f\leq\liminf_{k\to\infty}\int_{\bbR^n\smallsetminus E}f_k.
\end{equation}
Now, since $f\in L^1(\bbR^n)$, by equation \eqref{eq:prep:2:8} and
properties of the $\liminf$ and $\limsup$\footnote{Namely, for any sequence
  of positive real numbers $\{a_k\}$ the inequality $\liminf a_k\leq\limsup
  a_k$ holds} we have
\begin{equation}
\label{eq:prep:2:9}
\begin{aligned}
\int_Ef=\int_{\bbR^n}f-\int_{\bbR^n\smallsetminus E}f
&\geq\limsup_{k\to\infty}\int_{\bbR^n}f-\liminf_{k\to\infty}\int_{\bbR^n\smallsetminus
E}f_k\\
&\geq\limsup_{k\to\infty}\int_{\bbR^n}f_k-\limsup_{k\to\infty}\int_{\bbR^n\smallsetminus
  E}f_k\\
&=\limsup_{k\to\infty}\left[\int_{\bbR^n}f_k-\int_{\bbR^n\smallsetminus
    E}f_k\right]\\
&=\limsup_{k\to\infty}\int_Ef_k.
\end{aligned}
\end{equation}
By equations \eqref{eq:prep:2:7} and \eqref{eq:prep:2:9} it follows
that $\lim_{k\to\infty}\int_E f_k$ exists and is equal to $\int_E f$.
\\\\
To see that the result need not be true if $\int_Ef=\infty$, consider the
following example: Let $f_k\colon\bbR\to\bbR$ be given by
\begin{equation}
\label{eq:prep:2:11}
f_k(x)=
\begin{cases}
k^2/2&\text{if $x\in(-1/k,1/k)$},\\
1&\text{otherwise}
\end{cases}
\end{equation}
and $f=1$.

It is easy to see that $f_k\to f$ a.e.\@ in $\bbR$ and that both $\int_\bbR
f=\infty$ and $\lim_{k\to\infty}f_k=\infty$. However, if $E=(-1,1)$
then $\int_E f=1$, but $\lim_{k\to\infty}\int_Ef=\infty$.
\end{proof}

\begin{problem}
Assume that $E$ is a measurable set of $\bbR^n$, with
$|E|<\infty$. Prove that a nonnegative function $f$ defined
on $E$ is integrable if and only if
\[
\sum_{k=0}^\infty\left|\left\{\,\bfx\in E:f(\bfx)\geq
    k\,\right\}\right|<\infty.
\]
\end{problem}
\begin{proof}
If $f$ is integrable over a measurable subset $E$ of $\bbR^n$, then
\begin{equation}
\label{eq:integrability-3}
\int_E f(\bfx) d \bfx<\infty.
\end{equation}
Set $E_k=\left\{\,\bfx\in E:k+1>f(\bfx)\geq k\,\right\}$ and
$F_k=\left\{\,\bfx\in E:f(\bfx)\geq k\,\right\}$. Note the
following properties about the sets we have just defined: first, the
$E_k$'s are pairwise disjoint and the $F_k$'s are nested in the following
way $F_{k+1}\subset F_k$; second, $E=\bigcup_{k=1}^\infty E_k$ and
$E_k=F_k\smallsetminus F_{k+1}$. By Theorem 3.23, since the $E_k$'s are disjoint,
we have
\begin{equation}
  \label{eq:disjoint-measurable-sets-3}
|E|=\sum_{k=1}^\infty|E_k|<\infty.
\end{equation}
Now, since $k\chi_{E_k}(\bfx)\leq f(\bfx)\leq (k+1)\chi_{E_k}(\bfx)$ on
$E_k$, we have
\begin{equation}
\label{eq:estimates-E-k-3}
k|E_k|\leq\int_{E_k}f(\bfx) d \bfx\leq (k+1)|E_k|.
\end{equation}
Then we have the following upper and lower estimates on the integral of $f$
over $E$
\begin{equation}
\label{eq:upper-lower-estimates-3}
\sum_{k=0}^\infty k|E_k|\leq\int_E f(\bfx) d \bfx\leq\sum_{k=0}^\infty(k+1)|E_k|.
\end{equation}
But note that $|E_k|=|F_k\smallsetminus F_{k+1}|=|F_k|-|F_{k+1}|$ by Corollary 3.25
since the measures of $E_k$, $F_k$, and $F_{k+1}$ are all finite. Hence,
\eqref{eq:upper-lower-estimates-3} becomes
\begin{equation}
\label{eq:new-upper-lower-estimates-3}
\sum_{k=0}^\infty k\left(|F_k|-|F_{k+1}|\right)\leq
\int_E f(\bfx) d \bfx\leq
\sum_{k=0}^\infty (k+1)\left(|F_k|-|F_{k+1}|\right).
\end{equation}
A little manipulation of the series in the leftmost estimate gives us
\begin{equation}
\label{eq:leftmost-estimate-3}
\begin{aligned}
\sum_{k=0}^\infty k\left(|F_k|-|F_{k+1}|\right)
={}&\sum_{k=1}^\infty k|F_k|-\sum_{k=1}^\infty k|F_{k+1}|\\
={}&|F_1|+\sum_{k=2}^\infty k|F_k|-\sum_{k=1}^\infty k|F_{k+1}|\\
={}&|F_1|+\sum_{k=1}^\infty(k+1)|F_{k+1}|-\sum_{k=1}^\infty k|F_{k+1}\\
={}&|F_1|+\sum_{k=1}^\infty |F_{k+1}|\\
={}&\sum_{k=1}^\infty|F_{k+1}|
\end{aligned}
\end{equation}
and
\begin{equation}
\label{eq:rightmost-estimate-3}
\begin{aligned}
\sum_{k=0}^\infty(k+1)\left(|F_k|-|F_{k+1}|\right)
={}&\sum_{k=0}^\infty(k+1)|F_k|-\sum_{k=0}^\infty(k+1)|F_{k+1}|\\
={}&|F_0|+\sum_{k=1}^\infty(k+1)|F_k|-\sum_{k=0}^\infty(k+1)|F_{k+1}|\\
={}&|F_0|+\sum_{k=0}^\infty(k+2)|F_{k+1}|-\sum_{k=0}^\infty(k+1)|F_{k+1}|\\
={}&|F_0|+\sum_{k=0}^\infty|F_{k+1}|\\
={}&\sum_{k=0}^\infty|F_k|.
\end{aligned}
\end{equation}
Thus, from \eqref{eq:leftmost-estimate-3} and
\eqref{eq:rightmost-estimate-3}
\begin{equation}
\label{eq:final-upper-lower-estimates-3}
\sum_{k=1}^\infty|F_k|\leq\int_E f(\bfx) d \bfx\leq\sum_{k=0}^\infty|F_k|
\end{equation}
so the integral $\int_E f$ converges if and only if the sum
$\sum_{k=0}^\infty|F_k|$ converges.
\end{proof}
\begin{problem}
Suppose that $E$ is a measurable subset of $\bbR^n$, with
$|E|<\infty$. If $f$ and $g$ are measurable functions on
$E$, define
\[
\rho(f,g)=\int_E\frac{|f-g|}{1+|f-g|}.
\]
Prove that $\rho(f_k,f)\to 0$ as $k\to\infty$ if and only if $f_k$
converges to $f$ as $k\to\infty$.
\end{problem}
\begin{proof}
\end{proof}

\begin{problem}
Define the \emph{gamma function} $\Gamma\colon\bbR^+\to\bbR$ by
\[
\Gamma(y)=\int_0^\infty e^{-u}u^{y-1}\diff u,
\]
and the \emph{beta function} $\beta\colon\bbR^+\times\bbR^+\to\bbR$
by
\[
\beta(x,y)=\int_0^1 t^{x-1}(1-t)^{y-1}\diff t.
\]
\begin{enumerate}[label=(\alph*),noitemsep]
\item Prove that the definition of the gamma function is well-posed, i.e.,
the function $u\mapsto e^{-u}u^{y-1}$ is in $L(\bbR^+)$ for all
$y\in\bbR^+$.
\item Show that
\[
\beta(x,y)=\frac{\Gamma(x)\Gamma(y)}{\Gamma(x+y)}.
\]
\end{enumerate}
\end{problem}
\begin{proof}
\end{proof}

\begin{problem}
Let $f\in L(\bbR^n)$ and for $\mathbf{h}\in\bbR^n$ define
$f_{\mathbf{h}}\colon\bbR^n\to\bbR$ be $f_{\mathbf{h}}(\bfx)=
f(\bfx-\mathbf{h})$. Prove that
\[
\lim_{\mathbf{h}\to\mathbf{0}}\int_{\bbR^n}\left|f_{\mathbf{h}}-f\right|=0.
\]
\end{problem}
\begin{proof}
\end{proof}

\begin{problem}
\begin{enumerate}[label=(\alph*),noitemsep]
\item If $f_k,g_k,f,g\in L(\bbR^n)$, $f_k\to f$ and $g_k\to g$ a.e.\@ in
  $\bbR^n$, $|f_k|\leq g_k$ and
\[
\int_{\bbR^n}g_k\longrightarrow\int_{\bbR^n}g,
\]
prove that
\[
\int_{\bbR^n} f_k\longrightarrow\int_{\bbR^n}f.
\]
\item Using part (a) show that if $f_k,f\in L(\bbR^n)$ and $f_k\to f$
  a.e.\@ in $\bbR^n$, then
\[
\int_{\bbR^n}|f_k-f|\longrightarrow 0\qquad\text{as $k\to\infty$}
\]
if and only if
\[
\int_{\bbR^n}|f_k|\longrightarrow\int_{\bbR^n}|f|\qquad\text{as $k\to\infty$}.
\]
\end{enumerate}
\end{problem}
\begin{proof}
(a) $\implies$ (b): Assume part (a) then $\implies$ if
\begin{equation}
\label{eq:prep:2:12}
\int_{\bbR^n}|f_k-f|\longrightarrow 0
\end{equation}
as $k\to\infty$, we have
\\\\
(b):
\end{proof}

%%% Local Variables:
%%% mode: latex
%%% TeX-master: "../MA544-Quals"
%%% End:
