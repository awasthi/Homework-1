\subsubsection{Homework 8}
\setcounter{exercise}{0}
\setcounter{equation}{0}

\begin{problem}[Wheeden \& Zygmund Ch.\@ 5, Ex.\@ 2]
  Show that the conclusion of (5.32) are not true without the assumption
  that $\varphi\in L(E)$. [In part (ii), for example, take
  $f_k=\chi_{(k,\infty)}$.]
\end{problem}
\begin{solution}
\end{solution}

\begin{problem}[Wheeden \& Zygmund Ch.\@ 5, Ex.\@ 4]
  If $f\in L(0,1)$, show that $x^kf(x)\in L(0,1)$ for $k=1,2,...$, and
  $\int_0^1 x^kf(x)\diff x\to 0$.
\end{problem}
\begin{solution}
\end{solution}

\begin{problem}[Wheeden \& Zygmund Ch.\@ 5, Ex.\@ 6]
  Let $f(x,y)$, $0\leq x,y\leq 1$, satisfy the following conditions: for
  each $x$, $f(x,y)$ is an integrable function of $y$, and
  $\partial f(x,y)/\partial x$ is a bounded function of $(x,y)$. Show that
  $\partial f(x,y)/\partial x$ is a measurable function of $y$ for each $x$
  and
  \[
    \frac{\rmd}{\rmd x}\int_0^1f(x,y)\diff
    y=\int_0^1\frac{\partial}{\partial x}f(x,y)\diff y.
  \]
\end{problem}
\begin{solution}
\end{solution}

\begin{problem}[Wheeden \& Zygmund Ch.\@ 5, Ex.\@ 7]
  Give an example of an $f$ that is not integrable, but whose improper
  Riemann integral exists and is finite.
\end{problem}
\begin{solution}
\end{solution}

\begin{problem}[Wheeden \& Zygmund Ch.\@ 5, Ex.\@ 21]
  If $\int_A f=0$ for every measurable subset A of a measurable set $E$,
  show that $f=0$ a.e.\@ in $E$.
\end{problem}
\begin{solution}
\end{solution}

\begin{problem}[Wheeden \& Zygmund Ch.\@ 6, Ex.\@ 10]
  Let $V_n$ be the volume of the unit ball in $\bfR^n$. Show by using
  Fubini's theorem that
  \[
    V_n=2V_{n-1}\int_0^1\left(1-t^2\right)^{(n-1)/2}\diff t.
  \]
  (We also observe that by setting $w=t^2$, the integral is a multiple of a
  classical $\beta$-function and so can be expressed in terms of the
  $\Gamma$-function: $\Gamma(s)=\int_0^\infty e^{-t}t^{s-1}\diff t$,
  $s>0$.)
\end{problem}
\begin{solution}
\end{solution}

\begin{problem}[Wheeden \& Zygmund Ch.\@ 6, Ex.\@ 11]
  Use Fubini's theorem to prove that
  \[
    \int_{\bfR^n}e^{-|\bfx|^2}\diff\bfx=\pi^{n/2}.
  \]
  (For $n=1$, write
  $\left(\int_{-\infty}^\infty e^{-x^2}\diff
    x\right)^2=\int_{-\infty}^\infty\int_{-\infty}^\infty e^{-x^2-y^2}\diff
  xdy$ and use polar. For $n>1$, use the formula
  $e^{-|\bfx|^2}=e^{-{x_1}^2}\cdots e^{-{x_n}^2}$ and Fubini's theorem to
  reduce the case $n=1$.)
\end{problem}
\begin{solution}
\end{solution}

%%% Local Variables:
%%% mode: latex
%%% TeX-master: "../MA544-Quals"
%%% End:
