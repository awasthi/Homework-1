% \section{MA 544 Past Quals}
\subsection{Danielli: Winter 2012}
\setcounter{exercise}{0}
\setcounter{equation}{0}

\begin{problem}
  Let \(f(x,y)\), \(0\leq x,y\leq 1\), satisfy the following conditions:
  for each \(x\), \(f(x,y)\) is an integrable function of \(y\), and
  \(\partial f(x,y)/\partial x\) is a bounded function of \((x,y)\). Prove
  that \(\partial f(x,y)/\partial x\) is a measurable function of \(y\) for
  each \(x\) and
  \[
    \frac{\rmd}{\rmd x}\int_0^1f(x,y)\diff y= \int_0^1\frac{\partial
      f(x,y)}{\partial x}\diff y.
  \]
\end{problem}
\begin{solution}
  The end points can be dealt with separately. Fix a point \(x_0\in(0,1)\)
  and consider the sequence of measurable functions \(\{f_n'\}\) where
  \[
    f_n'(y)=\frac{f(x_0+h_n,y)-f(x_0,y)}{h_n}
  \]
  where \(\{h_n\}\) is a sequence of numbers converging to \(0\). Since
  \(f\) is differentiable as a function of \(x\), the sequence
  \(\{f_n'(x_0,y)\}\) converges to \(\partial f/\partial x(x_0,y)\). Now,
  since \(|\partial f/\partial x(x,y)|\leq M\) for some \(M\in\bbR^+\) for
  all \((x,y)\in[0,1]\times[0,1]\), by the bounded convergence theorem
  \begin{align*}
    \lim_{n\to\infty}\int_0^1 f'_n(y)\diff y
    &=\int_0^1\lim_{n\to\infty} f'_n(y)\diff y\\
    &=\int_0^1 \frac{\partial f(x_0,y)}{\partial x}\diff y.
  \end{align*}
\end{solution}

\begin{problem}
  Let \(f\) be a function of bounded variation on \([a,b]\),
  \(-\infty<a<b<\infty\). If \(f=g+h\), with \(g\) absolutely continuous
  and \(h\) singular, show that
  \[
    \int_a^b\varphi\diff f=\int_a^b\varphi f'\diff x+\int_a^b\varphi\diff
    h.
  \]
  \\\\
  \emph{Hint}: A function \(h\) is said to be singular if \(h'=0\).
\end{problem}
\begin{solution}
\end{solution}

\begin{problem}
  Let \(E\subseteq\bbR\) be a measurable set, and let \(K\) be a measurable
  function on \(E\times E\). Assume that there exists a positive constant
  \(C\) such that
  \[
    \label{eq:jan-12:1}
    \tag{\(\bigstar\)}
    \int_E K(x,y)\diff x\leq C
  \]
  for a.e.\@ \(y\in E\), and
  \[
    \label{eq:jan-12:2}
    \tag{\(\clubsuit\)}
    \int_E K(x,y)\diff y\leq C
  \]
  for a.e.\@ \(x\in E\).

  Let \(1<p<\infty\), \(f\in L^p(E)\), and define
  \[
    T_f(x)=\int_E K(x,y)f(y)\diff y.
  \]
  \begin{enumerate}[label=(\alph*),noitemsep]
  \item Prove that \(T_f\in L^p(E)\) and
    \[
      \label{eq:jan-12:3}
      \tag{\(\spadesuit\)}
      \|T_f\|_p\leq C\|f\|_p.
    \]
  \item Is \eqref{eq:jan-12:3} still valid if \(p=1\) or \(\infty\)? If so,
    are assumptions \eqref{eq:jan-12:1} and \eqref{eq:jan-12:2} needed?
  \end{enumerate}
\end{problem}
\begin{solution}
\end{solution}

\begin{problem}
  Let \(f\) be a nonnegative measurable function on \([0,1]\) satisfying
  \[
    \label{eq:jan-12:4}%
    \tag{\(\blacklozenge\)}%
    m\left\{\,x\in[0,1]:f(x)>\alpha\,\right\}<\frac{1}{1+\alpha^2}
  \]
  for \(\alpha>0\).
  \begin{enumerate}[label=(\alph*),noitemsep]
  \item Determine values of \(p\in[1,\infty)\) for which \(f\in L^p[0,1]\).
  \item If \(p_0\) is the minimum value of \(p\) for which \(p\) may fail
    to be in \(L^p\), give an example of a function which satisfies
    \eqref{eq:jan-12:4}, but which is not in \(L^{p_0}[0,1]\).
\end{enumerate}
\end{problem}
\begin{solution}
\end{solution}

%%% Local Variables:
%%% mode: latex
%%% TeX-master: "../MA544-Quals"
%%% End:
