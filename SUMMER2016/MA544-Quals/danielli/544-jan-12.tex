% \section{MA 544 Past Quals}
\setcounter{exercise}{0}
\setcounter{equation}{0}
\subsection{Danielli: Winter 2012}
\begin{problem}
Let $f(x,y)$, $0\leq x,y\leq 1$, satisfy the following conditions: for each
$x$, $f(x,y)$ is an integrable function of $y$, and $\partial
f(x,y)/\partial x$ is a bounded function of $(x,y)$. Prove that $\partial
f(x,y)/\partial x$ is a measurable function of $y$ for each $x$ and
\[
\frac{\rmd}{\rmd x}\int_0^1f(x,y)\diff y=
\int_0^1\frac{\partial f(x,y)}{\partial x}\diff y.
\]
\end{problem}
\begin{solution}
\end{solution}

\begin{problem}
Let $f$ be a function of bounded variation on $[a,b]$,
$-\infty<a<b<\infty$. If $f=g+h$, with $g$ absolutely continuous and $h$
singular, show that
\[
\int_a^b\varphi\diff f=\int_a^b\varphi f'\diff x+\int_a^b\varphi\diff h.
\]
\\\\
\emph{Hint}: A function $h$ is said to be singular if $h'=0$.
\end{problem}
\begin{solution}
\end{solution}

\begin{problem}
Let $E\subset\bfR$ be a measurable set, and let $K$ be a measurable
function on $E\times E$. Assume that there exists a positive constant $C$
such that
\begin{equation}
\label{eq:jan-12:1}
\int_E K(x,y)\diff x\leq C
\end{equation}
for a.e.\@ $y\in E$, and
\begin{equation}
\label{eq:jan-12:2}
\int_E K(x,y)\diff y\leq C
\end{equation}
for a.e.\@ $x\in E$.

Let $1<p<\infty$, $f\in L^p(E)$, and define
\[
T_f(x)=\int_E K(x,y)f(y)\diff y.
\]
\begin{enumerate}[label=(\alph*),noitemsep]
\item Prove that $T_f\in L^p(E)$ and
\begin{equation}
\label{eq:jan-12:3}
\|T_f\|_p\leq C\|f\|_p.
\end{equation}
\item Is \eqref{eq:jan-12:3} still valid if $p=1$ or $\infty$? If so, are
  assumptions \eqref{eq:jan-12:1} and \eqref{eq:jan-12:2} needed?
\end{enumerate}
\end{problem}
\begin{solution}
\end{solution}

\begin{problem}
Let $f$ be a nonnegative measurable function on $[0,1]$ satisfying
\begin{equation}
  \label{eq:jan-12:4}
\left|\left\{\,x\in[0,1]:f(x)>\alpha\,\right\}\right|<\frac{1}{1+\alpha^2}
\end{equation}
for $\alpha>0$.
\begin{enumerate}[label=(\alph*),noitemsep]
\item Determine values of $p\in[1,\infty)$ for which $f\in L^p[0,1]$.
\item If $p_0$ is the minimum value of $p$ for which $p$ may fail to be in
  $L^p$, give an example of a function which satisfies \eqref{eq:jan-12:4},
  but which is not in $L^{p_0}[0,1]$.
\end{enumerate}
\end{problem}
\begin{solution}
\end{solution}

%%% Local Variables:
%%% mode: latex
%%% TeX-master: "../MA544-Quals"
%%% End:
