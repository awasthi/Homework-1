\subsection{Exam Preparation}
\setcounter{exercise}{0}

\subsubsection{Exam 1 Practice}
\begin{problem}
Let $E\subset\bbR^n$ be a measurable set, $r\in\bbR$ and define the set
$rE=\left\{\,r\bfx:\bfx\in E\,\right\}$. Prove that $rE$ is
measurable, and that $|rE|=|r|^n|E|$.
\end{problem}
\begin{proof}
Define a map $T\colon\bbR^n\to\bbR^n$ by $T\bfx= r\bfx$. Note
that $T$ is
\href{https://en.wikipedia.org/wiki/Lipschitz_continuity}{\emph{Lipschitz
    continuous}} since for any $\bfx,\bfy\in\bbR^n$, the equality
\begin{equation}
\label{eq:prep:1:1}
|T\bfx-T\bfy|=|r\bfx-r\bfy|=|r||\bfx-\bfy|
\end{equation}
is satisfied. By Theorem 3.33 from \cite[Ch.\@ 3, p.\@55]{wheeden-zygmund},
the image of $E$ under $T$ is measurable. Moreover, by Theorem 3.35
\cite[Ch.\@ 3, p.\@ 56]{wheeden-zygmund}, since $T$ is linear, it follows
that $|T(E)|=|{\det T}||E|$ where $\det T=|r|^n$. Lastly, we note that the
image of $E$ under $T$ is precisely the set $rE$ so that
$|T(E)|=|rE|=|r|^n|E|$, as was to be shown.
\end{proof}

\begin{problem}
Let $\left\{ E_k \right\}$, $k\in\bfN$ be a collection of measurable
sets. Define the set
\[
\liminf_{k\to\infty} E_k
=\bigcup_{k=1}^\infty\left(\bigcap_{n=k}^\infty E_n\right).
\]
Show that
\[
\left|\liminf_{k\to\infty} E_k\right|\leq\liminf_{k\to\infty}\left|E_k\right|.
\]
\end{problem}
\begin{proof}
Following the style of \cite[Ch.\@ 1, p.\@ 2]{wheeden-zygmund},
particularly, the sets defined after the introduction of equation (1.1),
set
\begin{equation}
\label{eq:prep:1:2}
V_k=\bigcap_{\ell=k}^\infty E_\ell.
\end{equation}
Note that the collection of sets $\{V_k\}$ forms an increasing
sequence, that is, if $\bfx\in V_k$ then, by \eqref{eq:prep:1:2}, $\bfx$ is
in the intersection $E_k\cap\bigl(\bigcap_{\ell=k+1}E_\ell\bigr)$, but, by
\eqref{eq:prep:1:2}, $\bigcap_{\ell=k+1}E_\ell=V_{k+1}$ thus, $\bfx$
is in $V_{k+1}$ so $V_{k+1}\supset V_k$. Hence, we have $V_k\nearrow\liminf
E_k$.

Now, consider the sequence $\{|V_k|\}$ formed by the Lebesgue measure of
the $V_k$. By Theorem 3.26 from \cite[Ch.\@ 3, p.\@
51]{wheeden-zygmund}, since $V_k\nearrow\liminf E_k$,
\begin{equation}
  \label{eq:prep:1:3}
\lim_{k\to\infty}|V_k|=
\lim_{k\to\infty}\left|\bigcap_{\ell=k}^\infty E_\ell\right|=
\left|\liminf_{k\to\infty} E_k\right|.
\end{equation}
But note that, by the monotonicity of the Lebesgue measure, we have
\begin{equation}
  \label{eq:prep:1:4}
\left|\bigcap_{\ell=k}^\infty E_\ell\right|\leq |E_k|,
\end{equation}
so, by properties of the $\liminf$, in particular, by Theorem 19(v) from
\cite[Ch.\@ 1, p.\@ 23]{royden}, we have
\begin{equation}
\label{eq:prep:1:5}
\limsup_{k\to\infty}|V_k|\leq\liminf_{k\to\infty}|E_k|.
\end{equation}
Hence, by \eqref{eq:prep:1:3} and Proposition 19 (iv), since the sequence
$\{|V_k|\}$ converges and is equal to the measure of $\liminf E_k$, by
\eqref{eq:prep:1:5}, we have
\begin{equation}
\label{eq:prep:1:6}
\left|\liminf_{k\to\infty} E_k\right|\leq\liminf_{k\to\infty}|E_k|
\end{equation}
as was to be shown.
\end{proof}

\begin{problem}
Consider the function
\[
F(x)=
\begin{cases}
|B(\mathbf{0},x)|&x>0\\
0&x=0
\end{cases}.
\]
Here
$B(\mathbf{0},r)=\left\{\,\bfy\in\bbR^n:|\bfy|<r\,\right\}$. Prove
that $F$ is monotonic increasing and continuous.
\end{problem}
\begin{proof}
Define the linear map $T\colon[0,\infty)\times\bbR^n\to\bbR^n$ by
$T(r)\bfx= r\bfx$. We claim that
$B(\mathbf{0},r)=T(r,B(\mathbf{0},1))$. To reduce notation, set
$B_1= B(\mathbf{0},1)$ and $B_r= B(\mathbf{0},r)$.
\begin{proof}[Proof of claim]
\renewcommand{\qedsymbol}{$\clubsuit$}
$\subset$: Let $\bfx\in B_r$. Then $|\bfx|<r$ so $|\bfx|/r<1$. Thus,
$|\bfx|/r\in B_1$ so it is in the image of $B_1$ under the map
$T(r,\cdot)$.

$\supset$: On the other hand, suppose $\bfx\in T(r,B_1)$. Then
$\bfx=r\bfy$ for some $\bfy\in B_1$. Then, since $|\bfy|<1$,
$|\bfx|=r|\bfy|<r$ so $\bfx\in B_r$.
\end{proof}

From the claim, we see that $F(x)=|T(x,B(\mathbf{0},1))|$ which, by Problem
1, is nothing more that the polynomial $|B_1|x^n$. It is clear,
from this equivalence, that $F$ is monotonically increasing: Take
$x,y\in[0,\infty)$ such that $x<y$, then $x^n<y^n$ so
\begin{equation}
\label{eq:prep:1:7}
F(x)=|B_1|x^n<|B_1|y^n=F(y).
\end{equation}
Thus, $F$ is monotonically increasing.

In the argument above, since $F(x)=|B_1|x^n$ is a polynomial in
$[0,\infty)$ (and polynomials are continuous on $\bbR$) $F$ is continuous
on $[0,\infty)$.
\end{proof}

\begin{problem}
Let $f\colon\bbR\to\bbR$ be a function. Let $C$ be the set of all points
at which $f$ is continuous. Show that $C$ is a set of type $G_\delta$.
\end{problem}
\begin{proof}
(Without much motivation) let us consider the collection of sets $\{E_k\}$
defined by
\begin{equation}
\label{eq:prep:1:8}
E_k=\left\{\,x\in\bbR:
\text{there exists $\delta>0$ such that $y,z\in B(x,\delta)$ implies $\left|f(y)-f(z)\right|<\frac{1}{k}$}\,\right\}.
\end{equation}
We claim that $C=\bigcap_{k=1}^\infty E_k$ and that each $E_k$ is open.
\begin{proof}[Proof of claim]
\renewcommand{\qedsymbol}{$\clubsuit$}
First, we demonstrate equality. $\subset$: Suppose $x\in C$. Then, by the
definition of continuity, for every $\varepsilon>0$, there exists a
$\delta>0$ such that $y\in B(x,\delta)$ implies $|f(x)-f(y)|<\delta$. In
particular, for every $k$, there exists $\delta>0$ such that for $y\in
B(x,\delta)$ the inequality $|f(x)-f(y)|<1/k$ holds. Thus, $x$ is in
$\bigcap_{k=1}^\infty E_k$.

$\supset$: On the other hand, suppose that $x\in\bigcap_{k=1}^\infty
E_k$. Then, given $\varepsilon>0$, by the Archimedean property, there
exists a positive integer $N$ such that $1/N<\varepsilon$. Then, since
$x\in\bigcap_{k=1}^\infty E_k$, $x\in E_N$ so
\begin{equation}
  \label{eq:prep:1:9}
|f(x)-f(y)|<\frac{1}{N}<\varepsilon.
\end{equation}
Thus, $x$ is in $C$ and $C=\bigcap_{k=1}^\infty E_k$.

All that remains to be shown is that the $E_k$ are open. But this is clear
by the way we defined $E_k$ in \eqref{eq:prep:1:8}: Let $x\in E_k$, then
there exists $\delta>0$ such that for any $y,z\in B(x,\delta)$,
$|f(y)-f(z)|<1/k$; Let $x'\in B(x,\delta)$ and set
$\delta'=\min\{|(x+\delta)-x'|,|(x-\delta)-x|\}$. Then, since
$B(x',\delta')\subset B(x,\delta)$, for every $y,z\in B(x',\delta')$, we
have $|f(y)-f(z)|<1/k$. Hence, $x'\in E_k$ for any $x'\in B(x,\delta)$
so $B(x,\delta)\subset E_k$.
\end{proof}
Since $C$ can be expressed as the countable intersection of open sets
$E_k$, it follows that $C$ is a $G_\delta$ set.
\end{proof}
\begin{problem}
Let $f\colon\bbR\to\bbR$ be a function. Is it true that if the sets
$\left\{\,f=r\,\right\}$ are measurable for all $r\in\bbR$, then $f$ is
measurable?
\end{problem}
\begin{proof}
If $\left\{\,f=r\,\right\}$ are measurable for all $r\in\bbR$, it is not
necessarily the case that $f$ is measurable. Consider the following
construction: Let $E\subset(0,1)$ be an unmeasurable set.\footnote{It's
  construction does not concern us. The interested reader such direct their
  refer to Theorem 3.38 from \cite[Ch.\@ 3, p.\@ 57-58]{wheeden-zygmund} or
  Theorem 17 from \cite[Ch.\@ 2\S 7, p.\@ 48]{royden}.} Define a map
$f\colon\bbR\to\bbR$ by
\begin{equation}
\label{eq:prep:1:11}
f(x)=
\begin{cases}
x&\text{if $x\in\bbR\smallsetminus((0,1)\smallsetminus E)$},\\
x+1&\text{if $x\in (0,1)\smallsetminus E$.}
\end{cases}
\end{equation}
By the definition, it is clear that $\left\{\,f=r\,\right\}$ is measurable
and $\left|\left\{\,f=r\,\right\}\right|=0$ since $\{\,f=r\,\}$ contains at
most two elements. However, the set $\left\{\,0<f<1\,\right\}=E$ is not
measurable. Thus, $f$ is not measurable.
\end{proof}

\begin{problem}
Let $\left\{f_k\right\}_{k=1}^\infty$ be a sequence of measurable functions
on $\bbR$. Prove that the set
$\left\{\,x:\text{$\lim_{k\to\infty} f_k(x)$ exists}\,\right\}$
is measurable.
\end{problem}
\begin{proof}
By Theorem 4.12 from \cite[Ch.\@ 4, p.\@
67]{wheeden-zygmund}, $\liminf_{k\to\infty}f_k$ and
$\limsup_{k\to\infty}f_k$ are measurable. By Theorem 4.7 from \cite[Ch.\@ 4,
p.\@ 66]{wheeden-zygmund}
\begin{equation}
\label{eq:prep:1:12}
\left\{\,\liminf_{k\to\infty} f_k<\limsup_{k\to\infty} f_k\,\right\}
\end{equation}
is measurable. Since
\begin{equation}
  \label{eq:prep:1:13}
\left\{\,\text{$\lim_{k\to\infty}f_k$ exists}\,\right\}=
\left\{\,{\limsup_{k\to\infty}f_k=\liminf_{k\to\infty}f_k}\,\right\}=
\bbR\smallsetminus
\left\{\,{\liminf_{k\to\infty} f_k<\limsup_{k\to\infty} f_k}\,\right\},
\end{equation}
by Theorem 3.17 from \cite[Ch.\@ 3, p.\@ 48]{wheeden-zygmund},
the set $\left\{\,\text{$\lim_{k\to\infty}f_k$ exists}\,\right\}$ is
measurable.
% In a fashion similar to that of Problem 4, consider the set collection of
% sets $\{E_k\}$ given by
% \begin{equation}
% \label{eq:prep:1:11}
% E_k=
% \left\{\,
% x\in\bbR:\text{there exists $N$ such that $m,n\geq N$ implies $\left|f_n(x)-f_m(x)\right|<\frac{1}{k}$}
% \,\right\}.
% \end{equation}
% You can show that the $E_k$ are open and that
% $\left\{\,x:\text{$\lim_{x\to\infty}f_k(x)$
%     exists}\,\right\}=\bigcap_{k=1}^\infty E_k$. Then, since open sets are
% measurable and, by Theorem 3.18 from \cite[Ch.\@ 3, p.\@
% 48]{wheeden-zygmund}, the countable intersection of measurable sets is
% measurable, $\left\{\,x:\text{$\lim_{x\to\infty}f_k(x)$ exists}\,\right\}$
% is measuable.
\end{proof}
\begin{problem}
A real valued function $f$ on an interval $[a,b]$ is said to be
\emph{absolutely continuous} if for every $\varepsilon>0$, there exists a
$\delta>0$ such that for every finite disjoint collection
$\left\{(a_k,b_k)\right\}_{k=1}^N$ of open intervals in $(a,b)$ satisfying
$\sum_{k=1}^Nb_k-a_k<\delta$, one has
$\sum_{k=1}^N\left|f(b_k)-f(a_k)\right|<\varepsilon$. Show that an
absolutely continuous function on $[a,b]$ is of bounded variation on
$[a,b]$.
\end{problem}
\begin{proof}
Suppose $f$ is absolutely continuous on $[a,b]$. Let $\varepsilon=
1$. Then, there exists $\delta>0$ such that for every finite disjoint
collection $\left\{(a_k,b_k)\right\}_{k=1}^N$ of open intervals in $(a,b)$
satisfying $\sum_{k=1}^Nb_k-a_k<\delta$, one has
$\sum_{k=1}^N\left|f(b_k)-f(a_k)\right|<1$. Let
$N=\lceil(b-a)/\delta\rceil$, that is, $N$ is the smallest integer
greater than $(b-a)/\delta$, and consider the partition $\Gamma=\{x_k\}$
where $x_k= a+k(b-a)/N$, for $k=0,\dotsc,N$. Then
$x_k-x_{k-1}<(b-a)/N<\delta$ so, by Theorem 2.2(i) from \cite[Ch.\@ 2, p.\@
19]{wheeden-zygmund}, we have $V[f;x_{k-1},x_k]<1$ for $k=0,\dotsc,N$. In
follows by Theorem 2.2(ii) that
\begin{equation}
\label{eq:prep:1:14}
V[f;a,b]=\sum_{k=1}^N V[f;x_{k-1},x_k]<N.
\end{equation}
Thus, $f$ is b.v.\@ on $[a,b]$.
\end{proof}

\begin{problem}
Let $f$ be a continuous function from $[a,b]$ into $\bbR$. Let
$\chi_{\{c\}}$ be the characteristic function of a singleton
$\left\{c\right\}$, that is, $\chi_{\{c\}}(x)=0$ if $x\neq c$ and
$\chi_{\{c\}}(c)=1$. Show that
\[
\int_a^b f d \chi_{\{c\}}=
\begin{cases}
0&\text{if $c\in(a,b)$,}\\
-f(a)&\text{if $c=a$,}\\
f(b)&\text{if $c=b$.}
\end{cases}
\]
\end{problem}
\begin{proof}
The result follows quite easily from more sophisticated measure theoretic
arguments. At this point, however, such language has not been discussed so
we shall prove this using nothing but the definition of the
Riemann--Stieltjes integral and properties thereof.

Let us consider each case $c\in(a,b)$, $c=a$, and $c=b$ separately.

Recall that the given a partition $\Gamma=\{x_0,\dotsc,x_m\}$ of $[a,b]$,
the Riemann--Stieltjes sum of $f$ with respect to $\varphi$ is
\begin{equation}
  \label{eq:prep:1:15}
R_\Gamma=\sum_{k=1}^mf(\xi_k)[\varphi(x_k)-\varphi(x_{k-1})].
\end{equation}
The Riemann--Stieltjes integral is defined as the limit
\begin{equation}
\label{eq:prep:1:16}
\int_a^b f\diff\varphi=\lim_{|\Gamma|\to 0} R_\Gamma
\end{equation}
if it exists.

Suppose $c\in(a,b)$. Then, for any partition $\Gamma$ of $[a,b]$, either
$c\in\Gamma$ or $c\notin\Gamma$. In the latter case, $R_\Gamma=0$. In the
former case $c$ is one of the $x_k$, say $c=x_\ell$ for $0<\ell<m$. Then
\begin{equation}
\label{eq:prep:1:17}
\begin{aligned}
R_\Gamma&=\sum_{k=1}^mf(\xi_k)[\chi_{\{c\}}(x_k)-\chi_{\{c\}}(x_{k-1})]\\
&=0+\dotsb+0+f(\xi_{\ell-1})-f(\xi_\ell)+0+\dotsb+0\\
&=f(\xi_{\ell-1})-f(\xi_\ell).
\end{aligned}
\end{equation}
Since $f$ is continuous, given $\varepsilon>0$ there exists $\delta>0$ such
that $|\xi_\ell-\xi_{\ell-1}|<\delta$ implies
$|f(\xi_{\ell})-f(\xi_{\ell-1})|<\varepsilon$. It follows that the quantity
in \eqref{eq:prep:1:17} approaches $0$ as $|\Gamma|$ approaches
$0$. Therefore, $\int_a^b f\diff\chi_{\{c\}}=0$.

Suppose $c=a$. Then, since any partition $\Gamma$ of $[a,b]$ must contain
the point $a$, we have
\begin{equation}
\label{eq:prep:1:18}
\begin{aligned}
R_\Gamma
&=\sum_{k=1}^mf(\chi_k)[\chi_{\{c\}}(x_k)-\chi_{\{c\}}(x_{k-1})]\\
&
\begin{aligned}
=f(\xi_1)[\chi_{\{c\}}(x_1)-\chi_{\{c\}}(x_0)]&
+f(\xi_2)[\chi_{\{c\}}(x_2)-\chi_{\{c\}}(x_1)]\\
&+\dotsb+f(\xi_m)[\chi_{\{c\}}(x_m)-\chi_{\{c\}}(x_{m-1})]
\end{aligned}\\
&=-f(\xi_1)+0+\dotsb+0\\
&=-f(\xi_1)
\end{aligned}
\end{equation}
Taking the limit as $|\Gamma|\to 0$, $\xi_1\to a$ so, by continuity of $f$,
$f(\xi_1)\to f(a)$. Thus, $\int_a^b f\diff\chi_{\{c\}}=-f(a)$.

A similar argument to the one above shows that, if $c=b$, the
Riemann--Stieltjes integral $\int_a^bf\diff\chi_{\{c\}}=f(b)$.
\end{proof}

%%% Local Variables:
%%% mode: latex
%%% TeX-master: "../MA544-Quals"
%%% End:
