\section{Danielli}
\subsection{Danielli: Practice Exams Spring 2016}
\setcounter{exercise}{0}
\setcounter{equation}{0}

\subsubsection{Exam 1 Practice}
\begin{problem}
  Let \(E\subseteq\bbR^n\) be a measurable set, \(r\in\bbR\) and define the
  set \(rE=\left\{\,rx : x\in E\,\right\}\). Prove that \(rE\) is
  measurable, and that \(|rE|=|r|^n|E|\).
\end{problem}
\begin{solution}
  Define a map a linear map \(T\colon\bbR^n\to\bbR^n\) by
  \(T(x)=rx\). Since a the image of a measurable set \(E\) under linear map
  is measurable and \(m(T(E))=|{\det T}|m(E)=|r|^nm(E)\), it suffices to
  show that \(T(E)=rE\).

  Let \(y\in T(E)\) then \(y=rx\) for some \(x\in E\). Thus, \(y\in
  rE\). Let \(y\in rE\). Then, \(y=rx=T(x)\) for some \(x\in E\). Thus,
  \(y\in T(E)\). It follows that \(m(rE)=|r|^nm(E)\).
\end{solution}

\begin{problem}
  Let \(\{E_n\}\), \(n\in\bbN\) be a collection of measurable sets. Define
  the set
  \[
    \liminf_{n\to\infty} E_n
    =\bigcup_{n=1}^\infty\left(\bigcap_{k=n}^\infty E_k\right).
  \]
  Show that
  \[
    m\left(\liminf_{n\to\infty}
      E_n\right)\leq\liminf_{n\to\infty}m(E_n).
  \]
\end{problem}
\begin{solution}
  Here's a quick and dirty way of proving this: let \(\chi_{E_n}\) be the
  characteristic function of \(E_n\). Then, by Fatou's lemma,
  \begin{equation}
    \label{eq:ex-1-prep:fatou}
    \int\liminf_{n\to\infty}\chi_{E_n}(x)\diff x
    \leq\liminf_{n\to\infty}\int\chi_{E_n}(x)\diff x.
  \end{equation}
  By definition of the characteristic function, it is easy to see that the
  right hand-side of the Equation \eqref{eq:ex-1-prep:fatou} is
  \[
    \liminf_{k\to\infty}m(E_k).
  \]
  But what about the left-hand side of \eqref{eq:ex-1-prep:fatou}? We claim
  that
  \[
    \liminf_{n\to\infty}\chi_{E_n}=\chi_{E}
  \]
  where \(E=\liminf_{n\to\infty} E_n\).
  \begin{quote}
    \begin{proof}[Proof of claim]
      Suppose \(x\in E\). We must show that
      \(\liminf_{n\to\infty}\chi_{E_n}(x)=1\). By definition
      \[
        \liminf_{n\to\infty}\chi_{E_n}=%
        \lim_{n\to\infty}\left[\inf_{k\geq n}\chi_{E_k}\right].
      \]
      Now \(x\in E\) if and only if \(x\in \bigcap_{k=N}^\infty E_k\) for
      some \(N\in\bbN\). Then for \(k\geq N\)
      \[
        \inf_{k\geq n}\chi_{E_k}(x)=1
      \]
      so \(\liminf_{n\to\infty}\chi_{E_n}(x)=1\).

      On the other hand, if \(x\notin E\) then
      \(x\notin\bigcap_{k=n}^\infty E_k\) for all \(n\in\bbN\). Thus, for
      all \(n\in\bbN\),
      \[
        \inf_{k\geq n}\chi_{E_k}(x)=0
      \]
      so \(\liminf_{n\to\infty}\chi_{E_k}=0\).
    \end{proof}
  \end{quote}
  Having established this equivalence, we have
  \[
    m\left(\liminf_{n\to\infty}E_n\right)=%
    \int\liminf_{n\to\infty}\chi_{E_n}(x)\diff x\leq%
    \liminf_{n\to\infty}\int\chi_{E_n}(x)\diff x=%
    \liminf_{n\to\infty}m(E_n).
  \]
\end{solution}

\begin{problem}
  Consider the function
  \[
    F(x)=
    \begin{cases}
      m(B(\mathbf{0},x))&x>0,\\
      0&x=0.
    \end{cases}
  \]
  Here \(B(\mathbf{0},r)=\left\{\, y \in\bbR^n:| y |<r\,\right\}\). Prove
  that \(F\) is monotonic increasing and continuous.
\end{problem}

\begin{solution}
  Let \(T\colon\bbR^n\times[0,x)\to\bbR^n\) be the linear map given by
  \(T(x,r)=rx\). By Problem 1, we know that
  \(T(B(\mathbf{0},1),r)=B(\mathbf{0},r)\) and consequently,
  \(m(B(\mathbf{0},1))=|r|^nm(B(\mathbf{0},1))\). Interpreting
  \(B(\mathbf{0},0)=\emptyset\), we have \(F(x)=|r|^nm(B(\mathbf{0},1))\)
  and it is easy to see that \(F\) is both monotonically increasing and
  continuous since it is a polynomial in \(r\).
\end{solution}

\begin{problem}
  Let \(f\colon\bbR\to\bbR\) be a function. Let \(C\) be the set of all
  points at which \(f\) is continuous. Show that \(C\) is a set of type
  \(G_\delta\).
\end{problem}
\begin{solution}
  Let \(C\) be the subset of \(\bbR\) where \(f\) is continuous, i.e., the
  set
  \[
    C=\left\{\,x\in\bbR:\text{given \(\varepsilon>0\) there exist
        \(\delta>0\) such that \(|f(x)-f(y)|<\varepsilon\) whenever
      \(|x-y|<\delta\)}\,\right\}.
  \]
  In light of the latter equality, for each \(n\in\bbN\) define the
  following family of subsets of \(C\),
  \[
    G_n=\left\{\,x\in\bbR:\text{there exists \(\delta_n>0\) such that
        \(|f(x)-f(y)|<\frac{1}{n}\) whenever \(|x-y|<\delta_n\)}\,\right\}.
  \]
  We claim that (i) the \(G_n\) are open and (ii)
  \(C=\bigcup_{n\in\bbN}G_n\).

  The proof of (i) is easy: let \(x\in G_n\) then there exists
  \(\delta_n>0\) such that
  \[
    |f(x)-f(y)|<\frac{1}{n}.
  \]
  Then $B(x,\delta_n)\subseteq G_n$ since \(x'\in B(x,\delta_n)\) implies
  that \(|x-x'|<\delta\) so
  \[
    |f(x)-f(x')|<\frac{1}{n}.
  \]

  The proof of (ii) is also straight forward: let \(x\in C\) then given
  \(\varepsilon>0\) there exists \(\delta>0\) such that
  \[
    |f(x)-f(y)|<\varepsilon
  \]
  whenever \(|x-y|<\delta\). By the Archimedean property of \(\bbR\) there
  exist a positive integer \(N\) such that \(1/N<\varepsilon\). Let
\end{solution}

\begin{problem}
  Let \(f\colon\bbR\to\bbR\) be a function. Is it true that if the sets
  \(\left\{\,f=r\,\right\}\) are measurable for all \(r\in\bbR\), then
  \(f\) is measurable?
\end{problem}
\begin{solution}
\end{solution}

\begin{problem}
  Let \(\left\{f_k\right\}_{k=1}^\infty\) be a sequence of measurable
  functions on \(\bbR\). Prove that the set
  \[
    \left\{\,x:\text{\(\lim_{k\to\infty} f_k(x)\) exists}\,\right\}
  \]
  is measurable.
\end{problem}
\begin{solution}
\end{solution}

\begin{problem}
  A real valued function \(f\) on an interval \([a,b]\) is said to be
  \emph{absolutely continuous} if for every \(\varepsilon>0\), there exists
  a \(\delta>0\) such that for every finite disjoint collection
  \(\left\{(a_k,b_k)\right\}_{k=1}^N\) of open intervals in \((a,b)\)
  satisfying \(\sum_{k=1}^Nb_k-a_k<\delta\), one has
  \(\sum_{k=1}^N\left|f(b_k)-f(a_k)\right|<\varepsilon\). Show that an
  absolutely continuous function on \([a,b]\) is of bounded variation on
  \([a,b]\).
\end{problem}
\begin{solution}
\end{solution}

\begin{problem}
  Let \(f\) be a continuous function from \([a,b]\) into \(\bbR\). Let
  \(\chi_{\{c\}}\) be the characteristic function of a singleton
  \(\left\{c\right\}\), that is, \(\chi_{\{c\}}(x)=0\) if \(x\neq c\) and
  \(\chi_{\{c\}}(c)=1\). Show that
  \[
    \int_a^b f d \chi_{\{c\}}=
    \begin{cases}
      0&\text{if \(c\in(a,b)\),}\\
      -f(a)&\text{if \(c=a\),}\\
      f(b)&\text{if \(c=b\).}
    \end{cases}
  \]
\end{problem}
\begin{solution}
\end{solution}

%%% Local Variables:
%%% mode: latex
%%% TeX-master: "../MA544-Quals"
%%% End:
