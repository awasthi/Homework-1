% MA 54400 Homework assignment - Spring 2016
% Problems in parentheses are relative to material not yet covered in class.
% #1 - Due Jan. 20 Read Sections 2.1, 2.3; Chapter 2: # 1, 2, 3, 11, 13.
% #2 - Due Jan. 25 Read Section 3.1; Problem #1: Show that the boundary of
% any interval has outer measure zero; Problem #2: Show that a set
% consisting of a single point has outer measure zero.
% #3 - Due Feb. 1 Read Sections 3.2-4.  Chapter 3: # 5, 7, 8, 9, 10.
% #4 - Due Feb. 8 Read Section 3.5. Chapter 3: # 12, 13, 15.
% #5 - Due Feb. 15 Read Sections 3.6, 4.1 Chapter 3: # 14, 16, 18; Chapter
% 4: # 1, 2.
% Practice Problems for Midterm 1 - Solutions
% #6 - Due Feb. 22 Read Sections 4.2-3. Chapter 4: # 4, 7, 8.
% #7 - Due Mar. 2 Read Sections 5.1-2. Chapter 4: #9, 11, 15, 18 (ignore
% the second part concerning convergence in measure); Chapter 5: # 1, 3.
% #8 - Due Mar. 7 Read Section 5.3, 6.1. Chapter 5: # 2, 4, 6, 7,
% 21. Chapter 6: # 10, 11.
% #9 - Due Mar. 21 Read Section 6.2-3. Chapter 5: # 20; 6: # 1, 3, 4, 6, 7,
% 9.
% Practice Problems for Midterm 2 (You can ignore #4) – Solutions
% #10 – Due Apr. 11 Read Section 7.1-6. Chapter 7: # 1, 2, 12, 13, 6, 8, 9
% #11 – Due Apr. 18 Read Section 5.4, Sections 8.1-4. Chapter 7: # 11, 15;
% Chapter 5: # 8, 11, 12, 17; Prove Theorem 8.3.
% #12 – Due Apr. 25 Chapter 8: # 2, 4, 5, 6, 9, 11, 16, 18.
% Final Exam on Tuesday, May 3, 8-10 am, UNIV 117
% Practice Problems for the Final Exam - Solutions


% Interesting symbols
% $\mbfalpha,\mbfvartheta,\mbfvarphi,\mbfvarkappa,\mbfvarepsilon,\mbfvarpi$
\chapter{MA 544 Spring 2016}
\thispagestyle{empty} This is material from the course MA 544 as it was
taught in the spring of 2016.  \bigskip
\section{Homework}
These exercises were assigned from Wheeden and Zygmund's \emph{Measure and
  Integral}, therefore, most of the theorems I reference will be from
\cite{wheeden-zygmund}. Other resources include \cite{folland} and
\cite{royden}. For more elementary results, I cite \cite{rudin-1}. Unless
otherwise stated, whenever we quote a result, e.g., Theorem 1.1, it is
understood to come from Wheeden and Zygmund's \emph{Measure and Integral}.

\bigskip

Throughout these notes

\begin{tabular}{cl}
  $\bbR$ & is the set of real numbers\\
  $\bbR^+$ & is the set of positive real numbers, that is, $x\in\bbR$ with
             $x\geq 0$\\
  $\bbC$ & is the set of complex numbers\\
  $\bbQ$ & is the set of rational numbers\\
  $\bbZ$ & is the set of the integers\\
  $\bbZ^+$ & is the set of positive integers, that is, $x\in\bbZ$ with
             $x\geq 0$\\
  $\bbN$ & is the set of the natural numbers $1,2,\dotsc$\\
  $A\smallsetminus B$ & is the set difference of $A$ and $B$, that is, the
                        complement of $A\cap B$ in $A$\\
  $m^*(E)$ & the outer measure of $E$\\
  $m_*(E)$ & the inner measure of $E$\\
  $m(E)$ & the Lebesgue measure of $E$\\
  $\|\cdot\|$ & the standard Euclidean norm on $\bbR^n$\\
  $f\asymp g$ & means $f$ is asymptotically equivalent to $g$, that is,
                $\lim_{x\to\infty} g(x)/f(x)=1$\\
\end{tabular}

\newpage

\subsection{Homework 1}
\begin{problem}[Wheeden \& Zygmund Ch.\@ 2, Ex.\@ 1]
  Let $f(x)=x\sin(1/x)$ for $0<x\leq 1$ and $f(0)=0$. Show that $f$ is
  bounded and continuous on $[0,1]$, but that $V[f;0,1]=+\infty$.
\end{problem}
\begin{proof}
  Set $f\coloneq x\sin (1/x)$. We will show that $f$ is bounded and
  continuous on $[0,1]$, but that it is not of bounded variation on
  $[0,1]$.

  First we show that $f$ is bounded. To see this, note that both $|x|$ and
  $|\sin (1/x)|$ are bounded by $1$ on $[0,1]$ so that their product
  $|f|=|x||\sin (1/x)|\leq 1$. Thus, $f$ is bounded on $[0,1]$.

  Next we show that $f$ is continuous. Note that since $x$ is continuous on
  $(0,1]$ and $\sin(1/x)$ is continuous on $(0,1]$ since it is the
  composition $1/x\colon (0,1]\to (1,\infty)$,
  $\sin\colon(1,\infty)\to [0,1]$ both of which are continuous
  functions. All that is left to show is that $f$ is continuous at $0$. To
  that end, it suffices to show that for any sequence of point
  ${\{x_n\}}\subset[0,1]$ converging to $0$, the sequence ${\{f(x_n)\}}$
  converges to $f(0)=0$. But, given $\varepsilon>0$, for sufficiently large
  $N$, we can bound the distance $|x_n-0|<\varepsilon$ so that for any
  $n\geq N$, we have
  \begin{equation*}
    |x_n\sin(1/x_n)-0|\leq |x_n|<\varepsilon.
  \end{equation*}
  Letting $\varepsilon\to 0$, we have $\lim_{x_n\to 0}f(x_n)=0=f(0)$. Thus,
  $f$ is continuous at $0$ and consequently on all of $[0,1]$.

  Last but not least, we show that $f$ is not of bounded variation on
  $[0,1]$. Note that $f$ is differentiable on $[0,1]$ with derivative
  $f'(x)=\sin(1/x)-(1/x)\cos(1/x)$ so by Corollary 2.10 we have
  \begin{align*}
    V[f;0,1]
    &=\int_0^1|f'|\diff x\\
    &=\int_0^1|\sin(1/x)-(1/x)\cos(1/x)|\diff x\\
    &=\int_1^\infty\frac{1}{u^2}|\sin u-u\cos u|\diff u\\
    &\geq\int_M^\infty\frac{1}{2u}\diff u\\
    &=\infty,
  \end{align*}
  where, for sufficiently large $M$, for $u\geq M$, we have
  $|\sin u-u\cos u|>u/2$. Thus, $f$ is not b.v.\@ on $[0,1]$.
\end{proof}

\begin{problem}[Wheeden \& Zygmund Ch.\@ 2, Ex.\@ 2]
  Prove theorem (2.1).
\end{problem}
\begin{proof}
  Recall the statement of theorem (2.1):
  \begin{quote}
    \begin{enumerate}[label=(\alph*),noitemsep]
    \item If $f$ is of bounded variation on $[a,b]$, then $f$ is bounded on
      $[a,b]$.
    \item Let $f$ and $g$ be of bounded variation on $[a,b]$. Then $cf$
      (for any real constant $c$), $f+g$, and $fg$ are of bounded variation
      on $[a,b]$. Moreover, $f/g$ is of bounded variation on $[a,b]$ if
      there exists an $\varepsilon>0$ such that $|g(x)|\geq\varepsilon$ for
      $x\in[a,b]$.
    \end{enumerate}
  \end{quote}
  \noindent%
  (a) We shall proceed by contradiction. Suppose $f$ is of b.v.\@ on
  $[a,b]$ with total variation $V$, but that $f$ is not bounded on
  $[a,b]$. Then, for every $M\in\bbR^+$, there exist $x\in[a,b]$ such that
  $|f(x)|>M$. Hence, there exist $x^*\in[a,b]$ such that
  $|f(x^*)|>V+|f(a)|$ so by the reverse triangle inequality we have
  \begin{align*}
    |f(a)-f(x^*)|+|f(x^*)-f(b)| &\geq
                                  |(V-|f(a)|)-|f(a)||+|V+(|f(a)|-|f(b)|)|\\
                                &=V+|V+(|f(a)|-|f(b)|)|\\
                                &>V.
  \end{align*}
  Since $\Gamma^*\coloneq\{a,x^*,b\}$ is a partition of $[a,b]$ and $f$ is
  b.v.\@ on $[a,b]$, we must have $S_{\Gamma^*}\leq V$. This is a
  contradiction. Thus, it must be the case that $f$ is bounded on
  $[a,b]$. %
  \\\\
  (b) Suppose $f$ and $g$ are b.v.\@ on $[a,b]$ with variation $V_f$ and
  $V_g$, respectively.  We will show that $cf$, $f+g$, and $fg$ are b.v.\@
  on $[a,b]$. Moreover, we show $f/g$ is b.v.\@ on $[a,b]$ if there exists
  $\varepsilon>0$ such that $|g(x)|\geq\varepsilon$ for all $x\in[a,b]$.
  \\\\
  \underline{($cf$ is b.v.\@ on $[a,b]$)}: Let $c$ be a real number. Given
  a partition $\Gamma=\{x_0,\dotsc,x_n\}$ of $[a,b]$, we have
  \begin{align*}
    S_\Gamma&=\sum_{i=0}^n|cf(x_i)-cf(x_{i-1})|\\
            &=\sum_{i=1}^n|c||f(x_i)-f(x_{i-1})|\\
            &=|c|\underbrace{\left(\sum_{i=1}^n|f(x_i)-f(x_{i-1})|\right)}_{S_{\Gamma,f}}.
  \end{align*}
  But since $f$ is b.v.\@ on $[a,b]$, $S_{\Gamma,f}\leq V_F$ so
  $S_\Gamma\leq cV_f$. Since $\Gamma$ was chosen arbitrarily, it follows
  that $cf$ is b.v.\@ on $[a,b]$.  $x\in[a,b]$.
  \\\\
  \underline{($f+g$ is b.v.\@ on $[a,b]$)}: Given a partition
  $\Gamma=\left\{ x_0,\dotsc,x_n \right\}$ of the interval $[a,b]$, we have
  the sums associated to $f+g$
  \begin{align*}
    S_{\Gamma,f+g}
    &=\sum_{i=1}^n\left|(f(x_i)+g(x_i))-(f(x_{i-1})+g(x_{i-1}))\right|\\
    &=\sum_{i=1}^n\left|(f(x_i)-f(x_{i-1}))+(g(x_i)-g(x_{i-1}))\right|\\
    &\leq\sum_{i=1}^n|f(x_i)-f(x_{i-1})|+\sum_{i=1}^n|g(x_i)-g(x_{i-1})|\\
    &\leq V_f+V_g.
  \end{align*}
  Thus, $f+g$ is b.v.\@ on $[a,b]$.
  \\\\
  \underline{($fg$ is b.v.\@ on $[a,b]$)}: First, recall that, since $f$
  and $g$ are b.v.\@ on $[a,b]$, $f$ and $g$ are bounded on $[a,b]$ by,
  say, $M_f>0$ and $M_g>0$, respectively. Now, given a partition
  $\Gamma=\{x_0,\dotsc,x_n\}$ of $[a,b]$ consider the sums associated to
  the product $fg$
  \begin{align*}
    S_{\Gamma,fg}
    &=\sum_{i=1}^n\left|f(x_i)g(x_i)-f(x_{i-1})g(x_{i-1})\right|\\
    &\begin{aligned} =\sum_{i=1}^n
      &\left|f(x_i)g(x_i)-f(x_{i-1})g(x_{i-1})\right.\\
      &\left.+f(x_i)g(x_{i-1})-f(x_i)g(x_{i-1})\right|
    \end{aligned}\\
    &\begin{aligned} =\sum_{i=1}^n
      &\left|(f(x_i)g(x_i)-f(x_i)g(x_{i-1}))\right.\\
      &\left.-(f(x_{i-1})g(x_{i-1})-f(x_i)g(x_{i-1}))\right|
    \end{aligned}\\
    &\begin{aligned}
      \leq&\sum_{i=1}^n|f(x_i)g(x_i)-f(x_i)g(x_{i-1})|\\
      &+\sum_{i=1}^n|f(x_{i-1})g(x_{i-1})-f(x_i)g(x_{i-1})|
    \end{aligned}\\
    &=\sum_{i=1}^n|f(x_i)||g(x_i)-g(x_{i-1})|+\sum_{i=1}^n|g(x_{i-1})||f(x_i)-f(x_{i-1})|\\
    &=\sum_{i=1}^n
      M_f|g(x_i)-g(x_{i-1})|+\sum_{i=1}^nM_g|f(x_i)-f(x_{i-1})|\\
    &\leq M_fV_g+M_gV_f.
  \end{align*}
  Thus, $fg$ is b.v.\@ on $[a,b]$.
  \\\\
  \underline{($f/g$ is b.v.\@ on $[a,b]$ if there exists $\varepsilon>0$
    such that $|g(x)|\geq\varepsilon$ for all $x\in[a,b]$.)}: Suppose there
  exists $\varepsilon>0$ such that $|g(x)|\geq\varepsilon$ for all
  $x\in[a,b]$. Suppose $\Gamma=\{x_0,\dotsc,x_n\}$ is a partition of
  $[a,b]$ and consider the sum associated to the quotient $f/g$
  \begin{align*}
    V_{\Gamma,f/g}
    &=\sum_{i=1}^n |f(x_i)/g(x_i)-f(x_{i-1})/g(x_{i-1})|\\
    &=\sum_{i=1}^n\left|\frac{f(x_i)g(x_{i-1})-
      f(x_{i-1})g(x_i)}{g(x_i)g(x_{i-1})}\right|\\
    &\leq\frac{1}{\varepsilon^2}\sum_{i=1}^n|f(x_i)g(x_{i-1})-f(x_{i-1})g(x_i)|\\
    &
      \begin{aligned}
        =\frac{1}{\varepsilon^2}\sum_{i=1}^n |&f(x_i)g(x_{i-1})-f(x_{i-1})g(x_{i-1})\\
        &-(f(x_{i-1})g(x_i)-f(x_{i-1})g(x_{i-1}))|
      \end{aligned}\\
    &\leq
      \frac{1}{\varepsilon^2}\sum_{i=1}^n|g(x_{i-1})||f(x_i)-f(x_{i-1})|
      +\frac{1}{\varepsilon^2}\sum_{i=1}^n|f(x_{i-1})||g(x_i)-g(x_{i-1})|\\
    &=\frac{1}{\varepsilon^2}\sum_{i=1}^nM_g|f(x_i)-f(x_{i})|
      +\frac{1}{\varepsilon^2}\sum_{i=1}^nM_f|g(x_i)-g(x_i)|\\
    &=\frac{1}{\varepsilon^2}M_g\sum_{i=1}^n|f(x_i)-f(x_{i})|
      +\frac{1}{\varepsilon^2}M_f\sum_{i=1}^n|g(x_i)-g(x_i)|\\
    &=\frac{1}{\varepsilon^2}(M_gV_f+M_fV_g).
  \end{align*}
  Thus, $f/g$ is b.v.\@ on $[a,b]$.
\end{proof}

\begin{problem}[Wheeden \& Zygmund Ch.\@ 2, Ex.\@ 3]
  If $[a',b']$ is a subinterval of $[a,b]$ show that $P[a',b']\leq P[a,b]$
  and $N[a',b']\leq N[a,b]$.
\end{problem}
\begin{proof}
  Recall that, given a partition $\Gamma=\{x_0,x_1,\dotsc,x_n\}$ of the
  interval$[a,b]$, $P_\Gamma$ and $N_\Gamma$ are defined to be the sum of
  the positive and, respectively, the negative terms of $S_\Gamma$, that
  is, the sums
  \[
    P_\Gamma\coloneq\sum_{i=1}^n\left[f(x_i)-f(x_{i-1})\right]^+
    \quad\text{and}\quad
    N_\Gamma\coloneq\sum_{i=1}^n\left[f(x_i)-f(x_{i-1})\right]^-,
  \]
  so that the positive variation $P$ and negative variation $N$ are defined
  to be
  \[
    P\coloneq\sup_\Gamma P_\Gamma \quad\text{and}\quad N\coloneq\sup_\Gamma
    N_\Gamma.
  \]

  Now, we aim to show that if $[a',b']\subset[a,b]$ then
  $P[a',b']\leq P[a,b]$ and $N[a',b']\leq N[a,b]$. We can do this in two
  ways.

  First, by Theorem 2.6 we have
  \[
    P+N=V\quad\text{and}\quad P-N=f(b)-f(a).
  \]
  Moreover, Theorem 2.2 (i) tells us that $V[a',b']\leq V[a,b]$ so
  \[
    P[a',b']=\frac{1}{2}(V[a',b']+f(b')-f(a'))<\frac{1}{2}(V[a,b]+f(b')-f(a'))
  \]
\end{proof}

\begin{problem}[Wheeden \& Zygmund Ch.\@ 2, Ex.\@ 11]
  Show that $\int_a^bf\diff\varphi$ exists if and only if given
  $\varepsilon>0$ there exists $\delta>0$ such that
  $\left|R_\Gamma-R_{\Gamma'}\right|<\varepsilon$ if
  $|\Gamma|,|\Gamma'|<\delta$.
\end{problem}
\begin{proof}
\end{proof}

\begin{problem}[Wheeden \& Zygmund Ch.\@ 2, Ex.\@ 13]
  Prove theorem (2.16).
\end{problem}
\begin{proof}
  Recall the statement of Theorem 2.16:
  \begin{quote}
    \begin{enumerate}[label=(\roman*),noitemsep]
    \item If $\int_a^b f\diff\varphi$ exists, then so do
      $\int_a^bcf\diff\varphi$ and $\int_a^b f\diff(c\varphi)$ for any
      constant $c$, and
      \[
        \int_a^bcf\diff\varphi=\int_a^bf\diff(c\varphi)=c\int_a^bf\diff\varphi.
      \]
    \item If $\int_a^b f_1\diff\varphi$ and $\int_a^bf_2\diff\varphi$ both
      exist, so does $\int_a^b\left(f_1+f_2\right)\diff\varphi$, and
      \[
        \int_a^b\left(f_1+f_2\right)\diff\varphi=\int_a^bf_1\diff\varphi+\int_a^bf_2\diff\varphi.
      \]
    \item If $\int_a^bf\diff\varphi_1$ and $\int_a^bf\diff\varphi_2$ both
      exist, so does $\int_a^bf\diff\left(\varphi_1+\varphi_2\right)$, and
      \[
        \int_a^bf\diff\left(\varphi_1+\varphi_2\right)=\int_a^bf\diff\varphi_1+\int_a^bf\diff\varphi_2.
      \]
    \end{enumerate}
  \end{quote}
\end{proof}

%%% Local Variables:
%%% mode: latex
%%% TeX-master: "../MA544-Quals"
%%% End:
