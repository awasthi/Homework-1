% MA 54400 Homework assignment - Spring 2016
% Problems in parentheses are relative to material not yet covered in class.
% #1 - Due Jan. 20 Read Sections 2.1, 2.3; Chapter 2: # 1, 2, 3, 11, 13.
% #2 - Due Jan. 25 Read Section 3.1; Problem #1: Show that the boundary of
% any interval has outer measure zero; Problem #2: Show that a set
% consisting of a single point has outer measure zero.
% #3 - Due Feb. 1 Read Sections 3.2-4.  Chapter 3: # 5, 7, 8, 9, 10.
% #4 - Due Feb. 8 Read Section 3.5. Chapter 3: # 12, 13, 15.
% #5 - Due Feb. 15 Read Sections 3.6, 4.1 Chapter 3: # 14, 16, 18; Chapter
% 4: # 1, 2.
% Practice Problems for Midterm 1 - Solutions
% #6 - Due Feb. 22 Read Sections 4.2-3. Chapter 4: # 4, 7, 8.
% #7 - Due Mar. 2 Read Sections 5.1-2. Chapter 4: #9, 11, 15, 18 (ignore
% the second part concerning convergence in measure); Chapter 5: # 1, 3.
% #8 - Due Mar. 7 Read Section 5.3, 6.1. Chapter 5: # 2, 4, 6, 7,
% 21. Chapter 6: # 10, 11.
% #9 - Due Mar. 21 Read Section 6.2-3. Chapter 5: # 20; 6: # 1, 3, 4, 6, 7,
% 9.
% Practice Problems for Midterm 2 (You can ignore #4) – Solutions
% #10 – Due Apr. 11 Read Section 7.1-6. Chapter 7: # 1, 2, 12, 13, 6, 8, 9
% #11 – Due Apr. 18 Read Section 5.4, Sections 8.1-4. Chapter 7: # 11, 15;
% Chapter 5: # 8, 11, 12, 17; Prove Theorem 8.3.
% #12 – Due Apr. 25 Chapter 8: # 2, 4, 5, 6, 9, 11, 16, 18.
% Final Exam on Tuesday, May 3, 8-10 am, UNIV 117
% Practice Problems for the Final Exam - Solutions


% Interesting symbols
% $\mbfalpha,\mbfvartheta,\mbfvarphi,\mbfvarkappa,\mbfvarepsilon,\mbfvarpi$
\chapter{MA 544 Spring 2016}
\thispagestyle{empty} This is material from the course MA 544 as it was
taught in the spring of 2016.  \bigskip
\section{Homework}
These exercises were assigned from Wheeden and Zygmund's \emph{Measure and
  Integral}, therefore, most of the theorems I reference will be from
\cite{wheeden-zygmund}. Other resources include \cite{folland} and
\cite{royden}. For more elementary results, I cite \cite{rudin-1}. Unless
otherwise stated, whenever we quote a result, e.g., Theorem 1.1, it is
understood to come from Wheeden and Zygmund's \emph{Measure and Integral}.

\bigskip

Throughout these notes

\begin{tabular}{cl}
  $\bbR$ & is the set of real numbers\\
  $\bbR^+$ & is the set of positive real numbers, that is, $x\in\bbR$ with
             $x\geq 0$\\
  $\bbC$ & is the set of complex numbers\\
  $\bbQ$ & is the set of rational numbers\\
  $\bbZ$ & is the set of the integers\\
  $\bbZ^+$ & is the set of positive integers, that is, $x\in\bbZ$ with
             $x\geq 0$\\
  $\bbN$ & is the set of the natural numbers $1,2,\ldots$\\
  $A\setminus B$ & is the set difference of $A$ and $B$, that is, the
                        complement of $A\cap B$ in $A$\\
  $m^*(E)$ & the outer measure of $E$\\
  $m_*(E)$ & the inner measure of $E$\\
  $m(E)$ & the Lebesgue measure of $E$\\
  $\|\blank\|$ & the standard Euclidean norm on $\bbR^n$\\
  $f\asymp g$ & means $f$ is asymptotically equivalent to $g$, that is,
                $\lim_{x\to\infty} g(x)/f(x)=1$\\
\end{tabular}

\newpage

\subsection{Homework 1}
\begin{problem}[Wheeden \& Zygmund Ch.\@ 2, Ex.\@ 1]
  Let $f(x)=x\sin(1/x)$ for $0<x\leq 1$ and $f(0)=0$. Show that $f$ is
  bounded and continuous on $[0,1]$, but that $V[f;0,1]=+\infty$.
\end{problem}
\begin{solution}
  Let $f$ equal $x\sin(1/x)$. We will show that $f$ is bounded and
  continuous on $[0,1]$, but that it is not of bounded variation on
  $[0,1]$.

  First we will show that $f$ is bounded. Note that both $|x|$ and
  $|\sin(1/x)|$ are bounded by $1$ on the interval $[0,1]$. Since
  $|f|=|x||\sin(1/x)|$, it follows that $|f|\leq 1$ on $[0,1]$. Thus, $f$
  is bounded on $[0,1]$.

  Next we show that $f$ is continuous. It is easy to show that $f$ is
  continuous on the subinterval $(0,1]$ since both $|x|$ and $\sin(1/x)$
  are continuous on that interval and we know that the product of
  continuous functions is continuous. To see that $f$ is continuous at $0$
  we must show that $f(x^+)=f(0)$; that is, the limit of $f$ as $x$
  approaches $0$ from the right is $f(0)$ which by definition is $0$. To
  this end, it suffices to take a (monotonically decreasing) sequence
  $x_n\searrow 0$ and show that the limit of the sequence
  ${\{f(x_n)\}}_{n=1}^\infty$ is $0$. Let $\varepsilon>0$ be given then,
  since $x_n$ converges to $0$ there exists an index $N$ such that
  $|0-x_n|<\varepsilon$ whenever $n\geq N$. Since $|f(x_n)|\leq |x_n|$ on
  $[0,1]$, the following inequality holds
  \begin{align*}
    |0-f(x_n)|
    &=\left|0-x_n\sin(1/x_n)\right|\\
    &\leq |x_n|\\
    &<\varepsilon.
  \end{align*}
  Thus, $f$ is continuous at $0$ and it converges to $0$.

  Despite the nice properties that $f$ seemingly possesses, $f$ is not
  b.v.\@ on $[0,1]$. To show that $f$ is not b.v.\@ on $[0,1]$ we must show
  that for any positive real number $M$ there exists some partition
  $\Gamma=\{\,x_0<x_1<\cdots<x_n\,\}$ of $[0,1]$ such that the sum
  associated to $\Gamma$
  \[
    \sum_{i=1}^n|f(x_i)-f(x_{i-1})|>M.
  \]
  Let $N$ be the smallest integer greater than $M$ and let $n$ be the
  smallest integer greater than or equal to $N/2$. Then the partition
  $\Gamma=\{\,x_0=1<x_1<\cdots<x_{n+1}=1\,\}$ where
  $x_i=2/((3+(n-i))\pi)$\marginpar{\textcolor{Red}{I took care to choose
      $x_i$ such that $x_{i}<x_{i+1}$ in the partition.}} for
  $1\leq i\leq N$. Then we have the inequality
  \begin{align*}
    S_\Gamma
    &=\sum_{i=1}^{n+1}|f(x_i)-f(x_{i-1})|\\
    &=\sum_{i=2}^n
      |f(x_i)-f(x_{i-1})|+|f(x_{n+1})-f(x_n)|+|f(x_0)-f(x_1)|\\
    &=N+|f(x_{n+1})-f(x_n)|+|f(x_0)-f(x_1)|\\
    &>M.
  \end{align*}
  Thus, $f$ is not b.v.\@ on $[0,1]$.
\end{solution}

\begin{problem}[Wheeden \& Zygmund Ch.\@ 2, Ex.\@ 2]
  Prove theorem (2.1).
\end{problem}
\begin{solution}
  Recall the statement of Theorem 2.1:
  \begin{quote}
    \begin{enumerate}[label=(\alph*),noitemsep]
    \item If $f$ is of bounded variation on $[a,b]$, then $f$ is bounded on
      $[a,b]$.
    \item Let $f$ and $g$ be of bounded variation on $[a,b]$. Then $cf$
      (for any real constant $c$), $f+g$, and $fg$ are of bounded variation
      on $[a,b]$. Moreover, $f/g$ is of bounded variation on $[a,b]$ if
      there exists an $\varepsilon>0$ such that $|g(x)|\geq\varepsilon$ for
      $x\in[a,b]$.
    \end{enumerate}
  \end{quote}
  \noindent%
  We shall prove these in alphabetical order:

  For part (a) we shall proceed by contradiction. First, without loss of
  generality, we may assume that $f(a)=0$ since the function the variation
  of $g(x)=f(x)-f(a)$ is equal to the variation of $f$ and $g(a)=0$.
  Suppose that $f$ is b.v.\@ on $[a,b]$ with variation $V=V[f;a,b]$, but
  that $f$ is unbounded on $[a,b]$; that is, given a positive real number
  $M$ there exists a point $x$ in $[a,b]$ such that $|f(x)|>M$. In
  particular, there exists $x\in[a,b]$ such that $|f(x)|>V$. Hence, for any
  $x\in[a,b]$ by the triangle inequality we have
  \begin{align*}
    V&<|f(x)|\\
     &=|f(x)-f(a)+f(a)|\\
     &\leq|f(x)-f(a)|+|f(a)|\\
     &\leq V.
  \end{align*}
  This is a contradiction. Therefore, it must be the case that if $f$ is
  b.v.\@ on $[a,b]$ then $f$ is bounded on $[a,b]$.

  We break part (b) into three sections. Suppose $f$ and $g$ are b.v.\@ on
  $[a,b]$ with variation $V$ and $V'$, respectively.  We will show that (i)
  $cf$; (ii) $f+g$; and (iii) $fg$ are b.v.\@ on $[a,b]$. Moreover, we show
  that (iv) $f/g$ is b.v.\@ on $[a,b]$ if there exists $\varepsilon>0$ such
  that $|g(x)|\geq\varepsilon$ for all $x\in[a,b]$.

  For part (i) above let $c$ be a real number. Given a partition
  $\Gamma=\{\,x_0<x_1<\cdots<x_n\,\}$ of $[a,b]$, we have
  \begin{align*}
    S_\Gamma
    &=\sum_{i=1}^n |cf(x_i)-cf(x_{i-1})|\\
    &=\sum_{i=1}^n |c||f(x_i)-cf(x_{i-1})|\\
    &=|c|\sum_{i=1}^n|f(x_i)-f(x_{i-1})|\\
    &\leq |c| V
  \end{align*}
  since $V$ is the supremum of the sums of the form
  $\sum_{i=1}^m|f(x_i)-f(x_{i-1})|$ over all partitions of $[a,b]$. Thus,
  $V[cf;a,b]\leq |c|V$ so $cf$ is b.v.\@ on $[a,b]$.

  For part (ii) given a partition $\Gamma=\{\,x_0<x_1<\cdots<x_n\,\}$ of
  the interval $[a,b]$, by the triangle inequality we have
  \begin{align*}
    S_\Gamma
    &=\sum_{i=1}^n\left|(f(x_i)+g(x_i))-(f(x_{i-1})+g(x_{i-1}))\right|\\
    &=\sum_{i=1}^n\left|(f(x_i)-f(x_{i-1}))+(g(x_i)-g(x_{i-1}))\right|\\
    &\leq\sum_{i=1}^n|f(x_i)-f(x_{i-1})|+\sum_{i=1}^n|g(x_i)-g(x_{i-1})|\\
    &\leq V+V'.
  \end{align*}
  Thus, $f+g$ is b.v.\@ on $[a,b]$

  For part (iii) since $f$ and $g$ are b.v.\@ on $[a,b]$ by part (a) $f$
  and $g$ are bounded on $[a,b]$ by, say, $M$ and $N$, respectively. Now,
  given a partition $\Gamma=\{\,x_0<x_1<\cdots<x_n\,\}$ of $[a,b]$, by the
  triangle inequality we have
  \begin{align*}
    S_{\Gamma}
    &=\sum_{i=1}^n\left|f(x_i)g(x_i)-f(x_{i-1})g(x_{i-1})\right|\\
    &\begin{aligned} =\sum_{i=1}^n
      \left|f(x_i)g(x_i)\right.{}&{}\left.-f(x_{i-1})g(x_{i-1})\right.\\
      &\left.+f(x_i)g(x_{i-1})-f(x_i)g(x_{i-1})\right|
    \end{aligned}\\
    &\begin{aligned} =\sum_{i=1}^n
      \left|(f(x_i)g(x_i)\right.{}&{}\left.-f(x_i)g(x_{i-1}))\right.\\
      &\left.-(f(x_{i-1})g(x_{i-1})-f(x_i)g(x_{i-1}))\right|
    \end{aligned}\\
    &\begin{aligned}
      \leq\sum_{i=1}^n|f(x_i)g(x_i){}&{}-f(x_i)g(x_{i-1})|\\
      &+\sum_{i=1}^n|f(x_{i-1})g(x_{i-1})-f(x_i)g(x_{i-1})|
    \end{aligned}\\
    &=\sum_{i=1}^n|f(x_i)||g(x_i)-g(x_{i-1})|+\sum_{i=1}^n|g(x_{i-1})||f(x_i)-f(x_{i-1})|\\
    &=\sum_{i=1}^n
      M|g(x_i)-g(x_{i-1})|+\sum_{i=1}^n N|f(x_i)-f(x_{i-1})|\\
    &\leq MV'+NV.
  \end{align*}
  Thus, $fg$ is b.v.\@ on $[a,b]$.

  Finally, for part (iv) suppose there exists $\varepsilon>0$ such that
  $|g(x)|\geq\varepsilon$ for all $x\in[a,b]$. Then, given a partition
  $\Gamma=\{\,x_0<x_1<\cdots<x_n\,\}$ of $[a,b]$, largely by the triangle
  inequality, we have
  \begin{align*}
    S_\Gamma
    &=\sum_{i=1}^n |f(x_i)/g(x_i)-f(x_{i-1})/g(x_{i-1})|\\
    &=\sum_{i=1}^n\left|\frac{f(x_i)g(x_{i-1})-
      f(x_{i-1})g(x_i)}{g(x_i)g(x_{i-1})}\right|\\
    &\leq\frac{1}{\varepsilon^2}\sum_{i=1}^n|f(x_i)g(x_{i-1})-f(x_{i-1})g(x_i)|\\
    &
      \begin{aligned}
        =\frac{1}{\varepsilon^2}\sum_{i=1}^n |f(x_i)g(x_{i-1}){}&{}-f(x_{i-1})g(x_{i-1})\\
        &-(f(x_{i-1})g(x_i)-f(x_{i-1})g(x_{i-1}))|
      \end{aligned}\\
    &\leq
      \frac{1}{\varepsilon^2}\sum_{i=1}^n|g(x_{i-1})||f(x_i)-f(x_{i-1})|
      +\frac{1}{\varepsilon^2}\sum_{i=1}^n|f(x_{i-1})||g(x_i)-g(x_{i-1})|\\
    &=\frac{1}{\varepsilon^2}\sum_{i=1}^nM_g|f(x_i)-f(x_{i})|
      +\frac{1}{\varepsilon^2}\sum_{i=1}^nM_f|g(x_i)-g(x_i)|\\
    &=\frac{1}{\varepsilon^2}M_g\sum_{i=1}^n|f(x_i)-f(x_{i})|
      +\frac{1}{\varepsilon^2}M_f\sum_{i=1}^n|g(x_i)-g(x_i)|\\
    &\leq\frac{1}{\varepsilon^2}(NV+MV')
  \end{align*}
  where, as above, $f$ is bounded by $M$ and $g$ is bounded by $N$. Thus,
  $f/g$ is b.v.\@ on $[a,b]$.
\end{solution}

\begin{problem}[Wheeden \& Zygmund Ch.\@ 2, Ex.\@ 3]
  If $[a',b']$ is a subinterval of $[a,b]$ show that $P[a',b']\leq P[a,b]$
  and $N[a',b']\leq N[a,b]$.
\end{problem}
\begin{solution}
  Recall that, given a partition $\Gamma=\{x_0,x_1,\ldots,x_n\}$ of the
  interval $[a,b]$, $P_\Gamma$ and $N_\Gamma$ are defined to be the sum of
  the positive and, respectively, the negative terms of $S_\Gamma$, that
  is, the sums
  \begin{equation}
    \label{eq:1.1}
    P_\Gamma\coloneq\sum_{i=1}^n\left[f(x_i)-f(x_{i-1})\right]^+
    \quad\text{and}\quad
    N_\Gamma\coloneq\sum_{i=1}^n\left[f(x_i)-f(x_{i-1})\right]^-,
  \end{equation}
  so that the positive variation $P$ and negative variation $N$ are defined
  to be
  \[
    P\coloneq\sup_\Gamma P_\Gamma \quad\text{and}\quad N\coloneq\sup_\Gamma
    N_\Gamma.
  \]

  Now, we aim to show that if $[a',b']\subset[a,b]$ then
  $P[a',b']\leq P[a,b]$ and $N[a',b']\leq N[a,b]$.  We shall proceed as
  follows: given a partition $\Gamma'=\{x_0',\ldots,x_n'\}$ of $[a,b]$,
  extend $\Gamma'=\{x_0,\ldots,x_m\}$ where $x_i'=x_k$ for some
  $0\leq k\leq m-n$ for $0\leq i\leq n$ (clearly $n\leq m$ for this to make
  sense) to a partition $\Gamma$ of $[a,b]$; by which we mean $\Gamma$ is a
  partition of $[a,b]$ with $\Gamma'\subset\Gamma$. Then by the definition
  in \eqref{eq:1.1} we have
  \begin{align*}
    P_\Gamma
    &=\sum_{i=1}^m\left[f(x_i)-f(x_{i-1})\right]^+\\
    &=\sum_{i=1}^k\left[f(x_i)-f(x_{i-1})\right]^+
      +\sum_{i=k}^n\left[f(x_i)-f(x_{i-1})\right]
      +\sum_{i=n}^m\left[f(x_i)-f(x_{i-1})\right]\\
    &=\sum_{i=1}^k\left[f(x_i)-f(x_{i-1})\right]^+
      +\sum_{i=n}^m\left[f(x_i)-f(x_{i-1})\right]
      +P_{\Gamma'}\\
    &\geq P_{\Gamma'}
  \end{align*}
  so that, taking the supremum on both sides, we have
  $P[a,b]\geq P[a',b']$. The same argument can be used to show that
  $N[a,b]\geq N[a',b']$ by replacing $N$ by $P$ in the statements we made
  above.
\end{solution}

\begin{problem}[Wheeden \& Zygmund Ch.\@ 2, Ex.\@ 11]
  Show that $\int_a^bf\diff\varphi$ exists if and only if given
  $\varepsilon>0$ there exists $\delta>0$ such that
  $\left|R_\Gamma-R_{\Gamma'}\right|<\varepsilon$ if
  $|\Gamma|,|\Gamma'|<\delta$.
\end{problem}
\begin{solution}
  One direction is straightforward $\impliedby$: suppose that given
  $\varepsilon>0$ there exists $\delta>0$ such that
  $|R_\Gamma-R_{\Gamma'}|<\varepsilon$ whenever
  $|\Gamma|,|\Gamma'|<\delta$. Let $\{\Gamma_n\}$ be a sequence of nested
  partitions --- that is, $\Gamma_i\subset\Gamma_{i+1}$ --- of $[a,b]$ with
  $\lim_{n\to\infty}|\Gamma_n|=0$. Then there exists an index $N$ such that
  $m,n\geq N$ implies $|\Gamma_m|,|\Gamma_n|<\delta$. Then by the
  hypothesis we have
  \[
    \left|R_{\Gamma_n}-R_{\Gamma_m}\right|<\varepsilon.
  \]
  By Cauchy's criterion for convergence this implies that the
  Riemann--Stieltjes integral $\int_a^b f\diff\varphi$ exists.

  $\implies$: On the other hand, suppose that
  $I\coloneq\int_a^b f\diff\varphi$ exists. Then given $\varepsilon>0$
  there exists $\delta>0$ such that $|I-R_\Gamma|<\varepsilon/2$ whenever
  $|\Gamma|<\delta$. Take any two partitions $\Gamma_1$ and $\Gamma_2$ with
  $|\Gamma_1|,|\Gamma_2|<\delta$. Then
  \begin{align*}
    |R_{\Gamma_1}-R_{\Gamma_2}|
    &=\left|R_{\Gamma_1}-I-(R_{\Gamma_2}-I)\right|\\
    &\leq|R_{\Gamma_1}-I|+|R_{\Gamma_2}-I|\\
    &<\frac{\varepsilon}{2}+\frac{\varepsilon}{2}\\
    &=\varepsilon
  \end{align*}
  as desired.
\end{solution}

\begin{problem}[Wheeden \& Zygmund Ch.\@ 2, Ex.\@ 13]
  Prove theorem (2.16).
\end{problem}
\begin{solution}
  Recall the statement of Theorem 2.16:
  \begin{quote}
    \begin{enumerate}[label=(\roman*),noitemsep]
    \item If $\int_a^b f\diff\varphi$ exists, then so do
      $\int_a^bcf\diff\varphi$ and $\int_a^b f\diff(c\varphi)$ for any
      constant $c$, and
      \[
        \int_a^bcf\diff\varphi=\int_a^bf\diff(c\varphi)=c\int_a^bf\diff\varphi.
      \]
    \item If $\int_a^b f_1\diff\varphi$ and $\int_a^bf_2\diff\varphi$ both
      exist, so does $\int_a^b\left(f_1+f_2\right)\diff\varphi$, and
      \[
        \int_a^b\left(f_1+f_2\right)\diff\varphi=\int_a^bf_1\diff\varphi+\int_a^bf_2\diff\varphi.
      \]
    \item If $\int_a^bf\diff\varphi_1$ and $\int_a^bf\diff\varphi_2$ both
      exist, so does $\int_a^bf\diff\left(\varphi_1+\varphi_2\right)$, and
      \[
        \int_a^bf\diff\left(\varphi_1+\varphi_2\right)=\int_a^bf\diff\varphi_1+\int_a^bf\diff\varphi_2.
      \]
    \end{enumerate}
  \end{quote}
  \noindent%
  (i) Suppose that $I\coloneq\int_a^bf\diff\varphi$ exists. By Problem 4 we
  know that $I$ exists if and only if given $\varepsilon>0$ there exists
  $\delta>0$ such that whenever
  $\Gamma_1=\{x_1^1,\ldots,x_{n_1}^1\},\Gamma_2=\{x_1^2,\ldots,x_{n_2}^2\}$
  are partitions of $[a,b]$ with norm $|\Gamma_1|,|\Gamma_2|<\delta$ then
  $|R_{\Gamma_1}-R_{\Gamma_2}|<\varepsilon/|c|$. Now, consider the
  Riemann--Stieltjes sums of the pair $cf$, $\varphi$ with respect to
  $\Gamma_1$ and $\Gamma_2$, call them $R_{\Gamma_1}'$ and $R_{\Gamma_2}'$
  \begin{align*}
    |R_{\Gamma_1}'-R_{\Gamma_2}'|
    &=\left|\sum_{i=1}^{n_1}cf(\xi_i^1)\left[\varphi(x_i^1)-\varphi(x_{i-1}^1)\right]
      -\sum_{i=1}^{n_2}cf(\xi_i^2)\left[\varphi(x_i^2)-\varphi(x_{i-1}^2)\right]\right|\\
    &=|c|\left|\sum_{i=1}^{n_1}f(\xi_i^1)\left[\varphi(x_i^1)-\varphi(x_{i-1}^1)\right]
      -\sum_{i=1}^{n_2}f(\xi_i^2)\left[\varphi(x_i^2)-\varphi(x_{i-1}^2)\right]\right|\\
    &=|c||R_{\Gamma_1}-R_{\Gamma_2}|\\
    &<\frac{|c|\varepsilon}{|c|}\\
    &=\varepsilon.
  \end{align*}
  Thus, by Problem 4 $\int_a^b cf\diff\varphi$ exist. A slight modification
  to the argument we have made yields the same conclusion for
  $\int_a^b f\diff(c\varphi)$.
  \\\\
  (ii) Leaving the notation as in (i), let $R_{\Gamma_1}^1$ and
  $R_{\Gamma_2}^1$ be the Riemann--Stieltjes sums of $f_1$ with respect to
  $\Gamma_1$ and $\Gamma_2$ and $R_{\Gamma_1}^2$ and $R_{\Gamma_2}^2$ be
  the Riemann--Stieltjes sums of $f_2$ with respect to $\Gamma_1$ and
  $\Gamma_2$. By assumption, we can make the sums
  \[
    |R_{\Gamma_1}^1-R_{\Gamma_2}^1|<\frac{\varepsilon}{2}
    \qquad\text{and}\qquad
    |R_{\Gamma_1}^2-R_{\Gamma_2}^2|<\frac{\varepsilon}{2}.
  \]
  Then we have
  \begin{align*}
    |R_{\Gamma_1}-R_{\Gamma_2}|
    &{}={}\left|
      \sum_{i=1}^{n_1}(f_1(\xi_i^1)+f_2(\xi_i^1))
      \left[\varphi(x_i^1)-\varphi(x_{i-1}^1)\right]
      -\sum_{i=1}^{n_1}(f_1(\xi_i^2)+f_2(\xi_i^2))
      \left[\varphi(x_i^2)-\varphi(x_{i-1}^2)\right]
      \right|\\
    {}&{}\begin{aligned}
      &=\left|\sum_{i=1}^{n_1}f_1(\xi_i^1)
        \left[\varphi(x_i^1)-\varphi(x_{i-1}^1)\right]
        -\sum_{i=1}^{n_1}f_1(\xi_i^2)
        \left[\varphi(x_i^2)-\varphi(x_{i-1}^2)
        \right]\right.\\
      &\phantom{{}={}}+\left. \sum_{i=1}^{n_1}f_2(\xi_i^1)
        \left[\varphi(x_i^1)-\varphi(x_{i-1}^1)\right]
        -\sum_{i=1}^{n_1}f_2(\xi_i^2)
        \left[\varphi(x_i^2)-\varphi(x_{i-1}^2) \right] \right|
    \end{aligned}\\
    &\begin{aligned}
      &\leq\left|\sum_{i=1}^{n_1}f_1(\xi_i^1)
        \left[\varphi(x_i^1)-\varphi(x_{i-1}^1)\right]
        -\sum_{i=1}^{n_1}f_1(\xi_i^2)
        \left[\varphi(x_i^2)-\varphi(x_{i-1}^2)
        \right]\right|\\
      &\phantom{{}={}}+\left|\sum_{i=1}^{n_1}f_2(\xi_i^1)
        \left[\varphi(x_i^1)-\varphi(x_{i-1}^1)\right]
        -\sum_{i=1}^{n_1}f_2(\xi_i^2)
        \left[\varphi(x_i^2)-\varphi(x_{i-1}^2) \right] \right|
    \end{aligned}\\
    &{}={}|R_{\Gamma_1}^1-R_{\Gamma_2}^1|+|R_{\Gamma_1}^2-R_{\Gamma_2}^2|\\
    &{}<{}\frac{\varepsilon}{2}+\frac{\varepsilon}{2}.
  \end{align*}
  Thus, $\int_a^b (f_1+f_2)\diff\varphi$ exists and it is equal to
  $I_1+I_2$.
  \\\\
  (iii) Suppose that $J_1\coloneq\int_a^bf\diff\varphi_1$ and
  $J_2\coloneq\int_a^bf\diff\varphi_2$ exist. By Problem 4, given
  $\varepsilon>0$ there exists $\delta>0$ such that for any pair $\Gamma_1$,
  $\Gamma_2$ of partitions of $[a,b]$ with $|\Gamma_1|,|\Gamma_2|<\delta$
  we have
  \[
    |R_{\Gamma_1}^1-R_{\Gamma_2}^1|<\frac{\varepsilon}{2}\quad\text{and}\quad
    |R_{\Gamma_1}^2-R_{\Gamma_2}^2|<\frac{\varepsilon}{2},
  \]
  where $R_{\Gamma_1}^1$ and $R_{\Gamma_2}^1$ is the Riemann--Stieltjes sum
  of $(f,\varphi_1)$ with respect to $\Gamma_1$ and $\Gamma_2$ and
  $R_{\Gamma_1}^2$ and $R_{\Gamma_2}^2$ is the Riemann--Stieltjes sum of
  $(f,\varphi_2)$ with respect to $\Gamma_1$ and $\Gamma_2$. Then for the
  pair $(f,\varphi_1+\varphi_2)$ we have
  \begin{align*}
    |R_{\Gamma_1}-R_{\Gamma_2}|
    &=%
      \left|%
      \sum_{i=1}^{n_1}f(\xi_i^1)\left[\varphi_1(x_i^1)+\varphi_2(x_i^1)
      -(\varphi_1(x_i^1)+\varphi_2(x_i^1))\right]\right.\\
    &\phantom{{}={}}%
      \left.%
      -\sum_{i=1}^{n_2}f(\xi_i^2)\left[\varphi_1(x_i^2)+\varphi_2(x_i^2)
      -(\varphi_1(x_i^2)+\varphi_2(x_i^2))\right]
      \right|\\
    &=\left|%
      R_{\Gamma_1}^1-R_{\Gamma_2}^1+(R_{\Gamma_1}^2-R_{\Gamma_2}^2)
      \right|\\
    &\leq|%
      R_{\Gamma_1}^1-R_{\Gamma_2}^1
      |+
      |%
      (R_{\Gamma_1}^2-R_{\Gamma_2}^2)
      |\\
    &<\frac{\varepsilon}{2}+\frac{\varepsilon}{2}\\
    &=\varepsilon.
  \end{align*}
  Thus, $\int_a^bf\diff(\varphi_1+\varphi_2)$ exists and we can easily see
  that it is equal to the sum
  \[
    \int_a^bf\diff\varphi_1
    +\int_a^bf\diff\varphi_2.
  \]
\end{solution}

%%% Local Variables:
%%% mode: latex
%%% TeX-master: "../MA544-Quals"
%%% End:
