% MA 54400 Homework assignment - Spring 2016
% Problems in parentheses are relative to material not yet covered in class.
% #1 - Due Jan. 20 Read Sections 2.1, 2.3; Chapter 2: # 1, 2, 3, 11, 13.
% #2 - Due Jan. 25 Read Section 3.1; Problem #1: Show that the boundary of
% any interval has outer measure zero; Problem #2: Show that a set
% consisting of a single point has outer measure zero.
% #3 - Due Feb. 1 Read Sections 3.2-4.  Chapter 3: # 5, 7, 8, 9, 10.
% #4 - Due Feb. 8 Read Section 3.5. Chapter 3: # 12, 13, 15.
% #5 - Due Feb. 15 Read Sections 3.6, 4.1 Chapter 3: # 14, 16, 18; Chapter
% 4: # 1, 2.
% Practice Problems for Midterm 1 - Solutions
% #6 - Due Feb. 22 Read Sections 4.2-3. Chapter 4: # 4, 7, 8.
% #7 - Due Mar. 2 Read Sections 5.1-2. Chapter 4: #9, 11, 15, 18 (ignore
% the second part concerning convergence in measure); Chapter 5: # 1, 3.
% #8 - Due Mar. 7 Read Section 5.3, 6.1. Chapter 5: # 2, 4, 6, 7,
% 21. Chapter 6: # 10, 11.
% #9 - Due Mar. 21 Read Section 6.2-3. Chapter 5: # 20; 6: # 1, 3, 4, 6, 7,
% 9.
% Practice Problems for Midterm 2 (You can ignore #4) – Solutions
% #10 – Due Apr. 11 Read Section 7.1-6. Chapter 7: # 1, 2, 12, 13, 6, 8, 9
% #11 – Due Apr. 18 Read Section 5.4, Sections 8.1-4. Chapter 7: # 11, 15;
% Chapter 5: # 8, 11, 12, 17; Prove Theorem 8.3.
% #12 – Due Apr. 25 Chapter 8: # 2, 4, 5, 6, 9, 11, 16, 18.
% Final Exam on Tuesday, May 3, 8-10 am, UNIV 117
% Practice Problems for the Final Exam - Solutions


% Interesting symbols
% $\mbfalpha,\mbfvartheta,\mbfvarphi,\mbfvarkappa,\mbfvarepsilon,\mbfvarpi$
\chapter{MA 544 Spring 2016}
\thispagestyle{empty}
This is material from the course MA 544 as it was taught in the spring of
2016.
\bigskip
\section{Homework}
These exercises were assigned from Wheeden and Zygmund's \emph{Measure and
  Integral}, therefore, most of the theorems I reference will be from
\cite{wheeden-zygmund}. Other resources include \cite{folland} and
\cite{royden}. For more elementary results, I cite \cite{rudin-1}.

\bigskip

Throughout these notes

\begin{tabular}{cl}
  $\bbR$ & is the set of real numbers\\
  $\bbR^+$ & is the set of positive real numbers, that is, $x\in\bbR$ with
             $x\geq 0$\\
  $\bbC$ & is the set of complex numbers\\
  $\bbQ$ & is the set of rational numbers\\
  $\bbZ$ & is the set of the integers\\
  $\bbZ^+$ & is the set of positive integers, that is, $x\in\bbZ$ with
             $x\geq 0$\\
  $\bbN$ & is the set of the natural numbers $1,2,\dotsc$\\
  $A\smallsetminus B$ & is the set difference of $A$ and $B$, that is, the
                        complement of $A\cap B$ in $A$\\
  $m^*(E)$ & the outer measure of $E$\\
  $m_*(E)$ & the inner measure of $E$\\
  $m(E)$ & the Lebesgue measure of $E$\\
  $\|\cdot\|$ & the standard Euclidean norm on $\bbR^n$
\end{tabular}

\newpage

\subsection{Homework 1}
\begin{problem}[Wheeden \& Zygmund Ch.\@ 2, Ex.\@ 1]
Let $f(x)=x\sin(1/x)$ for $0<x\leq 1$ and $f(0)=0$. Show that $f$ is
bounded and continuous on $[0,1]$, but that $V[f;0,1]=+\infty$.
\end{problem}
\begin{proof}
 Set $f\coloneqq x\sin (1/x)$. We will show that $f$ is bounded and
 \gls{cont} on $[0,1]$ but that, nevertheless, $f$ is not of \gls{bv} on
 $[0,1]$.

 To see that $f$ is bounded, we note that both $x$ and $\sin(1/x)$ are
 bounded by $1$ on $[0,1]$. Thus, the absolute value of the product product
 $|{x\sin(1/x)}|$ is bounded by $1$ so $f$ is bounded on $[0,1]$.

 To see that $f$ is continuous, we use note that, since $\sin(1/x)$ and $x$
 are continuous on $(0,1]$, by properties of continuous functions (you may
 refer to \cite[Ch.\@ 4, p.\@ 87]{rudin-1}), $f$ is continuous on $(0,1]$.
 What is not obvious is continuity at $0$. To show that $f$ is continuous
 at $0$, by Theorem 4.6 from \cite[Ch.\@ 4, p.\@ 86]{rudin-1}, it suffices
 to show that $\lim_{x\to 0}f(x)=0$. Consider the sequence $\{a_n\}$ where
 $a_n\coloneqq 1/n$. The sequence $a_n\to 0$ as $n\to\infty$. Now, given
 $\varepsilon>0$, by the \gls{ArchProp}, for sufficiently  large
 $N\in\bbZ^+$, $1/N<\varepsilon$ holds so for every $n\geq N$. Thus, we have
 \begin{equation}
   \label{eq:1:1}
   |{(1/k)\sin(k)-0}|\leq
   |1/k|<\varepsilon.
 \end{equation}
 Thus, $\lim_{k\to\infty}f(1/k)=0$. Thus, $f$ is continuous on all of
 $[0,1]$.

 Nevertheless, $f$ is not b.v.\@ on $[0,1]$. To see this, note that since
 $f$ is differentiable on $[0,1]$, by Corollary 2.10 from \cite[Ch.\@ 2,
 p.\@ 23]{wheeden-zygmund}, we have
 \begin{equation}
   \label{eq:1:2}
   \begin{aligned}
     V&=\int_0^1|f'|\diff x\\
     &=\int_0^1|{\sin(1/x)-(1/x)\cos(1/x)}|\diff x\\
     &=\int_1^\infty\frac{1}{u^2}|{\sin u-u\cos u}|\diff x\\
     &\geq\int_M^\infty \frac{1}{2u}\diff u\\
     &=\infty,
   \end{aligned}
 \end{equation}
 where, for sufficiently large $M\in\bbZ^+$, for $u\geq M$ we have
 $|{\sin u- u\cos u}|>u/2$. Thus, $f$ is not b.v.\@ on $[0,1]$.
\end{proof}

\begin{problem}[Wheeden \& Zygmund Ch.\@ 2, Ex.\@ 2]
Prove theorem (2.1).
\end{problem}
\begin{proof}
Recall the statement of theorem (2.1):
\begin{quote}
\begin{enumerate}[label=(\alph*),noitemsep]
\item If $f$ is of bounded variation on $[a,b]$, then $f$ is bounded on
  $[a,b]$.
\item Let $f$ and $g$ be of bounded variation on $[a,b]$. Then $cf$ (for
  any real constant $c$), $f+g$, and $fg$ are of bounded variation on
  $[a,b]$. Moreover, $f/g$ is of bounded variation on $[a,b]$ if there
  exists an $\varepsilon>0$ such that $|g(x)|\geq\varepsilon$ for
  $x\in[a,b]$.
\end{enumerate}
\end{quote}
\noindent
(a)
\\\\
(b)
\end{proof}

\begin{problem}[Wheeden \& Zygmund Ch.\@ 2, Ex.\@ 3]
If $[a',b']$ is a subinterval of $[a,b]$ show that $P[a',b']\leq P[a,b]$
and $N[a',b']\leq N[a,b]$.
\end{problem}
\begin{proof}
\end{proof}
\begin{problem}[Wheeden \& Zygmund Ch.\@ 2, Ex.\@ 11]
Show that $\int_a^bf\diff\varphi$ exists if and only if given $\varepsilon>0$
there exists $\delta>0$ such that
$\left|R_\Gamma-R_{\Gamma'}\right|<\varepsilon$ if
$|\Gamma|,|\Gamma'|<\delta$.
\end{problem}
\begin{proof}
\end{proof}

\begin{problem}[Wheeden \& Zygmund Ch.\@ 2, Ex.\@ 13]
Prove theorem (2.16).
\end{problem}
\begin{proof}
Recall the statement of Theorem 2.16:
\begin{quote}
\begin{enumerate}[label=(\roman*),noitemsep]
\item If $\int_a^b f\diff\varphi$ exists, then so do $\int_a^bcf\diff\varphi$ and
  $\int_a^b f\diff(c\varphi)$ for any constant $c$, and
\[
\int_a^bcf\diff\varphi=\int_a^bf\diff(c\varphi)=c\int_a^bf\diff\varphi.
\]
\item If $\int_a^b f_1\diff\varphi$ and $\int_a^bf_2\diff\varphi$ both exist, so
  does $\int_a^b\left(f_1+f_2\right)\diff\varphi$, and
\[
\int_a^b\left(f_1+f_2\right)\diff\varphi=\int_a^bf_1\diff\varphi+\int_a^bf_2\diff\varphi.
\]
\item If $\int_a^bf\diff\varphi_1$ and $\int_a^bf\diff\varphi_2$ both exist, so
  does $\int_a^bf\diff\left(\varphi_1+\varphi_2\right)$, and
\[
\int_a^bf\diff\left(\varphi_1+\varphi_2\right)=\int_a^bf\diff\varphi_1+\int_a^bf\diff\varphi_2.
\]
\end{enumerate}
\end{quote}
\end{proof}

%%% Local Variables:
%%% mode: latex
%%% TeX-master: "../MA544-Quals"
%%% End:
