\subsection{Homework 2}
\begin{problem}
  Show that the boundary of any interval has outer measure zero.
\end{problem}
\begin{solution}
  Let $I\coloneq\prod_{i=1}^n I_i$ be a closed interval in $\bbR^n$ and let
  $J$ be the boundary of $I$. We must show that given $\varepsilon>0$ there
  exists a countable collection of intervals $\{I_n\}_{n\in J}$ covering
  $J$ such that
  \[
    \sum_{n\in J}\vol(I_n)<\varepsilon.
  \]
  First, note that we can write $J$ as the union $\bigcup_{i=1}^n J_i$
  where
  \[
    J_i\coloneq%
    % \underbrace{
      [a_1,b_1]\times\cdots\times\{a_i\}\times\cdots\times[a_n,b_n]%
    % }_{J_i^1}
    \cup
    [a_1,b_1]\times\cdots\times\{b_i\}\times\cdots\times[a_n,b_n].
  \]
  Since the countable union of null sets has measure zero, it suffices to
  show that the set
  \[
    [a_1,b_1]\times\cdots\times[a_{n-1},b_{n-1}]\times\{a_n\}%
  \]
  has measure zero. Consider the collection $\{I_\varepsilon\}$ consisting
  of the single interval
  \[
    I_\varepsilon\coloneq [a_1,b_1]\times\cdots\times[a_{n-1},b_{n-1}]
    \times\left[a_n-\frac{\varepsilon}{2\prod_{i=1}^{n-1}(b_i-a_i)},
      a_n+\frac{\varepsilon}{2\prod_{i=1}^{n-1}(b_i-a_i)}\right].
  \]
  It is clear that $I_\varepsilon\supset J$. Now, computing the volume of
  this interval, we have
  \begin{align*}
    \vol(I_\varepsilon)%
    &=\prod_{i=1}^{n-1}(b_i-a_i)%
    \left[%
    a_n+\frac{\varepsilon}{2\prod_{i=1}^{n-1}(b_i-a_i)}%
    -\left(%
    a_n -\frac{\varepsilon}{2\prod_{i=1}^{n-1}(b_i-a_i)} \right)
    \right]\\
    &=\left[\prod_{i=1}^{n-1}(b_i-a_i)\right]%
      \frac{\varepsilon}{\prod_{i=1}^{n-1}(b_i-a_i)}\\
    &=\varepsilon.
  \end{align*}
  Thus, $J$ has measure zero.
\end{solution}

\begin{problem}
  Show that a set consisting of a single point has outer measure zero.
\end{problem}
\begin{solution}
  Let $\{a\}$ be the set consisting of a single point $a\in\bbR$. Then we
  must show that given $\varepsilon>0$ there exists a countable collection
  of intervals $\{I_n\}$ such that
  \[
    \sum_{n\in J} m(I_n)<\varepsilon.
  \]
  Consider the collection $\{I_\varepsilon\}$ consisting of the single
  interval
  \[
    I_\varepsilon\coloneq
    \left[a-\frac{\varepsilon}{2},a+\frac{\varepsilon}{2}\right].
  \]
  It is clear that $\{a\}\subset I_\varepsilon$. Moreover,
  \begin{align*}
    \vol(I_\varepsilon)
    &=a+\frac{\varepsilon}{2}-\left(a-\frac{1}{\varepsilon}\right)\\
    &=\varepsilon.
  \end{align*}
  Thus, $\{a\}$ has measure zero.
\end{solution}

%%% Local Variables:
%%% mode: latex
%%% TeX-master: "../MA544-Quals"
%%% End:
