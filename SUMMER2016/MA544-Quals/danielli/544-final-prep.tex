\subsubsection{Final Exam Practice Problems}
\setcounter{exercise}{0}
\setcounter{equation}{0}

\begin{problem}
Suppose $f\in L^1(\bbR^n)$ and that $x$ is a point in the Lebesgue set of
$f$. For $r>0$, let
\[
A(r)=\frac{1}{|r|^n}\int_{B(0,r)}|f( x - y )-f( x )|\diff y .
\]
Show that:
\begin{enumerate}[label=(\alph*),noitemsep]
\item $A(r)$ is a continuous function of $r$, and $A(r)\to 0$ as $r\to 0$;
\item there exists a constant $M>0$ such that $A(r)\leq M$ for all $r>0$.
\end{enumerate}
\end{problem}
\begin{solution}
(a) Without loss of generality, we may assume $r<s$. Then, we want to show
that as $r\to s$, the quantity
\[
|A(s)-A(r)|\longrightarrow 0.
\]
Set $F( y )= |f( x - y )-f( x )|$ and consider said quantity
\begin{align*}
|A(s)-A(r)|
&=\left|
\frac{1}{|s|^n}\int_{B_s}F( y )\diff y
-
\frac{1}{|r|^n}\int_{B_r}F( y )\diff y
\right|\\
&=\left|
\frac{1}{|s|^n}\int_{B_s\setminus B_r}F( y )\diff y +
\frac{1}{|s|^n}\int_{B_r}F( y )\diff y -
\frac{1}{|r|^n}\int_{B_r}F( y )\diff y
\right|\\
&=\left|
\frac{1}{|s|^n}\int_{B_s\setminus B_r}F( y )\diff y
+\left(\frac{1}{|s|^n}-\frac{1}{|r|^n}\right)\int_{B_r}F( y )\diff y
\right|\\
&\leq
\underbrace{\frac{1}{|s|^n}\int_{B_s\setminus B_r}F( y )\diff y }_{I_1}
+\underbrace{\left(\frac{1}{|s|^n}-\frac{1}{|r|^n}\right)\int_{B_r}F( y )\diff y }_{I_2}.
\end{align*}
Hence, we must show that the quantities $I_1,I_2\to 0$ as $r\to s$.

To see that $A(r)\to 0$ as $r\to 0$, note that $x$ is a point of the
Lebesgue set of $f$ and that
\[
0=\lim_{B_r\searrow x }\frac{1}{|B_1||r|^n}\int_{B_r}|f( y )-f( x )|\diff y =\frac{1}{|B_1|}\lim_{B_r\searrow x }\frac{1}{|r|^n}\int_{B_r}|f(\bbt)-f( x )|\diff\bbt=\lim_{r\to
0}A(r).
\]
by making the change of variables $\bbt= x - y $.
\\\\
(b)
\end{solution}

\begin{problem}
Let $E\subset\bbR^n$ be a measurable set, $1\leq n<\infty$. Assume
$\{f_k\}$ is a sequence in $L^p(E)$ converging pointwise a.e.\@ on $E$ to a
function $f\in L^p(E)$. Prove that
\[
\|{f_k-f}\|_p\longrightarrow 0
\]
if and only if
\[
\|{f_k}\|_p\longrightarrow\|f\|_p
\]
as $k\to\infty$.
\end{problem}
\begin{solution}
\end{solution}

\begin{problem}
Let $1<p<\infty$, $f\in L^p(E)$, $g\in L^{p'}(E)$.
\begin{enumerate}[label=(\alph*),noitemsep]
\item Prove that $f*g\in C(\bbR^n)$.
\item Does this conclusion continue to be valid when $p=1$ and $p=\infty$?
\end{enumerate}
\end{problem}
\begin{solution}
\end{solution}

\begin{problem}
Let $f\in L(\bbR)$, and let $F(t)=\int_{\bbR}f(x)\cos(tx)dx$.
\begin{enumerate}[label=(\alph*),noitemsep]
\item Prove that $F(t)$ is continuous for $t\in\bbR$.
\item Prove the following
  \href{https://en.wikipedia.org/wiki/Riemann–Lebesgue_lemma}{\emph{Riemann--Lebesgue
      lemma}}:
\[
\lim_{t\to\infty}F(t)=0.
\]
\end{enumerate}
\end{problem}
\begin{solution}
\end{solution}

\begin{problem}
Let $f$ be of bounded variation on $[a,b]$, $-\infty<a<b<\infty$. If
$f=g+h$, with $g$ absolutely continuous and $h$ singular. Show that
\[
\int_a^b\varphi \diff f=\int_a^b\varphi f'dx+\int_a^b\varphi \diff h
\]
for all functions $\varphi$ continuous on $[a,b]$.
\end{problem}
\begin{solution}
\end{solution}

%%% Local Variables:
%%% mode: latex
%%% TeX-master: "../MA544-Quals"
%%% End:
