\subsubsection{Homework 7}
\setcounter{exercise}{0}
\setcounter{equation}{0}

% #7 - Due Feb. 29 Read Sections 5.1-2. Chapter 4: #9, 11, 15, 18 (ignore
% the second part concerning convergence in measure);
\begin{problem}[Wheeden \& Zygmund Ch.\@ 4, Ex.\@ 9]
  \hfill
  \begin{enumerate}[label=(\alph*),noitemsep]
  \item Show that the limit of a decreasing (increasing) sequence of
    functions u.s.c.\@ (l.s.c.) at $x_0$ is u.s.c.\@ (l.s.c.) at $x_0$. In
    particular, the limit of a decreasing (increasing) sequence of
    functions continuous at $x_0$ is u.s.c.\@ (l.s.c.) at $x_0$.
  \item Let $f$ be u.s.c.\@ and less than $\infty$ on $[a,b]$. Show that there
    exists continuous $f_k$ on $[a,b]$ such that $f_k\downarrow f$.
  \end{enumerate}
\end{problem}
\begin{solution}
  For part (a) we may as well assume that $f\geq 0$ for all $x$. Let
  $\{f_n\}$, $n\in\bbN$, be a sequence of decreasing functions with limit
  $f$ which are u.s.c.\@ at $x_0$. Then, for every $n\in\bbN$, for every
  sequence $x\to x_0$,
  \[
    \limsup_{x\to x_0}f_n(x)\leq f_n(x_0).
  \]
  Now, we claim that $f(x)\leq f_n(x)$ for every $x$ and every $n\in\bbN$.
  \begin{subproof}[Proof of claim]
    Suppose $f(x)>f_{N_1}(x)$ for some $x$, $N_1\in\bbN$. Then there exists
    a real number $\varepsilon>0$ such that $0<\varepsilon<|f(x)-f_n(x)|$
    (we may, for example, take $\varepsilon$ to be in $\bbQ$ which is dense
    in $\bbR$). Then, since $f_n\downarrow f$, there exists an index
    $N_1\in\bbN$ such that
    \[
      |f(x)-f_n(x)|<\varepsilon.
    \]
    However, since the sequence $f_n$ decreases to $f$, for
    $n\geq\max\{N_1,N_2\}$, $f_n(x)\leq f_{N_1}(x)$ so
    \[
      |f(x)-f_n(x)|>|f(x)-f_{N_1}(x)|>\varepsilon.
    \]
    This is a contradiction.
  \end{subproof}
  Having established this, for every sequence $x\to x_0$, we have
  \[
    \limsup_{x\to x_0} f(x)\leq \limsup_{x\to x_0} f_n(x)\leq f_n(x_0).
  \]
  Letting $n\to\infty$,
  \[
    \limsup_{x\to x_0} f(x)\leq \lim_{n\to\infty}f_n(x_0)=f(x_0).
  \]

  For part (b) suppose $f\colon [a,b]\to\bbR$ is u.s.c.\@ on $[a,b]$ and
  $f(x)<\infty$ for all $x\in [a,b]$. For a fixed $x\in[a,b]$, $f$ is
  u.s.c.\@ at $x$ if for every $\varepsilon>0$, there exists a neighborhood
  $B(x,\delta)$ such that $f(y)<f(x)+\varepsilon$. Now, let
  $\varepsilon=1/n$. Then, for each $x\in [a,b]$, there exists a
  neighborhood $B(x,\delta_x)$ such that $f(y)<f(x)+\varepsilon$ for
  $y\in B(x,\delta_x)$.

  The following
  \href{http://math.stackexchange.com/questions/462534/recognizing-uppersemicontinuous-function-as-a-pointwise-decreasing-limit}{post}
  on the Mathematics \textsf{StackExchange} contains a solution to part (b)
  of this problem.

  First, we claim that $f(x)\neq\infty$ for any $x\in[a,b]$, it must be
  bounded.
  \begin{subproof}[Proof of claim]
    By Theorem 4.14 (a), sets of the form $\{\,x\in[a,b]:f(x)<a\,\}$ is
    relatively open for all finite $a$. Define
    \[
      E_n=\left\{\,x\in [a,b]:f(x)<n\,\right\}.
    \]
    Then, the collection $\calE=\{E_n\}$, $n\in\bbN$, is an open cover of
    $[a,b]$. Since $[a,b]$ is compact, there exists a finite subcover
    $\{E_{n_1},\ldots,E_{n_m}\}$ of $\calE$. Letting
    $M=\max\{n_1,\ldots,n_m\}$, we have $f<M$ for all $x\in[a,b]$. Thus,
    $f$ is bounded on $[a,b]$.
  \end{subproof}\noindent
  Now that we have established that $f$ is bounded on $[a,b]$ by, say, $M$
  then $\sup_{x\in[a,b]} f\leq M$. Define
  \[
    f_n(x)=\sup_{y\in[a,b]}\bigl[f(y)-n|x-y|\bigr].
  \]
  We claim that this family of functions $\{f_n\}$, $n\in\bbN$, is
  continuous and that $f_n\to f$. To see that $f$ is continuous, we observe
  that this family of functions is in fact Lipschitz continuous
  \begin{align*}
    |f_n(x)-f_n(y)|
    &=\left|\sup_{z\in[a,b]}\bigl[ f(z)-n|x-z|
      \bigr]-\sup_{z\in[a,b]}\bigl[ f(z)-n|y-z| \bigr]\right|\\
    &\leq\left|\sup_{z\in[a,b]}\bigl[ f(z)-n|x-z|-f(z)-n|y-z| \bigr]\right|\\
    &=\left|\sup_{z\in[a,b]}\bigl[-n|x-z|-n|y-z|\bigr]\right|\\
    &=\left|\sup_{z\in[a,b]}\bigl[-n|x-y+(y-z)|-n|y-z|\bigr]\right|\\
    &\leq\left|\sup_{z\in[a,b]}\bigl[-n|x-y|-2n|y-z|\bigr]\right|\\
    &=n|x-y|.
  \end{align*}
  Thus, $f_n$ is Lipschitz and in particular, it is continuous.

  To see that $f_n\to f$ pointwise, let $\varepsilon>0$ be given then we
  must show that there exists some index $N$ such that $n\geq N$ implies
  \[
    |f(x)-f_n(x)|<\varepsilon.
  \]
  Expanding the equation above, we see that
  \[
    |f(x)-f_n(x)|=
    \left|
      f(x)-\sup_{y\in[a,b]}\bigl[f(y)-n|x-y|\bigr].
    \right|
  \]
\end{solution}

\begin{problem}[Wheeden \& Zygmund Ch.\@ 4, Ex.\@ 11]
  Let $f$ be defined on $\bbR^n$ and let $B(x)$ denote the open ball
  $\left\{\,y:|x-y|<r\,\right\}$ with center $x$ and fixed radius $r$. Show
  that the function $g(x)=\sup\left\{\,f(y):y\in B(x)\,\right\}$ is
  l.s.c.\@ and the function $h(x)=\inf\left\{\,f(y):y\in B(x)\,\right\}$ is
  u.s.c.\@ on $\bbR^n$. Is the same true for the closed ball
  $\left\{\,y:|x-y|\leq r\,\right\}$?
\end{problem}
\begin{solution}
  Note that, by properties of the infimum/supremum for any set of real
  numbers $S\subset\bbR$,
  \[
    \sup S=-\inf (-S)
  \]
  where $-S=\left\{\,-s:s\in S\,\right\}$. Thus,
  \begin{align*}
    g(x)
    &=-\inf\left\{\,-f(y):y\in B(x,r)\,\right\}\\
    &=\sup\left\{\,f(y):y\in B(x,r)\, \right\}.
  \end{align*}
  Letting $f'=-f$, it suffices to show that $g'(x)=\inf\{\,f'(y):y\in
  B(x,r)\,\}$ is u.s.c.\@ since for any u.s.c.\@ function $f$, $-f$ is
  l.s.c. Therefore, we show that $h$ is u.s.c.

  To see that $h$ is u.s.c., let $M>h(x_0)$. Then we must show that there
  exists a neighborhood $B(x_0,\delta)$ such that $M>h(x)$ for every
  $x\in B(x_0,\delta)$. Since $h(x_0)$ is the infimum of $f(x)$ over all
  $x\in B(x_0,r)$, given $\varepsilon>0$ there exists $x\in B(x_0,r)$ such
  that $f(x)<h(x_0)+\varepsilon<M$. Define $\delta=(r-|x-y|)/2$. Then we
  claim that for any $x\in B(x_0,\delta)$,
  \[
    g(x)<M.
  \]
  \begin{subproof}[Proof of claim]
    Let $x\in B(x_0,\delta)$. Then $y\in B(x_0,\delta)$ since
    \begin{align*}
      |x-y|&=|x-x_0-(y-x_0)|\\
           &\leq |x-x_0|+|y-x_0|\\
           &=(r-|y-x_0|)/2+|y-x_0|\\
           &=r/2+|y-x_0|/2\\
           &<r.
    \end{align*}
    Thus,
    \[
      g(x)\leq f(y)<g(x_0)+\varepsilon<M.
    \]
  \end{subproof}
  It follows that $g$ is u.s.c.
\end{solution}

\begin{problem}[Wheeden \& Zygmund Ch.\@ 4, Ex.\@ 15]
  Let $\left\{f_k\right\}$ be a sequence of measurable functions defined on
  a measurable set $E$ with $m(E)<\infty$. If $|f_k(x)|\leq M_x<\infty$ for
  all $k$ for each $x\in E$, show that given $\varepsilon>0$, there is
  closed $F\subseteq E$ and finite $M$ such that
  $m(E\setminus F)<\varepsilon$ and $|f_k(x)|\leq M$ for all $x\in F$.
\end{problem}
\begin{solution}
  Set $f=\sup_{n\in\bbN} |f_n|$; then, $f$ is measurable since it is the
  supremum of measurable functions $|f_n|$. By Lusin's theorem $f$
  satisfies the $\calC$-property, i.e., there exists a closed subset $F'$
  of $E$ with $m(E\setminus F')<\varepsilon/2$ and a continuous function
  $\bar f\colon E\to\bbR$ such that $f\restrict{F'}=\bar
  f\restrict{F'}$. Now, let $B$ be the closed ball centered at $\mathbf{0}$
  such that $|E\setminus B|<\varepsilon/2$ (remember, this is all taking
  place in $\bbR^n$, so we can do this). Thus, $F'\cap B$ is compact since
  it is a closed subset of $B$ the latter being a compact set. Let
  $F=F'\cap B$ then,
  \begin{align*}
    |E\setminus F|
    &=|E\setminus(F'\cap B)|\\
    &=|(E\setminus F')\cup(E\setminus B)|\\
    &\leq|E\setminus F'|+|E\setminus B|\\
    &<\frac{\varepsilon}{2}+\frac{\varepsilon}{2}\\
    &=\varepsilon
  \end{align*}
  so $F$ has the desired measure. Lastly, by the mean value theorem, $f$
  achieves its maximum, call it $M$, on $F$ since $F$ is compact. It
  follows that $f_n\restrict{F}\leq M$ for all $n\in\bbN$.
\end{solution}

\begin{problem}[Wheeden \& Zygmund Ch.\@ 4, Ex.\@ 18]
  If $f$ is measurable on $E$, define
  $\omega_f(a)=m\{\,f>a\,\}$ for $-\infty<a<\infty$. If
  $f_k\uparrow f$, show that $\omega_{f_k}\uparrow\omega_f$. If $f_k\to f$,
  show that $\omega_{f_k}\to\omega_f$ at each point of continuity of
  $\omega_f$. [For the second part, show that if $f_k\to f$, then
  $\limsup_{k\to\infty}\omega_{f_k}(a)\leq\omega_f(a-\varepsilon)$ and
  $\liminf_{k\to\infty}\omega_{f_k}(a)\geq\omega_f(a+\varepsilon)$ for
  every $\varepsilon>0$.]
\end{problem}
\begin{solution}
  For the initial part of this problem, suppose that $\{f_n\}$, $n\in\bbN$,
  is a sequence of measurable functions that increase to $f$ pointwise and
  consider the sequence of distribution functions $\{\omega_{f_n}\}$,
  $n\in\bbN$. We aim to show that
  $\lim_{n\to\infty}\omega_{f_n}=\omega_f$; call the latter limit,
  $\omega_f'$. First we will show that $\omega_f'$ exists.
  \begin{subproof}[Proof of claim]
    Since $f_n\uparrow f$, given $\varepsilon>0$ there exists an index $N$
    such that $m,n\geq N$ implies
    \[
      |f_n(x)-f_m(x)|<\varepsilon
    \]
    for a fixed $x\in E$. Then, assuming $n>m$, we have
    \begin{align*}
      |\omega_{f_n}(x)-\omega_{f_m}(x)|
      &=\left|
        m\left\{\,f_n>x\,\right\}
        -m\left\{\,f_m>x\,\right\}
        \right|\\
      &=
    \end{align*}
  \end{subproof}
\end{solution}

% (Chapter 5: # 1, 2, 3, 4).
\begin{problem}[Wheeden \& Zygmund Ch.\@ 5, Ex.\@ 1]
  If $f$ is a simple measurable function (not necessarily positive) taking
  values $a_j$ on $E_j$, $j=1,\ldots,N$, show that
  $\int_E f=\sum_{j=1}^N a_jm(E_j)$. [Use (5.24)].
\end{problem}
\begin{solution}
\end{solution}

\begin{problem}[Wheeden \& Zygmund Ch.\@ 5, Ex.\@ 3]
  Let $\left\{f_k\right\}$ be a sequence of nonnegative measurable
  functions defined on $E$. If $f_k\to f$ and $f_k\leq f$ a.e.\@ on $E$,
  show that $\int_E f_k\to\int_E f$.
\end{problem}
\begin{solution}
\end{solution}

%%% Local Variables:
%%% mode: latex
%%% TeX-master: "../MA544-Quals"
%%% End:
