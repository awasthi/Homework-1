\subsubsection{Homework 7}
\setcounter{exercise}{0}
\setcounter{equation}{0}

% #7 - Due Feb. 29 Read Sections 5.1-2. Chapter 4: #9, 11, 15, 18 (ignore
% the second part concerning convergence in measure);
\begin{problem}[Wheeden \& Zygmund Ch.\@ 4, Ex.\@ 9]
  \hfill
  \begin{enumerate}[label=(\alph*),noitemsep]
  \item Show that the limit of a decreasing (increasing) sequence of
    functions u.s.c.\@ (l.s.c.) at $x_0$ is u.s.c.\@ (l.s.c.) at $x_0$. In
    particular, the limit of a decreasing (increasing) sequence of
    functions continuous at $x_0$ is u.s.c.\@ (l.s.c.) at $x_0$.
  \item Let $f$ be u.s.c.\@ and less than $\infty$ on $[a,b]$. Show that there
    exists continuous $f_k$ on $[a,b]$ such that $f_k\downarrow f$.
  \end{enumerate}
\end{problem}
\begin{solution}
  For part (a) we may as well assume that $f\geq 0$ for all $x$. Let
  $\{f_n\}$, $n\in\bbN$, be a sequence of decreasing functions with limit
  $f$ which are u.s.c.\@ at $x_0$. Then, for every $n\in\bbN$, for every
  sequence $x\to x_0$,
  \[
    \limsup_{x\to x_0}f_n(x)\leq f_n(x_0).
  \]
  Now, we claim that $f(x)\leq f_n(x)$ for every $x$ and every $n\in\bbN$.
  \begin{subproof}[Proof of claim]
    Suppose $f(x)>f_{N_1}(x)$ for some $x$, $N_1\in\bbN$. Then there exists
    a real number $\varepsilon>0$ such that $0<\varepsilon<|f(x)-f_n(x)|$
    (we may, for example, take $\varepsilon$ to be in $\bbQ$ which is dense
    in $\bbR$). Then, since $f_n\downarrow f$, there exists an index
    $N_1\in\bbN$ such that
    \[
      |f(x)-f_n(x)|<\varepsilon.
    \]
    However, since the sequence $f_n$ decreases to $f$, for
    $n\geq\max\{N_1,N_2\}$, $f_n(x)\leq f_{N_1}(x)$ so
    \[
      |f(x)-f_n(x)|>|f(x)-f_{N_1}(x)|>\varepsilon.
    \]
    This is a contradiction.
  \end{subproof}
  Having established this, for every sequence $x\to x_0$, we have
  \[
    \limsup_{x\to x_0} f(x)\leq \limsup_{x\to x_0} f_n(x)\leq f_n(x_0).
  \]
  Letting $n\to\infty$,
  \[
    \limsup_{x\to x_0} f(x)\leq \lim_{n\to\infty}f_n(x_0)=f(x_0).
  \]

  For part (b), suppose $f\colon [a,b]\to\bbR$ is u.s.c.\@ on $[a,b]$ and
  $f(x)<\infty$ on $[a,b]$. For fixed $x_0\in[a,b]$, for every sequence
  $x\to x_0$ in $[a,b]$,
  \[
    \limsup_{x\to x_0} f(x)\leq f(x_0).
  \]
  Intuitively, an u.s.c.\@ function is one that is shaped like a stair
  case; once the function begins increasing, it cannot decrease and it
  cannot vary too much within a small neighborhood of any point. We can
  therefore take a sequence of piecewise continuous functions that increase
  linearly at these ``jumps'' and take the limit as the distance between
  the left endpoint of the line segment gets closer and closer to the
  point. Maybe this is all nonsense.
\end{solution}

\begin{problem}[Wheeden \& Zygmund Ch.\@ 4, Ex.\@ 11]
  Let $f$ be defined on $\bbR^n$ and let $B(x)$ denote the open ball
  $\left\{\,y:|x-y|<r\,\right\}$ with center $x$ and fixed radius $r$. Show
  that the function $g(x)=\sup\left\{\,f(y):y\in B(x)\,\right\}$ is
  l.s.c.\@ and the function $h(x)=\inf\left\{\,f(y):y\in B(x)\,\right\}$ is
  u.s.c.\@ on $\bbR^n$. Is the same true for the closed ball
  $\left\{\,y:|x-y|\leq r\,\right\}$?
\end{problem}
\begin{solution}
  For a fixed $x_0\in\bbR^n$ we must show that for every sequence of points
  $x\to x_0$,
  \[
    \liminf_{x\to x_0} g(x)\geq g(x_0).
  \]
  Since $g(x_0)$ is the supremum over all $x\in B(x_0,r)$ of $f(x)$, given
  $\varepsilon>0$ there exists
\end{solution}

\begin{problem}[Wheeden \& Zygmund Ch.\@ 4, Ex.\@ 15]
  Let $\left\{f_k\right\}$ be a sequence of measurable functions defined on
  a measurable set $E$ with $m(E)<\infty$. If $|f_k(M)|\leq M<\infty$ for
  all $k$ for each $x\in E$, show that given $\varepsilon>0$, there is
  closed $F\subseteq E$ and finite $M$ such that
  $m(E\setminus F)<\varepsilon$ and $|f_k(x)|\leq M$ for all $x\in F$.
\end{problem}
\begin{solution}
\end{solution}

\begin{problem}[Wheeden \& Zygmund Ch.\@ 4, Ex.\@ 18]
  If $f$ is measurable on $E$, define $\omega_f(a)=|\{\,f>a\,\}|$ for
  $-\infty<a<\infty$. If $f_k\uparrow f$, show that
  $\omega_{f_k}\uparrow\omega_f$. If $f_k\to f$, show that
  $\omega_{f_k}\to\omega_f$ at each point of continuity of $\omega_f$. [For
  the second part, show that if $f_k\to f$, then
  $\limsup_{k\to\infty}\omega_{f_k}(a)\leq\omega_f(a-\varepsilon)$ and
  $\liminf_{k\to\infty}\omega_{f_k}(a)\geq\omega_f(a+\varepsilon)$ for
  every $\varepsilon>0$.]
\end{problem}
\begin{solution}
\end{solution}

% (Chapter 5: # 1, 2, 3, 4).
\begin{problem}[Wheeden \& Zygmund Ch.\@ 5, Ex.\@ 1]
  If $f$ is a simple measurable function (not necessarily positive) taking
  values $a_j$ on $E_j$, $j=1,\ldots,N$, show that
  $\int_E f=\sum_{j=1}^N a_jm(E_j)$. [Use (5.24)].
\end{problem}
\begin{solution}
\end{solution}

\begin{problem}[Wheeden \& Zygmund Ch.\@ 5, Ex.\@ 3]
  Let $\left\{f_k\right\}$ be a sequence of nonnegative measurable
  functions defined on $E$. If $f_k\to f$ and $f_k\leq f$ a.e.\@ on $E$,
  show that $\int_E f_k\to\int_E f$.
\end{problem}
\begin{solution}
\end{solution}

%%% Local Variables:
%%% mode: latex
%%% TeX-master: "../MA544-Quals"
%%% End:
