\chapter{MA 544 Spring 2016}
\thispagestyle{empty}
This is material from the course MA 544 as taught in the spring of 2016.
\bigskip
\section{Homework}
These exercises were assigned from Wheeden and Zygmund's \emph{Measure and
  Integral}. Therefore, most of the theorems I reference will be from
\cite{wheeden-zygmund}. Other resources include \cite{folland} and
\cite{royden}. For more elementary results, I cite \cite{rudin-1}.
\subsection{Homework 1}
\begin{problem}[Wheeden \& Zygmund Ch.\@ 2, Ex.\@ 1]
Let $f(x)=x\sin(1/x)$ for $0<x\leq 1$ and $f(0)=0$. Show that $f$ is
bounded and continuous on $[0,1]$, but that $V[f;0,1]=+\infty$.
\end{problem}
\begin{proof}
It is clear that the function $f(x)=x\sin(1/x)$ is bounded on $[0,1]$ since
$|{\sin(1/x)}|\leq 1$ and $|x|\leq 1$ on $[0,1]$. Moreover, by properties
of continuous functions on $\bbR$, $x\sin(1/x)$ is continuous on
$(0,1)$.\footnote{You can, for example, check Ch.\@ }
\end{proof}

\begin{problem}[Wheeden \& Zygmund Ch.\@ 2, Ex.\@ 2]
Prove theorem (2.1).
\end{problem}
\begin{proof}
Recall the statement of theorem (2.1):
\begin{theorem*}[Wheeden \& Zygmund, 2.1]
\begin{enumerate}[label=(\alph*)]
\item If $f$ is of bounded variation on $[a,b]$, then $f$ is bounded on
  $[a,b]$.
\item Let $f$ and $g$ be of bounded variation on $[a,b]$. Then $cf$ (for
  any real constant $c$), $f+g$, and $fg$ are of bounded variation on
  $[a,b]$. Moreover, $f/g$ is of bounded variation on $[a,b]$ if there
  exists an $\varepsilon>0$ such that $|g(x)|\geq\varepsilon$ for
  $x\in[a,b]$.
\end{enumerate}
\end{theorem*}
\end{proof}

\begin{problem}[Wheeden \& Zygmund Ch.\@ 2, Ex.\@ 3]
If $[a',b']$ is a subinterval of $[a,b]$ show that $P[a',b']\leq P[a,b]$
and $N[a',b']\leq N[a,b]$.
\end{problem}
\begin{proof}
\end{proof}
\begin{problem}[Wheeden \& Zygmund Ch.\@ 2, Ex.\@ 11]
Show that $\int_a^bf\diff\phi$ exists if and only if given $\varepsilon>0$
there exists $\delta>0$ such that
$\left|R_\Gamma-R_{\Gamma'}\right|<\varepsilon$ if
$|\Gamma|,|\Gamma'|<\delta$.
\end{problem}
\begin{proof}
\end{proof}

\begin{problem}[Wheeden \& Zygmund Ch.\@ 2, Ex.\@ 13]
Prove theorem (2.16).
\end{problem}
\begin{proof}
\begin{theorem*}[Wheeden \& Zygmund, 2.16]
\begin{enumerate}[label=(\roman*)]
\item If $\int_a^b f\diff\phi$ exists, then so do $\int_a^bcf\diff\phi$ and
  $\int_a^b f\diff(c\phi)$ for any constant $c$, and
\[
\int_a^bcf\diff\phi=\int_a^bf\diff(c\phi)=c\int_a^bf\diff\phi.
\]
\item If $\int_a^b f_1\diff\phi$ and $\int_a^bf_2\diff\phi$ both exist, so
  does $\int_a^b\left(f_1+f_2\right)\diff\phi$, and
\[
\int_a^b\left(f_1+f_2\right)\diff\phi=\int_a^bf_1\diff\phi+\int_a^bf_2\diff\phi.
\]
\item If $\int_a^bf\diff\phi_1$ and $\int_a^bf\diff\phi_2$ both exist, so
  does $\int_a^bf\diff\left(\phi_1+\phi_2\right)$, and
\[
\int_a^bf\diff\left(\phi_1+\phi_2\right)=\int_a^bf\diff\phi_1+\int_a^bf\diff\phi_2.
\]
\end{enumerate}
\end{theorem*}
\end{proof}

%%% Local Variables:
%%% mode: latex
%%% TeX-master: "../MA544-Quals"
%%% End:
