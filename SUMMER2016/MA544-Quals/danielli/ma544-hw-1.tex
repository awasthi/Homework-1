\chapter{MA 544 Spring 2016}
\thispagestyle{empty}
This is material from the course MA 544 as taught in the spring of 2016.
\bigskip
\section{Homework}
These exercises were assigned from Wheeden and Zygmund's \emph{Measure and
  Integral}. Therefore, most of the theorems I reference will be from
\cite{wheeden-zygmund}. Other resources include \cite{folland} and
\cite{royden}. For more elementary results, I cite \cite{rudin-1}.
\subsection{Homework 1}
\begin{problem}[Wheeden \& Zygmund Ch.\@ 2, Ex.\@ 1]
Let $f(x)=x\sin(1/x)$ for $0<x\leq 1$ and $f(0)=0$. Show that $f$ is
bounded and continuous on $[0,1]$, but that $V[f;0,1]=+\infty$.
\end{problem}
\begin{proof}
  It is clear that the function $f(x)=x\sin(1/x)$ is
  \href{https://en.wikipedia.org/wiki/Bounded_function}{\emph{bounded}} on
  $[0,1]$ since $|{\sin(1/x)}|\leq 1$ and $|x|\leq 1$ on $[0,1]$. Moreover,
  by properties of
  \href{https://en.wikipedia.org/wiki/Continuous_function#Weierstrass_definition_.28epsilon.E2.80.93delta.29_of_continuous_functions}{\emph{continuous
      functions}} on $\bbR$, it is obvious that $f$ is continuous on
  $(0,1)$.\footnote{You can, for example, take a look at Theorem 4.9 from
    \cite[Ch.\@ 4, p.\@ 87]{rudin-1}.} What is not obvious is continuity at
  $0$. To show that $f$ is continuous at $0$, by Theorem 4.6 from
  \cite[Ch.\@ 4, p.\@ 86]{rudin-1}, it suffices to show that
  $\lim_{x\to 0}f(x)=0$. Consider the sequence $\{1/k\}$. This sequence
  converges to $0$. Moreover, given $\varepsilon>0$, by the Archimedean
  principle, for sufficiently large $K$, the inequality $1/K<\varepsilon$
  holds so for every $k\geq K$ we have
\begin{equation}
\label{eq:1:1}
\left|(1/k)\sin(k)-0\right|\leq
\left|1/k\right|<\varepsilon.
\end{equation}
Thus, $\lim_{k\to\infty}f(1/k)=0$. Thus, $f$ is continuous on all of
$[0,1]$.

Nevertheless, $f$ is not of
\href{https://en.wikipedia.org/wiki/Bounded_variation}{\emph{bounded
    variation}} on $[0,1]$. By Corollary 2.10 from \cite[Ch.\@ 2, p.\@
23]{wheeden-zygmund}, the
\href{https://en.wikipedia.org/wiki/Total_variation}{\emph{total
    variation}} $V$ of $f$ on $[0,1]$ is given by
\begin{equation}
\label{eq:sp16:1:2}
\begin{aligned}
V&=\int_0^1|f'|\diff x\\
&=\int_0^1|\sin(1/x)-(1/x)\cos(1/x)|\diff x\\
&=\int_1^\infty\frac{1}{u^2}|\sin u-u\cos u|\diff x\\
&\geq\int_M^\infty \frac{1}{2u}\diff u\\
&=\infty,
\end{aligned}
\end{equation}
where, for sufficiently large $M$, for $u\geq M$ we have $|{\sin u- u\cos
  u}|>u/2$. Thus, $f$ is not of bounded variation.
\end{proof}

\begin{problem}[Wheeden \& Zygmund Ch.\@ 2, Ex.\@ 2]
Prove theorem (2.1).
\end{problem}
\begin{proof}
Recall the statement of theorem (2.1):
\begin{quote}
\begin{enumerate}[label=(\alph*)]
\item If $f$ is of bounded variation on $[a,b]$, then $f$ is bounded on
  $[a,b]$.
\item Let $f$ and $g$ be of bounded variation on $[a,b]$. Then $cf$ (for
  any real constant $c$), $f+g$, and $fg$ are of bounded variation on
  $[a,b]$. Moreover, $f/g$ is of bounded variation on $[a,b]$ if there
  exists an $\varepsilon>0$ such that $|g(x)|\geq\varepsilon$ for
  $x\in[a,b]$.
\end{enumerate}
\end{quote}
\noindent
(a) Recall that $f$ is of b.v.\@ on $[a,b]$ if the
total variation $V$ of $f$ on $[a,b]$ is finite, where $V$ is defined to be
the supremum of the sum $\sum_{i=1}^m|f(x_i)-f(x_{i-1})|$ over all
\href{https://en.wikipedia.org/wiki/Partition_of_an_interval}{\emph{partitions}}
$\Gamma=\{x_0,\dotsc,x_m\}$ of $[a,b]$ of the sum.
\\\\
(b)
\end{proof}

\begin{problem}[Wheeden \& Zygmund Ch.\@ 2, Ex.\@ 3]
If $[a',b']$ is a subinterval of $[a,b]$ show that $P[a',b']\leq P[a,b]$
and $N[a',b']\leq N[a,b]$.
\end{problem}
\begin{proof}
\end{proof}
\begin{problem}[Wheeden \& Zygmund Ch.\@ 2, Ex.\@ 11]
Show that $\int_a^bf\diff\varphi$ exists if and only if given $\varepsilon>0$
there exists $\delta>0$ such that
$\left|R_\Gamma-R_{\Gamma'}\right|<\varepsilon$ if
$|\Gamma|,|\Gamma'|<\delta$.
\end{problem}
\begin{proof}
\end{proof}

\begin{problem}[Wheeden \& Zygmund Ch.\@ 2, Ex.\@ 13]
Prove theorem (2.16).
\end{problem}
\begin{proof}
Recall the statement of Theorem 2.16:
\begin{quote}
\begin{enumerate}[label=(\roman*)]
\item If $\int_a^b f\diff\varphi$ exists, then so do $\int_a^bcf\diff\varphi$ and
  $\int_a^b f\diff(c\varphi)$ for any constant $c$, and
\[
\int_a^bcf\diff\varphi=\int_a^bf\diff(c\varphi)=c\int_a^bf\diff\varphi.
\]
\item If $\int_a^b f_1\diff\varphi$ and $\int_a^bf_2\diff\varphi$ both exist, so
  does $\int_a^b\left(f_1+f_2\right)\diff\varphi$, and
\[
\int_a^b\left(f_1+f_2\right)\diff\varphi=\int_a^bf_1\diff\varphi+\int_a^bf_2\diff\varphi.
\]
\item If $\int_a^bf\diff\varphi_1$ and $\int_a^bf\diff\varphi_2$ both exist, so
  does $\int_a^bf\diff\left(\varphi_1+\varphi_2\right)$, and
\[
\int_a^bf\diff\left(\varphi_1+\varphi_2\right)=\int_a^bf\diff\varphi_1+\int_a^bf\diff\varphi_2.
\]
\end{enumerate}
\end{quote}
\end{proof}

%%% Local Variables:
%%% mode: latex
%%% TeX-master: "../MA544-Quals"
%%% End:
