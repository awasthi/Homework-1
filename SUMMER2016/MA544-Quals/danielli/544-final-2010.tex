\subsubsection{Final Exam 2010}
\setcounter{exercise}{0}
\setcounter{equation}{0}

\begin{problem}
  Suppose that \(f\in L^1(\bbR^n)\), and that \( x \) is a point in the
  Lebesgue set of \(f\). For \(r>0\), let
\[
A(r)= \frac{1}{r^n}\int_{B_r}|f( x - y )-f( x )|\diff y ,
\]
where \(B_r= B(\mathbf{0},r)\).
\\\\
Show that
\begin{enumerate}[label=(\alph*),noitemsep]
\item \(A(r)\) is a continuous function of \(r\), and \(A(r)\to 0\) as
  \(r\to 0\).
\item There exists a constant \(M>0\) such that \(A(r)\leq M\) for all
  \(r>0\).
\end{enumerate}
\end{problem}
\begin{solution}
(a)
\\\\
(b)
\end{solution}

\begin{problem}
  Let \(E\subseteq\bbR^n\) be a measurable set, \(1\leq p<\infty\). assume
  that \(\left\{ f_k \right\}\) is a sequence in \(L^p(E)\) converging
  pointwise a.e.\@ on \(E\) to a function \(f\in L^p(E)\). Prove that
\[
\|f_k-f\|_p\longrightarrow 0\iff
\|f_k\|_p\longrightarrow\|f\|_p
\]
\emph{Hint}: To prove one of the implications, you can use the following
fact without proving it:
\[
\left|
\frac{a-b}{2}
\right|
\leq
\frac{|a|^p+|b|^p}{2}
\]
for all \(a,b\in\bbR\).
\end{problem}
\begin{solution}
\end{solution}

\begin{problem}
  Let \(0<p<q<r\leq\infty\), \(E\subseteq\bbR^n\) be a measurable set. Show
  that each \(f\in L^q(E)\) is the sum of a function \(g\in L^p(E)\) and a
  function \(h\in L^r(E)\).
\end{problem}
\begin{solution}
\end{solution}

\begin{problem}
  Prove that \(f\colon[a,b]\to\bbR\) is Lipschitz continuous if and only if
  \(f\) is absolutely continuous and there exists a constant \(M>0\) such
  that \(|f'|<M\) a.e.\@ on \([a,b]\).
\end{problem}
\begin{solution}
\end{solution}

\begin{problem}
  Let \(1<p<\infty\), \(f\in L^p(\bbR^n)\), \(g\in L^{p'}(\bbR^n)\).
\begin{enumerate}[label=(\alph*),noitemsep]
\item Prove that \(f*g\in C(\bbR^n)\).
\item Does this conclusion continue to be valid when \(p=1\) or
  \(p=\infty\)?.
\end{enumerate}
\end{problem}
\begin{solution}
\end{solution}

%%% Local Variables:
%%% mode: latex
%%% TeX-master: "../MA544-Quals"
%%% End:
