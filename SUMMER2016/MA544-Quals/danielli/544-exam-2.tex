\subsubsection{Exam 2}
\setcounter{exercise}{0}
\setcounter{equation}{0}

\begin{problem}
Assume that $f\in L(\bbR^n)$. Show that for every $\varepsilon>0$ there
exists a ball $B$, centered at the origin, such that
\[
\int_{\bbR^n\setminus B}|f|<\varepsilon.
\]
\end{problem}
\begin{solution}
\end{solution}

\begin{problem}
Let $f\in L(E)$, and let $\{E_j\}$ be a countable collection of pairwise
disjoint measurable subsets of $E$, such that $E=\bigcup_{j=1}^\infty
E_j$. Prove that
\[
\int_E f=\sum_{j=1}^\infty\int_{E_j}f.
\]
\end{problem}
\begin{solution}
\end{solution}

\begin{problem}
Let $\{f_k\}$ be a family in $L(E)$ satisfying the following property:
For any $\varepsilon>0$ there exits $\delta>0$ such that $|A|<\delta$
implies
\[
\int_A |f_k|<\varepsilon
\]
for all $k\in\bfN$. Assume $|E|<\infty$, and $f_k(x)\to f(x)$ as
$k\to\infty$ for a.e.\@ $x\in E$. Show that
\[
\lim_{k\to\infty}\int_E f_k=\int_E f.
\]
(\emph{Hint:} Use Egorov's theorem.)
\end{problem}
\begin{solution}
\end{solution}

\begin{problem}
Let $I=[0,1]$, $f\in L(I)$, and define $g(x)=\int_x^1
t^{-1}f(t) \diff t$ for $x\in I$. Prove that $g\in L(I)$ and
\[
\int_I g=\int_I f.
\]
\end{problem}
\begin{solution}
\end{solution}

%%% Local Variables:
%%% mode: latex
%%% TeX-master: "../MA544-Quals"
%%% End:
