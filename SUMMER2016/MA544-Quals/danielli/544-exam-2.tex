\subsubsection{Exam 2}
\setcounter{exercise}{0}
\setcounter{equation}{0}

\begin{problem}
  Assume that \(f\in L(\bbR^n)\). Show that for every \(\varepsilon>0\)
  there exists a ball \(B\), centered at the origin, such that
  \[
    \int\limsclap{\bbR^n\setminus B}|f|<\varepsilon.
  \]
\end{problem}
\begin{solution}
  This was done in the previous section: define a sequence of functions
  \(\{f_n:n\in\bbN\}\) by
  \[
    f_n(x)=f(x)\indicate_{B(\mathbf{0},n)}.
  \]
  Then \(f_n\uparrow f\) so by the monotone convergence theorem, given
  \(\varepsilon>0\) there exists an index \(N\in\bbN\) such that
  \[
    \int_{\bbR^n}f(x)\diff x-\int_{\bbR^n}f_n(x)\diff
    x=\int\limsclap{\bbR^n\setminus B(\mathbf{0},n)} f(x)\diff x<\varepsilon
  \]
  whenever \(n\geq N\). Set \(B=B(\mathbf{0},N+1)\).
\end{solution}

\begin{problem}
  Let \(f\in L^1(E)\), and let \(\{E_j\}\) be a countable collection of
  pairwise disjoint measurable subsets of \(E\), such that
  \(E=\bigcup_{j=1}^\infty E_j\). Prove that
  \[
    \int_E f=\sum_{j=1}^\infty\int_{E_j}f.
  \]
\end{problem}
\begin{solution}
  Since \(f\in L^1(E)\), both \(\int_E f^+\diff x\) and
  \(int_E f^-\diff x\) are finite. Because we can make this decomposition,
  it suffices to prove the result for \(f\geq 0\). Using the countable
  sequence of measurable sets \(E_n\) define a sequence of measurable
  functions \(\{f_n:n\in\bbN\}\) by
  \[
    f_n(x)=\sum_{k=1}^n f(x)\chi_{E_k}(x).
  \]
  Then \(f_n\uparrow f\) so by the monotone convergence theorem, we have
  \begin{align*}
    \lim_{n\to\infty}\int_E f_n(x)\diff x
    &=\lim_{n\to\infty}\left[\int_E\sum_{n=1}^\infty f_n(x)\diff x\right]\\
    &=\lim_{n\to\infty}\left[\sum_{k=1}^n\int_E f(x)\chi_{E_n}(x)\diff x\right]\\
    &=\lim_{n\to\infty}\left[\sum_{k=1}^n\int_{E_n} f(x)\diff x\right]\\
    &=\sum_{n=1}^\infty\int_{E_n}f(x)\diff x\\
    &=\int_E \lim_{n\to\infty}f_n(x)\diff x\\
    &=\int_E f(x)\diff x,
  \end{align*}
  as desired.
\end{solution}

\begin{problem}
  Let \(\{f_k\}\) be a family in \(L^1(E)\) satisfying the following
  property: For any \(\varepsilon>0\) there exits \(\delta>0\) such that
  \(m(A)<\delta\) implies
  \[
    \int_A |f_k|<\varepsilon
  \]
  for all \(k\in\bbN\). Assume \(m(E)<\infty\), and \(f_k(x)\to f(x)\) as
  \(k\to\infty\) for a.e.\@ \(x\in E\). Show that
  \[
    \lim_{k\to\infty}\int_E f_k=\int_E f.
  \]
  (\emph{Hint:} Use Egorov's theorem.)
\end{problem}
\begin{solution}
  Let \(\varepsilon>0\) be given. Then, there exists \(\delta>0\) such that
  \(m(A)<\delta\) implies
  \[
    \int_A|f_n|<\left(1+2^{-n}\right)\varepsilon.
  \]
  Then by Egorov's theorem, there exists a closed subset \(F\) of \(E\)
  such that \(m(E\setminus F)<\delta\) and \(f_n\to f\) uniformly on
  \(F\). Thus,
  \begin{align*}
    \lim_{n\to\infty}\int_E f_n\diff x
    &=\lim_{n\to\infty}
      \left[\int_F f_n\diff x
      +\int_{E\setminus F}f_n\right]\\
    &=\lim_{n\to\infty}\int_F f_n\diff x+
      \lim_{n\to\infty}\int_{E\setminus F}f_n\diff x\\
    &\leq\int_F f\diff x+\varepsilon.
  \end{align*}
  Now, note that by Fatou's lemma
  \[
    \int_F\liminf_{n\to\infty}f_n(x)\diff x
    \leq \liminf_{n\to\infty}\int_F f_n\diff x.
  \]
\end{solution}

\begin{problem}
  Let \(I=[0,1]\), \(f\in L^1(I)\), and define
  \(g(x)=\int_x^1 t^{-1}f(t) \diff t\) for \(x\in I\). Prove that
  \(g\in L^1(I)\) and
  \[
    \int_I g=\int_I f.
  \]
\end{problem}
\begin{solution}
\end{solution}

%%% Local Variables:
%%% mode: latex
%%% TeX-master: "../MA544-Quals"
%%% End:
