\subsection{Homework 5}
% #5 - Due Feb. 15 Read Sections 3.6, 4.1 Chapter 3: # 14, 16, 18, Chapter 4: # 1, 2.
\begin{problem}[Wheeden \& Zygmund Ch.\@ 3, Ex.\@ 14]
  Show that the conclusion of part (ii) of Exercise 13 (Problem) is false
  if $m^*(E)=\infty$.
\end{problem}
\begin{solution}
\end{solution}

\begin{problem}[Wheeden \& Zygmund Ch.\@ 3, Ex.\@ 16]
  Prove (3.34).
\end{problem}
\begin{solution}
\end{solution}

\begin{problem}[Wheeden \& Zygmund Ch.\@ 3, Ex.\@ 18]
  Prove that outer measure is \emph{translation invariant}; that is, if
  $E_{\textbf{h}}=\left\{\,\bfx+\mathbf{h}:\bfx\in E\,\right\}$ is
  the translate of $E$ by $\mathbf{h}$, $\mathbf{h}\in\bbR^n$, show that
  $m^*(E_{\mathbf{h}})=m^*(E)$. If $E$ is measurable, show that
  $E_{\mathbf{h}}$ is also measurable. [This fact was used in proving
  (3.37).]
\end{problem}
\begin{solution}
\end{solution}

\begin{problem}[Wheeden \& Zygmund Ch.\@ 4, Ex.\@ 1]
  Prove corollary (4.2) and theorem (4.8)
\end{problem}
\begin{solution}
\end{solution}

\begin{problem}[Wheeden \& Zygmund Ch.\@ 4, Ex.\@ 2]
  Let $f$ be a simple function, taking its distinct values on disjoint sets
  $E_1,\ldots,E_N$. Show that $f$ is measurable if and only if
  $E_1,\ldots,E_N$ are measurable.
\end{problem}
\begin{solution}
\end{solution}

%%% Local Variables:
%%% mode: latex
%%% TeX-master: "../MA544-Quals"
%%% End:
