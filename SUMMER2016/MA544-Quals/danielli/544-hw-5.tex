\subsubsection{Homework 5}
\setcounter{exercise}{0}
\setcounter{equation}{0}

% #5 - Due Feb. 15 Read Sections 3.6, 4.1 Chapter 3: # 14, 16, 18, Chapter 4: # 1, 2.
\begin{problem}[Wheeden \& Zygmund Ch.\@ 3, Ex.\@ 14]
  Show that the conclusion of part (ii) of Exercise 13 is false if
  $m^*(E)=\infty$.
\end{problem}
\begin{solution}
  Part (ii) of Exercise 13 is part (ii) of Problem 2 from the last section
  (Homework 4). In that problem we showed that if the outer measure of $E$
  is finite, then $E$ is measurable if and only if its outer and inner
  measure agree. Here we construct a counter example to this when the outer
  measure of $E$ is $\infty$; that is, we show that there exists a set $E$
  with $m^*(E)=\infty$ such that $m^*(E)\neq m_*(E)$. So, which set shall
  it be? Since we are unoriginal, we will pull an example from Wheeden and
  Zygmund itself.

  Let $V\subset[0,1]$ be Vitali's unmeasurable (Theorem 3.38) set as
  constructed in Wheeden and Zygmund and consider the union
  $E=V\cup(2,\infty)$. It is clear that the inner and outer measure of $E$
  are both $\infty$. However, $E$ itself must be unmeasurable for otherwise
  $E\cap [0,1]=V$ is measurable.
\end{solution}

\begin{problem}[Wheeden \& Zygmund Ch.\@ 3, Ex.\@ 16]
  Prove (3.34).
\end{problem}
\begin{solution}
  We must prove Equation 3.34; that is, if $P$ is a parallellepiped
  \[
    m(P)=\vol(P).
  \]

  This is an uninteresting and slightly technical problem. We gain nothing
  from retracing its proof.
\end{solution}

\begin{problem}[Wheeden \& Zygmund Ch.\@ 3, Ex.\@ 18]
  Prove that outer measure is \emph{translation invariant}; that is, if
  $E_h=\left\{\,x+h:x\in E\,\right\}$\marginremark{To use \textbf{boldface}
    or not to use \textbf{boldface} with points in $\bfR^n$, that is the
    question. Whether 'tis nobler...
    \\
    For now, let us just use math italic for points in a set and boldface
    math for vectors. Yes, I like this convention.}  is the translate of
  $E$ by $h$, $h\in\bfR^n$, show that $m^*(E_h)=m^*(E)$. If $E$ is
  measurable, show that $E_h$ is also measurable. [This fact was used in
  proving (3.37).]
\end{problem}
\begin{solution}
  Let $E\subset\bfR^n$ and $h\in\bfR^n$ and define the set $E_h$ to be the
  set $E_h=\left\{\,x+h:x\in E\,\right\}$. We will show that the outer
  measure of $E$ is preserved under such translations. But first, let us
  point out that $E_h$ is nothing more than the image of $E$ under the
  linear transformation $T\colon\bfR^n\to\bfR^n$ given by $x\mapsto
  x+h$. By Theorem 3.35, such a map preserves measurability of sets and for
  any measurable set $E'\subset\bfR^n$, $m(T(E'))=(\det T)m(E')=m(E')$
  (since $\det T=1$). Now, by Theorem 3.6, for every $\varepsilon>0$, there
  exist an open set $G\supset E$ such that $m^*(G)\leq
  m^*(E)+\varepsilon$. Consider the image of $G$ under $T$, $T(G)$ is an
  open set containing $E_h$ so $m^*(G)\geq m^*(E)$ and
  \[
    m^*(T(G))=m^*(G)<m^*(E)+\varepsilon.
  \]
  Letting $\varepsilon\to 0$, we achieve the inequality
  \[
    m^*(E_h)\leq m^*(E).
  \]

  To get the other inequality, take the map $T^{-1}\colon\bfR^n\to\bfR^n$
  which takes $x\mapsto x-h$; this sends $E_h$ to $E$ and the same argument
  shows that
  \[
    m^*(E)\leq m^*(E).
  \]
  Thus, we have $m^*(E)=m^*(E_h)$, as was to be shown.
\end{solution}

\begin{problem}[Wheeden \& Zygmund Ch.\@ 4, Ex.\@ 1]
  Prove corollary (4.2) and theorem (4.8)
\end{problem}
\begin{solution}
  The corollary and theorem in question are:
  \begin{quote}
    \emph{If $f$ is measurable, then $\left\{\,f>-\infty\,\right\}$,
      $\left\{\,f<+\infty\,\right\}$, $\left\{\,f=+\infty\,\right\}$,
      $\left\{\,a\leq f\leq b\,\right\}$, $\left\{\,f=a\,\right\}$, etc.,
      are all measurable. Moreover $f$ is measurable if and only if
      $\left\{\,a<f<+\infty\,\right\}$ is measurable for every finite $a$.}
  \end{quote}
  and
  \begin{quote}
    \emph{If $f$ is measurable and $\lambda$ is any real number, then
      $f+\lambda$ and $\lambda f$ are measurable.}
  \end{quote}

  Their proofs are quite simple. For the corollary: Suppose
  $f\colon E\to\bfR$ is a measurable function. By Theorem 4.1, $f$ is
  measurable if and only if for every finite $\alpha\in\bfR$, the sets
  \begin{align*}
    &\left\{\,x\in E: f(x)\geq\alpha\,\right\}\\
    &\left\{\,x\in E:f(x)<\alpha\,\right\}\\
    &\left\{\,x\in E:f(x)\leq\alpha\,\right\}
  \end{align*}
  are measurable. Since measurable sets form a $\sigma$-algebra on
  $\bfR^n$, we know that the countable union and intersection of measurable
  sets is measurable. Thus,
  \begin{align*}
    \left\{\,x\in E:f(x)>-\infty\,\right\}
    &=\bigcup_{\alpha\in\bfZ}\left\{\,x\in E:f(x)>\alpha\,\right\}\\
    \left\{\,x\in E:f(x)=\infty\,\right\}
    &=\bigcap_{n=1}^\infty\left\{\,x\in E:f(x)>n\,\right\}\\
    \left\{\,x\in E:f(x)<\infty\,\right\}
    &=\bigcup_{\alpha\in\bfZ}\left\{\,x\in E:f(x)<\alpha\,\right\}
  \end{align*}
  are easily seen to be measurable.

  Showing that $\{\,x\in E:f(x)=\alpha\,\}$ and
  $\left\{\,x\in E:\alpha<f(x)<\beta\,\right\}$ are measurable requires
  some clever (but not too clever) intersection/union of the sets we get
  from Theorem 4.1.

  % Viewing $\lambda$ as the constant mapping $g\colon E\to \bfR$ where
  % $x\mapsto\lambda$ we have that $\lambda$ is measurable since it is
  % continuous.

  For the theorem: Suppose $f$ is measurable and $\lambda$ is a
  constant. By Theorem 4.1, for any finite $\alpha\in\bfR$ we have
  \[
    \left\{\,x\in E:f(x)>\alpha-\lambda\,\right\}
  \]
  so
  \[
    \left\{\,x\in E:f(x)+\lambda>\alpha\,\right\}
  \]
  is measurable. Thus, $f+\lambda$ is measurable. Similarly, for
  $\lambda\neq 0$, taking the set
  \[
    \left\{\,x\in E:f(x)>\alpha/\lambda\,\right\}
    =
    \left\{\,x\in E:\lambda f(x)>\alpha\,\right\}
  \]
  shows that $\lambda f$ is measurable; otherwise, if $\lambda=0$,
  $\lambda f=0$ is constant and hence is continuous which in turn implies
  that it is measurable.
\end{solution}

\begin{problem}[Wheeden \& Zygmund Ch.\@ 4, Ex.\@ 2]
  Let $f$ be a simple function, taking its distinct values on disjoint sets
  $E_1,\ldots,E_N$. Show that $f$ is measurable if and only if
  $E_1,\ldots,E_N$ are measurable.
\end{problem}
\begin{solution}
  $\implies$ Suppose that $f$ is measurable. Then, by Corollary 4.2, the
  sets of the form $\left\{\,f=\alpha_n\,\right\}=E_n$ are measurable. So
  the sets $E_n$ are measurable.

  $\impliedby$ On the other hand, suppose that the sets $E_n$ are
  measurable. Then, $\chi_{E_n}$ is measurable so by Theorem 4.8, $f$ is
  measurable since it is the sum
  \[
    f=\sum_{n=1}^N \alpha_{E_n}.\qedhere
  \]
\end{solution}

%%% Local Variables:
%%% mode: latex
%%% TeX-master: "../MA544-Quals"
%%% End:
