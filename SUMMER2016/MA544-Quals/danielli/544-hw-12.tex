\subsubsection{Homework 12}
\setcounter{exercise}{0}
\setcounter{equation}{0}

\begin{problem}[Wheeden \& Zygmund Ch.\@ 8, Ex.\@ 2]
  Prove the converse of Hölder's inequality for \(p=1\) and
  \(\infty\). Show also that for \(1\leq p\leq\infty\), a real-valued
  measurable \(f\) belongs to \(L^p(E)\) if \(fg\in L^1(E)\) for every
  \(g\in L^{p'}(E)\), \(1/p+1/p'=1\). The negation is also of interest: if
  \(f\in L^p(E)\) then there exists \(g\in L^{p'}(E)\) such that
  \(fg\notin L^1(E)\). (To verify the negation, construct \(g\) of the form
  \(\sum a_kg_k\) satisfying \(\int_E fg_k\to\infty\).)
\end{problem}
\begin{solution}
\end{solution}

\begin{problem}[Wheeden \& Zygmund Ch.\@ 8, Ex.\@ 3]
  Prove Theorems 8.12 and 8.13. Show that Minkowski’s inequality for series
  fails when \(p<1\).
\end{problem}
\begin{solution}
\end{solution}

\begin{problem}[Wheeden \& Zygmund Ch.\@ 8, Ex.\@ 4]
  Let \(f\) and \(g\) be real-valued and not identically \(0\) (i.e.,
  neither function equals \(0\) a.e.), and let \(1<p<\infty\). Prove that
  equality holds in the inequality \(|\int fg|\leq\|f\|_p\|g\|_{p'}\) if
  and only if \(fg\) has constant sign a.e.\@ and \(|f|^p\) is a multiple
  of \(|g|^{p'}\) a.e.
  \\\\
  If \(\|f+g\|_p=\|f\|_p+\|g\|_{p}\) and \(g\neq 0\) in Minkowski's
  inequality, show that \(f\) is a multiple of \(g\).
  \\\\
  Find analogues of these results for the spaces \(\ell^p\).
\end{problem}
\begin{solution}
\end{solution}

\begin{problem}[Wheeden \& Zygmund Ch.\@ 8, Ex.\@ 5]
  For \(0<p\leq\infty\) and \(0<|E|<\infty\), define
  \[
    N_p[f]=\left(\frac{1}{E}\int_E|f|^p\right)^{1/p},
  \]
  where \(N_\infty[f]\) means \(\|f\|_\infty\). Prove that if \(p_1<p_2\),
  then \(N_{p_1}[f]\leq N_{p_2}[f]\). Prove also that if
  \(1\leq p\leq \infty\), then \(N_p[f+g]\leq N_p[f]+N_p[g]\),
  \((1/|E|)\int_E|fg|\leq N_p[f]N_{p'}[g]\), \(1/p+1/p'=1\), and
  \(\lim_{p\to\infty} N_p[f]=\|f\|_\infty\). Thus, \(N_p\) behaves like
  \(\|\cdot\|_p\) but has the advantage of being monotone in \(p\). Recall
  Exercise 28 of Chapter 5.
\end{problem}
\begin{solution}
\end{solution}

\begin{problem}[Wheeden \& Zygmund Ch.\@ 8, Ex.\@ 6]
  \hfill
  \begin{enumerate}[label=(\alph*),noitemsep]
  \item Let \(1\leq p_i\), \(r\leq\infty\) and
    \(\sum_{i=1}^k1/p_i=1/r\). Prove the following generalization of
    Hölder's inequality:
    \[
      \|f_1\dotsm f_k\|_r\leq\|f_1\|_{p_1}\dotsm\|f_k\|_{p_k}.
    \]
  \item Let \(1\leq p<r<q\leq\infty\) and define \(\theta\in(0,1)\) by
    \(1/r=\theta/p+(1-\theta)/q\). Prove the interpolation estimate
    \[
      \|f\|_r\leq{\|f\|_p}^\theta{\|f\|_q}^{1-\theta}.
    \]
    In particular, if \(A=\max\left\{\|f\|_p,\|f\|_q\right\}\), then
    \(\|f\|_r\leq A\).
\end{enumerate}
\end{problem}
\begin{solution}
\end{solution}

\begin{problem}[Wheeden \& Zygmund Ch.\@ 8, Ex.\@ 9]
  If \(f\) is real-valued and measurable on \(E\), \(|E|>0\), define its
  essential infimum on \(E\) by
  \[
    \essinf f=\sup\left\{\,\alpha:|\{\,x\in
      E:f(x)<\alpha\,\}|=0\,\right\}.
  \]
  If \(f\geq 0\), show that \(\essinf_E f=(\esssup 1/f)^{-1}\).
\end{problem}
\begin{solution}
\end{solution}

\begin{problem}[Wheeden \& Zygmund Ch.\@ 8, Ex.\@ 11]
  If \(f_k\to f\) in \(L^p\), \(1\leq p<\infty\), \(g_k\to g\) pointwise,
  and \(\|g_k\|_\infty<M\) for all \(k\), prove that \(f_kg_k\to fg\) in
  \(L^p\).
\end{problem}
\begin{solution}
\end{solution}

%%% Local Variables:
%%% mode: latex
%%% TeX-master: "../MA544-Quals"
%%% End:
