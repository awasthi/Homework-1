\section{Bañuelos: Summer 2000}
\setcounter{exercise}{0}
\setcounter{equation}{0}
\begin{problem}
  Let $(X,\calF,\mu)$ be a measure space and suppose $\{f_n\}$ is a
  sequence of measurable functions with the property that for all
  $n\geq 1$
  \[
    \mu\left(\left\{\,x\in X:|f_n(x)|\geq\lambda\,\right\}\right)\leq C
    \exp(-\lambda^2/n)
  \]
  for all $\lambda>0$. (Here $C$ is a constant independent of $n$.) Let
  $n_k=2^k$. Prove that
  \[
    \limsup_{k\to\infty}\frac{|f_{n_k}|}{\sqrt{n_k\log(\log(n_k))}}\leq
    1\quad\text{a.e.}
  \]
\end{problem}
\begin{solution}
  Suppose ${\{f_n\}}_{n=1}^\infty$ is a sequence of measurable functions
  such that
  \begin{equation}
    \label{ban:aug00-1}%
    \mu\left(\left\{\,x\in X:|f_n(x)|\geq\lambda\,\right\}\right)\leq C
    \exp(-\lambda^2/n)
  \end{equation}
    for all $\lambda$. Now, consider the subsequence
  ${\{f_{2^k}\}}_{k=1}^\infty$ of ${\{f_n\}}_{n=1}^\infty$. We aim to show
  that
  \[
    \limsup_{k\to\infty}\frac{|f_{2^k}|}{\sqrt{2^k\log(\log(2^k))}}\leq 1
  \]
  almost everywhere. To that end, it suffices to show that the set
  \[
    E=%
    \left\{\,%
      x\in
      X:\limsup_{k\to\infty}\frac{|f_{2^k}|}{\sqrt{2^k\log(\log(2^k))}}>1%
      \,%
    \right\}
  \]
  has measure zero. Let $x\in E$ then
  \[
    \limsup_{k\to\infty} \frac{|f_{2^k}(x)|}{\sqrt{2^k\log(\log(2^k))}}>1.
  \]
  This means that there exists some subsequence
  $\{k_m\}_{m=1}^\infty\subset\{k\}_{n=1}^\infty$ such that
  \[
    \lim_{m\to\infty}\frac{|f_{2^{k_m}}(x)|}{\sqrt{2^{k_m}\log(\log(2^{k_m}))}}>1.
  \]
  This means that, for sufficiently large $N$
  \[
    |f_{2^{k_n}}(x)|>\sqrt{2^{k_n}\log(\log(2^{k_n}))}
  \]
  for all $n\geq N$. But by Equation \eqref{ban:aug00-1} we have
  \begin{equation}
    \label{eq:ban00-2}
    \begin{aligned}
      \mu\left(\left\{\,%
          x\in
          X:\frac{|f_{2^{k_n}}(x)|}{\sqrt{2^{k_n}\log(\log(2^{k_n}))}}\geq
          1 \,\right\}\right) &\leq%
      C\exp\left(-\left.
          \left(\sqrt{2^{k_n}\log(\log(2^{k_n}))}\right)^2\right/2^{k_n}%
      \right)\\
      &=C\exp\left(-\left.2^{k_n}\log(\log(2^{k_n}))\right/2^{k_n} \right)\\
      &=C\exp\left(-\log(\log(2^{k_n}))\right)\\
      &=C\exp\left(\log(1/\log(2^{k_n}))\right)\\
      &=\frac{C}{\log(2^{k_n})}.
    \end{aligned}
  \end{equation}
  Letting $n\to\infty$, we see that the measure of the set on the left-hand
  side of Equation \eqref{eq:ban00-2} must go to $0$ so $\mu(E)=0$.
\end{solution}

\begin{problem}
  Let $(X,\calF,\mu)$ be a finite measure space. Let $f_n$ be a sequence of
  measurable functions with $f_1\in L^1(\mu)$ and with the property that
  \[
    \mu\left(\left\{\,x\in X:|f_n(x)|>\lambda\,\right\}\right) \leq
    \mu\left(\left\{\,x\in X:|f_1(x)|>\lambda\,\right\}\right)
  \]
  for all $n$ and all $\lambda>0$. Prove that
  \[
    \lim_{n\to\infty}\frac{1}{n}\int_X\left[\max_{1\leq j\leq
        n}|f_j|\right]\diff\mu=0.
  \]
  [\emph{Hint}: You may use the fact that
  $\|f\|_1=\int_0^\infty\mu\left(\left\{\,|f(x)|>\lambda\,\right\}\right)\diff\lambda$.]
\end{problem}
\begin{solution}
  Without loss of generality we may assume that $\mu(X)=1$. Define
  $g,g_n\colon\calF\to[0,\infty]$ for $n\in\bbN$ by
  \begin{align*}
    g(\lambda)&=%
    \mu\left(\left\{\,x\in X:\max_{1\leq i\leq n}|f_i(x)|>\lambda\,\right\}\right),%
    &g_n(\lambda)%
    &=\mu\left(\left\{\,x\in X:|f_n(x)|>\lambda\,\right\}\right).
  \end{align*}
\end{solution}

\begin{problem}
  \hfill
  \begin{enumerate}[label=(\roman*)]
  \item Let $(X,\calF,\mu)$ be a finite measure space. Let $\{f_n\}$ be a
    sequence of measurable functions. Prove that $f_n\to f$ is measurable
    if and only if every subsequence $\{f_{n_k}\}$ contains a further
    subsequence $\{f_{n_{k_j}}\}$ that converges a.e.\@ to $f$.
  \item Let $(X,\calF,\mu)$ be a finite measure space. Let
    $F\colon\bbR\to\bbR$ be continuous and $f_n\to f$ in measure. Prove
    that $F(f_n)\to F(f)$ in measure. (You may assume, of course, that
    $f_n$, $F$, $F(f_n)$, and $F(f)$ are all measurable.)
  \end{enumerate}
\end{problem}
\begin{solution}
\end{solution}

\begin{problem}
  Let $(X,\calF,\mu)$ be a finite measure space and suppose $f\in L^1(\mu)$
  is nonnegative. Suppose $1<q<\infty$ and let $1\leq \leq\infty$ be its
  conjugate exponent, i.e., $1/p+1/q=1$. Suppose $f$ has the property that
  \[
    \int_Ef\diff\mu\leq \mu(E)^{1/q}
  \]
  for all measurable sets $E$. Prove that $f\in L^r(\mu)$ for any
  $1\leq r<p$. [\emph{Hint}: Consider
  $\left\{\,x\in X:2^n\leq f(x)<2^{n+1}\,\right\}$.]
\end{problem}
\begin{solution}
\end{solution}

\begin{problem}
  Let $f$ be a continuous function on $[-1,1]$. Find
  \[
    \lim_{n\to\infty}\int_{-1/n}^{1/n} f(x)(1-n|x|)\diff x.
  \]
\end{problem}
\begin{solution}
\end{solution}

\begin{problem}

\end{problem}

%%% Local Variables:
%%% mode: latex
%%% TeX-master: "../MA544-Quals"
%%% End:
