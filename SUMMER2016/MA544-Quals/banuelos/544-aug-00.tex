\section{Bañuelos: Summer 2000}
\setcounter{exercise}{0}
\setcounter{equation}{0}
\begin{problem}
  Let $(X,\calF,\mu)$ be a measure space and suppose $\{f_n\}$ is a
  sequence of measurable functions with the property that for all
  $n\geq 1$
  \[
    \mu\left(\left\{\,x\in X:|f_n(x)|\geq\lambda\,\right\}\right)\leq C
    \exp(-\lambda^2/n)
  \]
  for all $\lambda>0$. (Here $C$ is a constant independent of $n$.) Let
  $n_k=2^k$. Prove that
  \[
    \limsup_{k\to\infty}\frac{|f_{n_k}|}{\sqrt{n_k\log(\log(n_k))}}\leq
    1\quad\text{a.e.}
  \]
\end{problem}
\begin{solution}
  Suppose ${\{f_n\}}_{n=1}^\infty$ is a sequence of measurable functions
  such that
  \begin{equation}
    \label{ban:aug00-1}%
    \mu\left(\left\{\,x\in X:|f_n(x)|\geq\lambda\,\right\}\right)\leq C
    \exp(-\lambda^2/n)
  \end{equation}
    for all $\lambda$. Now, consider the subsequence
  ${\{f_{2^k}\}}_{k=1}^\infty$ of ${\{f_n\}}_{n=1}^\infty$. We aim to show
  that
  \[
    \limsup_{k\to\infty}\frac{|f_{2^k}|}{\sqrt{2^k\log(\log(2^k))}}\leq 1
  \]
  almost everywhere. To that end, it suffices to show that the set
  \[
    E=%
    \left\{\,%
      x\in
      X:\limsup_{k\to\infty}\frac{|f_{2^k}|}{\sqrt{2^k\log(\log(2^k))}}>1%
      \,%
    \right\}
  \]
  has measure zero. Let $x\in E$ then
  \[
    \limsup_{k\to\infty} \frac{|f_{2^k}(x)|}{\sqrt{2^k\log(\log(2^k))}}>1.
  \]
  This means that there exists some subsequence
  $\{k_m\}_{m=1}^\infty\subset\{k\}_{n=1}^\infty$ such that
  \[
    \lim_{m\to\infty}\frac{|f_{2^{k_m}}(x)|}{\sqrt{2^{k_m}\log(\log(2^{k_m}))}}>1.
  \]
  This means that, for sufficiently large $N$
  \[
    |f_{2^{k_n}}(x)|>\sqrt{2^{k_n}\log(\log(2^{k_n}))}
  \]
  for all $n\geq N$. But by Equation \eqref{ban:aug00-1} we have
  \begin{align*}
    \mu\left(\left\{\,%
    x\in X:\frac{|f_{2^{k_n}}(x)|}{\sqrt{2^{k_n}\log(\log(2^{k_n}))}}\geq 1
    \,\right\}\right)
    &\leq%
      C\exp\left(-\left.
      \left(\sqrt{2^{k_m}\log(\log(2^{k_m}))}\right)^2\right/2^{k_m}%
      \right)\\
    &=C\exp
  \end{align*}
\end{solution}

\begin{problem}
let $(X,\calF,\mu)$ be a finite measure space. Let $f_n$ be a sequence of
measurable functions with $f_1\in L^1(\mu)$ and with the property that
\[
\mu\left(\left\{\,x\in X:|f_n(x)|>\lambda\,\right\}\right)
\leq
\mu\left(\left\{\,x\in X:|f_1(x)|>\lambda\,\right\}\right)
\]
for all $n$ and all $\lambda>0$. Prove that
\[
\lim_{n\to\infty}\frac{1}{n}\int_X\left[\max_{1\leq j\leq n}|f_j|\right]\diff\mu=0.
\]
(\emph{Hint}: You may use the fact that
$\|f\|_1=\int_0^\infty\left\{\,|f(x)|>\lambda\,\right\}\diff\lambda$.)
\end{problem}
\begin{solution}
\end{solution}

\begin{problem}
\begin{enumerate}[label=(\roman*)]
\item Let $(X,\calF,\mu)$ be a finite measure space. Let $\{f_n\}$ be a sequence
of measurable functions. Prove that $f_n\to f$ is measurable if and only if
every subsequence $\{f_{n_k}\}$ contains a further subsequence
$\{f_{n_{k_j}}\}$ that converges a.e.\@ to $f$.
\item Let $(X,\calF,\mu)$ be a finite measure space. Let
$F\colon\bbR\to\bbR$ be continuous and $f_n\to f$ in measure. Prove that
$F(f_n)\to F(f)$ in measure. (You may assume, of course, that $f_n$, $F$,
$F(f_n)$, and $F(f)$ are all measurable.)
\end{enumerate}
\end{problem}
\begin{solution}
\end{solution}

\begin{problem}
Let $(X,\calF,\mu)$ be a finite measure space and suppose $f\in L^1(\mu)$
is nonnegative. Suppose $1<q<\infty$ and let $1\leq \leq\infty$ be its
conjugate exponent, i.e., $1/p+1/q=1$. Suppose $f$ has the property that
\[
\int_Ef\diff\mu\leq \mu(E)^{1/q}
\]
for all measurable sets $E$. Prove that $f\in L^r(\mu)$ for any $1\leq
r<p$.

(\emph{Hint}: Consider)
\end{problem}
\begin{solution}
\end{solution}

%%% Local Variables:
%%% mode: latex
%%% TeX-master: "../MA544-Quals"
%%% End:
