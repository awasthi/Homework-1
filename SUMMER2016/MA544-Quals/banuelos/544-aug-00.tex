\section{Bañuelos}
\subsection{Bañuelos: Summer 2000}
\setcounter{exercise}{0}
\setcounter{equation}{0}
\begin{problem}
  Let $(X,\calF,\mu)$ be a measure space and suppose $\{f_n\}$ is a
  sequence of measurable functions with the property that for all
  $n\geq 1$
  \[
    \mu\left(\left\{\,x\in X:|f_n(x)|\geq\lambda\,\right\}\right)\leq C
    \exp(-\lambda^2/n)
  \]
  for all $\lambda>0$. (Here $C$ is a constant independent of $n$.) Let
  $n_k=2^k$. Prove that
  \[
    \limsup_{k\to\infty}\frac{|f_{n_k}|}{\sqrt{n_k\log(\log(n_k))}}\leq
    1\quad\text{a.e.}
  \]
\end{problem}
\begin{solution}
  Suppose ${\{f_n\}}_{n=1}^\infty$ is a sequence of measurable functions
  such that
  \begin{equation}
    \label{ban:aug00-1}%
    \mu\left(\left\{\,x\in X:|f_n(x)|\geq\lambda\,\right\}\right)\leq C
    \exp(-\lambda^2/n)
  \end{equation}
    for all $\lambda$. Now, consider the subsequence
  ${\{f_{2^k}\}}_{k=1}^\infty$ of ${\{f_n\}}_{n=1}^\infty$. We aim to show
  that
  \[
    \limsup_{k\to\infty}\frac{|f_{2^k}|}{\sqrt{2^k\log(\log(2^k))}}\leq 1
  \]
  almost everywhere. To that end, it suffices to show that the set
  \[
    E=%
    \left\{\,%
      x\in
      X:\limsup_{k\to\infty}\frac{|f_{2^k}|}{\sqrt{2^k\log(\log(2^k))}}>1%
      \,%
    \right\}
  \]
  has measure zero. Let $x\in E$ then
  \[
    \limsup_{k\to\infty} \frac{|f_{2^k}(x)|}{\sqrt{2^k\log(\log(2^k))}}>1.
  \]
  This means that there exists some subsequence
  $\{k_m\}_{m=1}^\infty\subset\{k\}_{n=1}^\infty$ such that
  \[
    \lim_{m\to\infty}\frac{|f_{2^{k_m}}(x)|}{\sqrt{2^{k_m}\log(\log(2^{k_m}))}}>1.
  \]
  This means that, for sufficiently large $N$
  \[
    |f_{2^{k_n}}(x)|>\sqrt{2^{k_n}\log(\log(2^{k_n}))}
  \]
  for all $n\geq N$. But by Equation \eqref{ban:aug00-1} we have
  \begin{equation}
    \label{eq:ban00-2}
    \begin{aligned}
      \mu\left(\left\{\,%
          x\in
          X:\frac{|f_{2^{k_n}}(x)|}{\sqrt{2^{k_n}\log(\log(2^{k_n}))}}\geq
          1 \,\right\}\right) &\leq%
      C\exp\left(-\left.
          \left(\sqrt{2^{k_n}\log(\log(2^{k_n}))}\right)^2\right/2^{k_n}%
      \right)\\
      &=C\exp\left(-\left.2^{k_n}\log(\log(2^{k_n}))\right/2^{k_n} \right)\\
      &=C\exp\left(-\log(\log(2^{k_n}))\right)\\
      &=C\exp\left(\log(1/\log(2^{k_n}))\right)\\
      &=\frac{C}{\log(2^{k_n})}.
    \end{aligned}
  \end{equation}
  Letting $n\to\infty$, we see that the measure of the set on the left-hand
  side of Equation \eqref{eq:ban00-2} must go to $0$ so $\mu(E)=0$.
\end{solution}

\begin{problem}
  Let $(X,\calF,\mu)$ be a finite measure space. Let $f_n$ be a sequence of
  measurable functions with $f_1\in L^1(\mu)$ and with the property that
  \[
    \mu\left(\left\{\,x\in X:|f_n(x)|>\lambda\,\right\}\right) \leq
    \mu\left(\left\{\,x\in X:|f_1(x)|>\lambda\,\right\}\right)
  \]
  for all $n$ and all $\lambda>0$. Prove that
  \[
    \lim_{n\to\infty}\frac{1}{n}\int_X\left[\max_{1\leq j\leq
        n}|f_j|\right]\diff\mu=0.
  \]

  [\emph{Hint}: You may use the fact that
  $\|f\|_1=\int_0^\infty\mu\left(\left\{\,|f(x)|>\lambda\,\right\}\right)\diff\lambda$.]
\end{problem}
\begin{solution}
  Define $g_n,h_n\colon\calF\to[0,\infty]$ for $n\in\bbN$ by
  \begin{align*}
    g_n(\lambda)%
    &=\mu\left(\left\{\,x\in X:|f_n(x)|>\lambda\,\right\}\right),%
    &h_n(\lambda)%
    &=\mu\left(\left\{\,x\in X:\max_{1\leq i\leq n}|f_i(x)|>\lambda\,\right\}\right).%
  \end{align*}
  Now, note that, by the monotonicity of $\mu$, we have
  \[
    h_n(\lambda)\leq \sum_{i=1}^n g_n(\lambda)\leq ng_1(\lambda).
  \]
  Thus,
  \[
    \frac{h_n(\lambda)}{n}\leq g_1(\lambda).
  \]
  Since $\|f_1\|_1=\int_0^\infty g_1(\lambda)\diff\lambda$, by Lebesgue's
  dominated convergence theorem, we have
  \begin{align*}
    \lim_{n\to\infty}\frac{1}{n}%
    \int_X\left[\max_{1\leq j\leq n}|f_j|\right]\diff\mu
    &=\lim_{n\to\infty}\int_X\frac{h_n(x)}{n}\diff\mu\\
    &=\int_X\lim_{n\to\infty}\frac{h_n(x)}{n}\diff\mu\\
    &\leq\int_X\lim_{n\to\infty}\frac{\mu(X)}{n}\\
    &=0
  \end{align*}
  as we wanted to show.
\end{solution}

\begin{problem}
  \hfill
  \begin{enumerate}[label=(\roman*)]
  \item Let $(X,\calF,\mu)$ be a finite measure space. Let $\{f_n\}$ be a
    sequence of measurable functions. Prove that $f_n\to f$ is measurable
    if and only if every subsequence $\{f_{n_k}\}$ contains a further
    subsequence $\{f_{n_{k_j}}\}$ that converges a.e.\@ to $f$.
  \item Let $(X,\calF,\mu)$ be a finite measure space. Let
    $F\colon\bbR\to\bbR$ be continuous and $f_n\to f$ in measure. Prove
    that $F(f_n)\to F(f)$ in measure. (You may assume, of course, that
    $f_n$, $F$, $F(f_n)$, and $F(f)$ are all measurable.)
  \end{enumerate}
\end{problem}
\begin{solution}
  Recall that a sequence of measurable functions $\{f_n\}$ converge in
  measure to a limit $f$ if for every $\varepsilon>0$ the limit
  \[
    \lim_{n\to\infty}
    \mu\left(\left\{\,x\in X:|f(x)-f_n(x)|\geq\varepsilon\,\right\}\right)=0.
  \]

  For part (i) $\implies$ suppose that $f_n\to f$ in measure. Then given
  $\varepsilon>0$ and $\delta>0$ there exists $N\in\bbN$ such that
  $n\geq N$ implies
  \[
    \mu\left(\left\{\,%
        x\in X:|f(x)-f_n(x)|\geq\varepsilon%
        \,\right\}\right)%
    <\delta.
  \]
  In particular, given $\varepsilon=k^{-1}$ and $\delta=2^{-k}$, consider
  the countable collection of measurable sets ${\{E_k\}}_{k=1}^\infty$
  given by
  \[
    E_k=\left\{\,x\in X:|f(x)-f_{n_k}(x)|\geq\frac{1}{k}\,\right\},
  \]
  where $n_k\geq N(k)$ (which depends on our choice of $k$) such that
  \[
    \mu(E_k)<\frac{1}{2^k}.
  \]
  Now, by the Borel--Cantelli lemma, since
  \[
    \sum_{k=1}^\infty \mu(E_k)<\sum_{k=1}^\infty 2^{-k}=1<\infty,
  \]
  for almost every $x\in X$, there exists $N_x\in\bbN$ such that $x\notin
  E_k$ for $k\geq N_x$. This means that for $k\geq N_x$, we have
  \[
    |f(x)-f_{n_k}(x)|<\frac{1}{k}.
  \]
  Let $\{f_{n_{k+1}}\}$ be the subsequence of $\{f_{n_k}\}$. Then
  \[
    \lim_{k\to\infty} f_{n_{k+1}}=f
  \]
  as desired.

  $\impliedby$ On the other hand, suppose that every subsequence
  $\{f_{n_k}\}$ of $\{f_n\}$ contains a subsequence $\{f_{n_{k_j}}\}$ that
  converges to $f$. Seeking a contradiction, suppose that given
  $\varepsilon>0$ there exists a subsequence $\{f_{n_k}\}$ of $\{f_n\}$
  such that
  \[
    M=\mu\left(\left\{\,x\in
        X:|f(x)-f_{n_k}(x)|\geq\varepsilon\,\right\}\right)>0.
  \]
  But by assumption there exists a subsequence $\{f_{n_{k_j}}\}$ of
  $\{f_{n_k}\}$ that converges almost everywhere to $f$. We claim that this
  implies that $f_{n_{k_j}}\to f$ in measure.
  \begin{quote}
  \begin{proof}[Proof of claim]
    This is adapted from a proof in Royden, Proposition 3, Ch.\@ 5.

    First note that $f$ is measurable since it is the pointwise limit
    almost everywhere of a sequence of measurable functions. Let
    $\varepsilon,\delta>0$ be given. \hilight{Here is where the assumption
      that $\mu(X)<\infty$ is essential!} By Egorov's theorem, there is a
    measurable subset $E\subset X$ with $\mu(X\setminus E)<\delta$ such
    that $f_n\to f$ uniformly on $E$. Thus, there is an index $N$ such that
    $n\geq N$ implies
    \[
      |f_n(x)-f(x)|<\varepsilon
    \]
    for all $x\in E$. Thus, for $n\geq N$,
    \[
      \left\{\,x\in X:|f(x)-f_n(x)|\geq\varepsilon\,\right\}\subset
      X\setminus E
    \]
    so
    \[
      \mu\left(%
        \left\{\,x\in X:|f(x)-f_n(x)|\geq\varepsilon\,\right\}%
      \right)%
      <\varepsilon.
    \]
    Thus, we have
    \[
      \lim_{n\to\infty}\mu\left(%
        \left\{\,x\in X:|f(x)-f_n(x)|\geq\varepsilon\,\right\}%
      \right)%
      =0,
    \]
    i.e., $f_n\to f$ in measure.
  \end{proof}
  \end{quote}
  Hence, since $f_{n_{k_j}}\to f$ in measure, but $M>0$ we have a
  contradiction.

  For (ii) since $F$ is continuous given $\varepsilon>0$ there exist
  $\delta>0$ such that $|x-x'|<\delta$ implies
  $|F(x)-F(x')|<\varepsilon$. By part (i), $f_n\to f$ in measure if and
  only if every subsequence $\{f_{n_k}\}$ of $\{f_n\}$ contains a
  subsequence $\{f_{n_{k_j}}\}$ that converges to $f$ almost everywhere,
  i.e., given $\delta>0$ there exists an index $N$ such that $n_{k_j}\geq
  N$ implies
  \[
    |f(x)-f_{n_{k_j}}(x)|<\delta
  \]
  for almost every $x\in X$. Thus,
  \[
    \left|F(f(x))-F(f_{n_{k_j}}(x))\right|<\varepsilon
  \]
  and we see that for every subsequence $\{F\circ f_{n_k}\}$ of $\{F\circ
  f_n\}$ we can find a subsequence $\{F\circ f_{n_{k_j}}\}$ that converges
  almost everywhere to $F\circ f$.
\end{solution}

\begin{problem}
  Let $(X,\calF,\mu)$ be a finite measure space and suppose $f\in L^1(\mu)$
  is nonnegative. Suppose $1<p<\infty$ and let $1<q<\infty$ be its
  conjugate exponent, i.e., $1/p+1/q=1$. Suppose $f$ has the property that
  \[
    \int_Ef\diff\mu\leq \mu(E)^{1/q}
  \]
  for all measurable sets $E$. Prove that $f\in L^r(\mu)$ for any
  $1\leq r<p$.

  [\emph{Hint}: Consider $\left\{\,x\in X:2^n\leq f(x)<2^{n+1}\,\right\}$,
  if you like.]
\end{problem}
\begin{solution}
  By previous problems, we know that if $\mu(X)<\infty$ and $f\in L^p(X)$,
  then $f\in L^r(X)$ for $1\leq r<p$, so it suffices to show that
  $\|f\|_p<\infty$.

  Instead of following the hint, consider the set
  \[
    E_t=\left\{\,x\in X:f(x)\geq t\,\right\}
  \]
  and let
  \[
    \omega(t)=\mu\left(E_t\right),
  \]
  i.e., the distribution function of $f$. Then, we have
  \[
    \int_0^\infty\omega(t)\diff t=\int_X f\diff\mu.
  \]
  In particular, if we make the substitution $\alpha=t^{1/p}$,
  $\rmd\alpha=t^{1/q}/p\diff t=\alpha^{p/q}/p\diff t$, we have
  \[
    \int_X f^r\diff\mu=%
    \int_0^\infty p\alpha^{-p/q}\omega(\alpha)\diff\alpha.
  \]
  Now, by Chebyshev's inequality, we have
  \[
    t\omega(t)\leq\int_{E_t} f\diff\mu\leq\omega(t)^{1/q}
  \]
  so
  \[
    \omega(t)\leq t^{-p}.
  \]
  Thus,
  \[
    \int_X f^r\diff\mu=%
    \int_0^\infty p\alpha^{-p/q}\omega(\alpha)\diff\alpha%
    \leq
    \int_0^\infty p\alpha^{-p-p/q}\diff\alpha.
  \]
  Since $p+p/q>1$, the integral above is finite. Thus, $f\in L^p(X)$ and we
  have $f\in L^r(X)$ for all $1\leq r<p$.
\end{solution}

\begin{problem}
  Let $f$ be a continuous function on $[-1,1]$. Find
  \[
    \lim_{n\to\infty}\int_{-1/n}^{1/n} f(x)(1-n|x|)\diff x.
  \]
\end{problem}
\begin{solution}
  To find the limit of the integral
  \[
    \int_{-1/n}^{1/n} f(x)(1-n|x|)\diff x
  \]
  we first make the following substitutions: Let $y=nx$, $\rmd y=n\diff
  x$. Then
  \[
    \int_{-1/n}^{1/n} f(x)(1-n|x|)\diff x%
    =\frac{1}{n}\int_{-1}^1 f(y/n)(1-|y|)\diff y.
  \]
  By the extreme value theorem, since $f$ is continuous and $[-1,1]$ is
  compact $f$ is bounded on $[-1,1]$ by, say $M$. Let $g(x)=M$. Then $g\in
  L^1(X)$ since $\|g\|_1=2M$. Thus, by the Lebesgue dominated convergence
  theorem, since
  \[
    \left|f(y/n)(1-|y|)\right|\leq M
  \]
  on $[-1,1]$ and $g\in L^1([-1,1])$ it follows that
  \begin{align*}
    \lim_{n\to\infty}\int_{-1/n}^{1/n} f(x)(1-n|x|)\diff x
    &=\lim_{n\to\infty}\frac{1}{n}\int_{-1}^1 f(y/n)(1-|y|)\diff y\\
    &=\int_{-1}^1\lim_{n\to\infty}\left[\frac{f(y/n)(1-|y|)}{n}\right]\rmd
      y\\
    &=\int_{-1}^1\lim_{n\to\infty}%
      \left[%
      \frac{f(y/n)}{n}-\frac{|y|}{n}
      \right]\rmd y\\%
    &=0.
  \end{align*}
\end{solution}

\begin{problem}
  Let $(X,\calF,\mu)$ be a measure space and suppose $f\in L^p(\mu)$,
  $1\leq p<\infty$. Suppose $E_n$ is a sequence of measurable sets
  satisfying $\mu(E_n)=1/n$ for all $n$. Prove that
  \[
    \lim_{n\to\infty}\left[n^{(p-1)/p}\int_{E_n}|f|\diff\mu\right]=0.
  \]
\end{problem}
\begin{solution}
  The result follows immediately by Hölder's inequality. Let
  $C=\|f\|_p$. Since $f\in L^p(X)$, then $f\in L^p(E_n)$ for all
  $n\in\bbN$. Thus, by Hölder's inequality
  \[
    \begin{aligned}
      \|f\|_{L^1(E_n)}&\leq \|f\|_{L^p(E_n)}\mu(E)^{1/q}\\
      &\leq C\mu(E)^{1/q}\\
      &=C\mu(E)^{p/(p-1)}\\
      &=Cn^{-p/(p-1)}\\
      &=Cn^{p/(1-p)}.
    \end{aligned}
  \]
  Hence, the integral is bounded above by
  \begin{align*}
    0\leq
    n^{(p-1)/p}\int_{E_n}|f|\diff\mu
    &\leq Cn^{(p-1)/p+p/(1-p)}\\
    &=Cn^{(2p-1)/(p(1-p))}.
  \end{align*}
  Since $p>1$, $1-p<0$ and $2p-1>0$ so the exponent
  $(2p-1)/(p(1-p))<0$. Thus, as $n\to\infty$
  \[
    Cn^{(2p-1)/(p(1-p))}\longrightarrow 0.
  \]
  It follows that
  \[
    \lim_{n\to\infty}\left[n^{(p-1)/p}\int_{E_n}|f|\diff\mu\right]=0.
  \]
\end{solution}

\begin{problem}
  Let $(X,\calM,\mu)$ be a measure space and let $\{g_n\}$ be a sequence of
  nonnegative measurable functions with the property that $g_n\in L^1(\mu)$
  for every $n$ and $g_n\to g$ in $L^1(\mu)$. Let $\{f_n\}$ be another
  sequence of nonnegative measurable functions on $(X,\calF,\mu)$.
  \begin{enumerate}[label=(\roman*),noitemsep]
  \item If $f_n\leq g_n$ almost everywhere for every $n$, prove that
    \[
      \limsup_{n\to\infty}\int_X
      f_n\diff\mu\leq\int_X\limsup_{n\to\infty}f_n\diff\mu.
    \]

    [\emph{Hint}: Start by considering a subsequence $\{f_{n_k}\}$ such
    that
    \[
      \lim_{n_k\to\infty}\int_X
      f_{n_k}\diff\mu=\limsup_{n\to\infty}\int_X f_n\diff\mu
    \]
    and let $\{g_{n_{k_j}}\}$ be a subsequence of $\{g_{n_k}\}$ such
    that $g_{n_{k_j}}\to g$ almost everywhere.]
  \item If $f_n\to f$ almost everywhere and if $f_n\leq g_n$ almost
    everywhere for all $n$, then $\|f_n-f\|_1\to 0$ as $n\to\infty$.
  \end{enumerate}
\end{problem}
\begin{solution}
  Part (i) is a generalization of what is colloquially known as the reverse
  Fatou's lemma. Consider the sequence of measurable functions $\{h_n\}$
  where $h_n=g_n-f_n$. Note that $h_n\geq 0$ for all $x\in X$ since
  $g_n\geq f$ for all $x\in X$. Then
  \[
    h_n\leq\sup_{1\leq k\leq n} h_k
  \]
  for all $x\in X$.
\end{solution}

\begin{problem}
  Let $f\in L^1(\bbR)$. Consider the function
  \[
    F(x)=\int_{\bbR} \exp(\rmi xt) f(t)\diff t.
  \]
  \begin{enumerate}[label=(\roman*),noitemsep]
  \item Show that $F\in L^\infty(\bbR)$ and that $F$ is continuous at every
    $x\in\bbR$. Moreover, if $|t|^kf(t)\in L^\infty(\bbR)$ for all $k\geq
    1$, show that $F$ is infinitely differentiable, i.e., $F\in
    C^\infty(\bbR)$.
  \item Suppose $f$ is continuous as well as in $L^1(\bbR)$. Show that
    $\lim_{|x|\to\infty} F(x)=0$.
  \end{enumerate}
  [\emph{Hint}: Using $\exp(-\rmi\pi)=-1$, write
  $F(x)=\left(\int_{\bbR}(\exp(\rmi xt)-\exp(\rmi xt-\rmi\pi))\right)/2$.]
\end{problem}
\begin{solution}
\end{solution}

%%% Local Variables:
%%% mode: latex
%%% TeX-master: "../MA544-Quals"
%%% End:
