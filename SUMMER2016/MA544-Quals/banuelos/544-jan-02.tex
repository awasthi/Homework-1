\subsection{Bañuelos: Summer 2000}
\setcounter{exercise}{0}
\setcounter{equation}{0}

\begin{problem}
  For any two subsets $A$ and $B$ of $\bbR$ define $A+B=\left\{\,a+b:a\in
    A,b\in B\,\right\}$.
  \begin{enumerate}[label=(\roman*),noitemsep]
  \item Suppose $A$ is closed and $B$ is compact. Prove that $A+B$ is
    closed.
  \item Give an example that shows that (i) may be false if we only assume
    that $A$ and $B$ are closed.
  \end{enumerate}
\end{problem}
\begin{solution}
  For part (i): Let $x\in\overline{A+B}$. Then there is a sequence
  $\{a_n+b_n\}$ in $A+B$, with $a\in A$, $b_n\in B$, that converges to
  $x$. Now, consider the sequences $\{a_n\}$, $\{b_n\}$ in $A$ and $B$,
  respectively. Since $A$ is closed $a_n\to a$ for some $a\in A$. Moreover,
  since $\bbR$ is a complete metric space and $B$ is compact, there is a
  subsequence $\{b_{n_k}\}$ of $\{b_n\}$ that converges to a point $b$ in
  $B$. Thus, taking the subsequence $\{a_{n_k}+b_{n_k}\}$ of $\{a_n+b_n\}$,
  $a_{n_k}+b_{n_k}\to a+b$ which is in $A+B$. But $a_n+b_n\to x$ so every
  subsequence of $\{a_n+b_n\}$ converges to $x$. By the uniqueness of the
  limit, we must have $x=a+b$.

  For part (ii): Consider the subsets $A=\bbZ$ and $B=\sqrt{2}\bbZ$ of
  $\bbR$. Both $A$ and $B$ are discrete subsets of $\bbR$, hence
  closed. However, we claim that $A+B$ is dense in $\bbR$; in particular,
  $\overline{A+B}\neq A+B$. To see this, choose a point $x\in\bbR$; we may
  as well assume that $x\in[0,1]$ since $0\leq x-n<1$ where
  $n=\lceil x \rceil$ if $x>0$ (or if $x<0$, let $n=\lfloor n\rfloor$),
  i.e., the largest integer smaller than $x$. Thus, it suffices to show
  that $0$ is a limit point of $A+B$.

  Suppose there exists $\varepsilon>0$ such that
  $(0,\varepsilon)\cap(A+B)=\emptyset$. Let
  $\alpha=\inf (A+B)\cap(0,\infty)$. We claim that $\alpha\in A+B$. Seeking
  a contradiction, suppose that $\alpha\notin (A+B)\cap(0,\infty)$. Then
  there exists elements $x,y\in (A+B)\cap(0,\infty)$ such that
  \[
    \alpha\leq x\leq y\leq\left(1+\frac{1}{4}\right)\alpha
  \]
  (by the infimum property). Then $x-y\in A+B$ (since $A+B$ is a ring),
  such that
  \[
    0\leq x-y\leq\tfrac{1}{4}\alpha<\alpha.
  \]
  This contradicts that $\alpha$ is the infimum of $A+B$. Thus, $\alpha\in
  A+B$.

  Now, let $x\in (A+B)\cap(0,\infty)$ and let $n=\lfloor x/\alpha\rfloor$
  be the largest integer smaller than $x/\alpha$. Then
  \[
    n\alpha\leq x<(n+1)\alpha
  \]
  so that $0\leq x-n\alpha<\alpha$. This tells us that
  $x-n\alpha\in[0,\alpha)$, but since $\alpha$ is the infimum of
  $(A+B)\cap(0,\infty)$, we must have $x-n\alpha=0$ so $x=n\alpha$. It
  follows that every element in $A+B$ is a multiple of $\alpha$.

  In particular, since $1,\sqrt{2}\in A+B$, there exists integers $n$ and
  $m$ such that
  \[
    \sqrt{2}=\frac{\sqrt{2}}{1}=\frac{m\alpha}{n\alpha}.
  \]
  Since $\sqrt{2}$ is irrational, this is a contradiction. Thus, it must be
  that $(0,\varepsilon)\cap (A+B)$ is nonempty for every $\varepsilon>0$
  and it follows that we can reasonably approximate any $x\in[0,1]$ and by
  extension, any $x\in\bbR$. Thus, $A+B$ is dense in $\bbR$ but it is not
  $\bbR$.
\end{solution}

\begin{problem}
  Suppose $f\colon[0,1]\to\bbR$ is differentiable at every $x\in[0,1]$
  where by differentiability at $0$ and $1$ we mean right and left
  differentiability, respectively. Prove that $f'$ is continuous if and
  only if $f$ is uniformly differentiable. That is, if and only if for all
  $\varepsilon>0$ there is an $h_0>0$ such that
  \[
    \left|\frac{f(x+h)-f(x)}{h}-f'(x)\right|<\varepsilon
  \]
  whenever $0\leq x$, $x+h\leq 1$, $0<|h|<h_0$.
\end{problem}
\begin{solution}
  $\impliedby$ If $f$ is uniformly differentiable then given
  $\varepsilon>0$, there exists $h_0$ such that $|x-y|<h_0$ implies
  \begin{align*}
    \left|
    f'(x)-\frac{f(x)-f(y)}{x-y}
    \right|
    &<\frac{\varepsilon}{2}\\
    \left|
    f'(y)-\frac{f(y)-f(x)}{x-y}
    \right|
    &<\frac{\varepsilon}{2}.
  \end{align*}
  Then, for $|x-y|<h_0$, we have
  \begin{align*}
    |f'(x)-f'(y)|
    &=
      \left|
      f'(x)-\frac{f(x)-f(y)}{x-y}-
      \left(
      f'(y)-
      \frac{f(y)-f(x)}{x-y}
      \right)
      \right|\\
    &\leq
      \left|
      f'(x)-\frac{f(x)-f(y)}{x-y}
      \right|
      +
      \left|
      f'(y)-
      \frac{f(y)-f(x)}{x-y}
      \right|\\
    &<\frac{\varepsilon}{2}+\frac{\varepsilon}{2}\\
    &=\varepsilon.
  \end{align*}
  Thus, $f'$ is continuous on $[0,1]$.

  $\implies$ Conversely, if $f'$ is continuous on $[0,1]$ then given
  $\varepsilon>0$ there exists $\delta>0$ such that $|y-x|<\delta$ implies
  \[
    |f'(y)-f'(x)|<\varepsilon.
  \]
  Now, consider the estimate
  \[
    0<\left|\frac{f(x+h)-f(x)}{h}-f'(x)\right|
  \]
  for $|h|<\delta$. By the mean value theorem, there exists
  $\xi\in(x-\delta,x+\delta)$ such that
  \[
    \frac{f(x+h)-f(x)}{h}=f'(\xi).
  \]
  Thus,
  \begin{align*}
    \left|\frac{f(x+h)-f(x)}{h}-f'(x)\right|
    &=
      \left|f'(\xi)-f'(x)\right|\\
    &<\varepsilon
  \end{align*}
  since $|\xi-x|<\delta$. Thus, $f$ is uniformly differentiable.
\end{solution}

\begin{problem}
  Let $(X,\calF,\mu)$ be a measure space with $\mu(X)=1$ and let
  $F_1,\ldots,F_{17}$ be seventeen measurable subsets of $X$ with
  $\mu(F_j)=1/4$ for every $j$.
  \begin{enumerate}[label=(\roman*),noitemsep]
  \item Prove that five of these subsets must have an intersection of
    positive measure. That is, if $E_1,\ldots,E_k$ denotes the collection
    of all nonempty intersections of the $F_j$ taken five at a time ($k\leq
    6188$), show that at least one of these sets must have positive
    measure.
  \item Is the conclusion in (i) true if we take sixteen sets instead of
    seventeen?
  \end{enumerate}
\end{problem}
\begin{solution}

\end{solution}

\begin{problem}
  Let $f_n\colon X\to[0,\infty)$ be a sequence of measurable functions on
  the measure space $(X,\calF,\mu)$. Suppose there is a positive constant
  $M$ such that the functions $g_n(x)=f(x)\chi_{\left\{\,f_n\leq
      M\,\right\}}(x)$ satisfy $\|g_n\|_1\leq A/n^{4/3}$ and for which
  $\mu\left(\left\{\,x\in X:f_n(x)>M\,\right\}\right)\leq B/n^{5/4}$, where
  $A$ and $B$ are positive constants independent of $n$. Prove that
  \[
    \sum_{n=1}^\infty f_n<\infty
  \]
  almost everywhere.
\end{problem}
\begin{solution}
\end{solution}

\begin{problem}
  Let $\{g_n\}$ be a bounded sequence of functions on $[0,1]$ which is
  uniformly Lipschitz. That isthere is a constant $M$ (independent of $n$)
  such that for all $n$, $|g_n(x)-g_n(y)|\leq M|x-y|$ for all $x,y\in
  [0,1]$ and $|g_n(x)|\leq M$ for all $x\in[0,1]$.
  \begin{enumerate}[label=(\roman*),noitemsep]
  \item Prove that for any $0\leq a\leq b\leq 1$,
    \[
      \lim_{n\to\infty}\int_a^b g_n(x)\sin(2n\pi x)\diff x=0.
    \]
  \item Prove that for any $f\in L^1[0,1]$,
    \[
      \lim_{n\to\infty}\int_0^1 f(x)g_n(x)\sin(2n\pi x)\diff x=0.
    \]
  \end{enumerate}
\end{problem}
\begin{solution}
\end{solution}

\begin{problem}
  Let $\{f_n\}$ be a sequence of nonnegative functions in $L^1[0,1]$ with
  the property that $\int_0^1 f_n(t)\diff t=1$ and $\int_{1/n}^1
  f_n(t)\diff t\leq 1/n$ for all $n$. Define $h(x)=\sup_n f_n(x)$. Prove
  that $h\notin L^1[0,1]$.
\end{problem}
\begin{solution}
\end{solution}

%%% Local Variables:
%%% mode: latex
%%% TeX-master: "../MA544-Quals"
%%% End:
