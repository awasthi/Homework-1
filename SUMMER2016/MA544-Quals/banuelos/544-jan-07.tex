\subsection{Bañuelos: Winter 2007}
\begin{problem}
  Let $f\colon[0,1]\to\bbR$.
  \begin{enumerate}[label=(\roman*),noitemsep]
  \item Define what it means for $f$ to be absolutely continuous.
  \item Define what it means for $f$ to be of bounded variation.
  \item Let $V(f;0,x)$ be the total variation of $f$ on $[0,x]$. Prove that
    if $f$ is absolutely continuous on $[0,1]$ so is $V(f;0,x)$.
  \end{enumerate}
\end{problem}
\begin{solution}
\end{solution}

\begin{problem}
  \hfill
  \begin{enumerate}[label=(\roman*),noitemsep]
  \item Suppose that $f\colon[0,1]\to\bbR$ is nondecreasing with $f(0)=0$
    and $f(1)=1$. For $a>0$, let $A$ be set of all $x\in(0,1)$ for which
    \[
      \limsup_{h\to 0}\frac{f(x+h)-f(x)}{h}>a.
    \]
    Prove that $m^*(A)<1/a$, where $m^*$ denotes the Lebesgue outer
    measure.
  \item Prove that there is no Lebesgue measurable set $A$ in $[0,1]$ with
    the property that $m(A\cap I)=m(I)/4$ for every interval $I$.
  \end{enumerate}
  [\emph{Hint}: Consider the function $f(x)=\chi_A(x)$.]
\end{problem}
\begin{solution}
\end{solution}

\begin{problem}
\end{problem}
\begin{solution}
\end{solution}

\begin{problem}
\end{problem}
\begin{solution}
\end{solution}

\begin{problem}
\end{problem}
\begin{solution}
\end{solution}

\begin{problem}
\end{problem}
\begin{solution}
\end{solution}

%%% Local Variables:
%%% mode: latex
%%% TeX-master: "../MA544-Quals"
%%% End:
