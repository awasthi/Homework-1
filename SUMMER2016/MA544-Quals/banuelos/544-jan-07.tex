\subsection{Bañuelos: Winter 2007}
\setcounter{exercise}{0}
\begin{problem}
  Let $f\colon[0,1]\to\bbR$.
  \begin{enumerate}[label=(\roman*),noitemsep]
  \item Define what it means for $f$ to be absolutely continuous.
  \item Define what it means for $f$ to be of bounded variation.
  \item Let $V(f;0,x)$ be the total variation of $f$ on $[0,x]$. Prove that
    if $f$ is absolutely continuous on $[0,1]$ so is $V(f;0,x)$.
  \end{enumerate}
\end{problem}
\begin{solution}
\end{solution}

\begin{problem}
  \hfill
  \begin{enumerate}[label=(\roman*),noitemsep]
  \item Suppose that $f\colon[0,1]\to\bbR$ is nondecreasing with $f(0)=0$
    and $f(1)=1$. For $a>0$, let $A$ be set of all $x\in(0,1)$ for which
    \[
      \limsup_{h\to 0}\frac{f(x+h)-f(x)}{h}>a.
    \]
    Prove that $m^*(A)<1/a$, where $m^*$ denotes the Lebesgue outer
    measure.
  \item Prove that there is no Lebesgue measurable set $A$ in $[0,1]$ with
    the property that $m(A\cap I)=m(I)/4$ for every interval $I$.
  \end{enumerate}
  [\emph{Hint}: Consider the function $f(x)=\chi_A(x)$.]
\end{problem}
\begin{solution}
\end{solution}

\begin{problem}
  Let ${\{E_j\}}_{j=1}^\infty$ be Lebesgue measurable sets in $[0,1]$ and
  let $E=\bigcup_{j=1}^\infty E_j$ and suppose there is an $\varepsilon>0$
  such that $\sum_{j=1}^\infty m(E_j)\leq m(E)+\varepsilon$.
  \begin{enumerate}[label=(\roman*),noitemsep]
  \item Show that for all measurable sets $A\subset[0,1]$
    \[
      \sum_{j=1}^\infty m(A\cap E_j)\leq m(A\cap E)+\varepsilon.
    \]
  \item Let $A$ be the set of all $x\in[0,1]$ which are in at least two of
    $E_j'$. Prove that $m(A)\leq\varepsilon$.
  \end{enumerate}
\end{problem}
\begin{solution}
\end{solution}

\begin{problem}
  Let $(X,\calF,\mu)$ be a finite measure space. Let
  $f_n\colon X\to[0,\infty)$ be a sequence measurable functions and suppose
  that $\|f_n\|_p\leq 1$, $1<p<\infty$, and that $f_n\to f$ almost
  everywhere. Prove
  \begin{enumerate}[label=(\roman*),noitemsep]
  \item $f\in L^p(\mu)$.
  \item $\|f_n-f\|_1\to 0$ as $n\to\infty$.
  \end{enumerate}
\end{problem}
\begin{solution}
  Check this ring out $R\llbracket X\rrbracket$
\end{solution}

\begin{problem}
\end{problem}
\begin{solution}
\end{solution}

\begin{problem}
\end{problem}
\begin{solution}
\end{solution}

%%% Local Variables:
%%% mode: latex
%%% TeX-master: "../MA544-Quals"
%%% End:
