\chapter{Workshop Notes}
I will write a summary of the work done at Snowbird over the week. Here is
some preparation for the workshop.
\section{Knot Theory}
\subsection{Definitions and First Examples}
\begin{definition}[Provisional]
  A \emph{knot} is a closed loop of string in $\bbR^3$; two knots are
  equivalent if one can be wiggled around, stretched, tangled and untangled
  until it coincides with the other. Cutting and rejoining is \emph{not}
  permitted.
\end{definition}

All this means is that if $K_1$ and $K_2$ are two knots in $\bbR^3$, then
we say $K_1\simeq K_2$ if and only if $K_1$ can be deformed
\emph{continuously} into $K_2$.

\begin{remark}
  Can one produce a table of the simplest knot types (a \emph{knot type}
  means an equvialence class of knots, in other words a \emph{topological}
  as opposed to a \emph{geometrical} knot: often, we will simply call it a
  knot). Simplest is clearly something we will need to define: How should
  one measure the complexity of knots?
\end{remark}

\begin{definition}
  A \emph{link} is simply a collection of (finitely-many) disjoint closed
  loops of string in $\bbR^3$; each loop is called a \emph{component} of
  the link. Equivalence is defined in the obvious way. A knot is therefore
  just a one-component link.
\end{definition}

\begin{definition}
  The \emph{crossing number $c(K)$} of a knot $K$ is the minimal number of
  crossings in any diagram of that knot. (This is a natural measure of
  complexity.) A \emph{minimal diagram} of $K$ is one with $c(K)$
  crossings.
\end{definition}
\subsection{Operations on Knots}
\begin{definition}
  The \emph{mirror-image $\bar K$} of a knot $K$ is obtained by reflecting
  it in a plane in $\bbR^3$. It may also be defined given by the diagram
  $D$ of $K$: One simply exchanges all crossings of $D$. This is evident if
  one considers reflecting in the plane of the page.
\end{definition}

\begin{definition}
  A knot is called \emph{amphichiral} if it is equivalent to its own
  mirror-image. How might one detect amphichirality? The trefoil is in fact
  not amphichiral, whilst the figure-eight is.
\end{definition}

\begin{definition}
  An \emph{oriented} knot is one with a chosen direction or arrow of
  circulation along the string. Under equivalence this direction is carried
  along as well, so one may talk about equivalence (meaning
  \emph{orientation-preserving equivalence}) of oriented knots.
\end{definition}

\begin{definition}
  The \emph{reverse $rK$} of an oriented knot $K$ is simply the same knot
  with the opposite orientation. One may also define the \emph{inverse
    $r\bar K$} as the composition of the reversal and the mirror-image. By
  analogy with amphichirality, we have a notion of a knot being
  \emph{reversible} or \emph{invertible} if it is equivalent to its reverse
  or inverse. Reversibility is very difficult to detect; the knot $8_{17}$
  is the first non-reversible one (discovered by Trotter in the 60s).
\end{definition}

\begin{definition}
  If $K_1$, $K_2$ are oriented knots, one may form their \emph{connected
    sum $K_1\# K_2$} by removing a little arc from each and splicing up the
  ends to get a single component making sure that the orientations glue to
  get a consistent orientation on the result.
\end{definition}

This operation behaves rather like multiplication on the positive
integers. It is commutative with the unknot as identity. A natural question
is whether there is an inverse; could one somehow cancel out the
knottedness of a knot $K$ by connect-summing with another knot? This seems
implausible, and we will prove it false. Thus, knots form a
\emph{semigroup} under connected-sum. In this semigroup, just as in the
positive integers under multiplication, there is a notion of \emph{prime
  factorisation}, which we shall study later.

\subsection{Alternating Diagrams}
\begin{definition}
  An \emph{alternating diagram $D$} of a knot $K$ is a diagram such that it
  passes alternatively over and under crossings, when circling completely
  around the diagram from some arbitrary starting point. An
  \emph{alternating knot $K$} is one which possesses \emph{some}
  alternating diagram.
\end{definition}

\subsection{Unknotting number}
\begin{definition}
  The \emph{unknotting number $u(K)$} of a knot $K$ is the minimum, over
  all diagrams $D$ of $K$, of the minimal number of crossing changes
  required to turn $D$ into a diagram of the unknot.
\end{definition}

\section{Formal Definitions and Reidemeister Moves}
\subsection{Knots and equivalence}
How should we formulate the definition of a knot?

The most obvious way is to consider parametric curves in $\bbR^3$. Let
$I\coloneq [0,1]$ be the closed unit interval in $\bbR$. A continuous
vector-valued function $\bfx(s)$ with domain $I$ defines such a curve; the
continuity requirement makes sure that it is unbroken. If we also impose
the condition $\bfx(0)=\bfx(1)$ then the initial and the final point are
made to coincide, so we have a parametric representation of a \emph{closed
  loop}, rather than just an arc. If we require that the map
$s\mapsto\bfx(s)$ is injective on the interval $[0,1)$, then we enforce
that the curve \emph{does not intersect itself}. These three conditions
constitute a reasonable definition of a knot, which we now proceed to study
further.

Next question: How should we formulate the notion of deformation of a knot?

Deformation is best visualized as a time-dependent process. Imagine
starting at time $t=0$ with a knot $K_0$, and deforming it through a family
of intermediate knots $K_t$ to a final one $K_1$. We need to make sure this
process of deformation is continuous in $t$. So we could consider
vector-valued functions of two variables $\bfx(s,t)$ for $(s,t)\in I\times
I$, with the requirement thatfor each fixed value $t$, the function
$s\mapsto\bfx(s,t)$ obeys the three conditions making it a knot $K_t$.

Thus, we can define two knots $K$ and $K'$ to be equivalent if there exists
a deformation running from $K_0=K$ to $K_1=K'$.

Unfortunately, this definition is not correct as it allows for
pathological knots with infinitely much knotting. Instead, we can define
two knots to be equivalent if they are \emph{ambient isotopic}, meaning
that there exists an (orientation-preserving) homeomorphism
$\bbR^3\to\bbR^3$ carrying one to the other. This definition turns out to
fix the second problem---now, not all knots are equivalent to one
another---but it does not rule out pathological knots.

The easiest way to do that is to require that each knot should be
representable as a \emph{knotted polygon} in $\bbR^3$, that is a subset
made up of \emph{finitely many} straight-line arc segments. All \emph{tame}
knots can be approximated arbitrarily closely by polygonal subsets---we
just have to use a very large number of tiny edges. On the other hand,
pathological knots (ones that are infinitely knotted) cannot be represented
using a knotted polygon.

\begin{definition}
  If $K$ is a subset of $\bbR^3$ which can be written as the union of arc
  segments
  \[
    K=[a_0,a_1]\cup[a_1,a_2]\cup\dotsb\cup[a_{n-2},a_{n-1}]\cup[a_{n-1},a_0]
  \]
  such that the segments are disjoint from one another except when
  consecutive (in which case they intersect at a single point, i.e.,
  $[a_{k-2},a_{k-1}]\cap [a_{k-1},a_k]=\{a_k\}$) then we will say $K$ is a
  \emph{knot}.
\end{definition}

\begin{definition}
  Suppose $K$ is a knot in $\bbR^3$ having $[a_i,a_{i+1}]$ as one of its
  edges, and suppose $T\coloneq[a_i,x,a_{i+1}]$ is a closed solid triangle
  in $\bbR^3$ which intersects $K$ only along the edge
  $[a_i,a_{i+1}]$. Then we may slide the edge across the triangle without
  hitting anything: Replace the edge $[a_i,a_{i+1}]$ of $K$ by two new
  edges $[a_i,x]\cup[x,a_{i+1}]$ so as to form a new knot $K'$ with one
  more edge in total. Such a move is called a $\Delta$-move.
\end{definition}

A $\Delta$-move is clearly an absolutely basic kind of deformation of knots
which should be regarded as an equivalence. We will in fact define our
equivalence relation to be the one ``generated by $\Delta$-moves'' as
follows:

\begin{definition}
  Two knots $K$, $J$ are \emph{equivalent} (or \emph{isotopic}) if there is
  a sequence of knots $K=K_0,K_1,K_2,\dotsc,K_n=J$ of knots such that each
  pair $K_i$, $K_{i+1}$ is related by a $\Delta$-move or the reverse of a
  $\Delta$-move.
\end{definition}

This clearly defines an equivalence relation on knots.

We will often confuse knots in $\bbR^3$ with their equivalence classes,
which are the things we are really interested in topologically.


\subsection{Projections and diagrams}
\begin{definition}
  If $K$ is a knot in $\bbR^3$, its \emph{projection} is
  $\pi(K)\subset\bbR^2$, where $\pi$ is the projection along the $z$-axis
  onto the $xy$-plane. The projection is said to be \emph{regular} if the
  preimage of a point of $\pi(K)$ conists of either one or two points of
  $K$, in the latter case neither being a vertex of $K$. Clearly a knot has
  an \emph{irregular} projection if it has any edges parallel to the
  $z$-axis, if it has three or more points lying above each otehr, or any
  vertex lying above or below another point of $K$; on the other hand, a
  regular projection of a knot consists of a polygonal circle drawn in the
  plane with only ``transverse double points'' as self-intersections.
\end{definition}

\begin{definition}
  If $K$ has a regular projection then we can define the corresponding
  \emph{knot diagram} $D$ by redrawing it with a broken arc near each
  \emph{crossing} to incorporate the over/under information. If $K$ had an
  irregular projection then we would not be able to easily reconstruct it
  from this sort of picture so it is important that we can find regular
  projections of knots easily.
\end{definition}

\begin{definition}
  Define an \emph{$\varepsilon$-perturbation} of a knot $K$ in $\bbR^3$ to
  be any knot $K'$ obtained by moving each of the vertices of $K$ a
  distance less than $\varepsilon$, and reconnecting them with straight
  edges in the same fashion as $K$.
\end{definition}

If $\varepsilon$ is chosen sufficiently small then all such
$\varepsilon$-perturbations of $K$ will be equivalent to it. Regular
projections are generic. This means that knots which have regular
projections form an open, dense set in the space of knots. Or, more
precisely, the following two properties hold:
\begin{enumerate}[label=(\arabic*)]
\item If $K$ has an iregular projection then there exists \emph{arbitrarily
  small} $\varepsilon$-perturbations $K'$
\end{enumerate}

%%% Local Variables:
%%% mode: latex
%%% TeX-master: "../Snowbird-Notes"
%%% End:
