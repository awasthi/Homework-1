\section{Algebraic Geometry}
Generally speaking, in order to do geometry we need
\begin{enumerate}[label=(\arabic*)]
\item A topological space.
\item A notion of locally standard objects. (For example, in the case of
  real manifolds, a ball in $\bbR^n$. In the case of complex manifolds, a
  ball in $\bbC^n$. In the case of algebraic varieties, a system of
  polynomials that are locally $0$.)
\item Some set of functions on the space (perhaps locally defined). For
  example, in the real case, $C^k$-functions, or smooth functions, or
  analytic functions In the complex case, holomorphic functions, or
  analytic functions. In the complex case, holomorphic functions.
\item Maps between the objects given by (1), (2), (3).
\end{enumerate}

Another theme in algebraic geometry is that of classifying a space (or
moduli space). Assume that we have some geometric algebraic object $X$.This
object $X$ is at least a topological space.

\begin{question}
  Given $X$, with some topological structure, classify all algebraic
  structures it carries, compatible with the underlying topological
  structure.
\end{question}

\begin{example}
  Consider the elliptic curve of equation
  \[
    y^2=ax^3+bx+c,\qquad(a,b,c\in\bbC),
  \]
  where the right hand side has distinct roots. Geometrically, this is a
  genus $1$ complex surface with one point missing. If we compactify, we
  obtain the usual torus.
\end{example}

Here are two things that we can ask ourveles.

What are the algebraic structures carried by the torus?

Given $X$ an algebraic variety, classify all subobjects of $X$.

This problem can only be handled if we fix some discrete invariants. Then
it might be possible to classify the subobjects, and the classifying space
might also be an algebraic variety.

Consider the special case where $\bbk=\bbC$ and $X=\bbA^n$. We would like
to classify all the subvarieties of $\bbA^n$ traced by polynomials
$\bff_1,\dotsc,\bff_m\in\bbC[X_1,\dotsc,X_n]$. Let us consider the easier
problem which is to classify the linear subvarieties of $\bbA^n$. Using
translation, we may assume without loss of generality that they pass
through the origin. The invariant is the dimension $d$, where $0\leq d\leq
n$. The cases $d=0$, $n$ are trivial. Let $G(n,d)$ denote the space of all
linear transformations of dimension $d$ in $\bbA^n$ through $\mathbf{0}$.

Observe that there is an isomorphism
\[
G(n,d)\simeq G(n,n-d)
\]
given by duality. We will treat the case $d=1$, since it is simpler. We
need to classify all lines through the origin $\mathbf{0}$ in $\bbA^n$. Let
$\Sigma$ be the unit sphere in $\bbA^n$, that is, the set
\[
  \Sigma\coloneq\left\{\,\bfz:\sum|z_i|^2=1,\bfz=(z_1,\dotsc,z_n)\,\right\}
\]

%%% Local Variables:
%%% mode: latex
%%% TeX-master: "../Snowbird-Notes"
%%% End:
