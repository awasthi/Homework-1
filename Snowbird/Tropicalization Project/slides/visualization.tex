\begin{frame}
  \frametitle{Visualizing Tropical Character Varieties}
  \framesubtitle{Part I}%
  Summary of work: Our group worked on visualization of tropicalized
  character varieties into $\SL_2\bbC$ of $\bbZ\times\bbZ$ and $\bbZ*\bbZ$
  as well as several knot groups whose $A$-polynomials we obtained from a
  paper by Eric Chesebro \emph{Formulas for Character Varieties of
    $2$-Bridge Knots} which can be found on his website
  \url{http://hs.umt.edu/math/research/technical-reports/documents/2012/KnotFormulas.pdf}.

  With the help of \texttt{Mathematica}, we made the following subdivision
  of the Newton polytopes of these character varieties.
\end{frame}

\begin{frame}
  \frametitle{Visualizing Tropical Character Varieties}
  \framesubtitle{Part II}
  Some pictures
  \begin{figure}
    \includegraphics[scale=0.5]{newtonsub_apoly_fig8}
    \caption{Subdivided Newton polytope for the $A$-polynomial of the
      figure-$8$ knot.}
  \end{figure}
\end{frame}
\begin{frame}
  \begin{figure}
    \includegraphics[scale=0.45]{newtonsub_apoly_twist4}
    \caption{Subdivided Newton polytope for the $A$-polynomial the
      $4$-twisted knot.}
  \end{figure}
\end{frame}

\begin{frame}
  \frametitle{Visualizing Tropical Character Varieties}
  \framesubtitle{Part III} Some more pictures for Newton polytopes of two
  $2$-bridge knots corresponding to $\varphi(1)$ and $\varphi(2)$ of
  Chesebro's equation.
  \[
    \includegraphics[scale=0.5]{newtonsub_phi1}
    \quad
    \includegraphics[scale=0.5]{newtonsub_phi2}
  \]
\end{frame}

%%% Local Variables:
%%% mode: latex
%%% TeX-master: "../MRC16-Trop-CharVars-Slides"
%%% End:
