\begin{frame}{$S_{2,g}$ - a recursive approach}

  \begin{definition}
    $S_{2,g}$ is the number of words over an alphabet of size $g$ such
    that no letter appears (considering absolute multiplicities) more
    than twice.
  \end{definition}

  \pause

  \begin{definition}
    $S_{2,g}^{d,l,u}$ is the number of words counted by $S_{2,g}$ which
    have exactly $d$ exponents of $\pm 2$, have \emph{reduced} length
    $l$ (exponents are ignored), and use exactly $u$ distinct letters.
  \end{definition}

  \pause

  \[S_{2,g} = \sum_{d=0}^{g} \sum_{l=0}^{2g} \sum_{u=0}^{g}
    S_{2,g}^{d,l,u}, \]
  and
  \[ S_{2,0}^{0,0,0} = 1. \]


\end{frame}

\begin{frame}{The recursion}

  {\tiny
    \begin{align*}
      S_{2,g}^{d,l,u} =& S_{2,g-1}^{d,l,u} \\
                       &+\sum_{j=0}^{u-1} {g-1 \choose j} \cdot j! \cdot 2^j \cdot (l - 2j) \cdot \left( S_{2,g-1-j}^{d, l-1-2j, u-j-1} + S_{2,g-1-j}^{d-1,l-1-2j,u-j-1} \right) \cdot 2 \\
                       &+  \sum_{j=0}^{u-1} \sum_{k=0}^{u-1-j} { g-1 \choose j + k} \cdot (j + k)! \cdot 2^{j + k} \cdot \frac{(l - 2(j + k) - 1) \cdot (l - 2 (j + k) - 2)}{2} \cdot S_{2,g-1-j-k}^{d,l-2(j+k) - 2,u-1-j-k} \cdot 4 \\
                       &+ \sum_{j=0}^{0} \sum_{k=1}^{u-1-j} { g-1 \choose j + k} \cdot (j + k)! \cdot 2^{j + k} \cdot \frac{(l - 2(j + k) - 1) \cdot 2}{2} \cdot S_{2,g-1-j-k}^{d,l-2(j+k) - 2,u-1-j-k} \cdot 4 \\
                       &+ \sum_{j=1}^{u-1} \sum_{k=0}^{0} { g-1 \choose j + k} \cdot (j + k)! \cdot 2^{j + k} \cdot \frac{(l - 2(j + k) - 1) \cdot 2}{2} \cdot S_{2,g-1-j-k}^{d,l-2(j+k) - 2,u-1-j-k} \cdot 4 \\
                       &- \sum_{j=1}^{u-1} \sum_{k=1}^{u-1-j} { g-1 \choose j + k} \cdot (j + k)! \cdot 2^{j + k} \cdot \frac{(l - 2(j + k) - 1) \cdot 2}{2} \cdot S_{2,g-1-j-k}^{d,l-2(j+k) - 2,u-1-j-k} \cdot 4 \\
                       &+ \sum_{j=1}^{u-1} \sum_{k=0}^{u-1-j} { g-1 \choose j + k} \cdot (j + k)! \cdot 2^{j + k} \cdot \frac{(l - 2(j + k) - 1) \cdot 2(j-1)}{2} \cdot S_{2,g-1-j-k}^{d,l-2(j+k) - 2,u-1-j-k} \cdot 4\\
                       &+ \sum_{j=1}^{u-1} \sum_{k=1}^{u-1-j} { g-1 \choose j + k} \cdot (j + k)! \cdot 2^{j + k} \cdot \frac{(l - 2(j + k) - 1) \cdot 2)}{2} \cdot S_{2,g-1-j-k}^{d,l-2(j+k) - 2,u-1-j-k} \cdot 4\\
    \end{align*}
  }

\end{frame}

\begin{frame}{Some calculations}
  This has been implemented, with the following results
  \begin{align*}
    S_{2,0} &= 1 \\
    S_{2,1} &= 5 \\
    S_{2,2} &= 105 \\
    S_{2,3} &= 6061 \\
    S_{2,4} &= 668753 \\
    &\vdots
  \end{align*}

  Conclusion: it is impractical to consider naïve generators when examining
  representations $F_n$ to $\SL_2\bbC$.
\end{frame}

%%% Local Variables:
%%% mode: latex
%%% TeX-master: "../MRC16-Trop-CharVars-Slides"
%%% End:
