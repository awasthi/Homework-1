\begin{frame}{$S_{2,g}$ - a recursive approach}

  \begin{definition}
    $S_{2,g}$ is the number of words over an alphabet of size $g$ such
    that no letter appears (considering absolute multiplicities) more
    than twice.
  \end{definition}

  \pause

  \begin{definition}
    $S_{2,g}^{d,\ell,u}$ is the number of words counted by $S_{2,g}$ which
    have exactly $d$ exponents of $\pm 2$, have \emph{reduced} length
    $\ell$ (exponents are ignored), and use exactly $u$ distinct letters.
  \end{definition}

  \pause

  \[S_{2,g} = \sum_{d=0}^{g} \sum_{\ell=0}^{2g} \sum_{u=0}^{g}
    S_{2,g}^{d,\ell,u}, \]
  and
  \[ S_{2,0}^{0,0,0} = 1. \]


\end{frame}

\begin{frame}{The recursion}

{\tiny
\[
\begin{aligned}
  S_{2,g}^{d,\ell,u}
  &= S_{2,g-1}^{d,\ell,u} \\
  &+\sum_{j=0}^{u-1} {g-1 \choose j}  j!  2^j  (\ell - 2j)  \left( S_{2,g-1-j}^{d, \ell-1-2j, u-j-1} + S_{2,g-1-j}^{d-1,\ell-1-2j,u-j-1} \right)  2 \\
  &+  \sum_{j=0}^{u-1} \sum_{k=0}^{u-1-j} { g-1 \choose j + k}  (j + k)!  2^{j + k}  \frac{(\ell - 2(j + k) - 1)  (\ell - 2 (j + k) - 2)}{2}  S_{2,g-1-j-k}^{d,\ell-2(j+k) - 2,u-1-j-k}  4 \\
  &+ \sum_{j=0}^{0} \sum_{k=1}^{u-1-j} { g-1 \choose j + k}  (j + k)!  2^{j + k}  \frac{(\ell - 2(j + k) - 1)  2}{2}  S_{2,g-1-j-k}^{d,\ell-2(j+k) - 2,u-1-j-k}  4 \\
  &+ \sum_{j=1}^{u-1} \sum_{k=0}^{0} { g-1 \choose j + k}  (j + k)!  2^{j + k}  \frac{(\ell - 2(j + k) - 1)  2}{2}  S_{2,g-1-j-k}^{d,\ell-2(j+k) - 2,u-1-j-k}  4 \\
  &- \sum_{j=1}^{u-1} \sum_{k=1}^{u-1-j} { g-1 \choose j + k}  (j + k)!  2^{j + k}  \frac{(\ell - 2(j + k) - 1)  2}{2}  S_{2,g-1-j-k}^{d,\ell-2(j+k) - 2,u-1-j-k}  4 \\
  &+ \sum_{j=1}^{u-1} \sum_{k=0}^{u-1-j} { g-1 \choose j + k}  (j + k)!  2^{j + k}  \frac{(\ell - 2(j + k) - 1)  2(j-1)}{2}  S_{2,g-1-j-k}^{d,\ell-2(j+k) - 2,u-1-j-k}  4\\
  &+ \sum_{j=1}^{u-1} \sum_{k=1}^{u-1-j} { g-1 \choose j + k}  (j + k)!  2^{j + k}  \frac{(\ell - 2(j + k) - 1)  2)}{2}  S_{2,g-1-j-k}^{d,-2(j+k) - 2,u-1-j-k}  4
\end{aligned}
\]
}
\end{frame}

\begin{frame}{Some calculations}
  This has been implemented, with the following results
  \begin{align*}
    S_{2,0} &= 1 \\
    S_{2,1} &= 5 \\
    S_{2,2} &= 105 \\
    S_{2,3} &= 6061 \\
    S_{2,4} &= 668753 \\
    &\vdots
  \end{align*}

  Conclusion: it is impractical to consider naïve generators when examining
  representations $F_n$ to $\SL_2\bbC$.
\end{frame}

%%% Local Variables:
%%% mode: latex
%%% TeX-master: "../MRC16-Trop-CharVars-Slides"
%%% End:
