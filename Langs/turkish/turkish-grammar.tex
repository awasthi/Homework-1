\chapter{Turkish}
\section{Grammar}
This is a primer on Turkish grammar. Most of the material is taken from the
web, but especially from the site \textsf{Manisa Turkish}:
\begin{quote}
  \url{http://www.turkishlanguage.co.uk/}.
\end{quote}
This is for personal use and especially for those times when I don't have
access to the internet and want to brush up on my grammar.

\subsection{Turkish verbs}
Here we talk about Turkish verbs, their use, their conjugation and the
cases that they take.

First, let us talk about the infinitive form
\subsubsection{The infinitive}
To form the infinitive of a Turkish verb, you must add the suffix
\textbf{-mek/-mak} to the stem of the verb: \textbf{gel} \(\to\)
\textbf{gelmek} -- \emph{to come}; \textbf{al} \(\to\) \textbf{almak} --
\emph{to take, buy}. The choice of adding \textbf{-mek} or \textbf{-mak}
abides by the laws of
\href{https://en.wikipedia.org/wiki/Vowel_harmony}{vowel harmony}.

\subsubsection{The negative infinitive}
The negative of a verb has a separate infinitive form formed by first
attaching the negative suffix \textbf{-me/-ma} to the stem of the verb and
then adding the infinitive suffix: \textbf{gel} \(\to\) \textbf{gelme}
\(\to\) \textbf{gelmemek} -- \emph{to not come}; \textbf{al} \(\to\)
\textbf{alma} \(\to\) \textbf{almamak}.

Here are some more examples of infinitives and negative infinitive forms:
\begin{itemize}
\item \textbf{vermek} -- \emph{to give}; \textbf{vermemek} -- \emph{to not
    give};
\item \textbf{bilmek} -- \emph{to know}; \textbf{bilmemek} -- \emph{to not
    know};
\item \textbf{görmek} -- \emph{to see}; \textbf{görmemek} -- \emph{to not
    see};
\item \textbf{gülmek} -- \emph{to laugh}; \textbf{gülmemek} -- \emph{to not
    laugh};
\item \textbf{yapmak} -- \emph{to do, make, perform}; \textbf{yapmamak} --
  \emph{to not do};
\item \textbf{ağrımak} -- \emph{to ache}; \textbf{ağrımamak} -- \emph{to
    not ache};
\item \textbf{kopmak} -- \emph{to snap}; \textbf{kopmamak} -- \emph{to not
    snap};
\item \textbf{kurumak} -- \emph{to dry}; \textbf{kurumamak} -- \emph{to not
  dry}.
\end{itemize}

\subsubsection{Turkish infinitive as the object of a verb}
When used in combination with other verbs, the infinitive is nominalized
and takes case and personal endings like other nouns:

\begin{itemize}
\item Geçen hafta ödevimi yapmayı unuttum.\\
  I forgot to do my homework last week.
\item Garajdan arabamı almayı unuttum.\\
  I forgot to pick up (lit.\@ take ) my car from the garage.
\item Ali, sana söylemeyi unuttu.\\
  Ali forgot to tell you.
\item Affedersin, seni aramayı unuttuk.\\
  We're sorry, we forgot to call you.
\item Filmi izlemeyi unuttum.\\
  I forgot to watch the film.
\item Kediyi beslemeyi unuttum.\\
  I forgot to feed the cat.
\item Mehmet'i sormayı unuttunuz.\\
  You forgot to ask Mehmet.
\item Kapıyı unuttum.\\
  I forgot to close the door.
\item Pencereyi açmayı unuttular.\\
  They forgot to open the window.
\item Pencereyi açmamayı unuttular.\\
  They forgot not to open the window.
\end{itemize}

\subsubsection{Verbal objects in the dative}
Although most Turkish verbs take the accusative form in combination, some
verbs require the dative form: \textbf{yazmaya başladı} -- \emph{she
  started to write/she started to (the) writing}.

\subsubsection{Exception -- istemek}
The one exception to the above rules is the verb \textbf{istemek} --
\emph{to want} which requires that the verb stay in the infinitive form if
the desire to act is being conveyed and if the subject of both
\textbf{istemek} and the verb it governs are the same:
\begin{itemize}
\item Yazmak istiyorum.\\
  I want to write.
\item İçmek istiyorlar.\\
  They want to drink.
\item Kalmak istemedin.\\
  You didn't want to stay.
\item Çalışmak istemeyecekler.\\
  They will not want to work.
\end{itemize}

Otherwise, \textbf{istemek} the verb being governed by \textbf{istemek} is
nominalized and takes the appropriate personal suffix plus the accusative:
\begin{itemize}
\item Kalmamanızı istiyoruz.\\
  We want you to not stay.
\item Kalmanızı istemiyoruz.\\
  We do not want you to stay.
\item Kalmasını istemiyorlar.\\
  They do not want him to stay.
\item Kalmamalarını istemiyorum.\\
  I don't want them to not stay.
\end{itemize}

\subsubsection{Examples of suffixed Turkish infinitives}
Vowel harmony and consonant mutation rules for Turkish must be followed
when adding the standard suffixes. In addition to this, the buffer letter
\textbf{-y} is used to keep the vowels introduced by the suffixes apart.

\begin{itemize}
\item Gelmeye çalıştı.\\
  He tried to come.
\item Yüzmeyi severim.\\
  I like to swim.
\item Onu yapmaktayım.\\
  I am just doing it.
\item Sigara içmeyi bıraktım.\\
  I have just quit (given up) smoking.
\end{itemize}

\subsubsection{Extended Turkish infinitive forms}
Here is a table of the different forms the nominalization of the verb
\textbf{gelmek} can take:
\begin{tabular}{ccc}
  positive form&&negative form\\
  gelmek&to come&gelmemek&to not come\\
  gelmeye&to come&gelmemeye&to not come\\
  gelmeyi&to come (obj.)&gelmemeyi&to not come (obj.)\\
  gelmekte&in coming&gelmemekte&not in coming\\
  gelmekten&from coming&gelmemekten&from not coming\\
  gelmekle&by/with coming&gelmemekle&by/with not coming.
\end{tabular}

\begin{itemize}
\item Kesmeyi bıraktı.\\
  He stopped cutting.
\item Sürmeyi öğreniyorum.\\
  I am learning to drive.
\item Gülmemeye çalışıyorlar.\\
  They are trying not  to laugh.
\end{itemize}

%%% Local Variables:
%%% mode: latex
%%% TeX-master: "../Language-Wiki-CES"
%%% End:
