\chapter{German}
Dialogues and exercises from Colloquial German 2.

\section{Eine neue Wohnung}

\begin{exercise}[Text]
Marina, Sven, Matthias, Jasmin und Kilian ziehen um!

Ab dem 5. August ist unsere neue Adresse
\begin{quote}
  Ferdinand--Lehmann--Str.\@ 17\\
  45657 Recklinghausen\\
  Tel: 02361/1486770\\
  Handy: 0179/8648305
\end{quote}
Die E-Mailadresse bleibt gleich \texttt{fam.feldmann@weblink.de}.

Wir freuen uns auf euren Besuch!
\end{exercise}

\begin{exercise}
  \hfill
  \begin{enumerate}[label=\arabic*.]
  \item Wann zieht die Familie Feldmann um?
  \item Haben sie eine neue E-Mailadresse?
  \item Worauf freuen sie sich?
  \end{enumerate}
\end{exercise}
\begin{solution}\hfill
  \begin{enumerate}[label=\arabic*.]
  \item Die Familie Feldmann ziehet ab dem 5 von August um.
  \item Nein, die E-Mailadresse bleibt gleich.
  \item Sie freuen sich auf Besuch.
  \end{enumerate}
\end{solution}

\begin{exercise}
  Underline all the verbs you find in Text 1. Write down their infinitive
  and what kind of verbs they are -- weak, strong, reflexive, separable.
\end{exercise}
\begin{solution}
  The verbs and their infinitives are:
  \begin{center}
  \begin{tabular}{|l|l|l|l|l|l|}
    \hline
    infinitive&meaning&weak&strong&reflexive&separable\\
    \hline
    umziehen&to move&no&yes&no&yes\\
    bleiben&to remain&no&yes&no&no\\
    freuen sich auf&to look forward to&no&no&no&no\\
    \hline
  \end{tabular}
  \end{center}
\end{solution}

%%% Local Variables:
%%% mode: latex
%%% TeX-master: "../Language-Wiki-CES"
%%% End:
