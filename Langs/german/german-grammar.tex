\chapter{German}
\section{Grammar}
\subsection{Orthography and Phonology}
German is one of those languages (like Spanish or Italian) whose phonology
very closely matches the way words are spelled (the orthography).

\begin{color}{Red}I will update this section in the future;
  perhaps.\end{color}

\subsection{The cases}
German, like many European languages (particularly Slavic and Northern
German languages), has a rather complicated system of declensions whereby
the shape of a noun changes depending on its role in the sentence. For
example, in English, when we say
\begin{quote}
  \emph{A shark ate Peter.}
\end{quote}
the grammar of the English language forbids us from making a lot of change
to the structure of this sentence and we have to do some language
gymnastics to get the same meaning across, for example,
\begin{quote}
  \emph{Peter a shark ate.}
\end{quote}
whereas,
\begin{quote}
  \emph{Peter ate a shark.}
\end{quote}
has a completely different meaning from the first sentence.

However, in German, this ambiguity is resolved by its use of a case
system. For example, we might translate the sentence above as
\begin{quote}
  \emph{Peter hat ein Hai gegessen.}
\end{quote}
That is, a shark ate Peter or more clearly
\begin{quote}
  \emph{Ein Hai hat Peter gegessen.}
\end{quote}
Whereas
\begin{quote}
  \emph{Peter hat \textcolor{Red}{einen} Hai gegessen.}
\end{quote}

\subsubsection{The Nominative}
The nominative case is

\subsection{The Accusative}
The accusative

%%% Local Variables:
%%% mode: latex
%%% TeX-master: "../Language-Wiki-CES"
%%% End:
