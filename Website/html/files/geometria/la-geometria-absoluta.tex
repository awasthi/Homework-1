\chapter{Geometría}
\section{La geometría absoulta}
En este capítulo y los siguientes nos ocuparemos de formalizar en el seno
de la teoría de conjuntos la geometría intuitiva, es decir, la geometría
con la que interpretamos nuestras percepciones que, como se sabe, no
coinciden con la geometría que los físicos usan para describir el espacio
físico. Podemos decir que lo que vamos a estudiar es el \emph{espacio}. No
podemos definir este concepto, pero todos tenemos imagen intuitiva del
mismo.

La primera aproximación a la caracterización matemática del espacio será
considerar al espacio como un conjunto $\bbE$ a cuyos elementos llamaremos
\emph{puntos}. Un punto es una posición en el espacio, carente de toda
extensión.

Hay dos conceptos más que se encuentran al mismo nivel elemental que el
concepto de punto. Se trata de los conceptos de \emph{recta} y
\emph{plano}. De nuevo es imposible definir la característica que
diferencia a una línea recta de una línea curva o a una superficie plana de
una superficie curva, pero intuitivamente todos sabemos distinguir las
rectas y los planos de las restantes curvas y superficies.
\subsection{Axiomas de incidencia}
\begin{definition}
Una \emph{geometría de Hilbert} está formada por un conjunto $\bbE$ al que
llamaremos \emph{espacio}, y a cuyos elementos llamaremos \emph{puntos},
junto con dos familias no vacías de subconjuntos no vacíos de $\bbE$ a
cuyos elementos llamaremos respectivamente \emph{rectos y planos}, de modo
que se cumplan los cinco axiomas indicados a continuación
\end{definition}

Diremos que una recta o plano $X$ \emph{pasa} por un punto $P$, o que $X$
\emph{incide} en el punto $P$, si $P\in X$.

\begin{axioma}
Por cada par de puntos distintos $P$ y $Q$ pasa una única recta, que
representaremos por $PQ$.
\end{axioma}

\begin{axioma}
Toda recta pasa por al menos dos puntos.
\end{axioma}

Diremos que tres o más puntos son \emph{colineales} si hay una recta que
pasa por todos ellos.

\begin{axioma}
Por cada tres puntos no colineales $P$, $Q$ y $R$ pasa un único plano, que
representaremos por $PQR$.
\end{axioma}

\begin{axioma}
Si una recta tiene dos puntos en común con un plano $\Pi$, entonces está
contenida en $\Pi$.
\end{axioma}

\begin{axioma}
Todo plano pasa por al menos tres puntos no colineales.
\end{axioma}

Éstos no son los únicos axiomas de incidencia que vamos a considerar, sino
que más adelante consideraremos algunos axiomas alternativos o adicionales,
pero conviene distinguir los resultados que pueden probarse únicamente a
partir de estos axiomas.

\begin{teorema}
Toda recta tiene un punto exterior que esté en un plano dado que la contenga.
\end{teorema}
Por Axioma 5.
\begin{teorema}
Si $P$ y $Q$ son puntos de una recta $r$ y $R$ es exterior a $r$, entonces
$P$, $Q$ y $R$ no son colineales.
\end{teorema}
Por Axioma 1.
\begin{teorema}
Dos rectas distintas tienen como máximo un punto en común.
\end{teorema}
Por Axioma 1.

\begin{definicion}
Diremos que dos rectas son
\end{definicion}


%%% Local Variables:
%%% mode: latex
%%% TeX-master: "../Notas-de-Matematicas"
%%% End:
