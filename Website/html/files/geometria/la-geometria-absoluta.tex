\chapter{Geometría}
\section{La geometría absoulta}
En este capítulo y los siguientes nos ocuparemos de formalizar en el seno
de la teoría de conjuntos la geometría intuitiva, es decir, la geometría
con la que interpretamos nuestras percepciones que, como se sabe, no
coinciden con la geometría que los físicos usan para describir el espacio
físico. Podemos decir que lo que vamos a estudiar es el \emph{espacio}. No
podemos definir este concepto, pero todos tenemos imagen intuitiva del
mismo.

La primera aproximación a la caracterización matemática del espacio será
considerar al espacio como un conjunto $\bbE$ a cuyos elementos llamaremos
\emph{puntos}. Un punto es una posición en el espacio, carente de toda
extensión.

%%% Local Variables:
%%% mode: latex
%%% TeX-master: "../Notas-de-Matematicas"
%%% End:
