\chapter{Midterms and Qualifying Exams}
\section{Qualifying Exam, August '04}
\begin{problem}
  Consider the initial value problem
  \[
    \left\{
      \begin{aligned}
        a(x,y)u_x+b(x,y)u_y&=-u,\\
        u&=f,&\text{on \(S^1=\bigl\{\,x^2+y^2=1\,\bigr\}\)},
      \end{aligned}
    \right.
  \]
  where \(a\) and \(b\) satisfy
  \[
    a(x,y)+b(x,y)y>0
  \]
  for any \(x,y\in\bbR^n\setminus(0,0)\).
  \begin{itemize}[noitemsep]
  \item[(a)] Show that the initial value problem has a unique solution in a
    neighborhood of \(S^1\). Assume that \(a\), \(b\), and \(f\) are
    smooth.
  \item[(b)] Show that the solution of the initial value problem actually
    exists in \(\bbR^2\setminus(0,0)\).
  \end{itemize}
\end{problem}
\begin{solution}
\end{solution}

\begin{problem}
  Let \(u\in C^2\bigl(\bbR\times[0,\infty)\bigr)\) be a solution of the
  initial value problem for the one-dimensional wave equation
  \[
    \left\{
      \begin{aligned}
        u_{tt}-u_{xx}&=0,&&&&\text{on \(\bbR\times(0,\infty)\),}\\
        u&=f,&u_t&=g,&&\text{in \(\bbR\times 0\),}
      \end{aligned}
    \right.
  \]
  where \(f\) and \(g\) have compact support. Define the kinetic energy by
  \[
    K(t)=\frac{1}{2}\int_{-\infty}^\infty u_t^2(x,t)\diff x,
  \]
  and the potential energy by
  \[
    P(t)=\frac{1}{2}\int_{-\infty}^\infty u_x^2(x,t)\diff x.
  \]
  Show that
  \begin{itemize}[noitemsep]
  \item[(a)] \(K(t)+P(t)\) is constant in \(t\),
  \item[(b)] \(K(t)=P(t)\) for all large enough times \(t\).
  \end{itemize}
\end{problem}
\begin{solution}
\end{solution}

\begin{problem}
  Use Kirchhoff's formula and Duhamel's principle to obtain an integral
  representation of the solution of the following Cauchy problem
  \[
    \left\{
      \begin{aligned}
        u_{tt}-\Delta u&=e^{-t} g(x),&&\text{for \(x\in\bbR^3\), \(t>0\),}\\
        u(x,0)&=u_t(x,0)=0,&&\text{for \(x\in\bbR^3\).}
      \end{aligned}
    \right.
  \]
  Verify that the integral representation reduces to the obvious solution
  \(u=e^{-t}+t-1\) when \(g(x)=1\).
\end{problem}
\begin{solution}
\end{solution}

\begin{problem}
  Let \(\Omega\) be a bounded open set in \(\bbR^n\) and \(g\in
  C_0^\infty(\Omega)\). Consider the solutions of the initial boundary
  value problem
  \[
    \left\{
      \begin{aligned}
        \Delta u-u_t&=0,&&\text{for \(x\in\Omega\), \(t>0\),}\\
        u(x,0)&=g(x)&&\text{for \(x\in\Omega\),}\\
        u(x,t)&=0&&\text{for \(xi\in\partial\Omega\), \(t\geq 0\),}
      \end{aligned}
    \right.
  \]
  and the Cauchy problem
  \[
    \left\{
      \begin{aligned}
        \Delta v-v_t&=0,&&\text{for \(x\in\bbR^n\), \(t>0\),}\\
        v(x,0)&=|g(x)|&&\text{for \(x\in\bbR^n\),}
      \end{aligned}
    \right.
  \]
  where we put \(g=0\) outside \(\Omega\).
  \begin{itemize}[noitemsep]
  \item[(a)] Show that
    \[
      -v(x,t)\leq u(x,t)\leq v(x,t)
    \]
    for any \(x\in\Omega\), \(t>0\).
  \item[(b)] Use (a) to conclude that
    \[
      \lim_{t\to\infty} u(x,t)=0,
    \]
    for any \(x\in\Omega\).
  \end{itemize}
\end{problem}
\begin{solution}
\end{solution}

\begin{problem}
  Let \(P_k(x)\) and \(P_m(x)\) be homogeneous harmonic polynomials in
  \(\bbR^n\) of degrees \(k\) and \(m\) respectively; i.e.,
  \[
    P_k(\lambda x)=\lambda^kP_k(x),\qquad P_m(\lambda x)=\lambda^mP_m(x),
  \]
  for any \(x\in\bbR^n\), \(\lambda>0\),
  \[
    \Delta P_k=0,\qquad\Delta P_m=0
  \]
  in \(\bbR^n\).
  \begin{itemize}[noitemsep]
  \item[(a)] Show that
    \[
      \frac{\partial P_k(x)}{\partial\nu}=kP_k(x),\qquad
      \frac{\partial P_m(x)}{\partial \nu}=mP_m(x)
    \]
    on \(\partial B_1\), where \(B_1=\{\,|x|<1\,\}\) and \(\nu\) is the
    outward normal on \(\partial B_1\).
  \item[(b)] Use (a) and Green's second identity to prove that
    \[
      \int_{\partial B_1} P_k(x)P_m(x)\diff S=0,
    \]
    if \(k\neq m\).
  \end{itemize}
\end{problem}
\begin{solution}
\end{solution}

%%% Local Variables:
%%% mode: latex
%%% TeX-master: "../MA523-HW-ALL"
%%% End:
