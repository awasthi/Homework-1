\subsection{Final Practice Problems}
\begin{problem}
  Let \(\Omega\) be a bounded domain in \(\R^n\) with smooth boundary. Show
  that the problem
  \[
    \left\{
      \begin{aligned}
        -\Lap u&=f&&\text{in \(\Omega\),}\\
        u+\alpha\frac{\partial u}{\partial\nu}&=g&&\text{on \(\partial\Omega\)}
      \end{aligned}
    \right.
  \]
  has at most one solution in \(C^2(\Omega)\cap C(\bar\Omega)\) if
  \(\alpha>0\). Here \(\nu\) is the outward normal on \(\partial\Omega\)
  and \(f\), \(g\) are assumed to be smooth.
\end{problem}
\begin{solution*}
  Let us assume that \(\Omega\) is also a connected subset of \(\R^n\). We
  will use energy methods to show that there is only one solution to the
  problem
  \begin{equation}
    \label{eq:finprac:laplace-eq}
    \left\{
      \begin{aligned}
        -\Lap u&=f&&\text{in \(\Omega\),}\\
        u+\alpha\frac{\partial u}{\partial\nu}&=g&&\text{on \(\partial\Omega\)}
      \end{aligned}
    \right.
  \end{equation}

  Suppose \(u_1\) and \(u_2\) are two distinct solutions to the problem
  \eqref{eq:finprac:laplace-eq}. Define \(v\defeq u_1-u_2\). Then \(v\)
  solves the problem
  \[
    \left\{
      \begin{aligned}
        -\Lap v&=0&&\text{in \(\Omega\),}\\
        v+\alpha\frac{\partial v}{\partial\nu}&=0&&\text{on \(\partial\Omega\).}
      \end{aligned}
    \right.
  \]
  Consider the energy
  \[
    E[v]=\frac{1}{2}\int_\Omega|Dv|^2\diff x.
  \]
  By Green's formula, we may recast the expression above as the sum
  \begin{align*}
    E[v]
    &=-\frac{1}{2}
      \left[%
      \int_\Omega v\Lap v\diff x%
      +\int_{\partial\Omega}\frac{\partial v}{\partial\nu }v\diff S(x)%
      \right]\\
    &=-\frac{\alpha}{2}\int_{\partial\Omega} v^2\diff S(x)\\
    &\geq 0.
  \end{align*}
  However, since \(\alpha>0\) and \(v^2\) is strictly positive, it must be
  the case that \(v\equiv 0\) on \(\partial\Omega\). The maximum principle
  then implies that \(v\equiv 0\) in \(\Omega\). It follows that
  \(u_1=u_2\); i.e., the solution is unique.
\end{solution*}

\begin{problem}
  Let \(g\) be a continuous function with compact support in
  \(\R^n\). Write the formula for the bounded solution of
  \[
    \left\{
      \begin{aligned}
        u_t-\Lap u&=0&&\text{in \(\R^n\times(0,\infty)\),}\\
        u&=g&&\text{on \(\R^n\times\{\,t=0\,\}\).}
      \end{aligned}
    \right.
  \]
  Prove that \(\lim_{t\to\infty} u(x,t)=0\), where the convergence is
  uniform in \(x\in\R^n\).
\end{problem}
\begin{solution*}
  From previous work on the heat equation, we know that the convolution
  \[
    u(x,t)
    =\frac{1}{(4\pi t)^{\frac{n}{2}}}%
    \int_{\R^n}\rme^{-\frac{|x-y|^2}{4t}}g(y)\diff y
  \]
  the initial-value problem above. A crude estimate on the magnitude of
  \(u\) gives us
  \begin{align*}
    |u(x,t)|
    &=%
      \frac{1}{(4\pi t)^{\frac{n}{2}}}%
      \left|
      \int_{\R^n}\rme^{-\frac{|x-y|^2}{4t}}g(y)\diff y
      \right|\\
    &\leq%
      \frac{1}{(4\pi t)^{\frac{n}{2}}}
      \int_{\R^n}\left|\rme^{-\frac{|x-y|^2}{4t}}g(y)\right|\diff y\\
    &<M t^{-\frac{n}{2}};
  \end{align*}
  where \(M<\infty\) is chosen such that
  \[
    \frac{1}{(4\pi)^{\frac{n}{2}}}
    \int_{\Supp g}\left|\rme^{-\frac{|x-y|^2}{4t}}g(y)\right|\diff y
    <M.
  \]
  Having established this estimate, we have \(\lim_{t\to\infty}u(x,t)=0\)
  uniformly.
\end{solution*}

\begin{problem}
  Find an explicit solution to the problem
  \[
    \left\{
      \begin{aligned}
        u_t-u_{xx}&=0&&\text{on \(\R\times(0,\infty)\),}\\
        u&=\rme^{3x}&&\text{in \(\R\times\{\,t=0\,\}\).}
      \end{aligned}
    \right.
  \]
\end{problem}
\begin{solution*}
  By separation of variables,


  All we need to do is compute the convolution
  \[
    u(x,t)=%
    \frac{1}{\sqrt{4\pi t}}
    \int_\R \rme^{-\frac{|x-y|^2}{4t}}
    \rme^{3x}\diff y.
  \]
  Putting this through \textsf{Mathematica} gives
  \[
    u(x,t)= \frac{1}{\sqrt{4\pi t}} \left( \sqrt{4\pi t}\rme^{9t+3x}
    \right)=\rme^{9t+3x}.
  \]
\end{solution*}

\begin{problem}
  Find a formula for the solution of
  \[
    \left\{
      \begin{aligned}
        u_{tt}-u_{xx}+u=0&&&&\text{in \(\R\times(0,\infty)\),}\\
        u(x,0)=f(x),&&u_t(x,0)=g(x),
      \end{aligned}
    \right.
  \]
  where \(f,g\in C_0^\infty(\R)\).

  \noindent (\emph{Hint:} Method I: Use Hadamard's method of
  descent. Namely, find \(h(y)\) such that \(v(x,y,t)\defeq h(y)u(x,t)\)
  solves
  \[
    v_{tt}-(v_{xx}+v_{yy})=0.
  \]

  \noindent Method II: Use the Fourier transform.)
\end{problem}
\begin{solution*}
\end{solution*}

\begin{problem}
  Let \(u\in C^2(\R^n\times[0,\infty))\) satisfy
  \[
    \left\{
      \begin{aligned}
        u_{tt}-\Lap u=0&&&\text{in \(\R^n\times(0,\infty),\)}\\
        u(x,0)=g(x),&&u_t(x,0)=h(x).
      \end{aligned}
    \right.
  \]
  Show that if both \(g\) and \(h\) are radial, then so is \(u(\blank,t)\)
  for any \(t>0\). (Recall that the function \(f\) is called radial if
  \(f(x)=f(|x|)\).)
\end{problem}
\begin{solution*}
\end{solution*}

\begin{problem}
  Find the value of the solution \(u\) of the initial value problem
  \[
    \left\{
      \begin{aligned}
        u_{tt}-\Lap u=0&&&\text{for \(x\in\R^3\), \(t>0\),}\\
        u(x,0)=0,&u_t(x,0)=\phi(x),
      \end{aligned}
    \right.
  \]
  where
  \[
    \phi(x)\defeq
    \begin{cases}
      1&\text{for \(|x|<a\),}\\
      0&\text{for \(|x|\geq a\)}
    \end{cases}
  \]
  at a point \((x,t)\) such that \(|x|+t<a\).
\end{problem}
\begin{solution*}
\end{solution*}

\begin{problem}
  LEt \(u\) be a nonzero harmonic function in
  \(B(0,R)\defeq\bigl\{\,x\in\R^n:|x|<R\,\bigr\}\). Define
  \[
    E(r)\defeq\fint_{\partial B(0,r)}u^2(y)\diff\sigma_y.
  \]
  Show that \(\ln E(r)\) is a convex function of \(\ln r\); i.e.,
  \[
    E\bigl(\sqrt{ab}\bigr)^2\leq E(a)E(b),\quad\text{for \(a,b>0\),}
  \]
  for any \(0<a\leq c<R\).

  \noindent (\emph{Hint:} Use the representation of \(u\) as a uniformly
  convergent series
  \[
    u(x)=\sum_{k=0}^\infty p_k(x),\qquad |x|<R,
  \]
  where \(p_k(x)\) is a homogeneous harmonic polynomial of order \(k\).)
\end{problem}
\begin{solution*}
\end{solution*}

\begin{problem}
  Use Kirchhoff's formula and Duhamel's principle to obtain an integral
  representation of the solution to the following Cauchy problem,
  \[
    \left\{
      \begin{aligned}
        u_{tt}-\Lap u=\rme^{-t}f(x)&&\text{\(x\in\R^3\), \(t>0\),}\\
        u(x,0)=u_t(x,0)=0,&&x\in\R^3.
      \end{aligned}
    \right.
  \]
  Verify that the integral representation reduces to the obvious solution
  \(u=\rme^{-t}+t-1\) when \(f(x)=1\).
\end{problem}
\begin{solution*}
\end{solution*}

\begin{problem}
  Let \(f(x)=\rme^{-|x|^2}\), \(x\in\R^n\). Find \(f*f\).

  \noindent (\emph{Hint:} Use either the heat equation or the Fourier
  transform.)
\end{problem}
\begin{solution*}
\end{solution*}

\begin{problem}
  Recall that a solution to the heat equation
  \[
    \left\{
      \begin{aligned}
        u_t-\Lap u=0&&\text{in \(\R^n\times(0,\infty)\),}\\
        u=g&&\text{on \(\R^n\times\{t=0\}\)}
      \end{aligned}
    \right.
  \]
  is given by
  \[
    u(x,t)=\int_{\R^n}\Phi(x-y,t)g(y)\diff t,
  \]
  where, for \(t>0\),
  \[
    \Phi(z,t)=(4\pi t)^{-\frac{n}{2}}\rme^{-\frac{|x|^2}{4t}}.
  \]
  Assume that \(g\) is continuous and compactly supported. Show that there
  exists a \(C>0\) such that
  \[
    |D u(x,t)|\leq Ct^{-\frac{1}{2}}\|g\|_{L^\infty}.
  \]
\end{problem}
\begin{solution*}
\end{solution*}

%%% Local Variables:
%%% mode: latex
%%% TeX-master: "../MA523-HW-ALL"
%%% End:
