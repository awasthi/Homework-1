\subsection{Homework 7}
\begin{problem}
  Solve the Dirichlet problem for the Laplace equation in \(\R^2\)
  \[
    \left\{
      \begin{aligned}
        &\Lap u=0&&\text{in \(1<|x|<2\),}\\
        &u=x_1&&\text{on \(|x|=1\),}\\
        &u=1+x_1x_2&&\text{on \(|x|=2\).}
      \end{aligned}
    \right.
  \]
  \\\\
  \emph{Hint:} Use Laurent series.
\end{problem}
\begin{solution*}
  First, let us make the change of variables \((x_1,x_2)\mapsto
  r\rme^{\rmi\theta}\) to the Dirichlet problem in question:
  \begin{equation}
    \label{eq:7:dirichlet-problem-polar}
    \left\{
      \begin{aligned}
        &\Lap u=0&&\text{in \(1<r<2\),}\\
        &u=\tfrac{1}{2}(\rme^{\rmi\theta}+\rme^{-\rmi\theta})&&\text{on \(r=1\),}\\
        &u=1+\tfrac{1}{\rmi}(\rme^{\rmi 2\theta}-\rme^{-\rmi
          2\theta})&&\text{on \(r=2\).}
      \end{aligned}
    \right.
  \end{equation}

  Now, suppose \(u\) is a solution, of the form
  \[
    u(r\rme^{\rmi\theta})=%
    b\ln r%
    +\sum_{n=-\infty}^\infty (a_nr^n+\overline{a_{-n}}r^{-n})\rme^{\rmi n\theta},
  \]
  to the problem \eqref{eq:7:dirichlet-problem-polar}. It is easy to see
  that \(u\) is in fact harmonic:
  \begin{align*}
    \Lap u%
    &=u_{rr}+\tfrac{1}{r}u_r+\tfrac{1}{r^2}u_{\theta\theta}\\
    &=-br^{-2}+br^{-2}\\
    &\qquad+\sum_{n=-\infty}^\infty
      \left[\bigl(n(n-1)+n-n^2\bigr) a_nr^n
      +\bigl(n(n-1)+n-n^2\bigr)\overline{a_{-n}}r^{-n}\right]\rme^{\rmi
      n\theta}\\
    &=0.
  \end{align*}

  Next we use the boundary data to compute the coefficients \(a_n\),
  \(n\in\Z\). Using the data \eqref{eq:7:dirichlet-problem-polar}, on
  \(r=1\) we have
  \[
    \tfrac{1}{2}(\rme^{\rmi\theta}+\rme^{-\rmi\theta})%
    =\sum_{n=-\infty}^\infty (a_n+\overline{a_{-n}})\rme^{\rmi n\theta},
  \]
  and on \(r=2\)
  \[
    1+\tfrac{1}{\rmi}(\rme^{\rmi 2\theta}-\rme^{-\rmi 2\theta}) = b\ln
    2+\sum_{n=-\infty}^\infty (2^na_n+2^{-n}a_{-n})\rme^{\rmi n\theta}.
  \]
  These equations immediately tell us that \(b=\frac{1}{\ln 2}\). Moreover,
  the following relations on the coefficients hold
  \[
    \left\{
    \begin{aligned}
      &\tfrac{1}{2}=a_1+\overline{a_{-1}},\quad
      \tfrac{1}{2}=a_{-1}+\overline{a_1},\\
      &\tfrac{1}{\rmi}=2^2a_2+2^{-2}\overline{a_{-2}},\quad
      -\tfrac{1}{\rmi}=2^2a_{-2}+2^{-2}\overline{a_{2}},\\
      &0=a_n+\overline{a_{-n}}&&&\text{for \(n\neq\pm 1\),}\\
      &0=2^{n}a_n+2^{-n}\overline{a_{-n}}&&&\text{for \(n\neq\pm 2\).}
    \end{aligned}
    \right.
  \]
  A little calculation shows that
  \[
    \left\{
      \begin{aligned}
        &a_1=-\tfrac{1}{6},\quad a_{-1}=\tfrac{2}{3},\\
        &a_2=-\tfrac{4\rmi}{15},\quad a_{-2}=-\tfrac{4\rmi}{15},\\
        &a_n=0&&\text{for \(n\neq \pm 1,\pm 2\).}
      \end{aligned}
    \right.
  \]
  Thus,
  \[
    \begin{aligned}
      u(r\rme^{\rmi\theta})
      &=\tfrac{1}{\ln 2}\ln r
      +\left(-\tfrac{4\rmi}{15}r^{-2}+\tfrac{4\rmi}{15}r^2\right)\rme^{-\rmi
        2\theta} +\left(\tfrac{2}{3}r^{-1}-\tfrac{1}{6}r\right)\rme^{-\rmi\theta}\\
      &\phantom{{}={}}
      +\left(-\tfrac{1}{6}r+\tfrac{2}{3}r^{-1}\right)\rme^{\rmi\theta}
      +\left(-\tfrac{4\rmi}{15}r^2+\tfrac{4\rmi}{15}r^{-2}\right)\rme^{\rmi
        2\theta}\\
      &=\tfrac{1}{\ln 2}\ln r-\tfrac{8}{15}r^{-4}
      \left(\frac{r^2\rme^{\rmi 2\theta}-r^2\rme^{-\rmi 2\theta}}{2\rmi}\right)
      +\tfrac{8}{15}
      \left(\frac{r^2\rme^{\rmi 2\theta}-r^2\rme^{-\rmi
            2\theta}}{2\rmi}\right)\\
      &\phantom{{}={}}
      +\tfrac{4}{3}r^{-2}
      \left(\frac{r\rme^{\rmi\theta}+r\rme^{-\rmi\theta}}{2}\right)
      -\tfrac{1}{3}
      \left(\frac{r\rme^{\rmi\theta}+r\rme^{-\rmi\theta}}{2}\right).
    \end{aligned}
  \]
  Substituting back, we have
  \[
    u(x_1,x_2)=\tfrac{1}{\ln 2}\ln(x_1^2+x_2^2)
    -\frac{16x_1x_2}{15{(x_1^2+x_2^2)}^2}+\frac{16x_1x_2}{15}
    +\frac{4x_1}{3(x_1^2+x_2^2)}-\frac{x_1}{3}
  \]
  which clearly satisfies the boundary data at \(|x|=1\) and \(|x|=2\).
\end{solution*}

\begin{problem}
  Let \(\Omega\) be a bounded domain with a \(C^1\) boundary, \(g\in
  C^2(\partial\Omega)\) and \(f\in C(\bar\Omega)\). Consider the so called
  \emph{Neumann problem}
  \[
    \label{eq:7:neumann-problem}%
    \tag{\(*\)}%
    \left\{
      \begin{aligned}
        &-\Lap u=f&&\text{in \(\Omega\),}\\
        &\frac{\partial u}{\partial\nu}=g&&\text{on \(\partial\Omega\),}
      \end{aligned}
    \right.
  \]
  where \(\nu\) is the outer normal on \(\partial\Omega\). Show that the
  solution of \eqref{eq:7:neumann-problem} in
  \(C^2(\Omega)\cap C^1(\bar\Omega)\) is unique up to a constant; i.e., if
  \(u_1\) and \(u_2\) are both solutions of \eqref{eq:7:neumann-problem},
  then \(u_2=u_1+\text{const.}\)\@ in \(\Omega\).
  \\\\
  \emph{Hint:} Look at the proof of the uniqueness for the Dirichlet
  problem by energy methods [E, 2.2.5a].
\end{problem}
\begin{solution*}
  Suppose \(u_1\) and \(u_2\) are solutions to the Neumann problem
  \eqref{eq:7:neumann-problem}. Define \(v\defeq u_1-u_2\). Then \(v\) is
  harmonic in \(\Omega\) and \(\frac{\partial v}{\partial \nu}=0\) on
  \(\partial\Omega\). Consider the energy functional
  \[
    E[v]=\frac{1}{2}\int_\Omega|Dv|^2\diff x.
  \]
  By Green's formula,
  \begin{align*}
    E[v]
    &=\frac{1}{2}\int_\Omega|Dv|^2\diff x\\
    &=-\frac{1}{2}\int_\Omega v\Lap v\diff x
      +\int_{\partial U}\tfrac{\partial
      v}{\partial\nu} v\diff S(x)\\
    &=0.
  \end{align*}
  This implies that \(|Dv|^2=Dv\cdot Dv=0\) which, since the standard inner
  product on \(\R^n\) is positive-definite, implies that \(Dw\equiv
  0\). It follows that \(u_1=u_2+\text{const}\), i.e., the solution \(u\)
  to \eqref{eq:7:neumann-problem} is unique up to a constant.
\end{solution*}

\begin{problem}
  Write down an explicit formula for a solution of
  \[
    \left\{
      \begin{aligned}
        &u_t-\Lap_x u+cu=f&&\text{in \(\R^n\times(0,\infty)\),}\\
        &u=g&&\text{on \(\R^n\times\{\,t=0\,\}\),}
      \end{aligned}
    \right.
  \]
  where \(c\in\R\).
  \\\\
  \emph{Hint:} Rewrite the problem in terms of
  \(v(x,t)\defeq\rme^{ct}u(x,t)\).
\end{problem}
\begin{solution*}
  Taking the hint, let us rewrite the problem in terms of
  \(v(x,t)=\rme^{ct}u(x,t)\):
  \begin{equation}
    \label{eq:7:heat-problem-exp}
    \left\{
      \begin{aligned}
        &v_t-\Lap_xv=\rme^{ct}f&&\text{in
          \(\R^n\times(0,\infty),\)}\\
        &v=g&&\text{on \(\R^n\times\{\,t=0\,\}\).}
      \end{aligned}
    \right.
  \end{equation}
  By Duhamel's principle, the problem \eqref{eq:7:heat-problem-exp} is
  solved by
  \[
    v(x,t)=\int_{\R^n}\Phi(x-y,t)g(y)\diff y
    +\int_0^t\int_{\R^n}\Phi(x-y,t-s)\rme^{cs}f(y,s)\diff yds,
  \]
  where \(\Phi\) is the fundamental solution to the heat equation. Thus,
  the formula
  \[
    u(x,t)=\rme^{-ct}v(x,t)=\rme^{-ct}\int_{\R^n}\Phi(x-y,t)g(y)\diff y
    +\rme^{-ct}\int_0^t\int_{\R^n}\Phi(x-y,t-s)\rme^{cs}f(y,s)\diff y\diff s
  \]
  solves the original problem.
\end{solution*}

%%% Local Variables:
%%% mode: latex
%%% TeX-master: "../MA523-HW-ALL"
%%% End:
