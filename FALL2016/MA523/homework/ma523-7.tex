\begin{problem}
  Solve the Dirichlet problem for the Laplace equation in \(\bbR^2\)
  \[
    \left\{
      \begin{aligned}
        \Delta u&=0&&\text{in \(1<|x|<2\),}\\
        u&=x_1&&\text{on \(|x|=1\),}\\
        u&=1+x_1x_2&&\text{on \(|x|=2\).}
      \end{aligned}
    \right.
  \]

  \noindent (\emph{Hint:} Use Laurent series.)
\end{problem}
\begin{solution}
  First, let us make the change of variables \((x_1,x_2)\mapsto
  r\rme^{\rmi\theta}\) to the Dirichlet problem in question:
  \begin{equation}
    \label{eq:7:dirichlet-problem-polar}
    \left\{
      \begin{aligned}
        \Delta u&=0&&\text{in \(1<r<2\),}\\
        u&=\tfrac{1}{2}(\rme^{\rmi\theta}+\rme^{-\rmi\theta})&&\text{on \(r=1\),}\\
        u&=1+\tfrac{1}{i}(\rme^{\rmi 2\theta}-\rme^{-\rmi 2\theta})&&\text{on \(r=2\).}
      \end{aligned}
    \right.
  \end{equation}

  Now, suppose \(u\) is a solution, of the form
  \[
    u(r\rme^{\rmi\theta})=%
    b\ln r%
    +\sum_{n\in\bbZ} (a_nr^n+\overline{a_{-n}}r^{-n})\rme^{\rmi n\theta},
  \]
  to the problem \eqref{eq:7:dirichlet-problem-polar}. It is easy to see
  that \(u\) is in fact harmonic:
  \begin{align*}
    \Delta u%
    &=u_{rr}+\tfrac{1}{r}u_r+\tfrac{1}{r^2}u_{\theta\theta}\\
    &=-br^{-2}+br^{-2}+\sum_{n\in\bbZ}
      \left[\bigl(n(n-1)+n-n^2\bigr) a_nr^n\right.\\
    &\phantom{{}={}-\frac{b}{r^2}+\frac{b}{r^2}+}
      \left.+\bigl(n(n-1)+n-n^2\bigr)\overline{a_{-n}}r^{-n}\right]\rme^{\rmi
      n\theta}\\
    &=0.
  \end{align*}

  Next we use the boundary data to compute the coefficients \(a_n\),
  \(n\in\bbZ\). Using the data \eqref{eq:7:dirichlet-problem-polar} on
  \(r=1\), we have
  \begin{equation}
    \label{eq:7:data-r-1}
    \tfrac{1}{2}(\rme^{\rmi\theta}+\rme^{-\rmi\theta})%
    =\sum_{n\in\bbZ}(a_n+\overline{a_{-n}})\rme^{\rmi n\theta}
  \end{equation}
  and on \(r=2\),
  \begin{equation}
    \label{eq:7:data-r-2}
    1+\tfrac{1}{i}(\rme^{\rmi 2\theta}-\rme^{-\rmi 2\theta})
    = b\ln 2+\sum_{n\in\bbZ}(2^na_n+2^{-n}a_{-n})\rme^{\rmi n\theta}.
  \end{equation}
  These equations immediately tell us that \(b=1/{\ln 2}\). Moreover,
  \[
    \begin{aligned}
      \frac{1}{2}&=a_1+\overline{a_{-1}},\\
      0&=a_n+\overline{a_{-n}}&&\text{if \(n\neq\pm 1\),}\\
      \frac{1}{i}&=2^2a_2+2^{-2}\overline{a_{-2}},\\
      -\frac{1}{i}&=2^2a_{-2}+2^{-2}\overline{a_{2}}\\
      0&=
    \end{aligned}
  \]
\end{solution}
\newpage

\begin{problem}
  Let \(\Omega\) be a bounded domain with a \(C^1\) boundary, \(g\in
  C^2(\partial\Omega)\) and \(f\in C(\bar\Omega)\). Consider the so called
  \emph{Neumann problem}
  \[
    \label{eq:7:neumann-problem}%
    \tag{\(*\)}%
    \left\{
      \begin{aligned}
        -\Delta u&=f&&\text{in \(\Omega\),}\\
        \frac{\partial u}{\partial\nu}&=g&&\text{on \(\partial\Omega\),}
      \end{aligned}
    \right.
  \]
  where \(\nu\) is the outer normal on \(\partial\Omega\). Show that the
  solution of \eqref{eq:7:neumann-problem} in
  \(C^2(\Omega)\cap C^1(\bar\Omega)\) is unique up to a constant; i.e., if
  \(u_1\) and \(u_2\) are both solutions of \eqref{eq:7:neumann-problem},
  then \(u_2=u_1+\text{const.}\)\@ in \(\Omega\).

  \noindent (\emph{Hint:} Look at the proof of the uniqueness for the
  Dirichlet problem by energy methods [E, 2.2.5a].)
\end{problem}
\begin{solution}
  Suppose \(u_1\) and \(u_2\) are solutions to the Neumann problem
  \eqref{eq:7:neumann-problem}. Define \(v\defeq u_1-u_2\). Then \(v\) is
  harmonic in \(\Omega\) and \(\partial v/\partial \nu=0\) on
  \(\partial\Omega\). Consider the energy functional
  \[
    E[v]=\frac{1}{2}\int_\Omega|Dv|^2\diff x.
  \]
  By Green's formula version (ii),
  \begin{align*}
    E[v]
    &=\frac{1}{2}\int_\Omega|Dv|^2\diff x\\
    &=-\frac{1}{2}\int_\Omega v\Delta v\diff x
      +\int_{\partial U}\frac{\partial
      v}{\partial\nu} v\diff S(x)\\
    &=0.
  \end{align*}
  This implies that \(|Dv|^2=Dv\cdot Dv=0\) which, since the standard inner
  product on \(\bbR^n\) is positive-definite, implies that \(Dw\equiv
  0\). It follows that \(u_1=u_2+\text{const}\), i.e., the solution \(u\)
  to \eqref{eq:7:neumann-problem} is unique up to a constant.
\end{solution}
\newpage

\begin{problem}
  Write down an explicit formula for a solution of
  \[
    \left\{
      \begin{aligned}
        u_t-\Delta u+cu&=f&&\text{in \(\bbR^n\times(0,\infty)\),}\\
        u&=g&&\text{on \(\bbR^n\times\{\,t=0\,\}\),}
      \end{aligned}
    \right.
  \]
  where \(c\in\bbR\).

  \noindent (\emph{Hint:} Rewrite the problem in terms of
  \(v(x,t)\defeq\rme^{ct}u(x,t)\).)
\end{problem}
\begin{solution}
\end{solution}

%%% Local Variables:
%%% mode: latex
%%% TeX-master: "../MA523-HW-Current"
%%% End:
