\begin{problem}
  Solve the Dirichlet problem for the Laplace equation in \(\bbR^2\)
  \[
    \left\{
      \begin{aligned}
        \Delta u&=0&&\text{in \(1<|x|<2\),}\\
        u&=x_1&&\text{on \(|x|=1\),}\\
        u&=1+x_1x_2&&\text{on \(|x|=2\).}
      \end{aligned}
    \right.
  \]

  \noindent (\emph{Hint:} Use Laurent series.)
\end{problem}
\begin{solution}
  First, let us convert the Dirichlet problem above to polar coordinates by
  the transformation \((x,y)\mapsto(r\cos\theta,r\sin\theta)\):
  \begin{equation}
    \label{eq:7:polar-dirichlet-problem}
    \left\{
      \begin{aligned}
        \Delta u&=0&&\text{in \(1<r<2\),}\\
        u&=r\cos\theta&&\text{on \(r=1\),}\\
        u&=1+\tfrac{1}{2}r^2\sin(2\theta)&&\text{on \(r=2\).}
      \end{aligned}
    \right.
  \end{equation}
  Now, suppose \(u(r,\theta)\) is a Laurent series solution to
  \eqref{eq:7:polar-dirichlet-problem}
  \[
    u(r,\theta)=\sum_{n=-\infty}^\infty a_nr^n\bigl(\cos(n\theta)+\sin(n\theta)\bigr).
  \]
\end{solution}
\newpage

\begin{problem}
  Let \(\Omega\) be a bounded domain with a \(C^1\) boundary, \(g\in
  C^2(\partial\Omega)\) and \(f\in C(\bar\Omega)\). Consider the so called
  \emph{Neumann problem}
  \[
    \label{eq:7:neumann-problem}%
    \tag{\(*\)}%
    \left\{
      \begin{aligned}
        -\Delta u&=f&&\text{in \(\Omega\),}\\
        \frac{\partial u}{\partial\nu}&=g&&\text{on \(\partial\Omega\),}
      \end{aligned}
    \right.
  \]
  where \(\nu\) is the outer normal on \(\partial\Omega\). Show that the
  solution of \eqref{eq:7:neumann-problem} in
  \(C^2(\Omega)\cap C^1(\bar\Omega)\) is unique up to a constant; i.e., if
  \(u_1\) and \(u_2\) are both solutions of \eqref{eq:7:neumann-problem},
  then \(u_2=u_1+\text{const.}\)\@ in \(\Omega\).

  \noindent (\emph{Hint:} Look at the proof of the uniqueness for the
  Dirichlet problem by energy methods [E, 2.2.5a].)
\end{problem}
\begin{solution}
  Suppose \(u_1\) and \(u_2\) are solutions to the Neumann problem
  \eqref{eq:7:neumann-problem}. Define \(v\defeq u_1-u_2\). Then \(v\) is
  harmonic in \(\Omega\) and \(\frac{\partial v}{\partial \nu}=0\) on
  \(\partial\Omega\). Consider the energy functional
  \[
    E[v]=\frac{1}{2}\int_\Omega|Dv|^2\diff x.
  \]
  By Green's formula version (ii),
  \begin{align*}
    E[v]
    &=\frac{1}{2}\int_\Omega|Dv|^2\diff x\\
    &=-\frac{1}{2}\int_\Omega v\Delta v\diff x
      +\int_{\partial U}\frac{\partial
      v}{\partial\nu} v\diff S(x)\\
    &=0.
  \end{align*}
  This implies that \(|Dv|^2=Dv\cdot Dv=0\) which, since the standard inner
  product on \(\bbR^n\) is positive-definite, implies that \(Dw\equiv
  0\). It follows that \(u_1=u_2+\text{const}\), i.e., the solution \(u\)
  to \eqref{eq:7:neumann-problem} is unique up to a constant.
\end{solution}
\newpage

\begin{problem}
  Write down an explicit formula for a solution of
  \[
    \left\{
      \begin{aligned}
        u_t-\Delta u+cu&=f&&\text{in \(\bbR^n\times(0,\infty)\),}\\
        u&=g&&\text{on \(\bbR^n\times\{\,t=0\,\}\),}
      \end{aligned}
    \right.
  \]
  where \(c\in\bbR\).

  \noindent (\emph{Hint:} Rewrite the problem in terms of
  \(v(x,t)\defeq\rme^{ct}u(x,t)\).)
\end{problem}
\begin{solution}
\end{solution}

%%% Local Variables:
%%% mode: latex
%%% TeX-master: "../MA523-HW-Current"
%%% End:
