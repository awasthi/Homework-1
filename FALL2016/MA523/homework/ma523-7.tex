\begin{problem}
  Solve the Dirichlet problem for the Laplace equation in \(\bbR^2\)
  \[
    \left\{
      \begin{aligned}
        \Delta u&=0&&\text{in \(1<|x|<2\),}\\
        u&=x_1&&\text{on \(|x|=1\),}\\
        u&=1+x_1x_2&&\text{on \(|x|=2\).}
      \end{aligned}
    \right.
  \]

  \noindent (\emph{Hint:} Use Laurent series.)
\end{problem}
\begin{solution}
\end{solution}
\newpage

\begin{problem}
  Let \(\Omega\) be a bounded domain with a \(C^1\) boundary, \(g\in
  C^2(\partial\Omega)\) and \(f\in C(\bar\Omega)\). Consider the so called
  \emph{Neumann problem}
  \[
    \label{eq:7:neumann-problem}%
    \tag{\(*\)}%
    \left\{
      \begin{aligned}
        -\Delta u&=f&&\text{in \(\Omega\),}\\
        \frac{\partial u}{\partial\nu}&=g&&\text{on \(\partial\Omega\),}
      \end{aligned}
    \right.
  \]
  where \(\nu\) is the outer normal on \(\partial\Omega\). Show that the
  solution of \eqref{eq:7:neumann-problem} in
  \(C^2(\Omega)\cap C^1(\bar\Omega)\) is unique up to a constant; i.e., if
  \(u_1\) and \(u_2\) are both solutions of \eqref{eq:7:neumann-problem},
  then \(u_2=u_1+\text{const.}\)\@ in \(\Omega\).

  \noindent (\emph{Hint:} Look at the proof of the uniqueness for the
  Dirichlet problem by energy methods [E, 2.2.5a].)
\end{problem}
\begin{solution}
  By energy methods, we have
\end{solution}
\newpage

\begin{problem}
  Write down an explicit formula for a solution of
  \[
    \left\{
      \begin{aligned}
        u_t-\Delta u+cu&=f&&\text{in \(\bbR^n\times(0,\infty)\),}\\
        u&=g&&\text{on \(\bbR^n\times\{\,t=0\,\}\),}
      \end{aligned}
    \right.
  \]
  where \(c\in\bbR\).

  \noindent (\emph{Hint:} Rewrite the problem in terms of
  \(v(x,t)\defeq\rme^{ct}u(x,t)\).)
\end{problem}
\begin{solution}
\end{solution}

%%% Local Variables:
%%% mode: latex
%%% TeX-master: "../MA523-HW-Current"
%%% End:
