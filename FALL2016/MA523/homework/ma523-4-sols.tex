\subsection{Homework 4}
\begin{problem}[Legendre transform]
  Let \(u(x_1,x_2)\) be a solution of the quasilinear equation
  \[
    a^{11}(Du)u_{x_1x_1}+ 2a^{12}(Du)u_{x_1x_2}+a^{22}(Du)u_{x_2x_2}=0
  \]
  in some region of \(\bbR^2\), where we can invert the relations
  \[
    p^1=u_{x_1}(x_1,x_2),\quad p^2=u_{x_2}(x_1,x_2)
  \]
  to solve for
  \[
    x^1=x^1(p_1,p_2),\quad x^2=x^2(p_1,p_2).
  \]
  Define then
  \[
    v(p)\defeq \bfx(p)\cdot p-u\bigl(\bfx(p)\bigr),
  \]
  where \(\bfx=(x^1,x^2)\), \(p=(p_1,p_2)\). Show that \(v\) satisfies the
  \emph{linear} equation
  \[
    a^{22}(p)v_{p_1p_2}-2a^{12}(p)v_{p_1p_2}+a^{11}(p)v_{p_1p_2}=0.
  \]
  \\\\
  \emph{Hint:} See [E, \S 4.4.3b], prove the identities (29).
\end{problem}
\begin{solution*}
  Assuming the regularity on \(v\) prescribed above, we compute
  \(v_{p_1p_1}\), \(v_{p_1p_2}\) and \(v_{p_2p_2}\).

  First, we compute \(v_{p_1p_2}\) since in the case of \(v_{p_1p_1}\) and
  \(v_{p_2p_2}\), there is some symmetry we can exploit. Taking the first
  partial with respect to \(p^1\), we have
  \begin{equation}
    \label{eq:4:vp1}
    \begin{aligned}
      v_{p_1} &=\tfrac{\partial}{\partial
        p_1}\left(x^1(p)p^1+x^2(p)p^2-u\bigl(\bfx(p)\bigr)\right)\\
      &=x^1(p)+x^1_{p_1}(p)p^1+x^2_{p_1}(p)p^2
      -u_{x_1}\bigl(\bfx(p)\bigr)x^1_{p_1}(p)-u_{x_2}\bigl(\bfx(p)\bigr)x^2_{p_1}(p)\\
      &=x^1+x^1_{p_1}p^1+x^2_{p_1}p^2-p^1x^1_{p_1}-p^2x^2_{p_1}\\
      &=x^1,
    \end{aligned}
  \end{equation}
  since \(u_{x_1}=p^1\) and \(u_{x_2}=p^2\).

  Similarly, for \(v_{p_2}\), we have
  \begin{equation}
    \label{eq:4:vp2}
    \begin{aligned}
      v_{p_2} &=\tfrac{\partial}{\partial
        p_2}\left(x^1(p)p^1+x^2(p)p^2-u\bigl(\bfx(p)\bigr)\right)\\
      &=x^1_{p_2}(p)x^1(p)+x^2(p)+x^2_{p_2}(p)p^2
      -u_{x_1}\bigl(\bfx(p)\bigr)x^1_{p_2}(p)
      -u_{x_2}\bigl(\bfx(p)\bigr)x^2_{p_2}(p)\\
      &=x^1_{p_2}x^1+x^2+x^2_{p_2}p^2
      -p^1x^1_{p_2}
      -p^2x^2_{p_2}\\
      &=x^2.
    \end{aligned}
  \end{equation}

  Now, taking the partial of \eqref{eq:4:vp1} with respect to \(p_1\) and
  then \(p_2\), we have
  \[
    v_{p_1p_1}=x^1_{p_1}=x^1_{u_{x_1}}, \qquad
    v_{p_1p_2}=x^1_{p_2}=x^1_{u_{x_2}},
  \]
  and similarly for \eqref{eq:4:vp2},
  \[
    v_{p_1p_2}=x^2_{p_1}=x^2_{u_{x_1}},\qquad
    v_{p_2p_2}=x^2_{p_2}=x^2_{u_{x_2}}.
  \]
  By the inverse function theorem, we have
  \begin{align*}
    \begin{bmatrix}
      v_{p_1p_1}&v_{p_1p_2}\\
      v_{p_1p_2}&v_{p_2p_2}
    \end{bmatrix}
    &=
      \begin{bmatrix}
        x^1_{u_{x_1}}&x^1_{u_{x_2}}\\
        x^2_{u_{x_1}}&x^2_{u_{x_2}}
      \end{bmatrix}
    \\
    &=\begin{bmatrix}
      u_{x_1x_1}&u_{x_1x_2}\\
      u_{x_1x_2}&u_{x_2x_2}
    \end{bmatrix}^{-1}\\
    &=\frac{1}{J}
      \begin{bmatrix}
      u_{x_2x_2}&-u_{x_1x_2}\\
      -u_{x_1x_2}&u_{x_1x_1}
    \end{bmatrix}.
  \end{align*}
  Hence,
  \begin{equation}
    \label{eq:4:verified-29}
    \left\{
      \begin{aligned}
        u_{x_1x_1}&=Jv_{p_2p_2}\\
        u_{x_1x_2}&=-Jv_{p_1p_2}\\
        u_{x_2x_2}&=Jv_{p_1p_1},
      \end{aligned}
    \right.
  \end{equation}
  which verifies Equation (29) from [E, \S 4.4.3b]. Substituting
  \eqref{eq:4:verified-29} into the original equation,
  \begin{align*}
    0&=Ja^{11}(p)v_{p_2p_2}-Ja^{12}(p)v_{p_1p_2}+Ja^{22}(p)v_{p_1p_1}\\
     &=a^{22}(p)v_{p_1p_1}-a^{12}(p)v_{p_1p_2}+a^{11}(p)v_{p_2p_2},
  \end{align*}
  as was to be shown.
\end{solution*}

\begin{problem}
  Find the solution \(u(x,t)\) of the one-dimensional wave equation
  \[
    u_{tt}-u_{xx}=0
  \]
  in the quadrant \(x>0,t>0\) for which
  \[
    \left\{
      \begin{aligned}
        &u=f,\quad u_t=g,%
        &&\text{for \((0,\infty)\times\{\,t=0\,\}\),}\\
        &u_t=\alpha u_x,%
        &&\text{for \(\{\,x=0\,\}\times(0,\infty)\)},
      \end{aligned}
    \right.
  \]
  where \(\alpha\neq -1\) is a given constant. Show that generally no
  solution exists when \(\alpha=-1\).
  \\\\
  \emph{Hint:} Use a representation \(u(x,t)=F(x-t)+G(x+t)\) for the
  solution.
\end{problem}
\begin{solution*}
  Suppose \(u(x,t)=F(x-t)+G(x+t)\) is a classical solution to the
  one-dimensional wave equation with the prescribed initial
  conditions. Then, we want to extend the data to all of \(x\) so that we
  can exploit d'Alembert's formula. Suppose we gave done this by, e.g.,
  taking the odd reflection of \(\tilde f\), \(\tilde g\), and
  \(\tilde h(x,t)\defeq \int_{x-t}^{x+t} \tilde g(s)ds\). All we need to do
  is use the initial data to find the relation between \(\tilde f\),
  \(\tilde g\), and \(\tilde g\) at \(x=0\).

  Using d'Alembert's formula,
  \[
    \tilde u(x,t) =\frac{\tilde f(x+t)+\tilde
      f(x-t)}{2}+\frac{1}{2}\int_{x-t}^{x+t}\tilde g(s)\diff s,
  \]
  we compute \(\tilde u_t(0,t)\) and \(\alpha\tilde u_x(0,t)\) to match
  our initial data.

  Hence,
  \begin{align*}
    u_t(0,t)&=\tfrac{1}{2}
              \bigl(-\tilde f(-t)+\tilde f(t)+\tilde g(t)-\tilde g(-t)\bigr),\\
    u_x(0,t)&=\tfrac{1}{2}
              \bigl(\tilde f(-t)+\tilde f(t)+\tilde g(t)-\tilde g(-t)\bigr),
  \end{align*}
  so
  \[
    0=\tfrac{1}{2}\bigl(-(1+\alpha)\tilde f(-t)+(1-\alpha)\tilde
    f(t)+(1-\alpha)(\tilde g(t)-\tilde g(t))\bigr).\qedhere
  \]
\end{solution*}

\begin{problem}
  \begin{enumerate}[label=(\alph*),noitemsep]
  \item Let \(u\) be a solution of the wave equation \(u_{tt}-c^2u_{xx}=0\)
    for \(0<x<\pi\), \(t>0\) such that \(u(0,t)=u(\pi,t)=0\). Show that the
    \emph{energy}
    \[
      E(t)=\frac{1}{2}\int_0^\pi \bigl(u_t^2+c^2u_x^2 \bigr)\diff x,\quad t>0
    \]
    is independent of \(t\); i.e., \(\frac{d}{dt}E=0\) for \(t>0\). Assume that
    \(u\) is \(C^2\) up to the boundary.
  \item Express the energy \(E\) of the Fourier series solution
    \[
      u(x,t)=\sum_{n=1}^\infty
      \bigl(a_n\cos(nct)+b_n\sin(nct)\bigr)\sin(nx)
    \]
    in terms of coefficients \(a_n\), \(b_n\).
  \end{enumerate}
\end{problem}
\begin{solution*}
  For part (a): Suppose that \(u\) is, as above, a solution to the wave
  equation which is \(C^2\) up to the boundary. We show that its energy is
  independent of \(t\), i.e., that \(\frac{d}{dt}E=0\). Assuming the energy
  is bounded, the dominated convergence theorem allows us to permute the
  order of integration and differentiation like so
  \begin{align*}
    \tfrac{d}{dt}E(t)%
    &=\frac{d}{dt}\left(\frac{1}{2}\int_0^\pi
      \bigl(u_t^2+c^2u_x^2\bigr)\diff x\right)\\
    &=\frac{1}{2}\int_0^\pi\tfrac{\partial}{\partial t}
      \bigl(u_t^2+c^2u_x^2\bigr)\diff x\\
    &=\frac{1}{2}\int_0^\pi 2u_tu_{tt}+2c^2u_xu_{xt}\diff x
      \intertext{which, after using the relation \(u_{tt}=c^2u_{xx}\), becomes}
    &=c^2\int_0^\pi u_tu_{xx}+u_xu_{xt}\diff x\\
    &=c^2\int_0^\pi\tfrac{\partial}{\partial x}(u_xu_t)\diff x\\
    &=c^2\bigl(u_x(\pi,t)u_t(\pi,t)-u_x(0,t)u_t(0,t)\bigr)\\
    &=0
  \end{align*}
  since the boundary conditions, i.e., \(u=0\), implies \(u_x=u_t=0\) at
  the boundary.
  \\\\
  For part (b): Suppose \(u\) is a Fourier series solution to the wave
  equation, i.e.,
  \[
    u(x,t)=\sum_{k=1}^\infty \bigl(a_k\cos(kct)+b_k\sin(kct)\bigr)\sin(kx).
  \]
  First, we compute \(u_x\) and \(u_t\), they are
  \[
    \begin{aligned}
      u_x(x,t)
      &=\sum_{k=1}^\infty k\bigl(a_k\cos(kct)+b_k\sin(kct)\bigr)\cos(kx),\\
      u_t(x,t)
      &=\sum_{k=1}^\infty kc\bigl(-a_k\sin(kct)+b_k\cos(kct)\bigr)\sin(kx).
    \end{aligned}
  \]
  Let \(u_x^n(x,t)\) and \(u_t^n(x,t)\) be the partial sums of the two
  equations above. Then
  \begin{align*}
    E_n(t)
    &=\frac{1}{2}\int_0^\pi (u_t^n)^2+c^2(u_x^n)^2\diff t
      \intertext{taking into account orthogonality relations of \(\cos\) and
      \(\sin\), we have}
    &=\frac{1}{2}\sum_{k=1}^n k^2c^2
      \left[
      \left(\int_0^\pi\bigl(a_k^2\sin(kct)+b_k^2\cos(kct)\bigr)\right)\sin^2(kx)\right.\\
    &\phantom{{}={}\frac{1}{2}\sum_{k=1}^n k^2c^2}
      \left.+\left(\int_0^\pi\bigl(a_k^2\cos(kct)+a_k^2\sin(kct)\bigr)\right)\cos^2(kx)
      \right]\\
    &=\frac{\pi}{2}\sum_{k=1}^n k^2c^2(a_k^2+b_k^2).
  \end{align*}
  Thus, since the limit is uniform
  \[
    E(t)=\lim_{n\to\infty}E_n(t)=\frac{\pi}{4}\sum_{n=1}^\infty
    n^2c^2(a_n^2+b_n^2).
    \qedhere
  \]
\end{solution*}

%%% Local Variables:
%%% mode: latex
%%% TeX-master: "../MA523-HW-ALL"
%%% End:
