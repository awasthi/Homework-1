\begin{problem}
  For \(n=2\) find Green's function for the quadrant
  \(U=\{\,x_1,x_2>0\,\}\) by repeated reflection.
\end{problem}
\begin{solution}
  Taking the hit, set \(x'=(x_1,-x_2)\) and \(x''=(-x_1,x_2)\) and define
  \begin{equation}
    \label{eq:6:corrector}
    \varphi^x(y)\defeq \Phi(x-y)+\Phi(x'-y)-\Phi(x''-y).
  \end{equation}

  We claim that \(\varphi^x\), as defined above, solves
  \[
    \left\{
      \begin{aligned}
        \Delta\varphi^x&=0&&\text{in \(U\),}\\
        \varphi^x(y)&=\Phi(x-y)&&\text{on \(\partial U\).}
      \end{aligned}
    \right.
  \]
\end{solution}
\newpage

\begin{problem}
  (Precise form of Harnack's inequality) Use Poisson's formula for the ball
  to prove
  \[
    \frac{r^{n-2}(r-|x|)}{(r+|x|)^{n-1}}u(0)%
    \leq u(x)%
    \leq \frac{r^{n-2}(r+|x|)}{(r-|x|)^{n-1}}u(0)
  \]
  whenever \(u\) is positive and harmonic in
  \(B(0,r)=\{\,x\in\bbR^n:|x|<r\,\}\).
\end{problem}
\begin{solution}
\end{solution}
\newpage

\begin{problem}
  Let \(P_k(x)\) and \(P_m(x)\) be homogeneous harmonic polynomials in
  \(\bbR^n\) of degrees \(k\) and \(m\) respectively; i.e.,
  \begin{align*}
    P_k(\lambda x)&=\lambda^k P_k(x),&P_m(\lambda x)&=\lambda^m P_m(x)
    &&\text{for every \(x\in\bbR^n\), \(\lambda>0\),}\\
    \Delta P_k&=0,&\Delta P_m&=0&&\text{in \(\bbR^n\).}
  \end{align*}
  \begin{enumerate}[label=(\alph*),noitemsep]
  \item Show that
    \[
      \frac{\partial P_k}{\partial \nu}=kP_k(x),\qquad
      \frac{\partial P_m}{\partial\nu}=mP_m(x)\qquad\text{on \(\partial B(0,1)\)}
    \]
    where \(B(0,1)=\{\,x\in\bbR^n:|x|<1\,\}\) and \(\nu\) is the outward
    normal on \(\partial B(0,1)\).
  \item Use (a) and Green's formula to prove that
    \[
      \int_{\partial B(0,1)}P_k(x)P_m(x)\diff\sigma=0,\qquad\text{if
        \(k\neq m\).}
    \]
  \end{enumerate}
\end{problem}
\begin{solution}
\end{solution}

%%% Local Variables:
%%% mode: latex
%%% TeX-master: "../MA523-HW-Current"
%%% End:
