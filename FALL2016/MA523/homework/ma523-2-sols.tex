\subsection{Homework 2}
\begin{problem}
  Verify assertion (36) in [E, \S 3.2.3], that when \(\Gamma\) is not flat
  near \(x^0\) the noncharacteristic condition is
  \[
    D_pF(p^0,z^0,x^0)\cdot \nu(x^0)\neq 0.
  \]
  (Here \(\nu(x^0)\) denotes the normal to the hypersurface \(\Gamma\) at
  \(x^0\)).
\end{problem}
\begin{solution*}
  First, note that the condition
  \[
    D_pF(p^0,z^0,x^0)\cdot \nu(x^0)\neq 0
  \]
  reduces to the standard noncharacteristic boundary condition if
  \(\Gamma\) is flat near \(x^0\) because in such case the normal to the
  hypersurface at \(x^0\) will be \((0,\dotsc,0,1)\) so
  \begin{align*}
    0&\neq D_pF(p^0,z^0,x^0)\cdot (0,\dotsc,0,1)\\
     &=F_{p_n}(p^0,z^0,x^0).
  \end{align*}

  We shall proceed to flatten the hypersurface \(\Gamma\) near \(x^0\) and
  apply the standard noncharacteristic boundary condition. Assuming
  \(\Gamma\) is reasonably tame, by the implicit function theorem, it can
  be written as the graph \(x_n=\phi(x_1,\dotsc,x_{n-1})\) on some
  neighborhood \(U\) of \(x^0\). Now consider the map \(\Phi\colon U\to V\)
  given by
  \[
    \left\{
      \begin{aligned}
        y_j&=\Phi^j(x)=x_j,\qquad 1\leq j\leq n-1,\\
        y_n&=\Phi^j(x)=x_n-\phi(x_1,\dotsc,x_{n-1}).
      \end{aligned}
    \right.
  \]
  Then the normal \(\nu\) to \(\Gamma\) at \(x^0\) is parallel to the
  gradient \(D\Phi^n=(-\phi_{x_1},\dotsc,-\phi_{x_{n-1}},1)\).
\end{solution*}

\begin{problem}
  Show that the solution of the quasilinear PDE
  \[
    u_t+a(u)u_x=0
  \]
  with initial conditions \(u(x,0)=g(x)\) is given implicitly by
  \[
    u=g(x-a(u)t).
  \]
  Show that the solution develops a shock (becomes singular) for some
  \(t>0\), unless \(a(g(x))\) is a nondecreasing function of
  \(x\).
\end{problem}
\begin{solution*}
\end{solution*}

\begin{problem}
  Show that the function \(u(x,t)\) defined for \(t\geq 0\) by
  \[
    u(x,t)=
    \begin{cases}
      -\tfrac{2}{3}\Bigl(t+\sqrt{3x+t^2}\Bigr)
      &\text{for \(4x+t^2>0\)}\\
      0
      &\text{for \(4x+t^2<0\)}
    \end{cases}
  \]
  is an (unbounded) entropy solution of the conservation law
  \(u_t+(u^2/2)_x=0\) (\emph{inviscid Burgers' equation}).
\end{problem}
\begin{solution*}
\end{solution*}

%%% Local Variables:
%%% mode: latex
%%% TeX-master: "../MA523-HW-ALL"
%%% End:
