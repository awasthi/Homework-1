\subsection{Homework 3}
\begin{problem}
  Consider the initial value problem
  \[
    \left\{
      \begin{aligned}
        u_t&=\sin u_x,\\
        u(x,0)&=\tfrac{\pi}{4}x.
      \end{aligned}
    \right.
  \]
  Verify that the assumptions of the Cauchy--Kovalevskaya theorem are
  satisfied and obtain the taylor series of the solution about the origin.
\end{problem}
\begin{solution*}
\end{solution*}

\begin{problem}
  Consider the Cauchy problem for \(u(x,y)\)
  \[
    \left\{
      \begin{aligned}
        u_y&=a(x, y, u)u_x+b(x,y,u),\\
        u(x,0)&=0,
      \end{aligned}
    \right.
  \]
  let \(a\) and \(b\) be analytic functions of their arguments. Assume that
  \(D^\alpha a(0,0,0)\geq 0\) and \(D^\alpha b(0,0,0)\geq 0\) for all
  \(\alpha\). (Remember by definition, if \(\alpha=0\) then
  \(D^\alpha f=f\).)
  \begin{enumerate}[label=(\alph*),noitemsep]
  \item Show that \(D^\beta u(0,0)\geq 0\) for all \(|\beta|\leq 2\).
  \item Prove that \(D^\beta u(0,0)\geq 0\) for all
    \(\beta=(\beta_1,\beta_2)\).
    \\\\
    \emph{Hint:} Argue as in the proof of the
    Cauchy--Kovalevskaya theorem; i.e., use induction in \(\beta_2\).
  \end{enumerate}
\end{problem}
\begin{solution*}
\end{solution*}

\begin{problem}
  (Kovalevskaya's example) show that the line \(\{\,t=0\,\}\) is
  characteristic for the heat equation \(u_t=u_{xx}\). Show there does not
  exist an analytic solution \(u\) of the heat equation in
  \(\bbR\times\bbR\), with \(u=1/(1+x^2)\) on \(\{t=0\}\).
  \\\\
  \emph{Hint:} Assume there is an analytic solution, compute its
  coefficients, and show instead that the resulting power series diverges
  in any neighborhood of \((0,0)\).
\end{problem}
\begin{solution*}
\end{solution*}

%%% Local Variables:
%%% mode: latex
%%% TeX-master: "../MA523-HW-ALL"
%%% End:
