\subsection{Homework 3}
\begin{problem}
  Consider the initial value problem
  \[
    \left\{
      \begin{aligned}
        u_t&=\sin u_x,\\
        u(x,0)&=\tfrac{\pi}{4}x.
      \end{aligned}
    \right.
  \]
  Verify that the assumptions of the Cauchy--Kovalevskaya theorem are
  satisfied and obtain the Taylor series of the solution about the origin.
\end{problem}
\begin{solution*}
  The equation \(F(p,z,x,t)\defeq\sin p_1-p_2=0\) is a fully nonlinear
  first-order PDE. We first verify that the curve \(\Gamma\) is
  characteristic near the origin; i.e., we must show that \(F_p\cdot\nu\neq
  0\) where \(\nu\) is the normal vector to \(\Gamma\) at the origin. In
  this case, \(\nu=(0,1)\) and \(F_p=(\cos p_1,-1)\); hence,
  \[
    F_p\cdot\nu= (\cos p_1,-1)\cdot(0,1)=-1\neq 0.
  \]

  Moreover, the curve \(\Gamma\) is analytic (since it is cut out by the
  equation \(\frac{\pi}{4}x\)) and the initial conditions are
  analytic. Therefore, the assumptions of the Cauchy--Kovalevskaya theorem
  are satisfied and we can obtain an analytic solution
  \[
    u(x,t)=\sum_{j,k\in\Z}\frac{a_{j,k}}{j!k!}x^jt^k
  \]
  about the origin.

  First, we must compute the coefficients \(a_{j,k}\). To this end, we must
  find the partial derivatives \(u_{j,k}\) and potentially, relations among
  them which will help us to find these coefficients. Naïvely listing the
  partials with respect to \(t\) and \(x\), we have
  \begin{align*}
    u(0,0)&=0&
    u_x(0,0)&=\tfrac{\pi}{4}\\
    u_t(0,0)&=\sin u_x(0,0)=\tfrac{\sqrt{2}}{2}&
    u_{xx}(0,0)&=0\\
    u_{tx}(0,0)&=0&
    u_{tt}(0,0)&=-\cos\bigl(u_x(0,0)\bigr)u_{xt}(0,0)=0\\
    u_{xxx}(0,0)&=0&
    u_{ttx}(0,0)&=0\\
    &\vdotswithin{=}&&\vdotswithin{=}
  \end{align*}
  It is not difficult to see that higher derivatives of \(u\) will be
  zero. Thus,
  \[
    u(x,t)=\tfrac{\pi}{4}x+\tfrac{\sqrt{2}}{2}t.
  \]
  Plugging this equation into \(F\) we see that
  \[
    u_t-\sin u_x=\tfrac{\sqrt{2}}{2}-\sin(\pi/4)=0;
  \]
  i.e., \(u(x,t)\), as defined above, is an analytic solution to the PDE
  \(F\).
\end{solution*}

\begin{problem}
  Consider the Cauchy problem for \(u(x,y)\)
  \[
    \left\{
      \begin{aligned}
        u_y&=a(x, y, u)u_x+b(x,y,u),\\
        u(x,0)&=0
      \end{aligned}
    \right.
  \]
  let \(a\) and \(b\) be analytic functions of their arguments. Assume that
  \(D^\alpha a(0,0,0)\geq 0\) and \(D^\alpha b(0,0,0)\geq 0\) for all
  \(\alpha\). (Remember by definition, if \(\alpha=0\) then
  \(D^\alpha f=f\).)
  \begin{enumerate}[label=(\alph*),noitemsep]
  \item Show that \(D^\beta u(0,0)\geq 0\) for all \(|\beta|\leq 2\).
  \item Prove that \(D^\beta u(0,0)\geq 0\) for all
    \(\beta=(\beta_1,\beta_2)\).

    \noindent \emph{Hint:} Argue as in the proof of the
    Cauchy--Kovalevskaya theorem; i.e., use induction in \(\beta_2\).
  \end{enumerate}
\end{problem}
\begin{solution*}
  For part (a):
  \\\\
  For part (b):
\end{solution*}

\begin{problem}
  (Kovalevskaya's example) show that the line \(\{\,t=0\,\}\) is
  characteristic for the heat equation \(u_t=u_{xx}\). Show there does not
  exist an analytic solution \(u\) of the heat equation in
  \(\R\times\R\), with \(u=1/(1+x^2)\) on \(\{t=0\}\).
  \\\\
  \emph{Hint:} Assume there is an analytic solution, compute its
  coefficients, and show instead that the resulting power series diverges
  in any neighborhood of \((0,0)\).
\end{problem}
\begin{solution*}
\end{solution*}

%%% Local Variables:
%%% mode: latex
%%% TeX-master: "../MA523-HW-ALL"
%%% End:
