\subsection{Homework 5}
\begin{problem}
  Prove that Laplace's equation \(\Lap u=0\) is rotation invariant; that
  is, if \(O\) is an orthogonal \(n\times n\) matrix and we define
  \(v(x)\defeq u(Ox)\), \(x\in\bbR^n\), then \(\Lap v=0\).
\end{problem}
\begin{solution*}
\end{solution*}

\begin{problem}
  Let \(n=2\) and \(U\) be the halfplane \(\{\,x_2>0\,\}\). Prove that
  \[
    \sup_U u=\sup_{\partial U}u
  \]
  for \(u\in C^2(U)\cap C(\bar U)\) which are harmonic in \(U\) under the
  additional assumption that \(u\) is bounded from above in \(\bar
  U\). (The additional assumption is needed to exclude examples like
  \(u=x_2\).)
  \\\\
  \emph{Hint:} Take for \(\varepsilon>0\) the harmonic function
  \[
    u(x_1,x_2)-\varepsilon\ln\sqrt{x_1^2+(x_2+1)^2}.
  \]
  Apply the maximum principle to a region
  \(\bigl\{\,x_1^2+(x_2+1)^2<a^2,x_2>0\,\bigr\}\) with large \(a\). Let
  \(\varepsilon\to 0\).
\end{problem}
\begin{solution*}
\end{solution*}

\begin{problem}
  Let \(U\subset\bbR^n\) be an open set. We say \(v \in C^2(U)\) is
  subharmonic if
  \[
    -\Lap v\leq 0\qquad\qquad\text{in \(U\).}
  \]
  \begin{itemize}
  \item[(a)] Let \(\phi\colon\bbR^m\to\bbR\) be smooth and
    convex. Assume \(u^1,\dotsc,u^m\) are harmonic in \(U\) and
    \[
      v\defeq\phi(u_1,\dotsc,u_m).
    \]
    Prove \(v\) is subharmonic.
    \\\\
    \emph{Hint:} Convexity for a smooth function \(\phi(z)\) is
    equivalent to \(\sum_{j,k=1}^m\phi_{z_j,z_k}(z)\xi_j\xi_k\geq 0\)
    for any \(\xi\in\bbR^m\).
  \item[(b)] Prove \(v\defeq|Du|^2\) is subharmonic, whenever \(u\) is
    harmonic. (Assume that harmonic functions are \(C^\infty\).)
  \end{itemize}
\end{problem}
\begin{solution*}
\end{solution*}

%%% Local Variables:
%%% mode: latex
%%% TeX-master: "../MA523-HW-ALL"
%%% End:
