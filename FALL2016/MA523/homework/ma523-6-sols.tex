\subsection{Homework 6}
\begin{problem}
  For \(n=2\) find Green's function for the quadrant
  \(U\defeq\{\,x_1,x_2>0\,\}\) by repeated reflection.
\end{problem}
\begin{solution*}
  Taking the hit, set \(x'\defeq(x_1,-x_2)\), \(x''\defeq(-x_1,x_2)\),
  \(x'''\defeq(-x_1,-x_2)\), and define
  \begin{equation}
    \label{eq:6:corrector}
    \phi^x(y)\defeq\Phi(y-x')+\Phi(y-x'')-\Phi(y-x''').
  \end{equation}

  We claim that \(\phi^x\), as defined above, solves
  \[
    \left\{
      \begin{aligned}
        &\Lap\phi^x=0&&\text{in \(U\),}\\
        &\phi^x(y)=\Phi(y-x)&&\text{on \(\partial U\).}
      \end{aligned}
    \right.
  \]

  It is clear that \(\Lap\phi^x=0\) since it is built up from the
  fundamental solutions on \(\R^n\) (this follows from the linearity of
  the Laplace operator). To see that \(\phi^x(y)=\Phi(x-y)\) on
  \(\partial U\), we do a case by case analysis.

  Note that on \(\{\,x_1=0\,\}\subset\partial U\), we have
  \begin{align*}
    \phi^x(y_1,0)
    &=\Phi(-x_1,y_2+x_2)+\Phi(-x_1,y_2-x_2)-\Phi(x_1,y_2+x_2),\\
    \intertext{where, since the fundamental solution is radial, we have
    \(\Phi(-x_1,y_2+x_2)=\Phi(x_1,y_2+x_2)\), and hence the above equals}
    &=\Phi(-x_1,y_2-x_2)\\
    &=\Phi(y-x)
  \end{align*}
  and on \(\{\,x_2=0\,\}\subset\partial U\), we have
  \begin{align*}
    \phi^x(0,y_2)
    &=\Phi(y_1-x_1,x_2)+\Phi(y_1+x_1,-x_2)-\Phi(y_1+x_1,x_2)\\
    \intertext{where, again because \(\Phi\) is radial,
    \(\Phi(y_1+x_1,-x_2)=\Phi(y_1+x_1,x_2)\), thus the above equals}
    &=\Phi(y_1-x_1,x_2)\\
    &=\Phi(y-x).
  \end{align*}
  Thus, \(\phi^x(y)=\Phi(y-x)\) on \(\partial U\).

  Therefore, Green's function on \(U\) is
  \[
    G(x,y)
    =\Phi(y-x)-\phi^x(y)%
    =\Phi(y-x)-\Phi(y-x')-\Phi(y-x'')+\Phi(y-x''').%
    \qedhere
  \]
\end{solution*}

\begin{problem}
  (Precise form of Harnack's inequality) Use Poisson's formula for the ball
  to prove
  \[
    \frac{r^{n-2}(r-|x|)}{(r+|x|)^{n-1}}u(0)%
    \leq u(x)%
    \leq \frac{r^{n-2}(r+|x|)}{(r-|x|)^{n-1}}u(0)
  \]
  whenever \(u\) is positive and harmonic in
  \(B(0,r)=\{\,x\in\R^n:|x|<r\,\}\).
\end{problem}
\begin{solution*}
  Recall Poisson's formula for the ball
  \begin{equation}
    \label{eq:6:poissons-formula}
    u(x)=\frac{r^2-|x|^2}{n\alpha_nr}
    \int_{\partial B(0,r)}\frac{g(y)}{|x-y|^n}\diff S(y),
  \end{equation}
  where \(x\in B(0,r)\) and \(u\) solves the boundary-value problem
  \[
    \left\{
      \begin{aligned}
        &\Lap u=0&&\text{in \(B(0,r)\),}\\
        &u=g&&\text{on \(\partial B(0,r)\).}
      \end{aligned}
    \right.
  \]

  For fixed \(x\in B(0,r)\), write
  \[
    u(x)=r^{n-2}(r+|x|)(r-|x|)%
    \left[%
      \frac{1}{n\alpha_n r^{n-1}} \int_{\partial%
        B(0,r)}\frac{g(y)}{|x-y|^n}\diff S(y)%
    \right]%
    .
  \]
  Now, since \(r+|x|\geq |x-y|\geq r-|x|\) for all \(y\in\partial B(0,r)\),
  we have
  \begin{equation}
    \label{eq:6:desired-inequality}
    \frac{r^{n-2}(r-|x|)}{(r+|x|)^{n-1}}\fint_{\partial
      B(0,r)}g(y)\diff S(y)\leq u(x)
    \leq\frac{r^{n-2}(r+|x|)}{(r-|x|)^{n-1}}\fint_{\partial
      B(0,r)}g(y)\diff S(y).
  \end{equation}

  Since \(u=g\) on the boundary \(\partial B(0,r)\), by applying the
  mean-value property on \eqref{eq:6:desired-inequality} we have
  \[
    \frac{r^{n-2}(r-|x|)}{(r+|x|)^{n-1}}u(0)\leq
    u(x)\leq
    \frac{r^{n-2}(r+|x|)}{(r-|x|)^{n-1}}u(0),
  \]
  as desired.
\end{solution*}

\begin{problem}
  Let \(P_k(x)\) and \(P_m(x)\) be homogeneous harmonic polynomials in
  \(\R^n\) of degrees \(k\) and \(m\) respectively; i.e.,
  \[
    \left\{%
      \begin{aligned}
        &P_k(\lambda x)=\lambda^k P_k(x),%
        &&P_m(\lambda x)=\lambda^m P_m(x)%
        &&\text{for every \(x\in\R^n\), \(\lambda>0\),}\\
        &\Lap P_k=0,%
        &&\Lap P_m=0&&\text{in \(\R^n\).}
      \end{aligned}
    \right.
  \]
  \begin{enumerate}[label=(\alph*),noitemsep]
  \item Show that
    \[
      \left\{%
        \begin{aligned}
          \frac{\partial P_k}{\partial \nu}&=kP_k(x),& \frac{\partial
            P_m}{\partial\nu}&=mP_m(x)&&\text{on \(\partial B(0,1)\),}
        \end{aligned}
      \right.
    \]
    where \(B(0,1)=\{\,x\in\R^n:|x|<1\,\}\) and \(\nu\) is the outward
    normal
    on \(\partial B(0,1)\).
  \item Use (a) and Green's formula to prove that
    \[
      \int_{\partial B(0,1)}P_k(x)P_m(x)\diff\sigma=0,\qquad\text{if
        \(k\neq m\).}
    \]
  \end{enumerate}
\end{problem}
\begin{solution*}
  For part (a): Let
  \[
    P_k(x)=\sum_{|\alpha|=k}a_\alpha x^\alpha.
  \]
  Then, since \(\nu=(x_1,\dotsc,x_n)\), the derivative along \(\nu\) is
  given by
  \begin{align*}
    \frac{\partial P_k(x)}{\partial\nu}
    &=\sum_{j=1}^n (P_k)_{x_j}x_j\\
    &=\sum_{j=1}^n\left(\sum\nolimits_{|\alpha|=k}a_\alpha
      x^\alpha\right)_{x_j}x_j\\
    &=\sum_{j=1}^n\left(\sum\nolimits_{l=1}^ma_\alpha
      x_1^{\alpha_1^l}\dotsm x^{\alpha_j^l}\dotsm
      x^{\alpha_n^l}\right)_{x_j}x_j
      \intertext{where \(\sum_{j=1}^n\alpha_j^l=k\) and \(1\leq j\leq
      \binom{n+k-1}{n}\eqdef m\) (by the stars and bars theorem)}
    &=\sum_{j=1}^n\sum_{l=1}^m
      \left(
      \alpha_j^l
      a_\alpha
      x_1^{\alpha_1^l}\dotsm x^{\alpha_j^l-1}\dotsm
      x^{\alpha_n^l}\right)x_j\\
    &=\sum_{j=1}^n\sum_{l=1}^m
      \alpha_j^l
      a_\alpha
      x_1^{\alpha_1^l}\dotsm x^{\alpha_j^l}\dotsm
      x^{\alpha_n^l}\\
    &=\sum_{j=1}^n\sum_{l=1}^m
      \alpha_j^l a_\alpha x^\alpha
      \intertext{switching the order of summation, we have}
    &=\sum_{l=1}^m\sum_{j=1}^n
      \alpha_j^l
      a_\alpha x^\alpha\\
    &=\sum_{l=1}^m ka_\alpha x^\alpha\\
    &=k\sum_{l=1}^m a_\alpha x^\alpha\\
    &=kP_k(x).
  \end{align*}
  This argument, of course, applies to every \(k\in\bbN\).

  For part (b): By Green's theorem, we have
  \begin{align*}
    \int_{B(0,r)}P_k(x)\Lap P_m(x)-(\Lap P_k(x))P_m(x)\diff x
    &=\int_{\partial
      B(0,r)}P_k(x)\tfrac{\partial}{\partial\nu}P_m(x)-\tfrac{\partial}{\partial\nu}
      P_k(x)P_m(x)\diff S(x)\\
    &=\int_{\partial B(0,r)}(m-k) P_k(x)P_m(x)\diff S(x),
  \end{align*}
  where the left-hand side is equal to zero since both \(\Lap P_k\) and
  \(\Lap P_m\) are zero. Since \(m\neq k\), it must be the case that
  \[
    \int_{\partial B(0,r)}P_k(x)P_m(x)\diff S(x)=0.\qedhere
  \]
\end{solution*}

%%% Local Variables:
%%% mode: latex
%%% TeX-master: "../MA523-HW-ALL"
%%% End:
