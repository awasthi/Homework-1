\begin{problem}[Legendre transform]
  Let \(u(x_1,x_2)\) be a solution of the quasilinear equation
  \[
    a^{11}(Du)u_{x_1x_1}+ 2a^{12}(Du)u_{x_1x_2}+a^{22}(Du)u_{x_2x_2}=0
  \]
  in some region of \(\bbR^2\), where we can invert the relations
  \[
    p^1=u_{x_1}(x_1,x_2),\quad p^2=u_{x_2}(x_1,x_2)
  \]
  to solve for
  \[
    x^1=x^1(p_1,p_2),\quad x^2=x^2(p_1,p_2).
  \]
  Define then
  \[
    v(p)\defeq \bfx(p)\cdot p-u\bigl(\bfx(p)\bigr),
  \]
  where \(\bfx=(x^1,x^2)\), \(p=(p_1,p_2)\). Show that \(v\) satisfies the
  \emph{linear} equation
  \[
    a^{22}(p)v_{p_1p_2}-2a^{12}(p)v_{p_1p_2}+a^{11}(p)v_{p_1p_2}=0.
  \]
  (\emph{Hint:} See [Evans, 4.4.3b], prove the identities (29)).
\end{problem}
\begin{solution}
\end{solution}
\newpage

\begin{problem}
  Find the solution \(u(x,t)\) of the one-dimensional wave equation
  \[
    u_{tt}-u_{xx}=0
  \]
  in the quadrant \(x>0,t>0\) for which
  \[
    \left\{
      \begin{aligned}
        u(x,0)&=f(x),&u_t(x,0)&=g(x),&\text{for \(x>0\)}\\
        u_t(0,t)&=\alpha u_x(0,t),&&&\text{for \(t>0\),}
      \end{aligned}
    \right.
  \]
  where \(\alpha\neq -1\) is a given constant. Show that generally no
  solution exists when \(\alpha=-1\). (\emph{Hint:} Use a representation
  \(u(x,t)=F(x-t)+G(x+t)\) for the solution.)
\end{problem}
\begin{solution}
\end{solution}
\newpage

\begin{problem}
  \begin{enumerate}[label=(\alph*),noitemsep]
  \item Let \(u\) be a solution of the wave equation \(u_{tt}-c^2u_{xx}=0\)
    for \(0<x<\pi\), \(t>0\) such that \(u(0,t)=u(\pi,t)=0\). Show that the
    \emph{energy}
    \[
      E(t)=\frac{1}{2}\int_0^\pi \bigl(u_t^2+c^2u_x^2 \bigr)\diff x,\quad t>0
    \]
    is independent of \(t\); i.e., \(\frac{d}{dt}E=0\) for \(t>0\). Assume that
    \(u\) is \(C^2\) up to the boundary.
  \item Express the energy \(E\) of the Fourier series solution
    \[
      u(x,t)=\sum_{n=1}^\infty
      \bigl(a_n\cos(nct)+b_n\sin(nct)\bigr)\sin(nx)
    \]
    in terms of coefficients \(a_n\), \(b_n\).
  \end{enumerate}
\end{problem}
\begin{solution}
  For part (a), suppose that \(u\) is, as above, a solution to the wave
  equation which is \(C^2\) up to the boundary. We show that its energy is
  independent of \(t\), i.e., that \(\frac{d}{dt}E=0\). Assuming the energy
  is bounded, the dominated convergence theorem allows us to permute the
  order of integration and differentiation like so
  \begin{align*}
    \tfrac{d}{dt}E(t)%
    &=\frac{d}{dt}\left(\frac{1}{2}\int_0^\pi
      \bigl(u_t^2+c^2u_x^2\bigr)\diff x\right)\\
    &=\frac{1}{2}\int_0^\pi\tfrac{\partial}{\partial t}
      \bigl(u_t^2+c^2u_x^2\bigr)\diff x\\
    &=\frac{1}{2}\int_0^\pi 2u_tu_{tt}+2c^2u_xu_{xt}\diff x
      \intertext{which, after using the relation \(u_{tt}=c^2u_{xx}\), becomes}
    &=c^2\int_0^\pi u_tu_{xx}+u_xu_{xt}\diff x\\
    &=c^2\int_0^\pi\tfrac{\partial}{\partial x}(u_xu_t)\diff x\\
    &=c^2\bigl(u_x(\pi,t)u_t(\pi,t)-u_x(0,t)u_t(0,t)\bigr)\\
    &=0
  \end{align*}
  since the boundary conditions, i.e., \(u=0\), implies \(u_x=u_t=0\) at
  the boundary.

  For part (b), suppose \(u\) is a Fourier series solution to the wave
  equation, i.e.,
  \[
    u(x,t)=\sum_{n=1}^\infty \bigl(a_n\cos(nct)+b_n\sin(nct)\bigr)\sin(nx).
  \]
  First we compute \(u_t\) and \(u_x\). They are
  \begin{align*}
    u_t(x,t)%
    &=\tfrac{\partial}{\partial t} u(x,t)\\
    &=\sum_{n=1}^\infty cn\bigl(b_n\cos(nct)-a_n\sin(nct)\bigr)\sin(nx)
    \intertext{and}
    u_x(x,t)%
    &=\tfrac{\partial}{\partial x} u(x,t)\\
    &=\sum_{n=1}^\infty n\bigl(a_n\cos(nct)+b_n\sin(nct)\bigr)\cos(nx).
  \end{align*}
  Thus,
  \begin{align*}
    E(t)%
    &=\frac{1}{2}\int_0^\pi%
      \left[\biggl( \sum_{n=1}^\infty cn\bigl(
      b_n\cos(nct)-a_n\sin(nct) \bigr)\sin(nx)
      \biggr)^2\right.\\
    &\phantom{=\frac{1}{2}\int_0^\pi\Biggl[}%
      \left.+c^2 \biggl(\sum_{n=1}^\infty
      n\bigl(a_n\cos(nct)+b_n\sin(nct)\bigr)\cos(nx) \biggr)^2 \right]
      \intertext{which, after expanding and using the fact that
      \(\cos(nct)\), \(\sin(nct)\), \(\cos(nx)\), and \(\sin(nx)\) are
      orthogonal, becomes}
    &=
  \end{align*}
\end{solution}

%%% Local Variables:
%%% mode: latex
%%% TeX-master: "../MA523-Current-HW"
%%% End:
