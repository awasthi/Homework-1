\section{Homework Solutions}
These are my (corrected) solutions to Petrosyan's Math 523 homework for the
fall semester of 2016.
\subsection{Homework 1}
\begin{problem}[Taylor's formula]
  Let \(f\colon\bbR^n\to\bbR\) be smooth, \(n\geq 2\). Prove that
  \[
    f(x)=\sum_{|\alpha|\leq k}
    \frac{1}{\alpha!}D^\alpha f(0)x^\alpha+O\bigl(|x|^{k+1}\bigr)
  \]
  as \(x\to\mathbf{0}\) for each \(k=1,2,\dotsc\), assuming that you know this
  formula for \(n=1\).
  \\\\
  \emph{Hint}: Fix \(x\in\bbR^n\) and consider the function of
  one variable \(g(t)\defeq f(tx)\). Prove that
  \[
    \frac{d^k}{dt^k}g(t)
    =\sum_{|\alpha|=k}\frac{k!}{\alpha!} D^\alpha f(tx)x^\alpha,
  \]
  by induction on \(m\).
\end{problem}
\begin{solution*}
  Taking the hint, let us consider the function in one variable
  \(g(t)\defeq f(tx)\) for \(x\in\bbR^n\) fixed. We show by induction on
  \(k\) that
  \begin{equation}
    \label{eq:1:taylors-formula-g}
    \frac{d^k}{dt^k}g(t)=%
    \sum_{|\alpha|=k}\frac{k!}{\alpha!} D^\alpha f(tx)x^\alpha.
  \end{equation}

  Once we have shown \eqref{eq:1:taylors-formula-g} holds, evaluating
  \(g\) at \(t=1\) gives us the desired equality; i.e,
  \begin{align*}
    f(x)
    &=g(1)%
      \shortintertext{which, by Taylor's formula in one variable, is}
    &=\sum_{j=0}^k\frac{g^{(j)}(0)}{j!}1^j+O\bigl(|x|^{k+1}\bigr)%
      \shortintertext{applying \eqref{eq:1:taylors-formula-g} here gives us}
    &=\sum_{j=0}^k\frac{1}{j!}
      \left[%
      \sum\nolimits_{|\alpha|=j}\frac{j!}{\alpha!}D^\alpha f(tx)x^\alpha
      \right]+O\bigl(|x|^{k+1}\bigr)\\
    &=\sum_{j=0}^k
      \left[%
      \sum\nolimits_{|\alpha|=j}\frac{1}{\alpha!}D^\alpha f(0)x^\alpha
      \right]+O\bigl(|x|^{k+1}\bigr)\\
    &=\sum_{|\alpha|\leq k}\frac{1}{\alpha!}D^\alpha f(0)x^\alpha+O\bigl(|x|^{k+1}\bigr)
  \end{align*}
  as desired.

  Let us now show that \eqref{eq:1:taylors-formula-g} holds. To prove this
  we consider the algebra on the differential operator \(d/dt\). By the
  chain rule, we have
  \[
    \frac{d}{dt}(\blank)=\sum_{j=1}^n x_j\frac{\partial}{\partial x_j}(\blank).
  \]
  Since \(f\) is smooth by Schwartz's theorem the differential operators
  \(\partial/\partial x_j\) and \(\partial/\partial x_l\) commute for all
  \(1\leq j,l\leq n\). Therefore, by the multinomial theorem,
  \[
    \frac{d^k}{dt^k}(\blank)=%
    \left(
      \sum\nolimits_{j=1}^n x_j\frac{\partial}{\partial x_j}(\blank)
    \right)^k
    =%
    \sum_{|\alpha|=k}\frac{m!}{\alpha!} x^\alpha D^\alpha(\blank).\qedhere
  \]
\end{solution*}

\begin{problem}
  Write down the characteristic equation for the PDE
  \[
    \label{eq:1:1}
    \tag{\(*\)}
    u_t+b\cdot Du=f
  \]
  on \(\bbR^n\times(0,\infty)\), where \(b\in\bbR^n\). Using the
  characteristic equation, solve \eqref{eq:1:1} subject to the initial
  condition \(u=g\) on \(\bbR^n\times\{t=0\}\). Make sure the answer agrees
  with formula (5) in \S 2.1.2 of [E].
\end{problem}
\begin{solution*}
  Write
  \[
    F(p,z,x,t)\defeq (b,1)\cdot p-f=0.
  \]
  Then the characteristic equations to the problem \eqref{eq:1:1} with the
  initial condition \(u(\blank,0)=g(\blank)\) are given by
  \[
    \left\{
      \begin{aligned}
        \dot p&=-D_{x,t}F-D_zFp=0,\\
        \dot z&=D_pF\cdot p=(b,1)\cdot p=f,\\
        (\dot x,\dot t)&=D_pF=(b,1).
      \end{aligned}
    \right.
  \]
\end{solution*}

\begin{problem}
  Solve using the characteristics:
  \begin{enumerate}[label=(\alph*)]
  \item \(x_1^2u_{x_1}+x_2^2u_{x_2}=u^2\), \(u=1\) on the line
    \(x_2=2x_1\).
  \item \(uu_{x_1}+u_{x_2}=1\), \(u(x_1,x_1)=x_1/2\).
  \item \(x_1u_{x_1}+2x_2u_{x_2}+u_{x_3}=3u\),
    \(u(x_1,x_2,0)=g(x_1,x_2)\).
  \end{enumerate}
\end{problem}
\begin{solution*}
\end{solution*}

\begin{problem}
  For the equation
  \[
    u=x_1u_{x_1}+x_2u_{x_2}+\tfrac{1}{2}(u_{x_1}^2+u_{x_2}^2)
  \]
  find a solution with \(u(x_1,0)=(1-x_1^2)/2\).
\end{problem}
\begin{solution*}
\end{solution*}

%%% Local Variables:
%%% mode: latex
%%% TeX-master: "../MA523-HW-ALL"
%%% End:
