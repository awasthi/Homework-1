\begin{problem}[Taylor's formula]
  Let \(f\colon\bbR^n\to\bbR\) be smooth, \(n\geq 2\). Prove that
  \[
    f(x)=\sum_{|\alpha|\leq k}
    \frac{1}{\alpha!}D^\alpha f(0)x^\alpha+\rmO\bigl(|x|^{k+1}\bigr)
  \]
  as \(x\to\mathbf{0}\) for each \(k=1,2,\dotsc\), assuming that you know this
  formula for \(n=1\).
  \\\\
  \emph{Hint}: Fix \(x\in\bbR^n\) and consider the function of one variable
  \(g(t)\defeq f(tx)\). Prove that
  \[
    \frac{\rmd^m}{\rmd t^m}g(t)
    =\sum_{|\alpha|=m}\frac{m!}{\alpha!} D^\alpha f(tx)x^\alpha,
  \]
  by induction on \(m\).
\end{problem}
\begin{solution}
  Taking the hint, fix \(x\in\bbR^n\) and consider the function of one
  variable \(g(t)\defeq f(tx)\). We claim that
  \[
    \frac{\rmd^m}{\rmd t^m}g(t)%
    =\sum_{|\alpha|=m}\frac{m!}{\alpha!}D^\alpha f(tx)x^\alpha.%
  \]
  \begin{subproof}[Proof of claim]
    We shall proceed by induction on \(m\). The case \(m=1\) follows easily
    from the chain rule:
    \begin{align*}
      \frac{\rmd}{\rmd t}g(t)
      &=\frac{\rmd}{\rmd t}f(tx)\\
      &=D^{(1,0,\dotsc,0)}f(tx)x_1+\dotsb+D^{(0,\dotsc,0,1)}f(tx)x_n\\
      &=\bigl(D^{(1,0,\dotsc,0)}x_1+\dotsb+D^{(0,\dotsc,0,1)}x_n\bigr)f(tx)
    \end{align*}
    More generally, applying the equation above recursively, we have
    \begin{align*}
      \frac{\rmd^m}{\rmd t^m}g(t)
      &=%
        \bigl(D^{(1,0,\dotsc,0)}x_1+\dotsb+D^{(0,\dotsc,0,1)}x_n\bigr)^m f(tx)
      \intertext{by the multinomial theorem}
      &=\sum_{|\alpha|=m}\binom{|\alpha|}{\alpha}D^\alpha x^\alpha f(tx)\\
      &=\sum_{|\alpha|=m}\binom{|\alpha|}{\alpha}D^\alpha f(tx)x^\alpha\\
      &=\sum_{|\alpha|=m}\frac{m!}{\alpha!}D^\alpha f(tx)x^\alpha
    \end{align*}
    as desired.
  \end{subproof}
  Now, applying Taylor's formula in \(1\) variable to \(g(t)\)
  \begin{align*}
    g(t)
    &=\sum_{i=0}^k\frac{g^{(i)}(0)}{i!}t^i+\rmR_k(g)\\
    &=\sum_{i=0}^k\frac{1}{i!}\sum_{|\alpha|=i}\frac{i!}{\alpha!}D^\alpha
      f(tx)x^\alpha+\rmR_k(g)\\
    &=\sum_{i=0}^k\sum_{|\alpha|=i}\frac{1}{\alpha!}D^\alpha f(0)x^\alpha t^i+\rmR_k(g)\\
    &=\sum_{|\alpha|\leq k}\frac{1}{\alpha!}D^\alpha f(0)x^\alpha t^i+\rmR_k(g)
  \end{align*}
  and evaluating at \(t=1\) we have
  \begin{align*}
    g(1)
    &=\sum_{|\alpha|\leq k}\frac{1}{\alpha!}D^\alpha f(0)x^\alpha t^i\\
    &=\sum_{|\alpha|\leq k}\frac{1}{\alpha!}D^\alpha
      f(0)x^\alpha+\rmR_k(g)\\
    &=\sum_{|\alpha|\leq k}\frac{1}{\alpha!}D^\alpha
      f(0)x^\alpha+\rmR_k(g)
      \intertext{where the remainder \(\rmR_k(g)\asymp |x|^{k+1}\)}
    &=\sum_{|\alpha|\leq k}\frac{1}{\alpha!}D^\alpha
      f(0)x^\alpha+\rmR_k\bigl(|x|^{k+1}\bigr).
  \end{align*}
\end{solution}
\newpage

\begin{problem}
  Write down the characteristic equation for the p.d.e.\@
  \[
    \label{eq:1:1}
    \tag{\(*\)}
    u_t+b\cdot Du=f
  \]
  on \(\bbR^n\times(0,\infty)\), where \(b\in\bbR^n\). Using the
  characteristic equation, solve \eqref{eq:1:1} subject to the initial
  condition
  \[
    u=g
  \]
  on \(\bbR^n\times\{t=0\}\). Make sure the answer agrees with formula (5)
  in \S 2.1.2 of [E].
\end{problem}
\begin{solution}

\end{solution}
\newpage

\begin{problem}
  Solve using the characteristics:
  \begin{enumerate}[label=(\alph*)]
  \item \(x_1^2u_{x_1}+x_2^2u_{x_2}=u^2\), \(u=1\) on the line
    \(x_2=2x_1\).
  \item \(uu_{x_1}+u_{x_2}=1\), \(u(x_1,x_2)=x_1/2\).
  \item \(x_1u_{x_1}+2x_2u_{x_2}+u_{x_3}=3u\),
    \(u(x_1,x_2,0)=g(x_1,x_2)\).
  \end{enumerate}
\end{problem}
\begin{solution}

\end{solution}
\newpage

\begin{problem}
  For the equation
  \[
    u=x_1u_{x_1}+x_2u_{x_2}
    +\frac{1}{2}\bigl(u_{x_1}^2+u_{x_2}^2\bigr)
  \]
  find a solution with \(u(x_1,0)=\bigl(1-x_1^2\bigr)/2\).
\end{problem}
\begin{solution}

\end{%solution

  %%% Local Variables:
  %%% mode: latex
  %%% TeX-master: "../MA523-Current-HW"
  %%% End:
