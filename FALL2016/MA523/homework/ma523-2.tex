\begin{problem}
  Verify assertion (36) in [E, \S 3.2.3], that when \(\Gamma\) is not flat
  near \(x^0\) the noncharacteristic condition is
  \[
    D_pF(p^0,z^0,x^0)\cdot \nu(x^0)\neq 0.
  \]
  (Here \(\nu(x^0)\) denotes the normal to the hypersurface \(\Gamma\) at
  \(x^0\)).
\end{problem}
\begin{solution}
  First, note that the condition
  \begin{equation}
    \label{eq:2-1}
    D_pF(p^0,z^0,x^0)\cdot \nu(x^0)\neq 0
  \end{equation}
  reduces to the standard noncharacteristic boundary condition if
  \(\Gamma\) is flat near \(x^0\) because in such case we have
  \(\nu(x^0)=(0,\dotsc,0,1)\) so
  \begin{align*}
    0&\neq D_pF(p^0,z^0,x^0)\cdot (0,\dotsc,0,1)\\
     &=F_{p_n}(p^0,z^0,x^0).
  \end{align*}

  We shall verify the noncharacteristic condition \eqref{eq:2-1} by first
  flattening the boundary near \(x^0\) and then applying the
  noncharacteristic boundary conditions to the flattened region. Make a
  change of variables \((y_1,\dotsc,y_n)=\bfy(x_1,\dotsc,x_n)\) where
  \(\bfy\colon\bbR^n\to\bbR^n\) is the change of coordinates
  \[
    \left\{
      \begin{aligned}
        y_1&=x_1,\\
        &\vdotswithin{=}\\
        y_{n-1}&=x_{n-1},\\
        y_n&=x_n-\varphi(x_1,\dotsc,x_{n-1}),
      \end{aligned}
    \right.
  \]
  with \(\varphi\) a sufficiently regular map
  \(\varphi\colon\bbR^{n-1}\to\bbR\). Let \(\bfx=\bfy^{-1}\). Now, note
  that \(y^0=y(x_1^0,\dotsc,x_n^0)=(y_1,\dotsc,y_{n-1},0)\) and hence
  \(\Delta=\bfy(\Gamma)\) is flat near \(y^0\) so we can apply the standard
  noncharacteristic boundary conditions on the transformed PDE,
  \[
    0\neq F_{u_n}\bigl(Du(\bfx(y^0)),u(\bfx(y^0)),\bfx(y^0)\bigr).
  \]
  First, consider the gradient \(D(\bfx(y))\). Looking at the \(i\)-th
  coordinate of this function, by the chain rule, we have
  \begin{align*}
    u_{x_i}(\bfx(y^0))
    &=\sum_{j=1}^n u_{x_i}\frac{\partial x_i}{\partial y_j}\\
    &=u_{x_i}(x^0)+u_{x_n}\varphi_{y_i}(y^0),\\
    u_{x_n}(\bfx(y^0))
    &=u_{x_n}(x^0).
  \end{align*}
\end{solution}
\newpage

\begin{problem}
  Show that the solution of the quasilinear PDE
  \[
    u_t+a(u)u_x=0
  \]
  with initial conditions \(u(x,0)=g(x)\) is given implicitly by
  \[
    u=g(x-a(u)t).
  \]
  Show that the solution develops a shock (becomes singular) for some
  \(t>0\), unless \(a(g(x))\) is a nondecreasing function of
  \(x\).
\end{problem}
\begin{solution}

\end{solution}
\newpage

\begin{problem}
  Show that the function \(u(x,t)\) defined by \(t\geq 0\) by
  \[
    u(x,t)=
    \begin{cases}
      \displaystyle-\frac{2}{3}\Bigl(t+\sqrt{3x+t^2}\Bigr)
      &\text{for \(4x+t^2>0\)}\\
      0
      &\text{for \(4x+t^2<0\)}
    \end{cases}
  \]
  is an (unbounded) entropy solution of the conservation law
  \(u_t+(u^2/2)_x=0\) (\emph{inviscid Burger's equation}).
\end{problem}
\begin{solution}
\end{solution}

%%% Local Variables:
%%% mode: latex
%%% TeX-master: "../MA523-Current-HW"
%%% End:
