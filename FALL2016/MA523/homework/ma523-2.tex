\begin{problem}
  Verify assertion (36) in [E, \S 3.2.3], that when \(\Gamma\) is not flat
  near \(x^0\) the noncharacteristic condition is
  \[
    D_pF(p^0,z^0,x^0)\cdot \nu(x^0)\neq 0.
  \]
  (Here \(\nu(x^0)\) denotes the normal to the hypersurface \(\Gamma\) at
  \(x^0\)).
\end{problem}
\begin{solution}
  First, note that the condition
  \begin{equation}
    \label{eq:2-1}
    D_pF(p^0,z^0,x^0)\cdot \nu(x^0)\neq 0
  \end{equation}
  reduces to the standard noncharacteristic boundary condition if
  \(\Gamma\) is flat near \(x^0\) because in such case we have
  \(\nu(x^0)=(0,\dotsc,0,1)\) so
  \begin{align*}
    0&\neq D_pF(p^0,z^0,x^0)\cdot (0,\dotsc,0,1)\\
     &=F_{p_n}(p^0,z^0,x^0).
  \end{align*}

  We shall verify the noncharacteristic condition \eqref{eq:2-1} by first
  flattening the boundary near \(x^0\) and then applying the
  noncharacteristic boundary conditions to the flattened region. Assuming
  some regularity near \(x^0\), \(\Gamma\) is a hypersurface of dimension
  \(n-1\). Assuming some degree of regularity near \(x^0\), e.g., that the
  boundary be smooth, we may express \(\Gamma\) near \(x^0\) as the graph
  of a smooth function \(f\colon\bbR^{n-1}\to\bbR\), i.e., \(x=(y,f(y))\)
  on \(\Gamma\) and \(x_n\geq f(y)\) after reorienting the coordinate
  axes. Then we flatten out \(\Gamma\) via the map
  \(\Phi(x)\colon\bbR^n\to\bbR^n\) given by
  \[
    \left\{
      \begin{aligned}
        y_1&=x_1=\Phi^1(x),\\
        &\vdotswithin{=}\\
        y_{n-1}&=x_{n-1}=\Phi^{n-1}(x),\\
        y_n&=x_n-f(x_1,\dotsc,x_{n-1})=\Phi^n(x)
      \end{aligned}
    \right.
  \]
  and write \(y=\Phi(x)\).
\end{solution}
\newpage
% We shall verify the noncharacteristic condition \eqref{eq:2-1} by first
% flattening the boundary near \(x^0\) and then applying the
% noncharacteristic boundary conditions to the flattened region. Assuming
% some regularity near \(x^0\), \(\Gamma\) is a hypersurface of dimension
% \(n-1\).

% Make a change of variables \((y_1,\dotsc,y_n)=\bfy(x_1,\dotsc,x_n)\)
% where \(\bfy\colon\bbR^n\to\bbR^n\) is the change of coordinates
%   \[
%     \left\{
%       \begin{aligned}
%         y_1&=x_1,\\
%         &\vdotswithin{=}\\
%         y_{n-1}&=x_{n-1},\\
%         y_n&=x_n-\varphi(x_1,\dotsc,x_{n-1}),
%       \end{aligned}
%     \right.
%   \]
%   with \(\varphi\) a sufficiently regular map
%   \(\varphi\colon\bbR^{n-1}\to\bbR\) such that
%   \(x_n^0=\varphi(x_1^0,\dotsc,x_{n-1}^0)\). Let \(\bfx=\bfy^{-1}\). Now,
%   note that \(y^0=y(x_1^0,\dotsc,x_n^0)=(y_1,\dotsc,y_{n-1},0)\) and hence
%   \(\Delta=\bfy(\Gamma)\) is flat near \(y^0\) so we can apply the standard
%   noncharacteristic boundary conditions on the transformed PDE,
%   \begin{equation}
%     \label{eq:2-2}
%     0\neq F_{u_{y_n}}\bigl(Du(\bfx(y^0)),u(\bfx(y^0)),\bfx(y^0)\bigr).
%   \end{equation}
%   First, consider the gradient \(D(\bfx(y))\). Looking at the \(i\)-th
%   coordinate of this function, by the chain rule, we have
%   \begin{align*}
%     u_{y_i}(\bfx(y))
%     &=\sum_{j=1}^n u_{x_j}(\bfx(y))\frac{\partial x_j}{\partial y_i}\\
%     &=u_{x_i}(\bfx(y))+u_{x_n}(\bfx(y))\varphi_{y_i}(y_1,\dotsc,y_{n-1}),\\
%     u_{y_n}(\bfx(y))
%     &=\sum_{j=1}^n u_{x_j}(\bfx(y))\frac{\partial x_j}{\partial y_n}\\
%     &=u_{x_n}(\bfx(y)).
%   \end{align*}
%   Thus,
%   \[
%     u_{y_i}(\bfx(y))=u_{x_i}(\bfx(y))+u_{y_n}(\bfx(y))\varphi_{y_i}(y_1,\dotsc,y_{n-1}).
%   \]
%   Now, evaluating the derivative in \eqref{eq:2-2}, we have
%   \begin{align*}
%     0
%     &\neq F_{u_{y_n}}\bigl(Du(\bfx(y^0)),u(\bfx(y^0)),\bfx(y^0)\bigr)\\
%     &=F_{u_{y_n}}\bigl(u_{x_1}(\bfx(y^0))+u_{y_n}(\bfx(y^0))\varphi_{y_1}(y^0),\dotsc,u_{y_n}(\bfx(y^0)),u(\bfx(y^0)),\bfx(y^0)\bigr)
%     \intertext{which, by the chain rule, becomes}
%     &=F_{u_{y_1}}(Du(\bfx(y^0)),u(\bfx(y^0)),y^0)\varphi_{y_1}(y^0)
%       +\dotsb+F_{u_{y_n}}(Du(\bfx(y^0)),u(\bfx(y^0)),y^0)\varphi_{y_1}(y^0)\\
%     &=D_{Du} F\bigl(Du(\bfx(y^0)),u(\bfx(y^0)),\bfx(y^0)\bigr)\cdot(D\varphi,1)\\
%     &=D_p F(p^0,z^0,y^0)\cdot \nu(x^0)
%   \end{align*}
%   where \(D\varphi,1\) is the normal to \(\Gamma\) at \(x^0\).

\begin{problem}
  Show that the solution of the quasilinear PDE
  \[
    u_t+a(u)u_x=0
  \]
  with initial conditions \(u(x,0)=g(x)\) is given implicitly by
  \[
    u=g(x-a(u)t).
  \]
  Show that the solution develops a shock (becomes singular) for some
  \(t>0\), unless \(a(g(x))\) is a nondecreasing function of
  \(x\).
\end{problem}
\begin{solution}

\end{solution}
\newpage

\begin{problem}
  Show that the function \(u(x,t)\) defined by \(t\geq 0\) by
  \[
    u(x,t)=
    \begin{cases}
      \displaystyle-\frac{2}{3}\Bigl(t+\sqrt{3x+t^2}\Bigr)
      &\text{for \(4x+t^2>0\)}\\
      0
      &\text{for \(4x+t^2<0\)}
    \end{cases}
  \]
  is an (unbounded) entropy solution of the conservation law
  \(u_t+(u^2/2)_x=0\) (\emph{inviscid Burger's equation}).
\end{problem}
\begin{solution}
\end{solution}

%%% Local Variables:
%%% mode: latex
%%% TeX-master: "../MA523-Current-HW"
%%% End:
