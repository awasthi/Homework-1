\begin{problem}
  Verify assertion (36) in [E, \S 3.2.3], that when \(\Gamma\) is not flat
  near \(x^0\) the noncharacteristic condition is
  \[
    D_pF(p^0,z^0,x^0)\cdot \nu(x^0)\neq 0.
  \]
  (Here \(\nu(x^0)\) denotes the normal to the hypersurface \(\Gamma\) at
  \(x^0\)).
\end{problem}
\begin{solution}
  First, note that the condition
  \begin{equation}
    \label{eq:2-1}
    D_pF(p^0,z^0,x^0)\cdot \nu(x^0)\neq 0
  \end{equation}
  reduces to the standard noncharacteristic boundary condition if
  \(\Gamma\) is flat near \(x^0\) because in such case we have
  \(\nu(x^0)=(0,\dotsc,0,1)\) so
  \begin{align*}
    0&\neq D_pF(p^0,z^0,x^0)\cdot (0,\dotsc,0,1)\\
     &=F_{p_n}(p^0,z^0,x^0).
  \end{align*}

  We shall verify the noncharacteristic condition \eqref{eq:2-1} by first
  flattening the boundary near \(x^0\) and then applying the
  noncharacteristic boundary conditions to the flattened region. Assuming
  some degree of regularity near \(x^0\), e.g., that the boundary of \(U\)
  be smooth, we may express \(\Gamma\) near \(x^0\) as the graph of a
  smooth function \(f\colon\bbR^{n-1}\to\bbR\), i.e.,
  \(x=(x_1,\dotsc,x_{n-1},f(x_1,\dotsc,x_{n-1}))\) on \(\Gamma\) and
  \(x_n\geq f(y)\) after reorienting the coordinate axes. Then we flatten
  out \(\Gamma\) via the map \(\Phi(x)\colon\bbR^n\to\bbR^n\) given by
  \[
    \left\{
      \begin{aligned}
        y_1&=x_1=\Phi^1(x),\\
        &\vdotswithin{=}\\
        y_{n-1}&=x_{n-1}=\Phi^{n-1}(x),\\
        y_n&=x_n-f(x_1,\dotsc,x_{n-1})=\Phi^n(x)
      \end{aligned}
    \right.
  \]
  and write \(y=\Phi(x)\). Let \(\Psi=\Phi^{-1}\) and rewrite our PDE \(F\)
  in terms of \(y\) as follows,
  \begin{equation}
    \label{eq:2-2}
    0=F\bigl(Du(\Psi(y)),u(\Psi(y)),\Psi(y)\bigr).
  \end{equation}
  Since \(\Delta=\Phi(\Gamma)\) is flat near
  \(y^0=\Phi(x^0)=(y_1^0,\dotsc,y_{n-1}^0,0)\), we may apply the standard
  noncharacteristic condition on \eqref{eq:2-2} and get
  \[
    0\neq F_{u_{y_n}}\bigl(Du(\Psi(y^0)),u(\Psi(y^0)),\Psi(y^0)\bigr).
  \]
  Before we move on to finding an expression for this derivative, let us
  consider the gradient \(Du(\Psi(y))\). By the chain rule, we have
  \begin{align*}
    u_{y_i}(\Psi(y))
    &=\sum_{j=1}^n u_{x_j}(\Psi(y))\frac{\partial x_j}{\partial y_i}\\
    &=u_{x_i}(\Psi(y))+u_{x_n}(\Psi(y))f_{y_i}(y_1,\dotsc,y_{n-1}),\\
    u_{y_n}(\Psi(y))
    &=\sum_{j=1}^n u_{x_j}(\Psi(y))\frac{\partial x_j}{\partial y_n}\\
    &=u_{x_n}(\Psi(y)),
  \end{align*}
  Then, substituting \(u_{y_n}\) for \(u_{x_n}\), we have
  \[
    u_{y_i}(\Psi(y))=u_{x_i}(\Psi(y))+u_{y_n}(\Psi(y))f_{y_i}(y_1,\dotsc,y_{n-1}),\\
  \]
  Now, by the chain rule on \eqref{eq:2-2}, we have
  \begin{align*}
    0&\neq F_{u_{y_n}}\bigl(Du(\Psi(y^0)),u(\Psi(y^0)),\Psi(y^0)\bigr)\\
     &=F_{u_{y_n}}
       \bigl(
       u_{x_1}+u_{y_n}f_{y_1},\dotsc,
       u_{x_{n-1}}+u_{y_n}f_{y_{n-1}},u_{y_n},z^0,x^0
       \bigr)\\
     &=F_{u_{x_1}}f_{y_1}+\dotsb+F_{u_{x_{n-1}}}f_{y_{n-1}}+F_{u_{x_n}}\\
     &=D_pF(p^0,z^0,x^0)\cdot(Df(x^0),1)\\
     &=D_pF(p^0,z^0,x^0)\cdot\nu(x^0),
  \end{align*}
  as we set out to show.
\end{solution}
\newpage

\begin{problem}
  Show that the solution of the quasilinear PDE
  \[
    u_t+a(u)u_x=0
  \]
  with initial conditions \(u(x,0)=g(x)\) is given implicitly by
  \[
    u=g(x-a(u)t).
  \]
  Show that the solution develops a shock (becomes singular) for some
  \(t>0\), unless \(a(g(x))\) is a nondecreasing function of
  \(x\).
\end{problem}
\begin{solution}
  The characteristic ODEs of this PDE are
  \begin{equation}
    \label{eq:2-3}
    \dot t=1,\qquad \dot x=a(z),\qquad \dot z=0.
  \end{equation}
  with initial conditions \(t^0=0\), \(x^0=x(0)\) and \(z(x^0,0)=g(x^0)\)
  with \((x^0,0)\in\bbR^n\times(0,\infty)\). Hence, we have
  \[
    t(s)=s,\qquad x(s)=a(g(x^0))s+x^0,\qquad z(s)=g(x^0).
  \]
  Thus, solving for \(x^0\) and \(s\) in terms of \(t\), \(x\) and \(z\),
  we have
  \begin{align*}
    x&=a(g(x^0))s+x^0\\
     &=a(z)t+x^0,
    \shortintertext{so, moving \(x^0\) to the left-hand side}
    x^0
     &=x-a(z)t
       \shortintertext{hence,}
       z&=g(x-a(z)t),
          \shortintertext{i.e.,}
          u&=g(x-a(u)t),
  \end{align*}
  as desired.

  For the latter half of the problem,
\end{solution}
\newpage

\begin{problem}
  Show that the function \(u(x,t)\) defined by \(t\geq 0\) by
  \[
    u(x,t)=
    \begin{cases}
      \displaystyle-\frac{2}{3}\Bigl(t+\sqrt{3x+t^2}\Bigr)
      &\text{for \(4x+t^2>0\)}\\
      0
      &\text{for \(4x+t^2<0\)}
    \end{cases}
  \]
  is an (unbounded) entropy solution of the conservation law
  \(u_t+(u^2/2)_x=0\) (\emph{inviscid Burger's equation}).
\end{problem}
\begin{solution}
\end{solution}

%%% Local Variables:
%%% mode: latex
%%% TeX-master: "../MA523-Current-HW"
%%% End:
