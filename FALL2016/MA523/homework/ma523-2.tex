\begin{problem}
  Verify assertion (36) in [E, \S 3.2.3], that when \(\Gamma\) is not flat
  near \(x^0\) the noncharacteristic condition is
  \[
    D_pF(p^0,z^0,x^0)\cdot \nu(x^0)\neq 0.
  \]
  (Here \(\nu(x^0)\) denotes the normal to the hypersurface \(\Gamma\) at
  \(x^0\)).
\end{problem}
\begin{solution}
  First, note that the condition
  \begin{equation}
    \label{eq:2-1}
    D_pF(p^0,z^0,x^0)\cdot \nu(x^0)\neq 0
  \end{equation}
  reduces to the noncharacteristic boundary condition when \(\Gamma\) is
  flat near \(x^0\) since \(\nu(x^0)=(0,\dotsc,0,-1)\) giving us
  \begin{align*}
    0&\neq D_pF(p^0,z^0,x^0)\cdot (0,\dotsc,0,-1)\\
     &=-F_{p_n}(p^0,z^0,x^0)\\
     &=F_{p_n}(p^0,z^0,x^0).
  \end{align*}

  To show \eqref{eq:2-1}, we will straighten the boundary near \(x^0\) and
  apply the noncharacteristic boundary conditions. Let
  \(\Phi,\Psi\colon\bbR^n\to\bbR^n\) be smooth maps such that
  \(\Psi=\Phi^{-1}\) and \(\Phi\) straightens out \(\partial U\) near
  \(x^0\). Then, setting \(y^0\defeq (y_1,\dotsc,y_{n-1},0)=\Phi(x^0)\) and
  \(v(y)=u(\Psi(y))\), our PDE becomes
  \[
    0=F\bigl(Dv(y)D\Phi(\Psi(y)),v(y),\Psi(y)\bigr).
  \]


  From here we follow the proof of Lemma 1 in [E, \S 3.2.3]. Let
  \(G\colon\bbR^n\times\bbR^n\to\bbR^n\) be the map given by
  \[
    \left\{
      \begin{aligned}
        G^i(p,y)&=p_i-g_{x_i}(y),\\
        G^n(p,y)&=F(p,g(y),y).
      \end{aligned}
    \right.
  \]
\end{solution}
\newpage

\begin{problem}
  Show that the solution of the quasilinear PDE
  \[
    u_t+a(u)u_x=0
  \]
  with initial conditions \(u(x,0)=g(x)\) is given implicitly by
  \[
    u=g(x-a(u)t).
  \]
  Show that the solution develops a shock (becomes singular) for some
  \(t>0\), unless \(a(g(x))\) is a nondecreasing function of
  \(x\).
\end{problem}
\begin{solution}

\end{solution}
\newpage

\begin{problem}
  Show that the function \(u(x,t)\) defined by \(t\geq 0\) by
  \[
    u(x,t)=
    \begin{cases}
      \displaystyle-\frac{2}{3}\Bigl(t+\sqrt{3x+t^2}\Bigr)
      &\text{for \(4x+t^2>0\)}\\
      0
      &\text{for \(4x+t^2<0\)}
    \end{cases}
  \]
  is an (unbounded) entropy solution of the conservation law
  \(u_t+(u^2/2)_x=0\) (\emph{inviscid Burger's equation}).
\end{problem}
\begin{solution}
\end{solution}

%%% Local Variables:
%%% mode: latex
%%% TeX-master: "../MA523-Current-HW"
%%% End:
