\subsection{Homework 8}
\begin{problem}
  Show that the function
  \[
    u(x,t)\defeq\sum_{k=-\infty}^\infty (-1)^k\Phi(x-2k,t)
  \]
  where
  \[
    \Phi(x,t)=\frac{\rme^{-\frac{x^2}{4t}}}{\sqrt{4\pi t}}
  \]
  is positive for \(|x|<1\), \(t>0\).
  \\\\
  \emph{Hint:} Show that \(u\) satisfies \(u_t=u_{xx}\) for \(t>0\),
  \[
    \left\{
      \begin{aligned}
        &u=0&&\text{on \(\{\,|x|=1\,\}\times\{\,t\geq 0\,\}\),}\\
        &u=\delta_0&&\text{on \(\{\,|x|\leq 1\,\}\times\{\,t=0\,\}\).}
      \end{aligned}
    \right.
  \]
  Then, carefully apply the maximum/minimum principle in a domain
  \(\{\,|x|\leq 1\,\}\times\{\,\varepsilon\leq t\leq T\,\}\) for small
  \(\varepsilon>0\) and large \(T>0\) pass to the limit as
  \(\varepsilon\to 0+\) and \(T\to\infty\).
\end{problem}
\begin{solution*}
  Taking the hint, let us verify that \(u_t=u_{xx}\), for \(t>0\). By
  direct computation, we have
  \begin{align*}
    \Phi_x(x,t)
    &=\frac{\partial}{\partial x}
      \left(\frac{\rme^{-\frac{x^2}{4t}}}{\sqrt{4\pi t}}\right)
    &\Phi_{xx}(x,t)
    &=\frac{\partial}{\partial x}
      \left(-\frac{x\rme^{-\frac{x^2}{4t}}}{2\sqrt{4\pi}t^{\frac{3}{2}}}\right)\\
    &=-\frac{x\rme^{-\frac{x^2}{4t}}}{2\sqrt{4\pi}t^{\frac{3}{2}}},
    &&=\frac{x^2\rme^{-\frac{x^2}{4t}}}{4\sqrt{4\pi}t^{\frac{5}{2}}}
      -\frac{\rme^{-\frac{x^2}{4t}}}{2\sqrt{4\pi}t^{\frac{3}{2}}}
    \\
    &&&=\frac{(x^2-2t)\rme^{-\frac{x^2}{4t}}}{4\sqrt{4\pi}t^{\frac{5}{2}}},
  \end{align*}
  and
  \begin{align*}
    \Phi_t(x,t)
    &=\frac{\partial}{\partial t}
      \left(\frac{\rme^{-\frac{x^2}{4t}}}{\sqrt{4\pi t}}\right)\\
    &=\frac{x^2\rme^{-\frac{x^2}{4t}}}{4\sqrt{4\pi}t^{\frac{5}{2}}}
      -\frac{\rme^{-\frac{x^2}{4t}}}{2\sqrt{4\pi}t^{\frac{3}{2}}}\\
    &=\frac{(x^2-2t)\rme^{-\frac{x^2}{4t}}}{4\sqrt{4\pi}t^{\frac{5}{2}}}.
  \end{align*}
  Since \(\Phi_t=\Phi_{xx}\) it follows that \(u_t=u_{xx}\) (assuming
  uniform convergence of \(u\)).

  Next we show that \(u=0\) on \(\{\,|x|=1\,\}\times\{\,t\geq 0\,\}\) and
  \(u=\delta_0\) on \(\{\,|x|=1\,\}\times\{\,t=0\,\}\). To show \(u=0\) fix
  a \(t\geq 0\) and, after relabeling if necessary, assume that \(x=1\)
  which gives us
  \begin{align*}
    u(1,t)
    &=\sum_{k=-\infty}^\infty(-1)^k\frac{\rme^{-\frac{(1-2k)^2}{4t}}}{\sqrt{4\pi
      t}}\\
    &=\frac{1}{\sqrt{4\pi t}}%
      \left(\dotsb-\rme^{-\frac{9}{4t}}%
      +\rme^{-\frac{1}{4t}}-\rme^{-\frac{1}{4t}}%
      +\rme^{-\frac{9}{4t}}+\dotsb\right)\\
    &=0.
  \end{align*}
  Similarly for \(u(-1,t)=0\).

  For \(u(|x|\leq 1,0)\), we have a
  \begin{align*}
    u(|x|\leq 1,0)
    &=\sum_{k=-\infty}^\infty(-1)^k\lim_{t\to 0+}
      \left[\rme^{-\frac{(x-2k)^2}{4t}}/\sqrt{4\pi t}\right]\\
    &=\sum_{k=-\infty}^\infty(-1)^k\delta_0(x-2k)\\
    &=\delta_0(x)
  \end{align*}
  since \(|x|\leq 1\) and values \(\delta_0\) is zero for values \(x-2k\)
  outside of the interval \([-1,1]\).

  At last we show that \(u\) is positive for \(|x|<1\), \(t>0\). Seeking a
  contradiction, suppose \(u\) is negative on some point \((x_0,t_0)\) in
  \(\{\,|x|<1\,\}\times\{\,\varepsilon\leq t\leq T\,\}\). Then by the
  minimum principle, \(u\) achieves its minimum somewhere on the bottom
  boundary \(\{\,|x|=1\,\}\times\{\,t=\varepsilon\,\}\). Therefore, there
  exists a sequence \((x_n,t_n{+})\to (x,0)\), where \(|x_n|,|x|<1\),
  such that \(u(x,0)<0\). However, we have shown above that
  \(u(x,0)=\delta_0(x)\) for \(|x|<1\); i.e., either \(u(x,0)=0\) or
  \(u(x,0)=+\infty\). This is a contradiction. Therefore, it must be the
  case that \(u\geq 0\) for \(|x|<1\), \(t>0\).
\end{solution*}

\begin{problem}[Tikhonov's example]
  Let
  \[
    g(t)\defeq
    \begin{cases}
      \rme^{-t^{-2}}&t>0,\\
      0&t\leq 0.
    \end{cases}
  \]
  Then \(g\in C^\infty(\R)\) and we define
  \[
    u(x,t)\defeq\sum_{k=0}^\infty\frac{g^{(k)}(t)}{(2k)!}x^{2k}.
  \]
  Assuming that the series is convergent, show that \(u(x,t)\) solves the
  heat equation in \(\R\times(0,\infty)\) with the initial condition
  \(u(x,0)=0\), \(x\in\R\). Why doesn't this contradict the uniqueness
  theorem for the initial value problem?
\end{problem}
\begin{solution*}
  Let \(u\) be as above. Then
  \begin{align*}
    u_t(x,t)
    &=\frac{\partial}{\partial t}
      \left(
      \sum_{k=0}^\infty\frac{g^{(k)}(t)}{(2k)!}x^{2k}
      \right)\\
    &=\sum_{k=0}^\infty \frac{g^{(k+1)}(t)}{(2k)!}x^{2k}\\
    &=\sum_{k=2}^\infty \frac{g^{(k)}(t)}{(2k-2)!}x^{2k-2},
  \end{align*}
  and
  \begin{align*}
    u_x(x,t)
    &=\frac{\partial}{\partial x}
      \left(
      \sum_{k=0}^\infty\frac{g^{(k)}(t)}{(2k)!}x^{2k}
      \right)
    &u_{xx}(x,t)
    &=\frac{\partial}{\partial x}
      \left(
      \sum_{k=0}^\infty\frac{g^{(k)}(t)}{(2k-1)!}x^{2k-1}
      \right)\\
    &=\sum_{k=0}^\infty\frac{g^{(k)}(t)}{(2k)!} 2kx^{2k-1}
    &&=\sum_{k=1}^\infty \frac{g^{(k)}(t)}{(2k-1)!}(2k-1)x^{2k-2}
       +\frac{\partial}{\partial x}g^{(0)}(t)\\
    &=\sum_{k=1}^\infty\frac{g^{(k)}(t)}{(2k-1)!}x^{2k-1},
    &&=\sum_{k=2}^\infty\frac{g^{(k)}(t)}{(2k-2)!}x^{2k-2}.
  \end{align*}
  Thus, \(u_t-\Lap u=0\); i.e., \(u\) solves the heat equation. As this
  example shows, unless some assumptions on \(u\) such as subexponential
  (cf.\@ [E \S 2.3], Theorem 7) growth is assumed.
\end{solution*}

\begin{problem}
  Evaluate the integral
  \[
    \int_{-\infty}^\infty \cos(ax)\rme^{-x^2}\diff x,\qquad (a>0).
  \]
  \\\\
  \emph{Hint:} Use the separation of variables to find the solution of the
  corresponding initial-value problem for the heat equation.
\end{problem}
\begin{solution*}
  By separation of variables,
  \[
    u(x,t)=\cos(ax)\rme^{-a^2t}
  \]
  is a solution to the Cauchy problem
  \[
    \left\{
      \begin{aligned}
        &u_t-\Lap u=0&&\text{in \(\R\times(0,\infty)\),}\\
        &u=\cos(ax)&&\text{on \(\R\times\{\,t=0\,\}\).}
      \end{aligned}
    \right.
  \]
  However, the convolution
  \[
    u(x,t)=%
    \frac{1}{\sqrt{4\pi t}}\int_{-\infty}^\infty%
    \cos(a y)\rme^{-\frac{|x-y|^2}{4t}}\diff y
  \]
  is also a solution to the Cauchy problem. Now note that
  \begin{align*}
    \int_{-\infty}^\infty\cos(ay)\rme^{-y^2}\diff y
    &=\sqrt{\pi}\cdot u(0,\frac{1}{4})\\
    &=\sqrt{\pi}\rme^{-\frac{a^2}{4}}.\qedhere
  \end{align*}
\end{solution*}

%%% Local Variables:
%%% mode: latex
%%% TeX-master: "../MA523-HW-ALL"
%%% End:
