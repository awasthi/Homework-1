\begin{problem}
  Prove that Laplace's equation \(\Delta u=0\) is rotation invariant; that
  is, if \(O\) is an orthogonal \(n\times n\) matrix and we define
  \(v(x)\defeq u(Ox)\), \(x\in\bbR^n\), then \(\Delta v=0\).
\end{problem}
\begin{solution}
  Let
  \[
    O=
    \begin{bmatrix}
      a_{11}&a_{12}&\dotsc&a_{1n}\\
      \vdots&\vdots&\ddots&\vdots\\
      a_{n1}&a_{n2}&\cdots&a_{nn}
    \end{bmatrix}
  \]
  be an orthogonal \(n\times n\) matrix. We will show that \(\Delta v=0\),
  where \(v(x)=u(Ox)\).

  First, let us compute the gradient of \(v\):
  \begin{align*}
    Dv(x)
    &=Du(Ox)\\
    &=Du\bigl(a_{11}x_1+\dotsb+a_{1n}x_n,\dotsc,a_{n1}x_1+\dotsb+a_{nn}x_n\bigr)\\
    &=\left(\sum\nolimits_{j=1}^na_{j1}u_{x_j},%
      \dotsc,%
      \sum\nolimits_{j=1}^n a_{jn}u_{x_j}\right)\\
    &=O^\rmT Du(x).
  \end{align*}

  Lastly, we compute the divergence of \(Dv\):
  \begin{align*}
    \Delta v(x)
    &=\Div Dv(x)\\
    &=\Div%
      \left(\sum\nolimits_{j=1}^na_{j1}u_{x_j},%
      \dotsc,%
      \sum\nolimits_{j=1}^n a_{jn}u_{x_j}\right).
  \end{align*}
  Here the partial derivatives become unwieldy so we will first examine the
  partial \(\frac{\partial}{\partial x_1}\) of the first term and proceed
  from there. In this case,
  \begin{align*}
    \frac{\partial}{\partial x_1}\sum_{j=1}^n a_{j1}u_{x_j}
    &=a_{11}\tfrac{\partial}{\partial
      x_1}u_{x_1}+a_{12}\tfrac{\partial}{\partial
      x_1}u_{x_2}+\dotsb+a_{1n}\tfrac{\partial}{\partial x_1}u_{x_n}\\
    &=a_{11}\bigl(a_{11}u_{x_1x_1}+a_{21}u_{x_1x_2}+\dotsb+a_{n1}u_{x_1x_n}\bigr)\\
    &\phantom{{}={}}+\dotsb+a_{1n}\bigl(a_{11}u_{x_1x_n}+a_{21}u_{x_2x_n}
      +\dotsb+a_{n1}u_{x_nx_n}\bigr).
  \end{align*}

  Similarly, taking the \(k\)\textsup{th} partial of the \(k\)\textsup{th}
  entry of \(Dv\), we have
  \begin{equation}
    \label{eq:5:laplacian-k-part}
    \begin{aligned}
      \frac{\partial}{\partial x_k}\sum_{j=1}^n a_{jk}u_{x_j}
      &=a_{k1}\bigl(a_{1k}u_{x_1x_1}+\dotsb+a_{nk}u_{x_1x_n}\bigr)\\
      &\phantom{{}={}}+\dotsb+a_{kn}
      \bigl(a_{1k}u_{x_1x_n}+\dotsb+a_{nk}u_{x_nx_n}\bigr).
    \end{aligned}
  \end{equation}

  Now, since \(O\) is orthogonal, we have
  \begin{align*}
    O^\rmT O
    &=%
      \begin{bmatrix}
      a_{11}&a_{21}&\dotsc&a_{n1}\\
      \vdots&\vdots&\ddots&\vdots\\
      a_{1n}&a_{2n}&\cdots&a_{nn}
      \end{bmatrix}
      \begin{bmatrix}
      a_{11}&a_{12}&\dotsc&a_{1n}\\
      \vdots&\vdots&\ddots&\vdots\\
      a_{n1}&a_{n2}&\cdots&a_{nn}
      \end{bmatrix}\\
    &=%
      \begin{bmatrix}
        a_{11}^2+\dotsb+a_{n1}^2&a_{11}a_{12}+\dotsb+a_{n1}a_{n2}
        &\cdots&a_{11}a_{1n}+\dotsb+a_{n1}a_{nn}\\
        \vdots&\vdots&\ddots&\vdots\\
        a_{1n}a_{11}+\dotsb+a_{nn}a_{n1}&a_{1n}a_{12}+\dotsb+a_{nn}a_{n2}
        &\cdots&a_{1n}^2+\dotsb+a_{nn}^2
      \end{bmatrix}\\
    &=\begin{bmatrix}
      1&0&\cdots&0\\
      \vdots&\vdots&\ddots&0\\
      0&0&\cdots&1
      \end{bmatrix}.
  \end{align*}
  Thus, for \(1\leq k,\ell\leq n\), \(k\neq\ell\), we have
  \[
    \left\{%
      \begin{aligned}
        \sum_{j=1}^na_{jk}^2=1,\\
        \sum_{j=1}^na_{jk}a_{j\ell}=0
      \end{aligned}
    \right.
  \]
\end{solution}
\newpage

\begin{problem}
  Let \(n=2\) and \(U\) be the halfplane \(\{\,x_2>0\,\}\). Prove that
  \[
    \sup_U u=\sup_{\partial U}u
  \]
  for \(u\in C^2(U)\cap C(\bar U)\) which are harmonic in \(U\) under the
  additional assumption that \(u\) is bounded from above in \(\bar
  U\). (The additional assumption is needed to exclude examples like
  \(u=x_2\).)

  \noindent
  [\emph{Hint:} Take for \(\varepsilon>0\) the harmonic function
  \[
    u(x_1,x_2)+\varepsilon\ln\sqrt{x_1^2+(x_2+1)^2}.
  \]
  Apply the maximum principle to a region
  \(\bigl\{\,x_1^2+(x_2+1)^2<a_2,x_2>0\,\bigr\}\) with large \(a\). Let
  \(\varepsilon\to 0\).]
\end{problem}
\begin{solution}
\end{solution}
\newpage

\begin{problem}
  Let \(U\subset\bbR^n\) be an open set. We say \(v \in C^2(U)\) is
  subharmonic if
  \[
    -\Delta v\leq 0\qquad\qquad\text{in \(U\).}
  \]
  \begin{itemize}
  \item[(a)] Let \(\varphi\colon\bbR^m\to\bbR\) be smooth and
    convex. Assume \(u^1,\dotsc,u^m\) are harmonic in \(U\) and
    \[
      v\defeq\varphi(u_1,\dotsc,u_m).
    \]
    Prove \(v\) is sub harmonic.

    \noindent
    [\emph{Hint:} Convexity for a smooth function \(\varphi(z)\) is
    equivalent to \(\sum_{j,k=1}^m\varphi_{z_j,z_k}(z)\xi_j\xi_j\geq 0\)
    for any \(\xi\in\bbR^m\).]
  \item[(b)] Prove \(v\defeq|Du|^2\) is subharmonic, whenever \(u\) is
    harmonic. (Assume that harmonic functions are \(C^\infty\).)
  \end{itemize}
\end{problem}
\begin{solution}

\end{solution}

%%% Local Variables:
%%% mode: latex
%%% TeX-master: "../MA523-HW-Current"
%%% End:
