\begin{problem}
  Consider the initial value problem
  \[
    u_t=\sin u_x;\qquad u(x,0)=\frac{\pi}{4}x.
  \]
  Verify that the assumptions of the Cauchy--Kovalevskaya theorem are
  satisfied and obtain the Taylor series of the solution about the origin.
\end{problem}
\begin{solution}
  The initial value problem certainly satisfies the assumptions of the
  Cauchy--Kovalevskaya theorem, that is, setting
  \(\bfu\defeq(u,u_x,u_t,t)\), the \(\bfB\) are all identically \(0\) and
  \(\bfc(\bfu,x)=\sin u_x(x,t)\) is analytic. Next we show that he Taylor
  series of \(u\) at \((0,0)\),
  \[
    \tilde u(x,t)=\sum_{(\alpha_1,\alpha_2)}
    \frac{u_{\alpha_1,\alpha_2}(0)}{\alpha_1!\alpha_2!} x^{\alpha_1}t^{\alpha_2},
  \]
  is a solution to our PDE.

  First, we must compute the coefficients
  \[
    \frac{u_{\alpha_1,\alpha_2}(0,0)}{\alpha_1!\alpha_2!}.
  \]
  To this end, we must find the partial derivatives
  \(u_{\alpha_1,\alpha_2}\) and potentially, relations among them which
  will help us to find these coefficients. Naïvely listing the partials
  with respect to \(t\) and \(x\), we have
  \begin{align*}
    u(0,0)&=0\\
    u_x(0,0)&=\frac{\pi}{4}\\
    u_t(0,0)&=\sin u_x(0,0)=\frac{\sqrt{2}}{2}\\
    u_{xx}(0,0)&=0\\
    u_{tx}(0,0)&=0\\
    u_{tt}(0,0)&=-\cos\bigl(u_x(0,0)\bigr)u_{xt}(0,0)=0\\
    u_{xxx}(0,0)&=0\\
    u_{ttx}(0,0)&=0,
  \end{align*}
  etc. Thus,
  \[
    \tilde u=\frac{\pi}{4}x+\frac{\sqrt{2}}{2}t.
  \]
  Plugging this equation into our PDE, we have
  \[
    \tilde u_t-\sin\tilde u_x=\frac{\sqrt{2}}{2}-\sin(\pi/4)=0,
  \]
  as desired.
\end{solution}
\newpage

\begin{problem}
  Consider the Cauchy problem for \(u(x,y)\)
  \begin{align*}
    u_y&=a(x, y, u)u_x+b(x,y,u)\\
    u(x,0)&=0
  \end{align*}
  Let \(a\) and \(b\) be analytic functions of their arguments. Assume that
  \(D^\alpha a(0,0,0)\geq 0\) and \(D^\alpha b(0,0,0)\geq 0\) for all
  \(\alpha\). (Remember by definition, if \(\alpha=0\) then
  \(D^\alpha f=f\).)
  \begin{enumerate}[label=(\alph*),noitemsep]
  \item Show that \(D^\beta u(0,0)\geq 0\) for all \(|\beta|\leq 2\).
  \item Prove that \(D^\beta u(0,0)\geq 0\) for all
    \(\beta=(\beta_1,\beta_2)\). (\emph{Hint:} Argue as in the proof of the
    Cauchy--Kovalevskaya theorem; i.e., use induction in \(\beta_2\))
  \end{enumerate}
\end{problem}
\begin{solution}
  Write
  \[
    a(x,y,u)=\sum_{\alpha,\beta,\gamma} a_{\alpha,\beta,\gamma}x^\alpha
    y^\beta u^\gamma,%
    \qquad%
    b(x,y,u)=\sum_{\alpha,\beta,\gamma} b_{\alpha,\beta,\gamma}x^\alpha
    y^\beta u^\gamma
  \]
  where the right-hand side of the expressions above converge to the
  left-hand side for \(|x|+|y|+|u|<r\) for some sufficiently small \(r\).

  For part (a) we show this explicitly by considering all cases. The case
  \(\beta=(0,0)\) is obvious as are the cases \(\beta=(0,1)\) and
  \(\beta=(1,0)\) since \(u_x(0,0)=0\) and
  \begin{align*}
    u_y(0,0)
    &=a\bigl(0,0,u(0,0)\bigr)u_x(0,0)+b\bigl( 0,0,u(0,0) \bigr)\\
    &=a(0,0,0)u_x(0,0)+b(0,0,0)\\
    &=b(0,0,0)\\
    &\geq 0
  \end{align*}
  since \(b\) is a series of strictly positive numbers. For
  \(\beta=(2,0)\), we have \(u_{xx}(0,0)=0\). For \(\beta=(1,1)\), we have
  \begin{align*}
    u_{xy}(0,0)&=a\bigl(0,0,u(0,0)\bigr)u_{xx}(0,0)+\frac{\partial}{\partial
                 x}a\bigl( 0,0,u(0,0) \bigr)u_x(0,0)
                 +\frac{\partial}{\partial x} b\bigl(0,0,u(0,0)\bigr)\\
               &=\frac{\partial}{\partial x} b(0,0,0)\\
               &\geq 0.
  \end{align*}
  For \(\beta=(0,2)\), we have
  \begin{align*}
    u_{yy}(0,0)&=a\bigl(0,0,u(0,0)\bigr)u_{xy}(0,0)+\frac{\partial}{\partial
                 y}a\bigl( 0,0,u(0,0) \bigr)u_x(0,0)
                 +\frac{\partial}{\partial y} b\bigl(0,0,u(0,0)\bigr)\\
               &=a(0,0,0)\frac{\partial}{\partial
                 y}b(0,0,0)+\frac{\partial}{\partial y}b(0,0,0)\\
               &\geq 0
  \end{align*}
  since the latter is a sum of positive numbers.

  For part (b), in the proof of the Cauchy--Kovalevskaya theorem, for
  \(\beta_2=0\), we have
  \[
    D^\beta u(0,0)=0
  \]
  since \(u\) is constant on the hypersurface \(\{\,y=0\,\}\). In
  particular, \(D^\beta u(0,0)\geq 0\).

  Now, suppose \(D^\beta u(0,0)\geq 0\) for all \(\beta_2\leq n-1\). Then,
  for \(\beta=(m,n)\), we have
  \begin{align*}
    D^\beta u(0,0)
    &=D^{(m,n-1)} u_y(0,0)\\
    &=D^{(m,n-1)}\bigl(au_x+b\bigr)(0,0)\\
    &=
  \end{align*}
\end{solution}
\newpage

\begin{problem}
  (Kovalevskaya's example) Show that the line \(\{\,t=0\,\}\) is
  characteristic for the heat equation \(u_t=u_{xx}\). Show there does not
  exist an analytic solution \(u\) of the heat equation in
  \(\bfR\times\bfR\), with \(u=1/(1+x^2)\) on \(\{t=0\}\). (\emph{Hint:}
  Assume there is an analytic solution, compute its coefficients, and show
  instead that the resulting power series diverges in any neighborhood of
  \((0,0)\).)
\end{problem}
\begin{solution}
  First we show that the line \(\Gamma\defeq\{\,t=0\,\}\) is characteristic
  for the heat equation. With \(\bfnu=(1,0)\) the normal to the line
  \(\Gamma\), the noncharacteristic condition reads
  \[
    \sum_{|\alpha|=2} a_\alpha\bfnu^\alpha\neq 0.
  \]
  However,
  \[
    \sum_{|\alpha|=2} a_\alpha\bfnu^\alpha=%
    1\cdot 1+a_{0,2}\cdot 0=%
    1\neq%
    0.
  \]
  Thus, \(\Gamma\) is characteristic for \(u_t=u_{xx}\).

  Next suppose \(u\) is an analytic solution to the heat equation near
  \((0,0)\). Then, since the Cauchy data satisfies
\end{solution}

%%% Local Variables:
%%% mode: latex
%%% TeX-master: "../MA523-Current-HW"
%%% End:
