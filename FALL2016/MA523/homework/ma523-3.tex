\begin{problem}
  Consider the initial value problem
  \[
    u_t=\sin u_x;\qquad u(x,0)=\frac{\pi}{4}x.
  \]
  Verify that the assumptions of the Cauchy--Kovalevskaya theorem are
  satisfied and obtain the Taylor series of the solution about the origin.
\end{problem}
\begin{solution}
  First we zero the boundary data, that is, set \(v\defeq u-(\pi/4)x\), then
  \[
    v(x,0)=u(x,0)=\frac{\pi}{4}x-\frac{\pi}{4}x=0
  \]
  and
  \begin{align*}
    0
    &=u_t-\sin u_x\\
    &=\left(v+\frac{\pi}{4}x\right)_t
      -\sin\left(v+\frac{\pi}{4}x\right)_x\\
    &=v_t-\sin(v_x+\pi/4).
  \end{align*}
\end{solution}
\newpage

\begin{problem}
  Consider the Cauchy problem for \(u(x,y)\)
  \begin{align*}
    u_y&=a(x, y, u)u_x+b(x,y,u)\\
    u(x,0)&=0
  \end{align*}
  Let \(a\) and \(b\) be analytic functions of their arguments. Assume that
  \(D^\alpha a(0,0,0)\geq 0\) and \(D^\alpha b(0,0,0)\geq 0\) for all
  \(\alpha\). (Remember by definition, if \(\alpha=0\) then
  \(D^\alpha f=f\).)
  \begin{enumerate}[label=(\alph*),noitemsep]
  \item Show that \(D^\beta u(0,0)\geq 0\) for all \(|\beta|\leq 2\).
  \item Prove that \(D^\beta u(0,0)\geq 0\) for all
    \(\beta=(\beta_1,\beta_2)\). (\emph{Hint:} Argue as in the proof of the
    Cauchy--Kovalevskaya theorem; i.e., use induction in \(\beta_2\))
  \end{enumerate}
\end{problem}
\begin{solution}
\end{solution}
\newpage

\begin{problem}
  (Kovalevskaya's example) Show that the line \(\{t=0\}\) is characteristic
  for the heat equation \(u_t=u_{xx}\). Show there does not exist an
  analytic solution \(\) of the heat equation in \(\bfR\times\bfR\), with
  \(u=1/(1+x^2)\) on \(\{t=0\}\). (\emph{Hint:} Assume there is an analytic
  solution, compute its coefficients, and show that the resulting power
  series diverges in any neighborhood of \((0,0)\).)
\end{problem}
\begin{solution}
\end{solution}

%%% Local Variables:
%%% mode: latex
%%% TeX-master: "../MA523-Current-HW"
%%% End:
