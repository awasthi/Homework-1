\begin{problem}
  Consider the initial value problem
  \[
    u_t=\sin u_x;\qquad u(x,0)=\frac{\pi}{4}x.
  \]
  Verify that the assumptions of the Cauchy--Kovalevskaya theorem are
  satisfied and obtain the Taylor series of the solution about the origin.
\end{problem}
\begin{solution}
  % First, we zero the boundary data by setting \(v\defeq u-(\pi/4)x\) so
  % that
  % \begin{align*}
  %   0&=v(x,0)\\
  %    &=u(x,0)-\frac{\pi}{4}x\\
  %    &=\frac{\pi}{4}x-\frac{\pi}{4}x\\
  %    &=0.
  % \end{align*}
  % Moreover,
  % \[
  %   \frac{\partial^n v}{\partial t^n}(x,0)=0
  % \]
  % for all \(n\), and all \(x\). Lastly, we note that
  % \begin{align*}
  %   0&=u_t-\sin u_x\\
  %    &=\left(v+\frac{\pi}{4}x\right)_t+
  %      \sin\left(v+\frac{\pi}{4}x\right)_x\\
  %    &=v_t+\sin\bigl(v_x+(\pi/4)\bigr).
  % \end{align*}

  We skip checking that the assumptions of the Cauchy--Kovalevskaya theorem
  are satisfied (since I cannot decipher Evans's notation), and show that
  he Taylor series of \(u\) at \((0,0)\),
  \[
    \tilde u(x,t)=\sum_{(\alpha_1,\alpha_2)}
    \frac{u_{\alpha_1,\alpha_2}(0)}{\alpha_1!\alpha_2!} x^{\alpha_1}t^{\alpha_2},
  \]
  is a solution to our PDE.

  First, we must compute the coefficients
  \[
    \frac{u_{\alpha_1,\alpha_2}(0,0)}{\alpha_1!\alpha_2!}.
  \]
  To this end, we must find the partial derivatives
  \(u_{\alpha_1,\alpha_2}\) and potentially, relations among them which
  will help us to find these coefficients. Naïvely listing the partials
  with respect to \(t\) and \(x\), we have
  \begin{align*}
    u(0,0)&=0\\
    u_x(0,0)&=\frac{\pi}{4}\\
    u_t(0,0)&=\sin u_x(0,0)=\frac{\sqrt{2}}{2}\\
    u_{xx}(0,0)&=0\\
    u_{tx}(0,0)&=0\\
    u_{tt}(0,0)&=-\cos\bigl(u_x(0,0)\bigr)u_{xt}(0,0)=0\\
    u_{xxx}(0,0)&=0\\
    u_{ttx}(0,0)&=0,
  \end{align*}
  etc. Thus,
  \[
    \tilde u=\frac{\pi}{4}x+\frac{\sqrt{2}}{2}t.
  \]
  Plugging this equation into our PDE, we have
  \[
    \tilde u_t-\sin\tilde u_x=\frac{\sqrt{2}}{2}-\sin(\pi/4)=0,
  \]
  as desired.
\end{solution}
\newpage

\begin{problem}
  Consider the Cauchy problem for \(u(x,y)\)
  \begin{align*}
    u_y&=a(x, y, u)u_x+b(x,y,u)\\
    u(x,0)&=0
  \end{align*}
  Let \(a\) and \(b\) be analytic functions of their arguments. Assume that
  \(D^\alpha a(0,0,0)\geq 0\) and \(D^\alpha b(0,0,0)\geq 0\) for all
  \(\alpha\). (Remember by definition, if \(\alpha=0\) then
  \(D^\alpha f=f\).)
  \begin{enumerate}[label=(\alph*),noitemsep]
  \item Show that \(D^\beta u(0,0)\geq 0\) for all \(|\beta|\leq 2\).
  \item Prove that \(D^\beta u(0,0)\geq 0\) for all
    \(\beta=(\beta_1,\beta_2)\). (\emph{Hint:} Argue as in the proof of the
    Cauchy--Kovalevskaya theorem; i.e., use induction in \(\beta_2\))
  \end{enumerate}
\end{problem}
\begin{solution}
  For part (a):

  For part (b):
\end{solution}
\newpage

\begin{problem}
  (Kovalevskaya's example) Show that the line \(\{t=0\}\) is characteristic
  for the heat equation \(u_t=u_{xx}\). Show there does not exist an
  analytic solution \(\) of the heat equation in \(\bfR\times\bfR\), with
  \(u=1/(1+x^2)\) on \(\{t=0\}\). (\emph{Hint:} Assume there is an analytic
  solution, compute its coefficients, and show instead that the resulting
  power series diverges in any neighborhood of \((0,0)\).)
\end{problem}
\begin{solution}
\end{solution}

%%% Local Variables:
%%% mode: latex
%%% TeX-master: "../MA523-Current-HW"
%%% End:
