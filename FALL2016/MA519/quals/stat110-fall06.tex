\section{Extra Problems}
These problems were taken from probability courses from other
universities.
\subsection{Stat 110 (Harvard) -- Final 2006}
\begin{problem}
  The number of fish in a certain lake is a \(\Poi(\lambda)\) random
  variable. Worried there might be no fish at all, a statistician adds one
  fish to the lake. Let \(Y\) be the resulting number of fish (so \(Y\) is
  \(1\) plus a \(\Poi(\lambda)\) random variable).
  \begin{alphlist}
  \item Find \(E(Y^2)\).
  \item Find \(E(\frac{1}{Y})\) (in terms of \(\lambda\); do not simplify
    yet).
  \item Find a simplified expression for \(E(\frac{1}{Y})\). \emph{Hint:}
    \(k!(k+1)=(k+1)!\).
  \end{alphlist}
\end{problem}
\begin{solution*}
  For part (a): Let \(X\sim\Poi(\lambda)\) such that \(X=Y-1\). Then we
  have
  \begin{align*}
    E(Y^2)
    &=E\bigl[(X+1)^2\bigr]\\
    &=E(X^2+2X+1)\\
    &=E(X^2)+2E(X)+1\\
    &=\lambda^2+\lambda+2\lambda+1\\
    &=\lambda^2+3\lambda+1.
  \end{align*}
  \\\\
  For part (b): Keeping \(X\) as above, we have
  \begin{align*}
    E\left(\tfrac{1}{Y}\right)
    &=E\left(\tfrac{1}{X+1}\right)\\
    &=\sum_{k=0}^\infty \frac{1}{k+1}\rme^{-\lambda}\frac{\lambda^k}{k!}\\
    &=\rme^{-\lambda}\sum_{k=0}^\infty\frac{\lambda^k}{(k+1)!}\\
    &=\rme^{-\lambda}\sum_{k=0}^\infty\frac{\lambda^{-1}\lambda^{k+1}}{(k+1)!}.
  \end{align*}
  \\\\
  For part (c): Taking the equation above and further simplifying it, we
  have
  \begin{align*}
    E\left( \tfrac{1}{Y} \right)
    &=\frac{\rme^{-\lambda}}{\lambda}
      \sum_{k=0}^\infty\frac{\lambda^{k+1}}{(k+1)!}\\
    &=\frac{\rme^{-\lambda}}{\lambda}
      \sum_{k=1}^\infty\frac{\lambda^k}{k!}\\
    &=\frac{\rme^{-\lambda}}{\lambda}(\rme^{\lambda}-1)\\
    &=\tfrac{1}{\lambda}(1-\rme^{-\lambda}).\qedhere
  \end{align*}
\end{solution*}

\begin{problem}
  Write the most appropriate of \(\leq\), \(\geq\), \(=\) or \(?\) in the
  blank for each part (where ``\(?\)'' means that no relation holds in
  general.)  It is not necessary to justify your answers for full credit;
  some partial credit is available for justified answers that are flawed
  but on the right track.  In (c) through (f), \(X\) and \(Y\) are IID
  (independent identically distributed) positive random variables. Assume
  that the various expected values exist.
  \begin{alphlist}
  \item the probability that a roll of \(2\) fair dice totals \(9\) \_ the
    probability that a roll of \(2\) fair dice totals \(10\).
  \item the probability that \(65\%\) of \(20\) children born are girls \_
    the probability that \(65\%\) of \(2000\) children born are girls.
  \item \(E\bigl(\sqrt{X}\bigr)\_\sqrt{E(X)}\).
  \item \(E(\sin X)\_\sin[E(X)]\).
  \item \(P(X+Y>4)\_ P(X>2)P(Y>2)\).
  \item \(E\bigl[ (X+Y)^2 \bigr]\_ 2E(X^2)+2[E(X)]^2\).
  \end{alphlist}
\end{problem}
\begin{solution*}
\end{solution*}

\begin{problem}
  A fair die is rolled twice, with outcomes \(X\) for the \(1\)\textsup{th}
  roll and \(Y\) for the \(2\)\textsup{th} roll.
  \begin{alphlist}
  \item
  \end{alphlist}
\end{problem}
\begin{solution*}
\end{solution*}

\begin{problem}
\end{problem}
\begin{solution*}
\end{solution*}

\begin{problem}
\end{problem}
\begin{solution*}
\end{solution*}

\begin{problem}
\end{problem}
\begin{solution*}
\end{solution*}

\begin{problem}
\end{problem}
\begin{solution*}
\end{solution*}

\begin{problem}
\end{problem}
\begin{solution*}
\end{solution*}

\begin{problem}
\end{problem}
\begin{solution*}
\end{solution*}

\begin{problem}
\end{problem}
\begin{solution*}
\end{solution*}

\begin{problem}
\end{problem}
\begin{solution*}
\end{solution*}

\begin{problem}
\end{problem}
\begin{solution*}
\end{solution*}

\begin{problem}
\end{problem}
\begin{solution*}
\end{solution*}

\begin{problem}
\end{problem}
\begin{solution*}
\end{solution*}

\begin{problem}
\end{problem}
\begin{solution*}
\end{solution*}

\begin{problem}
\end{problem}
\begin{solution*}
\end{solution*}

\begin{problem}
\end{problem}
\begin{solution*}
\end{solution*}

\begin{problem}
\end{problem}
\begin{solution*}
\end{solution*}

\begin{problem}
\end{problem}
\begin{solution*}
\end{solution*}

\begin{problem}
\end{problem}
\begin{solution*}
\end{solution*}

\begin{problem}
\end{problem}
\begin{solution*}
\end{solution*}

\begin{problem}
\end{problem}
\begin{solution*}
\end{solution*}

%%% Local Variables:
%%% mode: latex
%%% TeX-master: "../MA519-HW-ALL"
%%% End:
