\section{Homework Solutions}
These are my (corrected) solutions to DasGupta's Math/Stat 519 homework for
the fall semester of 2016.
\subsection{Homework 1}
\begin{problem}[Handout 1, \# 5]
  A closet contains five pairs of shoes. If four shoes are selected at
  random, what is the probability that there is at least one complete pair
  among the four?
\end{problem}
\begin{solution*}
  There are \(\binom{10}{4}=210\) to choose four shoes from five
  pairs. Therefore, the size of the sample space is \(|\Omega|=210\) and,
  under the equally-likely hypothesis, each sample point \(x\in\Omega\) has
  probability \(\frac{1}{210}\).

  We compute the probability of the complement; i.e., the probability that
  there is no complete pair among the four randomly selected shoes. To
  avoid selecting a pair we choose must choose of two (a left or a right)
  shoes from four pairs which we can do in \(\binom{5}{5}=5\) different
  ways.

  Thus, the probability that at least one pair of shoes is among the four
  randomly selected shoes is
  \begin{align*}
    P(\text{one pair is among the four})
    &=1-P(\text{no pair is among the four})\\
    &=1-\frac{5\cdot 2^4}{210}\\
    &=1-\frac{8}{21}\\
    &=\num{0.6190476190476191}.\qedhere
  \end{align*}
\end{solution*}

\begin{problem}[Handout 1, \# 7]
  A gene consists of \(10\) subunits, each of which is normal or
  mutant. For a particular cell, there are \(3\) mutant and \(7\) normal
  subunits. Before the cell divides into two daughter cells, the gene
  duplicates. The corresponding gene of cell 1 consists of \(10\) subunits
  chosen from the \(6\) mutant and \(14\) normal units. Cell 2 gets the
  rest. What is the probability that one of the cells consists of all
  normal subunits.
\end{problem}
\begin{solution*}
  Assuming that we can arrange these gene subunits in any possible order,
  there are \(\binom{20}{10}=\num{184756}\) possible gene combinations any
  given daughter cell can end up with.

  There are \(\binom{14}{10}=1001\) ways for cell 1 to end up without any
  mutant subunits. Hence, there are \(2\cdot 1001=2002\) ways for at least
  one of the daughter cells to end up without any mutant subunits.

  Therefore, the probability that at least one of the two daughter cells
  end up without any mutant subunits is
  \[
    P(\text{at least one daughter cell consists of all normal subunits})
    =\frac{2002}{\num{184756}}
    \approx\num{0.010835913312693499}.\qedhere
  \]
\end{solution*}

\begin{problem}[Handout 1, \# 9]
  From a sample of size \(n\), \(r\) elements are sampled at random. Find
  the probability that none of the \(N\) prespecified elements are included
  in the sample, if sampling is
  \begin{enumerate}[label=(\alph*),noitemsep]
  \item with replacement;
  \item without replacement.
  \end{enumerate}
  Compute it for \(r=N=10\), \(n=100\).
\end{problem}
\begin{solution*}
  For part (a): The size of the sample space is \(|\Omega|=n^r\) and
  therefore each element \(x\in\Omega\) has a probability of \(n^{-r}\).
  If \(n\) elements are specified, \(n-N\) have not and therefore, there
  are \((n-N)^r\) ways of drawing \(r\) of the non-prespecified
  elements. Thus, the probability that note of the \(N\) prespecified
  elements are drawn if we sample \(r\) elements randomly with replacement
  is
  \begin{equation}
    \label{eq:1:with-replacement}
    p_1(n,N,r)=\frac{(n-N)^r}{n^r}.
  \end{equation}

  For part (b): The argument leading to the probability of this event is
  similar to that of part (a). The size of the sample space is
  \(|\Omega|=\binom{n}{r}\); these correspond to the passible draws of
  \(r\) elements without replacement. As before, \(n-N\) of the elements
  have not been prespecified and therefore, we have \(\binom{n-N}{r}\) ways
  of drawing \(r\) of the non-prespecified elements. Thus, the probability
  that note of the \(N\) prespecified elements are drawn if we sample \(r\)
  elements randomly without replacement is
  \begin{equation}
    \label{eq:1:without-replacement}
    p_2(n,N,r)=\frac{\binom{n-N}{r}}{\binom{n}{r}}.
  \end{equation}

  Finally, we compute \(p_1\) and \(p_2\) for \(r=N=10\), \(n=100\) using
  equations \eqref{eq:1:with-replacement} and
  \eqref{eq:1:without-replacement} above:
  \begin{align*}
    p_1(100,10,10)&\approx\num{0.3486784401},\\
    p_2(100,10,10)&\approx\num{0.3304762208462021}.\qedhere
  \end{align*}
\end{solution*}

\begin{problem}[Handout 1, \# 11]
  Let \(E\), \(F\), and \(G\) be three events. Find expressions for the
  following events:
  \begin{enumerate}[label=(\alph*),noitemsep]
  \item only \(E\) occurs;
  \item both \(E\) and \(G\) occur, but not \(F\);
  \item all three occur;
  \item at least one of the events occurs;
  \item at most two of them occur.
  \end{enumerate}
\end{problem}
\begin{solution*}
  For part (a): The event ``from \(E\), \(F\), and \(G\) only \(E\)
  occurs'' is given by
  \[
    E\cap(\Omega\setminus F)\cap(\Omega\setminus G).
  \]

  For part (b): The event ``from \(E\), \(F\), and \(G\) both \(E\) and
  \(G\) occur, but not \(F\)'' is given by
  \[
    E\cap G\cap(\Omega\setminus F).
  \]

  For part (c): The event ``from \(E\), \(F\), and \(G\) all three occur''
  is given by
  \[
    E\cap F\cap G.
  \]

  For part (d): The event ``from \(E\), \(F\), and \(G\) at least one
  occurs'' is given by
  \[
    E\cup F\cup G.
  \]

  For part (e): The event ``from \(E\), \(F\), and \(G\) at most two
  occur'' is given by
  \[
    \bigl(E\cap F\cap(\Omega\setminus G)\bigr)\cup%
    \bigl(E\cap (\Omega\setminus F)\cap G\bigr)\cup%
    \bigl((\Omega\setminus E)\cap F\cap G\bigr).\qedhere
  \]
\end{solution*}

\begin{problem}[Handout 1, \# 12]
  Which is more likely:
  \begin{enumerate}[label=(\alph*),noitemsep]
  \item Obtaining at least one six in six rolls of a fair die, or
  \item Obtaining at least one double six in six rolls of a pair of fair
    dice.
  \end{enumerate}
\end{problem}
\begin{solution*}
  For part (a): The probability that we do not roll a six in six rolls of a
  fair die is
  \[
    q_1=\left(\frac{5}{6}\right)^6\approx\num{0.3348979766803842}.
  \]
  Therefore, the probability of seeing at least one six in six rolls of a
  fair die is
  \[
    p_1=1-q_1\approx\num{0.6651020233196159}.
  \]

  For part (b): The probability that we do not roll a double six in six
  rolls of a pair fair die is
  \[
    q_2=\left(\frac{35}{36}\right)^6\approx\num{0.8444875698403315}.
  \]
  Therefore, the probability of seeing at least one double six in six rolls
  of a pair of fair die is
  \[
    p_2=1-q_2\approx\num{0.15551243015966854}.
  \]

  Lastly, we see that the \(p_1>p_2\); i.e., the probability of rolling at
  least one six in six rolls of a fair die is more likely than the
  probability of obtaining two double sixes is six rolls of a pair of fair
  dice.
\end{solution*}

\begin{problem}[Handout 1, \# 13]
  There are \(n\) people are lined up at random for a photograph. What is
  the probability that a specified set of \(r\) people happen to be next to
  each other?
\end{problem}
\begin{solution*}
  The \(r\) prespecified people can stand as a group starting at positions
  \(1,\dotsc,n-r+1\). They can be permuted among themselves in \(r!\)
  ways. The remaining \(n-r\) people can be permuted among themselves in
  \((n-r)!\) ways. Thus, we have
  \[
    p=\frac{(n-r+1)r!(n-r)!}{n!}=\frac{(n-r+1)!r!}{n!}.\qedhere
  \]
\end{solution*}

\begin{problem}[Handout 1, \# 16]
  Consider a particular player, say North, in a Bridge game. Let \(X\) be
  the number of aces in his hand. Find the distribution of \(X\).
\end{problem}
\begin{solution*}
  The number of aces in North's hand \(X\) is a random variable which takes
  integer values between \(0\) and \(4\). Therefore, we must compute the
  probabilities \(P(X=x)\), \(0\leq x\leq 4\).

  From a deck with \(52\) cards, \(13\) cards can be selected in
  \(\binom{52}{13}\) ways among which the player will have \(x\) number of
  aces in \(\binom{4}{x}\binom{48}{13-x}\) ways. Thus, the distribution is
  given by
  \[
    P(X=x)=\frac{\binom{4}{x}\binom{48}{13-x}}{\binom{52}{13}}.
  \]
  The explicit values are (approximately)
  \begin{align*}
    P(X=0)&\approx 0.304,
    &P(X=1)&\approx 0.439,\\
    P(X=2)&\approx 0.213,
    &P(X=3)&\approx 0.041,\\
    P(X=4)&\approx 0.003.&&\qedhere
  \end{align*}
\end{solution*}

\begin{problem}[Handout 1, \# 20]
  If \(100\) balls are distributed completely at random into \(100\) cells,
  find the expected value of the number of empty cells.

  \noindent Replace \(100\) by \(n\) and derive the general expression. Now
  approximate it as \(n\) tends to \(\infty\).
\end{problem}
\begin{solution*}
  Let \(X\) be the random variable denoting the number of empty
  cells. Define \(X_1,\dotsc,X_n\) indicator variables by
  \[
    X_k\defeq
    \begin{cases}
      1&\text{if the \(i\)\textsup{th} cell is empty,}\\
      0&\text{otherwise.}
    \end{cases}
  \]
  Then
  \[
    X=\sum_{k=1}^n X_k
  \]
  and so the mean of this random variable is given by
  \begin{align*}
    E(X)
    &=\sum_{k=1}^n E(X_k)\\
    &=\sum_{k=1}^n P(X_k=1)\\
    &=\sum_{k=1}^n\frac{(n-1)^n}{n^n}\\
    &=n\left(1-\frac{1}{n}\right)^n,
  \end{align*}
  which approaches \(\infty\) as \(n\to\infty\); i.e., for any \(M>0\)
  there exists a positive integer \(N\) such that
  \begin{align*}
    M&>n\left(1+\frac{1}{n}\right)^n\\
    1&>\left(\frac{M}{n}\right)^{\frac{1}{n}}-\frac{1}{n}
  \end{align*}
  since both \(\frac{M}{n}\) and \(\frac{1}{n}\) approach \(0\) as
  \(n\to\infty\).
  \\\\
  For \(n=100\), we have \(E(X)\approx 36.60\).
\end{solution*}

%%% Local Variables:
%%% mode: latex
%%% TeX-master: "../MA519-HW-ALL"
%%% End:
