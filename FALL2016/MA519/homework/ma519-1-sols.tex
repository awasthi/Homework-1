\section{Homework Solutions}
These are my (corrected) solutions to DasGupta's Math/Stat 519 homework for
the fall semester of 2016.

Throughout this document, unless otherwise specified, \(\Omega\) denotes
the sample space in question. The symbol `\(\sim\)' is used both to denote
the distribution type of a random variable and to denote asymptotic
equivalence; i.e., if \(\{a_n\}\), \(\{b_n\}\) are convergent sequences
with limit \(a\) and \(b\), respectively, we say \(a_n\sim b_n\) if
\(\frac{a_n}{b_n}\to 1\).

For the sake of remaining consistent with DasGupta's book, we adopt the
following notation for some of the distributions considered in this course
(here and throughout the text \(q\bydef q(p)=1-p\)):

\begin{adjustwidth}{-.5cm}{}
\begin{tabular}{|L|L|L|L|L|}
\hline
\text{name}&\text{params.}&\text{PMF or PDF}&\text{mean}&\text{variance}\\
\hline
\text{Bernoulli}
&\Ber(p)%
&\text{\(1-p\), for \(k=0\); \(p\), for \(k=1\)}%
&p
&p(1-p)
\\
\text{binomial}
&\Bin(n,p)%
&\text{\(\binom{n}{k}p^k(1-p)^{n-k}\), for \(k=0,\dotsc,n\)}
&np
&np(1-p)
\\
\text{Poisson}
&\Poi(p)
&\rme^{-\lambda}\frac{\lambda^k}{k!}
&\lambda
&\lambda
\\
\text{geometric}
&\Geo(p)
&(1-p)^{k-1} p
&\frac{1}{p}
&\frac{q}{p^2}
\\
\text{negative binomial}
&\NB(r,p)
&\binom{k+r-1}{k}(1-p)^rp^k
&\frac{pr}{1-p}
&\frac{1-p}{p^2}
\\
\text{hypergeometric}
&\Hypergeo(n,D,N)
&\frac{\binom{D}{k}\binom{N-D}{n-k}}{\binom{N}{n}}
&\frac{nD}{N}
&\frac{nD}{N}(1-\frac{D}{N})\frac{N-n}{N-1}
\\
\text{uniform}
&U[a,b]
&\text{\(\frac{1}{b-a}\), for \(a\leq x\leq b\)}
&\frac{a+b}{2}
&\frac{(b-a)^2}{12}
\\
\text{exponential}
&\Exp(\lambda)
&\text{\(\frac{1}{\lambda}\rme^{-x/\lambda}\), for \(x\geq 0\)}
&\lambda
&\lambda^2
\\
\text{gamma}
&G(\alpha,\lambda)
&\text{\(\frac{\rme^{-x/\lambda}x^{\alpha-1}}
  {\lambda^\alpha\Gamma(\alpha)}\), for \(x\geq 0\)}
&\alpha\lambda
&\alpha\lambda^2
\\
\text{chi-squared}
&\chi^2_n
&\text{\(\frac{\rme^{-x/2}x^{m/2-1}}
  {2^{m/2}\Gamma(\frac{m}{2})}\), for \(x\geq 0\)}
&m
&2m
\\
\text{beta}
&B(\alpha,\beta)
&\text{\(\frac{x^{\alpha-1}(1-x)^{\beta-1}}{\rmB(\alpha,\beta)}\)
 , for \(0\leq x\leq 1\)}
&\frac{\alpha}{\alpha+\beta}
&\frac{\alpha\beta}{(\alpha+\beta)^2(\alpha+\beta+1)}
\\
\text{normal}
&N(\mu,\sigma)
&\frac{1}{\sqrt{2\pi\sigma^2}}\rme^{-(x-\mu)^2/2\sigma^2}
&\mu
&\sigma^2
\\
\text{Cauchy}
&C(\mu,\sigma)
&\bigl(\sigma\pi(1+\frac{(x-\mu)^2}{\sigma^2})\bigr)^{-1}
&\text{DNE}
&\text{DNE}
\\
\hline
\end{tabular}
\end{adjustwidth}

For numerical problems, we consistently stick to the rule of writing four
figures after the decimal point.

\subsection{Homework 1}
\begin{problem}[Handout 1, \# 5]
  A closet contains five pairs of shoes. If four shoes are selected at
  random, what is the probability that there is at least one complete pair
  among the four?
\end{problem}
\begin{solution*}
  Let \(A\subset\Omega\) denote the event ``there is at least one complete
  pair of shoes among the four randomly selected shoes.''

  First, we compute the size of the sample space \(\Omega\). There are
  \(\binom{10}{4}=210\) ways to select four shoes from five pairs of
  shoes. Therefore, \(|\Omega|=210\) so each sample point \(x\in\Omega\)
  has an associated probability of \(P(x)=\frac{1}{210}\).

  Now, let us compute \(P(A)\). Oftentimes it is easier to compute the
  probability \(P(\Omega\setminus A)\) and use the identity
  \[
    P(A)+P(\Omega\setminus A)=1;
  \]
  this is one of those times. Let us now find the number of sample points
  in \(\Omega\setminus A\); i.e., the event ``there is no complete pair of
  shoes among the four randomly selected shoes.'' In order that we do not
  choose a complete pair of shoes we must choose from four different pairs,
  this can be done in \(\binom{5}{4}=5\) ways, and for each pair we have
  the possibility of choosing on of two shoes belonging to that pair
  (either a left or a right shoe). Therefore,
  \(|\Omega\setminus A|=5\cdot 2^4=80\) and so
  \begin{align*}
    P(A)
    &=1-P(\Omega\setminus A)\\
    &=1-\frac{80}{210}\\
    &\approx0.619.\qedhere
  \end{align*}
\end{solution*}

\begin{problem}[Handout 1, \# 7]
  A gene consists of \(10\) subunits, each of which is normal or
  mutant. For a particular cell, there are \(3\) mutant and \(7\) normal
  subunits. Before the cell divides into two daughter cells, the gene
  duplicates. The corresponding gene of cell 1 consists of \(10\) subunits
  chosen from the \(6\) mutant and \(14\) normal units. Cell 2 gets the
  rest. What is the probability that one of the cells consists of all
  normal subunits.
\end{problem}
\begin{solution*}
  Let \(A\subset\Omega\) denote the event ``at least one daughter cell
  contains all normal subunits.''

  When the cell duplicates, there will be a total of \(20\) subunits
  (\(14\) normal and \(6\) mutant ones). There are
  \(\binom{20}{10}=184756\) ways to distribute these subunits to a given
  daughter cell. Therefore, \(|\Omega|=184756\).

  Now, let us count the number of sample points in \(A\). Suppose, with out
  loss of generality, that cell 1 receives all of the normal subunits;
  there are \(\binom{14}{10}=1001\) ways to do this. Since we may as well have
  chosen cell 2 to give the normal units to, the number of sample points in
  \(A\) is twice the figure above; i.e., \(|A|=2002\).

  Therefore,
  \[
    P(A)=\frac{2002}{184756}\approx0.0108.\qedhere
  \]
\end{solution*}

\begin{problem}[Handout 1, \# 9]
  From a sample of size \(n\), \(r\) elements are sampled at random. Find
  the probability that none of the \(N\) prespecified elements are included
  in the sample, if sampling is
  \begin{enumerate}[label=(\alph*),noitemsep]
  \item with replacement;
  \item without replacement.
  \end{enumerate}
  Compute it for \(r=N=10\), \(n=100\).
\end{problem}
\begin{solution*}
  For part (a): The size of the sample space is \(|\Omega|=n^r\) so for
  each \(x\in\Omega\), \(P(x)=\frac{1}{n^r}\). If \(n\) elements are
  prespecified, there are \(n-N\) non-prespecified elements and thus we
  have \((n-N)^r\) ways to draw \(r\) non-prespecified elements. Thus, the
  probability that none of the \(N\) prespecified elements are drawn if we
  sample \(r\) elements randomly with replacement is
  \begin{equation}
    \label{eq:1:with-replacement}
    p_1(n,N,r)=\frac{(n-N)^r}{n^r}.
  \end{equation}
  \\\\
  For part (b): The argument leading to the probability of this event is
  similar to that of part (a). The size of the sample space is
  \(|\Omega|=\binom{n}{r}\); these correspond to the possible draws of
  \(r\) elements without replacement. As before, \(n-N\) of the elements
  have not been prespecified and therefore, we have \(\binom{n-N}{r}\) ways
  of drawing \(r\) of the non-prespecified elements. Thus, the probability
  that note of the \(N\) prespecified elements are drawn if we sample \(r\)
  elements randomly without replacement is
  \begin{equation}
    \label{eq:1:without-replacement}
    p_2(n,N,r)=\frac{\binom{n-N}{r}}{\binom{n}{r}}.
  \end{equation}

  Finally, we compute \(p_1\) and \(p_2\) for \(r=N=10\), \(n=100\) using
  equations \eqref{eq:1:with-replacement} and
  \eqref{eq:1:without-replacement} above:
  \begin{align*}
    p_1(100,10,10)&\approx\num{0.3486784401},\\
    p_2(100,10,10)&\approx\num{0.3304762208462021}.\qedhere
  \end{align*}
\end{solution*}

\begin{problem}[Handout 1, \# 11]
  Let \(E\), \(F\), and \(G\) be three events. Find expressions for the
  following events:
  \begin{enumerate}[label=(\alph*),noitemsep]
  \item only \(E\) occurs;
  \item both \(E\) and \(G\) occur, but not \(F\);
  \item all three occur;
  \item at least one of the events occurs;
  \item at most two of them occur.
  \end{enumerate}
\end{problem}
\begin{solution*}
  For part (a): The event ``from \(E\), \(F\), and \(G\) only \(E\)
  occurs'' is given by
  \[
    E\cap(\Omega\setminus F)\cap(\Omega\setminus G).
  \]
  \\\\
  For part (b): The event ``from \(E\), \(F\), and \(G\) both \(E\) and
  \(G\) occur, but not \(F\)'' is given by
  \[
    E\cap G\cap(\Omega\setminus F).
  \]
  \\\\
  For part (c): The event ``from \(E\), \(F\), and \(G\) all three occur''
  is given by
  \[
    E\cap F\cap G.
  \]
  \\\\
  For part (d): The event ``from \(E\), \(F\), and \(G\) at least one
  occurs'' is given by
  \[
    E\cup F\cup G.
  \]
  \\\\
  For part (e): The event ``from \(E\), \(F\), and \(G\) at most two
  occur'' is given by
  \[
    \bigl(E\cap F\cap(\Omega\setminus G)\bigr)\cup%
    \bigl(E\cap (\Omega\setminus F)\cap G\bigr)\cup%
    \bigl((\Omega\setminus E)\cap F\cap G\bigr).\qedhere
  \]
\end{solution*}

\begin{problem}[Handout 1, \# 12]
  Which is more likely:
  \begin{enumerate}[label=(\alph*),noitemsep]
  \item Obtaining at least one six in six rolls of a fair die, or
  \item Obtaining at least one double six in six rolls of a pair of fair
    dice.
  \end{enumerate}
\end{problem}
\begin{solution*}
  For part (a): The probability that we do not roll a six in six rolls of a
  fair die is
  \[
    q_1=\left(\frac{5}{6}\right)^6\approx\num{0.3348979766803842}.
  \]
  Therefore, the probability of seeing at least one six in six rolls of a
  fair die is
  \[
    p_1=1-q_1\approx\num{0.6651020233196159}.
  \]
  \\\\
  For part (b): The probability that we do not roll a double six in six
  rolls of a pair fair die is
  \[
    q_2=\left(\frac{35}{36}\right)^6\approx\num{0.8444875698403315}.
  \]
  Therefore, the probability of seeing at least one double six in six rolls
  of a pair of fair die is
  \[
    p_2=1-q_2\approx\num{0.15551243015966854}.
  \]

  Lastly, we see that the \(p_1>p_2\); i.e., the probability of rolling at
  least one six in six rolls of a fair die is more likely than the
  probability of obtaining two double sixes is six rolls of a pair of fair
  dice.
\end{solution*}

\begin{problem}[Handout 1, \# 13]
  There are \(n\) people are lined up at random for a photograph. What is
  the probability that a specified set of \(r\) people happen to be next to
  each other?
\end{problem}
\begin{solution*}
  The \(r\) prespecified people can stand as a group starting at positions
  \(1,\dotsc,n-r+1\). They can be permuted among themselves in \(r!\)
  ways. The remaining \(n-r\) people can be permuted among themselves in
  \((n-r)!\) ways. Thus, we have
  \[
    p=\frac{(n-r+1)r!(n-r)!}{n!}=\frac{(n-r+1)!r!}{n!}.\qedhere
  \]
\end{solution*}

\begin{problem}[Handout 1, \# 16]
  Consider a particular player, say North, in a Bridge game. Let \(X\) be
  the number of aces in his hand. Find the distribution of \(X\).
\end{problem}
\begin{solution*}
  Let \(X\) denote the number of aces in North's hand; this is a random
  variable taking integer values between \(0\) and \(4\). We are asked to
  find the PMF of \(X\); i.e., the values \(P(X=x)\) for all
  \(x=0,\dotsc,4\).

  From a deck of \(52\) cards, \(13\) cards can be selected in
  \(\binom{52}{13}\) ways. From these North can have \(x\) number of aces
  in \(\binom{4}{x}\binom{48}{13-x}\) ways. Therefore, the PMF of \(X\) is
  precisely
  \[
    P(X=x)=\frac{\binom{4}{x}\binom{48}{13-x}}{\binom{52}{13}}.
  \]

  The values of \(P\) at each \(x=0,\dotsc,4\) are
  \begin{align*}
    P(X=0)&\approx 0.304,
    &P(X=1)&\approx 0.439,\\
    P(X=2)&\approx 0.213,
    &P(X=3)&\approx 0.041,\\
    P(X=4)&\approx 0.003.&&\qedhere
  \end{align*}
\end{solution*}

\begin{problem}[Handout 1, \# 20]
  If \(100\) balls are distributed completely at random into \(100\) cells,
  find the expected value of the number of empty cells.

  \noindent Replace \(100\) by \(n\) and derive the general expression. Now
  approximate it as \(n\) tends to \(\infty\).
\end{problem}
\begin{solution*}
  Let \(X\) denote the number of empty cells. Define \(I_1,\dotsc,I_n\)
  indicator variables as
  \[
    I_k\defeq
    \begin{cases}
      1&\text{if the \(k\)\textsup{th} cell is empty,}\\
      0&\text{otherwise.}
    \end{cases}
  \]
  Then
  \[
    X=\sum_{k=1}^n I_k
  \]
  so the mean of \(X\) is
  \begin{align*}
    E(X)
    &=\sum_{k=1}^n E(I_k)\\
    &=\sum_{k=1}^n P(I_k=1)\\
    &=\sum_{k=1}^n\frac{(n-1)^n}{n^n}\\
    &=n\left(1-\frac{1}{n}\right)^n,
  \end{align*}
  which approaches \(\infty\) as \(n\to\infty\) since given any positive
  real number \(M\), we have
  \begin{align*}
    M&<n\left(1-\frac{1}{n}\right)^{1/n}\\
    1&>\left(\frac{M}{n}\right)^{1/n}+\frac{1}{n}
  \end{align*}
  for sufficiently large \(n\).

  For \(n=100\), we have \(E(X)\approx\num{36.60323412732292}\).
\end{solution*}

%%% Local Variables:
%%% mode: latex
%%% TeX-master: "../MA519-HW-ALL"
%%% End:
