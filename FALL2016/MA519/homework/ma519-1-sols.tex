\section{Homework Solutions}
These are my (corrected) solutions to DasGupta's Math/Stat 519 homework for
the fall semester of 2016.
\subsection{Homework 1}
\begin{problem}[Handout 1, \# 5]
  A closet contains five pairs of shoes. If four shoes are selected at
  random, what is the probability that there is at least one complete pair
  among the four?
\end{problem}
\begin{solution*}
  Let \(A\) denote the event ``at least one complete pair is among the four
  shoes that were selected at random''. We compute the probability
  \(P(\Omega\setminus A)\), i.e., the probability of the event ``there is
  no complete pair among the four shoes that were selected at random''.

  Under the equal likeliness assumption, the problem reduces to
  combinatorics. There are \(\binom{10}{4}=210\) ways to choose four shoes
  from five pairs. Therefore, \(|\Omega|=210\) and every point
  \(x\in\Omega\) has a corresponding probability of \(\frac{1}{210}\). Now,
  in order to avoid selecting a pair, we can choose one of two (a left or a
  right) shoe from four pairs of shoes which we can choose in
  \(\binom{5}{4}=5\) ways.

  Thus, the probability that there is at least one pair of shoes among the
  four shoes selected at random is
  \begin{align*}
    P(A)
    &=1-P(\Omega\setminus A)\\
    &=1-\frac{5\cdot 2^4}{210}\\
    &=1-\frac{8}{21}\\
    &\approx\num{0.6190476190476191}.\qedhere
  \end{align*}
\end{solution*}

\begin{problem}[Handout 1, \# 7]
  A gene consists of \(10\) subunits, each of which is normal or
  mutant. For a particular cell, there are \(3\) mutant and \(7\) normal
  subunits. Before the cell divides into \(2\) daughter cells, the gene
  duplicates. The corresponding gene of cell \(1\) consists of \(10\)
  subunits chosen from the \(6\) mutant and \(14\) normal units. Cell \(2\)
  gets the rest. What is the probability that one of the cells consists of
  all normal subunits.
\end{problem}
\begin{solution*}
  Let \(A\) denote the event ``at least one of the two cells consists of
  all normal subunits''. Assuming we can arrange these subunits in any
  order, each daughter cell can receive \(10\) of the \(20\) subunits in
  \(\binom{20}{10}=184756\) ways. Hence, \(|\Omega|=184756\). Now, suppose
  one of the daughter cells receives a full \(10\) normal subunits. There
  are \(\binom{14}{10}=1001\) ways to do this. Since we can do this for the
  other cell as well, there are \(2\cdot 1001=2002\) ways of giving
  daughter one cell a full \(10\) normal subunits. Thus, the probability
  that at least one of the two daughter cells consists entirely of normal
  subunits is
  \[
    P(A)=\frac{2002}{184756}\approx\num{0.0108359133}.
  \]
\end{solution*}

\begin{problem}[Handout 1, \# 9]
  From a sample of size \(n\), \(r\) elements are sampled at random. Find
  the probability that none of the \(N\) prespecified elements are included
  in the sample, if sampling is
  \begin{enumerate}[label=(\alph*),noitemsep]
  \item with replacement;
  \item without replacement.
  \end{enumerate}
  Compute it for \(r=N=10\), \(n=100\).
\end{problem}
\begin{solution*}
  For part (a) there are a total of \(n^r\) possible draws (this is the
  order of the sample space). There are \(n-N\) elements that have not been
  prespecified and \((n-N)^r\) ways to draw them. Therefore, the
  probability of the event \(A\) that none of the \(N\) prespecified
  elements are included in the sample with replacement is
  \[
    P(A)=\frac{(n-N)^r}{n^r}.
  \]
  \\\\
  For part (b) there are a total of \(\binom{n}{r}\) possible draws (this
  is the order of the sample space). Again, there are \(n-N\) elements that
  have not been prespecified and \(\binom{n-N}{r}\) ways to draw
  them. Thus, the probability of the event \(A\) that none of the \(N\)
  prespecified elements are included in the sample without replacement is
  \[
    P(A)=\frac{\binom{n-N}{r}}{\binom{n}{r}}.
  \]
\end{solution*}

\begin{problem}[Handout 1, \# 11]
  Let \(E\), \(F\), and \(G\) be three events. Find expressions for the
  following events:
  \begin{enumerate}[label=(\alph*),noitemsep]
  \item only \(E\) occurs;
  \item both \(E\) and \(G\) occur, but not \(F\);
  \item all three occur;
  \item at least one of the events occurs;
  \item at most two of them occur.
  \end{enumerate}
\end{problem}
\begin{solution*}
  The solution to this problem reduces to naive set theory.
  \\\\
  For part (a), the event that only \(E\) occurs is the event
  \[
    E\cap\lnot F\cap \lnot G.
  \]
  \\\\
  For part (b), the event that both \(E\) and \(G\) occur, but not \(F\) is
  \[
    E\cap G\cap \lnot F.
  \]
  \\\\
  For part (c), the event that all three of \(E\), \(F\), and \(G\) occur
  is
  \[
    E\cap F\cap G.
  \]
  \\\\
  For part (d), the event that at least one of \(E\), \(F\), or \(G\)
  occur is
  \[
    E\cup F\cup G.
  \]
  \\\\
  For part (e), the event that at most two ocur is the event that not all
  three occur, this is
  \[
    \lnot(E\cap F\cap G)=\lnot E\cup\lnot F\cup\lnot G.
  \]
\end{solution*}

\begin{problem}[Handout 1, \# 12]
  Which is more likely:
  \begin{enumerate}[label=(\alph*),noitemsep]
  \item Obtaining at least one six in six rolls of a fair die, or
  \item Obtaining at least one double six in six rolls of a pair of fair
    dice.
  \end{enumerate}
\end{problem}
\begin{solution*}
  For part (a), we compute the probability of the complement of the event,
  i.e., the event that we do not roll a \(6\) in \(6\) rolls of a fair
  die. This is
  \[
    P(A)=1-P(\lnot
    A)=1-\left(\frac{5}{6}\right)^6\approx\num{0.6651020233196159}.
  \]
  \\\\
  For part (b), we, again, compute the probability of the complement of the
  event, i.e., the event that no double \(6\) is rolled. This is
  \[
    P(A)=1-P(\lnot
    A)=1-\left(\frac{35}{36}\right)^6\approx\num{0.15551243015966854}.
  \]
  \\\\
  Therefore, it is more likely that you roll a \(6\) in \(6\) rolls of a
  fair die than that you roll a double \(6\) in \(6\) rolls of a pair of
  fair dice.
\end{solution*}

\begin{problem}[Handout 1, \# 13]
  There are \(n\) people are lined up at random for a photograph. What is
  the probability that a specified set of \(r\) people happen to be next to
  each other?
\end{problem}
\begin{solution*}
  The \(r\) specified people can stand as a group starting at positions
  \(1,2,\dotsc,n-r+1\). They can be permuted among themselves in \(r!\)
  ways. The remaining \(n-r\) people can be permuted among themselves in
  \((n-r)!\) ways. Thus, we have
  \[
    P=\frac{(n-r+1)r!(n-r)!}{n!}=\frac{(n-r+1)!r!}{n!}.
  \]
\end{solution*}

\begin{problem}[Handout 1, \# 16]
  Consider a particular player, say North, in a Bridge game. Let \(X\) be
  the number of aces in his hand. Find the distribution of \(X\).
\end{problem}
\begin{solution*}
  The number of aces in North's hand \(X\) is a random variable which takes
  integer values between \(0\) and \(4\). Therefore, we must compute the
  probabilities \(P(X=x)\), \(0\leq x\leq 4\).

  From a deck with \(52\) cards, \(13\) cards can be selected in
  \(\binom{52}{13}\) ways among which the player will have \(x\) number of
  aces in \(\binom{4}{x}\binom{48}{13-x}\) ways. Thus, the distribution is
  given by
  \[
    P(X=x)=\frac{\binom{4}{x}\binom{48}{13-x}}{\binom{52}{13}}.
  \]
  The explicit values are approximately
  \begin{align*}
    P(X=0)&\approx 0.304,
    &P(X=1)&\approx 0.439,\\
    P(X=2)&\approx 0.213,
    &P(X=3)&\approx 0.041,\\
    P(X=4)&\approx 0.003.
  \end{align*}
\end{solution*}

\begin{problem}[Handout 1, \# 20]
  If \(100\) balls are distributed completely at random into \(100\) cells,
  find the expected value of the number of empty cells.

  \noindent Replace \(100\) by \(n\) and derive the general expression. Now
  approximate it as \(n\) tends to \(\infty\).
\end{problem}
\begin{solution*}
  Let \(X\) be the random variable denoting the number of empty
  cells. Define indicator variables \(X_1,\dotsc,X_n\) by
  \[
    X_i=
    \begin{cases}
      1&\text{if the \(i\)\textsup{th} cell is empty,}\\
      0&\text{otherwise.}
    \end{cases}
  \]
  Then
  \[
    X=\sum_{i=1}^n X_i
  \]
  and so the mean of this random variable is given by
  \begin{align*}
    E(X)
    &=\sum_{i=1}^n E(X_i)\\
    &=\sum_{i=1}^n P(X_i=1)\\
    &=\sum_{i=1}^n\frac{(n-1)^n}{n^n}\\
    &=n\left(1-\frac{1}{n}\right)^n,
  \end{align*}
  which approaches \(\infty\) as \(n\to\infty\) since
  \(\bigl(1-\frac{1}{n}\bigr)^n\to \rme^{-1}\).
  \\\\
  For \(n=100\), we have \(E(X)\approx 36.60\).
\end{solution*}

%%% Local Variables:
%%% mode: latex
%%% TeX-master: "../MA519-HW-ALL"
%%% End:
