\section{Homework Solutions}
\subsection{Homework 1}
\begin{problem}[Handout 1, \# 5]
  A closet contains five pairs of shoes. If four shoes are selected at
  random, what is the probability that there is at least one complete pair
  among the four?
\end{problem}
\begin{solution}
  Let \(A\) denote the event that ``at least \(1\) complete pair is among
  the \(4\) selected shoes.'' We compute the probability of \(\lnot A\) the
  event that there is no complete pair among the \(4\) selected shoes.

  We can choose \(4\) shoes from the \(5\) pairs in \(\binom{10}{4}=210\)
  ways (this is the order of the sample space), in order that we do not
  select a pair, we choose one of \(2\) a left or a right from \(4\) pairs
  of shoes which can be chosen in \(\binom{5}{4}\) ways.

  Thus,
  \begin{align*}
    P(A)
    &=1-P(\lnot A)\\
    &=1-\frac{\binom{5}{4}2^4}{\binom{10}{4}}\\
    &=1-\frac{8}{21}\\
    &\approx\boxed{\num{0.6190476190476191}.}
  \end{align*}
\end{solution}

\begin{problem}[Handout 1, \# 7]
  A gene consists of \(10\) subunits, each of which is normal or
  mutant. For a particular cell, there are \(3\) mutant and \(7\) normal
  subunits. Before the cell divides into \(2\) daughter cells, the gene
  duplicates. The corresponding gene of cell \(1\) consists of \(10\)
  subunits chosen from the \(6\) mutant and \(14\) normal units. Cell \(2\)
  gets the rest. What is the probability that one of the cells consists of
  all normal subunits.
\end{problem}
\begin{solution}
  Let \(A\) denote the event that at least one of the cells consists of all
  normal subunits. Assuming that these cell units can be arranged in any
  order, each new cell can get \(10\) of the \(20\) subunits in
  \(\binom{20}{10}=184756\) ways. Suppose cell \(1\) receives all \(14\) of
  the normal subunits. There are \(\binom{14}{10}=1001\) ways to do
  this. Since we can do this for cell \(2\) as well, we have
  \[
    P(A)=\frac{2\binom{14}{10}}{\binom{20}{10}}\approx\boxed{\num{0.0108359133}.}
  \]
\end{solution}

\begin{problem}[Handout 1, \# 9]
  From a sample of size \(n\), \(r\) elements are sampled at random. Find
  the probability that none of the \(N\) prespecified elements are included
  in the sample, if sampling is
  \begin{enumerate}[label=(\alph*)]
  \item with replacement;
  \item without replacement.
  \end{enumerate}
  Compute it for \(r=N=10\), \(n=100\).
\end{problem}
\begin{solution}
\end{solution}

\begin{problem}[Handout 1, \# 11]
  Let \(E\), \(F\), and \(G\) be three events. Find expressions for the
  following events:
  \begin{enumerate}[label=(\alph*),noitemsep]
  \item only \(E\) occurs;
  \item both \(E\) and \(G\) occur, but not \(F\);
  \item all three occur;
  \item at least one of the events occurs;
  \item at most two of them occur.
  \end{enumerate}
\end{problem}
\begin{solution}
\end{solution}
\newpage

\begin{problem}[Handout 1, \# 12]
  Which is more likely:
  \begin{enumerate}[label=(\alph*),noitemsep]
  \item Obtaining at least one six in six rolls of a fair die, or
  \item Obtaining at least one double six in six rolls of a pair of fair
    dice.
  \end{enumerate}
\end{problem}
\begin{solution}
\end{solution}

\begin{problem}[Handout 1, \# 13]
  There are \(n\) people are lined up at random for a photograph. What is
  the probability that a specified set of \(r\) people happen to be next to
  each other?
\end{problem}
\begin{solution}
\end{solution}

\begin{problem}[Handout 1, \# 16]
  Consider a particular player, say North, in a Bridge game. Let \(X\) be
  the number of aces in his hand. Find the distribution of \(X\).
\end{problem}
\begin{solution}
\end{solution}

\begin{problem}[Handout 1, \# 20]
  If \(100\) balls are distributed completely at random into \(100\) cells,
  find the expected value of the number of empty cells.

  \noindent Replace \(100\) by \(n\) and derive the general expression. Now
  approximate it as \(n\) tends to \(\infty\).
\end{problem}
\begin{solution}
\end{solution}

%%% Local Variables:
%%% mode: latex
%%% TeX-master: "../MA519-HW-ALL"
%%% End:
