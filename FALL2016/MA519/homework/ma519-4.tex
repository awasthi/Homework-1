\begin{problem}[Handout 5, \# 2]
  In an urn, there are \(12\) balls. \(4\) of these are white. Three
  players: \(A\), \(B\), and \(C\), take turns drawing a ball from the urn,
  in the alphabetical order. The first player to draw a white ball is the
  winner. Find the respective winning probabilities: assume that at each
  trial, the ball drawn in the trial before is put back into the urn (i.e.,
  selection \emph{with replacement}).
\end{problem}
\begin{solution}
  Denote the events that player $A$ wins, player $B$ wins, and player $C$
  wins by $A$, $B$, and $C$ respectively.

  Note that $A$ wins if some multiple of 3 losses occurs, followed by a
  win. Also, $B$ wins if some multiple of three losses occurs, followed by
  a loss, then a win. Last, $C$ wins if some multiple of three losses
  occurs, followed by two losses then a win.

  A win occurs with probability $1/3$ each time, and a loss occurs with
  probability $2/3$ each time. Thus, because each draw is independent,
  \begin{align*}
    P(A) &= \sum\limits_{i=0}^\infty \frac{1}{3} \left(\frac{2}{3}\right)^{3i}, &
    P(B) &= \sum\limits_{i=0}^\infty \frac{1}{3} \left(\frac{2}{3}\right)^{3i+1}, &
    P(C) &= \sum\limits_{i=0}^\infty \frac{1}{3} \left(\frac{2}{3}\right)^{3i+2}.
  \end{align*}
  That is,
  \begin{align*}
    P(A) &= \sum\limits_{i=0}^\infty \frac{9}{19} \approx 0.47,&
    P(B) &= \sum\limits_{i=0}^\infty \frac{6}{19} \approx 0.32,&
    P(C) &= \sum\limits_{i=0}^\infty \frac{4}{19} \approx 0.21.
  \end{align*}
\end{solution}
\newpage

\begin{problem}[Handout 5, \# 8]
  Consider \(n\) families with \(4\) children each. How large must \(n\) be
  to have a \(90\%\) probability that at least \(3\) of the \(n\) families
  are all girl families?
\end{problem}
\begin{solution}
  The probability that a family has all girls is $(0.5)^4$.

  In $n$ families, the probability that at least $3$ are all-girl is
  $1-P(0)-P(1)-P(2)$, where $P(m)$ is the probability that exactly $m$
  families are all-girl.

  Note that
  \begin{align*}
    P(0) &= (1-(0.5)^4)^n \\
    P(1) &= \binom{n}{1}(0.5)^4(1-(0.5)^4)^{n-1} \\
    P(2) &= \binom{n}{2}(0.5)^8(1-(0.5)^4)^{n-2}
  \end{align*}
  (we choose a number of families to be all-girl, and then find the
  probability that that family is all-girl, while all of the rest are not
  all-girl).

  So that the probability that at least $3$ are all girl is
  \[
    1-\bigl(1-(0.5)^4\bigr)^n - n(0.5)^4\bigl(1-(0.5)^4\bigr)^{n-1} -
    n(n-1)(0.5)(0.5)^8\bigl(1-(0.5)^4\bigr)^{n-2}.
  \]
\end{solution}
\newpage

\begin{problem}[Handout 5, \# 10]
  \emph{(Yahtzee).} In Yahtzee, five fair dice are rolled. Find the
  probability of getting a Full House, which is three rolls of one number
  and two rolls of another, in Yahtzee.
\end{problem}
\begin{solution}
  There are $30$ different kinds of full house ($6$ different three of a
  kinds, and $5$ different kinds of different two of a kind).

  The probability of rolling a specific kind of full house is
  \[
    \binom{5}{3} \left( \frac{1}{6}\right)^3 \left(\frac{1}{6} \right)^2
  \]
  (Choose 3 dice to be the three of a kind, have them all rolled the
  specific number, have the other two dice rolled the other specific
  number.)

  So the probability of rolling some kind of full house is
  \[
    30 \binom{5}{3} \left( \frac{1}{6}\right)^3 \left(\frac{1}{6}
    \right)^2 = \frac{25}{648} \approx 0.039.
  \]
\end{solution}
\newpage

\begin{problem}[Handout 5, \# 12]
  The probability that a coin will show all heads or all tails when tossed
  four times is \(0.25\). What is the probability that it will show two
  heads and two tails?
\end{problem}
\begin{solution}
  Let $H$ denote the event that, on a given coin toss, that coin is heads,
  and let $T$ denote the event that, on a given coin toss, that coin is
  tails. Then, because each coin toss is independent, this says that
  \[
    P(H)^4 + P(T)^4 = \frac{1}{4}.
  \]
  Moreover, because this is a coin,
  \[
    P(H) + P(T) = 1.
  \]

  So,
  \begin{align*}
    P(H)^4 + (1-P(H))^4 &= \frac{1}{4} \\
    2P(H)^4 -4P(H)^3 + 6P(H)^2 - 4P(H) +1 &= \frac{1}{4} \\
  \end{align*}
  We can approximate a solution to the above by $P(H) \approx 0.299$ or by
  $P(H) \approx 0.701$.

  Now, the probability that two coin flips of four are heads is
  \[
    \binom{4}{2}P(H)^2P(T)^2 \approx 0.263.
  \]
\end{solution}
\newpage

\begin{problem}[Handout 5, \# 13]
  Let the events \(A_1, A_2,\dotsc,A_n\) be independent and
  \(P(A_k)=p_k\). Find the probability \(p\) that none of the events
  occurs.
\end{problem}
\begin{solution}
  Let the events $A_1, A_2,\dotsc, A_n$ be independent. Then the events
  $A_1^\sim, A_2^\sim, \dotsc, A_n^\sim$ are independent. (Where $A$
  denotes the event that $A$ does \emph{not} happen, i.e.,
  \(A^\sim=\Omega\setminus A\)). The events $A_i^\sim$ each have
  probability $1-p_i$ of occurring. Thus, the probability, $p$, that none
  of the events occur is
  \[
    p = \prod_{k=1}^n (1-p_k)
  \]
\end{solution}
\newpage

\begin{problem}[Handout 6, \# 5]
  Suppose a fair die is rolled twice and suppose \(X\) is the absolute
  value of the difference of the two rolls. Find the PMF and the CDF of
  \(X\) and plot the CDF. Find a median of \(X\); is the median unique?
\end{problem}
\begin{solution}
  First, we compute the probability mass function. Note that $X$ is an
  integer between $0$ and $5$, so computing the probilities that $X$ is
  each of $0$ through $5$ suffices to describe the PMF of $X$.

  We calculate the probabilities, by counting.
  \begin{align*}
    P(0) &= \frac{6}{36} = \frac{1}{6}\\
    P(1) &= \frac{10}{36} = \frac{5}{18}\\
    P(2) &= \frac{8}{36} = \frac{2}{9}\\
    P(3) &= \frac{6}{36} = \frac{1}{6}\\
    P(4) &= \frac{4}{36} = \frac{1}{9}\\
    P(5) &= \frac{2}{36} = \frac{1}{18}\\
  \end{align*}
  We calculate the CDF by summing.
  \begin{align*}
    \CDF(0) &= \frac{6}{36} = \frac{1}{6}\\
    \CDF(1) &= \frac{16}{36} = \frac{4}{9}\\
    \CDF(2) &= \frac{24}{36} = \frac{2}{3}\\
    \CDF(3) &= \frac{30}{36} = \frac{5}{6}\\
    \CDF(4) &= \frac{34}{36} = \frac{17}{18}\\
    \CDF(5) &= \frac{36}{36} = 1\\
  \end{align*}
  The median is $2$.
\end{solution}
\newpage

\begin{problem}[Handout 6, \# 7]
  Find a discrete random variable \(X\) such that \(E(X)=E(X^3)=0\);
  \(E(X^2)=E(X^4)=1\).
\end{problem}
\begin{solution}
  Set \(\Omega=\{0,1\}\) and define a random variable
  \(X\colon\Omega\to\bbR\) by \(X(0)= -1\), \(X(1)= 1\) as well as a
  probability \(P(0)=P(1)=1/2\). Then
  \[
    \rmE(X)=-1(1/2)+1(1/2)=0=(-1)^3(1/2)+1^3(1/2)=\rmE(X^3),
  \]
  whereas
  \[
    \rmE(X^2)=(-1)^2(1/2)+1^2(1/2)=1=(-1)^4(1/2)+1^4(1/2)=\rmE(X^4),
  \]
  as desired.
\end{solution}
\newpage

\begin{problem}[Handout 6, \# 9]
  \emph{(Runs).} Suppose a fair die is rolled \(n\) times. By using the
  indicator variable method, find the expected number of times that a six
  is followed by at least two other sixes. Now compute the value when
  \(n=100\).
\end{problem}
\begin{solution}
  Let \(\Omega\) denote the sample space and define a random variable
  \(X\colon\Omega\to\bbR\) by \(x\mapsto k\) the number of times a
  subsequence of three or more consecutive sixes shows up in \(x\). To
  compute the mean value of \(X\), we must first find the PMF of \(X\) to
  assign probabilities to the various possible outcomes.

  Fix an integer \(k\leq n\). We must count all possible sequences
  \(x\in\Omega\) which contain exactly \(k\) subsequences of three or more
  consecutive sixes.
\end{solution}
\newpage

\begin{problem}[Handout 6, \# 10]
  \emph{(Birthdays).} For a group of \(n\) people find the expected number
  of days of the year which are birthdays of exactly \(k\) people. (Assume
  \(365\) days and that all arrangements are equally probable.)
\end{problem}
\begin{solution}
  Let \(\Omega\) denote the sample space and \(X\colon\Omega\to\bfR\).
\end{solution}
\newpage

\begin{problem}[Handout 6, \# 11]
  \emph{(Continuation).} Find the expected number of multiple
  birthdays. How large should \(n\) be to make this expectation exceed
  \(1\)?
\end{problem}
\begin{solution}

\end{solution}
\newpage

\begin{problem}[Handout 6, \# 12]
  \emph{(The blood-testing problem).} A large number, \(N\), of people are
  subject to a blood test. This can be administered in two ways, (i) Each
  person can be tested separately. In this case \(N\) tests are required,
  (ii) The blood samples of \(k\) people can be pooled and analyzed
  together. If the test is negative, this one test suffices for the \(k\)
  people. If the test is positive, each of the \(k\) persons must be tested
  separately, and in all \(k+1\) tests are required for the \(k\)
  people. Assume the probability \(p\) that the test is positive is the
  same for all people and that people are stochastically independent.
  \begin{itemize}[noitemsep]
  \item[(b)] What is the expected value of the number, \(X\), of tests
    necessary under plan (ii)?
  \item[(c)] Find an equation for the value of \(k\) which will minimize
    the expected number of tests under the second plan. (Do not try
    numerical solutions.)
  \end{itemize}
  \end{problem}
\begin{solution}

\end{solution}
\newpage

\begin{problem}[Handout 6, \# 13]
  \emph{(Sample structure).} A population consists of \(r\) (classes whose
  sizes are in the proportion \(p_1:p_2:\dotsb:p_r\). A random sample of
  size \(n\) is taken with replacement. Find the expected number of classes
  not represented in the sample.
\end{problem}
\begin{solution}

\end{solution}

%%% Local Variables:
%%% mode: latex
%%% TeX-master: "../MA519-Current-HW"
%%% End:
