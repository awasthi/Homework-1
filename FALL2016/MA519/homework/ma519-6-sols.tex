\subsection{Homework 6}
\begin{problem}[Handout 8, \# 2]
  Identify the parameters \(n\) and \(p\) for each of the following
  binomial distributions:
  \begin{enumerate}[label=(\alph*),noitemsep]
  \item \(\#\) boys in a family with \(5\) children;
  \item \(\#\) correct answers in a multiple choice test if each
    question has a \(5\) alternatives, there are \(25\) questions, and the
    student is making guesses at random.
  \end{enumerate}
\end{problem}
\begin{solution}
\end{solution}

\begin{problem}[Handout 8, \# 10]
  A newsboy purchases papers at \(20\) cents and sells them for \(35\)
  cents. He cannot return unsold papers. If the daily demand for papers is
  modeled as a \(\Bin(50,0.5)\) random variable, what is the optimum
  number of papers the newsboy should purchase?
\end{problem}
\begin{solution}
\end{solution}

\begin{problem}[Handout 8, \# 12]
  How many independent bridge dealings are required in order for the
  probability of a preassigned player having four aces at least once to be
  \(1/2\) or better? Solve again for some player instead of a given one.
\end{problem}
\begin{solution}
\end{solution}

\begin{problem}[Handout 8, \# 13]
  A book of \(500\) pages contains \(500\) misprints. Estimate the chances
  that a given page contains at least three misprints.
\end{problem}
\begin{solution}
\end{solution}


\begin{problem}[Handout 8, \# 14]
  Colorblindness appears in \(1\) per cent of the people in a certain
  population. How large must a random sample (with replacements) be if the
  probability of its containing a colorblind person is to be \(0.95\) or
  more?
\end{problem}
\begin{solution}
\end{solution}

\begin{problem}[Handout 8, \# 15]
  Two people toss a true coin \(n\) times each. Find the probability that
  they will score the same number of heads.
\end{problem}
\begin{solution}
\end{solution}

\begin{problem}[Handout 8, \# 16]
  \emph{(Binomial approximation to the hypergeometric distribution).} A
  population of TV elements is divided into red and black elements in the
  proportion \(p:q\) (where \(p+q=1\)). A sample of size \(n\) is taken
  without replacement. The probability that it contains exactly \(k\) red
  elements is given by the hypergeometric distribution of II, 6. Show that
  as \(n\to\infty\) this probability approaches \(\Bin(n,p)\).
\end{problem}
\begin{solution}
\end{solution}

\begin{problem}[Handout 9, \# 3]
  Suppose \(X\), \(Y\), \(Z\) are mutually independent random variables,
  and \(E(X)=0\), \(E(Y)=-1\), \(E(Z)=1\), \(E(X^2)=4\), \(E(Y^2)=3\),
  \(E(Z^2)=10\). Find the variance and the second moment of
  \(2Z-Y/2+\rme X\), where \(\rme\) is the number such that \(\ln \rme=1\).
\end{problem}
\begin{solution}
\end{solution}

\begin{problem}[Handout 9, \# 14]
  \emph{(Variance of Product).} Suppose \(X\), \(Y\) are independent random
  variables. Can it ever be true that \(\Var(XY)=\Var(X)\Var(Y)\)?

  \noindent If it can, when?
\end{problem}
\begin{solution}
\end{solution}

%%% Local Variables:
%%% mode: latex
%%% TeX-master: "../MA519-HW-ALL"
%%% End:
