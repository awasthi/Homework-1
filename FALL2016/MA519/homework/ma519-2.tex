\begin{problem}[Handout 2, \# 5]
  Four men throw their watches into the sea, and the sea brings each man
  one watch back at random. What is the probability that at least one man
  gets his own watch back?
\end{problem}
\begin{solution}
  Let \(\Omega\) denote the sample space and let \(A\) denote the event
  that at least one man gets his own watch back. Since the order of
  distributing wallets to each man matters,
  \[
    \#\Omega=4!=24.
  \]
  Since, at least in this case, it is easier to list the ways in which no
  man gets his own watch back, we will compute the probability of the
  complement of \(A\), i.e., the probability that no man gets his own watch
  back. We count this in the following ways

  In this case, we can explicitly list the ways in which no man gets his
  own watch back: We can translate, i.e., each man receives the \(i\)-th
  man to the left's watch (there are \(3\) ways to do this before the
  everybody receives their own watch back); and we can transpose watches
  between any two any two men (there are \(3\) ways to do this as choosing
  one pair of men to trade watches determines the other pair). In summary,
  there are \(6\) ways for each man to not get his own wallet back. Thus,
  \[
    P(\Omega\setminus A)=\frac{6}{24}=0.25
  \]
  and hence
  \[
    P(A)=1-P(\Omega\setminus A)=0.75.
  \]
\end{solution}
\newpage

\begin{problem}[Handout 2, \# 7]
  Calculate the probability that in Bridge, the hand of at least one player
  is void in a particular suit.
\end{problem}
\begin{solution}
  Let \(\Omega\) denote the sample space and \(A\) denote the event that at
  least one player is void in a particular suit. As in the previous
  problem, it is easier to compute the probability of \(\Omega\setminus A\)
  the event that no player is void in a particular suit. But first, we
  count the number of sample points in \(\Omega\)
  \[
    \#\Omega=\binom{52}{13,13,13,13}
    =53644737765488792839237440000.
  \]

  Now, partition the deck into the \(4\) suits and from each deck draw a
  card to put into each player's hand. For the first player, this gives us
  \[
    13^4\cdot\binom{36}{9}=2688826220080
  \]
  potential hands for the first player
  \[
    12^4\cdot{27}{9}=97186003200
  \]
  for the second and
  \begin{align*}
    11^4\cdot\binom{18}{9}&=711845420
    &10^4\cdot\binom{9}{9}&=10000.
  \end{align*}
  Therefore, we have
  \[
    \#(\Omega\setminus A)=2786724078700
  \]
  total hands where each player has a card from each suit. Thus,
  \[
    P(A)=1-P(\Omega\setminus
    A)=1-\frac{2786724078700}{53644737765488792839237440000}\approx 1.
  \]
\end{solution}
\newpage

\begin{problem}[Handout 2, \# 12]
  If \(n\) balls are placed at random into \(n\) cells, find the
  probability that exactly one cell remains empty.
\end{problem}
\begin{solution}
  Let $n$ balls be placed at random into $n$ cells. (As was pointed out in
  the feedback, we will assume that these $n$ balls are distinct.)

  There are $n^n$ ways to do this.

  There are
  \[
    \binom{n}{1} \cdot \binom{n}{2} \cdot \binom{n-1}{1} \cdot (n-2)!
  \]
  ways to put $n$ balls into $n$ slots with exactly one remaining empty (we
  pick $1$ of our $n$ slots to be empty, we pick $2$ of our $n$ balls to
  share a slot, we pick one of our remaining $n-1$ slots for those $2$
  balls to go into, we fill the rest with one ball each.)

  Thus, the probability that exactly $1$ cell remains empty is
  \begin{align*}
    \frac{\binom{n}{1} \cdot \binom{n}{2} \cdot \binom{n-1}{1} \cdot (n-2)!}{n^n} &= \frac{n \cdot \binom{n}{2} \cdot (n-1) \cdot (n-2)!}{n^n} \\
                                                                                  &= \frac{n! \cdot \frac{n-1}{2}}{n^n} \\
\end{align*}
\end{solution}
\newpage


\begin{problem}[Handout 2, \# 13]
  \emph{Spread of rumors.} In a town of \(n+1\) inhabitants, a person tells
  a rumor to a second person, who in turn repeats it to a third person,
  etc. At each step the recipient of the rumor is chosen at random from the
  \(n\) people available. Find the probability that the rumor told \(r\)
  times without:
  \begin{enumerate}[label=(\alph*),noitemsep]
  \item returning to the originator,
  \item being repeated to any person.
  \end{enumerate}
  Do the same problem when at each step the rumor told by one person to a
  gathering of \(N\) randomly chosen people. (The first question is the
  special case \(N=1\)).
\end{problem}
\begin{solution}
  For part (a), the originator can chose from among \(n\) different people
  and at each step, each person can chose from among \(n-1\) people
  (excluding the originator). This gives us a probability of
  \[
    \frac{n(n-1)^r}{n^r}=\left(\frac{n-1}{n}\right)^{r-1}
  \]
  of the rumor not returning to the originator.

  For part (b), the originator can chose from among \(n\) different people
  and at each step \(i\) each person can chose from among \(n-i+1\) people
  excluding the people who told him the rumor. This gives us a probability
  of
  \[
    \frac{n(n-1)\dotsc(n-r+1)}{n^r}
  \]
  of the rumor not being repeated to anybody.

  Lastly,
\end{solution}
\newpage

\begin{problem}[Handout 2, \# 14]
  \emph{A family problem.} In a certain family four girls take turns at
  washing dishes. Out of a total of four breakages, three were caused by
  the youngest girl, and she was thereafter called clumsy. Was she
  justified in attributing the frequency of breakages to chance? Discuss
  the connection with random placement of balls.
\end{problem}
\begin{solution}
  By assuming that every sample point in our sample space is equally
  likely, we can relate the problem at hand to the probability of the event
  \(A\) that out of four indistinguishable balls, three fall in a specific
  bin, say, the first one. First, \(\#\Omega=4^4\) since we are including
  the possibility that no sister is responsible for breaking the dish. Now,
  there are \(3\) ways to attribute dishes to the youngest sister and the
  last dish can go into one of \(4\) places, the three other sisters or
  none of them. This gives us a probability of
  \[
    P(A)=\frac{4\cdot 3}{4^4}=\frac{4\cdot 3}{265}\approx 0.0469.
  \]
  Since this probability is rather small, the sisters are justified in
  calling her clumsy.
\end{solution}
\newpage

\begin{problem}[Handout 2, \# 15]
  A car is parked among \(N\) cars in a row, not at either end. On his
  return the owner finds exactly \(r\) of the \(N\) places still
  occupied. What is the probability that both neighboring places are empty?
\end{problem}
\begin{solution}
  There are \(r-1\) cars not belonging to the owner with \(N-1\) spots for
  them. Thus, there are \(\binom{N-1}{r-1}\) possible
  arrangements. Out of those arrangements, we insist that that the spots to
  the left and to the right of the car's owner be empty, leaving \(N-3\)
  spots for \(r-1\) cars. This gives us a probability of
  \[
    \frac{\binom{N-3}{r-1}}{\binom{N-1}{r-1}}.
  \]
\end{solution}
\newpage

\begin{problem}[Handout 2, \# 16]
  Find the probability that in a random arrangement of \(52\) bridge card
  no two aces are adjacent.
\end{problem}
\begin{solution}
  We can simplify this problem by using the sticks and stars analogy. There
  are \(4\) aces and, if we want to avoid placing the aces next to each
  other, we count the spaces between the remaining cards \(48\) cards, this
  is \(52-4+1=49\) and place the \(4\) aces in between. This gives a total
  of \(\binom{48}{4}\) arrangements (where the order in which
  we do it does not matter) out \(\binom{52}{4}\) possible
  arrangements including the possibility of two aces being next to each
  other. Thus, the probability that no two aces are adjacent is
  \[
    \frac{\binom{48}{4}}{\binom{52}{4}}\approx 0.7187.
  \]
\end{solution}
\newpage

\begin{problem}[Handout 2, \# 17]
  Suppose \(P(A)=3/4\), and \(P(B)=1/3\).
  \\\\
  Prove that \(P(A\cap B)\geq 1/12\). Can it be equal to \(1/12\)?
\end{problem}
\begin{solution}
  By the inclusion-exclusion principle, we have
  \[
    P(A\cup B)=P(A)+P(B)-P(A\cap B)
  \]
  so
  \[
    P(A\cap B)=P(A)+P(B)-P(A\cup B).
  \]
  But, since \(A\cup B\subseteq\Omega\) and \(P(\Omega)=1\), \(P(A\cup
  B)\leq P(\Omega)=1\) so \(-P(A\cup B)\geq -1\). Thus,
  \begin{align*}
    P(A\cap B)&=P(A)+P(B)-P(A\cup B)\\
              &\geq P(A)+P(B)-1\\
              &=\frac{3}{4}+\frac{1}{3}-1\\
              &=\frac{9}{12}+\frac{4}{12}-\frac{12}{12}\\
              &=\frac{1}{12}.
  \end{align*}

  Lastly, we show that \(P(A\cap B)\) can in fact be equal to
  \(1/12\). Consider the closed unit interval \(I=[0,1]\) equipped with the
  Lebesgue measure. Set \(A=(0,3/4)\) and \(B=(1-1/3,1)=(2/3,1)\). Then the
  intersection \(A\cap B=(2/3,3/4)\) has Lebesgue measure
  \[
    m(A\cap B)=\frac{3}{4}-\frac{2}{3}=\frac{1}{12}.
  \]
\end{solution}
\newpage

\begin{problem}[Handout 2, \# 18]
  Suppose you have infinitely many events \(A_1,A_2,\dotsc\), and each one
  is sure to occur, i.e., \(P(A_i)=1\) for any \(i\).
  \\\\
  Prove that \(P\bigl(\bigcap_{i=1}^n A_i\bigr)=1\).
\end{problem}
\begin{solution}
  Consider the sequence of probabilities \(\{P_n:n\in\bbN\}\) where
  \(P_n=P\bigl(\bigcap_{i=1}^n A_i\bigr)\). Note that
  \(\bigcap_{i=1}^n A_i\downarrow\bigcap_{i=1}^\infty A_i\). First we show,
  by induction, that \(P_n=1\). The case \(n=1\) is trivial. Now, assume
  the result holds for \(n-1\) and consider
  \(P_n=P\bigl(\bigcap_{i=1}^n A_i\bigr)\). Writing
  \(A'=\bigcap_{i=1}^{n-1}A_i\), we have
  \[
    P_n=P(A'\cap A_n).
  \]
  By the inclusion-exclusion principle,
  \begin{align*}
    P_n&=P(A')+P(A_n)-P(A'\cup A_n)\\
       &=1+1-P(A'\cup A_n)
         \intertext{and since \(P(A'\cup A_n)\geq P(A')=1\) by the monotonicity
         of the probability measure, \(P(A'\cup A_n)=1\) since \(P(A'\cup
         A_n)\leq P(\Omega)=1\), thus}
       &=1+1-1\\
       &=1.
  \end{align*}
  It follows that \(\{P_n:n\in\bbN\}\) is the constant sequence
  \(\{1\}\). By Theorem 1.1 from DasGupta, we have
  \[
    P\left(\bigcap\nolimits_{i=1}^\infty A_i\right)=\lim_{n\to\infty} 1=1.
  \]
\end{solution}
\newpage

\begin{problem}[Handout 2, \# 19]
  There are \(n\) blue, \(n\) green, \(n\) red, and \(n\) white balls in an
  urn. Four balls are drawn from the urn with replacement. Find the
  probability that there are balls of at least three different colors among
  the four drawn.
\end{problem}
\begin{solution}
  Each ball has an equal probability of being drawn on each draw. That is,
  each (ordered) set of 4 draws is equally likely to occur.

  There are $4^4$ ways to draw $4$ balls.

  There are $4!$ ways to draw one ball of each color.

  There are $4 \cdot 3 \cdot \binom{4}{2} \cdot 2!$ ways to draw 4 balls,
  missing exactly one color (pick a color to be missed, pick a color to be
  drawn twice, pick two draws for that color to be drawn on, rearrange the
  other two colors into the other two draws.)

  Thus, the probability of having at least three different colors being
  drawn among those four is
\begin{align*}
  \frac{4! + 4 \cdot 3 \cdot \binom{4}{2} \cdot 2!}{4^4} &= \frac{3!+6 \cdot \frac{4!}{2!2!}}{4^3} \\
                                                         &= \frac{6+36}{4^3}\\
                                                         &\cong 0.65625\\
\end{align*}
\end{solution}

%%% Local Variables:
%%% mode: latex
%%% TeX-master: "../MA519-Current-HW"
%%% End:
