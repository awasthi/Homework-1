\begin{problem}[Handout 2, \# 5]
  Four men throw their watches into the sea, and the sea brings each man
  one watch back at random. What is the probability that at least one man
  gets his own watch back?
\end{problem}
\begin{solution}
  The sample space \(\Omega\) is in correspondence with \(S_4\) the set of
  bijections from the set \(\{1,\dotsc,4\}\) to itself and therefore
  \begin{equation}
    \label{eq:2-1}
    \#\Omega=\# S_4=4!=24.
  \end{equation}
  Let \(A\) denote the event that at least one man gets his own watch
  back. This is a case where it is easier find the probability of the
  complement of \(A\), i.e., the event \(\Omega\setminus A\) that no man
  gets his own watch back.

  By the inclusion-exclusion formula, we have
\end{solution}
\newpage

\begin{problem}[Handout 2, \#7]
  Calculate the probability that in Bridge, the hand of at least one player
  is void in a particular suit.
\end{problem}
\begin{solution}
\end{solution}
\newpage

\begin{problem}[Handout 2, \# 12]
  If \(n\) balls are placed at random into \(n\) cells, find the
  probability that exactly \(1\) cell remains empty.
\end{problem}
\begin{solution}
\end{solution}
\newpage


\begin{problem}[Handout 2, \# 13]
  \emph{Spread of rumors.} In a town of \(n+1\) inhabitants, a person tells
  a rumor to a second person, who in turn repeats it to a third person,
  etc. At each step the recipient of the rumor is chosen at random from the
  \(n\) people available. Find the probability that the rumor told \(r\)
  times without:
  \begin{enumerate}[label=(\alph*),noitemsep]
  \item returning to the originator,
  \item being repeated to any person.
  \end{enumerate}
  Do the same problem when at each step the rumor told by one person to a
  gathering of \(N\) randomly chosen people. (The first question is the
  special case \(N=1\)).
\end{problem}
\begin{solution}

\end{solution}
\newpage

\begin{problem}[Handout 2, \# 14]
  \emph{A family problem.} In a certain family four girls take turns at
  washing dishes. Out of a total of four breakages, three were caused by
  the youngest girl, and she was thereafter called clumsy. Was she
  justified in attributing the frequency of breakages to chance? Discuss
  the connection with random placement of balls.
\end{problem}
\begin{solution}

\end{solution}
\newpage

\begin{problem}[Handout 2, \# 15]
  A car is parked among \(N\) cars in a row, not at either end. On his
  return the owner finds exactly \(r\) of the \(N\) places still
  occupied. What is the probability that both neighboring places are empty?
\end{problem}
\begin{solution}

\end{solution}
\newpage

\begin{problem}[Handout 2, \# 16]
  Find the probability that in a random arrangement of \(52\) bridge card
  no two aces are adjacent.
\end{problem}
\begin{solution}

\end{solution}
\newpage

\begin{problem}[Handout 2, \# 17]
  Suppose \(P(A)=3/4\), and \(P(B)=1/3\).

  Prove that \(P(A\cap B)\geq 1/12\). Can it be equal to \(1/12\)?
\end{problem}
\begin{solution}

\end{solution}
\newpage

\begin{problem}[Handout 2, \# 18]
  Suppose you have infinitely many events \(A_1,A_2,\dotsc\), and each one
  is sure to occur, i.e., \(P(A_i)=1\) for any \(i\).
  \\\\
  Prove that \(P\bigl(\bigcap_{i=1}^n A_i\bigr)=1\).
\end{problem}
\begin{solution}

\end{solution}
\newpage

\begin{problem}[Handout 2, \# 19]
  There are \(n\) blue, \(n\) green, \(n\) red, and \(n\) white balls in an
  urn. Four balls are drawn from the urn with replacement. Find the
  probability that there are balls of at least three different colors among
  the four drawn.
\end{problem}
\begin{solution}

\end{solution}

%%% Local Variables:
%%% mode: latex
%%% TeX-master: "../MA519-Current-HW"
%%% End:
