\begin{problem}[DasGupta, 1.5]
  The population of Danville is \(20000\). Can it be said with certainty
  that there must be two or more people in Danville with exactly the same
  three initials?
\end{problem}
\begin{solution}
  Yes. To see this we provide the following purely combinatorial
  argument. Let \(A\) denote the event that two or more people in Danville
  have exactly the same three initials and consider its converse
  \(A^\sfc\), that is, the event that no two people in Danvillle have
  exactly the same three initials. Since the Latin alphabet consists of
  \(26\) letters to any given person we may assign one of \(26^3=17576\)
  possible initials. Since there are \(20000\) people living in Danville by
  the pigeon-hole principle there must be two people (in fact, as many as
  \(20000-17576=2424\)) with the same three initials.
\end{solution}
\newpage

\begin{problem}[DasGupta, 1.7]
  Let \(E\), \(F\), and \(G\) be three events. Find expressions for the
  following events:
  \begin{enumerate}[label=(\alph*),noitemsep]
  \item only \(E\) occurs;
  \item both \(E\) and \(G\) occur, but not \(F\);
  \item all three occur;
  \item at least one of the events occurs;
  \item at most two of them occur.
  \end{enumerate}
\end{problem}
\begin{solution}
  The following equalities can be derived from the axioms of set theory.
  \\\\
  For part (a), the event that only \(E\) occurs is equivalent to the
  collection of sample points in \(E\) not in \(F\) and not in \(G\), that
  is,
  \[
    \bigl(E\setminus(E\cap G)\bigr)\setminus(E\cap F)
    =E\cap F^\sfc\cap G^\sfc.
  \]

  For part (b), the event that both \(E\) and \(G\) but not \(F\) occur is
  equivalent to the collection of sample points contained in both \(E\) and
  \(G\) that are not in \(F\), that is,
  \[
    (E\cap G)\setminus F=E\cap G\cap F^\sfc.
  \]

  For part (c), the event that all three \(E\), \(F\) and \(G\) occur is
  equal to the collection of sample points contained in all three of \(E\),
  \(G\) and \(F\), that is, their intersection
  \[
    E\cap G\cap F.
  \]

  For part (d), the event that at least one \(E\), \(F\) or \(G\) occur is
  equal to the collection of sample points in any of \(E\), or \(G\) or
  \(F\), that is, their union
  \[
    E\cup G\cup F.
  \]

  For part (e), the event that at most two of \(E\), \(F\) and \(G\) occur
  is the collection of sample points in \(E\) or \(F\) but not \(G\), \(E\)
  or \(G\) but not \(F\), or \(F\) and \(G\) but not \(E\), that is,
  \[
   \bigl((E\cup F)\setminus G\bigr)\cup
   \bigl((E\cup G)\setminus F\bigr)\cup
   \bigl((F\cup G)\setminus E\bigr)
  \]
\end{solution}
\newpage

\begin{problem}[Feller, Prob.\@ 9, p.\@ 55]
  If \(n\) balls are placed at random into \(n\) cells, find the
  probability that exactly \(1\) cell remains empty.
\end{problem}
\begin{solution}
  Let \(\Omega\) denote the sample space and \(A\) denote the event that
  exactly \(1\) cell remains empty. Then the number of sample points in
  \(\Omega\) is
  \[
    \#\Omega=n^n.
  \]
  Now we count the number of points in \(A\): The number of ways in which
  we can choose the empty cell is exactly \(n\) and we have \(n-1\) ways in
  which the remaining
\end{solution}
\newpage

\begin{problem}[Feller, Prob.\@ 21, p.\@ 56]
  \emph{Spread of rumors.} In a town of \(n+1\) inhabitants, a person tells
  a rumor to a second person, who in turn repeats it to a third person,
  \etc{} At each step the recipient of the rumor is chosen at random from the
  \(n\) people available. Find the probability that the rumor told \(r\)
  times without: (a) returning to the originator, (b) being repeated to any
  person. Do the same problem when at each step the rumor told by one
  person to a gathering of \(N\) randomly chosen people. (The first
  question is the special case \(N=1\)).
\end{problem}
\begin{solution}

\end{solution}
\newpage

\begin{problem}[Feller, Prob.\@ 24, p.\@ 56]
  \emph{A family problem.} In a certain family four girls take turns at
  washing dishes. Out of a total of four breakages, three were caused by
  the youngest girl, and she was thereafter called clumsy. Was she
  justified in attributing the frequency of breakages to chance? Discuss
  the connection with random placement of balls.
\end{problem}
\begin{solution}

\end{solution}
\newpage

\begin{problem}[Feller, Prob.\@ 27, p.\@ 57]
  A car is parked among \(N\) cars in a row, not at either end. On his
  return the owner finds exactly \(r\) of the \(N\) places still
  occupied. What is the probability that both neighboring places are empty?
\end{problem}
\begin{solution}

\end{solution}
\newpage

\begin{problem}[Feller, Prob.\@ 42, p.\@ 58]
  Find the probability that in a random arrangement of \(52\) bridge card
  no two aces are adjacent.
\end{problem}
\begin{solution}

\end{solution}
\newpage

\begin{problem}
  Suppose \(P(A)=3/4\), and \(P(B)=1/3\).

  Prove that \(P(A\cap B)\geq 1/12\). Can it be equal to \(1/12\)?
\end{problem}
\begin{solution}

\end{solution}
\newpage

\begin{problem}
  Suppose you have infinitely many events \(A_1,A_2,\dotsc\), and each one
  is sure to occur, i.e., \(P(A_i)=1\) for any \(i\).

  Prove that \(P\bigl(\bigcap_{i=1}^n A_i\bigr)=1\).
\end{problem}
\begin{solution}

\end{solution}
\newpage

\begin{problem}
  There are \(n\) blue, \(n\) green, \(n\) red, and \(n\) white balls in an
  urn. Four balls are drawn from the urn with replacement. Find the
  probability that there are balls of at least three different colors among
  the four drawn.
\end{problem}
\begin{solution}

\end{solution}

%%% Local Variables:
%%% mode: latex
%%% TeX-master: "../MA519-Current-HW"
%%% End:
