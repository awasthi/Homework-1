\begin{problem}[Handout 18, \# 15]
  \((X,Y)\) is distributed uniformly inside of the unit circle. Find the
  density of \(X+Y\) and hence the mean of \(X+Y\). Was the value of the
  mean obvious? Why?
\end{problem}
\begin{solution}
  Suppose the random vector \((X,Y)\) is uniformly distributed in
  \(D\defeq \{\,x^2+y^2<1\,\}\). Let us first find the joint distribution
  of \((X,Y)\). Fix a number \(y\in[0,1]\eqdef I\), then the PDF of \(X\)
  given \(Y=y\) is
  \[
    f_{X\,|\,Y=y}(x)=%
    \begin{cases}
      \frac{1}{2}\sqrt{1-y^2}&\text{for \(x\in I\setminus A\),}\\
      0&\text{otherwise}
    \end{cases}
  \]
  where \(A\defeq \bigl[-1+\sqrt{1-y^2},1-\sqrt{1-y^2}\bigr]\).
\end{solution}
\newpage

\begin{problem}[Handout 18, \# 16]
  Let \(X\) be a random number in \([0,1]\). What is the probability that
  the number \(5\) is completely missing from the decimal expansion of
  \(X\)?
\end{problem}
\begin{solution}
  First let us establish some notation. Let \(\Omega\) denote the interval
  \([0,1]\) and let \(A\) denote the set of all real numbers in \(\Omega\)
  which do not contain a \(5\) in their decimal expansion. We show that the
  probability that \(X\) is in \(A\) is zero; i.e., that \(P(X\in A)=0\).

  To this end, let us consider the following Bernoulli process: Let \(N_k\)
  denote the \(k\)\textsup{th} digit in the decimal expansion of
  \(X\). Then \(N_k\) is uniformly distributed on \(\{0,\dotsc,9\}\) and
  the probability that \(N_k\) is not the number \(5\) is
  \(\frac{9}{10}\). Let \(I_k\) be an indicator random variable given by
  \[
    I_k\defeq%
    \begin{cases}%
      0&\text{for \(N_k=5\),}\\
      1&\text{otherwise.}
    \end{cases}
  \]
  Then the sequence of random variables \(\{I_k\}\) with \(p=\frac{9}{10}\)
  forms a sequence of independent Bernoulli trials wit expected value
  \[
    E\left(\prod\nolimits_{k=1}^n I_k\right)=%
    \prod_{k=1}^n E(I_k)=%
    \prod_{k=1}^n p=p^n.
  \]
  The expectation above is in fact the probability every digit of \(X\)
  from the first to the \(n\)\textsup{th} is not \(5\).

  Taking the limit of the expression above, the probability that \(X\) does
  not contain a \(5\) in its decimal expansion is the limit
  \[
    \lim_{n\to\infty}
    p^n=\lim_{n\to\infty}\left(\frac{9}{10}\right)^n=0.\qedhere
  \]
\end{solution}
\newpage

\begin{problem}[Handout 18, \# 17]
  A foot long stick is broken into three pieces. Find the density functions
  of the length of the longest part, the smallest part, and the medium
  part. What are the expected values for each part?
\end{problem}
\begin{solution}

\end{solution}

%%% Local Variables:
%%% mode: latex
%%% TeX-master: "../MA519-HW-Current"
%%% End:
