\begin{problem}[Handout 16, \# 2]
  A number \(N\) is chosen according to a Poisson distribution with mean
  \(10\). Then \(5\) numbers are chosen from \(\{0,1,\dotsc,N\}\). Suppose
  \(X\) is the maximum of these \(5\) numbers.

  What is \(P(X>10)\)?
\end{problem}
\begin{solution}
\end{solution}
\newpage

\begin{problem}[Handout 18, \# 15]
  \((X,Y)\) is distributed uniformly inside of the unit circle. Find the
  density of \(X+Y\) and hence the mean of \(X+Y\). Was the value of the
  mean obvious? Why?
\end{problem}
\begin{solution}
  Suppose the random vector \((X,Y)\sim
  U\bigl[\{\,(x,y):x^2+y^2<1\,\}\bigr]\).
\end{solution}
\newpage

\begin{problem}[Handout 18, \# 16]
  Let \(X\) be a random number in \([0,1]\). What is the probability that
  the number \(5\) is completely missing from the decimal expansion of
  \(X\)?
\end{problem}
\begin{solution}
  Suppose \(X\) is picked randomly from the interval
  \(\Omega\defeq [0,1]\). Let \(A\) be the set of all real numbers in
  \(\Omega\) without a \(5\) in their decimal expansion. We show that
  \(P(X\in \Omega\setminus A)=1\) by proving that \(P(X\in A)=0\).

  To compute the probability of \(A\) we first decompose \(A\) as the limit
  of \(\bigcap_{k=1}^n A_k\) where
  \[
    A_k\defeq\bigl\{\,x\in\Omega:\text{\(x=0.n_1n_2\dotsm n_k\dotsm\),
      \(n_k=5\)}\,\bigr\}.
  \]
  By properties of the probability measure, \(P(A)=\lim_{k\to\infty}
  P(A_k)\) so we need to find an expression for \(P(A_k)\). Let us begin
  with the case \(k=1\):
\end{solution}
\newpage

\begin{problem}[Handout 18, \# 17]
  A foot long stick is broken into three pieces. Find the density functions
  of the length of the longest part, the smallest part, and the medium
  part. What are the expected values for each part?
\end{problem}
\begin{solution}

\end{solution}

%%% Local Variables:
%%% mode: latex
%%% TeX-master: "../MA519-HW-Current"
%%% End:
