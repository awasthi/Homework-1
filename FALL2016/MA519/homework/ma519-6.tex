\begin{problem}[Handout 8, \# 2]
  Identify the parameters \(n\) and \(p\) for each of the following
  binomial distributions:
  \begin{enumerate}[label=(\alph*),noitemsep]
  \item \(\#\) boys in a family with \(5\) children;
  \item \(\#\) correct answers in a multiple choice test if each
    question has a \(5\) alternatives, there are \(25\) questions, and the
    student is making guesses at random.
  \end{enumerate}
\end{problem}
\begin{solution}
  For part (a), the distribution is binomial with \(k\) being the number of
  children in a given family and \(p\) the probability that a child is
  born, say, male. In this case, we can reasonably assume that
  \(p=0.5\). Thus, the binomial distribution is given by
  \(\Binom(5,0.5)\).
  \\\\
  For part (b), we use similar reasoning and we have \(\Binom(25,0.2)\)
  where \(k=25\) is the number of questions and \(p=1/5=0.2\) the
  probability of guessing a question correctly.
\end{solution}
\newpage

\begin{problem}[Handout 8, \# 10]
  A newsboy purchases papers at \(20\) cents and sells them for \(35\)
  cents. He cannot return unsold papers. If the daily demand for papers is
  modeled as a \(\Binom(50,0.5)\) random variable, what is the optimum
  number of papers the newsboy should purchase?
\end{problem}
\begin{solution}
  Let \(X\sim\Binom(50,0.5)\) denote the daily demand for papers and \(n\)
  the number of copies bought by the newsboy. Then, the random variable
  \(S=\min\{X,n\}\) denotes the number of copies actually sold by the
  newsboy. His daily profit is, therefore, measured by the random variable
  \[
    Y=0.35S-0.25n.
  \]

  Now let us compute the average sales of the newsboy. By the linearity of
  expected value, we have
  \begin{align*}
    E(S)
    &=\sum_{k=0}^nkP(S=k)\\
    &=\sum_{k=0}^{n-1} kP(\min\{X,n\}=k)+nP(\min\{X,n\}=n),
      \intertext{where \(P(\min\{X,n\}=k)=P(X=k)\) the probability that there
      is a demand for \(k\) copies, and \(P(\min\{X,n\}=n)=P(X\geq n)\) the
      probability that the demand exceeds the number of copies the newsboy
      bought, giving us}
    &=\sum_{k=0}^{n-1} kP(X=k)+nP(X\geq n)\\
    &=\sum_{k=0}^{n-1} kP(X=k)+n\bigl(1-P(X<n)\bigr)\\
    &=n+\sum_{k=0}^{n-1}kP(X=k)-nP(X<n)\\
    &=n+\sum_{k=0}^{n-1}kP(X=k)-n\sum_{k=0}^{n-1}P(X=k)\\
    &=n+\sum_{k=0}^{n-1}k\binom{n-1}{k}0.5^k0.5^{n-1-k}
      -n\sum_{k=0}^{n-1}\binom{n-1}{k}0.5^k0.5^{n-1-k}\\
    &=n+\sum_{k=0}^{n-1}k\binom{n-1}{k}0.5^{n-1}
      -n\sum_{k=0}^{n-1}\binom{n-1}{k}0.5^{n-1}\\
    &=n+(n-1)0.5-n0.5^{n-1} 2^{n-1}\\
    &=\frac{n-1}{2}.
  \end{align*}

  Thus,
  \[
    E(Y)=0.35E(S)-0.25n=0.175(n-1)-0.25n=
  \]
\end{solution}
\newpage

\begin{problem}[Handout 8, \# 12]
  How many independent bridge dealings are required in order for the
  probability of a preassigned player having four aces at least once to be
  \(1/2\) or better? Solve again for some player instead of a given one.
\end{problem}
\begin{solution}
  Without loss of generality, assume North is the preassigned player. Let
  \(X\) be the number of dealings required for North to have a probability
  greater than or equal to \(1/2\)
\end{solution}
\newpage

  % Without loss of generality, let us assume it is North that is the
  % preassigned player.

  % If North is dealt first, there is a
  % \[
  %   \frac{\binom{52-4}{13-4}}{\binom{52}{13}}=\frac{\num{1677106640}}{\num{635013559600}}
  %   \approx\num{0.0026410564225690276}
  % \]
  % probability that he is dealt four aces.

  % If North is dealt second, assuming none of the aces have been dealt to
  % the other players, we have a
  % \[
  %   \frac{\binom{52-13-4}{13-4}}{\binom{52-13}{13}}=
  %   \frac{70607460}{8122425444}
  %   \approx\num{0.008692903429745534}
  % \]
  % probability of being dealt four aces.

  % If North is dealt third,
  % \[
  %   \frac{\binom{52-13-13-4}{13-4}}{\binom{52-13-13}{13}}=
  %   \frac{497420}{10400600}
  %   \approx\num{0.04782608695652174}.
  % \]

  % If North is dealt last,
  % \[
  %   \frac{\binom{52-13-13-13-4}{13-4}}{\binom{52-13-13-13}{13}}=1.
  % \]

  % Now, the probability that at the \(i\)\textsup{th} dealing, the four aces
  % are still in the deck is
  % \[
  %   p_i=\left(1-\frac{\binom{52-13(i-1)-4}{13}}{\binom{52-13(i-1)}{13}}\right)p_{i-1}.
  % \]
  % Numerically, these values are
  % \begin{align*}
  %   p_1&\approx\num{0.6961824729891957}
  %   &p_2&\approx\num{0.5696438537482323}\\
  %   p_3&\approx\num{0.5424000172646213}
  %   &p_4&=0.
  % \end{align*}

\begin{problem}[Handout 8, \# 13]
  A book of \(500\) pages contains \(500\) misprints. Estimate the chances
  that a given page contains at least three misprints.
\end{problem}
\begin{solution}
  Let $X$ be the number of misprints on the given page. The probability
  that a given misprint is on that page is $1/500$. Now,
  $P(X \geq 3) = 1- P(X=0) - P(X=1) - P(X=2)$. Also,
  \begin{align*}
    P(X=0) &= \left(\frac{499}{500}\right)^{500}\\
    P(X=1) &= 500\left(\frac{1}{500}\right)\left(\frac{499}{500}\right)^{499} \\
    P(X=2) &= \frac{500 \cdot 499}{2}\left(\frac{1}{500}\right)^2\left(\frac{499}{500}\right)^{498}
  \end{align*}
  So that
  \begin{align*}
    P(X \geq 3) &= 1- P(X=0) - P(X=1) - P(X=2)\\
                &= 1- \left(\frac{499}{500}\right)^{500} - 500\left(\frac{1}{500}\right)\left(\frac{499}{500}\right)^{499} - \frac{500 \cdot 499}{2}\left(\frac{1}{500}\right)^2\left(\frac{499}{500}\right)^{498}\\
                &\approx 0.08
  \end{align*}
  that is, the probability that the given page has at least \(3\) misprints
  is about \(8\) percent.
\end{solution}

\newpage

\begin{problem}[Handout 8, \# 14]
  Colorblindness appears in \(1\) per cent of the people in a certain
  population. How large must a random sample (with replacements) be if the
  probability of its containing a colorblind person is to be \(0.95\) or more?
\end{problem}
\begin{solution}
Let $n$ be the sample size. The probability of the sample containing no colorblind people is $0.99^n$. Solving the equation $0.99^n=0.05$, we see that (for the naturals) taking $n = 299$ is (minimally) sufficient for $0.99^n$ to be less than $0.05$.

The probability that the sample has some colorblind person is equal to $1-0.99^n$. This is at least 95 percent if $0.99^n$ is less than $0.05$. That is, having $299$ people in the sample is (minimally) sufficient for there to be a 95 percent chance of having some colorblind person.
\end{solution}
\newpage

\begin{problem}[Handout 8, \# 15]
  Two people toss a true coin \(n\) times each. Find the probability that
  they will score the same number of heads.
\end{problem}
\begin{solution}

\end{solution}
\newpage

\begin{problem}[Handout 8, \# 16]
  Binomial approximation to the hypergeometric distribution. A population
  of TV elements is divided into red and black elements in the proportion
  \(p:q\) (where \(p+q=1\)). A sample of size \(n\) is taken without
  replacement. The probability that it contains exactly \(k\) red elements
  is given by the hypergeometric distribution of II, 6. Show that as
  \(n\to\infty\) this probability approaches \(\Binom(n,p)\).
\end{problem}
\begin{solution}

\end{solution}
\newpage

\begin{problem}[Handout 9, \# 3]
  Suppose \(X\), \(Y\), \(Z\) are mutually independent random variables,
  and \(E(X)=0\), \(E(Y)=-1\), \(E(Z)=1\), \(E(X^2)=4\), \(E(Y^2)=3\),
  \(E(Z^2)=10\). Find the variance and the second moment of \(2Z-Y/2+e X\),
  where \(e\) is the number such that \(\ln e=1\).
\end{problem}
\begin{solution}

Let $W = 2Z-Y/2+e X$. Then

\begin{align*}
E(W^2) &= E((2Z-Y/2+e X)^2)\\
&= E(4Z^2 -2ZY + 4eZX + Y^2/4 -eYX + e^2X^2)\\
&= 4E(Z^2) -2E(Z)E(Y) + 4eE(Z)E(X) + E(Y^2)/4 -eE(Y)E(X) + e^2E(X^2)\\
&= 4\cdot 10 + 2  + 3/4  + e^2\cdot 4\\
&= \frac{171}{4} + 4e^2\\
&\approx 72.31
\end{align*}

and

\begin{align*}
Var(W) &= E(W^2) - E(W)^2 \\
&= \frac{171}{4} + 4e^2 - (2E(Z) + E(Y)/2 +eE(X))^2\\
&= \frac{171}{4} + 4e^2 - (2 -1/2)^2\\
&= \frac{171}{4} + 4e^2 - (3/2)^2\\
&= \frac{81}{2} +4e^2\\
&\approx 70.06
\end{align*}

That is, the second moment is ``about $72.31$'' and the variance is ``about
$70.06$''

\end{solution}
\newpage

\begin{problem}[Handout 9, \# 14]
  \emph{(Variance of Product).} Suppose \(X\), \(Y\) are independent
  random variables. Can it ever be true that \(\Var(XY)=\Var(X)\Var(Y)\)?

  \noindent If it can, when?
\end{problem}
\begin{solution}

\end{solution}

%%% Local Variables:
%%% mode: latex
%%% TeX-master: "../MA519-HW-Current"
%%% End:
