\begin{problem}[Handout 8, \# 2]
  Identify the parameters \(n\) and \(p\) for each of the following
  binomial distributions:
  \begin{itemize}
  \item[(a)] \(\#\) boys in a family with \(5\) children;
  \item[(b)] \(\#\) correct answers in a multiple choice test if each
    question has a \(5\) alternatives, there are \(25\) questions, and the
    student is making guesses at random.
  \end{itemize}
\end{problem}
\begin{solution}

\end{solution}
\newpage

\begin{problem}[Handout 8, \# 10]
  A newsboy purchases papers at \(20\) \textcent{} and sells them for \(35\)
  \textcent{}. He cannot return unsold papers. If the daily demand for papers
  is modeled as a \(\Binom(50,0.5)\) random variable, what is the optimum
  number of papers the newsboy should purchase?
\end{problem}
\begin{solution}

\end{solution}
\newpage

\begin{problem}[Handout 8, \# 12]
\end{problem}
\begin{solution}

\end{solution}
\newpage

\begin{problem}[Handout 8, \# 13]
\end{problem}
\begin{solution}

\end{solution}
\newpage

\begin{problem}[Handout 8, \# 14]
\end{problem}
\begin{solution}

\end{solution}
\newpage

\begin{problem}[Handout 8, \# 15]
\end{problem}
\begin{solution}

\end{solution}
\newpage

\begin{problem}[Handout 8, \# 16]
\end{problem}
\begin{solution}

\end{solution}
\newpage

\begin{problem}[Handout 9, \# 3]
\end{problem}
\begin{solution}

\end{solution}
\newpage

\begin{problem}[Handout 9, \# 14]
\end{problem}
\begin{solution}

\end{solution}

%%% Local Variables:
%%% mode: latex
%%% TeX-master: "../MA519-HW-Current"
%%% End:
