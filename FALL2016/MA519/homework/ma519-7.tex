\begin{problem}[Handout 10, \# 4]
  \emph{(Poisson Approximation.)} One hundred people will each toss a fair
  coin \(200\) times. Approximate the probability that at least \(10\) of
  the \(100\) people would each have obtained exactly \(100\) heads and
  \(100\) tails.
\end{problem}
\begin{solution}
  Let \(X\) denote the number of people who obtain exactly \(100\) heads
  and (consequently) \(100\) tails. First, we compute the probability that
  any one given person obtains exactly \(100\) heads. There are \(2^{200}\)
  possible outcomes for \(200\) tosses of a fair coin, and
  \(\binom{200}{100}\) possible ways of obtaining exactly \(100\)
  heads. Thus, the probability that any one person obtains exactly \(100\)
  head in \(200\) tosses of a fair coin is
  \[
    p=\frac{\binom{200}{100}}{2^{200}}\approx\num{0.05634847900925642}.
  \]

  Now, assuming \(X\sim\Poisson(\num{5.634847900925642})\), the
  probability that at least \(10\) of the \(100\) people have each obtained
  exactly \(100\) heads and \(100\) tails is
  \begin{align*}
    P(X\geq 10)
    &=1-P(X<10)\\
    &=1-\sum_{i=1}^9 P(X=i)\\
    &=1
      -e^{-\num{5.634847900925642}}
      \sum_{i=0}^9\frac{\num{5.634847900925642}^i}{i!}\\
    &=1-\num{0.8825634032515405}\\
    &\approx \num{0.11743659674845952}.
  \end{align*}
\end{solution}
\newpage

\begin{problem}[Handout 10, \# 5]
  \emph{(A Pretty Question.)} Suppose \(X\) is a Poisson distributed random
  variable. Can three different values of \(X\) have an equal probability?
\end{problem}
\begin{solution}
  No. Let \(X\sim\Poisson(\lambda)\). First, we show that given any two
  values \(k_1,k_2\in\bbZ_{\geq 0}\), there exists \(\lambda\) such that
  \(p(k_1)=p(k_2)\). Observe that for \(p(k_1)=p(k_2)\) we must have
  \begin{align*}
    e^{-\lambda}\frac{\lambda^{k_1}}{k_1!}
    &=e^{-\lambda}\frac{\lambda^{k_2}}{k_2!}\\
    \lambda^{k_1-k_2}
    &=\frac{k_1!}{k_2!}
      \intertext{this implies that}
      (k_1-k_2)\ln\lambda&=\ln(k_1!/k_2!)\\
    \lambda&=e^{\ln(k_1!/k_2!)/(k_1-k_2)}.
  \end{align*}
  For example, if \(k_1=3\) and \(k_2=5\), we have
  \(\lambda\approx\num{4.472135954999579}\) and
  \[
    p(3)\approx\num{0.17028240507308132}\approx p(5).
  \]
\end{solution}
\newpage

\begin{problem}[Handout 10, \# 6]
  \emph{(Poisson Approximation.)} There are \(20\) couples seated at a
  rectangular table, husbands on one side and the wives on the other, in a
  random order. Using a Poisson approximation, find the probability that
  exactly two husbands are seated directly across from their wives; at
  least three are; at most three are.
\end{problem}
\begin{solution}
  Let \(X\) count the number of husbands which have been seated directly
  across from their wives. First, we find the probability exactly one man
  is seated across from his wife. To this end, we compute the probability
  of the complement, that either no men are seated next to their wives or
  at least two men are seated next to their wives. There are \(20!\) ways
  of pairing men with women. There are \(18!\) ways of seating at least two
  men with their wives.

  Now, assume that \(X\sim\Poisson\)
\end{solution}
\newpage

\begin{problem}[Handout 10, \# 7]
  \emph{(Poisson Approximation.)} There are \(5\) coins on a desk, with
  probabilities \(0.05\), \(0.1\), \(0.05\), \(0.01\), and \(0.04\) for
  heads. By using a Poisson approximation, find the probability of
  obtaining at least one head when the five coins are each tossed once. Is
  the number of heads obtained binomially distributed in this problem?
\end{problem}
\begin{solution}

\end{solution}
\newpage

\begin{problem}[Handout 10, \# 8]
  A book of \(500\) pages contains 500 misprints. Estimate the chances that
  a given page contains at least three misprints.
\end{problem}
\begin{solution}

\end{solution}
\newpage

\begin{problem}[Handout 10, \# 9]
  Estimate the number of raisins which a cookie should contain on the
  average if it is desired that not more than one cookie out of a hundred
  should be without raisin.
\end{problem}
\begin{solution}

\end{solution}
\newpage

\begin{problem}[Handout 10, \# 10]
  The terms \(\Poisson(k;X)\) of the Poisson distribution reach their
  maximum when \(k\) is the largest integer not exceeding \(X\).
\end{problem}
\begin{solution}

\end{solution}
\newpage

\begin{problem}[Handout 10, \# 11]
  Prove
  \[
    \Poisson(0,\lambda)+\dotsb+\Poisson(n,\lambda)
    =\frac{1}{n!}\int_\lambda^\infty e^{-x}x^n\diff x.
  \]
\end{problem}
\begin{solution}

\end{solution}
\newpage

\begin{problem}[Handout 10, \# 12]
  There is a random number \(N\) of coins in your pocket, where \(N\) has a
  Poisson distribution with mean \(\mu\). Each one is tossed once.

  \noindent Let \(X\) be the number of times a head shows.

  \noindent Find the distribution of \(X\).
\end{problem}
\begin{solution}

\end{solution}
\newpage

\begin{problem}[Handout 10, \# 14]
  Find the MGF of a general Poisson distribution, and hence prove that the
  mean and the variance of an arbitrary Poisson distribution are equal.
\end{problem}
\begin{solution}

\end{solution}
\newpage

\begin{problem}[Handout 10, \# 17 (a)]
  \emph{(Poisson approximations.)} \(20\) couples are seated in a
  rectangular table, husbands on one side and the wives on the
  other. First, find the expected number of husbands that sit directly
  across from their wives. Then, using a Poisson approximation, find the
  probability that two do; three do; at most five do.
\end{problem}
\begin{solution}

\end{solution}

%%% Local Variables:
%%% mode: latex
%%% TeX-master: "../MA519-HW-Current"
%%% End:
