\begin{problem}[Handout 3, \# 3]
  \(n\) sticks are broken into one short and one long part. The \(2n\)
  parts are then randomly paired up to form \(n\) new sticks. Find the
  probability that
  \begin{enumerate}[label=(\alph*),noitemsep]
  \item the parts are joined in their original order, i.e., the new sticks
    are the same as the old sticks;
  \item each long part is paired up with a short part.
  \end{enumerate}
\end{problem}
\begin{solution}
  For part (a): let \(A_i\) denote the event that at the \(i\)th we choose
  a pair of sticks we get pair that joins up to one of the original
  sticks. We are after the probability of \(A\) the event that all the
  parts are joined in their original order, i.e.,
  \(A=\bigcap_{i=1}^n A_i\). First, we find \(P(A_i)\). There are \(2n\)
  ways of choosing a part of a stick and, once we have made that choice,
  only one way of choosing the original complement to it. The probability
  of making this choice on our first try is
  \[
    P(A_1)=\frac{2n}{2n(2n-1)}=\frac{1}{2n-1}.
  \]
  Now, assuming that the event of choosing a part of a stick and its
  original complement is independent from our other choices, we have
  \(2n-2\) choices for our next stick and only one way to chose its
  complement. Therefore, the probability of making this choice is
  \[
    P(A_2)=\frac{2n-2}{(2n-2)(2n-3)}=\frac{1}{2n-3}.
  \]
  Proceeding in this way we see that at the \(i\)th step, the probability
  of choosing a two broken sticks that make up an original stick is
  \[
    P(A_i)=\frac{1}{2(n-i+1)-1}.
  \]
  Thus, by the hierarchical multiplicative formula the probability that the
  sticks are paired in their original order is
  \[
    P(A)=\left(\frac{1}{2n-1}\right)\left(\frac{1}{2n-3}\right) \dotsm
    \left(\frac{1}{3}\right)\left(\frac{1}{1}\right).
  \]

  For part (b): let \(A_i\) denote the event that at the \(i\)th step we
  pair a long stick with a short stick. Then, to find the probability of
  \(A=\bigcap_{i=1}^n A_i\), we first find the probabilities of the
  \(A_i\). There are \(2n\) ways to choose the first stick and \(n\) ways
  to chose either a long or a short part. Thus, the probability of choosing
  a long and a short part on the first try is
  \[
    P(A_1)=\frac{2n\cdot n}{2n(2n-1)}=\frac{n}{2n-1}.
  \]
  As in part (a), at the \(i\)th step, the probability of choosing a long
  and a short stick together is
  \[
    P(A_i)=\frac{n-i+1}n{2(n-i+1)-1}.
  \]
  Thus, the probability of event \(A\), that each long and short stick is
  paired together, is
  \[
    P(A)=\left(\frac{n}{2n-1}\right)\left(\frac{n-1}{2n-3}\right)\dotsm
    \left(\frac{2}{3}\right)\left(\frac{1}{1}\right).
  \]
\end{solution}
\newpage

\begin{problem}[Handout 3, \# 5]
  In a town, there are \(3\) plumbers. On a certain day, \(4\) residents
  need a plumber and they each call one plumber at random.
\end{problem}
\begin{solution}

\end{solution}
\newpage

\begin{problem}[Handout 4, \# 7]
  \emph{(Polygraphs).} Polygraphs are routinely administered to job
  applicants for sensitive government positions. Suppose someone actually
  lying fails the polygraph \(90\%\) of the time. But someone telling the
  truth also fails the polygraph \(15\%\) of the time. If a polygraph
  indicates that an applicant is lying, what is the probability that he is
  in fact telling the truth? Assume a general prior probability \(p\) that
  the person is telling the truth.
\end{problem}
\begin{solution}
  By the total probability formula, we have
  \[
    P(C)=P(C|A)P(A)+P(A)
  \]
\end{solution}
\newpage

\begin{problem}[Handout 4, \# 8]
  In a bolt factory machines \(A\), \(B\), \(C\) manufacture, respectively,
  \(25\), \(35\), and \(40\) per cent of the total. Of their output \(5\),
  \(4\), and \(2\) per cent are defective bolts. A bolt is drawn at random
  from the produce and is found defective. What are the probabilities that
  it was manufactured by machines \(A\), \(B\), \(C\)?
\end{problem}
\begin{solution}

\end{solution}
\newpage

\begin{problem}[Handout 4, \# 9]
  Suppose that \(5\) men out of \(100\) and \(25\) women out of
  \(\num{10000}\) are colorblind. A colorblind person is chosen at
  random. What is the probability of his being male? (Assume males and
  females to be in equal numbers.)
\end{problem}
\begin{solution}

\end{solution}
\newpage

\begin{problem}[Handout 4, \# 10]
  \emph{(Bridge).} In a bridge party West has no ace. What probability should
  he attribute to the event of his partner having
  \begin{enumerate}[label=(\alph*),noitemsep]
  \item no ace,
  \item two or more aces?
  \end{enumerate}
  Verify the result by a direct argument.
\end{problem}
\begin{solution}

\end{solution}
\newpage

\begin{problem}[Handout 4, \# 12]
  A true-false question will be posed to a couple on a game show. The
  husband and the wife each has a probability \(p\) of picking the correct
  answer. Should they decide to let one of the answer the question, or
  decide that they will give the common answer if they agree and toss a
  coin to pick the answer if they disagree?
\end{problem}
\begin{solution}

\end{solution}
\newpage

\begin{problem}[Handout 4, \# 13]
  An urn containing \(5\) balls has been filled up by taking \(5\) balls at
  random from a second urn which originally had \(5\) black and \(5\) white
  balls. A ball is chosen at random from the first urn and is found to be
  black. What is the probability of drawing a white ball if a second ball
  is chosen from among the remaining \(4\) balls in the first urn?
\end{problem}
\begin{solution}
  We shall use the total probability formula to figure out the
  probabilities in question. Let \(B\) denote the event that that on our
  second drawing we draw a white ball. First, we must find a suitable
  partition \(A_1,\dotsc,A_n\) of \(\Omega\) such that
  \(A_i\cap A_j=\emptyset\) whenever \(i\neq j\), for \(1\leq i,j\leq n\),
  i.e., the events \(A_i\) are mutually exclusive. Consider the following
  partition of \(\Omega\),
  \[
    A_i=\bigl\{\,\text{exactly \(i\) white balls are put into the second
      urn}\,\bigr\}.
  \]
  The events \(A_i\), \(1\leq i\leq 4\), are clearly mutually exclusive (if
  we have exactly \(i\) white balls in the urn, we cannot simultaneously
  have \(j\) white balls in the urn for \(i\neq j\)). Therefore, to find
  \(P(A)\), we need only find the probabilities \(P(B|A_i)\) and
  \(P(A_i)\).

  The probabilities of \(A_i\) are easy to calculate: there are
  \(\binom{10}{5}=252\) ways to chose \(5\) balls from the urn containing
  the \(5\) white and \(5\) black balls, and the number of ways of choosing
  exactly \(i\) black balls are \(\binom{5}{5-i}\). Thus,
  \[
    P(A_i)=\frac{\binom{5}{i}}{252}
    =\binom{5}{i}^{\mathrlap{2}}\Bigm/252.
  \]

  The probabilities of \(B\) given \(A_i\) are also easy to calculate
  \[
    P(B|A_i)=\frac{i}{4},
  \]
  since we have removed one ball and it was not white and there are \(i\)
  white balls remaining.

  Thus, by the total probability formula,
  \begin{align*}
    P(B)
    &=\left(\frac{1}{4}\right)\left(\frac{25}{252}\right)+
      \left(\frac{2}{4}\right)\left(\frac{100}{252}\right)\\
    &\phantom{{}={}}+\left(\frac{3}{4}\right)\left(\frac{100}{252}\right)+
      \left(\frac{4}{4}\right)\left(\frac{25}{252}\right)\\
    &\approx\num{0.6200396825396824}.
  \end{align*}
\end{solution}
\newpage

\begin{problem}[Handout 4, \# 15]
  Events \(A\), \(B\), \(C\) have probabilities \(p_1\), \(p_2\),
  \(p_3\). Given that exactly two of the three events occured, the
  probability that \(C\) occured is greater than \(1/2\) if and only if
  ...\@ (write down the necessary and sufficient condition).
\end{problem}
\begin{solution}
  For convenience, let \(D\) denote the event that two of \(A,B,C\)
  occured. By Bayes' theorem, we have
  \begin{align*}
    P(C|D)
    &=\frac{P(D|C)P(C)}{P(D)}\\
    &=\frac{P(D|C)P(C)}{P(D|C)P(C)+P(D|\lnot C)P(\lnot C)}\\
    &=\frac{P\bigl((A\cup B)\setminus (A\cap B)\bigr)p_3}
      {P\bigl((A\cup B)\setminus (A\cap B)\bigr)p_3
      +P\bigl((A\cup B)\setminus C\bigr)(1-p_3)},
      \intertext{but by the inclusion-exclusion principle, \(P\bigl((A\cup
      B)\setminus (A\cap B)\bigr)=P(A)+P(B)-2P(A\cap B)\) so, letting
      \(x=P(A\cap B)\), the above becomes}
    &=\frac{(p_1+p_2-2x)p_3}{(p_1+p_2-2x)p_3+x(1-p_3)}\\
    &=\frac{p_1p_3+p_2p_3-2xp_3}{p_1p_3+p_2p_3+x-3xp_3}\\
    &\geq\frac{1}{2}
  \end{align*}
  if and only if
  \begin{align*}
    2(p_1p_3+p_2p_3-2xp_3)&\geq p_1p_3+p_2p_3+x-3xp_3,
    \intertext{that is,}
    p_1p_3+p_2p_3&\geq x+xp_3.
  \end{align*}
\end{solution}
\newpage

\begin{problem}[Handout 5, \# 1]
  There are five coins on a desk: \(2\) are double-headed, \(2\) are
  double-tailed, and \(1\) is a normal coin.

  \noindent
  One of the coins is selected at random and tossed. It shows heads.

  \noindent
  What is the probability that the other side of this coin is a tail?
\end{problem}
\begin{solution}
  Let \(A\) denote the event that the other side of a coin is tail and let
  \(B\) denote the event that after picking a coin at random and tossing
  it, it comes up heads. By Bayes' theorem,
  \[
    P(A|B)=\frac{P(B|A)P(A)}{P(B)};
  \]
  where \(P(B|A)=1/3\) since there are \(3\) coins whose backside is tail,
  but only one of which can come up heads; \(P(A)=3/5\) since \(3\) out of
  the \(5\) coins have a tail; and \(P(B)=1/2\) since \(1+4=5\) of the
  \(10\) faces are heads. Thus,
  \[
    P(A|B)=\frac{(1/3)(3/5)}{1/2}=\frac{2}{5}=0.4.
  \]
\end{solution}
\newpage

\begin{problem}[Handout 5, \# 2]
  \emph{(Genetic testing).} There is a \(50\)-\(50\) chance that the Queen
  carries the gene for hemophilia. If she does, then each Prince has a
  \(50\)-\(50\) chance of carrying it. Three Princesses were recently
  tested and found to be non-carriers. Find the following probabilities:
  \begin{enumerate}[label=(\alph*),noitemsep]
  \item that the Queen is a carrier;
  \item that the fourth Princess is a carrier.
  \end{enumerate}
\end{problem}
\begin{solution}
  For part (a): let \(A\) denote the event that the Queen has hæmophilia
  and let \(B\) be the event that three princesses were tested and found to
  be non-carriers. Then, by Bayes' theorem,
  \[
    P(A|B)=\frac{P(B|A)P(A)}{P(B)}.
  \]
  By assumption, \(P(A)=1/2\). We cannot calculate \(P(B)\) directly
  but, by the total probability formula,
  \begin{align*}
    P(B)&=P(B|A)P(A)+P(B|\lnot A)P(\lnot A)\\
        &=\left(\frac{1}{8}\right)\left(\frac{1}{2}\right)
          +1\left(\frac{1}{2}\right)\\
        &=\frac{9}{16}.
  \end{align*}
  Thus,
  \[
    P(A|B)=\frac{(1/8)(1/2)}{9/16}=\frac{1}{9}\approx\num{0.1111}.
  \]

  For part (b): let \(A\) denote the event that the fourth princess is a
  carrier and \(B\) remain as above, i.e., the event that three princesses
  were found to be non-carriers. By Bayes' theorem,
  \[
    P(A|B)=\frac{P(B|A)P(A)}{P(B)}.
  \]
  If the fourth princess is a carrier then the Queen is a carrier. Thus,
  \(P(B)=9/16\) and \(P(B|A)=1/8\) as above. Now,
  \(P(A)=P(A|C)P(C)+P(A|\lnot C)P(\lnot C)\) where \(C\) denotes the event
  that the Queen is a carrier. Thus,
  \[
    P(A)=\left(\frac{1}{2}\right)\frac{1}{2}=\frac{1}{4}.
  \]
  Thus,
  \[
    P(A|B)=\frac{(1/8)(1/4)}{9/16}=\frac{1}{18}\approx\num{0.0556}.
  \]
\end{solution}
\newpage

\begin{problem}[Handout 5, \# 4]
  \emph{(Is Johnny in Jail).} Johnny and you are roommates. You are a
  terrific student and spend Friday evenings drowned in books. Johnny
  always goes out on Friday evenings. \(40\%\) of the times, he goes out
  with his girlfriend, and \(60\%\) of the times he goes to a bar. If he
  goes out with his girlfriend, \(30\%\) of the times he is just too lazy
  to come back and spends the night at hers. If he goes to a bar, \(40\%\)
  of the times he gets mad at the person sitting on his right, beats him
  up, and goes to jail.

  \noindent
  On one Saturday morning, you wake up to see Johnny is missing. Where is
  Johnny?
\end{problem}
\begin{solution}
  Let \(A\) denote the event that Johnny is in jail and \(B\) denote the
  event that Johnny is missing. Then, by Bayes' theorem,
  \[
    P(A|B)=\frac{P(B|A)P(A)}{P(B)}.
  \]
  The probability \(P(B|A)=1\) clearly. Let \(C\) denote the event that
  Johnny went to the bar. Then, by the total probability formula,
  \[
    P(A)=P(A|C)P(C)+P(A|\lnot C)P(\lnot C)=(0.4)(0.4)+0=0.16.
  \]
\end{solution}

%%% Local Variables:
%%% mode: latex
%%% TeX-master: "../MA519-Current-HW"
%%% End:
