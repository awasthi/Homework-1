\begin{problem}[Handout 3, \# 3]
  \(n\) sticks are broken into one short and one long part. The \(2n\)
  parts are then randomly paired up to form \(n\) new sticks. Find the
  probability that
  \begin{enumerate}[label=(\alph*),noitemsep]
  \item the parts are joined in their original order, i.e., the new sticks
    are the same as the old sticks;
  \item each long part is paired up with a short part.
  \end{enumerate}
\end{problem}
\begin{solution}
  For part (a): let \(A_i\) denote the event that at the \(i\)th we choose
  a pair of sticks we get pair that joins up to one of the original
  sticks. We are after the probability of \(A\) the event that all the
  parts are joined in their original order, i.e.,
  \(A=\bigcap_{i=1}^n A_i\). First, we find \(P(A_i)\). There are \(2n\)
  ways of choosing a part of a stick and, once we have made that choice,
  only one way of choosing the original complement to it. The probability
  of making this choice on our first try is
  \[
    P(A_1)=\frac{2n}{2n(2n-1)}=\frac{1}{2n-1}.
  \]
  Now, assuming that the event of choosing a part of a stick and its
  original complement is independent from our other choices, we have
  \(2n-2\) choices for our next stick and only one way to chose its
  complement. Therefore, the probability of making this choice is
  \[
    P(A_2)=\frac{2n-2}{(2n-2)(2n-3)}=\frac{1}{2n-3}.
  \]
  Proceeding in this way we see that at the \(i\)th step, the probability
  of choosing a two broken sticks that make up an original stick is
  \[
    P(A_i)=\frac{1}{2(n-i+1)-1}.
  \]
  Thus, by the hierarchical multiplicative formula the probability that the
  sticks are paired in their original order is
  \[
    P(A)=\left(\frac{1}{2n-1}\right)\left(\frac{1}{2n-3}\right) \dotsm
    \left(\frac{1}{3}\right)\left(\frac{1}{1}\right).
  \]

  For part (b): let \(A_i\) denote the event that at the \(i\)th step we
  pair a long stick with a short stick. Then, to find the probability of
  \(A=\bigcap_{i=1}^n A_i\), we first find the probabilities of the
  \(A_i\). There are \(2n\) ways to choose the first stick and \(n\) ways
  to chose either a long or a short part. Thus, the probability of choosing
  a long and a short part on the first try is
  \[
    P(A_1)=\frac{2n(n)}{2n(2n-1)}=\frac{n}{2n-1}.
  \]
  As in part (a), at the \(i\)th step, the probability of choosing a long
  and a short stick together is
  \[
    P(A_i)=\frac{n-i+1}n{2(n-i+1)-1}.
  \]
  Thus, the probability of event \(A\), that each long and short stick is
  paired together, is
  \[
    P(A)=\left(\frac{n}{2n-1}\right)\left(\frac{n-1}{2n-3}\right)\dotsm
    \left(\frac{2}{3}\right)\left(\frac{1}{1}\right).
  \]
\end{solution}
\newpage
\begin{problem}[Handout 3, \# 5]
  In a town, there are \(3\) plumbers. On a certain day, \(4\) residents
  need a plumber and they each call one plumber at random.
  \begin{enumerate}[label=(\alph*),noitemsep]
  \item What is the probability that all the calls go to one plumber (not
    necessarily a specific one)?
  \item What is the expected value of the number of plumbers who get a
    call?
  \end{enumerate}
\end{problem}
\begin{solution}
  Label the plumbers as $a$, $b$, and $c$. Let $A$ be the event that
  plumber $a$ gets every call, let $B$ be the event that plumber $b$ gets
  every call, and let $C$ be the event that plumber $c$ gets every call.

  For (a): let $A$ be the event that plumber $a$ gets every call, let $B$
  be the event that plumber $b$ gets every call, and let $C$ be the event
  that plumber $c$ gets every call.  Then $P(A)=P(B)=P(C)=(1/3)^4$. (The
  probability of plumber $a$ getting called by the first resident is
  $1/3$. The residents call independently, the probability that all $4$
  residents call him is $(1/3)^4$). Because all events are disjoint,
  $P(A\cup B\cup C)=1/3^4+1/3^4+1/3^4$. That is, the probability that some
  plumber gets every call is
  $(1/3)^3=1/27\approx \num{0.037037037037037035}$.

  For (b): let $X$ be the number of plumbers who get called.  As above,
  \[
    P(X = 1)=\frac{1}{27}.
  \]
  Similar to the above,
  \[
    P(X = 2) = 3\left(\left(\frac{2}{3}\right)^4 - \frac{2}{3^4}
    \right)
  \]
  (The probability of a resident calling either plumber $a$ or plumber $b$
  is $2/)$. The probability of every resident calling one of plumber $a$ or
  $b$ is $(2/3)^4$. The probability that plumber $a$ was the only plumber
  called is $(1/3)^4$. The probability that plumber $b$ was the only
  plumber called is $(1/3)^4$. Thus, the probability that plumber $a$ and
  plumber $b$ have both been called without calling plumber $c$ is
  $(2/3)^4-2/3^4$. The probability that two plumbers have been called is
  the sum of the probabilities that exactly plumbers $a$ and $b$ have been
  called, exactly plumbers $b$ and $c$ have been called, and exactly
  plumbers $a$ and $c$ have been called; that is, it is as given above.)
  Lastly,
  \[
    P(X = 3) = 1-P(X = 1)-P(X = 2)
  \]
  so that
  \begin{align*}
    E(X) &= P(X=1) + 2P(X=2) + 3P(X = 3) \\
             &= \frac{1}{27} + 6\left(\left(\frac{2}{3}\right)^4 - \frac{2}{3^4} \right) + 3\left(1-\frac{1}{27} -  3\left(\left(\frac{2}{3}\right)^4 - \frac{2}{3^4} \right)\right) \\
             &= \frac{1}{27} + 6\frac{2^4}{3^4} - \frac{12}{3^4} + 3-\frac{1}{9} -  9\left(\left(\frac{2}{3}\right)^4 - \frac{2}{3^4} \right)\\
             &= \frac{1}{27} + 6\frac{16}{81} - \frac{12}{81} + 3-\frac{1}{9} -  \frac{16}{9} - \frac{2}{9}\\
             &= \frac{53}{27}\\
             &\approx\num{1.962962962962963}.
  \end{align*}
  That is, the expected number of plumbers that get called is $53/27$,
  which is slightly less than two.
\end{solution}
\newpage

\begin{problem}[Handout 4, \# 7]
  \emph{(Polygraphs).} Polygraphs are routinely administered to job
  applicants for sensitive government positions. Suppose someone actually
  lying fails the polygraph \(90\%\) of the time. But someone telling the
  truth also fails the polygraph \(15\%\) of the time. If a polygraph
  indicates that an applicant is lying, what is the probability that he is
  in fact telling the truth? Assume a general prior probability \(p\) that
  the person is telling the truth.
\end{problem}
\begin{solution}
  Let $T$ denote the event that the person is telling the truth. Set
  $P(T) = p$. Let $F$ denote the event that the person fails the
  polygraph. Let $L$ denote the event that the person has lied. Then
  $P(F|L) = 0.9$, and $P(F|T) = 0.15$.

  By Bayes' theorem,
  \[
    P(T|F) = \frac{P(F|T)P(T)}{P(F|T)P(T) + P(F|L)P(L)}
  \]
  which reduces to
  \[
    P(T|F) = \frac{0.15p}{0.15p + 0.9 - 0.9p} = \frac{0.15p}{0.9 -
      0.75p}
  \]
\end{solution}
\newpage

\begin{problem}[Handout 4, \# 8]
  In a bolt factory machines \(A\), \(B\), \(C\) manufacture, respectively,
  \(25\), \(35\), and \(40\) per cent of the total. Of their output \(5\),
  \(4\), and \(2\) per cent are defective bolts. A bolt is drawn at random
  from the produce and is found defective. What are the probabilities that
  it was manufactured by machines \(A\), \(B\), \(C\)?
\end{problem}
\begin{solution}
  Let $D$ denote the event in which the random bolt we have drawn was
  defective. Let $A$, $B$, and $C$ denote the events in which we have drawn
  our bolts from machines $A$, $B$, and $C$ (respectively). Then
  \begin{align*}
    P(D|A) &=0.05 \\
    P(D|B) &=0.04 \\
    P(D|C) &=0.02 \\
    P(A)&=0.25 \\
    P(B)&=0.35 \\
    P(C)&=0.4 \\
  \end{align*}
  Which means that, by Bayes' theorem,
  \begin{align*}
    P(A|D) &=\frac{P(D|A)P(A)}{P(D|A)P(A)
             +P(D|B)P(B)+P(D|C)P(C)} \\
           &=\frac{0.05 \cdot 0.25}{0.05 \cdot 0.25 + 0.04 \cdot 0.35 + 0.02 \cdot 0.4 } \\
           &= \frac{0.0125}{0.0125 + 0.014 + 0.008} \\
           &= \frac{0.0125}{0.0345} \\
           &\approx \num{0.36231884057971014}, \\
    P(B|D) &= \frac{0.014}{0.0125 + 0.014 + 0.008} \\
           &= \frac{0.014}{0.0345} \\
           &\approx \num{0.40579710144927533}, \\
    P(C|D) &= \frac{0.008}{0.0125 + 0.014 + 0.008} \\
           &= \frac{0.008}{0.0345} \\
           &\approx \num{0.23188405797101447}.\\
  \end{align*}
  That is, there's about a $36$ percent chance that our defective bolt came
  from $A$, a $41$ percent chance it came from $B$, and a $23$ percent
  chance it came from $C$.
\end{solution}
\newpage

\begin{problem}[Handout 4, \# 9]
  Suppose that \(5\) men out of \(100\) and \(25\) women out of
  \(\num{10000}\) are colorblind. A colorblind person is chosen at
  random. What is the probability of his being male? (Assume males and
  females to be in equal numbers.)
\end{problem}
\begin{solution}
  Let $C$ be the event that a randomly chosen person is
  color blind. Let $M$ be the event that the randomly chosen person was
  male, and let $F$ be the event that the randomly chosen person was
  female. Then
  \begin{align*}
    P(C|M)&=0.05\\
    P(C|F)& 0.0025
  \end{align*}
  So, by Bayes' theorem,
  \begin{align*}
    P(M|C)
    &=\frac{P(C|M)P(M)}{P(C|M)P(M)+P(C|F)P(F)}\\
    &=\frac{0.05}{0.05+0.0025}\\
    &=\frac{0.05}{0.0525}\\
    &\approx\num{0.9523809523809524}.
  \end{align*}
  That is, there is about a \(95\) percent chance that the randomly chosen
  color blind person was male.
\end{solution}
\newpage

\begin{problem}[Handout 4, \# 10]
  \emph{Bridge.} In a Bridge party West has no ace. What probability should
  he attribute to the event of his partner having
  \begin{enumerate}[label=(\alph*),noitemsep]
  \item no ace,
  \item two or more aces?
  \end{enumerate}
  Verify the result by a direct argument.
\end{problem}
\begin{solution}
  For part (a): let \(A\) be the event of West's partner having an ace
\end{solution}
\newpage

\begin{problem}[Handout 4, \# 12]
  A true-false question will be posed to a couple on a game show. The
  husband and the wife each has a probability \(p\) of picking the correct
  answer. Should they decide to let one of the answer the question, or
  decide that they will give the common answer if they agree and toss a
  coin to pick the answer if they disagree?
\end{problem}
\begin{solution}
  The probability that the couple wins if they let one of them answer the
  question is $p$ (that is, it is the probability that that person gets it
  right.)

  The probability that the couple wins if they give a common answer if they
  agree and toss a coin to pick the answer if they disagree is the sum of
  the probabilities that they agree on the correct answer and the
  probability that they guess differently and then win on the coin flip.

  The probability that they agree on the correct answer is $p^2$.

  The probability that they guess differently is $2(1-p)p$; that is, this
  is the probability that the husband is wrong and the wife is right plus
  the probability that the wife is wrong and the husband is right.

  The probability that they win, given that they've guessed differently, is
  $1/2$. So, the probability that they win because of the coin flip after
  they guess differently is $(1-p)p$.

  So, the probability that they win in this fashion is $p^2 + (1-p)p = p$.
  That is, from the perspective of the game, it does not matter which
  course of action they take. They should do whichever thing is most likely
  to make them the least mad at each other if they lose.
\end{solution}
\newpage

\begin{problem}[Handout 4, \# 13]
  An urn containing \(5\) balls has been filled up by taking \(5\) balls at
  random from a second urn which originally had \(5\) black and \(5\) white
  balls. A ball is chosen at random from the first urn and is found to be
  black. What is the probability of drawing a white ball if a second ball
  is chosen from among the remaining \(4\) balls in the first urn?
\end{problem}
\begin{solution}
  We shall use the total probability formula to figure out the
  probabilities in question. Let \(B\) denote the event that that on our
  second drawing we draw a white ball. First, we must find a suitable
  partition \(A_1,\dotsc,A_n\) of \(\Omega\) such that
  \(A_i\cap A_j=\emptyset\) whenever \(i\neq j\), for \(1\leq i,j\leq n\),
  i.e., the events \(A_i\) are mutually exclusive. Consider the following
  partition of \(\Omega\),
  \[
    A_i=\bigl\{\,\text{exactly \(i\) white balls are put into the second
      urn}\,\bigr\}.
  \]
  The events \(A_i\), \(1\leq i\leq 4\), are clearly mutually exclusive (if
  we have exactly \(i\) white balls in the urn, we cannot simultaneously
  have \(j\) white balls in the urn for \(i\neq j\)). Therefore, to find
  \(P(A)\), we need only find the probabilities \(P(B|A_i)\) and
  \(P(A_i)\).

  The probabilities of \(A_i\) are easy to calculate: there are
  \(\binom{10}{5}=252\) ways to chose \(5\) balls from the urn containing
  the \(5\) white and \(5\) black balls, and the number of ways of choosing
  exactly \(i\) black balls are \(\binom{5}{5-i}\). Thus,
  \[
    P(A_i)=\frac{\binom{5}{i}}{252}
    =\binom{5}{i}^{\mathrlap{2}}\Bigm/252.
  \]

  The probabilities of \(B\) given \(A_i\) are also easy to calculate
  \[
    P(B|A_i)=\frac{i}{4},
  \]
  since we have removed one ball and it was not white and there are \(i\)
  white balls remaining.

  Thus, by the total probability formula,
  \begin{align*}
    P(B)
    &=\left(\frac{1}{4}\right)\left(\frac{25}{252}\right)+
      \left(\frac{2}{4}\right)\left(\frac{100}{252}\right)\\
    &\phantom{{}={}}+\left(\frac{3}{4}\right)\left(\frac{100}{252}\right)+
      \left(\frac{4}{4}\right)\left(\frac{25}{252}\right)\\
    &\approx\num{0.6200396825396824}.
  \end{align*}
\end{solution}
\newpage

\begin{problem}[Handout 4, \# 15]
  Events \(A\), \(B\), \(C\) have probabilities \(p_1\), \(p_2\),
  \(p_3\). Given that exactly two of the three events occured, the
  probability that \(C\) occured is greater than \(1/2\) if and only if
  ...\@ (write down the necessary and sufficient condition).
\end{problem}
\begin{solution}
  For convenience, let \(D\) denote the event that two of \(A,B,C\)
  occured. By Bayes' theorem, we have
  \begin{align*}
    P(C|D)
    &=\frac{P(D|C)P(C)}{P(D)}\\
    &=\frac{P(D|C)P(C)}{P(D|C)P(C)+P(D|\lnot C)P(\lnot C)}\\
    &=\frac{P\bigl((A\cup B)\setminus (A\cap B)\bigr)p_3}
      {P\bigl((A\cup B)\setminus (A\cap B)\bigr)p_3
      +P\bigl((A\cup B)\setminus C\bigr)(1-p_3)},
      \intertext{but by the inclusion-exclusion principle, \(P\bigl((A\cup
      B)\setminus (A\cap B)\bigr)=P(A)+P(B)-2P(A\cap B)\) so, letting
      \(x=P(A\cap B)\), the above becomes}
    &=\frac{(p_1+p_2-2x)p_3}{(p_1+p_2-2x)p_3+x(1-p_3)}\\
    &=\frac{p_1p_3+p_2p_3-2xp_3}{p_1p_3+p_2p_3+x-3xp_3}\\
    &\geq\frac{1}{2}
  \end{align*}
  if and only if
  \begin{align*}
    2(p_1p_3+p_2p_3-2xp_3)&\geq p_1p_3+p_2p_3+x-3xp_3,
    \intertext{that is,}
    p_1p_3+p_2p_3&\geq x+xp_3.
  \end{align*}
\end{solution}
\newpage

\begin{problem}[Handout 5, \# 1]
  There are five coins on a desk: \(2\) are double-headed, \(2\) are
  double-tailed, and \(1\) is a normal coin.

  \noindent
  One of the coins is selected at random and tossed. It shows heads.

  \noindent
  What is the probability that the other side of this coin is a tail?
\end{problem}
\begin{solution}
  Let \(A\) denote the event that the other side of a coin is tail and let
  \(B\) denote the event that after picking a coin at random and tossing
  it, it comes up heads. By Bayes' theorem,
  \[
    P(A|B)=\frac{P(B|A)P(A)}{P(B)};
  \]
  where \(P(B|A)=1/3\) since there are \(3\) coins whose backside is tail,
  but only one of which can come up heads; \(P(A)=3/5\) since \(3\) out of
  the \(5\) coins have a tail; and \(P(B)=1/2\) since \(1+4=5\) of the
  \(10\) faces are heads. Thus,
  \[
    P(A|B)=\frac{(1/3)(3/5)}{1/2}=\frac{2}{5}=0.4.
  \]
\end{solution}
\newpage

\begin{problem}[Handout 5, \# 2]
  \emph{(Genetic testing).} There is a \(50\)-\(50\) chance that the Queen
  carries the gene for hemophilia. If she does, then each Prince has a
  \(50\)-\(50\) chance of carrying it. Three Princesses were recently
  tested and found to be non-carriers. Find the following probabilities:
  \begin{enumerate}[label=(\alph*),noitemsep]
  \item that the Queen is a carrier;
  \item that the fourth Princess is a carrier.
  \end{enumerate}
\end{problem}
\begin{solution}
  For part (a): let \(A\) denote the event that the Queen has hæmophilia
  and let \(B\) be the event that three princesses were tested and found to
  be non-carriers. Then, by Bayes' theorem,
  \[
    P(A|B)=\frac{P(B|A)P(A)}{P(B)}.
  \]
  By assumption, \(P(A)=1/2\). We cannot calculate \(P(B)\) directly
  but, by the total probability formula,
  \begin{align*}
    P(B)&=P(B|A)P(A)+P(B|\lnot A)P(\lnot A)\\
        &=\left(\frac{1}{8}\right)\left(\frac{1}{2}\right)
          +1\left(\frac{1}{2}\right)\\
        &=\frac{9}{16}.
  \end{align*}
  Thus,
  \[
    P(A|B)=\frac{(1/8)(1/2)}{9/16}=\frac{1}{9}\approx\num{0.1111111111111111}.
  \]

  For part (b): let \(A\) denote the event that the fourth princess is a
  carrier and \(B\) remain as above, i.e., the event that three princesses
  were found to be non-carriers. By Bayes' theorem,
  \[
    P(A|B)=\frac{P(B|A)P(A)}{P(B)}.
  \]
  If the fourth princess is a carrier then the Queen is a carrier. Thus,
  \(P(B)=9/16\) and \(P(B|A)=1/8\) as above. Now,
  \(P(A)=P(A|C)P(C)+P(A|\lnot C)P(\lnot C)\) where \(C\) denotes the event
  that the Queen is a carrier. Thus,
  \[
    P(A)=\left(\frac{1}{2}\right)\frac{1}{2}=\frac{1}{4}.
  \]
  Thus,
  \[
    P(A|B)=\frac{(1/8)(1/4)}{9/16}=\frac{1}{18}\approx\num{0.05555555555555555}.
  \]
\end{solution}
\newpage

\begin{problem}[Handout 5, \# 4]
  \emph{(Is Johnny in Jail).} Johnny and you are roommates. You are a
  terrific student and spend Friday evenings drowned in books. Johnny
  always goes out on Friday evenings. \(40\%\) of the times, he goes out
  with his girlfriend, and \(60\%\) of the times he goes to a bar. If he
  goes out with his girlfriend, \(30\%\) of the times he is just too lazy
  to come back and spends the night at hers. If he goes to a bar, \(40\%\)
  of the times he gets mad at the person sitting on his right, beats him
  up, and goes to jail.

  \noindent
  On one Saturday morning, you wake up to see Johnny is missing. Where is
  Johnny?
\end{problem}
\begin{solution}
  Let \(A\) denote the event that Johnny is in jail and \(B\) denote the
  event that Johnny is missing. Then, by Bayes' theorem,
  \[
    P(A|B)=\frac{P(B|A)P(A)}{P(B)}.
  \]
  The probability \(P(B|A)=1\) clearly. Let \(C\) denote the event that
  Johnny went to the bar. Then, by the total probability formula,
  \begin{align*}
    P(A)&=P(A|C)P(C)+P(A|\lnot C)P(\lnot C)\\
        &=(0.4)(0.6)+0\\
        &=0.24.
  \end{align*}
  Also by the total probability formula,
  \begin{align*}
    P(B)&=P(B|C)P(C)+P(B|\lnot C)P(\lnot C)\\
        &=(0.4)(0.6)+(0.3)(0.4)\\
        &=0.36.
  \end{align*}
  Thus,
  \[
    P(A|B)=\frac{0.24}{0.36}=\frac{2}{3}\approx\num{0.6666666666666666}.
  \]
\end{solution}

%%% Local Variables:
%%% mode: latex
%%% TeX-master: "../MA519-Current-HW"
%%% End:
