\subsection{Homework 2}
\begin{problem}[Handout 2, \# 5]
  Four men throw their watches into the sea, and the sea brings each man
  one watch back at random. What is the probability that at least one man
  gets his own watch back?
\end{problem}
\begin{solution*}
  Suppose \(4\) men throw their watches into the sea. Label these men ``the
  \(i\)\textsup{th} man'' for \(1\leq i\leq 4\). Let \(A_i\) denote the
  probability that the \(i\)\textsup{th} man gets his own watch back. Then
  the probability that at least one man gets his own watch back is the
  event \(A=A_1\cup A_2\cup A_3\cup A_4\). By the inclusion-exclusion
  principle, we have
  \begin{equation}
    \label{eq:inclusion-exclusion-4}
    \begin{aligned}
      P(
      A)&=P(A_1)+P(A_2)+P(A_3)+P(A_4)\\
      &\phantom{{}={}}-P(A_1\cap A_2)-P(A_1\cap A_3)-P(A_1\cap A_4)\\
      &\phantom{{}={}}-P(A_2\cap A_3)-P(A_2\cap A_4)-P(A_3\cap A_4)\\
      &\phantom{{}={}}+P(A_1\cap A_2\cap A_3)+P(A_1\cap A_2\cap A_4)\\
      &\phantom{{}={}}+P(A_1\cap A_3\cap A_4)+P(A_2\cap A_3\cap A_4)\\
      &\phantom{{}={}}-P(A_1\cap A_2\cap A_3\cap A_4).
    \end{aligned}
  \end{equation}
  Since the probabilities \(P(A_i)=P(A_j)\), \(P(A_i\cap A_j)=P(A_k\cap
  A_\ell)\), etc., the equation above reduces to
  \[
    P(A)=4P(A_1)-6P(A_1\cap A_2)+4P(A_1\cap A_2\cap A_3)-P(A_1\cap A_2\cap
    A_3\cap A_4).
  \]
  All we need to do now is fill in the blanks. Intuitively, the probability
  that the \(1\)\textsup{st} man gets back his own wallet is
  \(\frac{1}{4}\) since only one wallet is his own out of the \(4\). Now,
  the probability that the \(1\)\textsup{st} and the \(2\)\textsup{nd} man
  get their own wallet back is
  \(\frac{1}{4}\cdot\frac{1}{3}=\frac{1}{2}\). Proceeding in this fashion,
  we have
  \[
    P(A)=4\cdot\frac{1}{4}-6\cdot\frac{1}{12}+4\cdot\frac{1}{24}-\frac{1}{24}
    =0.625.
  \]
\end{solution*}

\begin{problem}[Handout 2, \# 7]
  Calculate the probability that in Bridge, the hand of at least one player
  is void in a particular suit.
\end{problem}
\begin{solution*}
  Label the players ``player \(i\)'' for \(1\leq i\leq 4\). Let \(A_i\)
  denote the event that player \(i\) is void in a particular suit. Then
  the probability of the event \(A\) that at least one player is void in a
  particular suit is \(P(A)=P\bigl(\bigcup_{i=1}^4 A_i\bigr)\). Again, we
  can decompose this into
  \[
    P(A)=4P(A_1)-6P(A_1\cap A_2)+4P(A_1\cap A_2\cap A_3)-P(A_1\cap A_2\cap
    A_3\cap A_4).
  \]
  Thus, it suffices to fill in the blanks in the equation above, i.e., find
  the probabilities \(P(A_1)\), \(P(A_1\cap A_2)\), etc. To this end, the
  probability that player \(1\) is void in a particular suit, say red
  hearts, is
  \[
    P(A_1)=\frac{\binom{52-13}{13}}{\binom{52}{13}}.
  \]
  Similarly, the probability that player \(1\) and player \(2\) is void in
  red hearts is
  \[
    P(A_1)=\frac{\binom{52-13}{13}}{\binom{52}{13}}\cdot\frac{\binom{52-13-13}{13}}{\binom{52-13}{13}}=\frac{\binom{52-13-13}{13}}{\binom{52}{13}};
  \]
  and so on.

  Thus,
  \[
    P(A)=4\cdot\frac{\binom{52-13}{13}}{\binom{52}{13}}
    -6\cdot\frac{\binom{52-13-13}{13}}{\binom{52}{13}}
    +4\cdot\frac{\binom{52-13-13-13}{13}}{\binom{52}{13}}
    -1\cdot0\approx \num{0.051065521}.
  \]
\end{solution*}

\begin{problem}[Handout 2, \# 12]
  If \(n\) balls are placed at random into \(n\) cells, find the
  probability that exactly one cell remains empty.
\end{problem}
\begin{solution*}
  There are \(n\) ways to choose a cell to be left empty. Of the remaining
  \(n-1\) cells, one must contain \(2\) balls. That cell can be chosen in
  \(n-1\) ways and the two balls to be placed in it can be chosen in
  \(\binom{2}{n}\) ways. Of the remaining cells, each must contain \(1\)
  ball and this pairing can be done in \((n-2)!\) ways. Thus, the
  probability of the event \(A\) that exactly one cell remains empty is
  \[
    P(A)=\frac{n(n-1)\binom{2}{n}(n-2)!}{n^n}=\frac{\binom{2}{n}n!}{n^n}.
  \]
\end{solution*}

\begin{problem}[Handout 2, \# 13]
  \emph{Spread of rumors.} In a town of \(n+1\) inhabitants, a person tells
  a rumor to a second person, who in turn repeats it to a third person,
  etc. At each step the recipient of the rumor is chosen at random from the
  \(n\) people available. Find the probability that the rumor told \(r\)
  times without:
  \begin{enumerate}[label=(\alph*),noitemsep]
  \item returning to the originator,
  \item being repeated to any person.
  \end{enumerate}
  Do the same problem when at each step the rumor told by one person to a
  gathering of \(N\) randomly chosen people. (The first question is the
  special case \(N=1\)).
\end{problem}
\begin{solution*}
  For part (a), the originator can tell any one of the other \(n\)
  inhabitants the rumor. In turn, the non-originator can tell the other
  \(n-1\) inhabitants (not including the originator). Thus, the probability
  of the event \(A\) that the rumor told \(r\) does not return to the
  originator is
  \[
    P(A)=\frac{n(n-1)^{r-1}}{n^r}=\left(\frac{n-1}{n}\right)^{r-1}.
  \]
  \\\\
  For part (b),
\end{solution*}

\begin{problem}[Handout 2, \# 14]
  What is the probability that
  \begin{enumerate}[label=(\alph*),noitemsep]
  \item the birthdays of twelve people will fall in twelve different
    calendar months (assume equal probabilities for the twelve months),
  \item the birthdays of six people will fall in exactly two calendar
    months?
  \end{enumerate}
\end{problem}
\begin{solution*}
\end{solution*}

\begin{problem}[Handout 2, \# 15]
  A car is parked among \(N\) cars in a row, not at either end. On his
  return the owner finds exactly \(r\) of the \(N\) places still
  occupied. What is the probability that both neighboring places are empty?
\end{problem}
\begin{solution*}
\end{solution*}

\begin{problem}[Handout 2, \# 16]
  Find the probability that in a random arrangement of \(52\) bridge card
  no two aces are adjacent.
\end{problem}
\begin{solution*}
\end{solution*}

\begin{problem}[Handout 2, \# 17]
  Suppose \(P(A)=\frac{3}{4}\), and \(P(B)=\frac{1}{3}\).

  \noindent Prove that \(P(A\cap B)\geq \frac{1}{12}\). Can it be equal to
  \(\frac{1}{12}\)?
\end{problem}
\begin{solution*}
\end{solution*}

\begin{problem}[Handout 2, \# 18]
  Suppose you have infinitely many events \(A_1,A_2,\dotsc\), and each one
  is sure to occur, i.e., \(P(A_i)=1\) for any \(i\).

  \noindent Prove that \(P\bigl(\bigcap_{i=1}^n A_i\bigr)=1\).
\end{problem}
\begin{solution*}
\end{solution*}

\begin{problem}[Handout 2, \# 19]
  There are \(n\) blue, \(n\) green, \(n\) red, and \(n\) white balls in an
  urn. Four balls are drawn from the urn with replacement. Find the
  probability that there are balls of at least three different colors among
  the four drawn.
\end{problem}
\begin{solution*}
\end{solution*}

%%% Local Variables:
%%% mode: latex
%%% TeX-master: "../MA519-HW-ALL"
%%% End:
