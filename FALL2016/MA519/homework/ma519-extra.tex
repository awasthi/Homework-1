\subsection{Extra Problems}
These are problems (two to three) randomly chosen from among the 19
handouts Prof.\@ DasGupta gave us during the fall semester of `16.
\subsubsection{Handout 1}
This handout was skipped as its contents were too basic to bother
reviewing.
\subsubsection{Handout 2}
\begin{problem}[Handout 2, \# 3]
If State's football team has a \(10\%\) chance of winning Saturday's game,
a \(30\%\) chance of winning two weeks from now, and a \(65\%\) chance of
losing both games, what are their chances of winning exactly once?
\end{problem}
\begin{solution*}
\end{solution*}

\subsubsection{Handout 3}
\begin{problem}[Handout 3, \# 4]
  A die is rolled \(12\) times. What is the probability that each face
  occurs twice?
\end{problem}
\begin{solution*}
\end{solution*}

\begin{problem}[Handout 3, \# 6]
  In a state lotto, three numbers are picked at random from
  \(00,01,\dotsc,99\). If you pick three numbers, what is the probability
  that none of your numbers will match any of the winning numbers?
\end{problem}
\begin{solution*}
\end{solution*}

\subsubsection{Handout 4}
\begin{problem}[Handout 4, \# 4]
  At a parking lot, there are \(12\) parking places in a row. A man
  observed that \(4\) of them were empty, and that these \(4\) were
  adjacent to each other. Given that there were \(4\) empty spaces, is this
  adjacency surprising?
\end{problem}
\begin{solution*}
\end{solution*}

\begin{problem}[Handout 4, \# 16]
  Suppose for \(n>1\), a family has \(n\) children with probability
  \(\alpha p^n\); here \(\alpha\), \(p\) are fixed numbers between \(0\)
  and \(1\). The probability that the family has no children is \(1-\alpha
  p-\alpha p^-\dotsb\).

  \noindent Suppose a randomly selected family is known to have at least
  one boy. What is the probability that the family has more than one boy?
\end{problem}
\begin{solution*}
\end{solution*}

\subsubsection{Handout 5}
\begin{problem}[Handout 5, \# 6]
  In the World Series, two teams play until one team wins four
  games. Suppose an eccentric referee dictates that the outcome of each
  game will be determined by the toss of a coin.

  \noindent If the coin is truly fair, what is the probability that the
  Series will be over in four games?
\end{problem}
\begin{solution*}
\end{solution*}

\begin{problem}[Handout 5, \# 14]
  North and South had neither aces nor kings in three consecutive bridge
  plays. Do they have reasons to complain?
\end{problem}
\begin{solution*}
\end{solution*}

\subsubsection{Handout 6}
No problems from this handout.

\subsubsection{Handout 7}
\begin{problem}[Handout 7, \# 5]
  A couple decided to have children until they have a child of each
  sex. Let \(X\) be the number of children they will have. Find \(X\).
\end{problem}
\begin{solution*}
\end{solution*}

\begin{problem}[Handout 7, \# 8]
  \begin{enumerate}[label=(\alph*),noitemsep]
  \item Show that for a suitable positive constant \(c\), the function
    \(p(x)=\frac{c}{x^3}\), \(x=1,2,3,\dotsc,\) is a valid PMF.
  \item Show that in this case, the expectation of the underlying random
    variable exists, but the variance does not.
  \end{enumerate}
\end{problem}
\begin{solution*}
\end{solution*}

\subsubsection{Handout 8}
\begin{problem}[Handout 8, \# 4]
  Derive closed form formulas for the mean, variance, and the mean absolute
  deviation of the \(\Bin(n,p)\) distribution.
\end{problem}
\begin{solution*}
\end{solution*}

\begin{problem}[Handout 8, \# 6]
  In a jury trial with twelve jurors, suppose at least eight jurors have to
  vote `guilty' for conviction. Suppose the jurors cast their first vote
  independently, each makes the correct judgment with probability \(p\),
  and your personal probability that the defendant is guilty is
  \(\frac{1}{2}\).

  \noindent What is the probability of conviction on the first vote?
\end{problem}
\begin{solution*}
\end{solution*}

\begin{problem}[Handout 8, \# 11]
  Suppose \(X\) is \(\Bin(n,p)\) distributed.

  \noindent Find a closed form formula for \(E(\rme^{tX})\). Here \(t\) is
  a fixed real number. So the formula is a function of this number.
\end{problem}
\begin{solution*}
\end{solution*}

\subsubsection{Handout 9}
\begin{problem}[Handout 9, \# 6]
  Suppose a general coin is tossed until the first head is obtained. Find
  the first mean, variance, and third moment of \(X\), where \(X\) is the
  toss at which the first head was obtained
\end{problem}
\begin{solution*}
\end{solution*}

\begin{problem}
\end{problem}
\begin{solution*}
\end{solution*}

\begin{problem}
\end{problem}
\begin{solution*}
\end{solution*}

\subsubsection{Handout 10}
\begin{problem}
\end{problem}
\begin{solution*}
\end{solution*}

\begin{problem}
\end{problem}
\begin{solution*}
\end{solution*}

\begin{problem}
\end{problem}
\begin{solution*}
\end{solution*}

\subsubsection{Handout 11}
\begin{problem}
\end{problem}
\begin{solution*}
\end{solution*}

\begin{problem}
\end{problem}
\begin{solution*}
\end{solution*}

\begin{problem}
\end{problem}
\begin{solution*}
\end{solution*}

\subsubsection{Handout 12}
\begin{problem}
\end{problem}
\begin{solution*}
\end{solution*}

\begin{problem}
\end{problem}
\begin{solution*}
\end{solution*}

\begin{problem}
\end{problem}
\begin{solution*}
\end{solution*}

\subsubsection{Handout 13}
\begin{problem}
\end{problem}
\begin{solution*}
\end{solution*}

\begin{problem}
\end{problem}
\begin{solution*}
\end{solution*}

\begin{problem}
\end{problem}
\begin{solution*}
\end{solution*}

\subsubsection{Handout 14}
\begin{problem}
\end{problem}
\begin{solution*}
\end{solution*}

\begin{problem}
\end{problem}
\begin{solution*}
\end{solution*}

\begin{problem}
\end{problem}
\begin{solution*}
\end{solution*}

\subsubsection{Handout 15}
\begin{problem}
\end{problem}
\begin{solution*}
\end{solution*}

\begin{problem}
\end{problem}
\begin{solution*}
\end{solution*}

\begin{problem}
\end{problem}
\begin{solution*}
\end{solution*}

\subsubsection{Handout 16}
\begin{problem}
\end{problem}
\begin{solution*}
\end{solution*}

\begin{problem}
\end{problem}
\begin{solution*}
\end{solution*}

\begin{problem}
\end{problem}
\begin{solution*}
\end{solution*}

\subsubsection{Handout 17}
\begin{problem}
\end{problem}
\begin{solution*}
\end{solution*}

\begin{problem}
\end{problem}
\begin{solution*}
\end{solution*}

\begin{problem}
\end{problem}
\begin{solution*}
\end{solution*}

\subsubsection{Handout 18}
\begin{problem}
\end{problem}
\begin{solution*}
\end{solution*}

\begin{problem}
\end{problem}
\begin{solution*}
\end{solution*}

\begin{problem}
\end{problem}
\begin{solution*}
\end{solution*}

\subsubsection{Handout 19}
\begin{problem}
\end{problem}
\begin{solution*}
\end{solution*}

\begin{problem}
\end{problem}
\begin{solution*}
\end{solution*}

\begin{problem}
\end{problem}
\begin{solution*}
\end{solution*}


%%% Local Variables:
%%% mode: latex
%%% TeX-master: "../MA519-HW-ALL"
%%% End:
