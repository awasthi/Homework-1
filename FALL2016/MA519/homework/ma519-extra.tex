\subsection{Extra Problems}
These are problems (two to three) randomly chosen from among the 19
handouts Prof.\@ DasGupta gave us during the fall semester of `16.
\subsubsection{Handout 1}
This handout was skipped as its contents were too basic to bother
reviewing.
\subsubsection{Handout 2}
\begin{problem}[Handout 2, \# 3]
  If State's football team has a \(10\%\) chance of winning Saturday's
  game, a \(30\%\) chance of winning two weeks from now, and a \(65\%\)
  chance of losing both games, what are their chances of winning exactly
  once?
\end{problem}
\begin{solution*}
  First let us establish some notation. Let \(A\) denote the event that
  State's football team wins Saturday's game, \(B\) denote the event that
  they win two weeks from now, and \(C\) denote the event that they lose
  both games. Then the probability of the event \(D\) that State's football
  team wins exactly once is
  \begin{align*}
    P(D)
    &=P(A\cup B\setminus A\cap B)\\
    &=P(A)+P(B)-P(A\cap B)\\
    &=P(A)+P(B)-P(\Omega\setminus C)\\
    &=0.1+0.3-(1-0.65)\\
    &=0.05
  \end{align*}
  or \(5\%\). This analysis follows from the law of total probability.
\end{solution*}

\subsubsection{Handout 3}
\begin{problem}[Handout 3, \# 4]
  A die is rolled \(12\) times. What is the probability that each face
  occurs twice?
\end{problem}
\begin{solution*}
  For this problem a simple combinatorial argument suffices. Let us label
  the faces of the die with \(F_k\), \(1\leq k\leq 6\), where \(k\)
  represents the value (number of dots) on that face of the die. Under the
  equal-likelihood hypothesis, there are \(6^{12}\) potential outcomes so
  each sample point in our probability space has an associated probability
  of \(\frac{1}{6^{12}}\). Now let us count the number of ways in which we
  can achieve \(12\) rolls of a die with each face appearing twice.

  Let us first see how we can arrange \(F_1\) so that it appears exactly
  twice in \(12\) rolls of a die. The first occurrence of \(F_1\) can
  happen on any of the first \(11\) rolls. Once we have made that choice,
  there are \(11\) ways we can choose the last occurrence of \(F_1\).
\end{solution*}

\begin{problem}[Handout 3, \# 6]
  In a state lotto, three numbers are picked at random from
  \(00,01,\dotsc,99\). If you pick three numbers, what is the probability
  that none of your numbers will match any of the winning numbers?
\end{problem}
\begin{solution*}
\end{solution*}

\subsubsection{Handout 4}
\begin{problem}[Handout 4, \# 4]
  At a parking lot, there are \(12\) parking places in a row. A man
  observed that \(4\) of them were empty, and that these \(4\) were
  adjacent to each other. Given that there were \(4\) empty spaces, is this
  adjacency surprising?
\end{problem}
\begin{solution*}
\end{solution*}

\begin{problem}[Handout 4, \# 16]
  Suppose for \(n>1\), a family has \(n\) children with probability
  \(\alpha p^n\); here \(\alpha\), \(p\) are fixed numbers between \(0\)
  and \(1\). The probability that the family has no children is \(1-\alpha
  p-\alpha p^2-\dotsb\).

  \noindent Suppose a randomly selected family is known to have at least
  one boy. What is the probability that the family has more than one boy?
\end{problem}
\begin{solution*}
\end{solution*}

\subsubsection{Handout 5}
\begin{problem}[Handout 5, \# 6]
  In the World Series, two teams play until one team wins four
  games. Suppose an eccentric referee dictates that the outcome of each
  game will be determined by the toss of a coin.

  \noindent If the coin is truly fair, what is the probability that the
  Series will be over in four games?
\end{problem}
\begin{solution*}
\end{solution*}

\begin{problem}[Handout 5, \# 14]
  North and South had neither aces nor kings in three consecutive bridge
  plays. Do they have reasons to complain?
\end{problem}
\begin{solution*}
\end{solution*}

\subsubsection{Handout 6}
No problems from this handout.

\subsubsection{Handout 7}
\begin{problem}[Handout 7, \# 5]
  A couple decided to have children until they have a child of each
  sex. Let \(X\) be the number of children they will have. Find \(X\).
\end{problem}
\begin{solution*}
\end{solution*}

\begin{problem}[Handout 7, \# 8]
  \begin{enumerate}[label=(\alph*),noitemsep]
  \item Show that for a suitable positive constant \(c\), the function
    \(p(x)=\frac{c}{x^3}\), \(x=1,2,3,\dotsc,\) is a valid PMF.
  \item Show that in this case, the expectation of the underlying random
    variable exists, but the variance does not.
  \end{enumerate}
\end{problem}
\begin{solution*}
\end{solution*}

\subsubsection{Handout 8}
\begin{problem}[Handout 8, \# 4]
  Derive closed form formulas for the mean, variance, and the mean absolute
  deviation of the \(\Bin(n,p)\) distribution.
\end{problem}
\begin{solution*}
\end{solution*}

\begin{problem}[Handout 8, \# 6]
  In a jury trial with twelve jurors, suppose at least eight jurors have to
  vote `guilty' for conviction. Suppose the jurors cast their first vote
  independently, each makes the correct judgment with probability \(p\),
  and your personal probability that the defendant is guilty is
  \(\frac{1}{2}\).

  \noindent What is the probability of conviction on the first vote?
\end{problem}
\begin{solution*}
\end{solution*}

\begin{problem}[Handout 8, \# 11]
  Suppose \(X\) is \(\Bin(n,p)\) distributed.

  \noindent Find a closed form formula for \(E(\rme^{tX})\). Here \(t\) is
  a fixed real number. So the formula is a function of this number.
\end{problem}
\begin{solution*}
\end{solution*}

\subsubsection{Handout 9}
\begin{problem}[Handout 9, \# 6]
  Suppose a general coin is tossed until the first head is obtained. Find
  the first mean, variance, and third moment of \(X\), where \(X\) is the
  toss at which the first head was obtained.
\end{problem}
\begin{solution*}
\end{solution*}

\begin{problem}[Handout 9, \# 8]
  Calculate in closed form the MGF of \(X\), the toss at which the first
  head is obtained when the coin being tossed is a general coin.
\end{problem}
\begin{solution*}
\end{solution*}

\begin{problem}[Handout 9, \# 15]
  \hfill
  \begin{enumerate}[label=(\alph*),noitemsep]
  \item Always, \(E(X^2)\geq [E(X)]^2\); also \(E(X^2)\geq [E(|X|)]^2\).
  \item Always, \(E(|X|)\geq |E(X)|\).
  \item Jensen's inequality: For any convex function \(g\), and any
    arbitrary random variable \(X\) such that \(E[g(X)]\) exists,
    \(E[g(X)]\geq g[E(X)]\).
  \item The triangle inequality: Always, \(E(|X|)+E(|Y|)\geq E(|X+Y|)\).
  \item the Cauchy--Schwartz inequality: For \(s>r\geq 1\), and for any
    arbitrary nonnegative random variable \(X\) such that \(E(X^s)\)
    exists,
    \[
      [E(X^r)]^{1/r}\leq [E(X^s)]^{1/s}.
    \]
  \end{enumerate}
\end{problem}
\begin{solution*}
\end{solution*}

\subsubsection{Handout 10}
\begin{problem}[Handout 10, \# 3]
  A \(14\) inch circular pizza has been baked with \(20\) pieces of
  barbecued chicken. A triangular slice of dimension \(4\times 4\times 2\)
  was served to you. What is the probability that you got at least one
  piece of chicken.
\end{problem}
\begin{solution*}
\end{solution*}

\begin{problem}[Handout 10, \# 15]
  Prove that the sum of any finite number of independent Poissons is
  another Poisson.
\end{problem}
\begin{solution*}
\end{solution*}

\subsubsection{Handout 11}
\begin{problem}[Handout 11, 4]
  \(n\) numbers are selected with replacement at random from the first
  \(N\) positive integers.
  \begin{enumerate}[label=(\alph*),noitemsep]
  \item Find the expected value of the sum of \(n\) drawn numbers.
  \item What would be the story if sampling was without replacement.
  \end{enumerate}
\end{problem}
\begin{solution*}
\end{solution*}

\begin{problem}[Handout 11, \# 5]
  I offer you to play the following game of chance with me. You will
  continually toss a fair coin until you obtain a head for the first
  time. If you obtain your first head at the \(n\)\textsup{th} toss, you
  will receive \(2^n\) dollars. However, you have to pay \(30\) dollars up
  front to play.
  \begin{enumerate}[label=(\alph*),noitemsep]
  \item Is this a good game of chance for you to play?
  \item Suppose you can win at most \(100\) dollars regardless of how many
    tosses it takes to obtain the first head.

    \noindent what is a fair fee to pay up front under these circumstances?
  \item Suppose there is no bound placed on the maximum you can win, but
    you will win \(n\) dollars if you first head shows up at the
    \(n\)\textsup{th} toss. What is now a fair fee to pay up front?
  \end{enumerate}
\end{problem}
\begin{solution*}
\end{solution*}

\subsubsection{Handout 12}
\begin{problem}[Handout 12, \# 6]
  Let \(X\) be \(U[0,1]\) distributed. Evaluate the expected values of
  \begin{enumerate}[label=(\alph*),noitemsep]
  \item  \(X\);
  \item \(X^k\);
  \item \(\frac{1}{1+X}\);
  \item \(\rme^{tX}\);
  \item \(\ln X\).
  \end{enumerate}
\end{problem}
\begin{solution*}
\end{solution*}

\begin{problem}[Handout 12, \# 12]
  Anirban's dog got mad at him and broke his walking cane off at a random
  point. To hid his devious act, he then chewed off one of the two pieces
  chosen at random. What is the expected length of the piece that remains?
\end{problem}
\begin{solution*}
\end{solution*}

\begin{problem}[Handout 12, \# 14]
  Give an elementary reason for
  \begin{enumerate}[label=(\alph*),noitemsep]
  \item if \(X\) is \(U[0,1]\), \(X^2\) cannot also be \(U[0,1]\);
  \item if \(X\) is \(U[0,1]\), any continuous function \(g(X)\) has a
    finite mean;
  \item if \(X\) is \(U[0,1]\), no strictly monotone function of \(X\),
    other than \(X\), can also be \(U[0,1]\).
  \end{enumerate}
\end{problem}
\begin{solution*}
\end{solution*}

\subsubsection{Handout 13}
\begin{problem}
\end{problem}
\begin{solution*}
\end{solution*}

\begin{problem}
\end{problem}
\begin{solution*}
\end{solution*}

\begin{problem}
\end{problem}
\begin{solution*}
\end{solution*}

\subsubsection{Handout 14}
\begin{problem}[Handout 14, \# 2]
  Approximately find the probability of getting between \(45\) and \(55\)
  heads in \(100\) tosses of a fair coin.
\end{problem}
\begin{solution*}
\end{solution*}

\begin{problem}[Handout 14-II, \# 3]
  In which of the following cases is the sequence \(X=o_p(1)\), in which
  cases is it not \(o_p(1)\) but \(O_p(1)\), and in which cases is it
  neither?
  \begin{enumerate}[label=(\alph*),noitemsep]
  \item \(Y_k\) are IID \(N(0,1)\) and \(X_n=\bar Y_n\);
  \item \(Y_k\) are IID \(N(10,1)\) and \(X_n=Y_n\);
  \item \(X_n\) is \(N(\frac{1}{n},1)\) (the \(X_n\) need not be
    independent!);
  \item \(X_n\) is \(\Bin(n,\frac{1}{2})\);
  \item \(X_n\) is \(\Bin(n,\frac{1}{n})\);
  \item \(X_n\) is \(\Bin(n,\frac{1}{n^2})\).
  \end{enumerate}
\end{problem}
\begin{solution*}
\end{solution*}

\begin{problem}[Handout 14-II, \# 7]
  In which of the following cases does the sequence \(X_n\) converge in
  distribution and what is the limiting distribution?
  \begin{enumerate}[label=(\alph*),noitemsep]
  \item \(Y_k\) are IID \(\Exp(1)\) and \(X_n=\sqrt{n}(Y_n-1)\);
  \item \(X_n\) are \(N(0,1)\) (the \(X_n\) need not be independent!);
  \item \(Y_k\) are IID \(N(\mu,\sigma^2)\), and \(X_n=\sqrt{n}(\bar
    Y_n-\mu)/s\) where \(s^2\defeq\frac{1}{n-1}\sum_{k=1}^n(Y_k-\bar
    Y_n)^2\).
  \item \(X_n\) is \(\Bin(n,\frac{10}{n})\);
    \item \(Y_n\) is \(\Bin(n,p)\), \(0<p<1\), and
      \(X_n=\frac{Y_n-np}{\sqrt{np(1-p)}}\).
    \item \(Y_k\) are IID \(U[0,1]\) andh \(X=n[1-\bar Y_n]\).
  \end{enumerate}
\end{problem}
\begin{solution*}
\end{solution*}

\subsubsection{Handout 15}
\begin{problem}[Handout 15, \# 2]
  For two rolls of a fair die, find the joint distribution of \(X\) the
  larger of the two rolls, and \(Y\) the smaller of the two.

  \noindent Hence, find the marginal distributions of \(X\), \(Y\), and the
  correlation between \(X\) and \(Y\).
\end{problem}
\begin{solution*}
\end{solution*}

\begin{problem}[Handout 15, \# 6]
  Sisters Sara and Wendy have the following plans: Sara will have children
  until she has two boys and Wendy will have children until she has two
  girls.
  \begin{enumerate}[label=(\alph*),noitemsep]
  \item Find the conditional expectation of the size of Sara's family if
    the size of Wendy's family is \(5\) (\(3\) children plus parents).
  \item Find the conditiona lexpectation of the size of the larger family
    if the size of the smaller family is \(4\).
  \end{enumerate}
  Comment on how parts (a) and (b) differ conceptually.
\end{problem}
\begin{solution*}
\end{solution*}

\begin{problem}[Handout 15, \# 7]
  \(X\), \(Y\) are two independent Poisson variables, each with mean \(1\).
\end{problem}
\begin{solution*}
\end{solution*}

\subsubsection{Handout 16}
\begin{problem}[Handout 16, \# ]
\end{problem}
\begin{solution*}
\end{solution*}

\begin{problem}
\end{problem}
\begin{solution*}
\end{solution*}

\begin{problem}
\end{problem}
\begin{solution*}
\end{solution*}

\subsubsection{Handout 17}
\begin{problem}
\end{problem}
\begin{solution*}
\end{solution*}

\begin{problem}
\end{problem}
\begin{solution*}
\end{solution*}

\begin{problem}
\end{problem}
\begin{solution*}
\end{solution*}

\subsubsection{Handout 18}
\begin{problem}
\end{problem}
\begin{solution*}
\end{solution*}

\begin{problem}
\end{problem}
\begin{solution*}
\end{solution*}

\begin{problem}
\end{problem}
\begin{solution*}
\end{solution*}

\subsubsection{Handout 19}
\begin{problem}
\end{problem}
\begin{solution*}
\end{solution*}

\begin{problem}
\end{problem}
\begin{solution*}
\end{solution*}

\begin{problem}
\end{problem}
\begin{solution*}
\end{solution*}


%%% Local Variables:
%%% mode: latex
%%% TeX-master: "../MA519-HW-ALL"
%%% End:
