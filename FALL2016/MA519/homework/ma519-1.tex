\begin{problem}[Handout 1, \# 5 \protect{[Feller Vol.\@ 1]}]
  A closet contains five pairs of shoes. If four shoes are selected at
  random, what is the probability that there is at least one complete pair
  among the four?
\end{problem}
\begin{solution}
  Let \(\Omega\) denote the sample space and \(A\) denote the event that at
  least \(1\) complete pair of shoes is among the \(4\). We can reduce the
  problem of finding \(P(A)\) into finding the probabilities of the
  mutually exclusive events
 \[
   A_1\defeq\bigl\{\,\text{exactly \(1\) pair is among the \(4\)}\,\bigr\}
  \]
  and
  \[
    A_1\defeq\bigl\{\,\text{exactly \(2\) pairs are among the
      \(4\)}\,\bigr\}.
  \]
  since \(A=A_1\cup A_2\), and using the additivity of \(P\),
  \[
    P(A)=P(A_1)+P(A_2).
  \]
  (To keep the problem short, we will not show that
  \(A_1\cap A_2=\emptyset\) and \(A=A_1\cup A_2\).)

  First, let us count the number of sample points in \(\Omega\): since the
  closet contains \(5\) pairs of shoes it contains a total of \(10\) choose
  out of which we are selecting \(4\). Hence, the number of sample points
  is
  \begin{equation}
    \label{eq:1-1}
    \card\Omega=%
    \binom{10}{4}=%
    \frac{10!}{4!6!}=%
    \frac{10\cdot 9\cdot 8\cdot 7\cdot 6!}{4\cdot 3\cdot 2\cdot 6!}=%
    10\cdot 3\cdot 7=%
    210.
  \end{equation}

  Now we count the sample points in \(A_1\) and \(A_2\): counting the
  points in \(A_2\) is immediate since we are not taking into consideration
  the order in which we select the pair
  \begin{equation}
    \label{eq:1-2}
    \card A_2=%
    \binom{5}{2}=%
    \frac{5!}{2!3!}=%
    \frac{5\cdot 4\cdot 3!}{2\cdot 3!}=%
    5\cdot 2=10.
  \end{equation}
  Counting the points in \(A_1\) is not much harder:
  first, we observe that there are \(5\) pairs to
  choose from and for the remaining two shoes we must choose one shoe
  (either a right or a left) from the remaining \(4\) pairs which leaves
  \(7-1=6\) other shoes to choose from; \ie{} the number of sample points
  in \(A_1\) is
  \begin{equation}
    \label{eq:1-3}
    5\cdot 4\cdot 6=120.
  \end{equation}
  Taking the results of \eqref{eq:1-1}, \eqref{eq:1-2} and \eqref{eq:1-3},
  the probability that there is at least one complete pair among the four
  is
  \[
    P(A)=%
    P(A_1)+P(A_2)=%
    \frac{120}{210}+\frac{10}{210}=\frac{130}{210}\approx%
    0.6190.
  \]
\end{solution}
\newpage

\begin{problem}[Handout 1, \# 7 \protect{[Feller Vol.\@ 1]}]
  A gene consists of \(10\) subunits, each of which is normal or
  mutant. For a particular cell, there are \(3\) mutant and \(7\) normal
  subunits. Before the cell divides into \(2\) daughter cells, the gene
  duplicates. The corresponding gene of cell \(1\) consists of \(10\)
  subunits chosen from the \(6\) mutant and \(14\) normal units. Cell \(2\)
  gets the rest. What is the probability that one of the cells consists of
  all normal subunits.
\end{problem}
\begin{solution}
  We shall employ the sames strategy as that of Problem 1.1. Let \(A\)
  denote the event that one of the cells contains all normal units. Then,
  like Problem 1.1, we can reduce the problem of finding the probability of
  \(A\) to finding the probability of
  \[
    A_1\defeq\bigl\{\,\text{cell \(1\) consists of all normal subunits}\,\bigr\}
  \]
  and
  \[
    A_2\defeq\bigl\{\,\text{cell \(1\) contains \(6\) mutant cells}\,\bigr\}
  \]
  and taking their sum.

  Now, let us count the number of points in our sample space
  \(\Omega\). Assuming the configuration of the subunits in a gene does not
  matter, we have
  \begin{equation}
    \label{eq:1-4}
      \card\Omega=\binom{20}{10}=184756
  \end{equation}
  sample points.

  Now we count the number of points in \(A_1\) and \(A_2\) these are: for
  \(A_1\) we choose \(10\) subunits from among the \(14\) normal subunits
  giving us
  \begin{equation}
    \label{eq:1-5}
    \card A_1=\binom{14}{10}=1001
  \end{equation}
  sample points. For \(A_2\), we must choose all \(6\) mutant subunits
  leaving \(4\) choices from among the \(14\) normal subunits giving us
  \begin{equation}
    \label{eq:1-6}
    \card A_1=\binom{14}{4}=1001.
  \end{equation}
  Thus, we have
  \[
    P(A)=%
    P(A_1)+P(A_2)=%
    \frac{1001}{184756}+\frac{1001}{184756}\approx%
    0.01083.
  \]
\end{solution}
\newpage

\begin{problem}[Handout 1, \# 9 \protect{[Feller Vol.\@ 1]}]
  From a sample of size \(n\), \(r\) elements are sampled at random. Find
  the probability that none of the \(N\) prespecified elements are included
  in the sample, if sampling is
  \begin{enumerate}[label=(\alph*)]
  \item with replacement;
  \item without replacement.
  \end{enumerate}
  Compute it for \(r=N=10\), \(n=100\).
\end{problem}
\begin{solution}
  For part (a), with replacement, the number of points in the sample space
  \(\Omega_a\) is given by the expression
  \[
    \card\Omega_a=\binom{n+r-1}{r}.
  \]
  Let \(A_a\) be the event that none of the \(N\) prespecified elements
  appear (with \(N\leq r\)). Now to find \(P(A_a)\), we count the sample
  points in \(A_a\) these are: there are \(N\) elements to avoid so \(n-N\)
  elements to choose from with replacement. This gives us
  \[
    \card A_a=\binom{(n-N)+r-1}{r}.
  \]
  Thus, the probability of \(A_a\) happening is
  \begin{equation}
    \label{eq:prob-a-a}
    P(A_a)=%
    \binom{(n-N)+r-1}{r}\biggl/\binom{n+r-1}{r}=%
    \frac{(n-1)\dotsm\bigl((n-1)-N+1\bigr)}
    {(n+r-1)\dotsm\bigl((n+r-1)-N+1\bigr)}.
  \end{equation}

  For part (b), without replacement, the number of points in the sample
  space \(\Omega_b\) is given by the expression
  \[
    \card\Omega_b=\binom{n}{r}.
  \]
  Let \(A_b\) be the event that none of the \(N\) prespecified elements
  appear (with \(N\leq r\)). Again, to find \(P(A_b)\) we need only count
  the sample points in \(A_b\): there are \(N\) elements to avoid so
  \(n-N\) elements to choose from without replacement. Hence,
  \[
    \card A_b=\binom{n-N}{r}.
  \]
  Thus, the probability of \(A_b\) happening is
  \begin{equation}
    \label{eq:prob-a-b}
    P(A_b)=%
    \binom{n-N}{r}\biggl/\binom{n}{r}=%
    \frac{(n-1)\dotsm (n-N)}{(n+r-1)\dotsm (n+r-N)}.
  \end{equation}

  Lastly, we compute, using Eqs.\@ \eqref{eq:prob-a-a} and
  \eqref{eq:prob-a-b}, we compute the probabilities in (a) and (b) with
  \(r=N=10\) and \(n=100\). These are:
  \[
    P(A_a)=\frac{99\dotsm 90}{109\dotsm 100}\approx 0.3654,
  \]
  and
  \[
    P(A_b)=\frac{90\dotsm 81}{100\dotsm 91}\approx 0.3305.
  \]
\end{solution}
\newpage

\begin{problem}[Handout 1, \# 11 \protect{[Text 1.3]}]
  A telephone number consists of ten digits, of which the first digit is
  one of \(1,2,\dotsc,9\) and the others can be \(0,1,2,\dotsc,9\). What is
  the probability that \(0\) appears at most once in a telephone number, if
  all the digits are chosen completely at random?
\end{problem}
\begin{solution}
  Let \(\Omega\) be the sample space and let \(A\) be the event that at
  \(0\) appears at most once in a telephone number if all the digits are
  chosen completely at random. First, let us count the number of elements
  in the sample space, this is
  \[
    \card\Omega=9\cdot 10^9
  \]
  where the first digit is taken from among \(1,2,\dotsc,9\) and the
  remaining \(9\) out of \(0,1,2\dotsc,9\). Assuming randomness (\ie{} that
  every sample point is equally likely), it suffices to count the sample
  points in the event. We do this by decomposing \(A\) into the union of
  mutually exclusive events
  \[
    A_i=\bigl\{\,\text{telephone numbers with exactly one \(0\) in the
      \(i\)-th position}\,\bigr\}.
  \]
  The number of sample points in \(A_i\) is
  \[
    \card A_i=9\cdot 9^8
  \]
  since we must choose \(8\) digits of the number from among \(1,\dotsc,9\)
  digits (with repetition). Thus,
  \[
    P(A)=%
    P(A_1)+\dotsm P(A_9)=%
    \frac{9\cdot 9\cdot 9^8}{9\cdot 10^9}=%
    \left(\frac{9}{8}\right)^9\approx 0.3874.
  \]
\end{solution}
\newpage

\begin{problem}[Handout 1, \# 12 \protect{[Text 1.6]}]
  Events \(A\), \(B\) and \(C\) are defined in a sample space
  \(\Omega\). Find expressions for the following probabilities in terms of
  \(P(A)\), \(P(B)\), \(P(C)\), \(P(AB)\), \(P(AC)\), \(P(BC)\) and
  \(P(ABC)\); here \(AB\) means \(A\cap B\), \etc{}:
  \begin{enumerate}[label=(\alph*)]
  \item the probability that exactly two of \(A\), \(B\), \(C\) occur;
  \item the probability that exactly one of these events occur;
  \item the probability that none of these events occur.
  \end{enumerate}
\end{problem}
\begin{solution}
  These are all easy consequences of the inclusion-exclusion formula. For
  part (a) we have is \(AB+AC+BC-ABC\)
  \[
    P(AB+AC+BC)=P(AB)+P(AC)+P(BC)-2P(ABC).
  \]

  For part (b) we have
  \[
    P(A)+P(B)+P(C)-P(AB)-P(AC)-P(BC)+P(ABC).
  \]

  Lastly, for part (c) we have
  \[
    P(\Omega)-P(A)-P(B)-P(C)+P(AB)+P(AC)+P(BC)+2P(ABC).
  \]
\end{solution}
\newpage

\begin{problem}[Handout 1, \# 13 \protect{[Text 1.8]}]
  Mrs.\@ Jones predicts that if it rains tomorrow it is bound to rain the
  day after tomorrow. She also thinks that the chance of rain tomorrow is
  \(1/2\) and that the chance of rain the day after tomorrow is
  \(1/3\). Are these subjective probabilities consistent with the axioms
  and theorems of probability?
\end{problem}
\begin{solution}
  The probabilities seem to be consistent with the axioms and theorems of
  probability. If the event \(A\) it will rain tomorrow implies that \(B\)
  it will rain the day after tomorrow then \(P(B)\leq P(A)\).
\end{solution}
\newpage

\begin{problem}[Handout 1, \# 16]
  Consider a particular player, say North, in a Bridge game. Let \(X\) be
  the number of aces in his hand. Find the distribution of \(X\).
\end{problem}
\begin{solution}

\end{solution}
\newpage

\begin{problem}[Handout 1, \# 20]
  If \(100\) balls are distributed completely at random into \(100\) cells,
  find the expected value of the number of empty cells.
  \\\\
  Replace \(100\) by \(n\) and derive the general expression. Now
  approximate it as \(n\) tends to \(\infty\).
\end{problem}
\begin{solution}
\end{solution}

%%% Local Variables:
%%% mode: latex
%%% TeX-master: "../MA519-Current-HW"
%%% End:
