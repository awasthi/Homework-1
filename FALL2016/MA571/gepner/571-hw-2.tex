\section{Homework 2}
\begin{exercise}
  A topological space \(X\) is said to be totally disconnected if a
  subspace \(Y\subseteq X\) is connected if and only if \(Y=\{x\}\)
  consists of only a single point \(x\in X\). Show that if \(X\) is
  discrete (that is, all subsets of \(X\) are open) then \(X\) is totally
  disconnected. Find an example of a totally disconnected space which is
  not discrete.
\end{exercise}
\begin{solution}
\end{solution}

\begin{exercise}
  Let \(X\) be a simply ordered set equipped with the order topology. Show
  that if \(X\) is connected then \(X\) is a continuum.
\end{exercise}
\begin{solution}
\end{solution}

\begin{exercise}
  Show that a metric \(d\colon X\times X\to\bfR\) on a set \(X\) determines
  a coarsest topology \(\calU\) on \(X\) for which the distance function
  \(d\colon X\times X\to\bfR\) is continuous, and give an explicit basis
  for this topology. Recall that a function \(f\colon X\to Y\) between
  metric spaces is said to be continuous at \(x\) if, for all
  \(\varepsilon>0\) there exists a \(\delta>0\) such that if
  \(d(x,x_0)<\delta\) then \(d\bigl(f(x),f(x_0)\bigr)<\varepsilon\); show
  that \(f\) is continuous (in the sense of topology) if and only if it is
  continuous at \(x\) for all \(x\in X\). Finally, show that every compact
  subspace of a metric space is closed and bounded, and find an example of
  a metric space for which there exists a closed and bounded subspace which
  is not compact.
\end{exercise}
\begin{solution}
\end{solution}

\begin{exercise}
  Let \(X\) be a compact space, \(Y\) a Hausdorff space, and
  \(f\colon X\to Y\) a continuous function. Show that \(f\) is a closed map
  (that is, \(f\) sends closed sets to closed sets), and also that the
  projection \(p\colon X\times Y\to Y\) is a closed map.
\end{exercise}
\begin{solution}
\end{solution}

\begin{exercise}
  Let \(f\colon W\to X\) and \(g\colon W\to Y\) be continuous
  functions. The pushout \(X\amalg_W Y\) of \(f\) and \(g\) is the
  quotient of the disjoint union \(X\amalg Y\) by the equivalence relation
  generated by the relation \(x\sim y\) if there exists a \(w\in W\) such
  that \(x=f(w)\) and \(y=g(w)\). Show that \(X\amalg_W Y\) comes equipped
  with continuous functions \(i\colon X\to X\amalg_W Y\) and
  \(j\colon Y\to X\amalg_W Y\) such that \(i\circ f=j\circ g\), and is
  universal among topological spaces \(Z\) equipped with continuous
  functions \(i'\colon X\to Z\) and \(j'\colon Y\to Z\) such that
  \(i'\circ f=j'\circ g\) in the following sense: given any such space
  \(Z\), there exists a unique continuous function
  \(k\colon X\amalg_W Y\to Z\) such that \(i'=k\circ i\) and \(j'=k\circ
  j\).
\end{exercise}
\begin{solution}
\end{solution}


%%% Local Variables:
%%% mode: latex
%%% TeX-master: "../MA571-Quals"
%%% End:
