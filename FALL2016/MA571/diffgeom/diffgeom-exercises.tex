\chapter{Differential Geometry Review}
Review exercises for differential geometry taken from Fernandes's notes
from UIUC's differential geometry course.
\section{Basic Concepts}
\subsection{Manifolds and subsets of Euclidean spaces}
\begin{proposition}
  Let \(X\subseteq\bbR^n\), \(Y\subseteq\bbR^m\) and
  \(Z\subseteq\bbR^p\). If \(f\colon X\to Y\) and \(g\colon Y\to Z\) are
  smooth maps, then \(g\circ f\colon X\to Z\) is also a smooth map.
\end{proposition}
\begin{proof}
  Consider the composition \(g\circ f\colon X\to Z\) of the smooth maps
  \(f\colon X\to Y\) and \(g\colon Y\to Z\). Fix a point \(x\in X\) and set
  \(y\defeq f(x)\). Since \(f\) and \(g\) are smooth, there exists
  neighborhoods \(U\subseteq\bbR^n\) and \(V\subseteq\bbR^m\) of \(x\) and
  \(y\), respectively, such that there are smooth maps
  \(F\colon U\to\bbR^m\) and \(G\colon V\to\bbR^p\) with
  \(F\restrict{X\cap U}=f\) and \(G\restrict{Y\cap V}=g\). Then, setting
  \(W\defeq U\cap F^{-1}(V)\), the map
  \[
    G\circ F\colon W\too\bbR^p
  \]
  is a smooth map since, fixing all but one variable \(x_i\), the partial
  \[
    \frac{\partial G\circ F}{\partial x_i}=
  \]
  . Thus, since
  \(G\circ F\restrict{X\cap W}=g\circ f\restrict{X\cap W}\), the
  composition \(g\circ f\) is smooth.
\end{proof}

\section{Lie Theory}
\section{Differential Forms}


%%% Local Variables:
%%% mode: latex
%%% TeX-master: "../MA571-Quals"
%%% End:
