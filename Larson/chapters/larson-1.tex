\chapter{Heuristics}

Typical heuristics that have helped people solve problems in the past
\begin{enumerate}[label=(\arabic*),noitemsep]
\item Search for a pattern.
\item Draw a figure.
\item Formulate an equivalent problem.
\item Modify the problem.
\item Choose effective notation.
\item Exploit symmetry.
\item Divide into cases.
\item Work backward.
\item Argue by contradiction.
\item Pursue parity.
\item Consider extreme cases.
\item Generalize.
\end{enumerate}
\section{Search for a Pattern}
Virtually all problem solvers begin their analysis by getting a feel for
the problem, then convincing themselves of the plausibility of the
result. This is best done by examining the most immediate special cases;
when this exploration is undertaken in a systematic way, patterns may
emerge that will suggest ideas for proceeding with the problem.
\begin{exercise}
Prove that the set of $n$ (different) elements has exactly $2^n$
(different) subsets.
\end{exercise}
\begin{proof}[Solution 1]
We begin by examining what happens when the set contains $0,1,2,3$
elements; the results are show in the following table:
\begin{tabular}{llll}
$n$&Elements of $S$&Subsets of $S$&Number of subsets of $S$\\
$0$&none&$\emptyset$&$1$\\
$1$&$x_1$&$\emptyset,\{x_1\}$&$2$\\
$2$&$x_1,x_2$&$\emptyset,\{x_1\},\{x_2\},\{x_1,x_2\}$&$4$\\
$3$&$x_1,x_2,x_3$&$\emptyset,\{x_1\},\{x_2\},\{x_3\}$&$8$\\
&&$\{x_1,x_2\},\{x_2,x_3\},\{x_1,x_2,x_3\}$
\end{tabular}

Our purpose in constructing this table is not only to verify the result,
but also to look for patterns that might suggest how to proceed in the
general case. Thus, we aim to be as systematic as possible. In this case,
notice when $n=3$, we have listed first the subsets of $\{x_1,x_2\}$ and
then, in the second line, each of these subsets augmented by the element
$x_3$. This is the key idea that allows us to proceed to higher values of
$n$. For example, when $n=4$, the subsets of $S=\{x_1,x_2,x_3,x_4\}$ are
the eight subsets of $\{x_1,x_2,x_3\}$ (shown in the table) together with
the eight formed by adjoining $x_4$ to each of these. These sixteen subsets
constitute the entire collection of possibilities; thus, a set with $4$
elements has $2^4(=16)$ subsets.

A proof based on this idea is an easy application of mathematical
induction.
\end{proof}
\begin{proof}[Solution 2]
Another way to present the idea of the last solution is to argue as
follows. For each $n$, let $A_n$ denote the number of (different) subsets
of a set with $n$ (different) elements. Let $S$ be the set with $n+1$
elements, and designate one of its elements by $x$. There is a one-to-one
correspondence between those subsets of $S$ which do not contain $x$ and
those subsets that do contain $x$ (namely, a subset $T$ of the former type
corresponds to $T\cup\{x\}$). The former types are all subsets of
$S\minus\{x\}$, a set with $n$ elements, and therefore, it must be the case
that
\[
A_{n+1}=2A_n.
\]
This recurrence relation, true for $n=0,1,2,3,\dotsc$, combined with the
fact that $A_0=1$, implies that
$A_n=2^n$. ($A_n=2A_{n-1}=2^2A_{n-2}=\cdots=2^nA_0=2^n$.)
\end{proof}
\begin{proof}[Solution 3]
Another systematic enumeration of subsets can be carried out by
constructing a ``tree.'' For the case $n=3$ and $S=\{a,b,c\}$, the tree is
as shown below

Each branch of the tree corresponsd to a distinct subset of $$ (teh bar
over the name of the leement means that it is not included in the set
corresponding to that branch). The tree is constructed in three stages,
corresponding to the three elements of $S$. Each element of $S$ leads to
two possibilities: either it is in the subset or it is not, and these
choices are represented by two branches. As each element is considered, the
number of branches doubles. Thus, for three-elements, the number of
branches is $2\times 2\times 2=8$. For an $n$-element set, the number of
branches is
\[
\underbrace{2\times\dotsm\times 2
\end{proof}
\begin{proof}[My solution]
\end{proof}

%%% Local Variables:
%%% mode: latex
%%% TeX-master: "../Larson-Exercises"
%%% End:
