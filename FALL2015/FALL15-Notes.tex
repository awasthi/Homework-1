\def\documentauthor{Carlos Salinas}
\def\documenttitle{Fall 2015 Notes}
\def\coursename{FALL15-Notes}
\def\shorttitle{FALL15-Notes}
\def\documentsubject{commutative algebra, point-set topology,
  wavelets and approximation theory, abstract algebra}
\def\authoremail{salinac@purdue.edu}

\documentclass[article,oneside,10pt]{memoir}
\usepackage{geometry}
\usepackage[dvipsnames]{xcolor}
\usepackage[
    breaklinks,
    bookmarks=true,
    colorlinks=true,
    pageanchor=false,
    linkcolor=black,
    anchorcolor=black,
    citecolor=black,
    filecolor=black,
    menucolor=black,
    runcolor=black,
    urlcolor=black,
    hyperindex=false,
    hyperfootnotes=true,
    pdftitle={\shorttitle},
    pdfauthor={\documentauthor},
    pdfkeywords={\documentsubject},
    pdfsubject={\coursename}
    ]{hyperref}

\usepackage{graphicx}
\graphicspath{{figures/}}

%% PDFLaTeX
\usepackage[mathcal]{euscript}
\usepackage{mathrsfs}
\usepackage[T2A,T1]{fontenc}
\usepackage[utf8]{inputenc}
\usepackage[french,german,russian,spanish,english]{babel}
\babeltags{fr=french,
           de=german,
           ru=russian,
           es=spanish,
           en=english}
\def\spanishoptions{mexico}
\usepackage{CJKutf8}
\newcommand{\textzh}[1]{\begin{CJK}{UTF8}{bsmi}#1\end{CJK}}
\newcommand{\textko}[1]{\begin{CJK}{UTF8}{mj}#1\end{CJK}}

% Misc
\usepackage{microtype}
\usepackage{multicol}
\usepackage[inline]{enumitem}
\usepackage{listings}
\usepackage{mleftright}
\mleftright

%% Math
\usepackage{amsthm}
\usepackage{amssymb}
\usepackage{mathtools}

\newcommand\nsubset{\ensuremath{\not\subset}}
\renewcommand\qedsymbol{\ensuremath{\null\hfill\blacksquare}}

\def\upint{\mathchoice%
    {\mkern13mu\overline{\vphantom{\intop}\mkern7mu}\mkern-20mu}%
    {\mkern7mu\overline{\vphantom{\intop}\mkern7mu}\mkern-14mu}%
    {\mkern7mu\overline{\vphantom{\intop}\mkern7mu}\mkern-14mu}%
    {\mkern7mu\overline{\vphantom{\intop}\mkern7mu}\mkern-14mu}%
  \int}
\def\lowint{\mkern3mu\underline{\vphantom{\intop}\mkern7mu}\mkern-10mu\int}

%% XeTeX specific
% \usepackage{ifxetex}
% \ifxetex
% \usepackage{unicode-math}

% \setmainfont[Ligatures=TeX]{Latin Modern Roman}
% \setsansfont[Ligatures=TeX]{Latin Modern Sans}
% \setmonofont{Latin Modern Mono}
% \setmathfont{Latin Modern Math}

% %% Patch arrows for XeTeX
% \usepackage{etoolbox}
% \makeatletter
% \patchcmd{\arrowfill@}{-7mu}{-14mu}{}{}
% \patchcmd{\arrowfill@}{-7mu}{-14mu}{}{}
% \patchcmd{\arrowfill@}{-2mu}{-4mu}{}{}
% \patchcmd{\arrowfill@}{-2mu}{-4mu}{}{}
% \makeatother

% \renewcommand\qedsymbol{\ensuremath{\null\hfill\QED}}
% \fi

%% Theorems and definitions
\theoremstyle{plain}
\newtheorem{theorem}{Theorem}
\newtheorem{proposition}[theorem]{Proposition}
\newtheorem{corollary}[theorem]{Corollary}
\newtheorem{claim}[theorem]{Claim}
\newtheorem{lemma}[theorem]{Lemma}
\newtheorem{axiom}[theorem]{Axiom}

\newtheorem*{corollary*}{Corollary}
\newtheorem*{claim*}{Claim}
\newtheorem*{lemma*}{Lemma}
\newtheorem*{proposition*}{Proposition}
\newtheorem*{theorem*}{Theorem}

\theoremstyle{definition}
\newtheorem{definition}{Definition}
\newtheorem{example}{Examples}
\newtheorem{examples}[example]{Examples}
\newtheorem{exercise}{Exercise}[section]
\newtheorem{problem}[exercise]{Problem}

\newtheorem*{definition*}{Definition}
\newtheorem*{example*}{Examples}
\newtheorem*{examples*}{Examples}
\newtheorem*{exercise*}{Exercise}
\newtheorem*{problem*}{Problem}

\theoremstyle{remark}
\newtheorem{remark}{Remark}
\newtheorem{remarks}[remark]{Remarks}
\newtheorem{observation}[remark]{Observation}
\newtheorem{observations}[remark]{Observations}

\newtheorem*{remark*}{**Remark**}
\newtheorem*{remarks*}{**Remarks**}
\newtheorem*{observation*}{**Observation**}
\newtheorem*{observations*}{**Observations**}

%% Redefinitions & commands
\newcommand\restr[2]{{% we make the whole thing an ordinary symbol
  \left.\kern-\nulldelimiterspace % automatically resize the bar with \right
  {#1} % the function
  % \vphantom{\big|} % pretend it's a little taller at normal size
  \right|{#2} % this is the delimiter
  }}

%% Commands and operators
\DeclareMathOperator{\id}{id}
\DeclareMathOperator{\im}{im}
\DeclareMathOperator{\Int}{int}
\DeclareMathOperator{\Cl}{cl}

\newcommand{\clsr}[1]{\overline{#1}}
\newcommand{\CC}{\mathbf{C}}
\newcommand{\NN}{\mathbf{N}}
\newcommand{\QQ}{\mathbf{Q}}
\newcommand{\RR}{\mathbf{R}}
\newcommand{\ZZ}{\mathbf{Z}}

\begin{document}
\let\setminus\relax
\let\phi\relax
\let\epsilon\relax
\newcommand\setminus{\smallsetminus}
\newcommand\phi{\varphi}
\newcommand\epsilon{\varepsilon}

\author{\href{mailto:\authoremail}{\documentauthor}}
\title{\documenttitle}
\date{\today}
\maketitle
\tableofcontents

\renewcommand{\thefootnote}{\fnsymbol{footnote}}
\numberwithin{equation}{section}
% \chapter{MA557 Notes (Fall 2015)}
\section{Lecture 1}


%%% Local Variables:
%%% mode: latex
%%% TeX-master: "../../FALL15-Notes"
%%% End:

% \include{MA557/lectures/ma557-lecture-2}
% \include{MA557/lectures/ma557-lecture-3}
% \include{MA557/lectures/ma557-lecture-4}
% \include{MA557/lectures/ma557-lecture-5}
% \include{MA557/lectures/ma557-lecture-6}
% \include{MA557/lectures/ma557-lecture-7}
% \include{MA557/lectures/ma557-lecture-8}

% \include{MA571/lectures/ma571-lecture-1}
% \include{MA571/lectures/ma571-lecture-2}
% \include{MA571/lectures/ma571-lecture-3}
% \include{MA571/lectures/ma571-lecture-4}
% \include{MA571/lectures/ma571-lecture-5}
% \include{MA571/lectures/ma571-lecture-6}
% \include{MA571/lectures/ma571-lecture-7}
% \include{MA571/lectures/ma571-lecture-8}
\section{McClure's Practice Problems}
\begin{problem}[Munkres \S25, p.\,162, \#6]
A space $X$ is said to be \emph{weakly locally connected at $x$}
if for every neighborhood $U$ of $x$, there is a connected
subspace of $X$ contained in $U$ that contains a neighborhood of
$x$. Show that if $X$ is weakly locally connected at each of its
points, then $X$ is locally connected [\emph{Hint:} Show that
components of open sets are open.]
\end{problem}
\begin{problem}
Let $X$ be locally path connected
\end{problem}

%%% Local Variables:
%%% mode: latex
%%% TeX-master: "../MA571-HW-Current"
%%% End:

\chapter{Kaufmann's 571 Problems}
\section{Midterm (Fall 2014)}

\section{Final (Fall 2014)}

%%% Local Variables:
%%% mode: latex
%%% TeX-master: "../../FALL15-Notes"
%%% End:


% \chapter{MA692 (Wavelets and Approximation Theory) Notes (Fall 2015)}
\section{Lecture 1}

%%% Local Variables:
%%% mode: latex
%%% TeX-master: "../../FALL15-Notes"
%%% End:

% \include{MA692-Wav/lectures/ma692-wav-lecture-2}
% \include{MA692-Wav/lectures/ma692-wav-lecture-3}
% \include{MA692-Wav/lectures/ma692-wav-lecture-4}
% \include{MA692-Wav/lectures/ma692-wav-lecture-5}
% \include{MA692-Wav/lectures/ma692-wav-lecture-6}
% \include{MA692-Wav/lectures/ma692-wav-lecture-7}
% \include{MA692-Wav/lectures/ma692-wav-lecture-8}

\chapter{MA553 Qual Problems}
\section{Goins}
\section{Goldberg}
\section{Ulrich}

%%% Local Variables:
%%% mode: latex
%%% TeX-master: "../FALL15-Notes"
%%% End:

\end{document}

%%% Local Variables:
%%% mode: latex
%%% TeX-master: t
%%% End:
