\def\documentauthor{Carlos Salinas}
\def\documenttitle{Fall 2015 Notes}
\def\coursename{FALL15-Notes}
\def\shorttitle{FALL15-Notes}
\def\documentsubject{commutative algebra, point-set topology,
  wavelets and approximation theory, abstract algebra}
\def\authoremail{salinac@purdue.edu}

\documentclass[article,oneside,10pt]{memoir}
\usepackage{geometry}

\usepackage[dvipsnames]{xcolor}
\usepackage[
    breaklinks,
    bookmarks=true,
    colorlinks=true,
    pageanchor=false,
    linkcolor=black,
    anchorcolor=black,
    citecolor=black,
    filecolor=black,
    menucolor=black,
    runcolor=black,
    urlcolor=black,
    hyperindex=false,
    hyperfootnotes=true,
    pdftitle={\shorttitle},
    pdfauthor={\documentauthor},
    pdfkeywords={\documentsubject},
    pdfsubject={\coursename}
    ]{hyperref}

\usepackage{graphicx}
\graphicspath{{figures/}}

\usepackage{multicol}
\usepackage[inline]{enumitem}
\usepackage{listings}
\usepackage{mleftright}
\mleftright

\usepackage{microtype}

\usepackage{amsthm}
\usepackage{amssymb}
\usepackage{mathtools}
% \usepackage[T1]{fontenc}
% \usepackage{lmodern}
\usepackage{unicode-math}

\setmainfont[Ligatures=TeX]{Latin Modern Roman}
\setsansfont[Ligatures=TeX]{Latin Modern Sans}
\setmonofont{Latin Modern Mono}
\setmathfont{Latin Modern Math}

\theoremstyle{plain}
\newtheorem{theorem}{Theorem}[section]
\newtheorem{proposition}[theorem]{Proposition}
\newtheorem{corollary}[theorem]{Corollary}
\newtheorem{claim}[theorem]{Claim}
\newtheorem{lemma}[theorem]{Lemma}
\newtheorem{axiom}[theorem]{Axiom}

\newtheorem*{corollary*}{Corollary}
\newtheorem*{claim*}{Claim}
\newtheorem*{lemma*}{Lemma}
\newtheorem*{proposition*}{Proposition}
\newtheorem*{theorem*}{Theorem}

\theoremstyle{definition}
\newtheorem{definition}{Definition}[section]
\newtheorem{example}{Example}[section]
\newtheorem{examples}[example]{Examples}
\newtheorem{exercise}{Exercise}[section]
\newtheorem{problem}[exercise]{Problem}

\newtheorem*{definition*}{Definition}
\newtheorem*{example*}{Examples}
\newtheorem*{examples*}{Examples}
\newtheorem*{exercise*}{Exercise}
\newtheorem*{problem*}{Problem}

\theoremstyle{remark}
\newtheorem{remark}{Remark}
\newtheorem{remarks}[remark]{Remarks}
\newtheorem{observation}[remark]{Observation}
\newtheorem{observations}[remark]{Observations}

\newtheorem*{remark*}{**Remark**}
\newtheorem*{remarks*}{**Remarks**}
\newtheorem*{observation*}{**Observation**}
\newtheorem*{observations*}{**Observations**}

%%% Patch arrows
\usepackage{etoolbox}
\makeatletter
\patchcmd{\arrowfill@}{-7mu}{-14mu}{}{}
\patchcmd{\arrowfill@}{-7mu}{-14mu}{}{}
\patchcmd{\arrowfill@}{-2mu}{-4mu}{}{}
\patchcmd{\arrowfill@}{-2mu}{-4mu}{}{}
\makeatother

%% Redefinitions & commands
\renewcommand\qedsymbol{\ensuremath{\null\hfill\QED}}
% \renewcommand\qedsymbol{\ensuremath{\null\hfill\blacksquare}}

\newcommand\restr[2]{{% we make the whole thing an ordinary symbol
  \left.\kern-\nulldelimiterspace % automatically resize the bar with \right
  {#1} % the function
  \right|_{#2} % this is the delimiter
  }}

%% Commands and operators
\newcommand{\id}{\mathrm{id}}
\newcommand{\im}{\mathrm{im}}

\newcommand{\clsr}[1]{\overline{#1}}

\newcommand{\GL}{\mathrm{GL}}
\newcommand{\GO}{\mathrm{GO}}
\newcommand{\OO}{\mathrm{O}}
\newcommand{\SL}{\mathrm{SL}}
\newcommand{\SO}{\mathrm{SO}}

\newcommand{\CC}{\mathbf{C}}
\newcommand{\CP}{\mathbf{CP}}
\newcommand{\GG}{\mathbf{G}}
\newcommand{\NN}{\mathbf{N}}
\newcommand{\PP}{\mathbf{P}}
\newcommand{\QQ}{\mathbf{Q}}
\newcommand{\RP}{\mathbf{RP}}
\newcommand{\RR}{\mathbf{R}}
\newcommand{\Sphere}{\mathbf{S}}
\newcommand{\Torus}{\mathbf{T}}
\newcommand{\ZZ}{\mathbf{Z}}

\begin{document}
\let\setminus\relax
\let\phi\relax
\let\epsilon\relax
\newcommand\setminus{\smallsetminus}
\newcommand\phi{\varphi}
\newcommand\epsilon{\varepsilon}

\author{\href{mailto:\authoremail}{\documentauthor}}
\title{\documenttitle}
\date{\today}
\maketitle
\tableofcontents

\renewcommand{\thefootnote}{\fnsymbol{footnote}}
\numberwithin{equation}{section}
% \chapter{MA557 Notes (Fall 2015)}
\section{Lecture 1 (August 24, 2015)}
\subsection{General Facts on Rings}
\begin{itemize}[noitemsep]
\item $R$ is a ring if
  \begin{enumerate}[noitemsep,label=(\roman*)]
  \item $R$ is an Abelian group with respect to $+$.
  \item $\cdot$ is associative, commutative, distributive and has
    $1$.
  \end{enumerate}
\item $R$, $S$ rings, $\phi\colon R\to S$ is a
  \emph{homomorphism} (of rings) if
  \begin{enumerate}[noitemsep,label=(\roman*)]
  \item $\phi(x+y)=\phi(x)+\phi(y)$.
  \item $\phi(1_R)=1_S$.
  \end{enumerate}
\item $I\subset R$ is an \emph{$R$-ideal} if $I$ is a subgroup of
  $R$ with respect to $+$ and $RI\subset I$.
\end{itemize}



%%% Local Variables:
%%% mode: latex
%%% TeX-master: "../../FALL15-Notes"
%%% End:

% \section{}

%%% Local Variables:
%%% mode: latex
%%% TeX-master: "../../FALL15-Notes"
%%% End:

% \section{}

%%% Local Variables:
%%% mode: latex
%%% TeX-master: "../../FALL15-Notes"
%%% End:

% \section{}

%%% Local Variables:
%%% mode: latex
%%% TeX-master: "../../FALL15-Notes"
%%% End:

% \section{}

%%% Local Variables:
%%% mode: latex
%%% TeX-master: "../../FALL15-Notes"
%%% End:

% \section{}

%%% Local Variables:
%%% mode: latex
%%% TeX-master: "../../FALL15-Notes"
%%% End:

% \section{}

%%% Local Variables:
%%% mode: latex
%%% TeX-master: "../../FALL15-Notes"
%%% End:

% \section{}

%%% Local Variables:
%%% mode: latex
%%% TeX-master: "../../FALL15-Notes"
%%% End:


% \chapter{MA571 Notes (Fall 2015)}
\section{Lecture 1}

%%% Local Variables:
%%% mode: latex
%%% TeX-master: "../../FALL15-Notes"
%%% End:

% \section{}

%%% Local Variables:
%%% mode: latex
%%% TeX-master: "../../FALL15-Notes"
%%% End:

% \section{}

%%% Local Variables:
%%% mode: latex
%%% TeX-master: "../../FALL15-Notes"
%%% End:

% \section{}

%%% Local Variables:
%%% mode: latex
%%% TeX-master: "../../FALL15-Notes"
%%% End:

% \section{}

%%% Local Variables:
%%% mode: latex
%%% TeX-master: "../../FALL15-Notes"
%%% End:

% \section{}

%%% Local Variables:
%%% mode: latex
%%% TeX-master: "../../FALL15-Notes"
%%% End:

% \section{}

%%% Local Variables:
%%% mode: latex
%%% TeX-master: "../../FALL15-Notes"
%%% End:

% \section{}

%%% Local Variables:
%%% mode: latex
%%% TeX-master: "../../FALL15-Notes"
%%% End:

\chapter{McClure's 571 Problems}
\section{Midterm I (Fall 2015)}
\begin{problem}
Let $A\subset X$ and $B\subset Y$. Show that the space $X\times Y$,
\[\clsr{A\times B}=\clsr A\times\clsr B.\]
\end{problem}
\begin{proof}
\end{proof}
\begin{problem}
Let $X$ bea topological space and let $A$ be a dense subset of
$X$. Let $Y$ be a Hausdorff space and let $g,h\colon X\to Y$ be
continuous functions which agree on $A$. Prove that $g=h$.
\end{problem}
\begin{proof}
\end{proof}
\begin{problem}
Let $X$ and $Y$ be topological spaces and let $f\colon X\to Y$
be a continuous function. Let $G_f$ (called the \emph{graph} of
$f$) be the subspace $\left\{\,x\times f(x)\;\middle|\;x\in
  X\,\right\}$ of $X\times Y$. Prove that if $Y$ is Hausdorff then
$G_f$ is closed.
\end{problem}
\begin{proof}
\end{proof}
\begin{problem}
Let $X$ be a topological space and let $f,g\colon X\to\RR$ be
continuous. Define $h\colon X\to\RR$ by
\[
h(x)=\min\left\{(f(x),g(x)\right\}.
\]
Use the pasting lemma to prove that $h$ is continuous. (You will
not get full credit for any other method.)
\end{problem}
\begin{proof}
\end{proof}
\begin{problem}
Let $X$ and $Y$ be topological spaces and let $f\colon X\to Y$ be
a function with the property that
\[
f(\clsr A)\subset\clsr{f(A)}
\]
for all subsets $A$ of $X$. Prove that $f$ is continuous.
\end{problem}
\begin{proof}
\end{proof}
\begin{problem}
Let $X$ and $Y$ be topological spaces and let $f\colon X\to Y$ be
a continuous function. Prove that
\[
f(\clsr A)\subset\clsr{f(A)}
\]
for all subsets $A$ of $X$.
\end{problem}
\begin{proof}
\end{proof}
\begin{problem}
Let $X$ be any topological space and let $Y$ be a Hausdorff
space. Let $f,g\colon X\to Y$ be continuous functions. Prove that
the set $\left\{\,x\in X\;\middle|\;f(x)=g(x)\,\right\}$ is
closed.
\end{problem}
\begin{proof}
\end{proof}
\begin{problem}
Let $X$ be a topological space and $A$ a subset of $X$. Suppose that
\[
A\subset\clsr{X\setminus\clsr A}.
\]
Prove that $\clsr{A}$ does not contain any nonempty open set.
\end{problem}
\begin{proof}
\end{proof}
\begin{problem}
Let $X$ be a topological space with a countable basis. Prove that
every open cover of $X$ has a countable subcover.
\end{problem}
\begin{proof}
\end{proof}
\begin{problem}
Let $X_\alpha$ be an infinite family of topological spaces.
\begin{enumerate}[noitemsep,label=(\alph*)]
\item Define the product topology on $\prod X_\alpha$.
\item For each $\alpha$, let $A_\alpha$ be a subspace of
  $X_\alpha$. Prove that $\clsr{\prod
    A_\alpha}=\prod\clsr{A_\alpha}$.
\end{enumerate}
\end{problem}
\begin{proof}
\end{proof}
\begin{problem}
Suppose that we are given an indexing set $A$, and for each
$\alpha\in A$ a topological space $X_\alpha$. Suppose also that
for each $\alpha\in A$ we are given a point $b_\alpha\in
X_\alpha$. Let $Y=\prod X_\alpha$ with the product topology. Let
$\pi_\alpha\colon Y\to X_\alpha$ be the projection. Prove that
the set
\[
S=\left\{\,y\in Y\;\middle|\;\text{$\pi_\alpha(y)=b_\alpha$ except for
    finitely many $\alpha$}\,\right\}
\]
is dense in $Y$ (that is, its closure is $Y$).
\end{problem}
\begin{proof}
\end{proof}
\begin{problem}
Let $X$ be the Cartesian product
$\RR^\omega=\prod_{i=1}^\infty\RR$ with the box topology (recall
that a basis for this topology consists of all sets of the form
$\prod_{i=1}^\infty U_i$, where each $U_i$ is open in $\RR$). Let
$f\colon\RR\to X$ be the function which takes $t$ to
$(t,t,t,...)$. Prove that $f$ is not continuous.
\end{problem}
\begin{proof}
\end{proof}
\begin{problem}
Prove that the countable product $\RR^\omega$ (with the product
topology) has the following property: there is a countable family
$\mathcal{F}$ of neighborhoods of the point
$\mathbf{0}=(0,0,0,...)$ such that for every neighborhood $V$ of
$\mathbf{0}$ there is a $U\in\mathcal{F}$ with $U\subset V$.
\\\\
Note: the book proves that $\RR^\omega$ is a metric space, but
you may not use this in your proof. Use the definition of the
product topology.
\end{problem}
\begin{proof}
\end{proof}
\begin{problem}
Let $X$ be the two-point set $\{0,1\}$ with the discrete
topology. Let $Y$ be a countable product of copies of $X$, thus
an element of $Y$ is a sequence of $0$'s and $1$'s. For each
$n\geq 1$, let $y_0\in Y$ be the element
$(1,1,1,...,1,0,0,0,..)$, with $n$ $1$'s at the beginning and all
other entries $0$. Let $y\in Y$ be the element with all
$1$s. Prove that the set $\left\{y_n\right\}_{n\geq 1}\cup\{y\}$
is closed. Give a clear explanation. Do not use a metric.
\end{problem}
\begin{proof}
\end{proof}
\begin{problem}
Let $X$ be the two-point set $\{0,1\}$ with the discrete
topology. Let $Y$ be a countable product of copies of $X$; thus
an element of $Y$ is a sequence of $0$'s and $1$'s. Let $A$ be
the subset of $Y$ consisting of sequences with only a finite
number of $1$'s. Is $A$ closed? Prove or disprove.
\end{problem}
\begin{proof}
\end{proof}
\begin{problem}
Let $Y$ be a topological space.Let $X$ be a set and let $f\colon
X\to Y$ be a function. Give $X$ the topology in which the open
sets are the sets $f^{-1}(V)$ with $V$ open in $Y$ (you do not
have to verify that this is a topology). Let $a\in X$ and let $B$
be a closed set in $X$ not containing $a$. Prove that $f(a)$ is
not in the closure of $f(B)$.
\end{problem}
\begin{proof}
\end{proof}
\begin{problem}
Let $f\colon X\to Y$ be a function that takes closed sets to
closed sets. Let $y\in Y$ and let $U$ be an open set containing
$f^{-1}(y)$. Prove that there is an open set $V$ containing $y$
such that $f^{-1}(V)$ is contained in $U$.
\end{problem}
\begin{proof}
\end{proof}
\begin{problem}
Let $X$ be a topological space with an equivalence relation
$\sim$. Suppose that the quotient space $X/{\sim}$ is
Hausdorff. Prove that the set $S=\left\{\,x\times y\in X\times
  X\;\middle|\;x\sim y\,\right\}$ is a closed subset of $X\times
X$.
\end{problem}
\begin{proof}
\end{proof}
\begin{problem}
Let $p\colon X\to Y$ be a quotient map. Let us say that a subset
$S$ of $X$ is \emph{saturated} if it has the form $p^{-1}(T)$ for
some subset $T$ of $Y$. Suppose that for every $y\in Y$ and every
open neighborhood $U$ of $p^{-1}(y)$ there is a saturated open
set $V$ with $p^{-1}(y)\subset V\subset U$. Prove that $p$ takes
closed sets to closed sets.
\end{problem}
\begin{proof}
\end{proof}
\begin{problem}
Let $X$ be a topological space, let $D$ be a connected subset of
$X$, and let $\left\{E_\alpha\right\}$ be a collection of
connected subsets of $X$.
\end{problem}
\begin{proof}
\end{proof}
\begin{problem}
Let $X$ and $Y$ be connected. Prove that $X\times Y$ is connected.
\end{problem}
\begin{proof}
\end{proof}
\begin{problem}
For any space $X$, let us say that two points are ``inseparable''
if there is no separation $X=U\cup V$ into disjoint open sets
such that $x\in U$ and $y\in V$. Write $x\sim y$ if $x$ and $y$
are inseparable. Then $\sim$ is an equivalence relation (you
don't have to prove this). Now suppose that $X$ is locally
connected (this means that for every point $x$ and every open
neighborhood $U$ of $x$, there is a connected open neighborhood
$V$ of $x$ contained in $U$). Prove that ecah equivalence class
of the relation $\sim$ is connected.
\end{problem}
\begin{proof}
\end{proof}
\begin{problem}
Let $X$ be a topological space. Let $A\subset X$ be
connected. Prove $\clsr A$ is connected.
\end{problem}
\begin{proof}
\end{proof}
\begin{problem}
Let $X_1,X_2,...$ be topological spaces. Suppose
$\prod_{n=1}^\infty X_n$ is locally connected. Prove that al lbut
finitely many $X_n$ are connected.
\end{problem}
\begin{proof}
\end{proof}
\begin{problem}
LEt $X$ be a connected space and let $f\colon X\to Y$ be a
function which is continuous and onto. Prove that $Y$ is
connected. (This is a theorem in Munkres---prove it from the
definitions).
\end{problem}
\begin{proof}
\end{proof}
\begin{problem}
Give:
\begin{enumerate}[noitemsep,label=(\roman*)]
\item $p\colon X\to Y$ is a quotient map.
\item $Y$ is connected.
\item For every $y\in Y$, the set $p^{-1}(y)$ is connected.
\end{enumerate}
Prove that $X$ is connected.
\end{problem}
\begin{proof}
\end{proof}
\begin{problem}
Let $A$ be a subset of $\RR^2$ which is homeomorphic to the open
unit interval $(0,1)$. Prove that $A$ does not contain a nonempty
set which is open in $\RR^2$.
\end{problem}
\begin{proof}
\end{proof}
\begin{problem}
Let $X$ be a connected space. Let $\mathcal{U}$ be an open
covering of $X$ and let $U$ be a nonempty set in
$\mathcal{U}$. Say that a set $V$ in $\mathcal{U}$ is
\emph{reachable from $U$} if there is a sequence
$U=U_1,U_2,...,U_n=V$ of sets in $\mathcal{U}$ such that $U_i\cap
U_{i+1}\neq\emptyset$ for each $i$ from $1$ to $n-1$. Prove that
every nonempty $V$ in $\mathcal{U}$ is reachable from $U$.
\end{problem}
\begin{proof}
\end{proof}
\begin{problem}
Suppose that $X$ is connected and every point of $X$ has a
path-connected open neighborhood. Prove that $X$ is
path-connected.
\end{problem}
\begin{proof}
\end{proof}
\begin{problem}
Let $X$ be a topological space and let $f,g\colon X\to[0,1]$ be
continuous functions. Suppose that $X$ is connected and $f$ is
onto. Prove that there must be a point $x\in X$ with
$f(x)=g(x)$.
\end{problem}
\begin{proof}
\end{proof}

\section{Midterm II (Fall 2015)}

%%% Local Variables:
%%% mode: latex
%%% TeX-master: "../../FALL15-Notes"
%%% End:

\chapter{Kaufmann's 571 Problems}
\section{Midterm (Fall 2014)}
\section{Final (Fall 2014)}

%%% Local Variables:
%%% mode: latex
%%% TeX-master: "../../FALL15-Notes"
%%% End:


% \chapter{MA692 (Wavelets and Approximation Theory) Notes (Fall 2015)}
\section{Lecture 1}

%%% Local Variables:
%%% mode: latex
%%% TeX-master: "../../FALL15-Notes"
%%% End:

% \section{Lecture }

%%% Local Variables:
%%% mode: latex
%%% TeX-master: "../../FALL15-Notes"
%%% End:

% \section{Lecture }

%%% Local Variables:
%%% mode: latex
%%% TeX-master: "../../FALL15-Notes"
%%% End:

% \section{Lecture }

%%% Local Variables:
%%% mode: latex
%%% TeX-master: "../../FALL15-Notes"
%%% End:

% \section{Lecture }

%%% Local Variables:
%%% mode: latex
%%% TeX-master: "../../FALL15-Notes"
%%% End:

% \section{Lecture }

%%% Local Variables:
%%% mode: latex
%%% TeX-master: "../../FALL15-Notes"
%%% End:

% \section{Lecture }

%%% Local Variables:
%%% mode: latex
%%% TeX-master: "../../FALL15-Notes"
%%% End:

% \section{Lecture }

%%% Local Variables:
%%% mode: latex
%%% TeX-master: "../../FALL15-Notes"
%%% End:


\chapter{MA553 Qual Problems}
\section{Goins}

\section{Goldberg}
\section{Ulrich}

%%% Local Variables:
%%% mode: latex
%%% TeX-master: "../FALL15-Notes"
%%% End:

\end{document}

%%% Local Variables:
%%% mode: latex
%%% TeX-master: t
%%% End:
