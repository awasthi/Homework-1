\def\documentauthor{Carlos Salinas}
\def\documenttitle{Fall 2015 Notes}
\def\coursename{FALL15-Notes}
\def\shorttitle{FALL15-Notes}
\def\documentsubject{commutative algebra, point-set topology,
  wavelets and approximation theory, abstract algebra}
\def\authoremail{salinac@purdue.edu}

\documentclass[article,oneside,10pt]{memoir}
\usepackage{geometry}

\usepackage[dvipsnames]{xcolor}
\usepackage[
    breaklinks,
    bookmarks=true,
    colorlinks=true,
    pageanchor=false,
    linkcolor=black,
    anchorcolor=black,
    citecolor=black,
    filecolor=black,
    menucolor=black,
    runcolor=black,
    urlcolor=black,
    hyperindex=false,
    hyperfootnotes=true,
    pdftitle={\shorttitle},
    pdfauthor={\documentauthor},
    pdfkeywords={\documentsubject},
    pdfsubject={\coursename}
    ]{hyperref}

\usepackage{graphicx}
\graphicspath{{figures/}}

\usepackage{multicol}
\usepackage[inline]{enumitem}
\usepackage{listings}
\usepackage{mleftright}
\mleftright

\usepackage{microtype}

\usepackage{amsthm}
\usepackage{amssymb}
\usepackage{mathtools}
\usepackage{lualatex-math}
\usepackage{unicode-math}

\setmainfont[Ligatures=TeX]{Latin Modern Roman}
\setsansfont[Ligatures=TeX]{Latin Modern Sans}
\setmonofont{Latin Modern Mono}
\setmathfont{Latin Modern Math}

\theoremstyle{plain}
\newtheorem{theorem}{Theorem}[section]
\newtheorem{proposition}[theorem]{Proposition}
\newtheorem{corollary}[theorem]{Corollary}
\newtheorem{claim}[theorem]{Claim}
\newtheorem{lemma}[theorem]{Lemma}
\newtheorem{axiom}[theorem]{Axiom}

\newtheorem*{corollary*}{Corollary}
\newtheorem*{claim*}{Claim}
\newtheorem*{lemma*}{Lemma}
\newtheorem*{proposition*}{Proposition}
\newtheorem*{theorem*}{Theorem}

\theoremstyle{definition}
\newtheorem{definition}{Definition}[section]
\newtheorem{example}{Example}[section]
\newtheorem{examples}[example]{Examples}
\newtheorem{exercise}{Exercise}[section]
\newtheorem{problem}[exercise]{Problem}

\newtheorem*{definition*}{Definition}
\newtheorem*{example*}{Examples}
\newtheorem*{examples*}{Examples}
\newtheorem*{exercise*}{Exercise}
\newtheorem*{problem*}{Problem}

\theoremstyle{remark}
\newtheorem{remark}{Remark}
\newtheorem{remarks}[remark]{Remarks}
\newtheorem{observation}[remark]{Observation}
\newtheorem{observations}[remark]{Observations}

\newtheorem*{remark*}{**Remark**}
\newtheorem*{remarks*}{**Remarks**}
\newtheorem*{observation*}{**Observation**}
\newtheorem*{observations*}{**Observations**}

%% Redefinitions & commands
\renewcommand\qedsymbol{\ensuremath{\null\hfill\QED}}

\newcommand\restr[2]{{% we make the whole thing an ordinary symbol
  \left.\kern-\nulldelimiterspace % automatically resize the bar with \right
  {#1} % the function
  \right|_{#2} % this is the delimiter
  }}

%% Commands and operators
\newcommand{\id}{\mathrm{id}}
\newcommand{\im}{\mathrm{im}}

\newcommand{\GL}{\mathrm{GL}}
\newcommand{\GO}{\mathrm{GO}}
\newcommand{\OO}{\mathrm{O}}
\newcommand{\SL}{\mathrm{SL}}
\newcommand{\SO}{\mathrm{SO}}

\newcommand{\CC}{\mathbf{C}}
\newcommand{\CP}{\mathbf{CP}}
\newcommand{\GG}{\mathbf{G}}
\newcommand{\NN}{\mathbf{N}}
\newcommand{\PP}{\mathbf{P}}
\newcommand{\QQ}{\mathbf{Q}}
\newcommand{\RP}{\mathbf{RP}}
\newcommand{\RR}{\mathbf{R}}
\newcommand{\Sphere}{\mathbf{S}}
\newcommand{\Torus}{\mathbf{T}}
\newcommand{\ZZ}{\mathbf{Z}}

\begin{document}
\let\setminus\relax
\let\phi\relax
\let\epsilon\relax
\newcommand\setminus{\smallsetminus}
\newcommand\phi{\varphi}
\newcommand\epsilon{\varepsilon}

\author{\href{mailto:\authoremail}{\documentauthor}}
\title{\documenttitle}
\date{\today}
\maketitle
\tableofcontents

\renewcommand{\thefootnote}{\fnsymbol{footnote}}
\numberwithin{equation}{section}
\chapter{MA557 Notes (Fall 2015)}
\section{Lecture 1}


%%% Local Variables:
%%% mode: latex
%%% TeX-master: "../../FALL15-Notes"
%%% End:

\include{MA557/lectures/ma557-lecture-2}
\include{MA557/lectures/ma557-lecture-3}
\include{MA557/lectures/ma557-lecture-4}
\include{MA557/lectures/ma557-lecture-5}
\include{MA557/lectures/ma557-lecture-6}
\include{MA557/lectures/ma557-lecture-7}
\include{MA557/lectures/ma557-lecture-8}

\include{MA571/lectures/ma571-lecture-1}
\include{MA571/lectures/ma571-lecture-2}
\include{MA571/lectures/ma571-lecture-3}
\include{MA571/lectures/ma571-lecture-4}
\include{MA571/lectures/ma571-lecture-5}
\include{MA571/lectures/ma571-lecture-6}
\include{MA571/lectures/ma571-lecture-7}
\include{MA571/lectures/ma571-lecture-8}
\chapter{Kaufmann's 571 Problems}
\section{Midterm (Fall 2014)}

\section{Final (Fall 2014)}

%%% Local Variables:
%%% mode: latex
%%% TeX-master: "../../FALL15-Notes"
%%% End:


\chapter{MA692 (Wavelets and Approximation Theory) Notes (Fall 2015)}
\section{Lecture 1}

%%% Local Variables:
%%% mode: latex
%%% TeX-master: "../../FALL15-Notes"
%%% End:

\include{MA692-Wav/lectures/ma692-wav-lecture-2}
\include{MA692-Wav/lectures/ma692-wav-lecture-3}
\include{MA692-Wav/lectures/ma692-wav-lecture-4}
\include{MA692-Wav/lectures/ma692-wav-lecture-5}
\include{MA692-Wav/lectures/ma692-wav-lecture-6}
\include{MA692-Wav/lectures/ma692-wav-lecture-7}
\include{MA692-Wav/lectures/ma692-wav-lecture-8}

\include{MA553/past-quals-compilation}
\end{document}

%%% Local Variables:
%%% mode: latex
%%% TeX-master: t
%%% End:
