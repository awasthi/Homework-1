\def\documentauthor{Carlos Salinas}
\def\documenttitle{PHYS550 Problem Set \hwnum}
\def\hwnum{1}
\def\shorttitle{PHYS550 PSet \hwnum}
\def\coursename{PHYS550}
\def\documentsubject{introduction to quantum mechanics}
\def\authoremail{salinac@purdue.edu}

\documentclass[article,oneside,10pt]{memoir}
\usepackage{geometry}
\usepackage[dvipsnames]{xcolor}
\usepackage[
    breaklinks,
    bookmarks=true,
    colorlinks=true,
    pageanchor=false,
    linkcolor=black,
    anchorcolor=black,
    citecolor=black,
    filecolor=black,
    menucolor=black,
    runcolor=black,
    urlcolor=black,
    hyperindex=false,
    hyperfootnotes=true,
    pdftitle={\shorttitle},
    pdfauthor={\documentauthor},
    pdfkeywords={\documentsubject},
    pdfsubject={\coursename}
    ]{hyperref}

\usepackage{graphicx}
\graphicspath{{figures/}}

\usepackage{multicol}
\usepackage[inline]{enumitem}
\usepackage{listings}
\usepackage{mleftright}
\mleftright

\usepackage{microtype}

\usepackage{amsthm}
\usepackage{amssymb}
\usepackage{mathtools}
\usepackage{lualatex-math}
\usepackage{unicode-math}
% \usepackage{unicode-minionmath}

% \setmainfont[Ligatures=TeX]{Minion Pro}
% \setsansfont[Ligatures=TeX]{Myriad Pro}
% \setmonofont{Courier Std}

% \setmathfont{Minion Math}
% \setmathfont[range={\mathfrak}]{XITS Math}
% \setmathfont[range={\mathcal},StylisticSet=1]{XITS Math}
% \setmathfont[range={\mathscr}]{XITS Math}
% \setmathfont[range={}]{Minion Math}

\setmainfont[Ligatures=TeX]{Latin Modern Roman}
\setsansfont[Ligatures=TeX]{Latin Modern Sans}
\setmonofont{Latin Modern Mono}
\setmathfont{Latin Modern Math}

% \setmainfont[Ligatures=TeX]{CMU Serif}
% \setsansfont[Ligatures=TeX]{CMU Sans Serif}
% \setmonofont{CMU Typewriter Text}

% \setmainfont[Ligatures=TeX]{XITS}
% \setmathfont[range={\mathcal,\mathbfcal},StylisticSet=1]{XITS Math}
% \setmathfont[range=\mathscr]{XITS Math}
% \setmathfont{XITS Math}

\usepackage{polyglossia}
\setmainlanguage[variant=american]{english}
\setotherlanguage{french}
\setotherlanguage[spelling=new,latesthyphen,babelshorthands]{german}
\setotherlanguage[spelling=modern,babelshorthands]{russian}
\setotherlanguage{spanish}

\newfontfamily\cyrillicfont[Ligatures=TeX]{CMU Serif}
\newfontfamily\cyrillicfontsf[Ligatures=TeX]{CMU Sans Serif}
\newfontfamily\cyrillicfonttt{CMU Typewriter Text}

\theoremstyle{plain}
\newtheorem{theorem}{Theorem}
\newtheorem{proposition}[theorem]{Proposition}
\newtheorem{corollary}[theorem]{Corollary}
\newtheorem{claim}[theorem]{Claim}
\newtheorem{lemma}[theorem]{Lemma}
\newtheorem{axiom}[theorem]{Axiom}

\newtheorem*{corollary*}{Corollary}
\newtheorem*{claim*}{Claim}
\newtheorem*{lemma*}{Lemma}
\newtheorem*{proposition*}{Proposition}
\newtheorem*{theorem*}{Theorem}

\theoremstyle{definition}
\newtheorem{definition}{Definition}
\newtheorem{example}{Examples}
\newtheorem{examples}[example]{Examples}
% \newtheorem{exercise}{Exercise}[section]
% \newtheorem{problem}[exercise]{Problem}

\counterwithout{section}{chapter}
\usepackage[explicit]{titlesec}
\titleformat{\section}{\normalfont\Large\bfseries\scshape}{}{0em}{#1}
\newenvironment{problem}[1][]% environment name
{% begin code
  \par\vspace{\baselineskip}\noindent
  \ifx &#1&%
  \section{Problem~\hwnum.\thesection}
  \global\def\exercisename{Problem \hwnum.\thesection}%
  \else
  \section{Problem~\hwnum.\thesection~(#1)}
  \global\def\exercisename{Problem \hwnum.\thesection(#1)}
  \fi
  \par\vspace{\baselineskip}%
  \noindent\ignorespaces
}%
{% end code
  \par\vspace{\baselineskip}%
  \noindent\ignorespacesafterend
}

\newtheorem*{definition*}{Definition}
\newtheorem*{example*}{Examples}
\newtheorem*{examples*}{Examples}
\newtheorem*{exercise*}{Exercise}
\newtheorem*{problem*}{Problem}

\theoremstyle{remark}
\newtheorem{remark}{Remark}
\newtheorem{remarks}[remark]{Remarks}
\newtheorem{observation}[remark]{Observation}
\newtheorem{observations}[remark]{Observations}

\newtheorem*{remark*}{**Remark**}
\newtheorem*{remarks*}{**Remarks**}
\newtheorem*{observation*}{**Observation**}
\newtheorem*{observations*}{**Observations**}

%% Redefinitions & commands
\renewcommand\qedsymbol{\ensuremath{\null\hfill\QED}}

\newcommand\restr[2]{{% we make the whole thing an ordinary symbol
  \left.\kern-\nulldelimiterspace % automatically resize the bar with \right
  {#1} % the function
  % \vphantom{\big|} % pretend it's a little taller at normal size
  \right|_{#2} % this is the delimiter
  }}

%% Commands and operators
\newcommand{\id}{\mathrm{id}}
\newcommand{\im}{\mathrm{im}}
\newcommand{\CC}{\mathbf{C}}
\newcommand{\NN}{\mathbf{N}}
\newcommand{\QQ}{\mathbf{Q}}
\newcommand{\RR}{\mathbf{R}}
\newcommand{\ZZ}{\mathbf{Z}}

\begin{document}
\let\setminus\relax
\let\phi\relax
\let\epsilon\relax
\newcommand\setminus{\smallsetminus}
\newcommand\phi{\varphi}
\newcommand\epsilon{\varepsilon}

\frontmatter
\aliaspagestyle{title}{empty}
\pagestyle{title}
\author{\href{mailto:\authoremail}{\documentauthor}}
\title{\documenttitle}
\date{\today}
\maketitle
\cleartooddpage

\makeoddhead{headings}
        {\small{\MakeUppercase{\itshape\documentauthor}}}
        {}
        {\small{\MakeUppercase{\itshape\exercisename}}}
\makeoddfoot{headings}{{\itshape\documenttitle}}
                      {}
                      {\thepage}
\makeevenhead{headings}
        {\small{\MakeUppercase{\itshape\documentauthor}}}
        {}
        {\small{\MakeUppercase{\itshape\exercisename}}}
\makeevenfoot{headings}{{\itshape\documenttitle}}
                      {}
                      {\thepage}
\makeheadrule{headings}{\textwidth}{.25pt}
% \makerunningwidth{headings}{1.15\textwidth}
\pagestyle{headings}

\mainmatter
\begin{problem}
A particle's wave function is
\begin{equation}
\label{eq:1}
\psi(x)=C\exp\left[\frac{ip_0x}{\hbar}-\frac{(x-x_0)^2}{2a^2}\right]
\end{equation}
(all parameters are real). Find average values for the coordinate
$x$, the momentum $p$ and their fluctuations. Find WF of
Eq.\,(\ref{eq:1}) in momentum representation.
\end{problem}
\begin{proof}[Solution]

\end{proof}
\newpage

\begin{problem}
For a given WF $\Psi(x,y,z)$ find a probability for a particle to
be in the range $z_1<z<z_2$ and $p_1<p_y<p_2$.
\end{problem}
\begin{proof}[Solution]
\end{proof}
\newpage

\begin{problem}
Find eigenvalues and eigenfunctions of a quantity $\hat
f=\alpha\hat p_x+\beta\hat x$, where $\hat p_x$ and $\hat x$ are
the momentum and the coordinate operators.
\end{problem}
\begin{proof}[Solution]
\end{proof}
\newpage

\begin{problem}
Using dimensional analysis, estimate Bohr's orbit for the
motion of electron around proton. Disregard numerical factors
of the order of a few.
\\\\
(\emph{Hints:} The proton is heavy -- not moving, its mass is not
important. Assume the velocity is $\ll$ speed of light -- $c$ is
not important. Only the electron mass, elementary charge and
Planck's constant are the important quantities. Also, for a bound
orbit potential electric energy is of the order of the kinetic
energy.) Clearly explain your steps -- do not just write down the
answer.
\end{problem}
\begin{proof}[Solution]
\end{proof}

%%% Local Variables:
%%% mode: latex
%%% TeX-master: "../PHYS550-HW-Current"
%%% End:

\end{document}

%%% Local Variables:
%%% mode: latex
%%% TeX-master: t
%%% End:
