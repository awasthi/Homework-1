\begin{problem}
A particle's wave function is
\begin{equation}
\label{eq:1}
\psi(x)=C\exp\left[\frac{ip_0x}{\hbar}-\frac{(x-x_0)^2}{2a^2}\right]
\end{equation}
(all parameters are real). Find average values for the coordinate
$x$, the momentum $p$ and their fluctuations. Find WF of
Eq.\,(\ref{eq:1}) in momentum representation.
\end{problem}
\begin{proof}[Solution]
\end{proof}
\newpage

\begin{problem}
For a given WF $\Psi(x,y,z)$ find a probability for a particle to
be in the range $z_1<z<z_2$ and $p_1<p_y<p_2$.
\end{problem}
\begin{proof}[Solution]
\end{proof}
\newpage

\begin{problem}
Find eigenvalues and eigenfunctions of a quantity $\hat
f=\alpha\hat p_x+\beta\hat x$, where $\hat p_x$ and $\hat x$ are
the momentum and the coordinate operators.
\end{problem}
\begin{proof}[Solution]
\end{proof}
\newpage

\begin{problem}
Using dimensional analysis, estimate Bohr's orbit for the
motion of electron around proton. Disregard numerical factors
of the order of a few.
\\\\
(\emph{Hints:} The proton is heavy -- not moving, its mass is not
important. Assume the velocity is $\ll$ speed of light -- $c$ is
not important. Only the electron mass, elementary charge and
Planck's constant are the important quantities. Also, for a bound
orbit potential electric energy is of the order of the kinetic
energy.) Clearly explain your steps -- do not just write down the
answer.
\end{problem}
\begin{proof}[Solution]
\end{proof}

%%% Local Variables:
%%% mode: latex
%%% TeX-master: "../PHYS550-HW-Current"
%%% End:
