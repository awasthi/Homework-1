\chapter{General Preliminaries}
\section{Notation and terminology}
\subsection{General notation}
\begin{convention}
We will always assume the Zermelo--Frænkel axioms and the axiom
of choice.
\end{convention}
\begin{convention}
We denote by $\chi_A$ the characteristic (or indicator) function
of a subset $A$ in a set $X$, i.e., the function $\chi_A(x)=1$ if
$x\in A$, $\chi_A(x)=0$ otherwise.
\end{convention}
\subsection{Direct and inverse limits of spaces and groups}
Let $I$ be a \emph{directed set}, i.e., a partially ordered set,
where every two elements $i,j$ have an \emph{upper bound}, which
is some $k\in I$ such that $i\leq k$, $j\leq k$. A \emph{directed
  system} of sets (or topological spaces, or groups) indexed by
$I$ is a collection of sets (or topological spaces, or groups)
$\left\{A_i\right\}_{i\in I}$ and maps (or continuous maps, or
group homomorphisms) $\phi_{ij}\colon A_i\to A_j$, for $i\leq j$,
satisfying the following compatibility conditions:
\begin{enumerate}[noitemsep,label=(\arabic*)]
\item $\phi_{ik}=\phi_{jk}\circ\phi_{ij}$ for all $i\leq j\leq
  k$.
\item $f_{ii}=\id$.
\end{enumerate}
An \emph{inverse system} is defined similarly, except
$\phi_{ij}\colon A_j\to A_i$, for $i\leq j$, and, accordingly, in
the first condition we use $\phi_{ij}\circ\phi_{jk}$.

The \emph{direct limit} of the direct system is the set
\[
A=\underrightarrow{\lim} A_i=\left(\coprod_{i\in I} A_i\right)/\sim
\]

%%% Local Variables:
%%% mode: latex
%%% TeX-master: "../Geometric-Group-Theory"
%%% End:
