\begin{problem}
Consider the transmission line described in Jackson's problem
1.7. Suppose the conductors carry charges $\pm Q$.
\begin{enumerate}[noitemsep,label=(\alph*)]
\item Find the electric field at any point of the plane passing
  through the axes of the conductors. (Please neglect the end
  effects, i.e., assume the length of the line to be effectively
  infinite.)
\item Using your result from (a), verify the formula given in the
  problem for the capacitance. Why is the formula listed as
  approximate?
\end{enumerate}
\end{problem}
\begin{proof}[Solution]
\end{proof}
\newpage

\begin{problem}
An uncharged conducting sphere of a radius $a$ is placed in the
electric field produced by some large distant conductors
(basically, the plates of a large parallel capacitor). Without
the sphere, the field was uniform (i.e., independent of the
location) and equal to $\textbf{E}_0=E_0\hat\textbf{z}$. We will
be looking for the field $\textbf{E}(x,y,z)$ in the presence of
the sphere.
\\\\
Look for a solution to the Laplace equation outside the sphere in
the form
\[
\varphi(r,\theta)=-E_0z+\sum_{\ell=1}^\infty
\frac{c_\ell P_\ell(\cos\theta)}{r^{\ell+1}},
\]
where $r,\theta,\varphi$ are spherical coordinates (there is no
dependence on $\varphi$), $z=r\cos\theta$, $P_\ell$ are the Legendre
polynomials, and $c_\ell$ are the expansion coefficients to be
found.
\begin{enumerate}[noitemsep,label=(\alph*)]
\item Use the boundary condition at the surface of the plane to
  find all $c_\ell$.
\item Find the Cartesian components of the electric field
  $\mathbf{E}$.
\item Find the change in the electrostatic energy caused by the
  presence of the sphere.
\end{enumerate}
\end{problem}
\begin{proof}[Solution]
\end{proof}

%%% Local Variables:
%%% mode: latex
%%% TeX-master: "../PHYS630-HW-Current"
%%% End:
