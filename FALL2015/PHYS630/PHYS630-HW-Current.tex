\def\documentauthor{Carlos Salinas}
\def\documenttitle{PHYS630 Problem Set \hwnum}
\def\hwnum{1}
\def\shorttitle{PHYS 630 PSet \hwnum}
\def\coursename{PHYS630}
\def\documentsubject{electricity and magnetism}
\def\authoremail{salinac@purdue.edu}

\documentclass[article,oneside,10pt]{memoir}
\usepackage{geometry}
\usepackage[dvipsnames]{xcolor}
\usepackage[
    breaklinks,
    bookmarks=true,
    colorlinks=true,
    pageanchor=false,
    linkcolor=black,
    anchorcolor=black,
    citecolor=black,
    filecolor=black,
    menucolor=black,
    runcolor=black,
    urlcolor=black,
    hyperindex=false,
    hyperfootnotes=true,
    pdftitle={\shorttitle},
    pdfauthor={\documentauthor},
    pdfkeywords={\documentsubject},
    pdfsubject={\coursename}
    ]{hyperref}

\usepackage{graphicx}
\graphicspath{{figures/}}

\usepackage{multicol}
\usepackage[inline]{enumitem}
\usepackage{listings}
\usepackage{mleftright}
\mleftright

\usepackage{microtype}

\usepackage{amsthm}
\usepackage{amssymb}
\usepackage{mathtools}
\usepackage{lualatex-math}
\usepackage{unicode-math}
% \usepackage{unicode-minionmath}

% \setmainfont[Ligatures=TeX]{Minion Pro}
% \setsansfont[Ligatures=TeX]{Myriad Pro}
% \setmonofont{Courier Std}

% \setmathfont{Minion Math}
% \setmathfont[range={\mathfrak}]{XITS Math}
% \setmathfont[range={\mathcal},StylisticSet=1]{XITS Math}
% \setmathfont[range={\mathscr}]{XITS Math}
% \setmathfont[range={}]{Minion Math}

\setmainfont[Ligatures=TeX]{Latin Modern Roman}
\setsansfont[Ligatures=TeX]{Latin Modern Sans}
\setmonofont{Latin Modern Mono}
\setmathfont{Latin Modern Math}

% \setmainfont[Ligatures=TeX]{CMU Serif}
% \setsansfont[Ligatures=TeX]{CMU Sans Serif}
% \setmonofont{CMU Typewriter Text}

% \setmainfont[Ligatures=TeX]{XITS}
% \setmathfont[range={\mathcal,\mathbfcal},StylisticSet=1]{XITS Math}
% \setmathfont[range=\mathscr]{XITS Math}
% \setmathfont{XITS Math}

\usepackage{polyglossia}
\setmainlanguage[variant=american]{english}
\setotherlanguage{french}
\setotherlanguage[spelling=new,latesthyphen,babelshorthands]{german}
\setotherlanguage[spelling=modern,babelshorthands]{russian}
\setotherlanguage{spanish}

\newfontfamily\cyrillicfont[Ligatures=TeX]{CMU Serif}
\newfontfamily\cyrillicfontsf[Ligatures=TeX]{CMU Sans Serif}
\newfontfamily\cyrillicfonttt{CMU Typewriter Text}

\theoremstyle{plain}
\newtheorem{theorem}{Theorem}
\newtheorem{proposition}[theorem]{Proposition}
\newtheorem{corollary}[theorem]{Corollary}
\newtheorem{claim}[theorem]{Claim}
\newtheorem{lemma}[theorem]{Lemma}
\newtheorem{axiom}[theorem]{Axiom}

\newtheorem*{corollary*}{Corollary}
\newtheorem*{claim*}{Claim}
\newtheorem*{lemma*}{Lemma}
\newtheorem*{proposition*}{Proposition}
\newtheorem*{theorem*}{Theorem}

\theoremstyle{definition}
\newtheorem{definition}{Definition}
\newtheorem{example}{Examples}
\newtheorem{examples}[example]{Examples}
% \newtheorem{exercise}{Exercise}[section]
% \newtheorem{problem}[exercise]{Problem}

\counterwithout{section}{chapter}
\usepackage[explicit]{titlesec}
\titleformat{\section}{\normalfont\Large\bfseries\scshape}{}{0em}{#1}
\newenvironment{problem}[1][]% environment name
{% begin code
  \par\vspace{\baselineskip}\noindent
  \ifx &#1&%
  \section{Problem~\hwnum.\thesection}
  \global\def\exercisename{Problem \hwnum.\thesection}%
  \else
  \section{Problem~\hwnum.\thesection~(#1)}
  \global\def\exercisename{Problem \hwnum.\thesection(#1)}
  \fi
  \par\vspace{\baselineskip}%
  \noindent\ignorespaces
}%
{% end code
  \par\vspace{\baselineskip}%
  \noindent\ignorespacesafterend
}

\newtheorem*{definition*}{Definition}
\newtheorem*{example*}{Examples}
\newtheorem*{examples*}{Examples}
\newtheorem*{exercise*}{Exercise}
\newtheorem*{problem*}{Problem}

\theoremstyle{remark}
\newtheorem{remark}{Remark}
\newtheorem{remarks}[remark]{Remarks}
\newtheorem{observation}[remark]{Observation}
\newtheorem{observations}[remark]{Observations}

\newtheorem*{remark*}{**Remark**}
\newtheorem*{remarks*}{**Remarks**}
\newtheorem*{observation*}{**Observation**}
\newtheorem*{observations*}{**Observations**}

%% Redefinitions & commands
% \renewcommand\qedsymbol{\ensuremath{\null\hfill\mdlgblksquare}}
\renewcommand\qedsymbol{\ensuremath{\null\hfill\QED}}
\newcommand\restr[2]{{% we make the whole thing an ordinary symbol
  \left.\kern-\nulldelimiterspace % automatically resize the bar with \right
  {#1} % the function
  % \vphantom{\big|} % pretend it's a little taller at normal size
  \right|_{#2} % this is the delimiter
  }}

%% Commands and operators
\newcommand{\id}{\mathrm{id}}
\newcommand{\im}{\mathrm{im}}
\newcommand{\CC}{\mathbf{C}}
\newcommand{\NN}{\mathbf{N}}
\newcommand{\QQ}{\mathbf{Q}}
\newcommand{\RR}{\mathbf{R}}
\newcommand{\ZZ}{\mathbf{Z}}

\DeclareMathOperator{\Div}{div}
\DeclareMathOperator{\Grad}{grad}
\DeclareMathOperator{\Curl}{curl}

\begin{document}
\frontmatter
\aliaspagestyle{title}{empty}
\pagestyle{title}
\author{\href{mailto:\authoremail}{\documentauthor}}
\title{\documenttitle}
\date{\today}
\maketitle
\cleartooddpage

\makeoddhead{headings}
        {\small{\MakeUppercase{\itshape\documentauthor}}}
        {}
        {\small{\MakeUppercase{\itshape\exercisename}}}
\makeoddfoot{headings}{{\itshape\documenttitle}}
                      {}
                      {\thepage}
\makeevenhead{headings}
        {\small{\MakeUppercase{\itshape\documentauthor}}}
        {}
        {\small{\MakeUppercase{\itshape\exercisename}}}
\makeevenfoot{headings}{{\itshape\documenttitle}}
                      {}
                      {\thepage}
\makeheadrule{headings}{\textwidth}{.25pt}
% \makerunningwidth{headings}{1.15\textwidth}
\pagestyle{headings}

\mainmatter
\begin{problem}
Consider the transmission line described in Jackson's problem
1.7. Suppose the conductors carry charges $\pm Q$.
 \begin{enumerate}[noitemsep,label=(\alph*)]
\item Find the electric field at any point of the plane passing
  through the axes of the conductors. (Please neglect the end
  effects, i.e., assume the length of the line to be effectively
  infinite.)
\item Using your result from (a), verify the formula given in the
  problem for the capacitance. Why is the formula listed as
  approximate?
\end{enumerate}
\end{problem}
\begin{proof}[Solution]
(a) From Gauss's law, the electric field due to one conductor is
\begin{equation}
\label{eq:electric-field-conductor}
\textbf{E}=\frac{Q}{4\pi\epsilon_0 r}
\end{equation}
where $Q$ is the charge per unit length and $r$ is the
perpendicular distance from the point of interest, say $P$, to
the conductor. Along the perpendicular line joining the two
conductors, the field due to to the two conductors are in the
same direction. Therefore, the total field along this line is
\begin{equation}
\label{eq:total-electric-field}
\mathbf{E}=\frac{Q}{2\pi\epsilon_0
  r_1}+\frac{Q}{2\pi\epsilon_0 r_2}
\end{equation}
where $r_1$ and $r_2$ are the perpendicular distances from the
point $P$ to each conductor. The electric field $\mathbf{E}$
points from $+Q$ to $-Q$ and the potential difference is given by
the expression
\begin{align*}
V&=\Phi_1+\Phi_2\\
 &=-\int\mathbf{E}\cdot\diff\mathbf{l}\\
 &=\frac{Q}{2\pi\epsilon_0}
   \left(\int_{a_1}^{d-a_2}\frac{1}{r_1}\diff
   r_1+\int_{a_2}^{a_1-d}\frac{1}{r_2}\diff r_2\right)\\
 &=\frac{Q}{2\pi\epsilon_0}\log\left((d-a_1)(d-a_2)/a_1a_2\right).
\end{align*}
Thus the capacitance is
\begin{align*}
C&=Q/V\\
 &=\frac{2\pi\epsilon_0}{\log\left((d-a_1)(d-a_2)/a_1a_2\right)}\\
 &\approx
 \frac{2\pi\epsilon_0}{\log\left(d^2/\sqrt{a_1a_2}^2\right)}\\
 &=\frac{\pi\epsilon_0}{\log(d/a)}.
\end{align*}
\\\\
(b)
\end{proof}
\newpage

\begin{problem}
An uncharged conducting sphere of a radius $a$ is placed in the
electric field produced by some large distant conductors
(basically, the plates of a large parallel capacitor). Without
the sphere, the field was uniform (i.e., independent of the
location) and equal to $\textbf{E}_0=E_0\hat\textbf{z}$. We will
be looking for the field $\textbf{E}(x,y,z)$ in the presence of
the sphere.
\\\\
Look for a solution to the Laplace equation outside the sphere in
the form
\[
\phi(r,\theta)=-E_0z+\sum_{l=1}^\infty
\frac{c_\ell P_\ell(\cos\theta)}{r^{\ell+1}},
\]
where $r,\theta,\phi$ are spherical coordinates (there is no
dependence on $\phi$), $z=r\cos\theta$, $P_\ell$ are the Legendre
polynomials, and $c_\ell$ are the expansion coefficients to be
found.
\begin{enumerate}[noitemsep,label=(\alph*)]
\item Use the boundary condition at the surface of the plane to
  find all $c_\ell$.
\item Find the Cartesian components of the electric field
  $\mathbf{E}$.
\item Find the change in the electrostatic energy caused by the
  presence of the sphere.
\end{enumerate}
\end{problem}
\begin{proof}[Solution]
(a)
\\\\
(b)
\\\\
(c)
\end{proof}

%%% Local Variables:
%%% mode: latex
%%% TeX-master: "../PHYS630-HW-Current"
%%% End:

\end{document}

%%% Local Variables:
%%% mode: latex
%%% TeX-master: t
%%% End:
