\begin{problem}
Let $R$ be a Noetherian ring, $R\subset S$ an extension of rings,
and $x\in S$. Show that $x$ is integral over $R$ if and only if
for every minimal prime $\mathfrak{q}$ of $S$, the image of $x$
in $S/\mathfrak{q}$ is integral over $R/\mathfrak{q}\cap R$.
\end{problem}
\begin{proof}
$\implies$ Suppose that $x$ is integral over $R$. Then $x$ satisfies a
monic polynomial of degree $n$, say $f(X)=X^n+a_1X^{n-1}+\cdots+a_n$. Let
$\mathfrak{q}$ be a minimal prime of $S$ and consider the quotient ring
$S/\mathfrak{q}$. If $x\in\mathfrak{q}$ there is nothing to show as $\bar
x=\bar 0$ hence satisfies the polynomial $X$ over $R/\mathfrak{q}\cap
S$. Suppose $x\notin\mathfrak{q}$. Then
\[
\bar 0=\overline{x^n+a_1x^{n-1}+\cdots+a_n}
=\bar x^n+\bar a_1\bar x^{n-1}+\cdots+\bar a_n
\]
so $\bar x$ satisfies the polynomial $\bar f(X)$. Hence, $\bar x$ is
integral over $R/\mathfrak{q}\cap S$.
\\\\
$\impliedby$ Conversely, suppose that for $x\in S$ the image of $x$ in
$S/\mathfrak{q}$ is integral over $R/\mathfrak{q}\cap S$. Then we shall
show that $x$ is integral over $R$. For this, it suffices to show that
$R[x]$ is a finite $R$-module.

Since I've not been successful at showing my assertion let us make an extra
assumption on $S$. In particular, we shall assume that $S$ is
Noetherian. Since $S$ is Noetherian, $S$ contains finitely many minimal
primes $\mathfrak{q}_1,...,\mathfrak{q}_n$. Let $f_i(X)\in R[X]$ be the
minimal polynomial of $x$ in $S/\mathfrak{q}_i$, i.e.,
$f_i(x)\mathfrak{q}_i$. Then
\[
f(x)=f_1(x)\cdots f_n(x)\in\mathfrak{q}_1\cap\cdots\cap\mathfrak{q}_n=\nil S.
\]
Since $\nil S$ is nilpotent, $f(x)^m=0$ for some positive integer
$m$. Thus, $x$ is integral over $R$.
\end{proof}
\newpage
\begin{problem}
Let $d$ be a square-free integer and $R$ the integral closure of
$\ZZ$ in $\QQ(\sqrt{d})$. Show that
\[
R=
\begin{cases}
\ZZ[\sqrt{d}]&\text{if $d\not\equiv 1\mod 4$}\\
\ZZ\bigl[\frac{1+\sqrt{d}}{2}\bigr]&\text{if $d\equiv 1\mod 4$}
\end{cases}.
\]
\end{problem}
\begin{proof}
Courtesy of Dummit \& Foote: Since $d$ satisfies the polynomial $X^2-d$,
respectively $X^2-X+(1-d)/4$ for $d\equiv 2$ or $3\mod 4$, it follows that
$\sqrt{d}$ is integral in $\QQ(\sqrt{d})$ so $\ZZ[\sqrt{d}]$ is contained
in the integral closure of $\ZZ$ in $\QQ(\sqrt{d})$. To see the reverse
containment let $\alpha=a+b\sqrt{d}$ with $a,b\in\QQ$ and suppose that
$\alpha$ is integral. If $b=0$, then $\alpha\in\QQ$ so $a\in\ZZ$. So
suppose $b\neq 0$. Then the minimal polynomial of $\alpha$ is
$X^2-2aX+(a^2-b^2d)$ and $2a,a^2-b^2d\in\ZZ$. Thus,
\[
4(a^2-b^2d)=(2a)^2-(2b)^2d
\]
so $4b^2d\in\ZZ$. Since $d$ is square-free it follows that $2b$ is an
integer, $x^2-y^2d\equiv 0\mod 4$. Since $0$ and $1$ are the only squares
mod $4$ and $d$ is not divisible by $4$, it we claim that (i) $d\equiv 2$
or $3\mod 4$ and $x,y$ are both even, or (2) $d\equiv 1\mod 4$ and $x,y$
are both odd. In the first case, $a,b\in \ZZ$ and
$\alpha\in\ZZ[\sqrt{d}]$. In the latter case, $a+b\sqrt{d}=r+s\sqrt{d}$
where $r=(x-y)/2$ and $s=y$ are both integers, so again
$\alpha\in\ZZ[\sqrt{d}]$.
\end{proof}
\newpage
\begin{problem}
Let $R\subset S$ be an integral extension of rings and $I$ and
$R$-ideal. Show that
\begin{enumerate}[label=(\alph*)]
\item $\Ht IS\leq\Ht I$
\item $\Ht IS=\Ht I$ if $S$ is a domain and $R$ is normal.
\end{enumerate}
\end{problem}
\begin{proof}
(a) Let $s=\Ht I$ and let $\mathfrak{q}\supset I$ be a prime ideal in $R$
with height $s$, i.e., there exists a proper chain of ideals
\[
\mathfrak{q}_0\subsetneq\mathfrak{q}_1\subsetneq\cdots\subsetneq\mathfrak{q}_s=\mathfrak{q}.
\]
Then by lying over there exists a prime ideal $\mathfrak{p}_0\subset S$ which
contracts to $\mathfrak{q}_0$ so that by going up we get the chain
\begin{equation}
\label{eq:going-up-chain}
\mathfrak{p}_0\subsetneq\mathfrak{p}_1\subsetneq\cdots\subsetneq\mathfrak{p}_s=\mathfrak{p}
\end{equation}
where $\mathfrak{p}\cap R=\mathfrak{q}$. We claim that
$\Ht\mathfrak{q}=s$. It is clear that $\Ht\mathfrak{q}\geq s$ by
(\ref{eq:going-up-chain}). To see that $\Ht\mathfrak{q}\leq s$ suppose that
we have the refinement
\[
\mathfrak{p}_0'\subsetneq\mathfrak{p}_1'\subsetneq\cdots\subsetneq\mathfrak{p}_r'=\mathfrak{p}.
\]
Write $\mathfrak{q}_i'=(\mathfrak{p}_i')^c$. Then the contracted chain
\[
\mathfrak{q}_0'\subsetneq\mathfrak{q}_1'\subsetneq\cdots\subsetneq\mathfrak{q}_r'=\mathfrak{q}
\]
is a refinement of $\mathfrak{q}$. Hence, $r\leq s$. It follows that $\Ht
p=s$. Thus, $\Ht IS\leq\Ht I$.

(b)
\end{proof}

%%% Local Variables:
%%% mode: latex
%%% TeX-master: "../MA557-HW-Current"
%%% End:
