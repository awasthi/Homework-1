\begin{problem}
Show that $\rad(R[x])=\nil(R[x])$.
\end{problem}
\begin{proof}
Suppose $R$ is a commutative ring with identity and $R[x]$ is the
polynomial ring over $R$ in the indeterminate $x$. Then, it is
clear that $\rad(R[x])\supset\nil(R[x])$ since $\nil(R[x])$ is
the intersection of all prime ideals of $R[x]$ and every maximal
ideal is a prime ideal. To show the reverse containment, we will
first prove the following results found in Dummit and Foote,
\S7.3, p.\,33:
\begin{lemma}
Let $f=a_nx^n+\cdots+a_1x+a_0\in R[x]$. Then
\begin{enumerate}[noitemsep,label=(\alph*)]
\item $f$ is a unit in $R[x]$ if and only if $a_0$ is a unit and
  $a_1,...,a_n$ are nilpotent in $R$;
\item $f$ is nilpotent in $R[x]$ if and only if $a_0,a_1,...,a_n$
  are nilpotent elements of $R$.
\end{enumerate}
\end{lemma}
\begin{proof}[Proof of lemma]
\renewcommand\qedsymbol{$\vardiamondsuit$}
(a) Suppose $f$ is a unit. Then there exists
$g=b_mx^m+\cdots+b_0$ in $R[x]$ such that $fg=1$. In particular,
\begin{align*}
fg-1
&=\left(a_nx^n+\cdots+a_0\right)\left(b_mx^m+\cdots+b_0\right)-1\\
&=\sum_{\substack{k_0+\cdots+k_n=m\\\ell_0+\cdots+\ell_m=m}}
  a_0^{k_0}\cdots a_{n}^{k_n}
  b_0^{\ell_0}\cdots b_m^{\ell_m}
  x^{mn+\sum ik_i+\sum j\ell_j}
  -1\\
&=0,
\end{align*}
is true if and only if

(b)
\end{proof}
\end{proof}
\newpage
\begin{problem}
Let $I$ and $J$ be $R$-ideals. Show that
\[\sqrt{IJ}=\sqrt{I\cap J}=\sqrt{I}\cap\sqrt{J}.\]
\end{problem}
\begin{proof}

\end{proof}
\newpage
\begin{problem}
Let $S$ be a subset of a ring $R$. Show that the following are
equivalent:
\begin{enumerate}[noitemsep,label=(\roman*)]
\item $R\setminus S$ is a union of prime ideals.
\item $1\in S$, and for any elements $x$, $y$ of $R$, $x\in S$
  and $y\in S$ if and only if $xy\in S$.
\end{enumerate}
\end{problem}
\begin{proof}

\end{proof}
\newpage
\begin{problem}
Show that the set of all zero divisors in a ring is a union of
prime ideals.
\end{problem}
\begin{proof}
\end{proof}
\newpage
\begin{problem}
Let $\phi\colon R\to S$ be a surjective homomorphism of
rings.
\begin{enumerate}[noitemsep,label=(\alph*)]
\item Show that $\phi(\rad(R))\subset\rad(S)$, but that
  equality does not hold in general.
\item Show that $\phi(\rad(R))=\rad(S)$ if $R$ is semilocal.
\end{enumerate}
\end{problem}
\begin{proof}
\end{proof}
\newpage
\begin{problem}
An element $e\in R$is called \emph{idempotent} if $e^2=e$. Show
that in a local ring, $0$ and $1$ are the only idempotents.
\end{problem}
\begin{proof}
\end{proof}
\newpage
\begin{problem}
Let $I$ be an $R$-ideal. Show that $I$ is finitely generated and
$I^2=I$ if and only if $I=Re$ with $e$ idempotent.
\end{problem}
\begin{proof}
\end{proof}

%%% Local Variables:
%%% mode: latex
%%% TeX-master: "../MA557-HW-Current"
%%% End:
