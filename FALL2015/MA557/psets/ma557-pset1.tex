\begin{problem}
Show that $\rad(R[X])=\nil(R[X])$.
\end{problem}
\begin{proof}
We will first prove the following results (which can be found in
Dummit and Foote, \S7.3, p.\,33):
\begin{lemma}
Let $f=a_nX^n+\cdots+a_0\in R[X]$. Then
\begin{enumerate}[noitemsep,label=(\alph*)]
\item $f$ is nilpotent in $R[X]$ if and only if $a_0,a_1,...,a_n$
  are nilpotent elements of $R$;
\item $f$ is a unit in $R[X]$ if and only if $a_0$ is a unit and
  $a_1,...,a_n$ are nilpotent in $R$.
\end{enumerate}
\end{lemma}
\begin{proof}[Proof of lemma]
\renewcommand\qedsymbol{$\vardiamondsuit$}
(a) $\impliedby$: Suppose that $a_0,...,a_n$ are nilpotent. Then
$a_0,...,a_n\in\nil(R)$, hence
$f\in\nil(R)\subset\nil(R[X])$. $\implies$: Conversely, if
$f^k=0$ for some positive integer $k$, then $(a_nX^n)^k=0$, so
$a_nx^n\in\nil(R[X])$ so $f-a_nX^n\in\nil(R[X])$, in particular
$a_n\in\nil(R[X])$. By induction on $n$,
$a_0,...,a_n\in\nil(R[X])$.
\\\\
(b) $\impliedby$: Suppose $a_0$ is unit and $a_1,...,a_n$ are
nilpotent. Then, by (a), $f-a_0=a_nX^n+\cdots+a_1X$ is nilpotent
so $f-a_0\in\rad(R[X])$. By Proposition 1.13, $f$ is a
unit. $\implies$: On the other hand, if $f$ is a unit, there
exist $g=b_mX^m+\cdots+b_0$ in $R[X]$ with $fg=1$. Now, let
$\mathfrak{p}$ be an arbitrary prime ideal. Since $f$ is a unit
in $R[X]$, $\bar f=\bar a_nX^n+\cdots+\bar a_0$ is a unit in
$R[X]/\mathfrak{p}$. But since $R[X]/\mathfrak{p}$ is an integral
domain and $\bar f$ is a unit, $\deg\bar f=0$ so $\bar a_i=0$ for
every $i\in\{1,...,n\}$. Since $\mathfrak{p}$ was chosen
arbitrarily,
\end{proof}
By definition $\rad(R)$ is the intersection of every maximal
(hence prime) ideal of $R$ so, by Theorem 1.12,
$\rad(R)\supset\nil(R)$. To see the reverse containment let
$f=a_nX^n+\cdots+a_0$ be in $\rad(R[X])$. By Proposition 1.13,
$1+fg$ is a unit for every $g\in R[X]$. In particular, $1+fX$ is
a unit, so by Lemma 1(b), $a_0,...,a_n$ are nilpotent so
$f\in\nil(R[X])$.
\end{proof}
\newpage
\begin{problem}
Let $I$ and $J$ be $R$-ideals. Show that
\[\sqrt{IJ}=\sqrt{I\cap J}=\sqrt{I}\cap\sqrt{J}.\]
\end{problem}
\begin{proof}
$\sqrt{IJ}=\sqrt{I\cap J}$:~By contradiction, suppose that there
exists some prime ideal $\mathfrak{p}\supset IJ$, but
$\mathfrak{p}\nsupset I\cap J$. Then there exists some element
$x\in I\cap J$ with $x\notin\mathfrak{p}$. However, $x^2\in
IJ$. This contradicts the primality of $\mathfrak{p}$. Hence, if
$\mathfrak{p}$ is a prime ideal containing $IJ$, it must also
contain $I\cap J$ so $\sqrt{IJ}=\sqrt{I\cap J}$.
\\\\
$\sqrt{IJ}=\sqrt{I}\cap\sqrt{J}$: Let
$x\in\sqrt{I}\cap\sqrt{J}$. Then $x^n\in I$ for some $n>0$ and
$x^m\in J$ for some $m>0$. Then $x^{n+m}\in IJ$ so
$x\in\sqrt{IJ}$. Hence $\sqrt{IJ}\supset\sqrt{I}\cap\sqrt{J}$. To
see the reverse containment note that, by above, since
$\sqrt{IJ}=\sqrt{I\cap J}$, then $x\in\sqrt{IJ}$ implies $x^n\in
J$ an $x^n\in J$ for some $n>0$, hence $x\in\sqrt{I}\cap\sqrt{J}$
so $\sqrt{IJ}=\sqrt{I}\cap\sqrt{J}$.
\\\\
By transitivity of ``$=$'', it follows that
$\sqrt{IJ}=\sqrt{I\cap J}=\sqrt{I}\cap\sqrt{J}$.
\end{proof}
\newpage
\begin{problem}
Let $S$ be a subset of a ring $R$. Show that the following are
equivalent:
\begin{enumerate}[noitemsep,label=(\roman*)]
\item $R\setminus S$ is a union of prime ideals.
\item $1\in S$, and for any elements $x$, $y$ of $R$, $x\in S$
  and $y\in S$ if and only if $xy\in S$.
\end{enumerate}
\end{problem}
\begin{proof}
(ii) $\implies$ (i):~Suppose that $S$ is a saturated
multiplicative subset of $R$. Then $S\supset R^\times$ so every
element of $R\setminus S$ is a non-unit. By Corollary 1.5, for
every $x\in R\setminus S$, there exists a maximal ideal
$\mathfrak{m}\supset(x)$. Hence
\[
R\setminus S=\bigcup_{\mathfrak{m}\supset(x)}\mathfrak{m},
\]
in particular $R\setminus S$ is a union of prime ideals.
\\\\
(i) $\implies$ (ii):~Suppose that $R\setminus S$ is a union of
prime ideals. Then, it is clear that $R^\times \subset S$ so
$1\in S$. Now $x,y\in S$ if and only if $x,y\notin R\setminus S$
if and only if $xy\notin\mathfrak{p}$ for some prime ideal
$\mathfrak{p}\subset R\setminus S$. Hence, $S$ is a saturated
multiplicative subset of $R$, i.e., satisfies the conditions
given in (ii).
\end{proof}
\newpage
\begin{problem}
Show that the set of all zero divisors in a ring is a union of
prime ideals.
\end{problem}
\begin{proof}
By Problem 1.3, it suffices to show that the complement of the
set of all zero-divisors, call it $Z$, of a ring $R$ is a
saturated multiplicative subset. It is clear that $R\setminus
Z\supset R^\times$ (since, if $u\in R^\times$, $ub=0$ if and only
if $b=0$: $\implies$ is easily seen since $u^{-1}ub=1\cdot b=0$
so $b=0$; the converse is immediate). Now, $xy$ in $R$ is a
zero-divisor if and only if $x$ or $y$ are zero-divisors, hence
(by taking the negation of this statement) $xy\in R\setminus Z$
implies $x,y\in R\setminus Z$. Thus, $R\setminus Z$ is a
saturated multiplicative subset of $R$.
\end{proof}
\newpage
\begin{problem}
Let $\phi\colon R\to S$ be a surjective homomorphism of
rings.
\begin{enumerate}[noitemsep,label=(\alph*)]
\item Show that $\phi(\rad(R))\subset\rad(S)$, but that
  equality does not hold in general.
\item Show that $\phi(\rad(R))=\rad(S)$ if $R$ is semilocal.
\end{enumerate}
\end{problem}
\begin{proof}
(a) The containment $\phi(\rad(R))\subset\rad(S)$ follows easily
from Proposition 1.13: $x\in\rad(R)$ if and only if $1+xy$ is a
unit for every $y\in R$. Then
\begin{align*}
\phi(1+xy)&=\phi(1)+\phi(xy)\\
             &=\phi(1)+\phi(x)\phi(y)\\
             &1+\phi(x)\phi(y).
\end{align*}
Since $\phi$ is surjective, $1+\phi(x)s$ is a unit for every
$s\in S$ so $\phi(x)\in\rad(S)$.

To see that equality does not hold in general, take $R$ to be
$\ZZ$ and $S$ to be $\ZZ/(6)$. Then the canonical projection
$\pi\colon R\to S$ is a surjection. Since $R$ is a domain,
$\rad(R)=0$, but $\rad(S)= 3S\cap 2S\neq
0=\phi(0)=\phi(\rad(R))$.
\\\\
(b) By part (a) we have that $\phi(\rad(R))\subset\rad(S)$ so we
need only show the reverse containment. Now, suppose $R$ is
semilocal with maximal ideals
$\mathfrak{m}_1,...,\mathfrak{m}_n$. Then, by Corollary 1.15,
$\rad(R)=\bigcap_{i=1}^n=\mathfrak{m}_1\cdots\mathfrak{m}_n$. Now,
by the Homeomorphism Theorem, $S\cong R/\ker\phi$ so, by
Proposition 1.2, the maximal ideals of $S$ are in one-one
correspondence with the maximal ideals of $R$ that contain
$\ker\phi$. Assuming $S\neq 0$, $\ker\phi\neq R$ so, by Corollary
1.5, at least one of the maximal ideals
$\mathfrak{m}_i\supset\ker\phi$. Without loss of generality,
assume the first $k$ maximal ideals
$\mathfrak{m}_1,...,\mathfrak{m}_k$ contain $\ker\phi$ and the
last $\mathfrak{m}_{k+1},...,\mathfrak{m}_n$ do not. Since
$\ker\phi\nsubseteq\mathfrak{m}_i$ for $k+1\leq i\leq n$,
$\ker\phi+\mathfrak{m}_i=R$, i.e., $\ker\phi$ and
$\mathfrak{m}_i$ are comaximal, so there exits elements
$y\in\ker\phi$ and $x_i\in\mathfrak{m}_i$ such that
$y+x_i=1$.

 Then,
$y\in\prod_{i=1}^m\phi(\mathfrak{m}_i)$
so
\[
y
=
\sum_i\phi(x_{i1})\cdots\phi(x_{im})
=
\phi\left(\sum_i x_{i1}\cdots x_{im}\right)
\]
for $x_{ij}\in\phi(\mathfrak{m}_j)$. Now, for $i>n$,
$\mathfrak{m}_i$ and $\ker\phi$ are comaximal so $x+x_0=1$ for
some $x\in\mathfrak{m}_i$, $x_0\in\ker\phi$.
\end{proof}
\newpage
\begin{problem}
An element $e\in R$is called \emph{idempotent} if $e^2=e$. Show
that in a local ring, $0$ and $1$ are the only idempotents.
\end{problem}
\begin{proof}
Suppose $R$ is a local ring with maximal ideal
$\mathfrak{m}$. Suppose, by contradiction, that there exists some
$e\in R$, $e\neq 0$ or $1$, with $e^2=e$. Then $e^2-e=e(e-1)=0$
so $e$ and $e-1$ are zero-divisors, in particular, $e$ and $e-1$
are non-units and hence contained in the maximal ideal
$\mathfrak{m}$. But then $e-(e-1)=1\in\mathfrak{m}$. This
contradicts the maximality of $\mathfrak{m}$.
\end{proof}
\newpage
\begin{problem}
Let $I$ be an $R$-ideal. Show that $I$ is finitely generated and
$I^2=I$ if and only if $I=Re$ with $e$ idempotent.
\end{problem}
\begin{proof}
By Nakayama's lemma (Theorem 2.2), if can view $I$ as a finitely
generated $I$-module, then $I=I^2$ if and only if $aI=0$ for some
$a\in 1+I$. Then, for any element $b\in I$
\[
(1-a)b=b-ba=b.
\]
This implies that any element of $I$ is of the form $(1-a)r$ for
some $r\in R$. This gives that $I\subset (1-a)$. Note that
$(1-a)$ is idempotent since
\[
(1-a)(1-a)=1-a.
\]
Thus $I=R(1-a)$.
\end{proof}

%%% Local Variables:
%%% mode: latex
%%% TeX-master: "../MA557-HW-Current"
%%% End:
