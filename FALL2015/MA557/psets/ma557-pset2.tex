\begin{problem}
Let $\mathfrak{a}$ be an $R$-ideal and $M$ a finite
$R$-module. Show that
\[
\sqrt{\ann(M/\mathfrak{a}M)}=\sqrt{\ann(M)+\mathfrak{a}}.
\]
\end{problem}
\begin{proof}
One inclusion is immediate, namely,
\[
\sqrt{\ann(M/\mathfrak{a}M)}\subset\sqrt{\ann(M)+\mathfrak{a}}
\]
since $x\in\sqrt{\ann(M/\mathfrak{a}M)}$ if
$x^n\in\ann(M/\mathfrak{a}M)$ if $x^nM\subset\mathfrak{a}M$,
i.e., $x^nm=\sum y_im_i$ for $y_i\in\mathfrak{a}$, $m_i\in
M$. But $x'\in\sqrt{\ann(M)+\mathfrak{a}}$ if $x'^n=n+y$ for
$n\in\ann(M)$, $y\in\mathfrak{a}$, or $x'^nm=(n+y)m=nm+ym=ym$, in
particular, $x'^nm\in\mathfrak{a}M$ so
$x'\in\sqrt{\ann(M/\mathfrak{a}M)}$. To see the reverse inclusion
note that [cf.\,Matsumura, Theorem 2.1] if
$x^n\in\ann(M/\mathfrak{a}M)$ then there exists a
$y\in\mathfrak{a}$ such that $(x^n+y)M=0$ or
$x^nM=-yM\subset\mathfrak{a}M$ so
$x\in\sqrt{\ann(M/\mathfrak{a}M)}$. Thus,
$\sqrt{\ann(M)+\mathfrak{a}}\subset\sqrt{\ann(M/\mathfrak{a}M}$
and we have equality.
\end{proof}
\newpage
\begin{problem}
Let $R$ be a local ring and $M,N$ finite $R$-modules. Show that
$M\otimes_R N=0$ if and only if $M=0$ or $N=0$.
\end{problem}
\begin{proof}
$\impliedby$: If $M=0$ or $N=0$, it is immediate that $M\otimes_R
N=0$.

$\implies$: Let $\mathfrak{m}$ be a maximal ideal of $R$. Since
$M\otimes_R N=0$, by Theorem 2.7, we have

% Let m b e the maximal ideal of A. Since M ⊗ A N = 0 we have 0 =
% (M ⊗ A N ) ⊗ A (A/m. However, this module is isomorphic to (M ⊗ A
% A/m) ⊗ A/m (N ⊗ A A/m).  Thus, this module is 0. Note that the
% factors are isomorphic to M/mM and N/mN by a previous homework
% problem. Also, it is a tensor product of vector spaces over the
% field A/m, and each factor is finitely generated, hence
% finite-dimensional. By the dimension formula dim(V ⊗ F W ) =
% dim(V ) dim(W ) for finite-dimensional vector spaces over a field
% F , we get 0 = dim(M/mM ) dim(N/mN ), which forces M/mM = 0 or
% N/mN = 0.  By Nakayama, we get M = 0 or N = 0.
\end{proof}
\newpage
\begin{problem}
Show that $R^n\cong R^m$ if and only if $n=m$.
\end{problem}
\begin{proof}
\end{proof}
\newpage
\begin{problem}
Prove 2.7.
\end{problem}
\begin{proof}
Recall the statement of Theorem 2.7:
\begin{theorem*}
\begin{enumerate}[noitemsep,label=(\alph*)]
\item $M\otimes_R N\cong N\otimes_R M$ via $x\otimes y\mapsto
  y\otimes x$.
\item $(M\otimes_R N)\otimes_R P\cong M\otimes_R N\otimes_R
  P\cong M\otimes_R(N\otimes_R P)$ via $(x\otimes y)\otimes
  z\mapsto x\otimes y\otimes z\mapsto x\otimes (y\otimes z)$.
\item $(M\oplus N)\otimes_R P\cong (M\otimes_R P)\oplus
  (N\otimes_R P)$ via $(x+y)\otimes z\mapsto x\otimes z+y\otimes
  z$.
\item $R\otimes_R M\cong M$ via $r\otimes x\mapsto rx$.
\end{enumerate}
\end{theorem*}
\end{proof}
\newpage
\begin{problem}
Prove 2.8.
\end{problem}
\begin{proof}
Recall the statement of Proposition 2.8:
\begin{proposition*}
Let $M$ be an $R$-module, $N$ an $R$-$S$-bimodule and $P$ an
$S$-module. Then:
\begin{enumerate}[noitemsep,label=(\alph*)]
\item $M\otimes_R N$ is an $R$-$S$-bimodule via $\left(\sum
    m_i\otimes n_i\right)s=\sum m_i\otimes (sn_i).$
\item The free module $(M\otimes_R N)\otimes_S P\cong M\otimes_R
  (N\otimes_S P)$ as $R$-$S$-bimodules via $(x\otimes y)\otimes
  z\mapsto x\otimes (y\otimes z)$.
\end{enumerate}
\end{proposition*}
\end{proof}
\newpage
\begin{problem}
Prove 2.9.
\end{problem}
\begin{proof}
Recall the statement of Theorem 2.9:
\begin{theorem*}
Let $\psi\colon R\to S$ be a ring map and $M$ and $R$
module. Then $S\otimes_R M$ is an $S$-module (by Proposition
2.8) and $\mu\colon M\to S\otimes_R M$ with $\mu(m)=1\otimes m$
is an $R$-linear map. Moreover, for every $R$-linear map
$\phi\colon M\to N$, where $N$ is any $S$-module, there exists a
unique $S$-linear map $f$ so that $\phi=f\circ\mu$, i.e, the
diagram commmutes
\end{theorem*}
\end{proof}
\newpage
\begin{problem}
Prove 2.10.
\end{problem}
\begin{proof}
Recall the statement of Proposition 2.10:
\begin{proposition*}
Let $S$ and $T$ be $R$-algebras. Then there is an $R$-algebra
structure on $S\otimes_R T$ with $(s_1\otimes t_1)(s_2\otimes
t_2)=(s_1s_2)\otimes (t_1t_2)$.
\end{proposition*}
\end{proof}

%%% Local Variables:
%%% mode: latex
%%% TeX-master: "../MA557-HW-Current"
%%% End:
