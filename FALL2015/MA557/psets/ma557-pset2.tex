\begin{problem}
Let $\mathfrak{a}$ be an $R$-ideal and $M$ a finite
$R$-module. Show that
\[
\sqrt{\ann(M/\mathfrak{a}M)}=\sqrt{\ann(M)+\mathfrak{a}}.
\]
\end{problem}
\begin{proof}
One inclusion is immediate, namely,
\[
\sqrt{\ann(M/\mathfrak{a}M)}\subset\sqrt{\ann(M)+\mathfrak{a}}
\]
since $x\in\sqrt{\ann(M/\mathfrak{a}M)}$ if
$x^n\in\ann(M/\mathfrak{a}M)$ if $x^nM\subset\mathfrak{a}M$,
i.e., $x^nm=\sum y_im_i$ for $y_i\in\mathfrak{a}$, $m_i\in
M$. But $x'\in\sqrt{\ann(M)+\mathfrak{a}}$ if $x'^n=n+y$ for
$n\in\ann(M)$, $y\in\mathfrak{a}$, or $x'^nm=(n+y)m=nm+ym=ym$, in
particular, $x'^nm\in\mathfrak{a}M$ so
$x'\in\sqrt{\ann(M/\mathfrak{a}M)}$. To see the reverse inclusion
note that [cf.Atiyah \& MacDonald, Proposition 2.4 or Matsumura,
Theorem 2.1] if $x^n\in\ann(M/\mathfrak{a}M)$ then there exists a
$y\in\mathfrak{a}$ such that $(x^n+y)M=0$ or
$x^nM=-yM\subset\mathfrak{a}M$ so
$x\in\sqrt{\ann(M/\mathfrak{a}M)}$. Thus,
$\sqrt{\ann(M)+\mathfrak{a}}\subset\sqrt{\ann(M/\mathfrak{a}M}$
and we have equality.
\end{proof}
\newpage
\begin{problem}
Let $R$ be a local ring and $M,N$ finite $R$-modules. Show that
$M\otimes N=0$ if and only if $M=0$ or $N=0$.
\end{problem}
\begin{proof}
$\impliedby$ If either $M=0$ or $N=0$ it is immediate that
$M\otimes N=0$.

$\implies$ To see the forward direction, we take Atiyah and
MacDonald's hint and let $\mathfrak{m}$ be the maximal ideal of
$R$ and let $k=R/\mathfrak{m}$ denote its residue field. Let
$M_k=k\otimes M\cong M/\mathfrak{m}M$ by Theorem 2.13. But
$M\otimes N=0$ implies $M_k\otimes_k N_k=0$ as vector spaces so
$M_k=0$ or $N_k=0$. Thus, by Nakayama's lemma, $M=0$ or $N=0$
since $M_k=0$ or $N_k=0$, in other words, since $\mathfrak{m}M=M$
or $\mathfrak{m}N=N$ and $\mathfrak{m}=\rad(R)$.
\end{proof}
\newpage
\begin{problem}
Show that $R^n\cong R^m$ if and only if $n=m$.
\end{problem}
\begin{proof}
$\impliedby$ If $n=m$ then the isomorphism $R^n\cong R^m$ is
canonical.

$\implies$ Suppose that $R^n\cong R^m$. Let $\varphi\colon R^n\to
R^m$ be a $R$-linear isomorphism. Let $\mathfrak{m}$ be a maximal
ideal of $R$ and let $k=R/\mathfrak{m}$ be its residue
field. Then there is an induced $k$-linear isomorphism
$\varphi^*\colon k^n\to k^m$. By the Rank-Nullity theorem and
since $\varphi^*$ is a bijection, we have that the dimension of
$k^n$ and $k^m$ are equal, i.e., $n=m$.
\end{proof}
\newpage
\begin{problem}
Prove 2.7.
\end{problem}
\begin{proof}
Recall the statement of Theorem 2.7:
\begin{theorem*}
\begin{enumerate}[noitemsep,label=(\alph*)]
\item $M\otimes N\cong N\otimes M$ via $x\otimes y\mapsto
  y\otimes x$.
\item $(M\otimes N)\otimes P\cong M\otimes N\otimes
  P\cong M\otimes(N\otimes P)$ via $(x\otimes y)\otimes
  z\mapsto x\otimes y\otimes z\mapsto x\otimes (y\otimes z)$.
\item $(M\oplus N)\otimes P\cong (M\otimes P)\oplus
  (N\otimes P)$ via $(x+y)\otimes z\mapsto x\otimes z+y\otimes
  z$.
\item $R\otimes M\cong M$ via $r\otimes x\mapsto rx$.
\end{enumerate}
\end{theorem*}
\vspace{.25in}
\noindent
In all cases we must show that the prescribed mapping is well
defined.
\\\\
(a) The map $\phi\colon M\times N\to N\times M$ given by
$(x,y)\mapsto(y,x)$ is a homomorphism. Therefore $\phi$ induces a
homomorphism $R^{\bigoplus (M\times
  N)}\overset{\cong}{\longrightarrow}R^{\bigoplus(N\times M)}$
which preserves the elements in the tensor quotient. Thus, we
conclude that the map $x\otimes y\to y\otimes x$ is well
defined.
\\\\
(b) Fix an element $z\in P$. Then the mapping $(x,y)\mapsto
x\otimes y\otimes z$ is bilinear in $x$ and $y$ and so induces a
homomorphism $\phi_z\colon M\otimes N\to M\otimes P$ such
that $\phi_z(x\otimes y)=x\otimes y\otimes z$. Next we consider
the mapping $(t,z)\mapsto\phi_z(t)$ of $(M\otimes N)\times P\to
M\otimes N\otimes P$. This is bilinear and therefore induces a
homomorphism $\phi\colon (M\otimes N)\otimes P\to M\otimes
N\otimes P$ such that $\phi((x\otimes y)\otimes z)=x\otimes
y\otimes z$.

Similarly we can construct a map $\psi_0\colon M\times N\times
P\mapsto (M\otimes N)\otimes P$ which sends $(x,y,z)\mapsto
(x\otimes y)\otimes z$. This is a trilinear map and so it induces
a homomorphism $\psi\colon M\otimes N\otimes P\to(M\otimes
N)\otimes P$ such that $\psi(x\otimes y\otimes z)=(x\otimes
y)\otimes z$. It is clear that $\phi$ and $\psi$ are inverses of
each other. Therefore, we have that $(M\otimes N)\otimes P\cong
M\otimes N\otimes P$.

The proof that $M\otimes (N\otimes P)\cong M\otimes N\otimes P$
is analogous, fixing an element $x\in M$ and repeating the whole
process above.
\\\\
(c) Consider the map $\phi\colon(M\oplus N)\times P\to (M\otimes
P)\oplus (N\otimes P)$ which sends an element $(x+y,z)\mapsto
x\otimes z+y\otimes z$.
\\\\
(d)
\end{proof}
\newpage
\begin{problem}
Prove 2.8.
\end{problem}
\begin{proof}
Recall the statement of Proposition 2.8:
\begin{proposition*}
Let $M$ be an $R$-module, $N$ an $R$-$S$-bimodule and $P$ an
$S$-module. Then:
\begin{enumerate}[noitemsep,label=(\alph*)]
\item $M\otimes N$ is an $R$-$S$-bimodule via $\left(\sum
    m_i\otimes n_i\right)s=\sum m_i\otimes (sn_i).$
\item The free module $(M\otimes N)\otimes_S P\cong M\otimes
  (N\otimes_S P)$ as $R$-$S$-bimodules via $(x\otimes y)\otimes
  z\mapsto x\otimes (y\otimes z)$.
\end{enumerate}
\end{proposition*}
(i)
\\\\
(ii)
\end{proof}
\newpage
\begin{problem}
Prove 2.9.
\end{problem}
\begin{proof}
Recall the statement of Theorem 2.9:
\begin{theorem*}
Let $\psi\colon R\to S$ be a ring map and $M$ and $R$
module. Then $S\otimes M$ is an $S$-module (by Proposition
2.8) and $\mu\colon M\to S\otimes M$ with $\mu(m)=1\otimes m$
is an $R$-linear map. Moreover, for every $R$-linear map
$\phi\colon M\to N$, where $N$ is any $S$-module, there exists a
unique $S$-linear map $f$ so that $\phi=f\circ\mu$, i.e, the
diagram commmutes
\end{theorem*}
\end{proof}
\newpage
\begin{problem}
Prove 2.10.
\end{problem}
\begin{proof}
Recall the statement of Proposition 2.10:
\begin{proposition*}
Let $S$ and $T$ be $R$-algebras. Then there is an $R$-algebra
structure on $S\otimes T$ with $(s_1\otimes t_1)(s_2\otimes
t_2)=(s_1s_2)\otimes (t_1t_2)$.
\end{proposition*}
\end{proof}

%%% Local Variables:
%%% mode: latex
%%% TeX-master: "../MA557-HW-Current"
%%% End:
