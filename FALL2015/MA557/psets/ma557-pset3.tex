\begin{problem}
Let $R$ be a domain and $\Gamma$ the set of all principal ideals
in $R$. Show that $R$ is a unique factorization domain if and
only if $\Gamma$ satisfies the ascending chain condition and
every irreducible element of $R$ is prime.
\end{problem}
\begin{proof}
$\implies$ Suppose that $R$ is a UFD and let
$\mathfrak{a}_1\subset\mathfrak{a}_2\subset\cdots$ be an
ascending chain of ideals in $\Gamma$. Then
$\mathfrak{a}_1=\langle a_1\rangle$ for some $a_i\in R$ since
every ideal belonging to $\Gamma$ is principal. Now, since $R$ is
a UFD, $a_1$ factors uniquely (up to associates) as the finite
product $a_1=p_1\cdots p_k$ of (not necessarily distinct)
irreducible elements $p_1,...,p_k\in R$. If $a_1$ is irreducible
we are done (because in such a case $a_2\mid a_1$ if and only if
$a_2=ua_1$ where $u$ is a unit, hence
$\mathfrak{a}_1=\mathfrak{a}_2=\cdots$). Suppose $a_1$ is not
irreducible. Then since $a_2\mid a_1$, the irreducible factors of
$a_2$ consist of some (or all) of the irreducible factors of
$a_1$ (more precisely we can write $a_2=p_{\sigma(1)}\cdots
p_{\sigma(\ell)}$ for some injection
$\sigma\colon\{1,...,\ell\}\hookrightarrow\{1,...,k\}$ where
$\ell\leq k$). Inductively applying this argument to $a_n$ for
$n\geq 1$, we see that the process (of factoring $a_n$'s from
$a_1$) must terminate for some positive $r$ for otherwise we have
that
\[
a_1
=
a_2b_2=(a_3b_3)b_2
=
\cdots
=
(a_nb_n)b_{n-1}\cdots
b_2
=
\cdots,
\]
but every factorization of $a_1$ into irreducibles must have
length $k$. Thus, the ascending chain
$\mathfrak{a}_1\subset\mathfrak{a}_2\subset\cdots\subset\mathfrak{a}_r=\mathfrak{a}_{r+1}=\cdots$
is stationary for some positive integer $r$ and we say that
$\Gamma$ satisfies the acc.
\\\\
$\impliedby$ Conversely, suppose that $\Gamma$ satisfies the
acc. Let $a_1\in R$. If $a_1$ is irreducible we are done. Suppose
$a_1$ is reducible, then $a_1=a_2b_2$ for some non-units
$a_2,b_{11}\in R$. If both $a_2$ and $b_2$ are irreducible, we
are done. Without loss of generality we may assume $a_2$ is
reducible (as the argument to follow may be applied to the
$b_i$'s in case they are not irreducible). Then $a_2=a_3b_3$ (and
so $a_1=a_2b_2=(a_3b_3)b_2$) for some non-units $a_3,b_3\in
R$. Then we get the ascending chain of principal ideals
\[
\langle a_1\rangle\subset\langle a_2\rangle\subset\langle a_3\rangle\subset\cdots
\]
which must stabilize for some positive integer $r$ since $\Gamma$
satisfies the acc. This argument shows that there exits a
factorization of $a_1$ into irreducibles. We must now prove that
this factorization in unique (up to associates).

Suppose $a=p_1\cdots p_k=q_1\cdots q_\ell$ where
$p_1,...,p_k,q_1,...,q_\ell\in R$ are irreducibles.
\end{proof}
\newpage
\begin{problem}
Let $M$ be an Artinian $R$-module. Show that every injective
$R$-linear map $\phi\colon M\to M$ is an isomorphism.
\end{problem}
\begin{proof}
Suppose, towards a contradiction, that $\phi$ is not
surjective. Then $M\complement\im\phi\neq\emptyset$. Let $x\in
M\complement\im\phi$ and consider the descending chain of
submodules $\im\phi\supset\im\phi^2\supset\cdots$ (the
containment is clear since for any endomorphism $f\colon R\to R$
we have $f(R)\subset R$ so $f^2(R)=f(f(R))\subset f(R)$). By the
dcc on $R$, this chain must stabilize for some positive integer
$n$, i.e., $\im\phi^n=\im\phi^{n+1}=\cdots$. Then
$\phi^n(x)\in\im\phi^n$, but $\phi^n(x)\in\im\phi^{n+1}$ so that
$\phi^n(x)=\phi^{n+1}(y)$ for some $y\in M$. However, since
$\phi$ is injective $\phi^n(x)=\phi^{n+1}(y)=\phi^n(\phi(y))$
implies that $x=\phi(y)$ contrary to our choice of
$x$. Therefore, $\phi$ is an isomorphism.
\end{proof}
\newpage
\begin{problem}
Let $M$ be a finitely generated Artinian module. Show that $M$ is
Noetherian.
\end{problem}
\begin{proof}
We shall induct on $n$ the number of generators of $M$.

Base case: Suppose that $M=Rx$. Then $M$ is cyclic and $M\cong
R/{\ann(x)}$ is an Artinian $R$-module, hence an Artinian
$R/{\ann(x)}$-module (if $N_0\supset N_2\supset\cdots$ is a
descending chain in $M$ as an $R/{\ann(x)}$-module, then it is a
descendig chain in $M$ as an $R$-model, hence is stationary for
nome $n$). Then $M$ is an Artinian ring. By (3.17) $M$ is
Noetherian hence, $M$ is Noetherian as an $R$-module.

Induction step: Assume that every Artinian module with a minimal
generating set of cardinality $\leq n$ is Noetherian. Let
$M=Rx_1+\cdots+Rx_{n+1}$. Then
\[
0
\longrightarrow
Rx_1
\hooklongrightarrow
M=Rx_1+\cdots+Rx_n
\twoheadlongrightarrow
M/Rx_1
\rightarrow
0
\]
is exact. Since $Rx_1$ is Noetherian and $M/Rx_1$ is Noetherian
(since its minimal generating set has cardinality $\leq n$), by
3.5,  $M$ is Noetherian.
\end{proof}
\newpage
\begin{problem}
Let $R$ be a ring that is Artinian or Noetherian, and $x\in
R$. Show that for some $n>0$, the image of $x$ in $R/(\ann x^n)$
is a non-zerodivisor on that ring.
\end{problem}
\begin{proof}
By 3.7 it suffices to show the result for $R$ a Noetherian
module. Consider the ascending chain of submodules
$\ann x\subset\ann x^2\subset\cdots$ (this containment is clear
since if $y\in\ann x^n$ then $yx^n=0$ so $yx^{n+1}=0$). By the
acc on $R$, this chain stabilizes for some positive integer $n$,
i.e., $\ann x^n=\ann x^{n+1}=\cdots$. We claim that $x$ is a
non-zerodivisor in $R/{\ann x^n}$. Suppose $\bar x\bar y=0$ in
$R/{\ann x^n}$. Then $xy\in\ann x^n$ so
$(xy)x^n=x^{n+1}y=0$. Hence, $y\in\ann x^{n+1}$. But $\ann
x^{n+1}=\ann x^n$ so $y\in\ann x^n$, i.e., $\bar y=0$. Thus, $x$
is a non-zerodivisor in $R/{\ann x^n}$.
\end{proof}
\newpage
\begin{problem}
Let $R$ be an Artinian ring. Show that $R\cong
R_1\times\cdots\times R_n$ with $R_i$ Artinian local rings.
\end{problem}
\begin{proof}
Since $R$ is Artinian, it is semilocal (this was shown in the proof
of 3.16, but not given as a proposition). Let us prove this:
\begin{proposition*}
Let $R$ be an Artinian ring. Then $R$ is semilocal.
\end{proposition*}
\begin{proof}[Proof of proposition]
\renewcommand\qedsymbol{$\clubsuit$}
Let $\Gamma$ be the set of all maximal ideals of $R$ (which is
nonempty by 1.4). Suppose, towards a contradiction, that
$|\Gamma|=\infty$ and consider the descending chain of ideals
$\mathfrak{m}_1\supset\mathfrak{m}_1\cap\mathfrak{m}_2\supset\cdots$
where $\mathfrak{m}_i\in\Gamma$. By the dcc on $R$, this chain
stabilizes for some $n$, i.e.,
$\mathfrak{m}_1\cap\cdots\cap\mathfrak{m}_n=\left(\mathfrak{m}_1\cap\cdots\cap\mathfrak{m}_n\right)\cap\mathfrak{m}_{n+1}$. Since
$\mathfrak{m}_1\cdots\mathfrak{m}_n\subset\mathfrak{m}_1\cap\cdots\cap\mathfrak{m}_n$
then
$\mathfrak{m}_1\cdots\mathfrak{m}_n\subset\mathfrak{m}_{n+1}$. Then
since $\mathfrak{m}_{n+1}$ is maximal, it is prime so, by 1.8,
$\mathfrak{m}_i\subset\mathfrak{m}_{n+1}$ for some $1\leq i\leq
n$. This is a contradiction. Thus, $|\Gamma|<\infty$, i.e, $R$ is
semilocal.
\end{proof}
Then $\rad R=\mathfrak{m}_1\cap\cdots\cap\mathfrak{m}_n$. So, by
the Chinese remainder theorem, $\rad
R=\mathfrak{m}_1\cdots\mathfrak{m}_n$. By 3.15, $\rad R$ is
nilpotent so $(\rad
R)^k=\left(\mathfrak{m}_1\cdots\mathfrak{m}_n\right)^k=\mathfrak{m}_1^k\cdots\mathfrak{m}_n^k=0$
for some positive integer $k$. Then the $\mathfrak{m}_i^k$'s are
pairwise comaximal since the $\mathfrak{m}_i$'s are pairwise
comaximal (a consequence of the following lemma)
\begin{lemma*}
Let $\mathfrak{a}$ and $\mathfrak{b}$ be ideals. Then
$\sqrt{\mathfrak{a}^n+\mathfrak{b}^m}\supset\mathfrak{a}+\mathfrak{b}$.
\end{lemma*}
\begin{proof}[Proof of lemma]
\renewcommand\qedsymbol{$\clubsuit$}
Without loss of generality, suppose $n>m$. Let
$x\in\mathfrak{a}+\mathfrak{b}$. Then $x=a+b$ for some
$a\in\mathfrak{a}$, $b\in\mathfrak{b}$. Then by the binomial
expansion theorem
\[
x^{m+n-1}=(a+b)^{m+n-1}=\sum_{i+j=m+n-1}\binom{n+m-1}{j}x^iy^j\in\sqrt{\mathfrak{a}^n+\mathfrak{b}^m}
\]
since (ignoring coefficients) the product $x^iy^j$ is in
$\mathfrak{a}^n+\mathfrak{b}^m$ since if $i\leq n-1$ and $j\leq
m-1$ then $i+j\leq m+n-2$. Thus,
$x\in\sqrt{\mathfrak{a}^n+\mathfrak{b}^m}$ and the containment
$\sqrt{\mathfrak{a}^n+\mathfrak{b}^m}\supset\mathfrak{a}+\mathfrak{b}$
holds.
\end{proof}
It follows immediately from the lemma that
$\rad{\mathfrak{m}_i^k+\mathfrak{m}_j^k}\sup\mathfrak{m}_i+\mathfrak{m}_j=R$
so $\mathfrak{m}_i^k+\mathfrak{m}_j^k=R$. Thus, by the Chinese
remainder theorem, there is an isomorphism $R\cong R/(\rad
R)^k\cong R/\mathfrak{m}_1^k\times\cdots\times
R/\mathfrak{m}_n^k$. We claim that each $R/\mathfrak{m}_i^k$ is
local.

By 1.2, there is a one-one correspondence between the ideals of
$R/\mathfrak{m}_i^k$ and the ideals of $\mathfrak{m}$ of $R$
containing $\mathfrak{m}_i^k$. In particular, if $\mathfrak{m}$
is a maximal ideal containing $\mathfrak{m}_i^k$, then
$\mathfrak{m}_i\subset\mathfrak{m}$ (since $\mathfrak{m}$ is
prime). Thus, $\mathfrak{m}=\mathfrak{m}_i$ by the maximality of
$\mathfrak{m}_i$. Passing to the quotient, we see that
$\mathfrak{m}_i/\mathfrak{m}_i^k$ is the unique maximal ideal of
$R/\mathfrak{m}_i^k$. Hence, $R/\mathfrak{m}_i$ is local.
\end{proof}
\newpage
\begin{problem}
Let $R$ be an Artinian ring all of whose maximal ideals are
principal. Show that every ideal in $R$ is principal.
\end{problem}
\begin{proof}
\end{proof}
\newpage
\begin{problem}
Prove 2.11 (the snake lemma).
\end{problem}
\begin{proof}
\end{proof}

%%% Local Variables:
%%% mode: latex
%%% TeX-master: "../MA557-HW-Current"
%%% TeX-engine: default
%%% End:
