\begin{problem}
Let $R$ be a domain and $\Gamma$ the set of all principal ideals
in $R$. Show that $R$ is a unique factorization domain if and
only if $\Gamma$ satisfies the ascending chain condition and
every irreducible element of $R$ is prime.
\end{problem}
\begin{proof}
$\implies$ Suppose that $R$ is a UFD and let
$\mathfrak{a}_1\subset\mathfrak{a}_2\subset\cdots$ be an
ascending chain of ideals in $\Gamma$. Then
$\mathfrak{a}_1=\langle a_1\rangle$ for some $a_i\in R$ since
every ideal belonging to $\Gamma$ is principal. Now, since $R$ is
a UFD, $a_1$ factors uniquely (up to associates) as the finite
product $a_1=p_1\cdots p_k$ of (not necessarily distinct)
irreducible elements $p_1,...,p_k\in R$. If $a_1$ is irreducible
we are done (because in such a case $a_2\mid a_1$ if and only if
$a_2=ua_1$ where $u$ is a unit, hence
$\mathfrak{a}_1=\mathfrak{a}_2=\cdots$). Suppose $a_1$ is not
irreducible. Then since $a_2\mid a_1$, the irreducible factors of
$a_2$ consist of some (or all) of the irreducible factors of
$a_1$ (more precisely we can write $a_2=p_{\sigma(1)}\cdots
p_{\sigma(\ell)}$ for some injection
$\sigma\colon\{1,...,\ell\}\hookrightarrow\{1,...,k\}$ where
$\ell\leq k$). Inductively applying this argument to $a_n$ for
$n\geq 1$, we see that the process (of factoring $a_n$'s from
$a_1$) must terminate for some positive $r$ for otherwise we have
that
\[
a_1
=
a_2b_2=(a_3b_3)b_2
=
\cdots
=
(a_nb_n)b_{n-1}\cdots
b_2
=
\cdots,
\]
but every factorization of $a_1$ into irreducibles must have
length $k$. Thus, the ascending chain
$\mathfrak{a}_1\subset\mathfrak{a}_2\subset\cdots\subset\mathfrak{a}_r=\mathfrak{a}_{r+1}=\cdots$
is stationary for some positive integer $r$ and we say that
$\Gamma$ satisfies the acc.
\\\\
$\impliedby$ Conversely, suppose that $\Gamma$ satisfies the
acc. Let $a_1\in R$. If $a_1$ is irreducible we are done. Suppose
$a_1$ is reducible, then $a_1=a_2b_2$ for some non-units
$a_2,b_{11}\in R$. If both $a_2$ and $b_2$ are irreducible, we
are done. Without loss of generality we may assume $a_2$ is
reducible (as the argument to follow may be applied to the
$b_i$'s in case they are not irreducible). Then $a_2=a_3b_3$ (and
so $a_1=a_2b_2=(a_3b_3)b_2$) for some non-units $a_3,b_3\in
R$. Then we get the ascending chain of principal ideals
\[
\langle a_1\rangle\subset\langle a_2\rangle\subset\langle a_3\rangle\subset\cdots
\]
which must stabilize for some positive integer $r$ since $\Gamma$
satisfies the acc. This argument shows that there exits a
factorization of $a_1$ into irreducibles. We must now prove that
this factorization in unique (up to associates).

Suppose $a=p_1\cdots p_k=q_1\cdots q_\ell$ where
$p_1,...,p_k,q_1,...,q_\ell\in R$ are irreducibles.
\end{proof}
\newpage
\begin{problem}
Let $M$ be an Artinian $R$-module. Show that every injective
$R$-linear map $\phi\colon M\to M$ is an isomorphism.
\end{problem}
\begin{proof}
Suppose $\phi$ is not surjective. Then, there is some element
$x\in M$ that is not in the image of $\phi$. Now consider
cokernels $\coker(\phi^n)=M/\im(\phi^n)$
\end{proof}
\newpage
\begin{problem}
Let $M$ be a finitely generated Artinian module. Show that $M$ is
Noetherian.
\end{problem}
\begin{proof}
Suppose that $M$ is not Noetherian. Let $\Gamma$ be the set of
all non-finitely generated submodules $N$ of $M$.
\end{proof}
\newpage
\begin{problem}
Let $R$ be a ring that is Artinian or Noetherian, and $x\in
R$. Show that for some $n>0$, the image of $x$ in $R/(0:x)^n$ is
a nonzero-divisor on that ring.
\end{problem}
\begin{proof}
\end{proof}
\newpage
\begin{problem}
Let $R$ be an Artinian ring. Show that $R\cong
R_1\times\cdots\times R_n$ with $R_i$ Artinian local rings.
\end{problem}
\begin{proof}
\end{proof}
\newpage
\begin{problem}
Let $R$ be an Artinian ring all of whose maximal ideals are
principal. Show that every ideal in $R$ is principal.
\end{problem}
\begin{proof}
\end{proof}
\newpage
\begin{problem}
Prove 2.12.
\end{problem}
\begin{proof}
Recall the statement of Theorem 2.12:
\begin{theorem*}
Let $R$ be a ring, $M$, $M'$ and $M''$ be $R$-modules. Then
\begin{enumerate}[noitemsep,label=(\alph*)]
\item The following are equivalent:
\begin{enumerate}[noitemsep,label=(\arabic*)]
\item $\displaystyle 0\xrightarrow{}
  M'\xrightarrow{\phi}M\xrightarrow{\psi}M''$
  is exact
\item $\displaystyle
  0\xrightarrow{}\hom(N,M')\xrightarrow{\hom(N,\phi)}\hom(N,M)\xrightarrow{\hom(N,\psi)}\hom(N,M'')$
  is exact for all modules $N$.
\end{enumerate}
\item The following are equivalent:
\begin{enumerate}[noitemsep,label=(\arabic*)]
\item $\displaystyle
  M'\xrightarrow{\phi}M\xrightarrow{\psi}M''\xrightarrow{}
  0$ is exact.
\item $\displaystyle
  0\xrightarrow{}\hom(M'',N)\xrightarrow{\hom(\psi,N)}\hom(M,N)\xrightarrow{\hom(\psi,N)}\hom(M',N)$
  is exact for all modules $N$.
\item $\displaystyle M'\otimes N\xrightarrow{\phi\otimes
    N}M\otimes N\xrightarrow{\psi\otimes
    N}M''\otimes N\xrightarrow{} 0$ is exact
  for all modules $N$.
\end{enumerate}
\end{enumerate}
\end{theorem*}
\end{proof}

%%% Local Variables:
%%% mode: latex
%%% TeX-master: "../MA557-HW-Current"
%%% End:
