\begin{problem}
Let $R$ be a Noetherian domain. Show that the following are equivalent:
\begin{enumerate}[label=(\roman*)]
\item $R$ is a unique factorization domain
\item every prime ideal of $R$ of height one is principal
\item $R$ is normal with $\Cl(R)=0$.
\end{enumerate}
\end{problem}
\begin{proof}
(i) $\implies$ (ii) Suppose $R$ is a Noetherian domain. Let $\mathfrak{p}$
be a height one prime. Then there exists at least one nonzero element
$x\in\mathfrak{p}$. Let $x=p_1^{\alpha_1}\cdots p_k^{\alpha_k}$ be the
factorization of $x$ into irreducible (prime) elements of $R$. Set
$p\coloneqq p_i$ for any prime in the factorization of $x$. Then the ideal
generated by $p$ is a prime ideal contained in $\mathfrak{p}$, i.e.,
$\langle p \rangle\subset\mathfrak{p}$. But $\Ht(\mathfrak{p})=1$. Thus,
$\langle p \rangle=\mathfrak{p}$.
\\\\
(ii) $\implies$ (ii) Suppose that every height one prime ideal in $R$ is
principal. To show that $R$ is a UFD, it suffices to show that every
irreducible element $p$ is a prime element, that is, $\langle p\rangle$ is
a prime ideal. Let $\mathfrak{p}$ be the minimal prime containing
$p$. Since $\mathfrak{p}$ is principal, $\mathfrak{p}=\langle x\rangle$ for
some $x\in\mathfrak{p}$. Thus, $p=xy$ for some $y\in R$. But $p$ is prime
hence, irreducible so either $x$ or $y$ is a unit. If $x$ is a unit, then
$\mathfrak{p}=R$, which is a contradiction. Thus, $y$ must be a unit and we
see that $\langle p\rangle=\langle xy\rangle=\mathfrak{p}$ is
prime.

Now, for the following implications we need to know a couple of
denfinitions: Let $D(R)$ denote the set of divisional fractional $R$-ideals
and $F(R)$ denote the set of all principal fractional ideals. Then the
\emph{divisor class group of $R$} is the quotient $\Cl(R)\coloneqq
D(R)/F(R)$.
\end{proof}
\newpage
\begin{problem}
Let $R$ be a ring with total ring of quotients $K$, $M$ an $R$-module, and
\[
\mathcal{T}(M)=\left\{\,x\in M\;\middle|\;\text{$ax=0$ for some non zero-divisor $a$
    of $RR$}\,\right\}.
\]
The submodule $\mathcal{T}(M)$ is called the \emph{torsion of $M$}, and $M$ is called
\emph{torsion free} if $\mathcal{T}(M)=0$. Show
\begin{enumerate}[label=(\alph*)]
\item $\mathcal{T}(M)=\ker(M\to K\otimes_R M)$
\item $M/\mathcal{T}(M)$ is torsion free.
\end{enumerate}
\end{problem}
\begin{proof}
\end{proof}
\newpage
\begin{problem}
Let $R$ be a Dedekind domain and $M$ a finitely generated $R$-module of
rank $r$. Show that:
\begin{enumerate}[label=(\alph*)]
\item If $M$ is torsion free then $M$ is projective (hint: induct on $r$).
\item $M\cong \mathcal{T}(M)\oplus P$ with $P$ projective.
\item If $M\neq 0$ is projective then $M\cong R^{r-1}\oplus I$ with $I\neq
  0$ an ideal.
\item If $M$ is torsion (i.e., $M=\mathcal{T}(M)$) then
\[
M\cong R/I_1\oplus\cdots\oplus R/I_n\qquad\text{with}\qquad
I_1\supset\cdots\supset I_n\neq 0
\]
ideals (hint: for $p_1,...,p_s$ the minimal primes of $\ann(M)$ and
$S=R\minus(\mathfrak{p}_1\cup\cdots\cup\mathfrak{p}_s)$, show that
$S^{-1}R$ is a PID).
\end{enumerate}
\end{problem}
\begin{proof}
\end{proof}

%%% Local Variables:
%%% mode: latex
%%% TeX-master: "../MA557-HW-Current"
%%% End:
