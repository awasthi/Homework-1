\begin{problem}
For $I$ an $R$-ideal consider the multiplicatively
closed set $S=1+I$. Show that
\begin{enumerate}[noitemsep,label=(\alph*)]
\item $\tilde S=R\setminus\bigcup\mathfrak{m}$, where the
  union is taken over all
  $\mathfrak{m}\in\MSpec(R)\cap V(I)$.
\item $\MSpec(S^{-1}R)$ and $\MSpec(R/I)$ are homeomorphic.
\end{enumerate}
\end{problem}
\begin{proof}
(a) By 4.19, we have
\[
\tilde
S=R\setminus\bigcup_{\substack{\mathfrak{p}\in\Spec(R)\\\mathfrak{p}\cap
  S=\emptyset}}\mathfrak{p}.
\]
But $\mathfrak{p}\cap S=\mathfrak{p}\cap (1+I)=\emptyset$ if and
only if $\mathfrak{p}+I\neq R$ if and only if there is some
maximal ideal $\mathfrak{m}\supset \mathfrak{p}+I$.

For the former equivalence: $\implies$ Suppose that
$\mathfrak{p}\cap S=\mathfrak{p}\cap(1+I)=\emptyset$, then if
$\mathfrak{p}+I=R$ for some $x\in\mathfrak{p}$, $y\in I$ we have
$x+y=1$. But then $x=1-y\in\mathfrak{p}\cap S$; this is a
contradiction. $\impliedby$ Conversely, if
$\mathfrak{p}\cap S\neq\emptyset$, $x=1+y\in\mathfrak{p}$ for some
$y\in I$ so $x-y=(1+y)-y=1\in\mathfrak{p}+I$ implies
$\mathfrak{p}+I=R$.

For the latter equivalence: $\implies$ Suppose $\mathfrak{p}+I\neq R$,
then $\mathfrak{p}+I$ is a proper ideal of $R$ so, by 1.5, is
contained in a maximal ideal $\mathfrak{m}$. $\impliedby$ Conversely,
if $\mathfrak{m}\subsetneq R$ is a maximal ideal containing
$\mathfrak{p}+I$ then $\mathfrak{p}+I\neq R$ for otherwise
$\mathfrak{m}=R$. Then it suffices to take the union over all maximal
ideals $\mathfrak{m}\supset I$.
\\\\
(b) We will show that $\MSpec(S^{-1}R)\approx\MSpec(R)\cap V(I)$ and
$\MSpec(R/I)\approx\MSpec(R)\cap V(I)$ so that, by the transitivity of
homeomorphism, we have $\MSpec(S^{-1}R)\approx\MSpec(R/I)$. By
4.21(a), $\Spec(R/I)\approx V(I)$ so the restriction
$\MSpec(R/I)\approx\MSpec(R/I)\cap V(I)$. To see that
$\MSpec(S^{-1}R)\approx\MSpec(R)\cap V(I)$, let $\varphi\colon R\to
S^{-1}R$ be the canonical homomorphism sending $x\mapsto x/1$, then
$\phi$ induces a continuous closed map
$^a\phi\colon\Spec(S^{-1}R)\to\Spec(R)$ taking
$\bar{\mathfrak{p}}\mapsto\mathfrak{p}$, i.e., ideal extension. Thus,
by 4.13(d), there is a one-to-one correspondence between
$\bar{\mathfrak{p}}\in\Spec(S^{-1}M)$ and its extension
$\mathfrak{p}\in\Spec(R)$ with $\mathfrak{p}\cap S=\emptyset$ so that
it suffices to show that $^a\phi(\MSpec(S^{-1}R))=\MSpec(R)\cap
V(I)$. But this is easy: If $\bar{\mathfrak{m}}\in\MSpec(S^{-1}R)$
then its contraction is a maximal ideal $\mathfrak{m}\supset I$ by
part (a), hence is in $\MSpec(R)\cap V(I)$. Conversely, if
$\mathfrak{m}\in\MSpec(R)\cap V(I)$, again, by part (a),
$\mathfrak{m}$ is a maximal ideal not meeting $S$ so that by 4.13(d),
there exist some maximal ideal $\bar{\mathfrak{m}}$ contracting to
$\mathfrak{m}$. It follows that $\MSpec(S^{-1}R)\approx\MSpec(R/I)$.
\end{proof}
\newpage
\begin{problem}
Show that the following are equivalent for a ring $R$:
\begin{enumerate}[noitemsep,label=(\alph*)]
\item there exist rings $R_1\neq 0$ and $R_2\neq 0$ so that
  $R\cong R_1\times R_2$;
\item there exist an idempotent $e\in R$ with $e\neq 0$ and
  $e\neq 1$;
\item $\Spec(R)$ is disconnected.
\end{enumerate}
\end{problem}
\begin{proof}
(a) $\iff$ (b) is immediate for suppose $R\cong R_1\times
R_2$ by $\phi\colon R\to R_1\times R_2$. Then, since $\phi$ is a
bijection, there exist an $r\in R$ that maps to the idempotent
element $(1,0)\in R_1\times R_2$.

Conversely, suppose $e\in R$ is idempotent. Then $e'=1-e$ is also
idempotent since
\[
(e')^2=(1-e)^2=1-2e+e^2=1-2e+e=1-e.
\]
Moreover
\[
ee'=e(1-e)=e-e^2=e-e=0.
\]
Let $R_1$ and $R_2$ be the subrings of $R$ generated by $e$ and
$e'$, respectively. Then we claim that $R\cong R_1\times R_2$ via
$\phi(r)=(re,re')$. It is clear that $\phi$ is a ring
homomorphism: take $r_1,r_2\in R$ then
\begin{align*}
\phi(r_1+r_2)&=((r_1+r_2)e,(r_1+r_2)e')&
\phi(r_1r_2)&=(r_1r_2e,r_1r_2e')\\
&=(r_1e+r_2e,r_1e'+r_2e')&
&=\left(r_1r_2e^2,r_1r_2(e')^2\right)\\
&=(r_1e,r_1e')+(r_2e,r_2e')&
&=(r_1e,r_1e')(r_2e,r_2e')\\
&=\phi(r_1)+\phi(r_2)&
&=\phi(r_1)\phi(r_2).
\end{align*}
To prove surjective take $(r,s)\in R_1\times
R_2$ then, $r=r_1e$ and $s=r_2e'$ for $r_1,r_2\in R$ then
\begin{align*}
\phi(r_1e+r_2e')&=\phi(r_1e)+\phi(r_2e')\\
&=(r_1e,r_1ee')+(r_2e'e,r_2ee')\\
&=(r_1e,0)+(0,r_2e')\\
&=(r_1e,r_2e')\\
&=(r,s).
\end{align*}
To prove injectivity take $r\in\ker\phi$. Then
$\phi(r)=(re,re')=(0,0)$. Then $re-re'=r(e-e')=r\cdot 1=0$ so
$r=0$.

(a) $\implies$ (c) Recall that a topological space $X$ is
disconnected if there exist disjoint open sets $A,B$ with
$X=A\cup B$. Suppose $R\cong R_1\times R_2$. Then
$\Spec(R)\approx \Spec(R_1\times R_2)$: Keeping the notation as
before, $\phi$ is a set bijection so it induces a bijection, call
it $\phi^*$, on $\Spec(R)\to\Spec(R_1\times R_2)$ by sending
$\Spec(I)\mapsto\Spec(\phi(I))$; Now let $I\subset R$ be an
ideal, then
\[
\phi^*(V(I))=\phi^*\left(V(eI+e'I)\right)=V(\phi(eI)+\phi(e'I))=V(eI\times e'I)
\]
is closed. Thus, $\phi^*$ is a homeomorphism. Now, we claim that
the sets $A=V(R_1\times 0)$ and $B=V(0\times R_2)$ constitute a
separation of $R$. First note by 4.20(2) that
\[
A\cup B=V(R_1\times 0)\cup V(0\times R_2)=V((R_1\times
0)\cap(0\times R_2))=V(0)=\Spec(R).
\]
Moreover
\[
A\cap B=V(R_1\times 0)\cap V(0\times R_2)=V(R_1\times 0+0\times R_2)=V(R)=\emptyset.
\]

\end{proof}
\newpage
\begin{problem}
A topological space is called \emph{Noetherian} if the set of
closed sets satisfies the dcc. Show that if a ring $R$ is
Noetherian then so is $\Spec(R)$, but that the converse does not
hold.
\end{problem}
\begin{proof}
We will first prove the following useful results:
\begin{lemma*}
Let $R$ be a commutative ring with identity. Then
\begin{enumerate}[noitemsep,label=(\roman*)]
\item $V(I)=V(\sqrt{I})$.
\item $I\subset J$ implies $V(I)\supset V(J)$.
\item $V(I)\supset V(J)$ implies $\sqrt{I}\subset\sqrt{J}$.
\end{enumerate}
\end{lemma*}
\begin{proof}[Proof of lemma]
\renewcommand\qedsymbol{$\clubsuit$}
(i) It is clear that for every prime ideal
$\mathfrak{p}\supset\sqrt{I}$ we have $\mathfrak{p}\supset I$ so
it suffice to prove that if $\mathfrak{p}\supset I$ then
$\mathfrak{p}\supset\sqrt{I}$. But this is clear since if
$x\in\sqrt{I}$ then $x^k\in I$ for some positive integer $k$ so
$x^k\in\mathfrak{p}$ and since $\mathfrak{p}$ is prime
$x\in\mathfrak{p}$. Thus, $V(I)=V(\sqrt{I})$.
\\\\
(ii) Suppose $I\subset J$. Then every prime ideal
$\mathfrak{p}\supset J$ must also contain $I$. Thus, $V(I)\supset
V(J)$.
\\\\
(iii) Suppose $V(I)\supset V(J)$. Then, for every prime ideal
$\mathfrak{p}\supset J$, $\mathfrak{p}\supset I$ so
\[
\sqrt{J}=\bigcap_{\mathfrak{p}\supset J}\mathfrak{p}\supset
\bigcap_{\mathfrak{p}\supset
    J}\mathfrak{p}\cap\bigcap_{\substack{\mathfrak{q}\supset
    I\\\mathfrak{q}\nsupset J}}\mathfrak{q}=\sqrt{I}.\qedhere
\]
\end{proof}
It suffices to reduce to the case of varieties of ideals in
$R$ since varieties generate the Zariski topology on
$\Spec(R)$. Suppose
\[
V(I_1)\supset V(I_2)\supset\cdots
\]
is a descending chain of varieties in $\Spec(R)$. Then, by the (iii)
of the lemma and the nullstellensatz, the latter chain is in
one-to-one correspondence with the ascending chain of radical ideals
\[
\sqrt{I_1}\subset\sqrt{I_2}\subset\cdots
\]
which must stabilize since $R$ is Noetherian. It follows that the
chain $V(I_1)\supset V(I_2)\supset\cdots$ stabilizes so $\Spec(R)$ is
Noetherian.
\end{proof}
\newpage
\begin{problem}
A nonempty closed subset $V$ of a topological space is called
\emph{irreducible} if $V=V_1\cup V_2$, $V_1$ and $V_2$ closed
subset, implies $V_1=V$ or $V_2=V$.
\begin{enumerate}[noitemsep,label=(\alph*)]
\item Show that in a Noetherian topological space every nonempty
  closed subset is a finite union of irreducible closed subsets.
\item Show that $V(\mathfrak{p})$, $\mathfrak{p}\in\Spec(R)$, are
  exactly the irreducible closed subsets of $\Spec(R)$.
\end{enumerate}
\end{problem}
\begin{proof}
(a) Let $X$ be a Noetherian topological space. Let
\[
\Lambda=\left\{ \,V\subset X\;\middle|\;\text{$V$ is closed and not a
    finite union of irreducible closed subsets} \right\}.
\]
Then, by the dcc, $\Lambda$ contains a minimal element, say $W$. Then
$W$ is not irreducible so we can write $W=W_1\cup W_2$ where $W_1\neq
W$ and $W_2\neq W$. By minimality of $W$, $W_1$ and $W_2$ are finite
unions of irreducible closed subsets so $W_1=\bigcup_{i=1}^k
W_1^{(i)}$ and $W_2=\bigcup_{i=1}^\ell W_2^{(i)}$ so
\[
W=W_1\cup W_2=\left( \bigcup_{i=1}^\ell W_1^{(i)} \right)\cup\left(
  \bigcup_{i=1}^k W_2^{(i)} \right)
\]
a contradiction. Thus, every closed subset $V$ can be expressed as the
finite union of irreducible closed subsets.
\\\\
(b)
\end{proof}
\newpage
\begin{problem}
Show that a Noetherian ring has only finitely many minimal prime
ideals.
\end{problem}
\begin{proof}
\end{proof}
\newpage
\begin{problem}
Show that a nonzero ring has at least one minimal prime ideal.
\end{problem}
\begin{proof}
\end{proof}

%%% Local Variables:
%%% mode: latex
%%% TeX-master: "../MA557-HW-Current"
%%% End:
