\begin{problem}
For $I$ an $R$-ideal consider the multiplicatively
closed set $S=1+I$. Show that
\begin{enumerate}[noitemsep,label=(\alph*)]
\item $\widetilde S=R\setminus\bigcup\mathfrak{m}$, where the
  union is taken over all
  $\mathfrak{m}\in\Speck{\mathfrak{m}}(R)\cap V(I)$.
\item $\Speck{\mathfrak{m}}(R/I)$ are homeomorphic.
\end{enumerate}
\end{problem}
\begin{proof}
(a) We will show double inclusion. Recall from 4.18 that
\\\\
(b)
\end{proof}
\newpage
\begin{problem}
Show that the following are equivalent for a ring $R$:
\begin{enumerate}[noitemsep,label=(\alph*)]
\item there exist rings $R_1\neq 0$ and $R_2\neq 0$ so that
  $R\cong R_1\times R_2$;
\item there exist an idempotent $e\in R$ with $e\neq 0$ and
  $e\neq 1$;
\item $\Spec(R)$ is disconnected.
\end{enumerate}
\end{problem}
\begin{proof}
(a) $\iff$ (b) is immediate for suppose $R\cong R_1\times
R_2$ by $\phi\colon R\to R_1\times R_2$. Then, since $\phi$ is a
bijection, there exist an $r\in R$ that maps to the idempotent
element $(1,0)\in R_1\times R_2$.

Conversely, suppose $e\in R$ is idempotent. Then $e'=1-e$ is also
idempotent since
\[
(e')^2=(1-e)^2=1-2e+e^2=1-2e+e=1-e.
\]
Moreover
\[
ee'=e(1-e)=e-e^2=e-e=0.
\]
Let $R_1$ and $R_2$ be the subrings of $R$ generated by $e$ and
$e'$, respectively. Then we claim that $R\cong R_1\times R_2$ via
$\phi(r)=(re,re')$. It is clear that $\phi$ is a ring
homomorphism: take $r_1,r_2\in R$ then
\begin{align*}
\phi(r_1+r_2)&=((r_1+r_2)e,(r_1+r_2)e')&
\phi(r_1r_2)&=(r_1r_2e,r_1r_2e')\\
&=(r_1e+r_2e,r_1e'+r_2e')&
&=\left(r_1r_2e^2,r_1r_2(e')^2\right)\\
&=(r_1e,r_1e')+(r_2e,r_2e')&
&=(r_1e,r_1e')(r_2e,r_2e')\\
&=\phi(r_1)+\phi(r_2)&
&=\phi(r_1)\phi(r_2).
\end{align*}
To prove surjective take $(r,s)\in R_1\times
R_2$ then, $r=r_1e$ and $s=r_2e'$ for $r_1,r_2\in R$ then
\begin{align*}
\phi(r_1e+r_2e')&=\phi(r_1e)+\phi(r_2e')\\
&=(r_1e,r_1ee')+(r_2e'e,r_2ee')\\
&=(r_1e,0)+(0,r_2e')\\
&=(r_1e,r_2e')\\
&=(r,s).
\end{align*}
To prove injectivity take $r\in\ker\phi$. Then
$\phi(r)=(re,re')=(0,0)$. Then $re-re'=r(e-e')=r\cdot 1=0$ so
$r=0$.

(a) $\implies$ (c) Recall that a topological space $X$ is
disconnected if there exist disjoint open sets $A,B$ with
$X=A\cup B$. Suppose $R\cong R_1\times R_2$. Then
$\Spec(R)\approx \Spec(R_1\times R_2)$: Keeping the notation as
before, $\phi$ is a set bijection so it induces a bijection, call
it $\phi^*$, on $\Spec(R)\to\Spec(R_1\times R_2)$ by sending
$\Spec(I)\mapsto\Spec(\phi(I))$; Now let $I\subset R$ be an
ideal, then
\[
\phi^*(V(I))=\phi^*\left(V(eI+e'I)\right)=V(\phi(eI)+\phi(e'I))=V(eI\times e'I)
\]
is closed. Thus, $\phi^*$ is a homeomorphism. Now, we claim that
the sets $A=V(R_1\times 0)$ and $B=V(0\times R_2)$ constitute a
separation of $R$. First note by 4.20(2) that
\[
A\cup B=V(R_1\times 0)\cup V(0\times R_2)=V((R_1\times
0)\cap(0\times R_2))=V(0)=\Spec(R).
\]
Moreover
\[
A\cap B=V(R_1\times 0)\cap V(0\times R_2)=V(R_1\times 0+0\times R_2)=V(R)=\emptyset.
\]

\end{proof}
\newpage
\begin{problem}
A topological space is called \emph{Noetherian} if the set of
closed sets satisfies the dcc. Show that if a ring $R$ is
Noetherian then so is $\Spec(R)$, but that the converse does not
hold.
\end{problem}
\begin{proof}
Suppose $R$ is Noetherian, then for any ascending chain of ideals
\[
I_1\subset I_2\subset\cdots\subset I_N=I_{N+1}=\cdots
\]
the chain is stationary for some positive integer $N$. To show
that closed sets of $\Spec(R)$ satisfy the dcc, it suffices to
show that basic closed sets for the topology on $\Spec(R)$
satisfy the dcc. Consider the chain
\[
V(I_1)\supset V(I_2)=V(I_1+I_2)=V(I_1)\cap V(I_2)\supset
V(I_1)\cap V(I_2)\cap V(I_3)\supset\cdots.
\]
This chain, like before, stabilizes at $N$ so that we have
\[
V(I_1)\supset V(I_2)\supset\cdots\supset V(I_N)=V(I_{N+1})=\cdots.
\]
\end{proof}
\newpage
\begin{problem}
A nonempty closed subset $V$ of a topological space is called
\emph{irreducible} if $V=V_1\cup V_2$, $V_1$ and $V_2$ closed
subset, implies $V_1=V$ or $V_2=V$.
\begin{enumerate}[noitemsep,label=(\alph*)]
\item Show that in a Noetherian topological space every nonempty
  closed subset is a finite union of irreducible closed subsets.
\item Show that $V(\mathfrak{p})$, $\mathfrak{p}\in\Spec(R)$, are
  exactly the irreducible closed subsets of $\Spec(R)$.
\end{enumerate}
\end{problem}
\begin{proof}
\end{proof}
\newpage
\begin{problem}
Show that a Noetherian ring has only finitely many minimal prime
ideals.
\end{problem}
\begin{proof}
\end{proof}
\newpage
\begin{problem}
Show that a nonzero ring has at least one minimal prime ideal.
\end{problem}
\begin{proof}
\end{proof}

%%% Local Variables:
%%% mode: latex
%%% TeX-master: "../MA557-HW-Current"
%%% End:
