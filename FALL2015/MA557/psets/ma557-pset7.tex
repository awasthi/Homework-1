\begin{problem}
Let $R$ be a Noetherian ring and $I,J$ $R$-ideals. Write
$I^{\langle  J \rangle}=\bigcup_{n\geq 1}(I:J^n)$, which is
called the \emph{saturation of $I$ with respect to $J$}. Show:
\begin{enumerate}[label=(\alph*)]
\item If $I=\bigcap_{i=1}^m\mathfrak{q}_i$ with $\mathfrak{q}_i$
  $\mathfrak{p}_i$-primary, then $I^{\langle
    J\rangle}=\bigcap_{J\nsubset\mathfrak{p}_i}\mathfrak{q}_i$.
\item $I^{\langle  J \rangle}$ is the unique largest $R$-ideal
  that coincides with $I$ locally on the open set
  $\Spec(R)\smallsetminus V(J)$.
\end{enumerate}
\end{problem}
\begin{proof}
(a) We shall demonstrate double inclusion: Let
$\bigcap_{i=1}^m\mathfrak{q}_i$ be a minimal decomposition of $I$ into
primary ideals where $\mathfrak{q}_i$ is
$\mathfrak{p}_i$-primary. $\implies$ Suppose $x\in I^{\langle J\rangle}$
then $xJ^n\subset I$ for some $n\geq 1$. Given $i$ such that
$\mathfrak{p}_i\nsupset J$\footnote{Why does such an ideal exist? Well,
  suppose that $\mathfrak{p}_i\supset J$ for all $1\leq i\leq
  m$. Then
  $J\subset\bigcap_{i=1}^m\mathfrak{p}_i=\bigcap_{i=1}^m\sqrt{\mathfrak{q}_i}=\sqrt{\bigcap_{i=1}^m\mathfrak{q}_i}=\sqrt{I}$
so that }
take $y\in
J\smallsetminus\mathfrak{p}_i$. Then $xy^n\in\mathfrak{q}_i$ so
$x\in\mathfrak{q}_i$ since $\mathfrak{q}_i$ is primary and
$y\notin\mathfrak{p}_i$. Hence, $I^{\langle
  J\rangle}\subset\bigcap_{J\nsubset\mathfrak{p}_i}\mathfrak{q}_i$. $\impliedby$
Conversely, suppose that
$x\in\bigcap_{J\nsubset\mathfrak{p}_i}\mathfrak{q}_i$ then
$x\in\mathfrak{q}_i$ for all $\mathfrak{q}_i\nsupset J$. Take any
$\mathfrak{p}_j$ containing $J$. Then
$\mathfrak{p}_j=\nil(R/\mathfrak{q}_j)^c$ (this is easily seen
from the fact that $\mathfrak{p}_i=\sqrt{\mathfrak{q}_i}$, i.e.,
$\mathfrak{q}_i$ is $\mathfrak{p}_i$-primary and the
correspondence theorem for ideals) so there exists $n_j$ with
$xJ^{n_j}\subset\mathfrak{q}_j$ (since, in the quotient, $\bar J$
is nilpotent). Let $n$ be the maximum of all such $n_j$ then
$xJ^n\mathfrak{q}_i$ for all $i$, i.e,
$x\in(I:J^n)=\bigcap_{i}^m(\mathfrak{q}_i:J^n)$. Thus, $x\in
I^{\langle  J \rangle}$.
\\\\
(b) We will prove that $I^{\langle  J\rangle}$ is precisely the
set of all $x\in R$ such that $x/1$ vanishes in
$R_{\mathfrak{p}}$ for all $\mathfrak{p}\nsupset J$. $\implies$
Given $x\in I^{\langle J\rangle}$, $xJ^n\subset I$ for some
$n\geq 1$. Let $\mathfrak{p}$ be a prime ideal not containing $J$
and let $y\in J\smallsetminus\mathfrak{p}$. Then $xy^n\in I$ and
$y^n\notin\mathfrak{p}$ so $x/1=0$ in
$R_{\mathfrak{p}}$. $\impliedby$ Conversely, suppose that $x/1$
vanishes in $R_{\mathfrak{p}}$ for some prime ideal
$\mathfrak{p}\subset R$. Then $xy=0$ for some
$y\in R\smallsetminus\mathfrak{p}$. Since
$\mathfrak{p}=\sqrt{\mathfrak{q}_i}$ for some $i$,
$y^n\in\mathfrak{q}_i$ for some $n\geq 1$. Let
$\bigcap_{i=1}^m\mathfrak{q}_i$ be a minimal decomposition of $0$
(one exists since $R$ is Noetherian) where $\mathfrak{q}_i$ is
$\mathfrak{p}_i$-primary. By part (a), it suffices to show that
\end{proof}
\newpage
\begin{problem}
Let $R$ be a Noetherian ring. Show that $R$ is reduced if and
only if $\Quot(R)$ is a finite direct product of fields.
\end{problem}
\begin{proof}
$\implies$ Suppose that $R$ is reduced then the nilradical of $R$ is
precisely the $0$ ideal. By 5.1(b), we know that the nilradical is
equivalent to the union
$\bigcup_{\mathfrak{p}\in\Ass R}\mathfrak{p}$. Moreover, by 5.4, we know
that the set of associated primes of $R$ is finite, say
$\Ass R=\{\mathfrak{p}_1,...,\mathfrak{p}_n\}$. Therefore,
$\Quot(R)=S^{-1}R$ where $S=R\smallsetminus\bigcup_{i=1}^n\mathfrak{p}_i$.
Now, observe that by 5.3 the prime ideals of $\Quot(R)$
are precisely the ideals $S^{-1}\mathfrak{p}_i$
\end{proof}
\newpage
\begin{problem}
Let $R$ be a Noetherian ring and $x\in R$ an $R$-regular
element. Show that $\Ass_R\left(R/(x^n)\right)=\Ass_R(R/(x))$ for
every $n\geq 1$.
\end{problem}
\begin{proof}
\end{proof}
\newpage
\begin{problem}
Let $\phi\colon R\to T$ be a homomorphism of rings where $T$ is
Noetherian, let $^a\phi$ be the induced map on the spectra, and
let $N$ be a $T$-module. Show:
\begin{enumerate}[label=(\alph*)]
\item $\Ass_R(N)=^a\phi(\Ass_T(N))$.
\item If $N$ is finitely generated as a $T$-module then
  $\Ass_R(N)$ is finite.
\end{enumerate}
\end{problem}
\begin{proof}
\end{proof}
\newpage
\begin{problem}
Let $K$ be a field that is a finitely generated
$\ZZ$-algebra. Show that $K$ is a finite field.
\end{problem}
\begin{proof}
\end{proof}
\newpage
\begin{problem}
Let $k$ be a Noetherian ring, $R$ a finitely generated
$k$-algebra, and $\Aut_k(R)$ the group of $k$-algebra
automorphisms of $R$. For a subgroup $G$ of $\Aut_k(R)$ write
$R^G=\left\{\,x\in R\;\middle|\;\text{$\sigma(x)=x$ for every
    $\sigma\in G$}\,\right\}$, which is called the \emph{ring of
  invariants} of $G$. Show that if $G$ is finite then $R^G$ is a
finitely generated $k$-algebra (and hence a Noetherian ring).
\end{problem}
\begin{proof}
\end{proof}

%%% Local Variables:
%%% mode: latex
%%% TeX-master: "../MA557-HW-Current"
%%% End:
