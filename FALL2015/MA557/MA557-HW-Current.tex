\def\documentauthor{Carlos Salinas}
\def\documenttitle{MA557 Homework \hwnum}
\def\hwnum{6}
\def\shorttitle{MA557 HW \hwnum}
\def\coursename{MA557}
\def\documentsubject{commutative algebra i}
\def\authoremail{salinac@purdue.edu}

\documentclass[article,oneside,10pt]{memoir}
\usepackage{geometry}
\usepackage[dvipsnames]{xcolor}
\usepackage[
    breaklinks,
    bookmarks=true,
    colorlinks=true,
    pageanchor=false,
    linkcolor=black,
    anchorcolor=black,
    citecolor=black,
    filecolor=black,
    menucolor=black,
    runcolor=black,
    urlcolor=black,
    hyperindex=false,
    hyperfootnotes=true,
    pdftitle={\shorttitle},
    pdfauthor={\documentauthor},
    pdfkeywords={\documentsubject},
    pdfsubject={\coursename}
    ]{hyperref}

% Use symbols instead of numbers
\renewcommand*{\thefootnote}{\fnsymbol{footnote}}

\usepackage{graphicx}
\graphicspath{{figures/}}

% Misc
\usepackage{microtype}
\usepackage{multicol}
\usepackage[inline]{enumitem}
\usepackage{listings}
\usepackage{mleftright}
\mleftright

%% Math
\usepackage{amsthm}
\usepackage{amssymb}
\usepackage{mathtools}

%% PDFTeX specific
\usepackage[mathcal]{euscript}
\usepackage{mathrsfs}

\usepackage{cmap}
\usepackage[T2A,T1]{fontenc}
\usepackage[utf8]{inputenc}
\usepackage[french,german,russian,spanish,english]{babel}
\babeltags{fr=french,
           de=german,
           ru=russian,
           es=spanish,
           en=english}
\def\spanishoptions{mexico}
\usepackage{CJKutf8}
\newcommand{\textha}[1]{\begin{CJK}{UTF8}{mj}#1\end{CJK}}
\newcommand{\textni}[1]{\begin{CJK}{UTF8}{min}#1\end{CJK}}
\newcommand{\textzh}[1]{\begin{CJK}{UTF8}{bsmi}#1\end{CJK}}

%% Theorems and definitions
%% remove parentheses
\makeatletter
\def\thmhead@plain#1#2#3{%
  \thmname{#1}\thmnumber{\@ifnotempty{#1}{ }\@upn{#2}}%
  \thmnote{ {\the\thm@notefont#3}}}
\let\thmhead\thmhead@plain
\makeatother

\theoremstyle{plain}
\newtheorem{theorem}{Theorem}
\newtheorem{proposition}[theorem]{Proposition}
\newtheorem{corollary}[theorem]{Corollary}
\newtheorem{claim}[theorem]{Claim}
\newtheorem{lemma}[theorem]{Lemma}
\newtheorem{axiom}[theorem]{Axiom}

\newtheorem*{corollary*}{Corollary}
\newtheorem*{claim*}{Claim}
\newtheorem*{lemma*}{Lemma}
\newtheorem*{proposition*}{Proposition}
\newtheorem*{theorem*}{Theorem}

\theoremstyle{definition}
\newtheorem{definition}{Definition}
\newtheorem{example}{Examples}
\newtheorem{examples}[example]{Examples}
% \newtheorem{exercise}{Exercise}[section]
% \newtheorem{problem}[exercise]{Problem}

% \newtheorem{exercise}{Exercise}[section]
% \newtheorem{problem}[exercise]{Problem}

\newcounter{problem}
\newenvironment{problem}[1][]% environment name
{% begin code
  \stepcounter{problem}
  \par\vspace{\baselineskip}\noindent
  \ifx &#1&%
  {\normalfont\Large\bfseries\scshape Problem~\hwnum.\theproblem}
  \global\def\exercisename{Problem~\hwnum.\theproblem}%
  \else
  {\normalfont\Large\bfseries\scshape Problem~\hwnum.\theproblem~(#1)}
  \global\def\exercisename{Problem~\hwnum.\theproblem(#1)}
  \fi
  \par\vspace{\baselineskip}%
  \noindent\ignorespaces
}%
{% end code
  \par\vspace{\baselineskip}%
  \noindent\ignorespacesafterend
}

\newtheorem*{definition*}{Definition}
\newtheorem*{example*}{Examples}
\newtheorem*{examples*}{Examples}
\newtheorem*{exercise*}{Exercise}
\newtheorem*{problem*}{Problem}

\theoremstyle{remark}
\newtheorem{remark}{Remark}
\newtheorem{remarks}[remark]{Remarks}
\newtheorem{observation}[remark]{Observation}
\newtheorem{observations}[remark]{Observations}

\newtheorem*{remark*}{**Remark**}
\newtheorem*{remarks*}{**Remarks**}
\newtheorem*{observation*}{**Observation**}
\newtheorem*{observations*}{**Observations**}

%% Redefinitions & commands
% \newcommand\restr[2]{{% we make the whole thing an ordinary symbol
%   \left.\kern-\nulldelimiterspace % automatically resize the bar with \right
%   {#1} % the function
%   % \vphantom{\big|} % pretend it's a little taller at normal size
%   \right|{#2} % this is the delimiter
%   }}

\newcommand\xtwoheadrightarrow[2][]{%
  \mathrel{\ooalign{$\xrightarrow[#1\mkern4mu]{#2\mkern4mu}$\cr%
  \hidewidth$\rightarrow\mkern4mu$}}}

\newcommand{\nsubset}{\ensuremath{\not\subset}}
\newcommand{\nsupset}{\ensuremath{\not\supset}}

\renewcommand\qedsymbol{\ensuremath{\null\hfill\blacksquare}}

%% Commands and operators
\DeclareMathOperator{\Ass}{Ass}
\DeclareMathOperator{\Aut}{Aut}
\DeclareMathOperator{\End}{End}
\DeclareMathOperator{\Fitt}{Fitt}
\DeclareMathOperator{\Hom}{Hom}
\DeclareMathOperator{\Quot}{Quot}
\DeclareMathOperator{\Spec}{Spec}
\DeclareMathOperator{\MSpec}{\mathfrak{m}-Spec}
\DeclareMathOperator{\Supp}{Supp}

\DeclareMathOperator{\ann}{ann}
\DeclareMathOperator{\coker}{coker}
\DeclareMathOperator{\id}{id}
\DeclareMathOperator{\im}{im}
\DeclareMathOperator{\lcm}{lcm}
\DeclareMathOperator{\nil}{nil}
\DeclareMathOperator{\rad}{rad}
\DeclareMathOperator{\rk}{rk}

\newcommand{\CC}{\mathbf{C}}
\newcommand{\NN}{\mathbf{N}}
\newcommand{\QQ}{\mathbf{Q}}
\newcommand{\RR}{\mathbf{R}}
\newcommand{\ZZ}{\mathbf{Z}}

\begin{document}
% Renewcommands
% \renewcommand\setminus{\smallsetminus}
% \renewcommand\phi{\varphi}
% \renewcommand\epsilon{\varepsilon}

\frontmatter
\aliaspagestyle{title}{empty}
\pagestyle{title}
\author{\href{mailto:\authoremail}{\documentauthor}}
\title{\documenttitle}
\date{\today}
\maketitle
\cleartooddpage

\makeoddhead{headings}
        {\small{\MakeUppercase{\itshape\documentauthor}}}
        {}
        {\small{\MakeUppercase{\itshape\exercisename}}}
\makeoddfoot{headings}{{\itshape\documenttitle}}
                      {}
                      {\thepage}
\makeevenhead{headings}
        {\small{\MakeUppercase{\itshape\documentauthor}}}
        {}
        {\small{\MakeUppercase{\itshape\exercisename}}}
\makeevenfoot{headings}{{\itshape\documenttitle}}
                      {}
                      {\thepage}
\makeheadrule{headings}{\textwidth}{.25pt}
% \makerunningwidth{headings}{1.15\textwidth}
\pagestyle{headings}

\mainmatter
\begin{problem}
Let $R$ be a Noetherian ring and $I,J$ $R$-ideals. Write
$I^{\langle  J \rangle}=\bigcup_{n\geq 1}(I:J^n)$, which is
called the \emph{saturation of $I$ with respect to $J$}. Show:
\begin{enumerate}[label=(\alph*)]
\item If $I=\bigcap_{i=1}^m\mathfrak{q}_i$ with $\mathfrak{q}_i$
  $\mathfrak{p}_i$-primary, then $I^{\langle
    J\rangle}=\bigcap_{J\nsubset\mathfrak{p}_i}\mathfrak{q}_i$.
\item $I^{\langle  J \rangle}$ is the unique largest $R$-ideal
  that coincides with $I$ locally on the open set
  $\Spec(R)\smallsetminus V(J)$.
\end{enumerate}
\end{problem}
\begin{proof}
(a) We shall demonstrate double inclusion: Let
$\bigcap_{i=1}^m\mathfrak{q}_i$ be a minimal decomposition of $I$ into
primary ideals where $\mathfrak{q}_i$ is
$\mathfrak{p}_i$-primary. $\implies$ Suppose $x\in I^{\langle J\rangle}$
then $xJ^n\subset I$ for some $n\geq 1$. Given $i$ such that
$\mathfrak{p}_i\nsupset J$\footnote{Why does such an ideal exist? Well,
  suppose that $\mathfrak{p}_i\supset J$ for all $1\leq i\leq m$. Then} take $y\in J\smallsetminus\mathfrak{p}_i$. Then
$xy^n\in\mathfrak{q}_i$ so $x\in\mathfrak{q}_i$ since $\mathfrak{q}_i$ is
primary and $y\notin\mathfrak{p}_i$. Hence, $I^{\langle
  J\rangle}\subset\bigcap_{J\nsubset\mathfrak{p}_i}\mathfrak{q}_i$. $\impliedby$
Conversely, suppose that
$x\in\bigcap_{J\nsubset\mathfrak{p}_i}\mathfrak{q}_i$ then
$x\in\mathfrak{q}_i$ for all $\mathfrak{q}_i\nsupset J$. Take any
$\mathfrak{p}_j$ containing $J$. Then
$\mathfrak{p}_j=\nil(R/\mathfrak{q}_j)^c$ (this is easily seen from the
fact that $\mathfrak{p}_i=\sqrt{\mathfrak{q}_i}$, i.e., $\mathfrak{q}_i$ is
$\mathfrak{p}_i$-primary and the correspondence theorem for ideals) so
there exists $n_j$ with $xJ^{n_j}\subset\mathfrak{q}_j$ (since, in the
quotient, $\bar J$ is nilpotent). Let $n$ be the maximum of all such
$n_j$ then we claim that $xJ^n\subset\mathfrak{q}_i$ for all $i$. We have
already seen what happens in case that $\mathfrak{p}_i\supset J$ so let us
consider the case that $\mathfrak{p}_i\nsupset J$.
\end{proof}
\newpage
\begin{problem}
Let $R$ be a Noetherian ring. Show that $R$ is reduced if and
only if $\Quot(R)$ is a finite direct product of fields.
\end{problem}
\begin{proof}
$\implies$ Suppose that $R$ is reduced.
\end{proof}
\newpage
\begin{problem}
Let $R$ be a Noetherian ring and $x\in R$ an $R$-regular
element. Show that $\Ass_R\left(R/(x^n)\right)=\Ass_R(R/(x))$ for
every $n\geq 1$.
\end{problem}
\begin{proof}
\end{proof}
\newpage
\begin{problem}
Let $\phi\colon R\to T$ be a homomorphism of rings where $T$ is
Noetherian, let $^a\phi$ be the induced map on the spectra, and
let $N$ be a $T$-module. Show:
\begin{enumerate}[label=(\alph*)]
\item $\Ass_R(N)=^a\phi(\Ass_T(N))$.
\item If $N$ is finitely generated as a $T$-module then
  $\Ass_R(N)$ is finite.
\end{enumerate}
\end{problem}
\begin{proof}
\end{proof}
\newpage
\begin{problem}
Let $K$ be a field that is a finitely generated
$\ZZ$-algebra. Show that $K$ is a finite field.
\end{problem}
\begin{proof}
\end{proof}
\newpage
\begin{problem}
Let $k$ be a Noetherian ring, $R$ a finitely generated
$k$-algebra, and $\Aut_k(R)$ the group of $k$-algebra
automorphisms of $R$. For a subgroup $G$ of $\Aut_k(R)$ write
$R^G=\left\{\,x\in R\;\middle|\;\text{$\sigma(x)=x$ for every
    $\sigma\in G$}\,\right\}$, which is called the \emph{ring of
  invariants} of $G$. Show that if $G$ is finite then $R^G$ is a
finitely generated $k$-algebra (and hence a Noetherian ring).
\end{problem}
\begin{proof}
\end{proof}

%%% Local Variables:
%%% mode: latex
%%% TeX-master: "../MA557-HW-Current"
%%% End:

\end{document}

%%% Local Variables:
%%% mode: latex
%%% TeX-master: t
%%% End:
