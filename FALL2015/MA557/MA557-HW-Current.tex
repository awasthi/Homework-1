\def\documentauthor{Carlos Salinas}
\def\documenttitle{MA557 Homework \hwnum}
\def\hwnum{6}
\def\shorttitle{MA557 HW \hwnum}
\def\coursename{MA557}
\def\documentsubject{commutative algebra i}
\def\authoremail{salinac@purdue.edu}

\documentclass[article,oneside,10pt]{memoir}
\usepackage{geometry}
\usepackage[dvipsnames]{xcolor}
\usepackage[
    breaklinks,
    bookmarks=true,
    colorlinks=true,
    pageanchor=false,
    linkcolor=black,
    anchorcolor=black,
    citecolor=black,
    filecolor=black,
    menucolor=black,
    runcolor=black,
    urlcolor=black,
    hyperindex=false,
    hyperfootnotes=true,
    pdftitle={\shorttitle},
    pdfauthor={\documentauthor},
    pdfkeywords={\documentsubject},
    pdfsubject={\coursename}
    ]{hyperref}

\usepackage{graphicx}
\graphicspath{{figures/}}

% Misc
\usepackage{microtype}
\usepackage{multicol}
\usepackage[inline]{enumitem}
\usepackage{listings}
\usepackage{mleftright}
\mleftright

%% Math
\usepackage{amsthm}
\usepackage{amssymb}
\usepackage{mathtools}

%% PDFTeX specific
% \usepackage[mathcal]{euscript}
% \usepackage{mathrsfs}

\usepackage{cmap}
\usepackage[T2A,T1]{fontenc}
\usepackage[utf8]{inputenc}
\usepackage[french,german,russian,spanish,english]{babel}
\babeltags{fr=french,
           de=german,
           ru=russian,
           es=spanish,
           en=english}
\def\spanishoptions{mexico}
\usepackage{CJKutf8}
\newcommand{\textha}[1]{\begin{CJK}{UTF8}{mj}#1\end{CJK}}
\newcommand{\textni}[1]{\begin{CJK}{UTF8}{min}#1\end{CJK}}
\newcommand{\textzh}[1]{\begin{CJK}{UTF8}{bsmi}#1\end{CJK}}

%% Theorems and definitions
\theoremstyle{plain}
\newtheorem{theorem}{Theorem}
\newtheorem{proposition}[theorem]{Proposition}
\newtheorem{corollary}[theorem]{Corollary}
\newtheorem{claim}[theorem]{Claim}
\newtheorem{lemma}[theorem]{Lemma}
\newtheorem{axiom}[theorem]{Axiom}

\newtheorem*{corollary*}{Corollary}
\newtheorem*{claim*}{Claim}
\newtheorem*{lemma*}{Lemma}
\newtheorem*{proposition*}{Proposition}
\newtheorem*{theorem*}{Theorem}

\theoremstyle{definition}
\newtheorem{definition}{Definition}
\newtheorem{example}{Examples}
\newtheorem{examples}[example]{Examples}
% \newtheorem{exercise}{Exercise}[section]
% \newtheorem{problem}[exercise]{Problem}

% \newtheorem{exercise}{Exercise}[section]
% \newtheorem{problem}[exercise]{Problem}

\newcounter{problem}
\newenvironment{problem}[1][]% environment name
{% begin code
  \stepcounter{problem}
  \par\vspace{\baselineskip}\noindent
  \ifx &#1&%
  {\normalfont\Large\bfseries\scshape Problem~\hwnum.\theproblem}
  \global\def\exercisename{Problem~\hwnum.\theproblem}%
  \else
  {\normalfont\Large\bfseries\scshape Problem~\hwnum.\theproblem~(#1)}
  \global\def\exercisename{Problem~\hwnum.\theproblem(#1)}
  \fi
  \par\vspace{\baselineskip}%
  \noindent\ignorespaces
}%
{% end code
  \par\vspace{\baselineskip}%
  \noindent\ignorespacesafterend
}

\newtheorem*{definition*}{Definition}
\newtheorem*{example*}{Examples}
\newtheorem*{examples*}{Examples}
\newtheorem*{exercise*}{Exercise}
\newtheorem*{problem*}{Problem}

\theoremstyle{remark}
\newtheorem{remark}{Remark}
\newtheorem{remarks}[remark]{Remarks}
\newtheorem{observation}[remark]{Observation}
\newtheorem{observations}[remark]{Observations}

\newtheorem*{remark*}{**Remark**}
\newtheorem*{remarks*}{**Remarks**}
\newtheorem*{observation*}{**Observation**}
\newtheorem*{observations*}{**Observations**}

%% Redefinitions & commands
\newcommand\restr[2]{{% we make the whole thing an ordinary symbol
  \left.\kern-\nulldelimiterspace % automatically resize the bar with \right
  {#1} % the function
  % \vphantom{\big|} % pretend it's a little taller at normal size
  \right|{#2} % this is the delimiter
  }}

\newcommand{\nsubset}{\ensuremath{\not\subset}}
\newcommand{\nsupset}{\ensuremath{\not\supset}}

\renewcommand\qedsymbol{\ensuremath{\null\hfill\blacksquare}}

%% Commands and operators
\DeclareMathOperator{\Ass}{Ass}
\DeclareMathOperator{\End}{End}
\DeclareMathOperator{\Fitt}{Fitt}
\DeclareMathOperator{\Hom}{Hom}
\DeclareMathOperator{\Spec}{Spec}
\DeclareMathOperator{\MSpec}{\mathfrak{m}-Spec}
\DeclareMathOperator{\Supp}{Supp}

\DeclareMathOperator{\ann}{ann}
\DeclareMathOperator{\coker}{coker}
\DeclareMathOperator{\id}{id}
\DeclareMathOperator{\im}{im}
\DeclareMathOperator{\lcm}{lcm}
\DeclareMathOperator{\nil}{nil}
\DeclareMathOperator{\rad}{rad}
\DeclareMathOperator{\rk}{rk}

\newcommand{\CC}{\mathbf{C}}
\newcommand{\NN}{\mathbf{N}}
\newcommand{\QQ}{\mathbf{Q}}
\newcommand{\RR}{\mathbf{R}}
\newcommand{\ZZ}{\mathbf{Z}}

\begin{document}
% Renewcommands
\renewcommand\setminus{\smallsetminus}
\renewcommand\phi{\varphi}
\renewcommand\epsilon{\varepsilon}

\frontmatter
\aliaspagestyle{title}{empty}
\pagestyle{title}
\author{\href{mailto:\authoremail}{\documentauthor}}
\title{\documenttitle}
\date{\today}
\maketitle
\cleartooddpage

\makeoddhead{headings}
        {\small{\MakeUppercase{\itshape\documentauthor}}}
        {}
        {\small{\MakeUppercase{\itshape\exercisename}}}
\makeoddfoot{headings}{{\itshape\documenttitle}}
                      {}
                      {\thepage}
\makeevenhead{headings}
        {\small{\MakeUppercase{\itshape\documentauthor}}}
        {}
        {\small{\MakeUppercase{\itshape\exercisename}}}
\makeevenfoot{headings}{{\itshape\documenttitle}}
                      {}
                      {\thepage}
\makeheadrule{headings}{\textwidth}{.25pt}
% \makerunningwidth{headings}{1.15\textwidth}
\pagestyle{headings}

\mainmatter
\begin{problem}
For an $n$ by $n$ matrix $\phi$ with entries in $R$ write
$I_t(\phi)$ for the $R$-ideal generated by all the $t$ by $t$
minors of $\phi$ (set $I_t(\phi)=R$ for $t\leq 0$ and
$I_t(\phi)=0$ for $t>\min\{m,n\}$). Thinking of $\phi$ as an
$R$-linear map $\phi\colon R^m\to R^n$ set $M=\coker(\phi)$ and
define $F_i(M)=\Fitt_i(M)=I_{n-i}(\phi)$. This ideal is called
the \emph{$i$th Fitting ideal of $M$}. Show:
\begin{enumerate}[label=(\alph*)]
\item $F_i(M)$ only depends on $i$ and $M$ (but not on
  $m,n,\phi$).
\item $(\ann(M))^n\subset F_0(M)\subset\ann(M)$.
\item In case $R$ is local, $F_i(M)=R$ if and only if $\mu(M)\leq
  i$.
\item
  $V(F_i(M))=
\left\{\,\mathfrak{p}\in\Spec(R)\;\middle|\;\mu_{R_{\mathfrak{p}}}(M_{\mathfrak{p}})>i\,\right\}$.
\end{enumerate}
\end{problem}
\begin{proof}
\end{proof}
\newpage
\begin{problem}
Let $I$ be an ideal in a Noetherian ring. Show that either $I$
contains an $R$-regular element or else $aI=0$ for some $0\neq
a\in R$.
\end{problem}
\begin{proof}
Suppose $R$ is a Noetherian ring and $I\subset R$ an ideal. Then, by
3.2, $I$ is finitely generated, say $I=(a_1,...,a_n)$, for
$a_1,...,a_n\in R$. Then, either $I$ contains an $R$-regular element
or it does not. If $I$ does not contain an $R$-regular element, then
for every $a_i$ there exists $x_i\in R$ such that $a_ix_i=0$. Thus,
$I\subset\bigcup_{i=1}^n\ann(x_i)$, but each
$\ann(x_i)\subset\mathfrak{p}_i$ for some $\mathfrak{p}\in\Ass(R)$ so
$I\subset\bigcup_{i=1}^m\mathfrak{p}_i$ for $m\leq n$. By the prime
avoidance lemma, 1.7, it follows that
$I\subset\mathfrak{p}_i=\ann(y_i)$ for some $1\leq i\leq m$. Thus,
$y_iI=0$.
\end{proof}
\newpage
\begin{problem}
Let $I\subset J$ be ideals in a Noetherian ring. Show that if
$I_{\mathfrak{p}}=J_{\mathfrak{p}}$ for every associated prime
$\mathfrak{p}$ of $I$, then $I=J$.
\end{problem}
\begin{proof}

\end{proof}
\newpage
\begin{problem}
Let $R$ be a Noetherian ring and $M$ a finite $R$-module. Show
that $\ell(M)<\infty$ if and only  if
$\Supp(M)\subset\MSpec(R)$.
\end{problem}
\begin{proof}
\end{proof}
\newpage
\begin{problem}
Let $R$ be a Noetherian ring, $M\neq 0$ a finite $R$-module, and
\[
0=M_0\subset M_1\subset \cdots\subset M_n=M
\]
a chain of submodules with $M_i/M_{i-1}\cong R/\mathfrak{p}_i$,
$\mathfrak{p}_i\in\Spec(R)$.
\begin{enumerate}[label=(\alph*)]
\item Show that
  $\Ass(M)\subset\left\{\mathfrak{p}_1,...,\mathfrak{p}_n\right\}$
  and that the minimal elements of the two sets coincide (hence
  only depend on $M$).
\item Let $\mathfrak{p}$ be minimal in
  $\left\{\mathfrak{p}_1,...,\mathfrak{p}_n\right\}$. Show that
  in any chain as above, the multiplicity with witch the factor
  $R/\mathfrak{p}$ appears is
  $\ell_{R_{\mathfrak{p}}}(M_{\mathfrak{p}})$ (hence only depends
  on $M$).
\end{enumerate}
\end{problem}
\begin{proof}
\end{proof}
\newpage
\begin{problem}
Let $R=k[X,Y]$ be a polynomial ring over a field and
$I=(X^2,XY)\subset R$. Find two distinct shortest primary
decompositions of $I$.
\end{problem}
\begin{proof}
\end{proof}

%%% Local Variables:
%%% mode: latex
%%% TeX-master: "../MA557-HW-Current"
%%% End:

\end{document}

%%% Local Variables:
%%% mode: latex
%%% TeX-master: t
%%% End:
