\def\documentauthor{Carlos Salinas}
\def\documenttitle{MA557 Homework \hwnum}
\def\hwnum{12}
\def\shorttitle{MA557 HW \hwnum}
\def\coursename{MA557}
\def\documentsubject{commutative algebra i}
\def\authoremail{salinac@purdue.edu}

\documentclass[article,oneside,10pt]{memoir}
\usepackage{geometry}
\usepackage[dvipsnames]{xcolor}
\usepackage[
    breaklinks,
    bookmarks=true,
    colorlinks=true,
    pageanchor=false,
    linkcolor=black,
    anchorcolor=black,
    citecolor=black,
    filecolor=black,
    menucolor=black,
    runcolor=black,
    urlcolor=black,
    hyperindex=false,
    hyperfootnotes=true,
    pdftitle={\shorttitle},
    pdfauthor={\documentauthor},
    pdfkeywords={\documentsubject},
    pdfsubject={\coursename}
    ]{hyperref}

% Use symbols instead of numbers
\renewcommand*{\thefootnote}{\fnsymbol{footnote}}

%% Misc
\usepackage{microtype}
\usepackage{multicol}
\usepackage[inline]{enumitem}
\usepackage{mleftright}
\mleftright

%% Math
\usepackage{amsthm}
\usepackage{amssymb}
\usepackage{mathtools}
\usepackage{graphicx}
\graphicspath{{figures/}}

%% PDFTeX specific
\usepackage[mathcal]{euscript}
\usepackage{mathrsfs}
% \usepackage[charter]{mathdesign}

\usepackage[LAE,LFE,T2A,T1]{fontenc}
\usepackage[utf8]{inputenc}
\usepackage[farsi,french,german,spanish,dutch,russian,swedish,english]{babel}
\babeltags{pa=farsi,
           fr=french,
           de=german,
           es=spanish,
           nl=dutch,
           ru=russian,
           sv=swedish,
           en=english}
\def\spanishoptions{mexico}

\selectlanguage{english}

\newcommand{\textfa}[1]{\beginR\textpa{#1}\endR}
% \usepackage{newtxtext,newtxmath}
% \usepackage{stix}
% \usepackage[free]{frfonts}

\usepackage{cmap}
\usepackage{CJKutf8}
\newcommand{\textha}[1]{\begin{CJK}{UTF8}{mj}#1\end{CJK}}
\newcommand{\textni}[1]{\begin{CJK}{UTF8}{min}#1\end{CJK}}
\newcommand{\textzh}[1]{\begin{CJK}{UTF8}{bsmi}#1\end{CJK}}

%% Because French
\usepackage{listings}

%% Math font
% \usepackage[charter]{mathdesign}
% \usepackage[no-math]{fontspec}
% % \usepackage{unicode-math}
% \setmainfont[Ligatures=TeX]{Khartiya}

% %% Language
% \usepackage{polyglossia}

% \newfontfamily\cyrillicfont[Script=Cyrillic]{Khartiya}
% \newfontfamily\farsifont[Script=Arabic,
%                          Scale=MatchUppercase]{Amiri}
% \setmainlanguage[variant=american]{english}
% \setotherlanguage{farsi}
% \setotherlanguage{french}
% \setotherlanguage[spelling=new,latesthyphen,babelshorthands]{german}
% \setotherlanguage{spanish}
% \setotherlanguage[spelling=modern,babelshorthands]{russian}

% \usepackage{xeCJK}
% \usepackage[overlap]{ruby}
% \renewcommand\rubysep{-0.2ex}
% \setCJKmainfont[% BoldFont=AR PL UKai TW,
%                 % ItalicFont=AR PL Ukai TW
%                 ]{AR PL UMing TW}
% \setCJKmainfont[BoldFont=IPAGothic]{IPAMincho}

%% Theorems and definitions
%% remove parentheses
\makeatletter
\def\thmhead@plain#1#2#3{%
  \thmname{#1}\thmnumber{\@ifnotempty{#1}{ }\@upn{#2}}%
  \thmnote{ {\the\thm@notefont#3}}}
\let\thmhead\thmhead@plain
\makeatother

\theoremstyle{plain}
\newtheorem{theorem}{Theorem}
\newtheorem{proposition}[theorem]{Proposition}
\newtheorem{corollary}[theorem]{Corollary}
\newtheorem{claim}[theorem]{Claim}
\newtheorem{lemma}[theorem]{Lemma}
\newtheorem{axiom}[theorem]{Axiom}

\newtheorem*{corollary*}{Corollary}
\newtheorem*{claim*}{Claim}
\newtheorem*{lemma*}{Lemma}
\newtheorem*{proposition*}{Proposition}
\newtheorem*{theorem*}{Theorem}

\theoremstyle{definition}
\newtheorem{definition}{Definition}
\newtheorem{example}{Examples}
\newtheorem{examples}[example]{Examples}
% \newtheorem{exercise}{Exercise}[section]
% \newtheorem{problem}[exercise]{Problem}

\newcounter{problem}
\newenvironment{problem}[1][]% environment name
{% begin code
  \stepcounter{problem}
  \par\vspace{\baselineskip}\noindent
  \ifx &#1&%
  {\normalfont\Large\bfseries\scshape Problem~\hwnum.\theproblem}
  \global\def\exercisename{Problem~\hwnum.\theproblem}%
  \else
  {\normalfont\Large\bfseries\scshape Problem~\hwnum.\theproblem~(#1)}
  \global\def\exercisename{Problem~\hwnum.\theproblem(#1)}
  \fi
  \par\vspace{\baselineskip}%
  \noindent\ignorespaces
}%
{% end code
  \par\vspace{\baselineskip}%
  \noindent\ignorespacesafterend
}

\newtheorem*{definition*}{Definition}
\newtheorem*{example*}{Examples}
\newtheorem*{examples*}{Examples}
\newtheorem*{exercise*}{Exercise}
\newtheorem*{problem*}{Problem}

\theoremstyle{remark}
\newtheorem{remark}{Remark}
\newtheorem{remarks}[remark]{Remarks}
\newtheorem{observation}[remark]{Observation}
\newtheorem{observations}[remark]{Observations}

\newtheorem*{remark*}{**Remark**}
\newtheorem*{remarks*}{**Remarks**}
\newtheorem*{observation*}{**Observation**}
\newtheorem*{observations*}{**Observations**}

%% Redefinitions & commands
% \newcommand\restr[2]{{% we make the whole thing an ordinary symbol
%   \left.\kern-\nulldelimiterspace % automatically resize the bar with \right
%   {#1} % the function
%   % \vphantom{\big|} % pretend it's a little taller at normal size
%   \right|{#2} % this is the delimiter
%   }}

% \newcommand\xtwoheadrightarrow[2][]{%
%   \mathrel{\ooalign{$\xrightarrow[#1\mkern4mu]{#2\mkern4mu}$\cr%
%   \hidewidth$\rightarrow\mkern4mu$}}}

\newcommand\minus{\smallsetminus}
\newcommand{\nsubset}{\ensuremath{\not\subset}}
\newcommand{\nsupset}{\ensuremath{\not\supset}}

\renewcommand\qedsymbol{\ensuremath{\null\hfill\blacksquare}}

%% Commands and operators
\DeclareMathOperator{\Ass}{Ass}
\DeclareMathOperator{\Aut}{Aut}
\DeclareMathOperator{\Cl}{Cl}
\DeclareMathOperator{\End}{End}
\DeclareMathOperator{\Fitt}{Fitt}
\DeclareMathOperator{\Hom}{Hom}
\DeclareMathOperator{\Quot}{Quot}
\DeclareMathOperator{\Spec}{Spec}
\DeclareMathOperator{\MSpec}{\mathfrak{m}-Spec}
\DeclareMathOperator{\Supp}{Supp}
\DeclareMathOperator{\Tor}{Tor}

\DeclareMathOperator{\ann}{ann}
\DeclareMathOperator{\coker}{coker}
\DeclareMathOperator{\id}{id}
\DeclareMathOperator{\im}{im}
\DeclareMathOperator{\lcm}{lcm}
\DeclareMathOperator{\Ht}{ht}
\DeclareMathOperator{\nil}{nil}
\DeclareMathOperator{\rad}{rad}
\DeclareMathOperator{\rk}{rk}

\newcommand{\bbC}{\mathbb{C}}
\newcommand{\bfC}{\mathbf{C}}
\newcommand{\bbN}{\mathbb{N}}
\newcommand{\bfN}{\mathbf{N}}
\newcommand{\bbQ}{\mathbb{Q}}
\newcommand{\bfQ}{\mathbf{Q}}
\newcommand{\bbR}{\mathbb{R}}
\newcommand{\bfR}{\mathbf{R}}
\newcommand{\bbZ}{\mathbb{Z}}
\newcommand{\bfZ}{\mathbf{Z}}

\begin{document}
\frontmatter
\aliaspagestyle{title}{empty}
\pagestyle{title}
\author{\href{mailto:\authoremail}{\documentauthor}}
\title{\documenttitle}
\date{\today}
\maketitle
\cleartooddpage
\makepagestyle{my-headings}
\makeoddhead{my-headings}
        {\small{\MakeUppercase{\itshape\documentauthor}}}
        {}
        {\small{\MakeUppercase{\itshape\exercisename}}}
\makeoddfoot{my-headings}{{\itshape\documenttitle}}
                      {}
                      {\thepage}
\makeevenhead{my-headings}
        {\small{\MakeUppercase{\itshape\documentauthor}}}
        {}
        {\small{\MakeUppercase{\itshape\exercisename}}}
\makeevenfoot{my-headings}{{\itshape\documenttitle}}
                      {}
                      {\thepage}
\makeheadrule{my-headings}{\textwidth}{.25pt}
\pagestyle{my-headings}
% \makerunningwidth{headings}{1.15\textwidth}
\mainmatter

\begin{problem}
Let $R$ be a Noetherian domain. Show that the following are equivalent:
\begin{enumerate}[label=(\roman*)]
\item $R$ is a unique factorization domain
\item every prime ideal of $R$ of height one is principal
\item $R$ is normal with $\Cl(R)=0$.
\end{enumerate}
\end{problem}
\begin{proof}
(i) $\implies$ (ii) Suppose $R$ is a Noetherian domain. Let $\mathfrak{p}$
be a height one prime. Then there exists at least one nonzero element
$x\in\mathfrak{p}$. Let $x=p_1^{\alpha_1}\cdots p_k^{\alpha_k}$ be the
factorization of $x$ into irreducible (prime) elements of $R$. Set
$p\coloneqq p_i$ for any prime in the factorization of $x$. Then the ideal
generated by $p$ is a prime ideal contained in $\mathfrak{p}$, i.e.,
$\langle p \rangle\subset\mathfrak{p}$. But $\Ht(\mathfrak{p})=1$. Thus,
$\langle p \rangle=\mathfrak{p}$.
\\\\
(ii) $\implies$ (ii) Suppose that every height one prime ideal in $R$ is
principal. To show that $R$ is a UFD, it suffices to show that every
irreducible element $p$ is a prime element, that is, $\langle p\rangle$ is
a prime ideal. Let $\mathfrak{p}$ be the minimal prime containing
$p$. Since $\mathfrak{p}$ is principal, $\mathfrak{p}=\langle x\rangle$ for
some $x\in\mathfrak{p}$. Thus, $p=xy$ for some $y\in R$. But $p$ is prime
hence, irreducible so either $x$ or $y$ is a unit. If $x$ is a unit, then
$\mathfrak{p}=R$, which is a contradiction. Thus, $y$ must be a unit and we
see that $\langle p\rangle=\langle xy\rangle=\mathfrak{p}$ is
prime.

Now, for the following implications we need to know a couple of
denfinitions: Let $D(R)$ denote the set of divisional fractional $R$-ideals
and $F(R)$ denote the set of all principal fractional ideals. Then the
\emph{divisor class group of $R$} is the quotient $\Cl(R)\coloneqq
D(R)/F(R)$.
\end{proof}
\newpage
\begin{problem}
Let $R$ be a ring with total ring of quotients $K$, $M$ an $R$-module, and
\[
\mathcal{T}(M)=
\left\{\,x\in M\;\middle|\;
\text{$ax=0$ for some non zero-divisor $a$ of $R$}\,\right\}.
\]
The submodule $\mathcal{T}(M)$ is called the \emph{torsion of $M$}, and $M$
is called \emph{torsion free} if $\mathcal{T}(M)=0$. Show
\begin{enumerate}[label=(\alph*)]
\item $\mathcal{T}(M)=\ker(M\to K\otimes_R M)$
\item $M/\mathcal{T}(M)$ is torsion free.
\end{enumerate}
\end{problem}
\begin{proof}
Let $S$ denote the set of all regular elements of $R$ and let
$\varphi\colon R\to K$, where $K\coloneqq S^{-1}R$, be the canonical
localization map $a\mapsto a/1$. We show, by way of double inclusion, that
$\mathcal{T}(M)=\ker\Phi$, where $\Phi\colon M\to K\otimes_R M$ is the
canonical map $x\mapsto 1\otimes x$. Note that this map, $\Phi$, is well
defined by the UMP of the tensor product (HW 2). Now let us show the
containment $\mathcal{T}(M)\subset\ker\Phi$: Let $x\in\mathcal{T}(M)$, then
$x$ is a non-zero divisor of $R$ such that $ax=0$. Since $a$ is a non-zero
divisior, $a\in S$ so $a/1=0/1$ in $K$. Thus, we have
\[
\Phi(xm)=1\otimes x=a/1\otimes x=0\otimes x=0,
\]
so $x\in\ker\Phi$. Conversely, suppose that $x\in\ker(\Phi)$. By some
theorem from the localization section\footnote{Sorry! I misplaced my
notebook and I've been taking notes on sheets of computer paper so I hate
going through the mess.} we have $K\otimes_R M\cong S^{-1}M$. Thus
$1\otimes x=0$ implies that $x=0$ in the localization $S^{-1}M$. This is
true if and only if $ax=0$ for some non-zero divisor $a$ of $R$. Thus,
$x\in\ker\Phi$ and equality holds.
\\\\
(b) Let $N\coloneqqM/\mathcal{T}(M)$. We show that $\mathcal{T}(N)=0$. For
that is suffices to show that every element $x\in\mathcal{T}(N)=0$. But, if
$x$ is, as before, an element of $\mathcal{T}(N)$, then $x$ is in the
kernel $\ker\Phi$, where $\Phi\colon N\to K\otimes_R N$ is the canonical
map.
\end{proof}
\newpage
\begin{problem}
Let $R$ be a Dedekind domain and $M$ a finitely generated $R$-module of
rank $r$. Show that:
\begin{enumerate}[label=(\alph*)]
\item If $M$ is torsion free then $M$ is projective (hint: induct on $r$).
\item $M\cong \mathcal{T}(M)\oplus P$ with $P$ projective.
\item If $M\neq 0$ is projective then $M\cong R^{r-1}\oplus I$ with $I\neq
  0$ an ideal.
\item If $M$ is torsion (i.e., $M=\mathcal{T}(M)$) then
\[
M\cong R/I_1\oplus\cdots\oplus R/I_n\qquad\text{with}\qquad
I_1\supset\cdots\supset I_n\neq 0
\]
ideals (hint: for $p_1,...,p_s$ the minimal primes of $\ann(M)$ and
$S=R\minus(\mathfrak{p}_1\cup\cdots\cup\mathfrak{p}_s)$, show that
$S^{-1}R$ is a PID).
\end{enumerate}
\end{problem}
\begin{proof}
\end{proof}

%%% Local Variables:
%%% mode: latex
%%% TeX-master: "../MA557-HW-Current"
%%% End:

\end{document}

%%% Local Variables:
%%% mode: latex
%%% TeX-master: t
%%% TeX-engine: xetex
%%% End:
