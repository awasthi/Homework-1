\def\documentauthor{Carlos Salinas}
\def\documenttitle{MA557 Problem Set \hwnum}
\def\hwnum{2}
\def\shorttitle{MA557 PSet \hwnum}
\def\coursename{MA557}
\def\documentsubject{commutative algebra i}
\def\authoremail{salinac@purdue.edu}

\documentclass[article,oneside,10pt]{memoir}
\usepackage{geometry}
\usepackage[dvipsnames]{xcolor}
\usepackage[
    breaklinks,
    bookmarks=true,
    colorlinks=true,
    pageanchor=false,
    linkcolor=black,
    anchorcolor=black,
    citecolor=black,
    filecolor=black,
    menucolor=black,
    runcolor=black,
    urlcolor=black,
    hyperindex=false,
    hyperfootnotes=true,
    pdftitle={\shorttitle},
    pdfauthor={\documentauthor},
    pdfkeywords={\documentsubject},
    pdfsubject={\coursename}
    ]{hyperref}

\usepackage{graphicx}
\graphicspath{{figures/}}

\usepackage{multicol}
\usepackage[inline]{enumitem}
\usepackage{listings}
\usepackage{mleftright}
\mleftright

\usepackage{microtype}

\usepackage{amsthm}
\usepackage{amssymb}
\usepackage{mathtools}
% \usepackage{lualatex-math}
\usepackage{unicode-math}

\setmainfont[Ligatures=TeX]{Latin Modern Roman}
\setsansfont[Ligatures=TeX]{Latin Modern Sans}
\setmonofont{Latin Modern Mono}
\setmathfont{Latin Modern Math}

\theoremstyle{plain}
\newtheorem{theorem}{Theorem}
\newtheorem{proposition}[theorem]{Proposition}
\newtheorem{corollary}[theorem]{Corollary}
\newtheorem{claim}[theorem]{Claim}
\newtheorem{lemma}[theorem]{Lemma}
\newtheorem{axiom}[theorem]{Axiom}

\newtheorem*{corollary*}{Corollary}
\newtheorem*{claim*}{Claim}
\newtheorem*{lemma*}{Lemma}
\newtheorem*{proposition*}{Proposition}
\newtheorem*{theorem*}{Theorem}

\theoremstyle{definition}
\newtheorem{definition}{Definition}
\newtheorem{example}{Examples}
\newtheorem{examples}[example]{Examples}
% \newtheorem{exercise}{Exercise}[section]
% \newtheorem{problem}[exercise]{Problem}

\counterwithout{section}{chapter}
\usepackage[explicit]{titlesec}
\titleformat{\section}{\normalfont\Large\bfseries\scshape}{}{0em}{#1}
\newenvironment{problem}[1][]% environment name
{% begin code
  \par\vspace{\baselineskip}\noindent
  \ifx &#1&%
  \section{Problem~\hwnum.\thesection}
  \global\def\exercisename{Problem \hwnum.\thesection}%
  \else
  \section{Problem~\hwnum.\thesection~(#1)}
  \global\def\exercisename{Problem \hwnum.\thesection(#1)}
  \fi
  \par\vspace{\baselineskip}%
  \noindent\ignorespaces
}%
{% end code
  \par\vspace{\baselineskip}%
  \noindent\ignorespacesafterend
}

\newtheorem*{definition*}{Definition}
\newtheorem*{example*}{Examples}
\newtheorem*{examples*}{Examples}
\newtheorem*{exercise*}{Exercise}
\newtheorem*{problem*}{Problem}

\theoremstyle{remark}
\newtheorem{remark}{Remark}
\newtheorem{remarks}[remark]{Remarks}
\newtheorem{observation}[remark]{Observation}
\newtheorem{observations}[remark]{Observations}

\newtheorem*{remark*}{**Remark**}
\newtheorem*{remarks*}{**Remarks**}
\newtheorem*{observation*}{**Observation**}
\newtheorem*{observations*}{**Observations**}

%% Redefinitions & commands
\renewcommand\qedsymbol{\ensuremath{\null\hfill\QED}}

\newcommand\restr[2]{{% we make the whole thing an ordinary symbol
  \left.\kern-\nulldelimiterspace % automatically resize the bar with \right
  {#1} % the function
  % \vphantom{\big|} % pretend it's a little taller at normal size
  \right|{#2} % this is the delimiter
  }}

%% Commands and operators
\DeclareMathOperator{\ann}{ann}
\DeclareMathOperator{\id}{id}
\DeclareMathOperator{\im}{im}
\DeclareMathOperator{\lcm}{lcm}
\DeclareMathOperator{\nil}{nil}
\DeclareMathOperator{\rad}{rad}

\newcommand{\CC}{\mathbf{C}}
\newcommand{\NN}{\mathbf{N}}
\newcommand{\QQ}{\mathbf{Q}}
\newcommand{\RR}{\mathbf{R}}
\newcommand{\ZZ}{\mathbf{Z}}

%% Fields
\DeclareMathOperator{\FF}{\mathbf{F}}

\begin{document}
\let\setminus\relax
\let\phi\relax
\let\epsilon\relax
\newcommand\setminus{\smallsetminus}
\newcommand\phi{\varphi}
\newcommand\epsilon{\varepsilon}

\frontmatter
\aliaspagestyle{title}{empty}
\pagestyle{title}
\author{\href{mailto:\authoremail}{\documentauthor}}
\title{\documenttitle}
\date{\today}
\maketitle
\cleartooddpage

\makeoddhead{headings}
        {\small{\MakeUppercase{\itshape\documentauthor}}}
        {}
        {\small{\MakeUppercase{\itshape\exercisename}}}
\makeoddfoot{headings}{{\itshape\documenttitle}}
                      {}
                      {\thepage}
\makeevenhead{headings}
        {\small{\MakeUppercase{\itshape\documentauthor}}}
        {}
        {\small{\MakeUppercase{\itshape\exercisename}}}
\makeevenfoot{headings}{{\itshape\documenttitle}}
                      {}
                      {\thepage}
\makeheadrule{headings}{\textwidth}{.25pt}
% \makerunningwidth{headings}{1.15\textwidth}
\pagestyle{headings}

\mainmatter
\begin{problem}
Let $\mathfrak{a}$ be an $R$-ideal and $M$ a finite
$R$-module. Show that
\[
\sqrt{\ann(M/\mathfrak{a}M)}=\sqrt{\ann(M)+\mathfrak{a}}.
\]
\end{problem}
\begin{proof}

\end{proof}
\newpage
\begin{problem}
Let $R$ be a local ring and $M,N$ finite $R$-modules. Show that
$M\otimes_R N=0$ if and only if $M=0$ or $N=0$.
\end{problem}
\begin{proof}
\end{proof}
\newpage
\begin{problem}
Show that $R^n\cong R^m$ if and only if $n=m$.
\end{problem}
\begin{proof}
\end{proof}
\newpage
\begin{problem}
Prove 2.7.
\end{problem}
\begin{proof}
\end{proof}
\newpage
\begin{problem}
Prove 2.8.
\end{problem}
\begin{proof}
\end{proof}
\newpage
\begin{problem}
Prove 2.9.
\end{problem}
\begin{proof}
\end{proof}
\newpage
\begin{problem}
Prove 2.10.
\end{problem}
\begin{proof}
\end{proof}

%%% Local Variables:
%%% mode: latex
%%% TeX-master: "../MA557-HW-Current"
%%% End:

\end{document}

%%% Local Variables:
%%% mode: latex
%%% TeX-master: t
%%% End:
