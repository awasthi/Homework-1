\def\documentauthor{Carlos Salinas}
\def\documenttitle{MA557 Problem Set \hwnum}
\def\hwnum{3}
\def\shorttitle{MA557 PSet \hwnum}
\def\coursename{MA557}
\def\documentsubject{commutative algebra i}
\def\authoremail{salinac@purdue.edu}

\documentclass[article,oneside,10pt]{memoir}
\usepackage{geometry}
\usepackage[dvipsnames]{xcolor}
\usepackage[
    breaklinks,
    bookmarks=true,
    colorlinks=true,
    pageanchor=false,
    linkcolor=black,
    anchorcolor=black,
    citecolor=black,
    filecolor=black,
    menucolor=black,
    runcolor=black,
    urlcolor=black,
    hyperindex=false,
    hyperfootnotes=true,
    pdftitle={\shorttitle},
    pdfauthor={\documentauthor},
    pdfkeywords={\documentsubject},
    pdfsubject={\coursename}
    ]{hyperref}

\usepackage{graphicx}
\graphicspath{{figures/}}

% Misc
\usepackage{microtype}
\usepackage{multicol}
\usepackage[inline]{enumitem}
\usepackage{listings}
\usepackage{mleftright}
\mleftright

%% Math
\usepackage{amsthm}
\usepackage{amssymb}
\usepackage{mathtools}
\usepackage[mathcal]{euscript}
\usepackage{mathrsfs}

%% PDFTeX specific
\usepackage{iftex}
\ifPDFTeX
\usepackage{cmap}
\usepackage[T2A,T1]{fontenc}
\usepackage[utf8]{inputenc}
\usepackage[french,german,russian,spanish,english]{babel}
\babeltags{fr=french,
           de=german,
           ru=russian,
           es=spanish,
           en=english}
\def\spanishoptions{mexico}
\usepackage{CJKutf8}
\newcommand{\textha}[1]{\begin{CJK}{UTF8}{mj}#1\end{CJK}}
\newcommand{\textni}[1]{\begin{CJK}{UTF8}{min}#1\end{CJK}}
\newcommand{\textzh}[1]{\begin{CJK}{UTF8}{bsmi}#1\end{CJK}}
\fi

%% XeTeX specific
% \ifXeTeX
% \usepackage{unicode-math}

% \setmainfont[Ligatures=TeX]{Latin Modern Roman}
% \setsansfont[Ligatures=TeX]{Latin Modern Sans}
% \setmonofont{Latin Modern Mono}
% \setmathfont{Latin Modern Math}
% \fi

%% Theorems and definitions
\theoremstyle{plain}
\newtheorem{theorem}{Theorem}
\newtheorem{proposition}[theorem]{Proposition}
\newtheorem{corollary}[theorem]{Corollary}
\newtheorem{claim}[theorem]{Claim}
\newtheorem{lemma}[theorem]{Lemma}
\newtheorem{axiom}[theorem]{Axiom}

\newtheorem*{corollary*}{Corollary}
\newtheorem*{claim*}{Claim}
\newtheorem*{lemma*}{Lemma}
\newtheorem*{proposition*}{Proposition}
\newtheorem*{theorem*}{Theorem}

\theoremstyle{definition}
\newtheorem{definition}{Definition}
\newtheorem{example}{Examples}
\newtheorem{examples}[example]{Examples}
% \newtheorem{exercise}{Exercise}[section]
% \newtheorem{problem}[exercise]{Problem}

\counterwithout{section}{chapter}
\usepackage[explicit]{titlesec}
\titleformat{\section}{\normalfont\Large\bfseries\scshape}{}{0em}{#1}
\newenvironment{problem}[1][]% environment name
{% begin code
  \par\vspace{\baselineskip}\noindent
  \ifx &#1&%
  \section{Problem~\hwnum.\thesection}
  \global\def\exercisename{Problem \hwnum.\thesection}%
  \else
  \section{Problem~\hwnum.\thesection~(#1)}
  \global\def\exercisename{Problem \hwnum.\thesection(#1)}
  \fi
  \par\vspace{\baselineskip}%
  \noindent\ignorespaces
}%
{% end code
  \par\vspace{\baselineskip}%
  \noindent\ignorespacesafterend
}

\newtheorem*{definition*}{Definition}
\newtheorem*{example*}{Examples}
\newtheorem*{examples*}{Examples}
\newtheorem*{exercise*}{Exercise}
\newtheorem*{problem*}{Problem}

\theoremstyle{remark}
\newtheorem{remark}{Remark}
\newtheorem{remarks}[remark]{Remarks}
\newtheorem{observation}[remark]{Observation}
\newtheorem{observations}[remark]{Observations}

\newtheorem*{remark*}{**Remark**}
\newtheorem*{remarks*}{**Remarks**}
\newtheorem*{observation*}{**Observation**}
\newtheorem*{observations*}{**Observations**}

%% Redefinitions & commands
\newcommand\restr[2]{{% we make the whole thing an ordinary symbol
  \left.\kern-\nulldelimiterspace % automatically resize the bar with \right
  {#1} % the function
  % \vphantom{\big|} % pretend it's a little taller at normal size
  \right|{#2} % this is the delimiter
  }}

\ifPDFTeX
\newcommand{\nsubset}{\ensuremath{\not\subset}}
\newcommand{\hooklongrightarrow}{\lhook\joinrel\longrightarrow}
\newcommand{\twoheadlongrightarrow}{\relbar\joinrel\twoheadrightarrow}

\renewcommand\qedsymbol{\ensuremath{\null\hfill\blacksquare}}

%% upint and loint
\def\upint{\mathchoice%
    {\mkern13mu\overline{\vphantom{\intop}\mkern7mu}\mkern-20mu}%
    {\mkern7mu\overline{\vphantom{\intop}\mkern7mu}\mkern-14mu}%
    {\mkern7mu\overline{\vphantom{\intop}\mkern7mu}\mkern-14mu}%
    {\mkern7mu\overline{\vphantom{\intop}\mkern7mu}\mkern-14mu}%
  \int}
\def\lowint{\mkern3mu\underline{\vphantom{\intop}\mkern7mu}\mkern-10mu\int}
\fi

% \ifXeTeX
% %% Patch arrows for XeTeX
% \usepackage{etoolbox}
% \makeatletter
% \patchcmd{\arrowfill@}{-7mu}{-14mu}{}{}
% \patchcmd{\arrowfill@}{-7mu}{-14mu}{}{}
% \patchcmd{\arrowfill@}{-2mu}{-4mu}{}{}
% \patchcmd{\arrowfill@}{-2mu}{-4mu}{}{}
% \makeatother

% \renewcommand\qedsymbol{\ensuremath{\null\hfill\QED}}
% \fi

%% Commands and operators
\DeclareMathOperator{\ann}{ann}
\DeclareMathOperator{\coker}{coker}
\DeclareMathOperator{\id}{id}
\DeclareMathOperator{\im}{im}
\DeclareMathOperator{\lcm}{lcm}
\DeclareMathOperator{\nil}{nil}
\DeclareMathOperator{\rad}{rad}

\newcommand{\CC}{\mathbf{C}}
\newcommand{\NN}{\mathbf{N}}
\newcommand{\QQ}{\mathbf{Q}}
\newcommand{\RR}{\mathbf{R}}
\newcommand{\ZZ}{\mathbf{Z}}

\begin{document}
% Renewcommands
% \renewcommand\complement{\smallsetminus}
\renewcommand\setminus{\smallsetminus}
\renewcommand\phi{\varphi}
\renewcommand\epsilon{\varepsilon}

\frontmatter
\aliaspagestyle{title}{empty}
\pagestyle{title}
\author{\href{mailto:\authoremail}{\documentauthor}}
\title{\documenttitle}
\date{\today}
\maketitle
\cleartooddpage

\makeoddhead{headings}
        {\small{\MakeUppercase{\itshape\documentauthor}}}
        {}
        {\small{\MakeUppercase{\itshape\exercisename}}}
\makeoddfoot{headings}{{\itshape\documenttitle}}
                      {}
                      {\thepage}
\makeevenhead{headings}
        {\small{\MakeUppercase{\itshape\documentauthor}}}
        {}
        {\small{\MakeUppercase{\itshape\exercisename}}}
\makeevenfoot{headings}{{\itshape\documenttitle}}
                      {}
                      {\thepage}
\makeheadrule{headings}{\textwidth}{.25pt}
% \makerunningwidth{headings}{1.15\textwidth}
\pagestyle{headings}

\mainmatter
\begin{problem}
Let $R$ be a domain and $\Gamma$ the set of all principal ideals
in $R$. Show that $R$ is a unique factorization domain if and
only if $\Gamma$ satisfies the ascending chain condition and
every irreducible element of $R$ is prime.
\end{problem}
\begin{proof}
\end{proof}
\newpage
\begin{problem}
Let $M$ be an Artinian $R$-module. Show that every injective
$R$-lnear map $\phi\colon M\to M$ is an isomorphism.
\end{problem}
\begin{proof}
\end{proof}
\newpage
\begin{problem}
Let $M$ be a finitely generated Artinian module. Show that $M$ is
Noetherian.
\end{problem}
\begin{proof}
\end{proof}
\newpage
\begin{problem}
Let $R$ be a ring that is Artinian or Noetherian, and $x\in
R$. Show that for some $n>0$, the image of $x$ in $R/(0:x)^n$ is
a nonzero-divisor on that ring.
\end{problem}
\begin{proof}
\end{proof}
\newpage
\begin{problem}
Let $R$ be an Artinian ring. Show that $R\cong
R_1\times\cdots\times R_n$ with $R_i$ Artinian local rings.
\end{problem}
\begin{proof}
\end{proof}
\newpage
\begin{problem}
Let $R$ be an Artinian ring all of whose maximal ideals are
principal. Show that every ideal in $R$ is principal.
\end{problem}
\begin{proof}
\end{proof}
\newpage
\begin{problem}
Prove 2.12.
\end{problem}
\begin{proof}
Recall the statement of
\begin{theorem*}[2.12]
\begin{enumerate}[noitemsep,label=(\alph*)]
\item The following are equivalent:
\begin{enumerate}[noitemsep,label=(\arabic*)]
\item $\displaystyle 0\xrightarrow{}
  M'\xrightarrow{\phi}M\xrightarrow{\psi}M''$
  is exact
\item $\displaystyle
  0\xrightarrow{}\hom(N,M')\xrightarrow{\hom(N,\phi)}\hom(N,M)\xrightarrow{\hom(N,\psi)}\hom(N,M'')$
  is exact for all modules $N$.
\end{enumerate}
\item The following are equivalent:
\begin{enumerate}[noitemsep,label=(\arabic*)]
\item $\displaystyle
  M'\xrightarrow{\phi}M\xrightarrow{\psi}M''\xrightarrow{}
  0$ is exact.
\item $\displaystyle
  0\xrightarrow{}\hom(M'',N)\xrightarrow{\hom(\psi,N)}\hom(M,N)\xrightarrow{\hom(\psi,N)}\hom(M',N)$
  is exact for all modules $N$.
\item $\displaystyle M'\otimes N\xrightarrow{\phi\otimes
    N}M\otimes N\xrightarrow{\psi\otimes
    N}M''\otimes N\xrightarrow{} 0$ is exact
  for all modules $N$.
\end{enumerate}
\end{enumerate}
\end{theorem*}
\end{proof}

%%% Local Variables:
%%% mode: latex
%%% TeX-master: "../MA557-HW-Current"
%%% End:

\end{document}

%%% Local Variables:
%%% mode: latex
%%% TeX-master: t
%%% End:
