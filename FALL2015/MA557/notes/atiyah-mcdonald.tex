\chapter{Commutative Algebra: Atiyah and McDonald}
\section{Rings and Ideals}
\subsection{Rings and ring homomorphisms}
A \emph{ring} $A$ is a set with two binary operations (addition
and multiplication) such that
\begin{enumerate}[noitemsep,label=(\arabic*)]
\item $A$ is an abelian group with respect to addition (so that
  $A$ has a zero element, denoted by $0$, and every $x\in A$ has
  an (additive) inverse, $-x$).
\item Multiplication is associative ($(xy)z=x(yz)$) and
  distributive over addition ($(x(x+z)=xy+xz$,
  $(y+z)x=yx+zx$). We shall consider only rigs which are
  \emph{commutative}:
\item $xy=yx$ for all $x,y\in A$, and have an \emph{identity
    element} (denoted by $1$):
\item $\exists 1\in A$ such that $x1=1x=x$ for all $x\in A$. The
  identity element is then unique.
\end{enumerate}

A \emph{ring homomorphism} is a mapping $f$ of a ring $A$
into a ring $B$ such that
\begin{enumerate}[noitemsep,label=(\roman*)]
\item $f(x+y)=f(x)+f(y)$ (so that $f$ is a homomorphism of
  abelian groups, and therefore also $f(x-y)=f(x)-f(y)$,
  $f(-x)=-f(x)$, $f(0)=0$),
\item $f(xy)=f(x)f(y)$,
\item $f(1)=1$.
\end{enumerate}
In other words, $f$ respects addition, multiplication and the
identity element.

A subset $S$ of a ring $A$ is a \emph{subring} of $A$ if $S$ is
closed under addition and multiplication and contains the
identity element of $A$. The identity mapping of $S$ into $A$ is
then a ring homomorphism.

If $f\colon A\to B$, $g\colon B\to C$ are ring homomorphisms so
is their composition $g\circ f\colon A\to C$.
\subsection{Ideals. Quotient rings}
An \emph{ideal} $\mathfrak{a}$ of a ring $A$ is a subset of $A$
which is an additive subgroup and is such that
$A\mathfrak{a}\subset\mathfrak{a}$ (i.e., $x\in A$ and
$y\in\mathfrak{a}$). The quotient group $A/\mathfrak{a}$
inherits a uniquely defined multiplication from $A$ which makes
it into a ring, called the \emph{quotient ring} (or residue-class
ring) $A/\mathfrak{a}$. The elements of $A/\mathfrak{a}$ are the
cosets of $\mathfrak{a}$ in $A$, and the mapping $\phi\colon
A\to A/\mathfrak{a}$ which maps each $x\in A$ to its coset
$x+\mathfrak{a}$ is a surjective ring homomorphism.
\begin{proposition}
There is a $1$-to-$1$ order-preserving correspondence between the
ideals $\mathfrak{b}$ of $A$ which contains $\mathfrak{a}$, and
the ideals $\bar\mathfrak{b}$ of $A/\mathfrak{a}$, given by
$\mathfrak{b}=\phi^{-1}\left(\bar\mathfrak{b}\right)$.
\end{proposition}
If $f\colon A\to B$ is any ring homomorphism, the \emph{kernel}
of $f(=f^{-1}(0))$ is an ideal $\mathfrak{a}$ of $A$, and the
\emph{image} of $f(=(f(A))$ is a subring $C$ of $B$; and $f$
induces a ring isomorphism $A/\mathfrak{a}\cong C$.

We shall sometimes use the notation $x\equiv
y\pmod{\mathfrak{a}}$; this means that $x-y\in\mathfrak{a}$.
\subsection{Zero-divisors. Nilpotent elements. Units}
A \emph{zero-divisor} in a ring $A$ is an element $x$ which
``divides $0$'', i.e., for which there exists $y\neq 0$ in $A$
such that $xy=0$. A ring with no zero-divisors $\neq 0$ (and in
which $1\neq 0$) is called an \emph{integral domain}. For
example, $\ZZ$ and $k[x_1,...,x_n]$ ($k$ a field, $x_i$
indeterminates) are integral domains.

An element $x\in A$ is \emph{nilpotent} if $x^n=0$ for some
$n>0$. A nilpotent element is a zero-divisor (unless $A\neq 0$),
but not conversely (in general).

A \emph{unit} in $A$ is an element $x$ which ``divides $1$'',
i.e., an element $x$ such that $xy=1$ for some $y\in A$. The
element $y$ is then uniquely determined by $x$, and is written
$x^{-1}$. The units in $A$ form a (multiplicative) abelian group.

The multiples $ax$ of an element $x\in A$ from a
\emph{principal} ideal, denoted by $(x)$ or $Ax$. $x$ is a unit
$\iff$ $(x)=A$. The \emph{zero} ideal $(0)$ is usually denoted by
$0$.

A \emph{field} is a ring $A$ in which $1\neq 0$ and every nonzero
element is a unit. Every field is an integral domain (but not
conversely: $\ZZ$ is not a field).

\begin{proposition}
Let $A$ be a ring $\neq 0$. Then the following are equivalent:
\begin{enumerate}[noitemsep,label=(\roman*)]
\item $A$ is a field;
\item the only ideals in $A$ are $0$ and $(1)$;
\item every homomorphism of $A$ into a nonzero ring $B$ is
  injective.
\end{enumerate}
\end{proposition}
\begin{proof}
(i) $\implies$ (ii). Let $\mathfrak{a}\neq 0$ be an ideal in
$A$. Then $\mathfrak{a}$ contains a nonzero element $x$, $x$ is a
unit, hence $\mathfrak{a}\supset(x)=A$, hence $\mathfrak{a}=A$.

(ii) $\implies$ (iii). Let $\phi\colon A\to B$ be a ring
homomorphism. Then $\ker \phi$ is an ideal $\neq (1)$ in $A$,
hence $\ker\phi=0$, hence $\phi$ is injective.

(iii) $\implies$ (i). Let $x$ be an element of $A$ which is not a
unit. Then $(x)\neq(1)$, hence $B=A/(x)$ is not the zero ring. Let
$\phi\colon A\to B$ be the natural homomorphism of $A$ onto
$B$, with kernel $(x)$. By hypothesis, $\phi$ is injective,
hence $x=0$.
\end{proof}
\subsection{Prime ideals and maximal ideals}
An ideal $\mathfrak{p}$ in $A$ is \emph{prime} if
$\mathfrak{p}\neq(1)$ and if $xy\in\mathfrak{p}$ $\implies$
$x\in\mathfrak{p}$ or $y\in\mathfrak{p}$.

An ideal $\mathfrak{m}$ in $A$ is \emph{maximal} if
$\mathfrak{m}\neq(1)$ and if there is no ideal $\mathfrak{a}$ such
that $\mathfrak{a}\subsetneq\mathfrak{a}\subsetneq A$. Equivalently
\[
\text{$\mathfrak{p}$ is prime}\iff\text{$A/\mathfrak{p}$ is an
  integral domain;}
\]
\[
\text{$\mathfrak{m}$ is maximal}\iff\text{$A/\mathfrak{m}$ is a field.}
\]
Hence a maximal ideal is prime (but not conversely, in
general). The zero ideal is prime $\iff$ $A$ is an integral
domain.

If $f\colon A\to B$ is a ring homomorphism and $\mathfrak{q}$ is
a prime ideal in $B$, then $f^{-1}(\mathfrak{q})$ is a prime
ideal in $A$, for $A/f^{-1}(\mathfrak{q})$ is isomorphic to a
subring of $B/\mathfrak{q}$ and hence has a no zero-divisor $\neq
0$. But if $\mathfrak{n}$ is a maximal ideal of $B$ is not
necessarily true that $f^{-1}(\mathfrak{n})$ is maximal in $A$;
all we can say for sure is that it is prime. (Example: $A=\ZZ$,
$B=\QQ$, $\mathfrak{n}=0$.)
\begin{theorem}
Every ring $A\neq 0$ has at least one maximal ideal.
\end{theorem}
\begin{proof}
This is a standard application of Zorn's lemma. Let $\Sigma$ be
the set of all ideals $\neq(1)$ in $A$.  Order $\Sigma$ by
inclusion. $\Sigma$ is not empty, since $0\in\Sigma$. To apply
Zorn's lemma we must show that every chain in $\Sigma$ has an
upper bound in $\Sigma$; let then
$\left(\mathfrak{a}_\alpha\right)$ be a chain of ideals in
$\Sigma$, so that for each pair of indices $\alpha,\beta$ we have
either $\mathfrak{a}_\alpha\subset\mathfrak{a}_\beta$ or
$\mathfrak{a}_\beta\subset\mathfrak{a}_\alpha$. Let
$\mathfrak{a}=\bigcup_\alpha\mathfrak{a}_\alpha$. Then
$\mathfrak{a}$ is an ideal and $1\notin\mathfrak{a}$. Hence
$\mathfrak{a}\in\Sigma$, and $\mathfrak{a}$ is an upper bound of
the chain. Hence by Zorn's lemma $\Sigma$ has a maximal element.
\end{proof}
\begin{corollary}
If $\mathfrak{a}\neq(1)$ is an ideal of $A$, there exists a
maximal ideal of $A$ containing $\mathfrak{a}$.
\end{corollary}
\begin{proof}
Apply (1.3) to $A/\mathfrak{a}$ bearing in mind
(1.1). Alternatively, modify the proof of (1.3).
\end{proof}
\begin{corollary}
Every nonunit of of $A$ is contained in a maximal ideal.
\end{corollary}
\begin{remarks*}
\begin{enumerate}[noitemsep,label=(\arabic*)]
\item If $A$ is Noetherian we can avoid the use f Zorn's lemma:
  the set of all ideals $\neq (1)$ has a maximal element.
\item There exists rings with exactly one maximal ideal, for
  example fields. A ring $A$ with exactly one maximal ideal
  $\mathfrak{m}$ is called a \emph{local ring}. The field
  $k=A/\mathfrak{m}$ is called the \emph{residue field} of $A$.
\end{enumerate}
\end{remarks*}
\begin{proposition}
\begin{enumerate}[noitemsep,label=(\roman*)]
\item Let $A$ be a ring and $\mathfrak{m}\neq (1)$ be an ideal of
  $A$ such that every $x\in A-\mathfrak{m}$ is a unit in
  $A$. Then $A$ is a local ring and $\mathfrak{m}$ its maximal
  ideal.
\item Let $A$ be a ring and $\mathfrak{m}$ a maximal ideal of
  $A$, such that every element of $1+\mathfrak{m}$ (i.e., every
  $1+x$, where $x\in\mathfrak{m}$) is a unit in $A$. Then $A$ is
  a local ring.
\end{enumerate}
\end{proposition}
\begin{proof}
(i) Every ideal $\neq(1)$ consists of nonunits, hence is
contained in $\mathfrak{m}$. Hence $\mathfrak{m}$ is the only
maximal ideal of $A$.

(ii) Let $x\in A-\mathfrak{m}$. Since $\mathfrak{m}$ is maximal,
then the ideal generated by $x$ and $\mathfrak{m}$ is $(1)$,
hence there exists $y\in A$ and $\t\in\mathfrak{m}$ such that
$xy+t=1$; hence $xy=1-t$ belongs to $1+\mathfrak{m}$ and
therefore s a unit. Now use (i).
\end{proof}
A ring with only a finite number of maximal ideals is called
\emph{semilocal}.
\begin{examples}
\begin{enumerate}[noitemsep,label=(\arabic*)]
\item $A=k[x_1,...,x_n]$, $k$ a field. Let $f\in A$ be an
  irreducible polynomial. By unique factorization, the ideal
  $(f)$ is prime.
\item $A=\ZZ$. Every ideal in $\ZZ$ is of the form $(m)$ for some
  $m\geq 0$. The ideal $(m)$ is prime $\iff$ $m=0$ or a prime
  number. All ideals $(p)$, where $p$ is a prime number, are
  maximal: $\ZZ/(p)$ is the field of $p$ elements.
\item A \emph{principal ideal domain} is an integral domain in
  which every ideal is principal. In such a ring every nonzero
  ideal is maximal. For if $(x)\neq 0$ is a prime ideal and
  $(y)\supset (x)$, we have $x\in (y)$, say $x=yz$, so that
  $yz\in (x)$ and $y\notin (x)$, hence $z\in (x)$, say
  $z=tx$. Then $x=yz=ytx$, so that $yt=1$ and therefore $(y)=1$.
\end{enumerate}
\end{examples}
\subsection{Nilradical and Jacobson radical}
\begin{proposition}
The set $\mathfrak{N}$ of all nilpotent elements in a ring $A$ is
an ideal, and $A/\mathfrak{N}$ has no nilpotent element $\neq 0$.
\end{proposition}
\begin{proof}
If $x\in\mathfrak{N}$, clearly $ax\in\mathfrak{N}$ for all $a\in
A$. Let $x,y\in\mathfrak{N}$: say $x^m=0$, $y^n=0$. By the
binomial theorem, $(x+y)^{m+n-1}$ is a sum of integer multiples
of products $x^ry^s$, where $r+s=m+n-1$ we cannot have both $r<m$
and $s<n$, hence each of these products vanishes and therefore
$(x+y)^{m+n-1}=0$. Hence $x+y\in\mathfrak{N}$ and therefore
$\mathfrak{N}$ is an ideal.

Let $\bar x\in\mathfrak{N}$ be represented by $x\in A$. Then
$\bar x^n$ is prepresented by $x^n$, so that $\bar x^n=0$ implies
$x^n\in\mathfrak{N}$ implies $(x^n)^k=0$ for some $k>0$ implies
$x\in\mathfrak{N}$ implies $\bar x=0$.
\end{proof}
The ideal $\mathfrak{N}$ is called the \emph{nilradical} of
$A$. The following proposition gives an alternative definition of
$\mathfrak{N}$:
\begin{proposition}
The nilradical of $A$ is the intersection of all the prime ideals
of $A$.
\end{proposition}
\begin{proof}
Let $\mathfrak{N}'$ denote the intersection of all the prime
ideals of $A$. If $f\in A$ is nilpotent and if $\mathfrak{p}$ is
a prime ideal, then $f^n=0\in\mathfrak{p}$ for some $n>0$, hence
$f\in\mathfrak{p}$ (because $\mathfrak{p}$ is prime). Hence
$f\in\mathfrak{N}'$.

Conversely, suppose that $f$ is not nilpotent. Let $\Sigma$ be
the set of ideals $\mathfrak{a}$ with the property $n>0\implies
f^n\notin\mathfrak{a}$. Then $\Sigma$ is not empty because
$0\in\Sigma$. As in (1.3) Zorn's lemma can be applied to the set
$\Sigma$, ordered by inclusion, and therefore $\Sigma$ has a
maximal element. Let $\mathfrak{p}$ be a maximal element of
$\Sigma$. We shall show that $\mathfrak{p}$ is a prime ideal. Let
$x,y\notin\mathfrak{p}$. Then the ideals $\mathfrak{p}+(x)$,
$\mathfrak{p}+(y)$ strictly contain $\mathfrak{p}$ and therefore
do not belong to $\Sigma$; hence
\[
f^m\in\mathfrak{p}+(x),\qquad f^n\in\mathfrak{p}+(y)
\]
for some $m,n$. It follows that $f^{m+n}\in\mathfrak{p}+(xy)$,
hence the ideal $\mathfrak{p}+(xy)$ is not in $\Sigma$ and
therefore $xy\notin\mathfrak{p}$. Hence we have the  prime  ideal
$\mathfrak{p}$ such that $f\notin\mathfrak{p}$, so that
$f\notin\mathfrak{N}'$.
\end{proof}
The \emph{Jacobson radical} $\mathfrak{R}$ of $A$ is defined to
be the intersection of all maximal ideals of $A$. It can be
characterized as follows:
\begin{proposition}
$x\in\mathfrak{R}$ if and only if $1-xy$ is a unit for all $y\in
A$.
\end{proposition}
\begin{proof}
$\implies$: Suppose $1-xy$ is not a unit. By (1.5) it belongs to
some maximal ideal $\mathfrak{m}$; but
$x\in\mathfrak{R}\subset\mathfrak{m}$, hence $xy\in\mathfrak{m}$
and therefore $1\in\mathfrak{m}$ which is absurd.

$\impliedby$: Suppose $x\notin\mathfrak{m}$ for some maximal
ideal $\mathfrak{m}$. Then $\mathfrak{m}$ and $x$ generate the
unit ideal $(1)$, so that we have $u+xy=1$ for some
$u\in\mathfrak{m}$ and some $y\in A$. Hence $1-xy\in\mathfrak{m}$
and is therefore not a unit.
\end{proof}
\subsection{Operations on Ideals}
Two ideals $\mathfrak{a}$ and $\mathfrak{b}$ are said to be
\emph{coprime} (or \emph{comaximal}) if
$\mathfrak{a}+\mathfrak{b}=(1)$. Thus for coprime ideals we have
$\mathfrak{a}\cap\mathfrak{b}=\mathfrak{a}\mathfrak{b}$.

Let $A_1,...,A_n$ be rings. Their \emph{direct product}
\[
A=\prod_{i=1}^n A_i
\]
is the set of all sequences $x=(x_1,...,x_n)$ with $x_i\in A_i$
($1\leq i\leq n$) and componentwise addition and multiplication.

Let $A$ be a ring and $\mathfrak{a}_1,...,\mathfrak{a}_n$ ideals
of $A$. Define a homomorphism
\[
\phi\colon A\longrightarrow\prod_{i=1}^n\frac{A_i}{\mahfrak{a}_i}
\]
by the rule $\phi(x)=(x+\mathfrak{a}_1,...,x+\mathfrak{a_n})$.
\begin{proposition}
\begin{enumerate}[noitemsep,label=(\roman*)]
\item If $\mathfrak{a}_i$, $\mathfrak{a}_j$ are coprime whenever
  $i\neq j$, then $\prod\mathfrak{a}_i=\bigcap\mathfrak{a}_i$.
\item $\phi$ is injective $\iff$ $\mathfrak{a}_i$,
  $\mathfrak{a}_j$ are coprime whenever $i\neq j$.
\item $\phi$ is injective $\iff$ $\bigcap\mathfrak{a}_i=(0)$.
\end{enumerate}
\end{proposition}
\begin{proof}
(i) By induction on $n$. The case $n=2$ is dealt with
above. Suppose $n>2$ and the result true for
$\mathfrak{a}_1,...,\mathfrak{a}_{n-1}$, and let
$\mathfrak{b}=\prod_{i=1}^{n-1}\mathfrak{a}_i=\bigcap_{i=1}^{n-1}\mathfrak{a}_i$. Since
$\mathfrak{a}_i+\mahfrak{a}_j=(1)$ ($1\leq i\leq n-1$) we have
equations $x_i+y_i=1$ ($x_i\in\mathfrak{a}_i$,
$y_i\in\mathfrak{a}_n$) and therefore
\[
\prod_{i=1}^{n-1}x_i=\prod_{i=1}^{n-1}(1-y_i)\equiv 1\pmod{\mathfrak{a}_n}.
\]
Hence
\[
\prod_{i=1}^n\mathfrak{a}_i=\mahfrak{b}\mathfrak{a}_n=\mathfrak{b}\cap\mathfrak{a}_n=\bigcap_{i=1}^n\mathfrak{a}_i.
\]

(ii) $\implies$: Let us show for example that $\mathfrak{a}_1$
and $\mathfrak{a}_2$ are coprime. There exists $x\in A$ such that
$\phi(x)=(1,0,...,0)$; hence $x\equiv 1\pmod{\mathfrak{a}_1}$ and
$x\equiv 0\pmod{\mathfrak{a}_2}$, so that
\[
1=(1-x)+x\in\mathfrak{a}_1+\mathfrak{a}_2.
\]

$\impliedby$: It is enough to show, for example, that there is an
element $x\in A$ such that $\phi(x)=(1,0,...,0)$. Since
$\mathfrak{a}_1+\mathfrak{a}_2=(1)$ ($i>1$) we have equation
$u_i+v_i=1$ ($u_i\in\mathfrak{a}_1$,
$v_i\in\mathfrak{a}_i$). Take $x=\prod_{i=2}^n v_i$, then
$x=\prod(1-u_i)\equiv 1\pmod{\mathfrak{a}_i}$, and $x\equiv
0\pmod{\mathfrak{a}_i}$, $i>1$. Hence $\phi(x)=(1,0,...,0)$ as
required.

(iii) Clear, since $\bigcap\mathfrak{a}_i$ is the kernel of
$\phi$.
\end{proof}
\begin{proposition}
\begin{enumerate}[noitemsep,label=(\roman*)]
\item Let $\mathfrak{p}_1,...,\mathfrak{p}_n$ be prime ideals and
  let $\mathfrak{a}$ be an ideal contained in
  $\bigcup_{i=1}^n\mathfrak{p}_i$. Then
  $\mathfrak{a}\subset\mathfrak{p}_i$ for some $i$.
\item Let $\mathfrak{a}_1,...,\mathfrak{a}_n$ be ideals and let
  $\mathfrak{p}$ be a prime ideal containing
  $\bigcap_{i=1}^n\mathfrak{a}_i$. Then
  $\mathfrak{p}\supset\mathfrak{a}_i$ for some $i$.
\end{enumerate}
\end{proposition}
\begin{proof}
(i) Is proved by induction on $n$ in the form
\[
\text{$\mathfrak{a}\subsetneq\mathfrak{p}_i$ $(1\leq i\leq n)$}\implies\text{$\mathfrak{a}\nsubset\bigcup_{i=1}^n\mathfrak{p}_i$}.
\]
It is certainly true for $n=1$. If $n>1$ and the result is true
for $n-1$, then for each $i$ there exists $x_i\in\mathfrak{a}$
such that $x_i\in\mathfrak{p}_i$ for all $i$. Consider the element
\[
y=\sum_{i=1}^nx_1\cdots x_{i-1}x_{i+1}\cdots x_n;
\]
we have $y\in\mathfrak{a}$ and $y\notin\mathfrak{p}_i$ ($1\leq
i\leq n$). Hence $\mathfrak{a}\nsubset\bigcup_{i=1}^n\mathfrak{p}_i$.

(ii)
\end{proof}

%%% Local Variables:
%%% mode: latex
%%% TeX-master: "../../FALL15-Notes"
%%% End:
