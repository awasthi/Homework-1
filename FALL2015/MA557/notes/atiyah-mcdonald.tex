\chapter{Commutative Algebra: Atiyah and McDonald}
\section{Rings and Ideals}
\subsection{Rings and ring homomorphisms}
A \emph{ring} $A$ is a set with two binary operations (addition
and multiplication) such that
\begin{enumerate}[noitemsep,label=(\arabic*)]
\item $A$ is an abelian group with respect to addition (so that
  $A$ has a zero element, denoted by $0$, and every $x\in A$ has
  an (additive) inverse, $-x$).
\item Multiplication is associative ($(xy)z=x(yz)$) and
  distributive over addition ($(x(x+z)=xy+xz$,
  $(y+z)x=yx+zx$). We shall consider only rigs which are
  \emph{commutative}:
\item $xy=yx$ for all $x,y\in A$, and have an \emph{identity
    element} (denoted by $1$):
\item $\exists 1\in A$ such that $x1=1x=x$ for all $x\in A$. The
  identity element is then unique.
\end{enumerate}

A \emph{ring homomorphism} is a mapping $f$ of a ring $A$
into a ring $B$ such that
\begin{enumerate}[noitemsep,label=(\roman*)]
\item $f(x+y)=f(x)+f(y)$ (so that $f$ is a homomorphism of
  abelian groups, and therefore also $f(x-y)=f(x)-f(y)$,
  $f(-x)=-f(x)$, $f(0)=0$),
\item $f(xy)=f(x)f(y)$,
\item $f(1)=1$.
\end{enumerate}
In other words, $f$ respects addition, multiplication and the
identity element.

A subset $S$ of a ring $A$ is a \emph{subring} of $A$ if $S$ is
closed under addition and multiplication and contains the
identity element of $A$. The identity mapping of $S$ into $A$ is
then a ring homomorphism.

If $f\colon A\to B$, $g\colon B\to C$ are ring homomorphisms so
is their composition $g\circ f\colon A\to C$.
\subsection{Ideals. Quotient rings}
An \emph{ideal} $\mathfrak{a}$ of a ring $A$ is a subset of $A$
which is an additive subgroup and is such that
$A\mathfrak{a}\subset\mathfrak{a}$ (i.e., $x\in A$ and
$y\in\mathfrak{a}$). The quotient group $A/\mathfrak{a}$
inherits a uniquely defined multiplication from $A$ which makes
it into a ring, called the \emph{quotient ring} (or residue-class
ring) $A/\mathfrak{a}$. The elements of $A/\mathfrak{a}$ are the
cosets of $\mathfrak{a}$ in $A$, and the mapping $\phi\colon
A\to A/\mathfrak{a}$ which maps each $x\in A$ to its coset
$x+\mathfrak{a}$ is a surjective ring homomorphism.
\begin{proposition}
There is a $1$-to-$1$ order-preserving correspondence between the
ideals $\mathfrak{b}$ of $A$ which contains $\mathfrak{a}$, and
the ideals $\bar\mathfrak{b}$ of $A/\mathfrak{a}$, given by
$\mathfrak{b}=\phi^{-1}\left(\bar\mathfrak{b}\right)$.
\end{proposition}
If $f\colon A\to B$ is any ring homomorphism, the \emph{kernel}
of $f(=f^{-1}(0))$ is an ideal $\mathfrak{a}$ of $A$, and the
\emph{image} of $f(=(f(A))$ is a subring $C$ of $B$; and $f$
induces a ring isomorphism $A/\mathfrak{a}\cong C$.

We shall sometimes use the notation $x\equiv
y\pmod{\mathfrak{a}}$; this means that $x-y\in\mathfrak{a}$.
\subsection{Zero-divisors. Nilpotent elements. Units}
A \emph{zero-divisor} in a ring $A$ is an element $x$ which
``divides $0$'', i.e., for which there exists $y\neq 0$ in $A$
such that $xy=0$. A ring with no zero-divisors $\neq 0$ (and in
which $1\neq 0$) is called an \emph{integral domain}. For
example, $\ZZ$ and $k[x_1,...,x_n]$ ($k$ a field, $x_i$
indeterminates) are integral domains.

An element $x\in A$ is \emph{nilpotent} if $x^n=0$ for some
$n>0$. A nilpotent element is a zero-divisor (unless $A\neq 0$),
but not conversely (in general).

A \emph{unit} in $A$ is an element $x$ which ``divides $1$'',
i.e., an element $x$ such that $xy=1$ for some $y\in A$. The
element $y$ is then uniquely determined by $x$, and is written
$x^{-1}$. The units in $A$ form a (multiplicative) abelian group.

The multiples $ax$ of an element $x\in A$ from a
\emph{principal} ideal, denoted by $(x)$ or $Ax$. $x$ is a unit
$\iff$ $(x)=A$. The \emph{zero} ideal $(0)$ is usually denoted by
$0$.

A \emph{field} is a ring $A$ in which $1\neq 0$ and every nonzero
element is a unit. Every field is an integral domain (but not
conversely: $\ZZ$ is not a field).

\begin{proposition}
Let $A$ be a ring $\neq 0$. Then the following are equivalent:
\begin{enumerate}[noitemsep,label=(\roman*)]
\item $A$ is a field;
\item the only ideals in $A$ are $0$ and $(1)$;
\item every homomorphism of $A$ into a nonzero ring $B$ is
  injective.
\end{enumerate}
\end{proposition}
\begin{proof}
(i) $\implies$ (ii). Let $\mathfrak{a}\neq 0$ be an ideal in
$A$. Then $\mathfrak{a}$ contains a nonzero element $x$, $x$ is a
unit, hence $\mathfrak{a}\supset(x)=A$, hence $\mathfrak{a}=A$.

(ii) $\implies$ (iii). Let $\phi\colon A\to B$ be a ring
homomorphism. Then $\ker \phi$ is an ideal $\neq (1)$ in $A$,
hence $\ker\phi=0$, hence $\phi$ is injective.

(iii) $\implies$ (i). Let $x$ be an element of $A$ which is not a
unit. Then $(x)\neq(1)$, hence $B=A/(x)$ is not the zero ring. Let
$\phi\colon A\to B$ be the natural homomorphism of $A$ onto
$B$, with kernel $(x)$. By hypothesis, $\phi$ is injective,
hence $x=0$.
\end{proof}
\subsection{Prime ideals and maximal ideals}
An ideal $\mathfrak{p}$ in $A$ is \emph{prime} if
$\mathfrak{p}\neq(1)$ and if $xy\in\mathfrak{p}$ $\implies$
$x\in\mathfrak{p}$ or $y\in\mathfrak{p}$.

An ideal $\mathfrak{m}$ in $A$ is \emph{maximal} if
$\mathfrak{m}\neq(1)$ and if there is no ideal $\mathfrak{a}$ such
that $\mathfrak{a}\subsetneq\mathfrak{a}\subsetneq A$. Equivalently
\[
\text{$\mathfrak{p}$ is prime}\iff\text{$A/\mathfrak{p}$ is an
  integral domain;}
\]
\[
\text{$\mathfrak{m}$ is maximal}\iff\text{$A/\mathfrak{m}$ is a field.}
\]
Hence a maximal ideal is prime (but not conversely, in
general). The zero ideal is prime $\iff$ $A$ is an integral
domain.

If $f\colon A\to B$ is a ring homomorphism and $\mathfrak{q}$ is
a prime ideal in $B$, then $f^{-1}(\mathfrak{q})$ is a prime
ideal in $A$, for $A/f^{-1}(\mathfrak{q})$ is isomorphic to a
subring of $B/\mathfrak{q}$ and hence has a no zero-divisor $\neq
0$. But if $\mathfrak{n}$ is a maximal ideal of $B$ is not
necessarily true that $f^{-1}(\mathfrak{n})$ is maximal in $A$;
all we can say for sure is that it is prime. (Example: $A=\ZZ$,
$B=\QQ$, $\mathfrak{n}=0$.)
\begin{theorem}
Every ring $A\neq 0$ has at least one maximal ideal.
\end{theorem}
\begin{proof}
This is a standard application of Zorn's lemma. Let $\Sigma$ be
the set of all ideals $\neq(1)$ in $A$.  Order $\Sigma$ by
inclusion. $\Sigma$ is not empty, since $0\in\Sigma$. To apply
Zorn's lemma we must show that every chain in $\Sigma$ has an
upper bound in $\Sigma$; let then
$\left(\mathfrak{a}_\alpha\right)$ be a chain of ideals in
$\Sigma$, so that for each pair of indices $\alpha,\beta$ we have
either $\mathfrak{a}_\alpha\subset\mathfrak{a}_\beta$ or
$\mathfrak{a}_\beta\subset\mathfrak{a}_\alpha$. Let
$\mathfrak{a}=\bigcup_\alpha\mathfrak{a}_\alpha$. Then
$\mathfrak{a}$ is an ideal and $1\notin\mathfrak{a}$. Hence
$\mathfrak{a}\in\Sigma$, and $\mathfrak{a}$ is an upper bound of
the chain. Hence by Zorn's lemma $\Sigma$ has a maximal element.
\end{proof}
\begin{corollary}
If $\mathfrak{a}\neq(1)$ is an ideal of $A$, there exists a
maximal ideal of $A$ containing $\mathfrak{a}$.
\end{corollary}
\begin{proof}
Apply (1.3) to $A/\mathfrak{a}$ bearing in mind
(1.1). Alternatively, modify the proof of (1.3).
\end{proof}
\begin{corollary}
Every nonunit of of $A$ is contained in a maximal ideal.
\end{corollary}
\begin{remarks*}
\begin{enumerate}[noitemsep,label=(\arabic*)]
\item If $A$ is Noetherian we can avoid the use f Zorn's lemma:
  the set of all ideals $\neq (1)$ has a maximal element.
\item There exists rings with exactly one maximal ideal, for
  example fields. A ring $A$ with exactly one maximal ideal
  $\mathfrak{m}$ is called a \emph{local ring}. The field
  $k=A/\mathfrak{m}$ is called the \emph{residue field} of $A$.
\end{enumerate}
\end{remarks*}
\begin{proposition}
\begin{enumerate}[noitemsep,label=(\roman*)]
\item Let $A$ be a ring and $\mathfrak{m}\neq (1)$ be an ideal of
  $A$ such that every $x\in A-\mathfrak{m}$ is a unit in
  $A$. Then $A$ is a local ring and $\mathfrak{m}$ its maximal
  ideal.
\item Let $A$ be a ring and $\mathfrak{m}$ a maximal ideal of
  $A$, such that every element of $1+\mathfrak{m}$ (i.e., every
  $1+x$, where $x\in\mathfrak{m}$) is a unit in $A$. Then $A$ is
  a local ring.
\end{enumerate}
\end{proposition}
\begin{proof}

\end{proof}

%%% Local Variables:
%%% mode: latex
%%% TeX-master: "../../FALL15-Notes"
%%% End:
