\chapter{Commutative Algebra: Atiyah and McDonald}
\section{Rings and ring homomorphisms}
A \emph{ring} $A$ is a set with two binary operations (addition
and multiplication) such that
\begin{enumerate}[noitemsep,label=(\arabic*)]
\item $A$ is an abelian group with respect to addition (so that
  $A$ has a zero element, denoted by $0$, and every $x\in A$ has
  an (additive) inverse, $-x$).
\item Multiplication is associative ($(xy)z=x(yz)$) and
  distributive over addition ($(x(x+z)=xy+xz$,
  $(y+z)x=yx+zx$). We shall consider only rigs which are
  \emph{commutative}:
\item $xy=yx$ for all $x,y\in A$, and have an \emph{identity
    element} (denoted by $1$):
\item $\exists 1\in A$ such that $x1=1x=x$ for all $x\in A$. The
  identity element is then unique.
\end{enumerate}

A \emph{ring homomorphism} is a mapping $f$ of a ring $A$
into a ring $B$ such that
\begin{enumerate}[noitemsep,label=(\roman*)]
\item $f(x+y)=f(x)+f(y)$ (so that $f$ is a homomorphism of
  abelian groups, and therefore also $f(x-y)=f(x)-f(y)$,
  $f(-x)=-f(x)$, $f(0)=0$),
\item $f(xy)=f(x)f(y)$,
\item $f(1)=1$.
\end{enumerate}

%%% Local Variables:
%%% mode: latex
%%% TeX-master: "../../FALL15-Notes"
%%% End:
