\def\documentauthor{Carlos Salinas}
\def\documenttitle{MA571 Problem Set \hwnum}
\def\hwnum{6}
\def\shorttitle{MA571 PSet \hwnum}
\def\coursename{MA571}
\def\documentsubject{point-set topology}
\def\authoremail{salinac@purdue.edu}

\documentclass[article,oneside,10pt]{memoir}
\usepackage{geometry}
\usepackage[dvipsnames]{xcolor}
\usepackage[
    breaklinks,
    bookmarks=true,
    colorlinks=true,
    pageanchor=false,
    linkcolor=black,
    anchorcolor=black,
    citecolor=black,
    filecolor=black,
    menucolor=black,
    runcolor=black,
    urlcolor=black,
    hyperindex=false,
    hyperfootnotes=true,
    pdftitle={\shorttitle},
    pdfauthor={\documentauthor},
    pdfkeywords={\documentsubject},
    pdfsubject={\coursename}
    ]{hyperref}

\usepackage{graphicx}
\graphicspath{{figures/}}

% Misc
\usepackage{microtype}
\usepackage{multicol}
\usepackage[inline]{enumitem}
\usepackage{listings}
\usepackage{mleftright}
\mleftright

%% Math
\usepackage{amsthm}
\usepackage{amssymb}
\usepackage{mathtools}

%% PDFTeX specific
\usepackage{iftex}
\ifPDFTeX
\usepackage[mathcal]{euscript}
\usepackage{mathrsfs}

\usepackage{cmap}
\usepackage[T2A,T1]{fontenc}
\usepackage[utf8]{inputenc}
\usepackage[french,german,russian,spanish,english]{babel}
\babeltags{fr=french,
           de=german,
           ru=russian,
           es=spanish,
           en=english}
\def\spanishoptions{mexico}
\usepackage{CJKutf8}
\newcommand{\textha}[1]{\begin{CJK}{UTF8}{mj}#1\end{CJK}}
\newcommand{\textni}[1]{\begin{CJK}{UTF8}{min}#1\end{CJK}}
\newcommand{\textzh}[1]{\begin{CJK}{UTF8}{bsmi}#1\end{CJK}}
\fi

% %% XeTeX specific
\ifXeTeX
\usepackage{unicode-math}

\setmainfont[Ligatures=TeX]{Latin Modern Roman}
\setsansfont[Ligatures=TeX]{Latin Modern Sans}
\setmonofont{Latin Modern Mono}
\setmathfont{Latin Modern Math}

% \usepackage{unicode-minionmath}
% \setmainfont[Ligatures=TeX]{Minion Pro}
% \setsansfont[Ligatures=TeX]{Myriad Pro}
% \setmonofont[Scale=.85]{Courier Std}

% \setmathfont{Minion Math}
% \setmathfont[range={\mathfrak}]{XITS Math}
% \setmathfont[range={\mathcal},StylisticSet=1]{XITS Math}
% \setmathfont[range={\mathscr}]{XITS Math}
% \setmathfont[range={}]{Minion Math}
\fi

%% Theorems and definitions
\theoremstyle{plain}
\newtheorem{theorem}{Theorem}
\newtheorem{proposition}[theorem]{Proposition}
\newtheorem{corollary}[theorem]{Corollary}
\newtheorem{claim}[theorem]{Claim}
\newtheorem{lemma}[theorem]{Lemma}
\newtheorem{axiom}[theorem]{Axiom}

\newtheorem*{corollary*}{Corollary}
\newtheorem*{claim*}{Claim}
\newtheorem*{lemma*}{Lemma}
\newtheorem*{proposition*}{Proposition}
\newtheorem*{theorem*}{Theorem}

\theoremstyle{definition}
\newtheorem{definition}{Definition}
\newtheorem{example}{Examples}
\newtheorem{examples}[example]{Examples}
% \newtheorem{exercise}{Exercise}[section]
% \newtheorem{problem}[exercise]{Problem}

\newcounter{problem}
\newenvironment{problem}[1][]% environment name
{% begin code
  \par\vspace{\baselineskip}\noindent
  \ifx &#1&%
  {\normalfont\Large\bfseries\scshape Problem~Mid-\hwnum.\theproblem}
  \global\def\exercisename{Problem Mid-\hwnum.\theproblem}%
  \else
  {\normalfont\Large\bfseries\scshape Problem~Mid-\hwnum.\theproblem~(#1)}
  \global\def\exercisename{Problem Mid-\hwnum.\theproblem(#1)}
  \fi
  \par\vspace{\baselineskip}%
  \noindent\ignorespaces
}%
{% end code
  \par\vspace{\baselineskip}%
  \noindent\ignorespacesafterend
}

\newtheorem*{definition*}{Definition}
\newtheorem*{example*}{Examples}
\newtheorem*{examples*}{Examples}
\newtheorem*{exercise*}{Exercise}
\newtheorem*{problem*}{Problem}

\theoremstyle{remark}
\newtheorem{remark}{Remark}
\newtheorem{remarks}[remark]{Remarks}
\newtheorem{observation}[remark]{Observation}
\newtheorem{observations}[remark]{Observations}

\newtheorem*{remark*}{**Remark**}
\newtheorem*{remarks*}{**Remarks**}
\newtheorem*{observation*}{**Observation**}
\newtheorem*{observations*}{**Observations**}

%% Redefinitions & commands
\newcommand\restr[2]{{% we make the whole thing an ordinary symbol
  \left.\kern-\nulldelimiterspace % automatically resize the bar with \right
  {#1} % the function
  % \vphantom{\big|} % pretend it's a little taller at normal size
  \right|{#2} % this is the delimiter
  }}

\ifPDFTeX
\newcommand{\nsubset}{\ensuremath{\not\subset}}
\newcommand{\hooklongrightarrow}{\lhook\joinrel\longrightarrow}
\newcommand{\twoheadlongrightarrow}{\relbar\joinrel\twoheadrightarrow}

\renewcommand\qedsymbol{\ensuremath{\null\hfill\blacksquare}}

%% upint and loint
\def\upint{\mathchoice%
    {\mkern13mu\overline{\vphantom{\intop}\mkern7mu}\mkern-20mu}%
    {\mkern7mu\overline{\vphantom{\intop}\mkern7mu}\mkern-14mu}%
    {\mkern7mu\overline{\vphantom{\intop}\mkern7mu}\mkern-14mu}%
    {\mkern7mu\overline{\vphantom{\intop}\mkern7mu}\mkern-14mu}%
  \int}
\def\lowint{\mkern3mu\underline{\vphantom{\intop}\mkern7mu}\mkern-10mu\int}
\fi

\ifXeTeX
%% Patch arrows for XeTeX
\usepackage{etoolbox}
\makeatletter
\patchcmd{\arrowfill@}{-7mu}{-14mu}{}{}
\patchcmd{\arrowfill@}{-7mu}{-14mu}{}{}
\patchcmd{\arrowfill@}{-2mu}{-4mu}{}{}
\patchcmd{\arrowfill@}{-2mu}{-4mu}{}{}
\makeatother

\renewcommand\qedsymbol{\ensuremath{\null\hfill\QED}}
\fi

%% Commands and operators
\DeclareMathOperator{\id}{id}
\DeclareMathOperator{\im}{im}
\DeclareMathOperator{\Int}{int}
\DeclareMathOperator{\Cl}{cl}

\newcommand{\clsr}[1]{\overline{#1}}
\newcommand{\CC}{\mathbf{C}}
\newcommand{\NN}{\mathbf{N}}
\newcommand{\QQ}{\mathbf{Q}}
\newcommand{\RR}{\mathbf{R}}
\newcommand{\ZZ}{\mathbf{Z}}

\begin{document}
% \renewcommand\complement{\smallsetminus}
\renewcommand\setminus{\smallsetminus}
\renewcommand\phi{\varphi}
\renewcommand\epsilon{\varepsilon}

\frontmatter
\aliaspagestyle{title}{empty}
\pagestyle{title}
\author{\href{mailto:\authoremail}{\documentauthor}}
\title{\documenttitle}
\date{\today}
\maketitle
\cleartooddpage

\makeoddhead{headings}
        {\small{\MakeUppercase{\itshape\documentauthor}}}
        {}
        {\small{\MakeUppercase{\itshape\exercisename}}}
\makeoddfoot{headings}{{\itshape\documenttitle}}
                      {}
                      {\thepage}
\makeevenhead{headings}
        {\small{\MakeUppercase{\itshape\documentauthor}}}
        {}
        {\small{\MakeUppercase{\itshape\exercisename}}}
\makeevenfoot{headings}{{\itshape\documenttitle}}
                      {}
                      {\thepage}
\makeheadrule{headings}{\textwidth}{.25pt}
% \makerunningwidth{headings}{1.15\textwidth}
\pagestyle{headings}

\mainmatter
\setcounter{theorem}{13}
\begin{problem}[Munkres \S25, Ex.\,8]
Let $p\colon X\to Y$ be a quotient map. Show that if $X$ is
locally connected, then $Y$ is locally connected. [\emph{Hint:}
If $C$ is a component of the open set $U$ of $Y$, show that
$p^{-1}(C)$ is a union of components of $p^{-1}(U)$.]
\end{problem}
\begin{proof}
We will proceed from the hint. Let $U\subset Y$ be open and let
$C$ be a component of $U$. Then we will show that $p^{-1}(C)$ is
the union of components of $p^{-1}(U)$. Now, since $U$ is open in
$Y$, $p^{-1}(U)$ is open in $X$ and $p^{-1}(C)\subset
p^{-1}(U)$. Let $x\in p^{-1}(C)$ and let $C_x$ be the component
of $x$ in $p^{-1}(U)$. Then, by Theorem 25.3, since $X$ is
locally connected $C_x$ is open in $X$. The, we claim that
$C_x\subset p^{-1}(C)$. But this claim follows from Theorem 23.5
and Theorem 25.1 since $p(C_x)$ is connected and contains $p(x)$
so $p(C_x)\subset C$ so $C_x\subset p^{-1}(C)$. Taking the union
over every component $C_x$ corresponding to a point $x\in p^{-1}(U)$, we
have that
\[
p^{-1}(U)=\bigcup C_x
\]
is a union of components of $p^{-1}(U)$. It follows that
$p^{-1}(C)$ is open so, by the definition of the quotient
topology, $C$ is open and hence, it follows from Theorem 25.3
that $Y$ is locally path connected.
\end{proof}
\newpage
\begin{problem}[Munkres \S25, Ex.\,10(a,b)]
Let $X$ be a space. Let us define $x\sim y$ if there is no
separation $X=A\cup B$ of $X$ into disjoint open sets such that
$x\in A$ and $y\in B$.
\begin{enumerate}[label=(\alph*)]
\item Show this relation is an equivalence relation. The
  equivalence classes are called \emph{quasicomponents} of $X$.
\item Show that each component of $X$ lies in a quasicomponent of
  $X$, and that the components and quasicomponents of $X$ are the
  same if $X$ is locally connected.
\item Let $K$ denote the set
  $\left\{\,\frac{1}{n}\;\middle|\;n\in\ZZ_+\,\right\}$ and let
  $-K$ denote the set
  $\left\{\,-\frac{1}{n}\;\middle|\;n\in\ZZ_+\,\right\}$. Determine
  the components, path components, and quasicomponents of the
  following subspaces of $\RR^2$:
  \begin{align*}
    A&=(K\times[0,1])\cup\{0\times 0\}\cup\{0\times 1\}.\\
    B&=A\cup([0,1]\times\{0\}).\\
  \end{align*}
\end{enumerate}
\end{problem}
\begin{proof}
(a) To show that $\sim$ is a equivalence relation, we need to
check three things (i) reflexivity ($x\sim x$); (ii) symmetry (if $x\sim
y$ then $y\sim x$); and (iii) transitivity (if $x\sim y$ and
$y\sim z$ then $x\sim z$).

In order:
\begin{enumerate}[noitemsep,label=(\roman*)]
\item Seeking a contradiction, suppose that $A,B$ is a separation
  of $x$ such that $x\in A$ and $x\in B$ then $x\in A\cap B$ but
  $A\cap B=\emptyset$. Thus, $x\sim x$.
\item Suppose that $x\sim y$. Then, if $y\nsim x$ there exists a
  separation $A,B$ of $X$ such that $y\in A$, $x\in B$, but then
  $x\nsim y$.
\item Suppose $x\sim y$ and $y\sim z$. Seeking a contradiction,
  if $x\nsim z$ then there exists a separation $A,B$ of $X$ such
  that $x\in A$ and $z\in B$. Then, since $y\in X$ either $y\in
  A$ or $y\in B$. In the former case, $A,B$ is a separation with
  $y\in A$ and $z\in B$ contradicting $y\sim z$ and in the latter
  case $A,B$ is a separation with $x\in A$ and $y\in B$
  contradicting $x\sim y$. Thus, $x\sim z$.
\end{enumerate}
Thus, $\sim$ defines an equivalence relation on $X$.
\\\\
(b) Let $x\in X$ and let $Q$ and $C$ denote, respectively, the
quasicomponent and component of $x$. Then, we claim that
$C\subset Q$. For if $y\in C$ not in $Q$, then there exists a
separation $A,B$ of $X$ such that $x\in A$ and $y\in B$. But, by
Theorem 23.2, either since $C$ is connected either $C\subset A$
or $C\subset A$. In either case, we arrive at a contradiction
(it the former $C\subset A$ but $y\notin A$ and in the latter
$C\subset B$ but $x\notin A$). Thus, $C\subset Q$.

Keeping the notation the same as in the previous paragraph,
suppose $X$ is locally connected. Having shown $C\subset Q$ it
suffices holds, it suffices to show that $C\supset Q$. Suppose
not, then there exist some $y\in Q$ not in $C$. Since $X$ is
locally connected, then $C$ is open and closed so $C$ and
$X\setminus C$ is a separation of $X$. But then $x\in C$ and
$y\in X\setminus C$ which contradicts our assumption that $y\in
Q$, i.e, $x$ and $y$ lie in the same quasiconnected
component. Thus, $C\supset Q$ and in fact $C=Q$ holds.
\\\\
(c)
\end{proof}
\newpage
\begin{problem}[Munkres \S26, Ex.\,4]
Show that every compact subspace of a metric space is bounded in
that metric and is closed. Find a metric space in which not every
closed bounded subspace is compact.
\end{problem}
\begin{proof}
Let $(X,d)$ be a metric space and $Y\subset X$ a compact
subspace. By Theorem 26.3, $Y$ is closed since $X$ is Hausdorff
(since for any $x,y\in X$, let $\epsilon=d(x,y)$, then
$B(x,\epsilon/2)\cap B(y,\epsilon/2)=\emptyset$). Hence, we need
only show $Y$ is bounded.

Recall from Munkres \S20 that a set $A$ is bounded if there exist
some positive real number $M$ such that $d(a_1,a_2)\leq M$ for
every pair $a_1,a_2\in A$. Fix a $x_0\in Y$ and consider the
collection of open sets
\[
\mathcal{A}=\left\{\,B_d(x_0,M)\;\middle|\;M\in[0,\infty)\,\right\}.
\]
Then $\mathcal{A}$ is a covering of $Y$ (since for every $x\in
Y$, $x\in B_d(x_0,d(x_0,x)+1)$ so it is in the union of all of
them). Since $Y$ is compact $\mathcal{A}$, by Theorem 26.1, there
is a finite subcollection, say
$\left\{B(x_0,M_n)\right\}_{n=1}^N$, of $\mathcal{A}$ covering
$Y$. Let $M=\max\{M_1,...,M_n\}$. Then, for any pair $x,y\in Y$,
by the triangle inequality, we have
\begin{align*}
d(x,y)&\leq d(x_0,y)+d(x_0,y)\\
\intertext{but $x\in B(x_0,M_k)$ and $y\in B(x_0,M_\ell)$
  for some $1\leq k,\ell\leq N$ so}
      &\leq M_k+M_\ell\\
      &\leq 2M.
\end{align*}
Thus, $Y$ is bounded.
\end{proof}
\newpage
\begin{problem}[Munkres \S26, Ex.\,5]
Let $A$ and $B$ be disjoint compact subspaces of the Hausdorff
space $X$. Show that there exists disjoint open sets $U$ and $V$
containing $A$ and $B$, respectively.
\end{problem}
\begin{proof}
Suppose $A$ and $B$ are disjoint compact subpsaces of the
Hausdorff space $X$. By Theorem 26.4, there for every $x\in B$
there exists disjoint open sets $U_x\supset A$ and $V_x\ni x$. Then,
the collection of all such $V_x$, call it $\mathcal{V}$, is a
covering of $B$. By Theorem 26.1, there exists a finite
subcollection $\left\{V_n\right\}_{n=1}^N$ of $\mathcal{U}$
covering $B$. Let $\left\{U_n\right\}_{n=1}^N$ be the collection
of sets $U_i$ corresponding to $V_i$. Then $U=\bigcap_{i=1}^N
U_i$ and $V=\bigcup_{i=1}^N V_i$ are disjoint open subsets
containing $A$ and $B$ respectively since, by the distributive
property of ``$\cup$'', we have that
\[
U\cap V=U\cap\left(\bigcup V_i\right)=\bigcup U\cap V_i=\bigcup
U_i\cap V_i=\emptyset.
\]
\end{proof}
\newpage
\begin{problem}[Munkres \S26, Ex.\,7]
Show that if $Y$ is compact, then the projection $\pi_X\colon
X\times Y\to X$ is a closed map.
\end{problem}
\begin{proof}
We proceed by the tube lemma (Theorem 26.8). Let $C$ be a closed
subset of $X\times Y$. Then $N=(X\times Y)\setminus C$ is open in
$X\times Y$. Let $x_0\in X\setminus\pi_X(C)$. Then $x_0\times
Y\subset N$. By the tube lemma, there exists some $W$
neighborhood of $x_0$ in $X$ such that $W\times Y\subset N$. In
particular, $W\subset X\setminus\pi(C)$ for otherwise there is a
point $x\in W\cap\pi(C)$ which implies $x\times Y\subset N$ but
$x\times Y\cap C\neq\emptyset$ as $(x,y)\in x\times Y\cap C$ for
any $y\in\pi_Y(C)$. It follows, by Lemma C, that
$X\setminus\pi_X(C)$ is open so $\pi_X(C)$ is closed. Thus,
$\pi_X$ is a closed map.
\end{proof}
\newpage
\begin{problem}[A]
Let $X$ be a compact space and let $\sim$ be an equivalence
relation on $X$. Suppose that the set
\[
S=\left\{\,(x,y)\;\middle|\;x\sim y\,\right\}
\]
is a closed subset of $X\times X$. Prove that the quotient map
$q\colon X\to X/{\sim}$ is a closed map.
\end{problem}
\begin{proof}
Put $Y=X/{\sim}$.
We claim that:
\begin{lemma}
$B\subset Y$ is closed if and only if $q^{-1}(B)$ is closed in
$X$.
\end{lemma}
\begin{proof}
\renewcommand\qedsymbol{$\clubsuit$}
$B\subset Y$ is closed in $Y$ if and only if $Y\setminus B$ is
open in $Y$ if and only if $q^{-1}(Y\setminus B)=Y\setminus
q^{-1}(B)$ is open in $X$, i.e., $q^{-1}(B)$ is closed in $X$.
\end{proof}
Let $C\subset X$ be closed. By Lemma 14, it suffices to show that
$q^{-1}(q(C))$ is closed. But note that
\begin{align*}
q^{-1}(q(C))
  &=\left\{\,x\;\middle|\;q(x)\in q(C)\,\right\}\\
  &=\left\{\,x\;\middle|\;\text{for some $y\in C$, $x\sim
    y$}\,\right\}\\
  &=\pi_1(S\cap (X\times B)).
\end{align*}
By Problem 6.5, $\pi_1\colon X\times X\to X$ sending
$(x_1,x_2)\mapsto x_1$ is a closed map, therefore it suffices to
check that $S\cap (X\times B)$ is closed, in particular, we need
to check that $X\times B$ is closed (since $S$ closed is
given). But $\pi_2\colon X\times X\to X$ via $(x_1,x_2)\mapsto
x_2$ is continuous so $X\times B=\pi_2^{-1}(B)$ is closed. Thus,
$S\cap (X\times B)$ is closed. Thus, $q^{-1}(q(C))$ is closed so
by Lemma 14 $q(C)$ is closed.
\end{proof}
\newpage
\begin{problem}[B]
Let $S^2$ be the sphere
\[
\left\{\,(x,y,z)\in\RR^3\;\middle|\;x^2+y^2+z^2=1\,\right\}.
\]
Let $S_+^2$ be $S^2\cap\{z\geq 0\}$ (the upper hemisphere), let
$S_-^2$ be $S^2\cap\{z\leq 0\}$ (the lower hemisphere), and let
$E$ be $S^2\cap\{z=0\}$ (the equator). Recall the definition of
$Y/S$ from Homework \#4. Prove that $S^2/S^2_-$ is homeomorphic
to $S_+^2/E$. [\emph{Hint:} There are maps in both directions.]
\end{problem}
\begin{proof}
Let us begin by rewriting the sets in question in a more
descriptive way
\end{proof}

%%% Local Variables:
%%% mode: latex
%%% TeX-master: "../MA571-HW-Current"
%%% End:

\end{document}

%%% Local Variables:
%%% mode: latex
%%% TeX-master: t
%%% End:
