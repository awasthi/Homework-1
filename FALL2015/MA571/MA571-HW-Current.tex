\def\documentauthor{Carlos Salinas}
\def\documenttitle{MA571 Homework \hwnum}
\def\hwnum{14}
\def\shorttitle{MA571 HW \hwnum}
\def\coursename{MA571}
\def\documentsubject{point-set topology}
\def\authoremail{salinac@purdue.edu}

\documentclass[article,oneside,10pt]{memoir}
\usepackage{geometry}
\usepackage[dvipsnames]{xcolor}
\usepackage[
    breaklinks,
    bookmarks=true,
    colorlinks=true,
    pageanchor=false,
    linkcolor=black,
    anchorcolor=black,
    citecolor=black,
    filecolor=black,
    menucolor=black,
    runcolor=black,
    urlcolor=black,
    hyperindex=false,
    hyperfootnotes=true,
    pdftitle={\shorttitle},
    pdfauthor={\documentauthor},
    pdfkeywords={\documentsubject},
    pdfsubject={\coursename}
    ]{hyperref}

% Use symbols instead of numbers
\renewcommand*{\thefootnote}{\fnsymbol{footnote}}

%% Math
\usepackage{amsthm}
\usepackage{amssymb}
\usepackage{mathtools}

%%PDFTeX specific
\usepackage[mathcal]{euscript}
\usepackage{mathrsfs}

\usepackage[LAE,LFE,T2A,T1]{fontenc}
\usepackage[utf8]{inputenc}
\usepackage[farsi,french,german,spanish,dutch,russian,swedish,english]{babel}
\babeltags{pa=farsi,
           fr=french,
           de=german,
           es=spanish,
           nl=dutch,
           ru=russian,
           sv=swedish,
           en=english}
\def\spanishoptions{mexico}

\selectlanguage{english}

\newcommand{\textfa}[1]{\beginR\textpa{#1}\endR}

\usepackage{cmap}
\usepackage{CJKutf8}
\newcommand{\textha}[1]{\begin{CJK}{UTF8}{mj}#1\end{CJK}}
\newcommand{\textni}[1]{\begin{CJK}{UTF8}{min}#1\end{CJK}}
\newcommand{\textzh}[1]{\begin{CJK}{UTF8}{bsmi}#1\end{CJK}}

%%% XeTeX/LuaTeX
% \usepackage{unicode-math}
% \setmainfont[RawFeature={% +ss02,+cv01,
%                          +ss05,+dlig},
%              ItalicFeatures={RawFeature=+cv04,CharacterVariant=5:2},
%              Numbers=Lining,
%              Ligatures=TeX]{EB Garamond}
% \setmathfont{Minion Math}
% \setmathfont[range={\mathfrak}]{XITS Math}
% \setmathfont[range={\mathcal},StylisticSet=1]{XITS Math}
% \setmathfont[range={\mathscr}]{XITS Math}
% % \setmathfont[range={\mathfrak}]{Latin Modern Math}
% % \setmathfont[range={\mathcal}]{Latin Modern Math}
% \setmathfont[range={}]{Minion Math}

% \usepackage{polyglossia}

% \setmainlanguage[variant=american]{english}
% \setotherlanguage{french}
% \setotherlanguage[spelling=new,latesthyphen,babelshorthands]{german}
% \setotherlanguage{spanish}
% \setotherlanguage[spelling=modern,babelshorthands]{russian}

%% Misc
\usepackage{graphicx}
\graphicspath{{figures/}}

\usepackage{microtype}
\usepackage{multicol}
\usepackage[inline]{enumitem}
\usepackage{listings}
\usepackage{mleftright}
\mleftright

%% Theorems and definitions
%% remove parentheses
\makeatletter
\def\thmhead@plain#1#2#3{%
  \thmname{#1}\thmnumber{\@ifnotempty{#1}{ }\@upn{#2}}%
  \thmnote{ {\the\thm@notefont\textbf{#3}}}}
\let\thmhead\thmhead@plain
\makeatother

\theoremstyle{plain}
\newtheorem{theorem}{Theorem}
\newtheorem{proposition}[theorem]{Proposition}
\newtheorem{corollary}[theorem]{Corollary}
\newtheorem{claim}[theorem]{Claim}
\newtheorem{lemma}[theorem]{Lemma}
\newtheorem{axiom}[theorem]{Axiom}

\newtheorem*{corollary*}{Corollary}
\newtheorem*{claim*}{Claim}
\newtheorem*{lemma*}{Lemma}
\newtheorem*{proposition*}{Proposition}
\newtheorem*{theorem*}{Theorem}

\theoremstyle{definition}
\newtheorem{definition}{Definition}
\newtheorem{example}{Examples}
\newtheorem{examples}[example]{Examples}
% \newtheorem{exercise}{Exercise}[section]
% \newtheorem{problem}[exercise]{Problem}

\newcounter{problem}
\newenvironment{problem}[1][]% environment name
{% begin code
  \stepcounter{problem}
  \par\vspace{\baselineskip}\noindent
  \ifx &#1&%
  {\normalfont\Large\bfseries\scshape Problem~\hwnum.\theproblem}
  \global\def\exercisename{Problem~\hwnum.\theproblem}%
  \else
  {\normalfont\Large\bfseries\scshape Problem~\hwnum.\theproblem~(#1)}
  \global\def\exercisename{Problem~\hwnum.\theproblem(#1)}
  \fi
  \par\vspace{\baselineskip}%
  \noindent\ignorespaces
}%
{% end code
  \par\vspace{\baselineskip}%
  \noindent\ignorespacesafterend
}

\newtheorem*{definition*}{Definition}
\newtheorem*{example*}{Examples}
\newtheorem*{examples*}{Examples}
\newtheorem*{exercise*}{Exercise}
\newtheorem*{problem*}{Problem}

\theoremstyle{remark}
\newtheorem{remark}{Remark}
\newtheorem{remarks}[remark]{Remarks}
\newtheorem{observation}[remark]{Observation}
\newtheorem{observations}[remark]{Observations}

\newtheorem*{remark*}{**Remark**}
\newtheorem*{remarks*}{**Remarks**}
\newtheorem*{observation*}{**Observation**}
\newtheorem*{observations*}{**Observations**}

%% Redefinitions & commands
% \newcommand\restr[2]{{% we make the whole thing an ordinary symbol
%   \left.\kern-\nulldelimiterspace % automatically resize the bar with \right
%   {#1} % the function
%   % \vphantom{\big|} % pretend it's a little taller at normal size
%   \right|{#2} % this is the delimiter
%   }}

\newcommand\minus{\smallsetminus}
\newcommand{\nsubset}{\ensuremath{\not\subset}}
\newcommand{\nsupset}{\ensuremath{\not\supset}}

\renewcommand\qedsymbol{\ensuremath{\null\hfill\blacksquare}}

%% Commands and operators
\DeclareMathOperator{\diam}{diam}
\DeclareMathOperator{\id}{id}
\DeclareMathOperator{\im}{im}
\DeclareMathOperator{\Int}{int}
\DeclareMathOperator{\Cl}{cl}

\newcommand{\CC}{\mathbf{C}}
\newcommand{\NN}{\mathbf{N}}
\newcommand{\QQ}{\mathbf{Q}}
\newcommand{\RR}{\mathbf{R}}
\newcommand{\ZZ}{\mathbf{Z}}

% Renewcommands
% \renewcommand\setminus{\smallsetminus}
% \renewcommand\phi{\varphi}
% \renewcommand\epsilon{\varepsilon}

\begin{document}
\frontmatter
\aliaspagestyle{title}{empty}
\pagestyle{title}
\author{\href{mailto:\authoremail}{\documentauthor}}
\title{\documenttitle}
\date{\today}
\maketitle
\cleartooddpage
\makepagestyle{my-headings}
\makeoddhead{my-headings}
        {\small{\MakeUppercase{\itshape\documentauthor}}}
        {}
        {\small{\MakeUppercase{\itshape\exercisename}}}
\makeoddfoot{my-headings}{{\itshape\documenttitle}}
                      {}
                      {\thepage}
\makeevenhead{my-headings}
        {\small{\MakeUppercase{\itshape\documentauthor}}}
        {}
        {\small{\MakeUppercase{\itshape\exercisename}}}
\makeevenfoot{my-headings}{{\itshape\documenttitle}}
                      {}
                      {\thepage}
\makeheadrule{my-headings}{\textwidth}{.25pt}
\pagestyle{my-headings}
% \makerunningwidth{headings}{1.15\textwidth}
\mainmatter

\begin{problem}[Munkres \S74, Ex.\,6]
If $n>1$, show that the fundamental group of the $n$-fold torus is not
Abelian. [\emph{Hint:} Let $G$ be a free group on the set
$\{\alpha_1,\beta_1,...,\alpha_n,\beta_n\}$; let $F$ be the free group on
the set $\{\gamma,\delta\}$. Consider the homomorphism of $G$ onto $F$ that
sends $\alpha_1$ and $\beta_1$ to $\gamma$ and all other $\alpha_i$ and
$\beta_i$ to $\delta$.]
\end{problem}
\begin{proof}
Let $\bfT^n$ denote the $n$-fold torus and let us fix a base-point
$x_0\in\bfT^n$. By Theorem 74.3, the fundamental group of $\bfT^n$,
$\pi_1(\bfT^n,x_0)$, is isomorphic to the quotient of the free group $G$ on
the set $\{\alpha_1,\beta_1,...,\alpha_n,\beta_n\}$, by the least normal
subgroup $N$ containing
\begin{equation}
\label{eq:1}
\alpha_1\beta_1\alpha_1^{-1}\beta_1^{-1}
\cdots
\alpha_n\beta_n\alpha_n^{-1}\beta_n^{-1}.
\end{equation}
Now we proceed by the hint. Let $F$ be the free group on the set
$\{\gamma,\delta\}$. We define a homomorphism $\varphi\colon G\to F$ by the
rule $\alpha_1\mapsto\gamma$, $\beta_1\mapsto\gamma$ and
$\alpha_i\mapsto\delta$ and $\beta_i\mapsto\delta$ for all $i\neq
1$. By Lemma 69.1, $\varphi$ determines a homomorphism $G\to F$. Moreover,
note that $\varphi$ is surjective so by the 1st isomorphism theorem,
$G/\ker\varphi\cong F$. Now, our next goal is to use the universal mapping
property of the group quotient which guarantees the existence and
uniqueness of a map $\bar\varphi\colon G/N\to F$.\footnote{Munkres never
explicitly calls it this in his short exposition of group theory or,
indeed, the UMP of the free product, quotient topology, product
topology. These concepts are very illuminating and makes the whole process
of writing thinking about a particular algebraic/geometric object much
easier. In my opinion of course. tl;dr I don't know where this is stated in
Munkres; we all know some group theory---this is true; please don't take
off points.} To that end, we need to show that $N<\ker\varphi$. But $N$ is
the intersection of all normal subgroups of $G$ containing (\ref{eq:1})
hence, it suffices to show that $\ker\varphi$ contains (\ref{eq:1}). But
this is immediate since
\begin{align*}
\varphi(
\alpha_1\beta_1\alpha_1^{-1}\beta_1^{-1}
\cdots
\alpha_n\beta_n\alpha_n^{-1}\beta_n^{-1})
&=\varphi(\alpha_1)\varphi(\beta_1)\varphi(\alpha_1^{-1})\varphi(\beta_1)^{-1}\cdots
\varphi(\alpha_n)\varphi(\beta_n)\varphi(\alpha_n^{-1})\varphi(\beta_n)^{-1}\\
&=\delta\delta\delta^{-1}\delta^{-1}\gamma\gamma\gamma^{-1}\gamma^{-1}
\cdots\gamma\gamma\gamma^{-1}\gamma^{-1}\\
&=1.
\end{align*}
Thus, there exists a map $\bar\varphi\colon G/N\to F$ such that
$\varphi=\bar\varphi\circ\pi_N$ where $\pi$ is the canonical (group)
projection map $\pi\colon G\to G/N$ defined by the rule $g\mapsto
g+N$. Since $\varphi(G)=F$, $\bar\varphi(G/N)=F$ which is non-Abelian so
$G/N$.\footnote{This is by elementary group theory: Say $G$ is Abelian and
  $\varphi\colon G\to F$ is an homomorphism. Then $\varphi(G)<F$ is Abelian
  since, by the properties of the homomorphism, for any $g_1,g_2$,
  $\varphi(g_1g_2)=\varphi(g_2g_1)$ so
  $\varphi(g_1)\varphi(g_2)=\varphi(g_2)\varphi(g_2)$.}
\end{proof}
\newpage
\begin{problem}[Munkres \S75, Ex.\,1]
Calculate $H_1(\bfP^2\#\bfT)$. Assuming that the list of compact surfaces given
in Theorem 75.5 is a complete list, to which of these surfaces is $\bfP^2\#\bfT$
homeomorphic?
\end{problem}
\begin{proof}
We shall begin by setting up, but not actually computing the fundamental
group, $\pi_1(\bfP^2\#\bfT)$. On Munkres \S74, p.\,453, Munkres gives us
the labeling scheme
\begin{equation}
\label{eq:2}
aabcb^{-1}c^{-1}
\end{equation}
for the connected sum of $\bfP^2$ and $\bfT$; we shall not prove that this,
indeed, determines the quotient space $\bfP^2\#\bfT$ (unless we have time),
but instead we will use the labeling scheme to do our computations. Now,
given (\ref{eq:2}), by Theorem 74.3, the fundamental group of
$\bfP^2\#\bfT$ is the quotient of the free group on the set $\{a,b,c\}$,
$G$, by the least normal subgroup $N$ containing (\ref{eq:2}). Now, by
Corollary 75.2, the Abelianization of $\pi_1(\bfP^2\#bfT)$ is isomorphic to
\begin{align}
H_1(\bfP^2\#\bfT)
&\cong\frac{\bfZ(p(a))\times\bfZ(p(b))\times\bfZ(p(c))}
{p\left(\left<aabcb^{-1}c^{-1}\right>\right)}
\intertext{and simplifying the quotient above taking note that $p$ is a
  homomorphism whose image is lies inside an Abelian group, we have}
\label{eq:3}
&=\frac{\bfZ(p(a))\times\bfZ(p(b))\times\bfZ(p(c))}{\left<2p(a)\right>}
\end{align}
where we use the module notation $\bfZ(x)$ to denote the free Abelian group
generated by $x$, because why not; these happen to be $\bfZ$-modules after
all. By the 1st isomorphism theorem, (\ref{eq:3}) is isomorphic to
\begin{equation}
\label{eq:4}
\bfZ/(2)\times\bfZ\times\bfZ
\end{equation}
by the obvious homomorphism, i.e., the one sending $(p(a),0,0)\mapsto
(1,0,0)$, $(0,p(b),0)\mapsto(0,1,0)$, and $(0,0,p(c))\mapsto(0,0,1)$). This
map is bijective since it has an inverse, the map sending $(1,0,0)\mapsto
(p(a),0,0)$, $(0,1,0)\mapsto(0,p(b),0)$, and $(0,0,1)\mapsto(0,0,p(c))$. It
follows from Lemma 67.7 that both of the maps described above are
homomorphisms. From the list given in Theorem 75.5, $\bfP^2\#\bfT$ must be
homeomorphic to $\bfP^2\#\bfP^2$.
\end{proof}
\newpage
\begin{problem}[Munkres \S75, Ex.\,2]
If $\bfK$ is the Klein bottle, calculate $H_1(\bfK)$ directly.
\end{problem}
\begin{proof}
\end{proof}
\newpage
\begin{problem}[Munkres \S75, Ex.\,3(a,b,c)]
Let $X$ be the quotient space obtained from an $8$-sided polygonal region
$P$ by pasting its edges together according to the labelling scheme
$acadbcb^{-1}d$.
\begin{enumerate}[label=(\alph*)]
\item Check that all vertices of $P$ are mapped to the same point of the
  quotient space $X$ by the pasting map.
\item Calculate $H_1(X)$.
\item Assuming $X$ is homeomorphic to one of the surfaces given in Theorem
  75.5 (which it is), which surface is it ?
\end{enumerate}
\end{problem}
\begin{proof}
\end{proof}
\newpage
\begin{problem}[A]
Define $P^n$ to be the space $\bfS^n/{\sim}$ where $z\sim z'$ if and only
if $z=z'$ or $z=-z'$. Use the Seifert--van Kampen Theorem to calculate
$\pi_1(\bfP^n)$. (Hint: induction starting from the case $n=2$ that was
done in class.)
\end{problem}
\begin{proof}
\end{proof}
\newpage
\begin{problem}[B]
A topological space $X$ is called \emph{homogeneous} if for every pair of
points $x,y\in X$ there is a homeomorphism $\varphi\colon X\to X$ with
$\varphi(x)=y$. Prove that every connected $2$-manifold is
homogeneous. (Hint: use the optional problem from the previous assignment.)
\end{problem}
\begin{proof}
\end{proof}
\newpage
\begin{problem}[Optional problem]
\begin{enumerate}[label=(\roman*)]
\item Let $x\subset\bfR^3$ be the cylinder
\[
\left\{\,(x,y,z)\;\middle|\;
\text{$x^2+y^2=\tfrac{1}{\sqrt{2}}$ and $|z|\leq \tfrac{1}{\sqrt{2}}$}\,\right\}
\]
and let $f\colon X\to\bfR^3$ be the map
\[
f(x,y,z)=\left(2^{1/4}x\sqrt{1-z^2},2^{1/4},y\sqrt{1-z^2},z\right).
\]
Prove that $f$ is a homeomorphism from $X$ to the subspace
\[
Y=\bfS^2\cap\left\{\,(x,y,z)\;\middle|\;|z|\leq\tfrac{1}{\sqrt{2}}\,\right\}.
\]
\item Prove that the \textde{Möbius} band is homeomorphic to $P^2$ with an
  open disk removed (think of $\bfP^2$ as $\bfS^2/{\sim}$ and use part (i)).
\end{enumerate}
\end{problem}
\begin{proof}
\end{proof}
% \textru{Татарстаннын кызлары миңа бик килеше!}

%%% Local Variables:
%%% mode: latex
%%% TeX-master: "../MA571-HW-Current"
%%% End:

\end{document}

%%% Local Variables:
%%% mode: latex
%%% TeX-master: t
%%% End:
