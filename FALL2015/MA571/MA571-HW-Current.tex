\def\documentauthor{Carlos Salinas}
\def\documenttitle{MA571 Problem Set \hwnum}
\def\hwnum{5}
\def\shorttitle{MA571 PSet \hwnum}
\def\coursename{MA571}
\def\documentsubject{point-set topology}
\def\authoremail{salinac@purdue.edu}

\documentclass[article,oneside,10pt]{memoir}
\usepackage{geometry}
\usepackage[dvipsnames]{xcolor}
\usepackage[
    breaklinks,
    bookmarks=true,
    colorlinks=true,
    pageanchor=false,
    linkcolor=black,
    anchorcolor=black,
    citecolor=black,
    filecolor=black,
    menucolor=black,
    runcolor=black,
    urlcolor=black,
    hyperindex=false,
    hyperfootnotes=true,
    pdftitle={\shorttitle},
    pdfauthor={\documentauthor},
    pdfkeywords={\documentsubject},
    pdfsubject={\coursename}
    ]{hyperref}

\usepackage{graphicx}
\graphicspath{{figures/}}

\usepackage{multicol}
\usepackage[inline]{enumitem}
\usepackage{listings}
\usepackage{mleftright}
\mleftright

\usepackage{microtype}

\usepackage{amsthm}
\usepackage{amssymb}
\usepackage{mathtools}
\usepackage[mathcal]{euscript}
\usepackage{mathrsfs}
\usepackage[T2A,T1]{fontenc}
\usepackage[utf8]{inputenc}
\usepackage[french,ngerman,russian,spanish,english]{babel}
\newcommand\nsubset{\ensuremath{\not\subset}}
\renewcommand\qedsymbol{\ensuremath{\null\hfill\blacksquare}}

% \usepackage{ifxetex}
% %% XeTeX specific
% \ifxetex
% \usepackage{unicode-math}

% \setmainfont[Ligatures=TeX]{Latin Modern Roman}
% \setsansfont[Ligatures=TeX]{Latin Modern Sans}
% \setmonofont{Latin Modern Mono}
% \setmathfont{Latin Modern Math}

% %% Patch arrows for XeTeX
% \usepackage{etoolbox}
% \makeatletter
% \patchcmd{\arrowfill@}{-7mu}{-14mu}{}{}
% \patchcmd{\arrowfill@}{-7mu}{-14mu}{}{}
% \patchcmd{\arrowfill@}{-2mu}{-4mu}{}{}
% \patchcmd{\arrowfill@}{-2mu}{-4mu}{}{}
% \makeatother

% \renewcommand\qedsymbol{\ensuremath{\null\hfill\QED}}
% \fi

%% Theorems and definitions
\theoremstyle{plain}
\newtheorem{theorem}{Theorem}
\newtheorem{proposition}[theorem]{Proposition}
\newtheorem{corollary}[theorem]{Corollary}
\newtheorem{claim}[theorem]{Claim}
\newtheorem{lemma}[theorem]{Lemma}
\newtheorem{axiom}[theorem]{Axiom}

\newtheorem*{corollary*}{Corollary}
\newtheorem*{claim*}{Claim}
\newtheorem*{lemma*}{Lemma}
\newtheorem*{proposition*}{Proposition}
\newtheorem*{theorem*}{Theorem}

\theoremstyle{definition}
\newtheorem{definition}{Definition}
\newtheorem{example}{Examples}
\newtheorem{examples}[example]{Examples}
% \newtheorem{exercise}{Exercise}[section]
% \newtheorem{problem}[exercise]{Problem}

\counterwithout{section}{chapter}
\usepackage[explicit]{titlesec}
\titleformat{\section}{\normalfont\Large\bfseries\scshape}{}{0em}{#1}
\newenvironment{problem}[1][]% environment name
{% begin code
  \par\vspace{\baselineskip}\noindent
  \ifx &#1&%
  \section{Problem~\hwnum.\thesection}
  \global\def\exercisename{Problem \hwnum.\thesection}%
  \else
  \section{Problem~\hwnum.\thesection~(#1)}
  \global\def\exercisename{Problem \hwnum.\thesection(#1)}
  \fi
  \par\vspace{\baselineskip}%
  \noindent\ignorespaces
}%
{% end code
  \par\vspace{\baselineskip}%
  \noindent\ignorespacesafterend
}

\newtheorem*{definition*}{Definition}
\newtheorem*{example*}{Examples}
\newtheorem*{examples*}{Examples}
\newtheorem*{exercise*}{Exercise}
\newtheorem*{problem*}{Problem}

\theoremstyle{remark}
\newtheorem{remark}{Remark}
\newtheorem{remarks}[remark]{Remarks}
\newtheorem{observation}[remark]{Observation}
\newtheorem{observations}[remark]{Observations}

\newtheorem*{remark*}{**Remark**}
\newtheorem*{remarks*}{**Remarks**}
\newtheorem*{observation*}{**Observation**}
\newtheorem*{observations*}{**Observations**}

%% Redefinitions & commands
\newcommand\restr[2]{{% we make the whole thing an ordinary symbol
  \left.\kern-\nulldelimiterspace % automatically resize the bar with \right
  {#1} % the function
  % \vphantom{\big|} % pretend it's a little taller at normal size
  \right|{#2} % this is the delimiter
  }}

%% Commands and operators
\DeclareMathOperator{\id}{id}
\DeclareMathOperator{\im}{im}
\DeclareMathOperator{\Int}{int}
\DeclareMathOperator{\Cl}{cl}

\newcommand{\clsr}[1]{\overline{#1}}
\newcommand{\CC}{\mathbf{C}}
\newcommand{\NN}{\mathbf{N}}
\newcommand{\QQ}{\mathbf{Q}}
\newcommand{\RR}{\mathbf{R}}
\newcommand{\ZZ}{\mathbf{Z}}

\begin{document}
\let\setminus\relax
\let\phi\relax
\let\epsilon\relax
\newcommand\setminus{\smallsetminus}
\newcommand\phi{\varphi}
\newcommand\epsilon{\varepsilon}

\frontmatter
\aliaspagestyle{title}{empty}
\pagestyle{title}
\author{\href{mailto:\authoremail}{\documentauthor}}
\title{\documenttitle}
\date{\today}
\maketitle
\cleartooddpage

\makeoddhead{headings}
        {\small{\MakeUppercase{\itshape\documentauthor}}}
        {}
        {\small{\MakeUppercase{\itshape\exercisename}}}
\makeoddfoot{headings}{{\itshape\documenttitle}}
                      {}
                      {\thepage}
\makeevenhead{headings}
        {\small{\MakeUppercase{\itshape\documentauthor}}}
        {}
        {\small{\MakeUppercase{\itshape\exercisename}}}
\makeevenfoot{headings}{{\itshape\documenttitle}}
                      {}
                      {\thepage}
\makeheadrule{headings}{\textwidth}{.25pt}
% \makerunningwidth{headings}{1.15\textwidth}
\pagestyle{headings}

\mainmatter
\setcounter{theorem}{12}
\begin{problem}[Munkres \S23, Ex.\,3]
Let $\left\{A_\alpha\right\}$ be a collection of connected
subspaces of $X$, such that $A_n\cap A_{n+1}\neq\emptyset$ for
all $n$. Show that $\bigcup A_n$ is connected.
\end{problem}
\begin{proof}
\end{proof}
\newpage
\begin{problem}[Munkres \S23, Ex.\,6]
Let $A\subset X$. Show that if $C$ is a connected subspace of $X$
that intersects both $A$ and $X\setminus A$, then $C$ intersects
$\partial A$.
\end{problem}
\begin{proof}
\end{proof}
\newpage
\begin{problem}[Munkres \S23, Ex.\,7]
Is the space $\RR_\ell$ connected? Justify your answer.
\end{problem}
\begin{proof}
\end{proof}
\newpage
\begin{problem}[Munkres \S23, Ex.\,9]
Let $A$ be a proper subset of $X$, and let $B$ be a proper subset
of $Y$. If $X$ and $Y$ are connected, show that
\[
(X\times Y)\setminus(A\times B)
\]
is connected.
\end{problem}
\begin{proof}
\end{proof}
\newpage
\begin{problem}[Munkres \S24, Ex.\,1(ac)]
\begin{enumerate}[noitemsep]
\item[(a)] Show that no two of the spaces $(0,1)$, $(0,1]$ and
  $[0,1]$ are homeomorphic. [\emph{Hint:} What happens if you
  remove a point from each of these spaces?]
\item[(c)] Show $\RR^n$ and $\RR$ are not homeomorphic if $n>1$.
\end{enumerate}
\end{problem}
\begin{proof}
\end{proof}
\newpage
\begin{problem}[Munkres \S24, Ex.\,2]
Let $f\colon S^1\to\RR$ be a continuous map. Show there exists a
point $x$ of $S^1$ such that $f(x)=f(-x)$.
\end{problem}
\begin{proof}
\end{proof}
\newpage
\begin{problem}[Munkres \S25, Ex.\,2(b)]
\begin{enumerate}[noitemsep]
\item[(b)] Consider $\RR^\omega$ in the uniform topology. Show
  that $\mathbf{x}$ and $\mathbf{y}$ lie in the same component of
  $\RR^\omega$ if and only if the sequence
  \[
    \mathbf{x}-\mathbf{y}=\left(x_1-y_1,x_2-y_2,...\right)
  \]
  is bounded. [\emph{Hint:} It suffices to consider the case
  where $\mathbf{y}=\mathbf{0}$.]
\end{enumerate}
\end{problem}
\begin{proof}
\end{proof}
\newpage
\begin{problem}[Munkres \S25, Ex.\,4]
Let $X$ be locally path connected. Show that every connected open
set in $X$ is path connected.
\end{problem}
\begin{proof}
\end{proof}
\newpage
\begin{problem}[Munkres \S25, Ex.\,6]
A space $X$ is said to be \emph{weakly locally path connected at
  $x$} if for every neighborhood $U$ of $x$, there is a connected
subspace of $X$ contained in $U$ that contains a neighborhood of
$x$. Show that if $X$ is weakly locally connected at each of its
points, then $X$ is locally connected. [\emph{Hint:} H]
\end{problem}
\begin{proof}
\end{proof}
\newpage
\begin{problem}[A]
Let $X$ be a topological space. The quotient space
$(X\times[0,1])/(X\times 0)$ is called the \emph{cone} of $X$ and
denoted $CX$.
\\\\
Prove that if $X$ is homeomorphic to $Y$ then $CX$ is
homeomorphic to $CY$ (\emph{Hint:} There are maps in both
directions).
\end{problem}
\begin{proof}
\end{proof}
\newpage
\begin{problem}[Extra problem]
Notation: for positive integers $i,n,I,N$, let us write
$(i,n)\gg(I,N)$ if $i>I$ and $n>N$.
\begin{theorem}
A sequence $\left\{\mathbf{x}_n\right\}$ in $\RR^\omega$
converges to $\mathbf{0}$ in the box topology if and only if two
conditions hold:
\begin{enumerate}[noitemsep,label=(\roman*)]
\item for each $k$, $\lim_{n\to\infty} x_n^{(k)}=0$, and
\item there is a pair $(I,N)$ with $x_n^{(k)}=0$ whenever
  $(i,n)\gg(I,N)$.
\end{enumerate}
\end{theorem}
\end{problem}
\begin{proof}
\end{proof}

%%% Local Variables:
%%% mode: latex
%%% TeX-master: "../MA571-HW-Current"
%%% End:

\end{document}

%%% Local Variables:
%%% mode: latex
%%% TeX-master: t
%%% End:
