\def\documentauthor{Carlos Salinas}
\def\documenttitle{MA571 Problem Set \hwnum}
\def\hwnum{1}
\def\shorttitle{MA571 PSet \hwnum}
\def\coursename{MA571}
\def\documentsubject{point-set topology}
\def\authoremail{salinac@purdue.edu}

\documentclass[article,oneside,10pt]{memoir}
\usepackage{geometry}
\usepackage[dvipsnames]{xcolor}
\usepackage[
    breaklinks,
    bookmarks=true,
    colorlinks=true,
    pageanchor=false,
    linkcolor=black,
    anchorcolor=black,
    citecolor=black,
    filecolor=black,
    menucolor=black,
    runcolor=black,
    urlcolor=black,
    hyperindex=false,
    hyperfootnotes=true,
    pdftitle={\shorttitle},
    pdfauthor={\documentauthor},
    pdfkeywords={\documentsubject},
    pdfsubject={\coursename}
    ]{hyperref}

\usepackage{graphicx}
\graphicspath{{figures/}}

\usepackage{multicol}
\usepackage[inline]{enumitem}
\usepackage{listings}
\usepackage{mleftright}
\mleftright

\usepackage{microtype}

\usepackage{amsthm}
\usepackage{amssymb}
\usepackage{mathtools}
\usepackage{lualatex-math}
\usepackage{unicode-math}
% \usepackage{unicode-minionmath}

% \setmainfont[Ligatures=TeX]{Minion Pro}
% \setsansfont[Ligatures=TeX]{Myriad Pro}
% \setmonofont{Courier Std}

% \setmathfont{Minion Math}
% \setmathfont[range={\mathfrak}]{XITS Math}
% \setmathfont[range={\mathcal},StylisticSet=1]{XITS Math}
% \setmathfont[range={\mathscr}]{XITS Math}
% \setmathfont[range={}]{Minion Math}

\setmainfont[Ligatures=TeX]{Latin Modern Roman}
\setsansfont[Ligatures=TeX]{Latin Modern Sans}
\setmonofont{Latin Modern Mono}
\setmathfont{Latin Modern Math}

% \setmainfont[Ligatures=TeX]{CMU Serif}
% \setsansfont[Ligatures=TeX]{CMU Sans Serif}
% \setmonofont{CMU Typewriter Text}

% \setmainfont[Ligatures=TeX]{XITS}
% \setmathfont[range={\mathcal,\mathbfcal},StylisticSet=1]{XITS Math}
% \setmathfont[range=\mathscr]{XITS Math}
% \setmathfont{XITS Math}

\usepackage{polyglossia}
\setmainlanguage[variant=american]{english}
\setotherlanguage{french}
\setotherlanguage[spelling=new,latesthyphen,babelshorthands]{german}
\setotherlanguage[spelling=modern,babelshorthands]{russian}
\setotherlanguage{spanish}

\newfontfamily\cyrillicfont[Ligatures=TeX]{CMU Serif}
\newfontfamily\cyrillicfontsf[Ligatures=TeX]{CMU Sans Serif}
\newfontfamily\cyrillicfonttt{CMU Typewriter Text}

\theoremstyle{plain}
\newtheorem{theorem}{Theorem}
\newtheorem{proposition}[theorem]{Proposition}
\newtheorem{corollary}[theorem]{Corollary}
\newtheorem{claim}[theorem]{Claim}
\newtheorem{lemma}[theorem]{Lemma}
\newtheorem{axiom}[theorem]{Axiom}

\newtheorem*{corollary*}{Corollary}
\newtheorem*{claim*}{Claim}
\newtheorem*{lemma*}{Lemma}
\newtheorem*{proposition*}{Proposition}
\newtheorem*{theorem*}{Theorem}

\theoremstyle{definition}
\newtheorem{definition}{Definition}
\newtheorem{example}{Examples}
\newtheorem{examples}[example]{Examples}
% \newtheorem{exercise}{Exercise}[section]
% \newtheorem{problem}[exercise]{Problem}

\counterwithout{section}{chapter}
\usepackage[explicit]{titlesec}
\titleformat{\section}{\normalfont\Large\bfseries\scshape}{}{0em}{#1}
\newenvironment{problem}[1][]% environment name
{% begin code
  \par\vspace{\baselineskip}\noindent
  \ifx &#1&%
  \section{Problem~\hwnum.\thesection}
  \global\def\exercisename{Problem \hwnum.\thesection}%
  \else
  \section{Problem~\hwnum.\thesection~(#1)}
  \global\def\exercisename{Problem \hwnum.\thesection(#1)}
  \fi
  \par\vspace{\baselineskip}%
  \noindent\ignorespaces
}%
{% end code
  \par\vspace{\baselineskip}%
  \noindent\ignorespacesafterend
}

\newtheorem*{definition*}{Definition}
\newtheorem*{example*}{Examples}
\newtheorem*{examples*}{Examples}
\newtheorem*{exercise*}{Exercise}
\newtheorem*{problem*}{Problem}

\theoremstyle{remark}
\newtheorem{remark}{Remark}
\newtheorem{remarks}[remark]{Remarks}
\newtheorem{observation}[remark]{Observation}
\newtheorem{observations}[remark]{Observations}

\newtheorem*{remark*}{**Remark**}
\newtheorem*{remarks*}{**Remarks**}
\newtheorem*{observation*}{**Observation**}
\newtheorem*{observations*}{**Observations**}

%% Redefinitions & commands
% \renewcommand\qedsymbol{\ensuremath{\null\hfill\mdlgblksquare}}
\renewcommand\qedsymbol{\ensuremath{\null\hfill\QED}}
\newcommand\restr[2]{{% we make the whole thing an ordinary symbol
  \left.\kern-\nulldelimiterspace % automatically resize the bar with \right
  {#1} % the function
  % \vphantom{\big|} % pretend it's a little taller at normal size
  \right|{#2} % this is the delimiter
  }}

%% Operators & functions
\DeclareMathOperator{\id}{id}

%% Sets
\DeclareMathOperator{\im}{im}

%% Number sets
\DeclareMathOperator{\CC}{\mathbf{C}}
\DeclareMathOperator{\NN}{\mathbf{N}}
\DeclareMathOperator{\QQ}{\mathbf{Q}}
\DeclareMathOperator{\RR}{\mathbf{R}}
\DeclareMathOperator{\ZZ}{\mathbf{Z}}

\begin{document}
\frontmatter
\aliaspagestyle{title}{empty}
\pagestyle{title}
\author{\href{mailto:\authoremail}{\documentauthor}}
\title{\documenttitle}
\date{\today}
\maketitle
\cleartooddpage

\makeoddhead{headings}
        {\small{\MakeUppercase{\itshape\documentauthor}}}
        {}
        {\small{\MakeUppercase{\itshape\exercisename}}}
\makeoddfoot{headings}{{\itshape\documenttitle}}
                      {}
                      {\thepage}
\makeevenhead{headings}
        {\small{\MakeUppercase{\itshape\documentauthor}}}
        {}
        {\small{\MakeUppercase{\itshape\exercisename}}}
\makeevenfoot{headings}{{\itshape\documenttitle}}
                      {}
                      {\thepage}
\makeheadrule{headings}{\textwidth}{.25pt}
% \makerunningwidth{headings}{1.15\textwidth}
\pagestyle{headings}

\mainmatter
\begin{problem}[Munkres \S2, 1(a,b).]
Let $f\colon A\to B$. Let $A_0\subset A$ and $B_0\subset B$.
\begin{enumerate}[noitemsep,label=(\alph*)]
\item Show that $A_0\subset f^{-1}(f(A_0))$ and that equality
  holds if $f$ is injective.
\item Show that $f(f^{-1}(B_0))\subset B_0$ and that equality
  holds if $f$ is surjective.
\end{enumerate}
\end{problem}
\begin{proof}
(a). First, we will show $A_0\subset f^{-1}(f(A_0))$. Let $x\in
A_0$. Then $f(x)\in f(A_0)$. By definition, $f^{-1}(f(A_0))$ is
the set of those points $x_0\in A$ such that $f(x_0)\in f(A_0)$
and in particular we see that the containment $A_0\subset
f^{-1}(f(A_0))$ holds. Thus, $x\in f^{-1}(f(A_0))$.

Now, let us suppose the map $f$ is injective. By our former
argument, we have that $A_0\subset f^{-1}(f(A_0))$ therefore, we
will show the reverse containment. If $y\in f(A_0)$, then
$f(x)=y$ for some $x\in A_0$. By the injectivity of $f$, if
$f(x_0)=y$ for some $x_0\in A$, then we must have that
$x_0=x$. In particular, $x_0\in A_0$. Thus $f^{-1}(f(A_0))\subset
A_0$ and equality $A_0=f^{-1}(f(A_0))$ holds.
\\\\
(b). First, we will show that $f(f^{-1}(B_0))\subset
B_0$. Consider the preimage $f^{-1}(B_0)$ of $B_0$. Let $x\in
f^{-1}(B_0)$. Then $f(x)=y$ for some $y\in B_0$. Since
$f(f^{-1}(B_0))$ is, by definition, the set of all points
$f(x)\in B$ where $x\in f^{-1}(B_0)$ and $f(x)=y$ for $y\in B_0$,
we have that $f(f^{-1}(B_0))\subset B_0$.

Now, let us suppose the map $f$ is surjective. Let $y\in B_0$,
then there exists $x\in A$ such that $f(x)=y$. Thus, $x\in
f^{-1}(B_0)$. Then $y=f(x)\in f(f^{-1}(B_0))$ (in particular
$B_0\subset f(f^{-1}(B_0))$) and we have equality
$B_0=f(f^{-1}(B_0))$.
\end{proof}
\newpage

\begin{problem}[Munkres, \S2, 2(g).]
Let $f\colon A\to B$ and let $A_i\subset A$ and $B_i\subset B$
for $i=0$ and $i=1$. Show that $f^{-1}$ preserves inclusion,
unions, intersections, and differences of sets:
\begin{enumerate}[noitemsep]
\item[(g)] $f(A_0\cap A_1)\subset f(A_0)\cap f(A_1)$; show that
  equality holds if $f$ is injective.
\end{enumerate}
\end{problem}
\begin{proof}[Proof of (g)]
The claim is evident if $A_0$ and $A_1$ are disjoint
subsets. Suppose $A_0\cap A_1\neq\emptyset$. Let $y\in f(A_0\cap
A_1)$. Then $y=f(x)$ for some $x\in A_0$, $x\in A_1$. Then
$f(x)\in f(A_0)$ and $f(x)\in f(A_1)$ so $y\in f(A_0)\cap
f(A_1)$. Thus, $f(A_0\cap A_1)\subset f(A_0)\cap f(A_1)$.

Now, suppose $f$ is injective. Then, if $f(x)=f(x')=y$ for some
$y\in B$, then $x=x'$. Let $y\in f(A_0)\cap f(A_1)$. Then
$y=f(x_0)$, $y=f(x_1)$ for some $x_0\in A_0$, $x_1\in A_1$. But,
by the injectivity of $f$, $x_0=x_1$ so $x_0\in A_0\cap
A_1$. Hence, $y\in f(A_0\cap A_1)$ and the equality $f(A_0\cap
A_1)=f(A_0)\cap f(A_1)$ holds.
\end{proof}
\newpage

\begin{problem}[Munkres, \S13, 3.]
Show that the collection ${\mathcal{T}}_c$ given in Example 4 of
\S12 is a topology on the set $X$. Is the collection
\[
{\mathcal{T}}_\infty=
\left\{\,U\;\middle|\;
\text{$X\smallsetminus U$ is infinite or empty or all of $X$}
\,\right\}
\]
a topology on $X$?
\end{problem}
\begin{proof}

\end{proof}
\newpage

\begin{problem}[Munkres, \S13, 5.]
Show that if $\mathcal{A}$ is a basis for a topology on $X$, then
the topology generated by $\mathcal{A}$ equals the intersection
of all topologies on $X$ that contain $\mathcal{A}$. Prove the
same if $\mathcal{A}$ is a subbasis.
\end{problem}
\begin{proof}
\end{proof}
\newpage

\begin{problem}[Munkres, \S13, 8(b).]
\begin{enumerate}[noitemsep]
\item[(b)] Show that the collection
\[\mathcal{C}=\left\{\,[a,b)\;\middle|\;\text{$a<b$, $a$ and $b$ rational}\right\}\]
is a basis that generates a topology different from the lower
limit topology on ${\RR}$.
\end{enumerate}
\end{problem}
\newpage

\begin{problem}[Munkres, \S16, 1.]
Show that if $Y$ is a subspace of $X$, and $A$ is a subset of
$Y$, then the topology $A$ inherits as a subspace of $Y$ is the
same as the topology it inherits as a subspace of $X$.
\end{problem}
\begin{proof}
\end{proof}
\newpage

\begin{problem}[Munkres, \S16, 4.]
A map $f\colon X\to Y$ is said to be an \emph{open map} if for
every open set $U$ of $X$, the set $f(U)$ is open in $Y$. Show
that $\pi_1\colon X\times Y\to X$ and $\pi_2\colon X\times Y\to
Y$ are open maps.
\end{problem}
\begin{proof}
\end{proof}
\newpage

\begin{problem}[Munkres, \S16, 6.]
Show that the countable collection
\[
\left\{\,(a,b)\times(c,d)\;\middle|\;
\text{$a<b$ and $c<d$, and $a,b,c,d$ are rational}\,\right\}
\]
is a basis for ${\RR}^2$.
\end{problem}
\begin{proof}
\end{proof}
\newpage

\begin{problem}[Munkres, \S16, 9.]
Show that the dictionary order topology on the set
${\RR}\times{\RR}$ is the same as the product topology
${\RR}_d\times{\RR}$, where ${\RR}_d$ denotes ${\RR}$ in the
discrete topology. Compare this topology with the standard
topology on ${\RR}^2$.
\end{problem}
\begin{proof}
\end{proof}

%%% Local Variables:
%%% mode: latex
%%% TeX-master: "../MA571-HW-Current"
%%% End:

\end{document}

%%% Local Variables:
%%% mode: latex
%%% TeX-master: t
%%% End:
