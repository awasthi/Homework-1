\def\documentauthor{Carlos Salinas}
\def\documenttitle{MA571 Problem Set \hwnum}
\def\hwnum{2}
\def\shorttitle{MA571 PSet \hwnum}
\def\coursename{MA571}
\def\documentsubject{point-set topology}
\def\authoremail{salinac@purdue.edu}

\documentclass[article,oneside,11pt]{memoir}
\usepackage{geometry}
\usepackage[dvipsnames]{xcolor}
\usepackage[
    breaklinks,
    bookmarks=true,
    colorlinks=true,
    pageanchor=false,
    linkcolor=black,
    anchorcolor=black,
    citecolor=black,
    filecolor=black,
    menucolor=black,
    runcolor=black,
    urlcolor=black,
    hyperindex=false,
    hyperfootnotes=true,
    pdftitle={\shorttitle},
    pdfauthor={\documentauthor},
    pdfkeywords={\documentsubject},
    pdfsubject={\coursename}
    ]{hyperref}

\usepackage{graphicx}
\graphicspath{{figures/}}

\usepackage{multicol}
\usepackage[inline]{enumitem}
\usepackage{listings}
\usepackage{mleftright}
\mleftright

\usepackage{microtype}

\usepackage{amsthm}
\usepackage{amssymb}
\usepackage{mathtools}
\usepackage{lualatex-math}
\usepackage{unicode-math}
% \usepackage{unicode-minionmath}

% \setmainfont[Ligatures=TeX]{Minion Pro}
% \setsansfont[Ligatures=TeX]{Myriad Pro}
% \setmonofont{Courier Std}

% \setmathfont{Minion Math}
% \setmathfont[range={\mathfrak}]{XITS Math}
% \setmathfont[range={\mathcal},StylisticSet=1]{XITS Math}
% \setmathfont[range={\mathscr}]{XITS Math}
% \setmathfont[range={}]{Minion Math}

\setmainfont[Ligatures=TeX]{Latin Modern Roman}
\setsansfont[Ligatures=TeX]{Latin Modern Sans}
\setmonofont{Latin Modern Mono}
\setmathfont{Latin Modern Math}

% \setmainfont[Ligatures=TeX]{CMU Serif}
% \setsansfont[Ligatures=TeX]{CMU Sans Serif}
% \setmonofont{CMU Typewriter Text}

% \setmainfont[Ligatures=TeX]{XITS}
% \setmathfont[range={\mathcal,\mathbfcal},StylisticSet=1]{XITS Math}
% \setmathfont[range=\mathscr]{XITS Math}
% \setmathfont{XITS Math}

% \setmainfont[Ligatures=TeX]{Linux Libertine O}
% \setsansfont[Ligatures=TeX]{Linux Biolinum O}
% \setmonofont{Linux Libertine Mono O}

\usepackage{polyglossia}
\setmainlanguage[variant=american]{english}
\setotherlanguage{french}
\setotherlanguage[spelling=new,latesthyphen,babelshorthands]{german}
\setotherlanguage[spelling=modern,babelshorthands]{russian}
\setotherlanguage{spanish}

\newfontfamily\cyrillicfont[Ligatures=TeX]{CMU Serif}
\newfontfamily\cyrillicfontsf[Ligatures=TeX]{CMU Sans Serif}
\newfontfamily\cyrillicfonttt{CMU Typewriter Text}

\theoremstyle{plain}
\newtheorem{theorem}{Theorem}
\newtheorem{proposition}[theorem]{Proposition}
\newtheorem{corollary}[theorem]{Corollary}
\newtheorem{claim}[theorem]{Claim}
\newtheorem{lemma}[theorem]{Lemma}
\newtheorem{axiom}[theorem]{Axiom}

\newtheorem*{corollary*}{Corollary}
\newtheorem*{claim*}{Claim}
\newtheorem*{lemma*}{Lemma}
\newtheorem*{proposition*}{Proposition}
\newtheorem*{theorem*}{Theorem}

\theoremstyle{definition}
\newtheorem{definition}{Definition}
\newtheorem{example}{Examples}
\newtheorem{examples}[example]{Examples}
% \newtheorem{exercise}{Exercise}[section]
% \newtheorem{problem}[exercise]{Problem}

\counterwithout{section}{chapter}
\usepackage[explicit]{titlesec}
\titleformat{\section}{\normalfont\Large\bfseries\scshape}{}{0em}{#1}
\newenvironment{problem}[1][]% environment name
{% begin code
  \par\vspace{\baselineskip}\noindent
  \ifx &#1&%
  \section{Problem~\hwnum.\thesection}
  \global\def\exercisename{Problem \hwnum.\thesection}%
  \else
  \section{Problem~\hwnum.\thesection~(#1)}
  \global\def\exercisename{Problem \hwnum.\thesection(#1)}
  \fi
  \par\vspace{\baselineskip}%
  \noindent\ignorespaces
}%
{% end code
  \par\vspace{\baselineskip}%
  \noindent\ignorespacesafterend
}

\newtheorem*{definition*}{Definition}
\newtheorem*{example*}{Examples}
\newtheorem*{examples*}{Examples}
\newtheorem*{exercise*}{Exercise}
\newtheorem*{problem*}{Problem}

\theoremstyle{remark}
\newtheorem{remark}{Remark}
\newtheorem{remarks}[remark]{Remarks}
\newtheorem{observation}[remark]{Observation}
\newtheorem{observations}[remark]{Observations}

\newtheorem*{remark*}{**Remark**}
\newtheorem*{remarks*}{**Remarks**}
\newtheorem*{observation*}{**Observation**}
\newtheorem*{observations*}{**Observations**}

%% Redefinitions & commands
\renewcommand\qedsymbol{\ensuremath{\null\hfill\QED}}

\newcommand\restr[2]{{% we make the whole thing an ordinary symbol
  \left.\kern-\nulldelimiterspace % automatically resize the bar with \right
  {#1} % the function
  % \vphantom{\big|} % pretend it's a little taller at normal size
  \right|_{#2} % this is the delimiter
  }}

%% Commands and operators
\newcommand{\id}{\mathrm{id}}
\newcommand{\im}{\mathrm{im}}
\newcommand{\Int}{\mathrm{int}}
\newcommand{\Cl}{\mathrm{cl}}
\newcommand{\clsr}[1]{\overline{#1}}
\newcommand{\CC}{\mathbf{C}}
\newcommand{\NN}{\mathbf{N}}
\newcommand{\QQ}{\mathbf{Q}}
\newcommand{\RR}{\mathbf{R}}
\newcommand{\ZZ}{\mathbf{Z}}

\begin{document}
\let\setminus\relax
\let\phi\relax
\let\epsilon\relax
\newcommand\setminus{\smallsetminus}
\newcommand\phi{\varphi}
\newcommand\epsilon{\varepsilon}

\frontmatter
\aliaspagestyle{title}{empty}
\pagestyle{title}
\author{\href{mailto:\authoremail}{\documentauthor}}
\title{\documenttitle}
\date{\today}
\maketitle
\cleartooddpage

\makeoddhead{headings}
        {\small{\MakeUppercase{\itshape\documentauthor}}}
        {}
        {\small{\MakeUppercase{\itshape\exercisename}}}
\makeoddfoot{headings}{{\itshape\documenttitle}}
                      {}
                      {\thepage}
\makeevenhead{headings}
        {\small{\MakeUppercase{\itshape\documentauthor}}}
        {}
        {\small{\MakeUppercase{\itshape\exercisename}}}
\makeevenfoot{headings}{{\itshape\documenttitle}}
                      {}
                      {\thepage}
\makeheadrule{headings}{\textwidth}{.25pt}
% \makerunningwidth{headings}{1.15\textwidth}
\pagestyle{headings}

\mainmatter
\begin{problem}[Munkres, \S17, p.\,100, 3]
Show that if $A$ is closed in $Y$ and $Y$ is closed in $X$, then
$A$ is closed in $X$.
\end{problem}
\begin{proof}
\end{proof}
\newpage
\begin{problem}[Munkres, \S17, p.\,101, 6(b)]
Let $A$, $B$ and $A_\alpha$ denote subsets of a space $X$. Prove
the following:
\begin{enumerate}[noitemsep]
\item[(b)] $\clsr{A\cup B}=\bar A\cup\bar B$.
\end{enumerate}
\end{problem}
\begin{proof}
\end{proof}
\newpage
\begin{problem}[Munkres, \S17, p.\,101, 6(c)]
Let $A$, $B$ and $A_\alpha$ denote subsets of a space $X$. Prove
the following:
\begin{enumerate}[noitemsep]
\item[(b)] $\clsr{\bigcup A_\alpha}\supset\bigcup\clsr{A_\alpha}$.
\end{enumerate}
\end{problem}
\begin{proof}
\end{proof}
\newpage
\begin{problem}[Munkres, \S17, p.\,101, 7]
Criticize the following ``proof'' that $\clsr{\bigcup
  A_\alpha}\subset\bigcup\bar A_\alpha$: if
$\left\{A_\alpha\right\}$ is a collection of sets in $X$ and if
$x\in\clsr{\bigcup A_\alpha}$, then every neighborhood $U$ of $x$
intersects $\bigcup A_\alpha$. Thus $U$ must intersect some
$A_\alpha$, so $x$ must belong to the closure of some
$A_\alpha$. Therefore, $x\in\bigcup\bar A_\alpha$.
\end{problem}
\begin{proof}[Critique]
\end{proof}
\newpage
\begin{problem}[Munkres, \S17, p.\,101, 9]
Let $A\subset X$ and $B\subset Y$. Show that in the space
$X\times Y$,
\[
\clsr{A\times B}=\bar A\times\bar B.
\]
\end{problem}
\begin{proof}
\end{proof}
\newpage
\begin{problem}[Munkres, \S17, p.\,101, 10]
Show that every order topology is Hausdorff.
\end{problem}
\begin{proof}
\end{proof}
\newpage
\begin{problem}[Munkres, \S17, p.\,101, 13]
Show that $X$ is Hausdorff if and only if the \emph{diagonal}
$\Delta=\left\{\,x\times x\;\middle|\;x\in X\,\right\}$ is closed
in $X\times X$.
\end{problem}
\begin{proof}
\end{proof}
\newpage
\begin{problem}[Munkres, \S18, p.\,111, 4]
Given $x_0\in X$ and $y_0\in Y$, show that the maps $f\colon X\to
X\times Y$ and $g\colon Y\to X\times Y$ defined by
\[
f(x)=x\times y_0\quad\text{and}\quad g(y)=x_0\times y
\]
are imbeddings.
\end{problem}
\begin{proof}
\end{proof}
\newpage
\begin{problem}[Munkres, \S18, p.\,111-112, 8(a,b)]
Let $Y$ be an ordered set in the order topology. Let
$f,g\colon X\to Y$ be continuous.
\begin{enumerate}[noitemsep,label=(\alph*)]
\item Show that the set
  $\left\{\,x\;\middle|\;f(x)\leq g(x)\,\right\}$ is closed in $X$.
\item Let $h\colon X\to Y$ be the
  function \[h(x)=\min\{f(x),g(x)\}.\] Show that $h$ is
  continuous. [\emph{Hint:} Use the pasting lemma.]
\end{enumerate}
\end{problem}
\begin{proof}
\end{proof}
\newpage
\begin{problem}
Given: $X$ is a topological space with open sets $U_1,...,U_n$
such that $\bar U_i=X$ for all $i$. Prove that the closure of
$U_1\cap\cdots\cap U_n$ is $X$.
\end{problem}
\begin{proof}
\end{proof}

%%% Local Variables:
%%% mode: latex
%%% TeX-master: "../MA571-HW-Current"
%%% End:

\end{document}

%%% Local Variables:
%%% mode: latex
%%% TeX-master: t
%%% End:
