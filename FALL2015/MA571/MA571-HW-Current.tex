\def\documentauthor{Carlos Salinas}
\def\documenttitle{MA571 Homework \hwnum}
\def\hwnum{9}
\def\shorttitle{MA571 HW \hwnum}
\def\coursename{MA571}
\def\documentsubject{point-set topology}
\def\authoremail{salinac@purdue.edu}

\documentclass[article,oneside,10pt]{memoir}
\usepackage{geometry}
\usepackage[dvipsnames]{xcolor}
\usepackage[
    breaklinks,
    bookmarks=true,
    colorlinks=true,
    pageanchor=false,
    linkcolor=black,
    anchorcolor=black,
    citecolor=black,
    filecolor=black,
    menucolor=black,
    runcolor=black,
    urlcolor=black,
    hyperindex=false,
    hyperfootnotes=true,
    pdftitle={\shorttitle},
    pdfauthor={\documentauthor},
    pdfkeywords={\documentsubject},
    pdfsubject={\coursename}
    ]{hyperref}

\usepackage{graphicx}
\graphicspath{{figures/}}

% Misc
\usepackage{microtype}
\usepackage{multicol}
\usepackage[inline]{enumitem}
\usepackage{listings}
\usepackage{mleftright}
\mleftright

%% Math
\usepackage{amsthm}
\usepackage{amssymb}
\usepackage{mathtools}
% \usepackage{unicode-math}

%% PDFTeX specific
% \usepackage[mathcal]{euscript}
% \usepackage{mathrsfs}

\usepackage{cmap}
\usepackage[T2A,T1]{fontenc}
\usepackage[utf8]{inputenc}
\usepackage[french,german,russian,spanish,english]{babel}
\babeltags{fr=french,
           de=german,
           ru=russian,
           es=spanish,
           en=english}
\def\spanishoptions{mexico}
\usepackage{CJKutf8}
\newcommand{\textha}[1]{\begin{CJK}{UTF8}{mj}#1\end{CJK}}
\newcommand{\textni}[1]{\begin{CJK}{UTF8}{min}#1\end{CJK}}
\newcommand{\textzh}[1]{\begin{CJK}{UTF8}{bsmi}#1\end{CJK}}

%% Theorems and definitions
\theoremstyle{plain}
\newtheorem{theorem}{Theorem}
\newtheorem{proposition}[theorem]{Proposition}
\newtheorem{corollary}[theorem]{Corollary}
\newtheorem{claim}[theorem]{Claim}
\newtheorem{lemma}[theorem]{Lemma}
\newtheorem{axiom}[theorem]{Axiom}

\newtheorem*{corollary*}{Corollary}
\newtheorem*{claim*}{Claim}
\newtheorem*{lemma*}{Lemma}
\newtheorem*{proposition*}{Proposition}
\newtheorem*{theorem*}{Theorem}

\theoremstyle{definition}
\newtheorem{definition}{Definition}
\newtheorem{example}{Examples}
\newtheorem{examples}[example]{Examples}
% \newtheorem{exercise}{Exercise}[section]
% \newtheorem{problem}[exercise]{Problem}

% \newtheorem{exercise}{Exercise}[section]
% \newtheorem{problem}[exercise]{Problem}

\newcounter{problem}
\newenvironment{problem}[1][]% environment name
{% begin code
  \stepcounter{problem}
  \par\vspace{\baselineskip}\noindent
  \ifx &#1&%
  {\normalfont\Large\bfseries\scshape Problem~\hwnum.\theproblem}
  \global\def\exercisename{Problem~\hwnum.\theproblem}%
  \else
  {\normalfont\Large\bfseries\scshape Problem~\hwnum.\theproblem~(#1)}
  \global\def\exercisename{Problem~\hwnum.\theproblem(#1)}
  \fi
  \par\vspace{\baselineskip}%
  \noindent\ignorespaces
}%
{% end code
  \par\vspace{\baselineskip}%
  \noindent\ignorespacesafterend
}

\newtheorem*{definition*}{Definition}
\newtheorem*{example*}{Examples}
\newtheorem*{examples*}{Examples}
\newtheorem*{exercise*}{Exercise}
\newtheorem*{problem*}{Problem}

\theoremstyle{remark}
\newtheorem{remark}{Remark}
\newtheorem{remarks}[remark]{Remarks}
\newtheorem{observation}[remark]{Observation}
\newtheorem{observations}[remark]{Observations}

\newtheorem*{remark*}{**Remark**}
\newtheorem*{remarks*}{**Remarks**}
\newtheorem*{observation*}{**Observation**}
\newtheorem*{observations*}{**Observations**}

%% Redefinitions & commands
\newcommand\restr[2]{{% we make the whole thing an ordinary symbol
  \left.\kern-\nulldelimiterspace % automatically resize the bar with \right
  {#1} % the function
  % \vphantom{\big|} % pretend it's a little taller at normal size
  \right|{#2} % this is the delimiter
  }}

\newcommand{\nsubset}{\ensuremath{\not\subset}}
\newcommand{\nsupset}{\ensuremath{\not\supset}}
\renewcommand\qedsymbol{\ensuremath{\null\hfill\blacksquare}}

%% Commands and operators
\DeclareMathOperator{\diam}{diam}
\DeclareMathOperator{\id}{id}
\DeclareMathOperator{\im}{im}
\DeclareMathOperator{\Int}{int}
\DeclareMathOperator{\Cl}{cl}

\newcommand{\CC}{\mathbf{C}}
\newcommand{\NN}{\mathbf{N}}
\newcommand{\QQ}{\mathbf{Q}}
\newcommand{\RR}{\mathbf{R}}
\newcommand{\ZZ}{\mathbf{Z}}

% Renewcommands
\renewcommand\setminus{\smallsetminus}
\renewcommand\phi{\varphi}
\renewcommand\epsilon{\varepsilon}

\begin{document}
\frontmatter
\aliaspagestyle{title}{empty}
\pagestyle{title}
\author{\href{mailto:\authoremail}{\documentauthor}}
\title{\documenttitle}
\date{\today}
\maketitle
\cleartooddpage

\makeoddhead{headings}
        {\small{\MakeUppercase{\itshape\documentauthor}}}
        {}
        {\small{\MakeUppercase{\itshape\exercisename}}}
\makeoddfoot{headings}{{\itshape\documenttitle}}
                      {}
                      {\thepage}
\makeevenhead{headings}
        {\small{\MakeUppercase{\itshape\documentauthor}}}
        {}
        {\small{\MakeUppercase{\itshape\exercisename}}}
\makeevenfoot{headings}{{\itshape\documenttitle}}
                      {}
                      {\thepage}
\makeheadrule{headings}{\textwidth}{.25pt}
% \makerunningwidth{headings}{1.15\textwidth}
\pagestyle{headings}

\mainmatter
% \setcounter{theorem}{16}
\begin{problem}[Munkres \S46, Ex.\,6]
Show that the compact-open topology, $\mathcal{C}(X,Y)$ is
Hausdorff if $Y$ is Hausdorff, and regular if $Y$ is
regular. [\emph{Hint:} If $\overline U\subset V$, then
$\overline{S(C,U)}\subset S(C,V)$.]
\end{problem}
\begin{proof}
Suppose that $Y$ is regular. We shall proceed by the
hint and Lemma 31.1(b). Consider the subbasis element
$S(C,U)$. Since $Y$ is regular, there exists a neighborhood
$V\supset U$ such that $V\supset\overline{U}$. Let
$f\in\overline{S(C,U)}$. Then, we claim that $f\in S(C,V)$. For
suppose not, then there exists an element $x_0\in C$ such that
$f(x_0)\notin V$. Then, since $\overline{U}\subset V$, by
hypothesis, $f(x_0)\notin\overline{U}$. Consider the subbasic
neighborhood $S\left(\left\{x_0\right\},Y-\overline{U}\right)$ of
$f$. Then, $S\left(\left\{x_0\right\},Y-\overline{U}\right)\cap
S(C,U)$ is nonempty. Let $g$ be in the aforementioned
intersection. Then $g(x_0)\in g(C)\subset U$, but $g(x_0)\in
Y-\overline{U}$. This is a contradiction. It follows by Lemma
31.1(b) that $\mathcal{C}(X,Y)$ is regular.
\end{proof}
\newpage
\begin{problem}[Munkres \S46, Ex.\,9(a,b,c)]
Here is a (unexpected) application of Theorem 46.11 to quotient
maps. (Compare Exercise 11 of \S29.)
\begin{theorem*}
If $p\colon A\to B$ is a quotient map and $X$ is locally compact
Hausdorff, then $(\id_X,p)\colon X\times A\to X\times B$ is a
quotient map.
\begin{proof}
\renewcommand\qedsymbol{\null}
\begin{enumerate}[label=(\alph*)]
\item Let $Y$ be the quotient space induced by $(\id_X,p)$; let
  $q\colon X\times A\to Y$ be the quotient map. Show there is a
  bijective continuous map $f\colon Y\to X\times B$ such that
  $f\circ q=(\id_X,p)$.
\item Let $g=f^{-1}$. Let $G\colon B\to\mathcal{C}(X,Y)$ and
  $Q\colon A\to\mathcal{C}(X,Y)$ be the maps induced by $g$ and
  $q$, respectively. Show that $Q=G\circ p$.
\item Show that $Q$ is continuous; conclude that $G$ is
  continuous, so that $g$ is continuous.
\end{enumerate}
\end{proof}
\end{theorem*}
\end{problem}
\begin{proof}
\end{proof}
\newpage
\begin{problem}[Munkres \S52, Ex.\,1]
Show that if $h,h'\colon X\to Y$ are homotopic and $k,k'\colon
Y\to Z$ are homotopic, then $k\circ h$ and $k'\circ h'$ are
homotopic.
\end{problem}
\begin{proof}
\end{proof}
\newpage
\begin{problem}[Munkres \S52, Ex.\,2]
Given spaces $X$ and $Y$, let $[X,Y]$ denote the homotopy classes
of maps of $X$ into $Y$
\begin{enumerate}[label=(\alph*)]
\item Let $I=[0,1]$. Show that for any $X$, the set $[X,I]$ has a
  single element.
\item Show that if $Y$ is path connected, the set $[I,Y]$ has a
  single element.
\end{enumerate}
\end{problem}
\begin{proof}
\end{proof}
\newpage
\begin{problem}[Munkres \S52, Ex.\,3(a,b,c,)]
A space $X$ is said to be \emph{contractible} if the identity map
$\id_X\colon X\to X$ is nulhomotopic.
\begin{enumerate}[label=(\alph*)]
\item Show that $I$ and $\RR$ are contractible.
\item Show that a contractible space is path connected.
\item Show that f $Y$ is contractible, then for any $X$, the set
  $[X,Y]$ has a single element.
\end{enumerate}
\end{problem}
\begin{proof}
\end{proof}

%%% Local Variables:
%%% mode: latex
%%% TeX-master: "../MA571-HW-Current"
%%% End:

\end{document}

%%% Local Variables:
%%% mode: latex
%%% TeX-master: t
%%% End:
