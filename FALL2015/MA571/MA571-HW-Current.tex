\def\documentauthor{Carlos Salinas}
\def\documenttitle{MA571 Problem Set \hwnum}
\def\hwnum{7}
\def\shorttitle{MA571 PSet \hwnum}
\def\coursename{MA571}
\def\documentsubject{point-set topology}
\def\authoremail{salinac@purdue.edu}

\documentclass[article,oneside,10pt]{memoir}
\usepackage{geometry}
\usepackage[dvipsnames]{xcolor}
\usepackage[
    breaklinks,
    bookmarks=true,
    colorlinks=true,
    pageanchor=false,
    linkcolor=black,
    anchorcolor=black,
    citecolor=black,
    filecolor=black,
    menucolor=black,
    runcolor=black,
    urlcolor=black,
    hyperindex=false,
    hyperfootnotes=true,
    pdftitle={\shorttitle},
    pdfauthor={\documentauthor},
    pdfkeywords={\documentsubject},
    pdfsubject={\coursename}
    ]{hyperref}

\usepackage{graphicx}
\graphicspath{{figures/}}

% Misc
\usepackage{microtype}
\usepackage{multicol}
\usepackage[inline]{enumitem}
\usepackage{listings}
\usepackage{mleftright}
\mleftright

%% Math
\usepackage{amsthm}
\usepackage{amssymb}
\usepackage{mathtools}

%% PDFTeX specific
\usepackage[mathcal]{euscript}
\usepackage{mathrsfs}

\usepackage{cmap}
\usepackage[T2A,T1]{fontenc}
\usepackage[utf8]{inputenc}
\usepackage[french,german,russian,spanish,english]{babel}
\babeltags{fr=french,
           de=german,
           ru=russian,
           es=spanish,
           en=english}
\def\spanishoptions{mexico}
\usepackage{CJKutf8}
\newcommand{\textha}[1]{\begin{CJK}{UTF8}{mj}#1\end{CJK}}
\newcommand{\textni}[1]{\begin{CJK}{UTF8}{min}#1\end{CJK}}
\newcommand{\textzh}[1]{\begin{CJK}{UTF8}{bsmi}#1\end{CJK}}

%% Theorems and definitions
\theoremstyle{plain}
\newtheorem{theorem}{Theorem}
\newtheorem{proposition}[theorem]{Proposition}
\newtheorem{corollary}[theorem]{Corollary}
\newtheorem{claim}[theorem]{Claim}
\newtheorem{lemma}[theorem]{Lemma}
\newtheorem{axiom}[theorem]{Axiom}

\newtheorem*{corollary*}{Corollary}
\newtheorem*{claim*}{Claim}
\newtheorem*{lemma*}{Lemma}
\newtheorem*{proposition*}{Proposition}
\newtheorem*{theorem*}{Theorem}

\theoremstyle{definition}
\newtheorem{definition}{Definition}
\newtheorem{example}{Examples}
\newtheorem{examples}[example]{Examples}
% \newtheorem{exercise}{Exercise}[section]
% \newtheorem{problem}[exercise]{Problem}

% \newtheorem{exercise}{Exercise}[section]
% \newtheorem{problem}[exercise]{Problem}

\newcounter{problem}
\newenvironment{problem}[1][]% environment name
{% begin code
  \stepcounter{problem}
  \par\vspace{\baselineskip}\noindent
  \ifx &#1&%
  {\normalfont\Large\bfseries\scshape Problem~\hwnum.\theproblem}
  \global\def\exercisename{Problem~\hwnum.\theproblem}%
  \else
  {\normalfont\Large\bfseries\scshape Problem~\hwnum.\theproblem~(#1)}
  \global\def\exercisename{Problem~\hwnum.\theproblem(#1)}
  \fi
  \par\vspace{\baselineskip}%
  \noindent\ignorespaces
}%
{% end code
  \par\vspace{\baselineskip}%
  \noindent\ignorespacesafterend
}

\newtheorem*{definition*}{Definition}
\newtheorem*{example*}{Examples}
\newtheorem*{examples*}{Examples}
\newtheorem*{exercise*}{Exercise}
\newtheorem*{problem*}{Problem}

\theoremstyle{remark}
\newtheorem{remark}{Remark}
\newtheorem{remarks}[remark]{Remarks}
\newtheorem{observation}[remark]{Observation}
\newtheorem{observations}[remark]{Observations}

\newtheorem*{remark*}{**Remark**}
\newtheorem*{remarks*}{**Remarks**}
\newtheorem*{observation*}{**Observation**}
\newtheorem*{observations*}{**Observations**}

%% Redefinitions & commands
\newcommand\restr[2]{{% we make the whole thing an ordinary symbol
  \left.\kern-\nulldelimiterspace % automatically resize the bar with \right
  {#1} % the function
  % \vphantom{\big|} % pretend it's a little taller at normal size
  \right|{#2} % this is the delimiter
  }}

\newcommand{\nsubset}{\ensuremath{\not\subset}}

\renewcommand\qedsymbol{\ensuremath{\null\hfill\blacksquare}}

%% Commands and operators
\DeclareMathOperator{\diam}{diam}
\DeclareMathOperator{\id}{id}
\DeclareMathOperator{\im}{im}
\DeclareMathOperator{\Int}{int}
\DeclareMathOperator{\Cl}{cl}

\newcommand{\CC}{\mathbf{C}}
\newcommand{\NN}{\mathbf{N}}
\newcommand{\QQ}{\mathbf{Q}}
\newcommand{\RR}{\mathbf{R}}
\newcommand{\ZZ}{\mathbf{Z}}

% Renewcommands
\renewcommand\setminus{\smallsetminus}
\renewcommand\phi{\varphi}
\renewcommand\epsilon{\varepsilon}

\begin{document}
\frontmatter
\aliaspagestyle{title}{empty}
\pagestyle{title}
\author{\href{mailto:\authoremail}{\documentauthor}}
\title{\documenttitle}
\date{\today}
\maketitle
\cleartooddpage

\makeoddhead{headings}
        {\small{\MakeUppercase{\itshape\documentauthor}}}
        {}
        {\small{\MakeUppercase{\itshape\exercisename}}}
\makeoddfoot{headings}{{\itshape\documenttitle}}
                      {}
                      {\thepage}
\makeevenhead{headings}
        {\small{\MakeUppercase{\itshape\documentauthor}}}
        {}
        {\small{\MakeUppercase{\itshape\exercisename}}}
\makeevenfoot{headings}{{\itshape\documenttitle}}
                      {}
                      {\thepage}
\makeheadrule{headings}{\textwidth}{.25pt}
% \makerunningwidth{headings}{1.15\textwidth}
\pagestyle{headings}

\mainmatter
% \setcounter{theorem}{14}
\begin{problem}[Munkres \S26, Ex.\,8]
\begin{theorem*}
Let $f\colon X\to Y$; let $Y$ be compact Hausdorff. Then $f$ is
continuous if and only if the \emph{graph} of $f$,
\[
G_f=\left\{\,(x,f(x))\;\middle|\;x\in X\,\right\},
\]
is closed in $X\times Y$.
\end{theorem*}
[\emph{Hint:} If $G_f$ is closed and $V$ is a neighborhood of
$f(x_0)$, then the intersection of $G_f$ and $X\times(Y-V)$ is
closed. Apply Exercise 7.]
\end{problem}
\begin{proof}
As we demonstrated in Problem 2.7 (Munkres \S18, Ex.\,17) $Y$ is
Hausdorff if and only if the diagonal,
$\Delta_Y=\left\{\,(y,y)\;\middle|\;y\in Y\,\right\}$, is a closed
subset of $Y\times Y$. Consider the map $F\colon X\times Y\to Y\times
Y$ defined by $(x,y)\mapsto (f(x),y)$. This map is continuous by
Theorem 18.4 as $f$ is, by assumption, continuous and $\id_Y$ is
continuous by 18.2(b) (since it is the inclusion $Y\hookrightarrow
Y$). Then
\begin{align*}
F^{-1}(\Delta_Y)
&=\left\{\,(x,y)\;\middle|\;\text{$F(x,y)\in\Delta_Y$, $x\in X$, $y\in
  Y$}\,\right\}\\
&=\left\{\,(x,y)\;\middle|\;\text{$(f(x),y)\in\Delta_Y$, $x\in X$,
  $y\in Y$}\,\right\}\\
&=\left\{\,(x,y)\;\middle|\;\text{$f(x)=y$, $x\in X$, $y\in Y$}\,\right\}\\
&=\left\{\,(x,f(x))\;\middle|\;\text{$x\in X$, $y\in Y$}\,\right\}\\
&=G_f
\end{align*}
is closed by Theorem 18.1(3).

Conversely, suppose $G_f$ is closed in $X\times Y$. Fix a point
$x_0\in X$ and let $V\subset Y$ be an arbitrary neighborhood of
$f(x_0)$. Then $Y-V$ is a closed subset of $Y$ so, by Problem 2.1
(Munkres \S17, Ex.\,3), the product $X\times (Y-V)$ is closed in
$Y\times Y$. In particular, by Theorem 17.1(2), the intersection
$B=G_f\cap X\times(Y-V)$ is closed in $X\times Y$. Thus, by Problem
6.5 (Munkres \S26, Ex.\,7), since $Y$ is a compact Hausdorff space,
the projection $\pi_1(B)$ onto $X$ is a closed subset of $X$. But
\begin{align*}
B
&=\left\{\,(x,y)\;\middle|\;\text{$(x,y)\in G_f$ and $(x,y)\in X\times
  (Y-V)$}\,\right\}\\
&=\left\{\,(x,y)\;\middle|\;\text{$y=f(x)$ and $(x,y)\in X\times
  (Y-V)$}\right\}\\
&=\left\{\,(x,f(x))\;\middle|\;f(x)\in Y-V\,\right\}
\end{align*}
so we have that $\pi_1(B)=f^{-1}(Y-V)=X-f^{-1}(V)$. One
containment is easy to see, namely ``$\subset$'': if $x\in B$ then
$x=\pi_1(x,f(x))$ for at least one $f(x)\in Y-V$. To see the reverse
inclusion, take $x\in f^{-1}(Y-V)$, then $f(x)\in Y-V$ so $(x,f(x))\in
B$, hence $x\in\pi_1(B)$. Thus, $X-\pi_1(B)=f^{-1}(V)$ is open
so $f$ is continuous.
\end{proof}
\newpage
\begin{problem}[Munkres \S26, Ex.\,9]
Generalize the tube lemma as follows:
\begin{theorem*}
Let $A$ and $B$ be subspaces of $X$ and $Y$, respectively; let
$N$ be an open set in $X\times Y$ containing $A\times B$. If $A$
and $B$ are compact, then there exist open sets $U$ and $V$ in
$X$ and $Y$, respectively, such that
\[A\times B\subset U\times V\subset N.\]
\end{theorem*}
\end{problem}
\begin{proof}
We first prove the theorem for the case in which $A=\{a\}$. Since
$a\times B$ is canonically homeomorphic to $B$, by Theorem 26.5,
$a\times B$ is compact. Therefore, for every covering by open
subsets $\left\{ U_\alpha \right\}$ of $a\times B$,
$U_\alpha\subset X\times Y$, there exists a finite subcollection,
say $\left\{ U_i \right\}_{i=1}^n$ covering $a\times B$.

% We first show the partially generalized lemma when $A=\{a\}$ is a
% single point. Since $a\times B\cong B$ is compact, we obtain $$a\times
% B\subset \bigcup(U_i\times V_i)$$ when $U_i\subset X$ and $V_i\subset
% Y$ are open, for finitely many indices $i$. We may assume each $U_i$
% is a neighborhood of $a$. We let $U=\bigcap U_i$ and $V=\bigcup V_i$
% and claim they satisfy the lemma presented; this is evident since
% $U\times V_i\subset U_i\times V_i\subset N$ for all $i$ hence $U\times
% V=\bigcup (U\times V_i)\subset N$, and also $a\times b\in a\times B$
% implies $b\in V_i$ for some $i$ so $a\times B\subset U\times V$.

% So by this partially generalized tube lemma, for each $a\in A$ let
% $U_a\subset X$ be a neighborhood of $a$ and $V_a\subset Y$ be open
% such that $U_a\times V_a\subset N$. Since $A$ is compact, we may
% choose a finite subset $\mathcal{A}\subset A$ such that $A\subset
% \bigcup_{\alpha \in\mathcal{A}}U_\alpha $. We let $U=\bigcup U_\alpha
% $ and $V=\bigcap V_\alpha $ and claim they satisfy the lemma
% presented. Since $U_\alpha \times V\subset U_\alpha \times V_\alpha
% \subset N$ we see $U\times V=\bigcup (U_\alpha \times V)\subset N$. As
% well, for each $\alpha \in\mathcal{A}$ we have $\alpha \times B\subset
% U_\alpha \times V_\alpha $ so that $B\subset V_\alpha $, implying
% $B\subset V$. And since the $U_\alpha $ cover $A$ by construction we
% have $A\times B\subset \bigcup (U_\alpha \times V)\subset U\times V$.
\end{proof}
\newpage
\begin{problem}[Munkres \S26, Ex.\,12]
Let $p\colon X\to Y$ be a closed continuous surjective map
such that $p^{-1}(y)$ is compact, for each $y\in Y$. (Such a
map is called a \emph{perfect map}.) Show that if $Y$ is compact,
then $X$ is compact.

[\emph{Hint:} If $U$ is an open set containing $p^{-1}(y)$, there
is a neighborhood $W$ of $y$ such that $p^{-1}(W)$ is contained
in $U$.]
\end{problem}
\begin{proof}

\end{proof}
\newpage
\begin{problem}[Munkres \S27, Ex.\,2(b,d)]
Let $X$ be a metric space with metric $d$; let $A\subset X$ be
nonempty.
\begin{itemize}
\item[(b)] Show that if $A$ is compact, $d(x,A)=d(x,a)$ for some
  $a\in A$.
\item[(d)] Assume that $A$ is compact; let $U$ be an open set
  containing $A$. Show that some $\epsilon$-neighborhood of $A$
  is contained in $U$.
\end{itemize}
\end{problem}
\begin{proof}
\end{proof}
\newpage
\begin{problem}[Munkres \S27, Ex.\,5]
Let $X$ be a compact Hausdorff space; let $\left\{A_n\right\}$ be
a countable collection of closed sets of $X$. Show that if each
set $A_n$ has empty interior in $X$, then the union $\bigcup A_n$
has empty interior in $X$. [\emph{Hint:} Imitate the proof of
Theorem 27.7.]

This is a special case of the \emph{Baire category theorem},
which we shall study in Chapter 8.
\end{problem}
\begin{proof}
\end{proof}
\newpage
\begin{problem}[Munkres \S28, Ex.\,2(a)]
Let $\left\{X_\alpha\right\}$ be a nindexed family of nonempty
spaces.
\begin{itemize}
\item[(a)] Show that if $\prod X_\alpha$ is locally compact, then
  each $X_\alpha$ is locally compact and $X_\alpha$ is compact
  for all but finitely many values of $\alpha$.
\end{itemize}
\end{problem}
\begin{proof}
\end{proof}
\newpage
\begin{problem}[Munkres \S28, Ex.\,10]
Show that if $X$ is a Hausdorff space that is locally compact at
the point $x$, then for each neighborhood $U$ of $x$, there is a
neighborhood $V$ of $x$ such that $V$ is compact and $\overline
V\subset U$.
\end{problem}
\begin{proof}
\end{proof}
\newpage
\begin{problem}
\end{problem}
\begin{proof}
\end{proof}
\newpage
\begin{problem}[A]
Let $S^1$ denote the circle
\[
S^1=\left\{\,(x,y)\in\RR^2\;\middle|\;x^2+y^2=1\,\right\}
\]
and let $B^2$ denote the closed disk
\[
B^2=\left\{\,(x,y)\in\RR^2\;\middle|\;x^2+y^2\leq 1\,\right\}.
\]
Prove that the quotient space $(S^1\times[0,1])/(S^1\times 0)$
(see HW \#4 for the notation) is homeomorphic to $B^2$.
\end{problem}
\begin{proof}
\end{proof}

%%% Local Variables:
%%% mode: latex
%%% TeX-master: "../MA571-HW-Current"
%%% End:

\end{document}

%%% Local Variables:
%%% mode: latex
%%% TeX-master: t
%%% End:
