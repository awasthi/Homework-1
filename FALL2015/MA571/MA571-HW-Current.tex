\def\documentauthor{Carlos Salinas}
\def\documenttitle{MA571 Homework \hwnum}
\def\hwnum{11}
\def\shorttitle{MA571 HW \hwnum}
\def\coursename{MA571}
\def\documentsubject{point-set topology}
\def\authoremail{salinac@purdue.edu}

\documentclass[article,oneside,10pt]{memoir}
\usepackage{geometry}
\usepackage[dvipsnames]{xcolor}
\usepackage[
    breaklinks,
    bookmarks=true,
    colorlinks=true,
    pageanchor=false,
    linkcolor=black,
    anchorcolor=black,
    citecolor=black,
    filecolor=black,
    menucolor=black,
    runcolor=black,
    urlcolor=black,
    hyperindex=false,
    hyperfootnotes=true,
    pdftitle={\shorttitle},
    pdfauthor={\documentauthor},
    pdfkeywords={\documentsubject},
    pdfsubject={\coursename}
    ]{hyperref}

% Use symbols instead of numbers
\renewcommand*{\thefootnote}{\fnsymbol{footnote}}

%% Math
\usepackage{amsthm}
\usepackage{amssymb}
\usepackage{mathtools}
% \usepackage{unicode-math}

%% PDFTeX specific
\usepackage[mathcal]{euscript}
\usepackage{mathrsfs}

\usepackage{cmap}
\usepackage[T2A,T1]{fontenc}
\usepackage[utf8]{inputenc}
\usepackage[french,german,spanish,dutch,russian,swedish,english]{babel}
\babeltags{fr=french,
           de=german,
           es=spanish,
           nl=dutch,
           ru=russian,
           sv=swedish,
           en=english}
\def\spanishoptions{mexico}

\usepackage{CJKutf8}
\newcommand{\textha}[1]{\begin{CJK}{UTF8}{mj}#1\end{CJK}}
\newcommand{\textni}[1]{\begin{CJK}{UTF8}{min}#1\end{CJK}}
\newcommand{\textzh}[1]{\begin{CJK}{UTF8}{bsmi}#1\end{CJK}}

\usepackage{graphicx}
\graphicspath{{figures/}}

% Misc
\usepackage{microtype}
\usepackage{multicol}
\usepackage[inline]{enumitem}
\usepackage{listings}
\usepackage{mleftright}
\mleftright

%% Theorems and definitions
%% remove parentheses
\makeatletter
\def\thmhead@plain#1#2#3{%
  \thmname{#1}\thmnumber{\@ifnotempty{#1}{ }\@upn{#2}}%
  \thmnote{ {\the\thm@notefont#3}}}
\let\thmhead\thmhead@plain
\makeatother

\theoremstyle{plain}
\newtheorem{theorem}{Theorem}
\newtheorem{proposition}[theorem]{Proposition}
\newtheorem{corollary}[theorem]{Corollary}
\newtheorem{claim}[theorem]{Claim}
\newtheorem{lemma}[theorem]{Lemma}
\newtheorem{axiom}[theorem]{Axiom}

\newtheorem*{corollary*}{Corollary}
\newtheorem*{claim*}{Claim}
\newtheorem*{lemma*}{Lemma}
\newtheorem*{proposition*}{Proposition}
\newtheorem*{theorem*}{Theorem}

\theoremstyle{definition}
\newtheorem{definition}{Definition}
\newtheorem{example}{Examples}
\newtheorem{examples}[example]{Examples}
% \newtheorem{exercise}{Exercise}[section]
% \newtheorem{problem}[exercise]{Problem}

\newcounter{problem}
\newenvironment{problem}[1][]% environment name
{% begin code
  \stepcounter{problem}
  \par\vspace{\baselineskip}\noindent
  \ifx &#1&%
  {\normalfont\Large\bfseries\scshape Problem~\hwnum.\theproblem}
  \global\def\exercisename{Problem~\hwnum.\theproblem}%
  \else
  {\normalfont\Large\bfseries\scshape Problem~\hwnum.\theproblem~(#1)}
  \global\def\exercisename{Problem~\hwnum.\theproblem(#1)}
  \fi
  \par\vspace{\baselineskip}%
  \noindent\ignorespaces
}%
{% end code
  \par\vspace{\baselineskip}%
  \noindent\ignorespacesafterend
}

\newtheorem*{definition*}{Definition}
\newtheorem*{example*}{Examples}
\newtheorem*{examples*}{Examples}
\newtheorem*{exercise*}{Exercise}
\newtheorem*{problem*}{Problem}

\theoremstyle{remark}
\newtheorem{remark}{Remark}
\newtheorem{remarks}[remark]{Remarks}
\newtheorem{observation}[remark]{Observation}
\newtheorem{observations}[remark]{Observations}

\newtheorem*{remark*}{**Remark**}
\newtheorem*{remarks*}{**Remarks**}
\newtheorem*{observation*}{**Observation**}
\newtheorem*{observations*}{**Observations**}

%% Redefinitions & commands
% \newcommand\restr[2]{{% we make the whole thing an ordinary symbol
%   \left.\kern-\nulldelimiterspace % automatically resize the bar with \right
%   {#1} % the function
%   % \vphantom{\big|} % pretend it's a little taller at normal size
%   \right|{#2} % this is the delimiter
%   }}

\newcommand{\nsubset}{\ensuremath{\not\subset}}
\newcommand{\nsupset}{\ensuremath{\not\supset}}
\renewcommand\qedsymbol{\ensuremath{\null\hfill\blacksquare}}

%% Commands and operators
\DeclareMathOperator{\diam}{diam}
\DeclareMathOperator{\id}{id}
\DeclareMathOperator{\im}{im}
\DeclareMathOperator{\Int}{int}
\DeclareMathOperator{\Cl}{cl}

\newcommand{\CC}{\mathbf{C}}
\newcommand{\NN}{\mathbf{N}}
\newcommand{\QQ}{\mathbf{Q}}
\newcommand{\RR}{\mathbf{R}}
\newcommand{\ZZ}{\mathbf{Z}}

% Renewcommands
% \renewcommand\setminus{\smallsetminus}
% \renewcommand\phi{\varphi}
% \renewcommand\epsilon{\varepsilon}

\begin{document}
\frontmatter
\aliaspagestyle{title}{empty}
\pagestyle{title}
\author{\href{mailto:\authoremail}{\documentauthor}}
\title{\documenttitle}
\date{\today}
\maketitle
\cleartooddpage

\makeoddhead{headings}
        {\small{\MakeUppercase{\itshape\documentauthor}}}
        {}
        {\small{\MakeUppercase{\itshape\exercisename}}}
\makeoddfoot{headings}{{\itshape\documenttitle}}
                      {}
                      {\thepage}
\makeevenhead{headings}
        {\small{\MakeUppercase{\itshape\documentauthor}}}
        {}
        {\small{\MakeUppercase{\itshape\exercisename}}}
\makeevenfoot{headings}{{\itshape\documenttitle}}
                      {}
                      {\thepage}
\makeheadrule{headings}{\textwidth}{.25pt}
% \makerunningwidth{headings}{1.15\textwidth}
\pagestyle{headings}

\mainmatter
% \setcounter{theorem}{16}
\begin{problem}[Munkres \S53, Ex.\,7(abcd)]
Let $G$ be a topological group with operation $\cdot$ and identity element
$x_0$. Let $\Omega(G,x_0)$ denote the set of all loops in $G$ based at
$x_0$. If $f,g\in\Omega(G,x_0)$, let us define a loop $f\otimes g$ by the
rule
\[
(f\otimes g)(s)=f(s)\cdot g(s).
\]
\begin{enumerate}[label=(\alph*)]
\item Show that this operation makes the set $\Omega(G,x_0)$ into a group.
\item Show that this operation induces a group operation $\otimes$ on
  $\pi_1(G,x_0)$.
\item Show that the two group operations $*$ and $\otimes$ on
  $\pi_1(G,x_0)$ are the same. [\emph{Hint:} Compute
  $(f*e_{x_0})\otimes(e_{x_0}*g)$.]
\item Show that $\pi_1(G,x_0)$ is Abelian.
\end{enumerate}
\end{problem}
\begin{proof}
For part (a) we need to show that the operation (0) $\otimes$ is associative,
(1) $\Omega(G,x_0)$ is closed under $\otimes$, (2) $\Omega(G,x_0)$ contains
an identity element $e$ and (3) for every $f\in\Omega(G,x_0)$ there
exists an element $\bar f\in\Omega(G,x_0)$ such that $f\otimes\bar f=\bar
f\otimes f=e$. We shall proceed in order: (0) Let
$f,g,h\in\Omega(G,x_0)$. Then $(f\otimes g)\otimes h=f\otimes(g\otimes f)$
since the multiplication $\cdot$ is associative in $G$, i.e., since given
$t\in I$ we have $(f(t)\cdot g(t))\cdot h(t)=f(t)\cdot (g(t)\cdot h(t))$,
in particular this holds for all $\in I$. (1) Let $f$ and $g$ be loops at
$x_0$ then $f\otimes g=f(s)\cdot g(s)$
\end{proof}
\newpage
\begin{problem}[(A)]
Prove Proposition F from the note on the Fundamental Group of the
Circle.
\end{problem}
\begin{proof}
\end{proof}
\newpage
\begin{problem}[(B)]
Prove Lemma G from the note on the Fundamental Group of the Circle. (Hint:
one way to do this is to use the fact, which you don’t have to prove, that
if $\sim$ is the equivalence relation on $[a,a+1]$ which identifies $a$ and
$a+1$ then the restriction of $p$ induces a homeomorphism
$[a,a+1]/{\sim}\to S^1$.)
\end{problem}
\begin{proof}
\end{proof}
\newpage
\begin{problem}[(C)]
Show that for every point $x\in S^n$ the space $S^n-x$ is homeomorphic to
$\RR^n$. You may use the fact, shown in Step 1 of the proof of Theorem
59.3, that $S^n$ with the \emph{north pole} removed is homeomorphic to
$\RR^n$. (Hint: linear algebra.)
\end{problem}
\begin{proof}
\end{proof}
\newpage
\begin{problem}[(D)]
Show that every loop in $S^n$ which is not onto is path-homotopic to a
constant path. (Hint: use Problem C).
\end{problem}
\begin{proof}
\end{proof}
\newpage
\begin{problem}[(E)]
Let $X$ be a topological space and let $A\subset X$ be a deformation
retract. In the space $X/A$, the set $A$ is a point (because it is an
equivalence class). Show that this point is a deformation retract of
$X/A$. (Hint: use p.\,289 \# 9.)
\end{problem}
\begin{proof}
\end{proof}

%%% Local Variables:
%%% mode: latex
%%% TeX-master: "../MA571-HW-Current"
%%% End:

\end{document}

%%% Local Variables:
%%% mode: latex
%%% TeX-master: t
%%% End:
