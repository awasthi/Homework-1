\def\documentauthor{Carlos Salinas}
\def\documenttitle{MA571 Homework \hwnum}
\def\hwnum{8}
\def\shorttitle{MA571 HW \hwnum}
\def\coursename{MA571}
\def\documentsubject{point-set topology}
\def\authoremail{salinac@purdue.edu}

\documentclass[article,oneside,10pt]{memoir}
\usepackage{geometry}
\usepackage[dvipsnames]{xcolor}
\usepackage[
    breaklinks,
    bookmarks=true,
    colorlinks=true,
    pageanchor=false,
    linkcolor=black,
    anchorcolor=black,
    citecolor=black,
    filecolor=black,
    menucolor=black,
    runcolor=black,
    urlcolor=black,
    hyperindex=false,
    hyperfootnotes=true,
    pdftitle={\shorttitle},
    pdfauthor={\documentauthor},
    pdfkeywords={\documentsubject},
    pdfsubject={\coursename}
    ]{hyperref}

\usepackage{graphicx}
\graphicspath{{figures/}}

% Misc
\usepackage{microtype}
\usepackage{multicol}
\usepackage[inline]{enumitem}
\usepackage{listings}
\usepackage{mleftright}
\mleftright

%% Math
\usepackage{amsthm}
\usepackage{amssymb}
\usepackage{mathtools}
% \usepackage{unicode-math}

%% PDFTeX specific
% \usepackage[mathcal]{euscript}
% \usepackage{mathrsfs}

\usepackage{cmap}
\usepackage[T2A,T1]{fontenc}
\usepackage[utf8]{inputenc}
\usepackage[french,german,russian,spanish,english]{babel}
\babeltags{fr=french,
           de=german,
           ru=russian,
           es=spanish,
           en=english}
\def\spanishoptions{mexico}
\usepackage{CJKutf8}
\newcommand{\textha}[1]{\begin{CJK}{UTF8}{mj}#1\end{CJK}}
\newcommand{\textni}[1]{\begin{CJK}{UTF8}{min}#1\end{CJK}}
\newcommand{\textzh}[1]{\begin{CJK}{UTF8}{bsmi}#1\end{CJK}}

%% Theorems and definitions
\theoremstyle{plain}
\newtheorem{theorem}{Theorem}
\newtheorem{proposition}[theorem]{Proposition}
\newtheorem{corollary}[theorem]{Corollary}
\newtheorem{claim}[theorem]{Claim}
\newtheorem{lemma}[theorem]{Lemma}
\newtheorem{axiom}[theorem]{Axiom}

\newtheorem*{corollary*}{Corollary}
\newtheorem*{claim*}{Claim}
\newtheorem*{lemma*}{Lemma}
\newtheorem*{proposition*}{Proposition}
\newtheorem*{theorem*}{Theorem}

\theoremstyle{definition}
\newtheorem{definition}{Definition}
\newtheorem{example}{Examples}
\newtheorem{examples}[example]{Examples}
% \newtheorem{exercise}{Exercise}[section]
% \newtheorem{problem}[exercise]{Problem}

% \newtheorem{exercise}{Exercise}[section]
% \newtheorem{problem}[exercise]{Problem}

\newcounter{problem}
\newenvironment{problem}[1][]% environment name
{% begin code
  \stepcounter{problem}
  \par\vspace{\baselineskip}\noindent
  \ifx &#1&%
  {\normalfont\Large\bfseries\scshape Problem~\hwnum.\theproblem}
  \global\def\exercisename{Problem~\hwnum.\theproblem}%
  \else
  {\normalfont\Large\bfseries\scshape Problem~\hwnum.\theproblem~(#1)}
  \global\def\exercisename{Problem~\hwnum.\theproblem(#1)}
  \fi
  \par\vspace{\baselineskip}%
  \noindent\ignorespaces
}%
{% end code
  \par\vspace{\baselineskip}%
  \noindent\ignorespacesafterend
}

\newtheorem*{definition*}{Definition}
\newtheorem*{example*}{Examples}
\newtheorem*{examples*}{Examples}
\newtheorem*{exercise*}{Exercise}
\newtheorem*{problem*}{Problem}

\theoremstyle{remark}
\newtheorem{remark}{Remark}
\newtheorem{remarks}[remark]{Remarks}
\newtheorem{observation}[remark]{Observation}
\newtheorem{observations}[remark]{Observations}

\newtheorem*{remark*}{**Remark**}
\newtheorem*{remarks*}{**Remarks**}
\newtheorem*{observation*}{**Observation**}
\newtheorem*{observations*}{**Observations**}

%% Redefinitions & commands
\newcommand\restr[2]{{% we make the whole thing an ordinary symbol
  \left.\kern-\nulldelimiterspace % automatically resize the bar with \right
  {#1} % the function
  % \vphantom{\big|} % pretend it's a little taller at normal size
  \right|{#2} % this is the delimiter
  }}

\newcommand{\nsubset}{\ensuremath{\not\subset}}
\newcommand{\nsupset}{\ensuremath{\not\supset}}
\renewcommand\qedsymbol{\ensuremath{\null\hfill\blacksquare}}

%% Commands and operators
\DeclareMathOperator{\diam}{diam}
\DeclareMathOperator{\id}{id}
\DeclareMathOperator{\im}{im}
\DeclareMathOperator{\Int}{int}
\DeclareMathOperator{\Cl}{cl}

\newcommand{\CC}{\mathbf{C}}
\newcommand{\NN}{\mathbf{N}}
\newcommand{\QQ}{\mathbf{Q}}
\newcommand{\RR}{\mathbf{R}}
\newcommand{\ZZ}{\mathbf{Z}}

% Renewcommands
\renewcommand\setminus{\smallsetminus}
\renewcommand\phi{\varphi}
\renewcommand\epsilon{\varepsilon}

\begin{document}
\frontmatter
\aliaspagestyle{title}{empty}
\pagestyle{title}
\author{\href{mailto:\authoremail}{\documentauthor}}
\title{\documenttitle}
\date{\today}
\maketitle
\cleartooddpage

\makeoddhead{headings}
        {\small{\MakeUppercase{\itshape\documentauthor}}}
        {}
        {\small{\MakeUppercase{\itshape\exercisename}}}
\makeoddfoot{headings}{{\itshape\documenttitle}}
                      {}
                      {\thepage}
\makeevenhead{headings}
        {\small{\MakeUppercase{\itshape\documentauthor}}}
        {}
        {\small{\MakeUppercase{\itshape\exercisename}}}
\makeevenfoot{headings}{{\itshape\documenttitle}}
                      {}
                      {\thepage}
\makeheadrule{headings}{\textwidth}{.25pt}
% \makerunningwidth{headings}{1.15\textwidth}
\pagestyle{headings}

\mainmatter
\setcounter{theorem}{15}
\begin{problem}[Munkres \S 46, Ex.\,6]
Show that the compact-open topology, $\mathcal{C}(X,Y)$ is
Hausdorff if $Y$ is Hausdorff, and regular if $Y$ is
regular. [\emph{Hint:} If $\overline U\subset V$, then
$\overline{S(C,U)}\subset S(C,V)$.]
\end{problem}
\begin{proof}
% We will first prove the following fact:
% \begin{lemma*}
% If $C\subset X$ is finite, it is compact.
% \end{lemma*}
% \begin{proof}
% \renewcommand\qedsymbol{$\clubsuit$}
% Let $C\subset X$ be finite. Put $C=\{x_1,...,x_n\}$ and let
% $\left\{U_\alpha\right\}$ be an open cover of $C$. Suppose that there is no
% finite subcollection of $\left\{U_\alpha\right\}$ which covers $C$. Then,
% for every $U_\alpha$ there is a distinct point $x\in C\cap U_\alpha$. This
% contradicts the fact that $C$ is finite.
% \end{proof}
% Now, suppose $Y$ is Hausdorff. Let $f,g\in\mathcal{C}(X,Y)$ with $f\neq g$,
% i.e., there exists a point $x_0\in X$ such that $f(x_0)\neq g(x_0)$. Since
% $Y$ is Hausdorff, there exists disjoint neighborhoods $U$ and $V$ of
% $f(x_0)$ and $g(x_0)$, respectively. Let $U'=S(\{x_0\},U)$ and
% $V'=S(\{x_0\},V)$; note that $\{x_0\}$ is compact by the lemma and $U'$ and
% $V'$ are subbasis elements of the compact-open topology by the definition
% on Munkres \S 46, p.\,285. Then $U'\cap V'=\emptyset$ for otherwise, there
% is a function $h\in U'\cap V'$ such that $h(x_0)\in U\cap V$, but this
% contradicts $U\cap V=\emptyset$. Thus, $\mathcal{C}(X,Y)$ is Hausdorff.

\end{proof}
\newpage
\begin{problem}[Munkres \S 46, Ex.\,7]
Show that if $Y$ is locally compact Hausdorff, then composition
of maps
\[\mathcal{C}(X,Y)\times\mathcal{C}(Y,Z)\longrightarrow\mathcal{C}(X,Z)\]
is continuous, provided the compact-open topology is used
throughout. [\emph{Hint:} If $g\circ f\in S(C,U)$, find $V$ such
that $f(C)\subset V$ and $g\bigl(\overline{V}\bigr)\subset U$.]
\end{problem}
\begin{proof}
\end{proof}
\newpage
\begin{problem}[Munkres \S 46, Ex.\,8]
Let $\mathcal{C}'(X,Y)$ denote the set $\mathcal{C}(X,Y)$ in some
topology $\mathcal{T}$. Show that if the evaluation map
\[
e\colon X\times\mathcal{C}'(X,Y)\longrightarrow Y
\]
is continuous, then $\mathcal{T}$ contains the compact-open
topology. [\emph{Hint:} The induced map
$E\colon\mathcal{C}'(X,Y)\to\mathcal{C}(X,Y)$ is continuous.]
\end{problem}
\begin{proof}
\end{proof}
\newpage
\begin{problem}[(A)]
\begin{definition}
Definition. If $X$ is a locally compact Hausdorff space then the
space $Y$ given by Theorem 29.1 is called the \emph{one-point
  compactification} of $X$.
\end{definition}

Let $X$ be a compact Hausdorff space and let $W$ be an open
subset of $X$ (so $W$ is locally compact by Corollary 29.3) with
$W\neq X$. Prove that the one-point compactification of $W$ is
homeomorphic to the quotient space $X/(X-W)$.
\end{problem}
\begin{proof}
\end{proof}
\newpage
\begin{problem}[(B)]
Let $X$ be a compact Hausdorff space, let $Y$ be a topological
space, and let $p\colon X\to Y$ be a closed surjective continuous
map. Prove that $Y$ is Hausdorff. [\emph{Hint:} one ingredient in the
proof is p. 171 \# 5.]
\\\\
Note: combining this with HW 4 Problem E and HW 6 Problem A gives
a necessary and sufficient condition for a quotient of a compact
Hausdorff space to be Hausdorff.
\end{problem}
\begin{proof}
\end{proof}
\newpage
\begin{problem}[(C)]
Let $S^2\subset\RR^3$ be the subspace
\[
\left\{\,(x,y,z)\;\middle|\; x^2+y^2+z^2=1\,\right\}.
\]
Prove that $S^2$ is a $2$-manifold. (The definition of
$m$-manifold, where $m$ is a positive whole number, is given at
the top of page 225.)
\end{problem}
\begin{proof}
\end{proof}
\newpage
\begin{problem}[(D)]
Prove that the union of the $x$ and $y$-axes in $\RR^2$ is not a
$1$-manifold.
\end{problem}
\begin{proof}
\end{proof}

%%% Local Variables:
%%% mode: latex
%%% TeX-master: "../MA571-HW-Current"
%%% End:

\end{document}

%%% Local Variables:
%%% mode: latex
%%% TeX-master: t
%%% End:
