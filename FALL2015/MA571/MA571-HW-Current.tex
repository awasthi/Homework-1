\def\documentauthor{Carlos Salinas}
\def\documenttitle{MA571 Problem Set \hwnum}
\def\hwnum{3}
\def\shorttitle{MA571 PSet \hwnum}
\def\coursename{MA571}
\def\documentsubject{point-set topology}
\def\authoremail{salinac@purdue.edu}

\documentclass[article,oneside,10pt]{memoir}
\usepackage{geometry}
\usepackage[dvipsnames]{xcolor}
\usepackage[
    breaklinks,
    bookmarks=true,
    colorlinks=true,
    pageanchor=false,
    linkcolor=black,
    anchorcolor=black,
    citecolor=black,
    filecolor=black,
    menucolor=black,
    runcolor=black,
    urlcolor=black,
    hyperindex=false,
    hyperfootnotes=true,
    pdftitle={\shorttitle},
    pdfauthor={\documentauthor},
    pdfkeywords={\documentsubject},
    pdfsubject={\coursename}
    ]{hyperref}

\usepackage{graphicx}
\graphicspath{{figures/}}

\usepackage{multicol}
\usepackage[inline]{enumitem}
\usepackage{listings}
\usepackage{mleftright}
\mleftright

\usepackage{microtype}

\usepackage{amsthm}
\usepackage{amssymb}
\usepackage{mathtools}
\usepackage{lualatex-math}
\usepackage{unicode-math}

\setmainfont[Ligatures=TeX,Numbers=OldStyle]{Latin Modern Roman}
\setsansfont[Ligatures=TeX,Numbers=OldStyle]{Latin Modern Sans}
\setmonofont{Latin Modern Mono}
\setmathfont{Latin Modern Math}

\theoremstyle{plain}
\newtheorem{theorem}{Theorem}
\newtheorem{proposition}[theorem]{Proposition}
\newtheorem{corollary}[theorem]{Corollary}
\newtheorem{claim}[theorem]{Claim}
\newtheorem{lemma}[theorem]{Lemma}
\newtheorem{axiom}[theorem]{Axiom}

\newtheorem*{corollary*}{Corollary}
\newtheorem*{claim*}{Claim}
\newtheorem*{lemma*}{Lemma}
\newtheorem*{proposition*}{Proposition}
\newtheorem*{theorem*}{Theorem}

\theoremstyle{definition}
\newtheorem{definition}{Definition}
\newtheorem{example}{Examples}
\newtheorem{examples}[example]{Examples}
% \newtheorem{exercise}{Exercise}[section]
% \newtheorem{problem}[exercise]{Problem}

\counterwithout{section}{chapter}
\usepackage[explicit]{titlesec}
\titleformat{\section}{\normalfont\Large\bfseries\scshape}{}{0em}{#1}
\newenvironment{problem}[1][]% environment name
{% begin code
  \par\vspace{\baselineskip}\noindent
  \ifx &#1&%
  \section{Problem~\hwnum.\thesection}
  \global\def\exercisename{Problem \hwnum.\thesection}%
  \else
  \section{Problem~\hwnum.\thesection~(#1)}
  \global\def\exercisename{Problem \hwnum.\thesection(#1)}
  \fi
  \par\vspace{\baselineskip}%
  \noindent\ignorespaces
}%
{% end code
  \par\vspace{\baselineskip}%
  \noindent\ignorespacesafterend
}

\newtheorem*{definition*}{Definition}
\newtheorem*{example*}{Examples}
\newtheorem*{examples*}{Examples}
\newtheorem*{exercise*}{Exercise}
\newtheorem*{problem*}{Problem}

\theoremstyle{remark}
\newtheorem{remark}{Remark}
\newtheorem{remarks}[remark]{Remarks}
\newtheorem{observation}[remark]{Observation}
\newtheorem{observations}[remark]{Observations}

\newtheorem*{remark*}{**Remark**}
\newtheorem*{remarks*}{**Remarks**}
\newtheorem*{observation*}{**Observation**}
\newtheorem*{observations*}{**Observations**}

%% Redefinitions & commands
\renewcommand\qedsymbol{\ensuremath{\null\hfill\QED}}

\newcommand\restr[2]{{% we make the whole thing an ordinary symbol
  \left.\kern-\nulldelimiterspace % automatically resize the bar with \right
  {#1} % the function
  % \vphantom{\big|} % pretend it's a little taller at normal size
  \right|{#2} % this is the delimiter
  }}

%% Commands and operators
\DeclareMathOperator{\id}{id}
\DeclareMathOperator{\im}{im}

\newcommand{\clsr}[1]{\overline{#1}}
\newcommand{\CC}{\mathbf{C}}
\newcommand{\NN}{\mathbf{N}}
\newcommand{\QQ}{\mathbf{Q}}
\newcommand{\RR}{\mathbf{R}}
\newcommand{\ZZ}{\mathbf{Z}}

\begin{document}
\let\setminus\relax
\let\phi\relax
\let\epsilon\relax
\newcommand\setminus{\smallsetminus}
\newcommand\phi{\varphi}
\newcommand\epsilon{\varepsilon}

\frontmatter
\aliaspagestyle{title}{empty}
\pagestyle{title}
\author{\href{mailto:\authoremail}{\documentauthor}}
\title{\documenttitle}
\date{\today}
\maketitle
\cleartooddpage

\makeoddhead{headings}
        {\small{\MakeUppercase{\itshape\documentauthor}}}
        {}
        {\small{\MakeUppercase{\itshape\exercisename}}}
\makeoddfoot{headings}{{\itshape\documenttitle}}
                      {}
                      {\thepage}
\makeevenhead{headings}
        {\small{\MakeUppercase{\itshape\documentauthor}}}
        {}
        {\small{\MakeUppercase{\itshape\exercisename}}}
\makeevenfoot{headings}{{\itshape\documenttitle}}
                      {}
                      {\thepage}
\makeheadrule{headings}{\textwidth}{.25pt}
% \makerunningwidth{headings}{1.15\textwidth}
\pagestyle{headings}

\mainmatter
\setcounter{theorem}{5}
\begin{problem}[Munkres \S18, p.\,111, \#7(a)]
\begin{enumerate}[noitemsep]
\item[(a)] Suppose that $f\colon\RR\to\RR$ is ``continuous from
  the right,'' that is,
  \[
    \lim_{x\to a+}f(x)=f(a).
  \]
  for each $a\in\RR$. Show that $f$ is continuous when considered
  as a function from $\RR_\ell$ to $\RR$.
\end{enumerate}
\end{problem}
\begin{proof}
Recall the definition of ``right-hand limit,'':
\begin{definition*}[Rudin \S4, p.\,94, Def.\,4.25]
Let $f$ be defined on $(a,b)$. Consider any point $x$ such that
$a\leq x<b$. We write $f(x+)=q$ if $f(t_n)\to q$ as $n\to\infty$,
for all sequences $\left\{t_n\right\}$ in $(x,b)$ such that
$t_n\to x$.
\end{definition*}
\end{proof}
\newpage
\begin{problem}[Munkres \S18, p.\,112, \#13]
Let $A\subset X$; let $f\colon A\to Y$ be continuous; let $Y$ be
Hausdorff. Show that if $f$ may be extended to a continuous
function $g\colon\clsr A\to Y$, then $g$ is uniquely determined
by $f$.
\end{problem}
\begin{proof}
\end{proof}
\newpage
\begin{problem}[Munkres \S19, p.\,118, \#2]
Prove Theorem 19.3.
\end{problem}
\begin{proof}
Recall the exact statement of Theorem 19.3 from Munkres \S19,
p.\,116:
\begin{theorem*}
Let $A_\alpha$ be as subspace of $X_\alpha$, for each $\alpha\in
J$. Then $\prod A_\alpha$ is a subspace of $\prod X_\alpha$ if
both products are given the box topology, or if both products are
given the product topology.
\end{theorem*}
\end{proof}
\newpage
\begin{problem}[Munkres \S19, p.\,118, \#3]
Prove Theorem 19.4.
\end{problem}
\begin{proof}
Recall the exact statement of Theorem 19.4 from Munkres \S19,
p.\,116:
\begin{theorem*}
If each space $X_\alpha$ is a Hausdorff space, then $\prod
X_\alpha$ is a Hausdorff space in both the box and product
topologies.
\end{theorem*}
\end{proof}
\newpage
\begin{problem}[Munkres \S19, p.\,118, \#6]
Let $\mathbf{x}_1,\mathbf{x}_2,...$ be a sequence of the points
of the product space $\prod X_\alpha$. Show that this sequence
converges to the point $\mathbf{x}$ if and only if the sequence
$\pi_\alpha(\mathbf{x}_1),\pi_\alpha(\mathbf{x}_2),...$ converges
to $\pi_\alpha(\mathbf{x})$ for each $\alpha$. Is this fact true
if one uses the box topology instead of the product topology?
\end{problem}
\begin{proof}
\end{proof}
\newpage
\begin{problem}[Munkres \S19, p.\,118, \#7]
Let $\RR^\infty$ be the subset of $\RR^\omega$ consisting of all
sequences that are ``eventually zero,'' that is, all sequences
$(x_1,x_2,...)$ such that $x_i\neq 0$ for only finitely many
values of $i$. What is the closure of $\RR^\infty$ in
$\RR^\omega$ in the box and product topologies? Justify your
answer.
\end{problem}
\begin{proof}
\end{proof}
\newpage
\begin{problem}[Munkres \S20, p.\,126, \#3(b)]
Let $X$ be a metric space with metric $d$.
\begin{enumerate}[noitemsep]
\item[(b)] Let $X'$ denote a space having the same underlying set
  as $X$. show that if $d\colon X'\times X'\to\RR$ is continuous,
  then the topology of $X'$ is finer than the topology of $X$.
\end{enumerate}
\end{problem}
\begin{proof}
\end{proof}
\newpage
\begin{problem}[Munkres \S20, p.\,127, \#4(b)]
Consider the product, uniform and box topologies on $\RR^\omega$
\begin{enumerate}[noitemsep]
\item[(b)] In which topologies do the following sequences converge?
\begin{align*}
\mathbf{w}_1&=(1,1,1,1,...),&\mathbf{x}_1&=(1,1,1,1,...),\\
\mathbf{w}_2&=(0,2,2,2,...),&\mathbf{x}_2&=\left(0,\tfrac{1}{2},\tfrac{1}{2},\tfrac{1}{2},...\right),\\
\mathbf{w}_3&=(0,0,3,3,...),&\mathbf{x}_3&=\left(0,0,\tfrac{1}{3},\tfrac{1}{3},...\right),\\
&\vdotswithin{=}&&\vdotswithin{=}\\
\mathbf{y}_1&=(1,0,0,0,...)&\mathbf{z}_1&=(1,1,0,0,...),\\
\mathbf{y}_2&=\left(\tfrac{1}{2},\tfrac{1}{2},0,0,...\right)&\mathbf{z}_2&=\left(\tfrac{1}{2},\tfrac{1}{2},0,0,...\right),\\
\mathbf{y}_3&=\left(\tfrac{1}{3},\tfrac{1}{3},\tfrac{1}{3},0,...\right)&\mathbf{z}_3&=\left(\tfrac{1}{3},\tfrac{1}{3},0,0,...\right),\\
&\vdotswithin{=}&&\vdotswithin{=}
\end{align*}
\end{enumerate}
\end{problem}
\begin{proof}
\end{proof}
\newpage
\begin{problem}[A]
Given: $X$ a metric space, $A$ a countable subset of $X$, and
$\clsr A=X$. To prove: the topology of $X$ has a countable
basis.
\end{problem}
\begin{proof}
\end{proof}
\newpage
\begin{problem}[B]
Given: $Y$ is an ordered set, $(a,b)$ and $(c,d)$ are disjoint
open intervals, and there are elements $x\in(a,b)$ and
$y\in(c,d)$ with $x<y$. To prove: every element of $(a,b)$ less
than every element of $(c,d)$.
\end{problem}
\begin{proof}
\end{proof}
\newpage
\begin{problem}[C]
(This problem will be used when we discuss quotient spaces). Let
$S$ and $T$ be sets and let $f\colon S\to T$ be a function. Let
$A\subset S$.
\begin{enumerate}[noitemsep,label=(\roman*)]
\item Give an example to show that the equation
\begin{equation}
\label{eq:mcclure-1}
\tag{*}
f^{-1}(f(A))=A
\end{equation}
isn't always valid.
\item Define an equivalence relation $\sim$ on $S$ by $s\sim s'$
  if and only if $f(s)=f(s')$. Using this equivalence relation,
  describe the subsets $A$ of $S$ for which (\ref{eq:mcclure-1}) is
  true. Prove that your answer is correct.
\end{enumerate}
\end{problem}
\begin{proof}
\end{proof}

%%% Local Variables:
%%% mode: latex
%%% TeX-master: "../MA571-HW-Current"
%%% End:

\end{document}

%%% Local Variables:
%%% mode: latex
%%% TeX-master: t
%%% End:
