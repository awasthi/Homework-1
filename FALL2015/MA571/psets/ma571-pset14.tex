\begin{problem}[Munkres \S74, Ex.\,6]
If $n>1$, show that the fundamental group of the $n$-fold torus is not
Abelian. [\emph{Hint:} Let $G$ be a free group on the set
$\{\alpha_1,\beta_1,...,\alpha_n,\beta_n\}$; let $F$ be the free group on
the set $\{\gamma,\delta\}$. Consider the homomorphism of $G$ onto $F$ that
sends $\alpha_1$ and $\beta_1$ to $\gamma$ and all other $\alpha_i$ and
$\beta_i$ to $\delta$.]
\end{problem}
\begin{proof}
Let $\bfT^n$ denote the $n$-fold torus and let us fix a base-point
$x_0\in\bfT^n$. By Theorem 74.3, the fundamental group of $\bfT^n$,
$\pi_1(\bfT^n,x_0)$, is isomorphic to the quotient of the free group $G$ on
the set $\{\alpha_1,\beta_1,...,\alpha_n,\beta_n\}$, by the least normal
subgroup $N$ containing
\[
\alpha_1\beta_1\alpha_1^{-1}\beta_1^{-1}
\cdots
\alpha_n\beta_n\alpha_n^{-1}\beta_n^{-1}.
\]
Now we proceed by the hint. Let $F$ be the free group on the set
$\{\gamma,\delta\}$. We define a homomorphism $\varphi\colon G\to F$ by the
rule $\alpha_1\mapsto\gamma$, $\beta_1\mapsto\gamma$ and
$\alpha_i\mapsto\delta$ and $\beta_i\mapsto\delta$ for all $i\neq
1$. By Lemma 69.1, $\varphi$ determines a homomorphism $G\to F$. Moreover,
note that $\varphi$ is surjective since each generator is mapped onto by an
element of $G$ so by the 1st isomorphism theorem, the induced map
$\bar\varphi\colon G/\ker\varphi\to F$ is an isomorphism. Now, we claim
that it suffices to show that $\ker\varphi>N$ since
\end{proof}
\newpage
\begin{problem}[Munkres \S76, Ex.\,1]
Calculate $H_1(\bfP^2\#\bfT)$. Assuming that the list of compact surfaces given
in Theorem 75.5 is a complete list, to which of these surfaces is $\bfP^2\#\bfT$
homeomorphic?
\end{problem}
\begin{proof}
\end{proof}
\newpage
\begin{problem}[Munkres \S76, Ex.\,2]
If $\bfK$ is the Klein bottle, calculate $H_1(\bfK)$ directly.
\end{problem}
\begin{proof}
\end{proof}
\newpage
\begin{problem}[Munkres \S76, Ex.\,3(a,b,c)]
Let $X$ be the quotient space obtained from an $8$-sided polygonal region
$P$ by pasting its edges together according to the labelling scheme
$acadbcb^{-1}d$.
\begin{enumerate}[label=(\alph*)]
\item Check that all vertices of $P$ are mapped to the same point of the
  quotient space $X$ by the pasting map.
\item Calculate $H_1(X)$.
\item Assuming $X$ is homeomorphic to one of the surfaces given in Theorem
  75.5 (which it is), which surface is it ?
\end{enumerate}
\end{problem}
\begin{proof}
\end{proof}
\newpage
\begin{problem}[A]
Define $P^n$ to be the space $\bfS^n/{\sim}$ where $z\sim z'$ if and only
if $z=z'$ or $z=-z'$. Use the Seifert--van Kampen Theorem to calculate
$\pi_1(\bfP^n)$. (Hint: induction starting from the case $n=2$ that was
done in class.)
\end{problem}
\begin{proof}
\end{proof}
\newpage
\begin{problem}[B]
A topological space $X$ is called \emph{homogeneous} if for every pair of
points $x,y\in X$ there is a homeomorphism $\varphi\colon X\to X$ with
$\varphi(x)=y$. Prove that every connected $2$-manifold is
homogeneous. (Hint: use the optional problem from the previous assignment.)
\end{problem}
\begin{proof}
\end{proof}
\newpage
\begin{problem}[Optional problem]
\begin{enumerate}[label=(\roman*)]
\item Let $x\subset\bfR^3$ be the cylinder
\[
\left\{\,(x,y,z)\;\middle|\;
\text{$x^2+y^2=\tfrac{1}{\sqrt{2}}$ and $|z|\leq \tfrac{1}{\sqrt{2}}$}\,\right\}
\]
and let $f\colon X\to\bfR^3$ be the map
\[
f(x,y,z)=\left(2^{1/4}x\sqrt{1-z^2},2^{1/4},y\sqrt{1-z^2},z\right).
\]
Prove that $f$ is a homeomorphism from $X$ to the subspace
\[
Y=\bfS^2\cap\left\{\,(x,y,z)\;\middle|\;|z|\leq\tfrac{1}{\sqrt{2}}\,\right\}.
\]
\item Prove that the \textde{Möbius} band is homeomorphic to $P^2$ with an
  open disk removed (think of $\bfP^2$ as $\bfS^2/{\sim}$ and use part (i)).
\end{enumerate}
\end{problem}
\begin{proof}
\end{proof}
% \textru{Татарстаннын кызлары миңа бик килеше!}

%%% Local Variables:
%%% mode: latex
%%% TeX-master: "../MA571-HW-Current"
%%% End:
