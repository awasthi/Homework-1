\begin{problem}[Munkres \S74, Ex.\,6]
If $n>1$, show that the fundamental group of the $n$-fold torus is not
Abelian. [\emph{Hint:} Let $G$ be a free group on the set
$\{\alpha_1,\beta_1,...,\alpha_n,\beta_n\}$; let $F$ be the free group on
the set $\{\gamma,\delta\}$. Consider the homomorphism of $G$ onto $F$ that
sends $\alpha_1$ and $\beta_1$ to $\gamma$ and all other $\alpha_i$ and
$\beta_i$ to $\delta$.]
\end{problem}
\begin{proof}
Let $\TT^n$ denote the $n$-fold torus and let $x_0\in\TT^n$. By Theorem
74.3, the $\pi_1(\TT^n,x_0)$ is isomorphic to the quotient of the free
group on $2n$ letters, say $\alpha_1,\beta_1,...,\alpha_n,\beta_n$, by the
least normal subgroup, $N$, containing
$[\alpha_1,\beta_1][\alpha_2,\beta_2]\cdots[\alpha_n,\beta_n]$ where
$[\alpha,\beta]=\alpha\beta\alpha^{-1}\beta^{-1}$, i.e., the commutator of
$\alpha$ and $\beta$.
\end{proof}
\newpage
\begin{problem}[Munkres \S76, Ex.\,1]
Calculate $H_1(P^2\#T)$. Assuming that the list of compact surfaces given
in Theorem 75.5 is a complete list, to which of these surfaces is $P^2\#T$
homeomorphic?
\end{problem}
\begin{proof}
\end{proof}
\newpage
\begin{problem}[Munkres \S76, Ex.\,2]
If $K$ is the Klein bottle, calculate $H_1(K)$ directly.
\end{problem}
\begin{proof}
\end{proof}
\newpage
\begin{problem}[Munkres \S76, Ex.\,3(a,b,c)]
Let $X$ be the quotient space obtained from an $8$-sided polygonal region
$P$ by pasting its edges together according to the labelling scheme
$acadbcb^{-1}d$.
\begin{enumerate}[label=(\alph*)]
\item Check that all vertices of $P$ are mapped to the same point of the
  quotient space $X$ by the pasting map.
\item Calculate $H_1(X)$.
\item Assuming $X$ is homeomorphic to one of the surfaces given in Theorem
  75.5 (which it is), which surface is it ?
\end{enumerate}
\end{problem}
\begin{proof}
\end{proof}
\newpage
\begin{problem}[A]
Define $P^n$ to be the space $S^n/{\sim}$ where $z\sim z'$ if and only if
$z=z'$ or $z=-z'$. Use the Seifert--van Kampen Theorem to calculate
$\pi_1(P^n)$. (Hint: induction starting from the case $n=2$ that was done
in class.)
\end{problem}
\begin{proof}
\end{proof}
\newpage
\begin{problem}[B]
A topological space $X$ is called \emph{homogeneous} if for every pair of
points $x,y\in X$ there is a homeomorphism $\varphi\colon X\to X$ with
$\varphi(x)=y$. Prove that every connected $2$-manifold is
homogeneous. (Hint: use the optional problem from the previous assignment.)
\end{problem}
\begin{proof}
\end{proof}
\newpage
\begin{problem}[Optional problem]
\begin{enumerate}[label=(\roman*)]
\item Let $x\subset\RR^3$ be the cylinder
\[
\left\{\,(x,y,z)\;\middle|\;
\text{$x^2+y^2=\tfrac{1}{\sqrt{2}}$ and $|z|\leq \tfrac{1}{\sqrt{2}}$}\,\right\}
\]
and let $f\colon X\to\RR^3$ be the map
\[
f(x,y,z)=\left(2^{1/4}x\sqrt{1-z^2},2^{1/4},y\sqrt{1-z^2},z\right).
\]
Prove that $f$ is a homeomorphism from $X$ to the subspace
\[
Y=S^2\cap\left\{\,(x,y,z)\;\middle|\;|z|\leq\tfrac{1}{\sqrt{2}}\,\right\}.
\]
\item Prove that the \textde{Möbius} band is homeomorphic to $P^2$ with an
  open disk removed (think of $P^2$ as $S^2/{\sim}$ and use part (i)).
\end{enumerate}
\end{problem}
\begin{proof}
\end{proof}
% \textru{Татарстаннын кызлары миңа бик килеше!}

%%% Local Variables:
%%% mode: latex
%%% TeX-master: "../MA571-HW-Current"
%%% End:
