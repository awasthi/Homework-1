\begin{problem}[Munkres \S46, Ex.\,6]
Show that the compact-open topology, $\mathcal{C}(X,Y)$ is
Hausdorff if $Y$ is Hausdorff, and regular if $Y$ is
regular. [\emph{Hint:} If $\overline U\subset V$, then
$\overline{S(C,U)}\subset S(C,V)$.]
\end{problem}
\begin{proof}
Suppose that $Y$ is regular. We shall proceed by the
hint and Lemma 31.1(b). Consider the subbasis element
$S(C,U)$. Since $Y$ is regular, there exists a neighborhood
$V\supset U$ such that $V\supset\overline{U}$. Let
$f\in\overline{S(C,U)}$. Then, we claim that $f\in S(C,V)$. For
suppose not, then there exists an element $x_0\in C$ such that
$f(x_0)\notin V$. Then, since $\overline{U}\subset V$, by
hypothesis, $f(x_0)\notin\overline{U}$. Consider the subbasic
neighborhood $S\left(\left\{x_0\right\},Y-\overline{U}\right)$ of
$f$. Then, $S\left(\left\{x_0\right\},Y-\overline{U}\right)\cap
S(C,U)$ is nonempty. Let $g$ be in the aforementioned
intersection. Then $g(x_0)\in g(C)\subset U$, but $g(x_0)\in
Y-\overline{U}$. This is a contradiction. Thus, $\overline{S(C,U)}\subset
S(C,V)$.

Now, let $f\in\mathcal{C}(X,Y)$ and let $V=\bigcap_{i=1}^n S(C_i,V_i)$ be a
basic neighborhood of $f$. Then, since $Y$ is regular, for every
$y=f(x_i)\in f(C_i)$ there exists an open neighborhood $U_{x_i}$ such that
$\overline{U_{x_i}}\subset V_i$. These $U_{x_i}$'s form an open cover of
$f(C_i)$ which is compact by Theorem 25.6 so there exists a finite
collection of them, say $\left\{U_{x_i,j}\right\}_{j=1}^{n_i}$ that covers
$f(C_i)$. Let $U_i=\bigcup_{j=1}^{n_i}U_{x_i,j}$. Then
$\overline{U}_i=\bigcup_{j=1}^{n_i}\overline{U_{x_i,j}}\subset V_i$ by
induction on Problem 2.2 (Munkres \S17, Ex. 6(b)). Let
$U=\bigcap_{i=1}^nS(C_i,U_i)$. We claim that $U$ is the desired
neighborhood of $f$ that, by Theorem 31.1(b), shows that $\mathcal{C}(X,Y)$
is regular. Let us verify this. First, note that $f\in U$ since
$f(C_i)\subset U_i$ for all $i$ so $U$ is indeed a neighborhood of
$U$. Moreover, by the hint, we have that $\overline{S(C_i,U_i)}\subset
S(C_i,V_i)$ since $\overline{U_i}\subset V_i$. Then
$\overline{U}\subset\bigcap_{i=1}^n\overline{S(C_i,U_i)}\subset V$ by Lemma
B. It follows, by Theorem 31.1(b), that $\mathcal{C}(X,Y)$ is regular.
\end{proof}
\newpage
\begin{problem}[Munkres \S46, Ex.\,9(a,b,c)]
Here is a (unexpected) application of Theorem 46.11 to quotient
maps. (Compare Exercise 11 of \S29.)
\begin{theorem*}
If $p\colon A\to B$ is a quotient map and $X$ is locally compact
Hausdorff, then $(\id_X,p)\colon X\times A\to X\times B$ is a
quotient map.
\begin{proof}
\renewcommand\qedsymbol{\null}
\begin{enumerate}[label=(\alph*)]
\item Let $Y$ be the quotient space induced by $(\id_X,p)$; let
  $q\colon X\times A\to Y$ be the quotient map. Show there is a
  bijective continuous map $f\colon Y\to X\times B$ such that
  $f\circ q=(\id_X,p)$.
\item Let $g=f^{-1}$. Let $G\colon B\to\mathcal{C}(X,Y)$ and
  $Q\colon A\to\mathcal{C}(X,Y)$ be the maps induced by $g$ and
  $q$, respectively. Show that $Q=G\circ p$.
\item Show that $Q$ is continuous; conclude that $G$ is
  continuous, so that $g$ is continuous.
\end{enumerate}
\end{proof}
\end{theorem*}
\end{problem}
\begin{proof}[Actual proof]
% (a) First, let us note that $Y$ has the same underlying set as
% the Cartesian product $X\times B$ with the topology induced by
% the map $(\id_X,p)$. Moreover, the quotient map $q$ is (as a map
% between sets) equivalent to $(\id_X,p)$. Thus, the natural
% bijective map to consider between $Y$ and $X\times B$ is the
% identity map $f=\id_{X\times B}=(\id_X,\id_B)$. It is clear that
% $f$ is bijective. To see that $f$ is continuous we note that
% since the composition
% \[
% (f\circ q)(x,a)
% =(\id_X,\id_B)\circ(\id_X,p)(x,a)
% =(\id_X(a),\id_B(p(a)))
% =(x,p(a))
% =(\id_X,p)(x,a)
% \]
% and the map $(\id_X,p)$ is continuous by Theorem 18.4, it
% follows by Theorem Q.2 that $f$ is continuous.
% \\\\
% (b) Recall, from the definition given on Munkres \S46, p.\,287,
% that the induced map $G$ (respectively $Q$) are defined by the
% equation $(G(b))(x)=(x,b)$ (respectively
% $(Q(a))(x)=(x,p(a))$). Then we have that the composition
% \[
% (G\circ p)(a)=G(p(a))=(G(p(a)))(x)=(x,p(a))=(Q(a))(x)=Q(a)
% \]
% as desired.
% \\\\
% (c) By Theorem 46.11, since $q$ is continuous with respect to the
% quotient topology on $Y$, it follows that the induced map $Q$ is
% continuous. Additionally, since $Q$ is equal to the composition
% $G\circ p$ by part (b) so by Theorem Q.2 $G$ is continuous. Since
% $X$ is locally compact Hausdorff, it follows by Theorem 46.11
% that the map $g$ is continuous.
(a) Let
\\\\
(b)
\\\\
(c)
\end{proof}
\newpage
\begin{problem}[Munkres \S51, Ex.\,1]
Show that if $h,h'\colon X\to Y$ are homotopic and $k,k'\colon
Y\to Z$ are homotopic, then $k\circ h$ and $k'\circ h'$ are
homotopic.
\end{problem}
\begin{proof}
Let $H\colon X\times I\to Y$ and $K\colon Y\times I\to Z$ denote
the homotopies from $h$ to $h'$ and $k$ to $k'$,
respectively. Then, we claim that the map $L(x,t)=K(H(x,t),t)$ is
a homotopy from $k\circ h$ to $k'\circ h'$. First, we check that
$L$ starts and ends where we want it to, i.e.,
$L(x,0)=K(H(x,0),0)=k(h(x))$ and
$L(x,1)=K(H(x,1),1)=k'(h'(x))$. Lastly, we must assure ourselves
that $L$ is in fact continuous. But this last claim follows from
the fact that $L$ can be expressed as the composition $K\circ
(h_t,t)$ where $h_t$ denotes the continuous map $H(x,t)$ at time
$t$. Since $K$ is (by assumption) continuous and $(h_t,t)$ are
continuous by Theorem 18.4, it follows by Theorem 18.2(a) that
$L$ is continuous. Thus, $k\circ h\simeq k'\circ h'$ as desired.
\end{proof}
\newpage
\begin{problem}[Munkres \S51, Ex.\,2]
Given spaces $X$ and $Y$, let $[X,Y]$ denote the homotopy classes
of maps of $X$ into $Y$
\begin{enumerate}[label=(\alph*)]
\item Let $I=[0,1]$. Show that for any $X$, the set $[X,I]$ has a
  single element.
\item Show that if $Y$ is path connected, the set $[I,Y]$ has a
  single element.
\end{enumerate}
\end{problem}
\begin{proof}
(a) Let $f,g\colon X\to I$ be arbitrary continuous maps. Then we
claim that the straight line homotopy $H(x,t)=(1-t)f(x)+tg(x)$
gives a homotopy from $f$ to $g$. Note that the image of $H(x,t)$
stays in the interval $I$ since $(1-t)f(x)+tg(x)\leq (1-t)+t=1$
for all $x$ and for all $t$. Lastly, note that by Theorem 25.1
$H$ is continuous since it is the sum of a product of continuous
functions. Hence, $f\simeq g$. Since $f$ and $g$ were arbitrary,
it follows that $[X,I]$ consists of a single equivalence class.
\\\\
(b) Note that if $f,g\colon I\to Y$ are constant maps, say $f(x)=x_0$ and
$g(x)=x_1$ for all $x\in I$, then the path $p\colon I\to Y$ where
$p(0)=x_0$ and $p(1)=x_1$ defines a homotopy $H(x,t)=p(t)$. This map is
clearly continuous since for any open neighborhood $U$ of $Y$, since $p$ is
continuous, by Theorem 18.1(4) there exists a neighborhood $V\subset I$
such that $p(V)\subset U$ so $H(I\times V)=p(V)\subset U$ implies $H$ is
continuous by Theorem 18.1(4). Therefore, it suffices to show that given a
continuous map $f\colon I\to Y$, $f$ is nulhomotopic. Let $H(x,t)$ be the
map $f((t-1)x)$. The map $(t-1)x$ is continuous by Theorem 25.1 so the
composition $f\circ((t-1)x)$ is continuous by Theorem 18.2(c). Then,
observing that $H(x,0)=f(x)$ and $H(x,1)=f(0)$, $H(x,t)$ gives a homotopy
from $f$ to $f(0)$. It follows by Lemma 51.1 that given any $f,g\colon I\to
Y$ continuous maps $f\simeq g$ by transitivity of homotopy.
\end{proof}
\newpage
\begin{problem}[Munkres \S51, Ex.\,3(a,b,c,)]
A space $X$ is said to be \emph{contractible} if the identity map
$\id_X\colon X\to X$ is nullhomotopic.
\begin{enumerate}[label=(\alph*)]
\item Show that $I$ and $\RR$ are contractible.
\item Show that a contractible space is path connected.
\item Show that if $Y$ is contractible, then for any $X$, the set
  $[X,Y]$ has a single element.
\end{enumerate}
\end{problem}
\begin{proof}
(a) It is clear that $\id_I\colon I\to I$ is nulhomotopic, say to the
constant map $0$, via the homotopy $H(x,t)=(1-t)x$. Note that
$H(x,0)=x=\id_I(x)$ and $H(x,1)=0$ and $H(x,t)$ is continuous since
$(1-t)x$ is continuous by Theorem 25.1.\footnote{More generally, we showed
  that products, sums and quotients (when they are defined) of maps from a
  metric space $(X,d)$ to $\RR$ (or a subspace of $\RR$ by
  Theorem 18.2(d)) for that matter, are continuous.} In the case of $\RR$
the previous map $H(x,t)$ also works to show that $\id_{\RR}$ is
nulhomotopic since $H(x,0)=x=\id_{\RR}$ and $H(x,1)=0$ and $H(x,t)$ is
continuous by Theorem 25.1.
\\\\
(b) Suppose that $X$ is contractible. Then there exists a homotopy $H(x,t)$
with $H(x,0)=x$ and $H(x,1)=x_0$ for some point $x_0\in X$. Now, let
$x_1,x_2\in X$. Then the map $p_1(t)=H(x_1,t)$ and $p_2(t)=H(x_2,t)$ are
path homotopies from $x_1$ to $x_0$ and $x_2$ to $x_0$. It follows by the
fact that $\simeq_p$ is an equivalence relation that $x_1\simeq_p x_2$.
\\\\
(c) Since $Y$ is contractible there exist a homotopy $H(y,t)$ with
$H(y,0)=x$ and $H(y,1)=y_0$ for some fixed $y_0\in X$. Therefore, it
suffices to show that an arbitrary continuous map $f\colon X\to Y$ is
nulhomotopic. Consider the map $K(x,t)=H(f(x),t)$. This map is continuous
since it is the composition $H\circ (f,\id_I)$. Moreover,
$K(x,0)=\id_Y(f(x))=f(x)$ and $K(x,1)=e_{y_0}(f(x))=y_0$. Thus, $f$ is
nulhomotopic and it follows that $[X,Y]$ has a single element (all maps are
null homotopic and $Y$ is path connected by part (b)).
\end{proof}

%%% Local Variables:
%%% mode: latex
%%% TeX-master: "../MA571-HW-Current"
%%% End:
