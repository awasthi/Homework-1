\begin{problem}[Munkres \S46, Ex.\,6]
Show that the compact-open topology, $\mathcal{C}(X,Y)$ is
Hausdorff if $Y$ is Hausdorff, and regular if $Y$ is
regular. [\emph{Hint:} If $\overline U\subset V$, then
$\overline{S(C,U)}\subset S(C,V)$.]
\end{problem}
\begin{proof}
Suppose that $Y$ is regular. We shall proceed by the
hint and Lemma 31.1(b). Consider the subbasis element
$S(C,U)$. Since $Y$ is regular, there exists a neighborhood
$V\supset U$ such that $V\supset\overline{U}$. Let
$f\in\overline{S(C,U)}$. Then, we claim that $f\in S(C,V)$. For
suppose not, then there exists an element $x_0\in C$ such that
$f(x_0)\notin V$. Then, since $\overline{U}\subset V$, by
hypothesis, $f(x_0)\notin\overline{U}$. Consider the subbasic
neighborhood $S\left(\left\{x_0\right\},Y-\overline{U}\right)$ of
$f$. Then, $S\left(\left\{x_0\right\},Y-\overline{U}\right)\cap
S(C,U)$ is nonempty. Let $g$ be in the aforementioned
intersection. Then $g(x_0)\in g(C)\subset U$, but $g(x_0)\in
Y-\overline{U}$. This is a contradiction. It follows by Lemma
31.1(b) that $\mathcal{C}(X,Y)$ is regular.
\end{proof}
\newpage
\begin{problem}[Munkres \S46, Ex.\,9(a,b,c)]
Here is a (unexpected) application of Theorem 46.11 to quotient
maps. (Compare Exercise 11 of \S29.)
\begin{theorem*}
If $p\colon A\to B$ is a quotient map and $X$ is locally compact
Hausdorff, then $(\id_X,p)\colon X\times A\to X\times B$ is a
quotient map.
\begin{proof}
\renewcommand\qedsymbol{\null}
\begin{enumerate}[label=(\alph*)]
\item Let $Y$ be the quotient space induced by $(\id_X,p)$; let
  $q\colon X\times A\to Y$ be the quotient map. Show there is a
  bijective continuous map $f\colon Y\to X\times B$ such that
  $f\circ q=(\id_X,p)$.
\item Let $g=f^{-1}$. Let $G\colon B\to\mathcal{C}(X,Y)$ and
  $Q\colon A\to\mathcal{C}(X,Y)$ be the maps induced by $g$ and
  $q$, respectively. Show that $Q=G\circ p$.
\item Show that $Q$ is continuous; conclude that $G$ is
  continuous, so that $g$ is continuous.
\end{enumerate}
\end{proof}
\end{theorem*}
\end{problem}
\begin{proof}
\end{proof}
\newpage
\begin{problem}[Munkres \S52, Ex.\,1]
Show that if $h,h'\colon X\to Y$ are homotopic and $k,k'\colon
Y\to Z$ are homotopic, then $k\circ h$ and $k'\circ h'$ are
homotopic.
\end{problem}
\begin{proof}
\end{proof}
\newpage
\begin{problem}[Munkres \S52, Ex.\,2]
Given spaces $X$ and $Y$, let $[X,Y]$ denote the homotopy classes
of maps of $X$ into $Y$
\begin{enumerate}[label=(\alph*)]
\item Let $I=[0,1]$. Show that for any $X$, the set $[X,I]$ has a
  single element.
\item Show that if $Y$ is path connected, the set $[I,Y]$ has a
  single element.
\end{enumerate}
\end{problem}
\begin{proof}
\end{proof}
\newpage
\begin{problem}[Munkres \S52, Ex.\,3(a,b,c,)]
A space $X$ is said to be \emph{contractible} if the identity map
$\id_X\colon X\to X$ is nulhomotopic.
\begin{enumerate}[label=(\alph*)]
\item Show that $I$ and $\RR$ are contractible.
\item Show that a contractible space is path connected.
\item Show that f $Y$ is contractible, then for any $X$, the set
  $[X,Y]$ has a single element.
\end{enumerate}
\end{problem}
\begin{proof}
\end{proof}

%%% Local Variables:
%%% mode: latex
%%% TeX-master: "../MA571-HW-Current"
%%% End:
