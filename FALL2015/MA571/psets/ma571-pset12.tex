\begin{problem}[Munkres \S58, Ex.\,7(c)]
Let $A$ be a subspace of $X$; let $j\colon A\hookrightarrow X$ be the
inclusion map, and let $f\colon X\to A$ be a continuous map. Suppose there
is a homotopy $H\colon X\times I\to X$ between the map $j\circ f$ and the
identity map of $X$.
\begin{itemize}
\item[(c)] Give an example in which $j_*$ is not an isomorphism.
\end{itemize}
\end{problem}
\begin{proof}[Example]
\renewcommand\qedsymbol{$\spadesuit$}
\end{proof}
\newpage
\begin{problem}[Munkres \S58, Ex.9(a,b,c)]
We define the \emph{degree} of a continuous map $h\colon S^1\to S^1$ as
follows:
\\\\
Let $b_0$ be the point $(0,1)$ of $S^1$; choose a generator $\gamma$ for
the infinite cyclic group $\pi_1(S^1,b_0)$. If $x_0$ is any point of $S^1$,
choose a path $\alpha$ in $S^1$ from $b_0$ to $x_0$ and define
$\gamma(x_0)\coloneqq\hat\alpha(\gamma)$. Then $\gamma(x_0)$ generates
$\pi_1(S^1,x_0)$. The element $\gamma(x_0)$ is independent of the choice of
the path $\alpha$, since the fundamental group of $S^1$ is Abelian.

Now given $h\colon S^1\to S^1$, choose $x_0\in S^1$ and let
$h(x_0)=x_1$. consider the homomorphism
\[
h_*\colon\pi_1(S^1,x_0)\longrightarrow\pi_1(S^1,x_1).
\]
Since both groups are infinite cyclic, we have
\begin{equation}
\label{eq:1}
\tag{*}
h_*(\gamma(x_0))=d\cdot\gamma(x_1)
\end{equation}
for some integer $d$, if the group is written additively. The integer $d$
is called the \emph{degree} of $h$ is denoted by $\deg h$.

The degree of $h$ is independent of the choice of the generator $\gamma$;
choosing the other generator woul merely change the sign of both sides of
(\ref{eq:1}).
\begin{enumerate}[label=(\alph*)]
\item Show that $d$ is independent of the choice of $x_0$.
\item Show that if $h,k\colon S^1\to S^1$ are homotopic, they have the same
  degre.
\item Show that $\deg(h\circ k)=(\deg h)\cdot(\deg k)$.
\end{enumerate}
\end{problem}
\begin{proof}
\end{proof}
\newpage
\begin{problem}[Munkres \S60, Ex.\,2]
Let $X$ be the quotient space obtained from $B^2$ by identifying each point
$x$ of $S^1$ with its antipode $-x$. Show that $X$ is homeomorphic to the
projective plane $P^2$.
\end{problem}
\begin{proof}
\end{proof}
\newpage
\thispagestyle{empty}
For the problems to come we need the following definitions:
\begin{definition*}
Let $M$ be an  $m$-manifold.
\begin{enumerate}[label=(\roman*)]
\item A \emph{linear} path in $\RR^n$ is a path $f\colon[a,b]\to\RR^n$ with
  $f(s)\coloneqq\tfrac{1}{b-a}[(b-s)z_1+(s-a)z_2]$ for two points $z_1$ and $z_2$.
\item A \emph{quasi-linear} path in $M$ is a path $g\colon[a,b]\to M$ for
  which there is an open set $U$ containing $g([a,b])$ and a homeomorphism
  $h$ from $U$ to an open set $\RR^m$ such that $h\circ g$ is linear.
\item A \emph{piecewise quasi-linear} path in $M$ is a path
  $g\colon[a,b]\to M$ for which there is a partition of $[a,b]$ into
  subintervals such that the restriction of $g$ to each subinterval of the
  partition is  quasi-linear.
\end{enumerate}
\end{definition*}
\newpage
\begin{problem}[A]
\begin{enumerate}[label=(\roman*)]
\item Let $M$ be an $m$-manifold, let $U$ an open set in $M$ which is
  homeomorphic to an open \emph{ball} in $\RR^m$, and let $g$ be a path in
  $U$. Prove that $g$ is a path-homotopic to a quasi-linear path. (Hint:
  straight-line homotopy.)
\item Prove that every path in an $m$-manifold is path-homotopic to a
  piecewise quasi-linear path. (Hint: Theorem 51.3, Lebesgue Lemma and part
  (i)).
\end{enumerate}
\end{problem}
\begin{proof}
\end{proof}
\newpage
\begin{problem}[B]
Prove piecewise quasi-linear path in an $m$-manifold with $m>1$ cannot be
onto. (Hint: Use Problem A from HW 2; you may \emph{assume}, without
proving it, that the image of a linear path does not contain an open set of
$\RR^m$ if $m>1$.)
\end{problem}
\begin{proof}
\end{proof}
\newpage
\begin{problem}[C]
\begin{enumerate}[label=(\roman*)]
\item $S^m$ is an $m$-manifold for all $m$ (you don't have to prove this,
  it follows easily from the solution of HW 8 \#3). Prove that $S^m$ is
  simply connected for $m\geq 2$. Do not use Section 59. (Hint: Use
  Problems A and B from assignment and Problem C from HW 11.)
\item Prove that $\RR^n$ is not homeomorphic to $\RR^2$ for $n\neq
  2$. (Hint: You may use Theorem A from the note on the Fundamental Group
  of the Circle.)
\end{enumerate}
\end{problem}
\begin{proof}
\end{proof}

%%% Local Variables:
%%% mode: latex
%%% TeX-master: "../MA571-HW-Current"
%%% End:
