\begin{problem}[Munkres \S18, p.\,111, \#7(a)]
\begin{enumerate}[noitemsep]
\item[(a)] Suppose that $f\colon\RR\to\RR$ is ``continuous from
  the right,'' that is,
  \[
    \lim_{x\to a+}f(x)=f(a).
  \]
  for each $a\in\RR$. Show that $f$ is continuous when considered
  as a function from $\RR_\ell$ to $\RR$.
\end{enumerate}
\end{problem}
\begin{proof}
Recall the definition of ``right-hand limit,'':
\begin{definition*}[Rudin \S4, p.\,94, Def.\,4.25]
Let $f$ be defined on $(a,b)$. Consider any point $x$ such that
$a\leq x<b$. We write $f(x+)=q$ if $f(t_n)\to q$ as $n\to\infty$,
for all sequences $\left\{t_n\right\}$ in $(x,b)$ such that
$t_n\to x$.
\end{definition*}
This definition is not well suited for our purposes since it is
defined in terms of limits of sequences, which Rudin validates in
Theorem 4.6 (cf.\,Rudin, \S4, p.\,86) by proving that the limit-point
formulation of continuity coincides with the $\epsilon$-$\delta$
formulation. We shall, therefore, reformulate Rudin's definitions
in terms of $\epsilon$'s and $\delta$'s as follows:
\begin{definition*}
$f\colon\RR\to\RR$ is right-continuous at $x_0\in\RR$ if for
every $\epsilon>0$ there exists $\delta>0$ such that $x\in
[x_0,x_0+\delta)$ implies
$f(x)\in\left(f(x_0)-\epsilon,f(x_0)+\epsilon\right)$.
\end{definition*}
We will prove that $f\colon\RR_\ell\to\RR$ is continuous in the
sense: ``for each open subset $V$ of $\RR$, the set $f^{-1}(V)$
is an open subset of $\RR_\ell$'' (cf.\,Munkres, \S18,
p.\,102) and we shall do so in the spirit of Example 1 in Munkres
\S18, p.\,103 and employ Theorem 18.1(4). Recall what basic open
sets look like in the lower-limit topology (defined in Munkres
\S13, pp.\,81-82), they are intervals of the form
$[a,b)\subset\RR$. Without loss of generality, consider the basic
open set $V=(a,b)$ in $\RR$. Let $x_0\in f^{-1}(V)$. Then, since
$f$ is right-continuous, for
$\epsilon\leq\min\left\{f(x_0)-a,b-f(x_0)\right\}$, there exists
$\delta>0$ such that $x\in[x_0,x_0+\delta)$ implies
$f(x)\in\left(f(x_0)-\epsilon,f(x_0)+\epsilon\right)$, i.e.,
\[
f\left([x_0,x_0+\delta)\right)\subset\left(f(x_0)-\epsilon,f(x_0)+\epsilon\right)\subset
V.
\]
By Theorem 18.1(4), $f$ is continuous.
\end{proof}
\newpage
\begin{problem}[Munkres \S18, p.\,112, \#13]
Let $A\subset X$; let $f\colon A\to Y$ be continuous; let $Y$ be
Hausdorff. Show that if $f$ may be extended to a continuous
function $g\colon\clsr A\to Y$, then $g$ is uniquely determined
by $f$.
\end{problem}
\begin{proof}
We shall proceed by contradiction. Suppose that $g_1$ and $g_2$
are distinct continuous extensions of $f$ to the closure of $A$,
i.e., $g_1(x)\neq g_2(x)$ for some $x\in\clsr A\setminus
A$. Recall from Problem 2.7 (Munkres \S13, p.\,101, \#13) that
$Y$ is Hausdorff if and only the diagonal
$\Delta=\left\{\,y\times y\;\middle|\;y\in Y\,\right\}$ is closed
in $Y\times Y$. Now, consider the product map $G=g_1\times
g_2\colon X\to Y\times Y$. This map is continuous by Theorem
18.4 so, by Theorem 18.1(2), $G(\clsr
A)\subset\clsr{G(A)}$. However, since $g_1=g_2$ on $A$ we have
that $G(A)\subset\Delta$ so, by Lemma B (from Prof.\,McClure's
lectures), we have that
\[
G(\clsr{A})\subset\clsr{G(A)}\subset\Delta.
\]
But by assumption $g_1(x)\times g_2(x)\notin\Delta$. This is a
contradiction. Therefore, $g_1=g_2$ on $\clsr A$, i.e., the
extension of $f$ to a continuous function $g$ on $\clsr A$ is
unique.
\end{proof}
\newpage
\begin{problem}[Munkres \S19, p.\,118, \#2]
Prove Theorem 19.3.
\end{problem}
\begin{proof}
Recall the exact statement of Theorem 19.3 from Munkres \S19,
p.\,116:
\begin{theorem*}
Let $A_\alpha$ be a subspace of $X_\alpha$, for each $\alpha\in
J$. Then $\prod A_\alpha$ is a subspace of $\prod X_\alpha$ if
both products are given the box topology, or if both products are
given the product topology.
\end{theorem*}
Our goal is to show that the box (or product) topology on $\prod
A_\alpha$ coincides with the subspace topology on $\prod
A_\alpha$ as a subset of $\prod X_\alpha$ with the product (or
box) topology. That is, we will show that $U$ is open in the
box (or product) topology on $\prod A_\alpha$ if and only if it
is of the form $V\cap\prod A_\alpha$ where $V$ is open in $\prod
X_\alpha$ with the box (or product) topology (cf.\, Munkres \S16
p.\,88 and Munkres \S19, p.\,114 for the relevant
definitions as we will avoid naming these topologies explicitly
in this proof).

$\implies$ By Theorem 13.1, it suffices to consider basic open
sets in $\prod A_\alpha$ (equipped with the box or the product
topology). Let $U$ be a basic open set in $\prod A_\alpha$ with
the box (or product) topology , then, by Theorem 19.2, $U=\prod
U_\alpha$ where $U_\alpha$ is a basic open set in $A_\alpha$ (or
all of $A_\alpha$ for all but finitely many $\alpha$ in the case
of the product topology). Then, since $A_\alpha$ is a subspace of
$X_\alpha$, the set $U_\alpha=V_\alpha\cap A_\alpha$ for
$V_\alpha$ open in $X_\alpha$ (or equal to $X_\alpha$ for all but
finitely many $\alpha$). Then, $V=\prod V_\alpha$ is a basic open
set in $\prod X_\alpha$ (again, by Theorem 19.2) and we have that
\[
U=\prod U_\alpha=\prod\left(V_\alpha\cap A_\alpha\right).
\]
We must now prove that:
\begin{lemma}
\[
\prod (A_\alpha\cap B_\alpha)=\prod A_\alpha\cap\prod B_\alpha.
\]
\end{lemma}
\begin{proof}[Proof of Lemma 6]
\renewcommand\qedsymbol{$\clubsuit$}
The proof is just a matter of chasing definitions and so our
demonstration will be compact, but slightly informal. Recall
from the two lowermost definition in Munkres \S19, p.\,113  that
$\mathbf{x}\in\prod_{\alpha\in J} \left(A_\alpha\cap
  B_\alpha\right)$ if, by definition, $\mathbf{x}\colon
J\to\bigcup_{\alpha\in   J}\left(A_\alpha\cap B_\alpha\right)$
such that $\mathbf{x}(\alpha)\in A_\alpha\cap B_\alpha$ for each
$\alpha\in J$ if and only if $\mathbf{x}(\alpha)\in A_\alpha$,
$\mathbf{x}(\alpha)\in B_\alpha$ if and only if $\mathbf{x}\colon
J\to\bigcup_{\alpha\in J}A_\alpha$, $\mathbf{x}\colon
J\to\bigcup_{\alpha\in J}B_\alpha$ if and only if
$\mathbf{x}\in\prod_{\alpha\in J}A_\alpha$ and
$\mathbf{x}\in\prod_{\alpha\in J}B_\alpha$ if and only if
$\mathbf{x}\in\prod_{\alpha\in J}A_\alpha\cap\prod_{\alpha\in
  J}B_\alpha$.
\end{proof}
Thus, by Lemma 6
\[
U
=\prod\left(V_\alpha\cap A_\alpha\right)
=\prod V_\alpha\cap\prod A_\alpha
=V\cap\prod A_\alpha.
\]

$\impliedby$ Conversely, suppose $U$ is of the form $V\cap\prod
A_\alpha$, i.e., $U$ is a basic open set in the subspace topology
on $\prod A_\alpha$. Without loss of generality, assume that
$V=\prod V_\alpha$ is a basic open set in $\prod X_\alpha$. Then,
by Lemma 6,
\[
U=V\cap\prod A_\alpha=\prod\left(V_\alpha\cap A_\alpha\right).
\]
is a basic open set in the box (or product topology) on $\prod
A_\alpha$ since $V\alpha\cap A_\alpha$ is a basic open set in
$A_\alpha$ with the subspace topology.
\end{proof}
\newpage
\begin{problem}[Munkres \S19, p.\,118, \#3]
Prove Theorem 19.4.
\end{problem}
\begin{proof}
Recall the exact statement of Theorem 19.4 from Munkres \S19,
p.\,116:
\begin{theorem*}
If each space $X_\alpha$ is a Hausdorff space, then $\prod
X_\alpha$ is a Hausdorff space in both the box and product
topologies.
\end{theorem*}
To show that $\prod X_\alpha$ equipped with the box (or the
product) topology we will proceed in the following way: let
$\mathbf{x},\mathbf{y}\in\prod X_\alpha$, it is sufficient,
although not necessary, to show that there exists basic open sets
$U=\prod U_\alpha$ and $V=\prod V_\alpha$ neighborhoods of
$\mathbf{x}$ and $\mathbf{y}$, respectively, such that $U\cap
V=\emptyset$.

We will first demonstrate this for the product topology on $\prod
X_\alpha$. Since $X_\alpha$ is Hausdorff for each $\alpha$, there
exists basic open sets $U_\alpha$ and $V_\alpha$ neighborhoods of
$x_\alpha$ and $y_\alpha$, respectively, such that $U_\alpha\cap
V_\alpha=\emptyset$. For a finite collection of $\alpha$'s, say
$A$, let $U_\alpha'=U_\alpha$ and $V_\alpha'=V_\alpha$ for all
$\alpha\in A$ and $U_\alpha'=X_\alpha$, $V_\alpha'=X_\alpha$
otherwise. Then $U=\prod U_\alpha'$ and $V=\prod V_\alpha'$, by
Theorem 19.2, are open in $\prod X_\alpha$ with the product
topology and, by Lemma 6,
\[
U\cap V=\prod U_\alpha'\cap \prod V_\alpha'
=\prod_{\alpha\in A}\left(U_\alpha\cap
  V_\alpha\right)\times\prod_{\alpha\notin A}X_\alpha
=\prod_{\alpha\in A}\emptyset\times\prod_{\alpha\notin A}X_\alpha
=\emptyset.
\]
Thus $\prod X_\alpha$ with the product topology is Hausdorff.

In the case of $\prod X_\alpha$ with the box topology, we take
$U$ and $V$ to be the basis elements $U=\prod U_\alpha$ and
$V=\prod V_\alpha$ such that $U_\alpha\cap V_\alpha=\emptyset$
for all $\alpha$. Then, by Lemma 6, we have
\[
U\cap V=\prod U_\alpha\cap\prod V_\alpha
=\prod \left(U_\alpha\cap V_\alpha\right)
=\prod\emptyset=\emptyset.
\]
Hence $\prod X_\alpha$ with the box topology is Hausdorff.
\end{proof}
\newpage
\begin{problem}[Munkres \S19, p.\,118, \#6]
Let $\mathbf{x}_1,\mathbf{x}_2,...$ be a sequence of the points
of the product space $\prod X_\alpha$. Show that this sequence
converges to the point $\mathbf{x}$ if and only if the sequence
$\pi_\alpha(\mathbf{x}_1),\pi_\alpha(\mathbf{x}_2),...$ converges
to $\pi_\alpha(\mathbf{x})$ for each $\alpha$. Is this fact true
if one uses the box topology instead of the product topology?
\end{problem}
\begin{proof}
$\implies$ Suppose that the sequence
$\left\{\mathbf{x}_n\right\}$ converges to $\mathbf{x}\in\prod
X_\alpha$ then, from Munkres \S17, p.\,98, for every neighborhood
$U$ of $\mathbf{x}$, there is a positive integer $N$ such that
$\mathbf{x}_n\in U$ for $n\geq N$. Then, since
$\pi_\beta\colon\prod X_\alpha\to X_\beta$ is continuous,
$\pi_\beta(\mathbf{x}_n)=x_\beta^{(n)}\in\pi_\beta(U)$ a
neighborhood of $x_\alpha$ for $n\geq N$ for all $\alpha$. Note
that this holds for $\prod X_\alpha$ with the box topology.

$\impliedby$ Now, suppose that
$\pi_\alpha(\mathbf{x}_n)\to\pi_\alpha(\mathbf{x})$ for all
$\alpha$, i.e., the sequence $x_\alpha^{(n)}\to x_\alpha$ in
$X_\alpha$. Then, for every neighborhood $U_\alpha$ of
$x_\alpha$, there exists a positive integer $N$ such that
$x_\alpha^{(n)}\in U_\alpha$ for all $n\geq N$. Let $A$ be a
finite collection of $\alpha$'s
\end{proof}
\newpage
\begin{problem}[Munkres \S19, p.\,118, \#7]
Let $\RR^\infty$ be the subset of $\RR^\omega$ consisting of all
sequences that are ``eventually zero,'' that is, all sequences
$(x_1,x_2,...)$ such that $x_i\neq 0$ for only finitely many
values of $i$. What is the closure of $\RR^\infty$ in
$\RR^\omega$ in the box and product topologies? Justify your
answer.
\end{problem}
\begin{proof}
\end{proof}
\newpage
\begin{problem}[Munkres \S20, p.\,126, \#3(b)]
Let $X$ be a metric space with metric $d$.
\begin{enumerate}[noitemsep]
\item[(b)] Let $X'$ denote a space having the same underlying set
  as $X$. Show that if $d\colon X'\times X'\to\RR$ is continuous,
  then the topology of $X'$ is finer than the topology of $X$.
\end{enumerate}
\end{problem}
\begin{proof}

\end{proof}
\newpage
\begin{problem}[Munkres \S20, p.\,127, \#4(b)]
Consider the product, uniform and box topologies on $\RR^\omega$
\begin{enumerate}[noitemsep]
\item[(b)] In which topologies do the following sequences converge?
\begin{align*}
\mathbf{w}_1&=(1,1,1,1,...),&\mathbf{x}_1&=(1,1,1,1,...),\\
\mathbf{w}_2&=(0,2,2,2,...),&\mathbf{x}_2&=\left(0,\tfrac{1}{2},\tfrac{1}{2},\tfrac{1}{2},...\right),\\
\mathbf{w}_3&=(0,0,3,3,...),&\mathbf{x}_3&=\left(0,0,\tfrac{1}{3},\tfrac{1}{3},...\right),\\
&\vdotswithin{=}&&\vdotswithin{=}\\
\mathbf{y}_1&=(1,0,0,0,...)&\mathbf{z}_1&=(1,1,0,0,...),\\
\mathbf{y}_2&=\left(\tfrac{1}{2},\tfrac{1}{2},0,0,...\right)&\mathbf{z}_2&=\left(\tfrac{1}{2},\tfrac{1}{2},0,0,...\right),\\
\mathbf{y}_3&=\left(\tfrac{1}{3},\tfrac{1}{3},\tfrac{1}{3},0,...\right)&\mathbf{z}_3&=\left(\tfrac{1}{3},\tfrac{1}{3},0,0,...\right),\\
&\vdotswithin{=}&&\vdotswithin{=}
\end{align*}
\end{enumerate}
\end{problem}
\begin{proof}
\end{proof}
\newpage
\begin{problem}[A]
Given: $X$ a metric space, $A$ a countable subset of $X$, and
$\clsr A=X$. To prove: the topology of $X$ has a countable
basis.
\end{problem}
\begin{proof}
\end{proof}
\newpage
\begin{problem}[B]
Given: $Y$ is an ordered set, $(a,b)$ and $(c,d)$ are disjoint
open intervals, and there are elements $x\in(a,b)$ and
$y\in(c,d)$ with $x<y$. To prove: every element of $(a,b)$ less
than every element of $(c,d)$.
\end{problem}
\begin{proof}
\end{proof}
\newpage
\begin{problem}[C]
(This problem will be used when we discuss quotient spaces). Let
$S$ and $T$ be sets and let $f\colon S\to T$ be a function. Let
$A\subset S$.
\begin{enumerate}[noitemsep,label=(\roman*)]
\item Give an example to show that the equation
\begin{equation}
\label{eq:mcclure-1}
\tag{*}
f^{-1}(f(A))=A
\end{equation}
isn't always valid.
\item Define an equivalence relation $\sim$ on $S$ by $s\sim s'$
  if and only if $f(s)=f(s')$. Using this equivalence relation,
  describe the subsets $A$ of $S$ for which (\ref{eq:mcclure-1}) is
  true. Prove that your answer is correct.
\end{enumerate}
\end{problem}
\begin{proof}
\end{proof}

%%% Local Variables:
%%% mode: latex
%%% TeX-master: "../MA571-HW-Current"
%%% End:
