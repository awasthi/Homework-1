\begin{problem}[Munkres \S18, p.\,111, \#7(a)]
\begin{enumerate}[noitemsep]
\item[(a)] Suppose that $f\colon\RR\to\RR$ is ``continuous from
  the right,'' that is,
  \[
    \lim_{x\to a+}f(x)=f(a).
  \]
  for each $a\in\RR$. Show that $f$ is continuous when considered
  as a function from $\RR_\ell$ to $\RR$.
\end{enumerate}
\end{problem}
\begin{proof}
Recall the definition of ``right-hand limit,'':
\begin{definition*}[Rudin \S4, p.\,94, Def.\,4.25]
Let $f$ be defined on $(a,b)$. Consider any point $x$ such that
$a\leq x<b$. We write $f(x+)=q$ if $f(t_n)\to q$ as $n\to\infty$,
for all sequences $\left\{t_n\right\}$ in $(x,b)$ such that
$t_n\to x$.
\end{definition*}
This definition is not well suited for our purposes since it is
defined in terms of limits of sequences, which Rudin validates in
Theorem 4.6 (cf.\,Rudin, \S4, p.\,86) by proving that the limit-point
formulation of continuity coincides with the $\epsilon$-$\delta$
formulation. We shall, therefore, reformulate Rudin's definitions
in terms of $\epsilon$'s and $\delta$'s as follows:
\begin{definition*}
$f\colon\RR\to\RR$ is right-continuous at $x_0\in\RR$ if for
every $\epsilon>0$ there exists $\delta>0$ such that $x\in
[x_0,x_0+\delta)$ implies
$f(x)\in\left(f(x_0)-\epsilon,f(x_0)+\epsilon\right)$.
\end{definition*}
We will prove that $f\colon\RR_\ell\to\RR$ is continuous in the
sense: ``for each open subset $V$ of $\RR$, the set $f^{-1}(V)$
is an open subset of $\RR_\ell$'' (cf.\,Munkres, \S18,
p.\,102) and we shall do so in the spirit of Example 1 in Munkres
\S18, p.\,103 and employ Theorem 18.1(4). Recall what basic open
sets look like in the lower-limit topology (defined in Munkres
\S13, pp.\,81-82), they are intervals of the form
$[a,b)\subset\RR$. Without loss of generality, consider the basic
open set $V=(a,b)$ in $\RR$. Let $x_0\in f^{-1}(V)$. Then, since
$f$ is right-continuous, for
$\epsilon\leq\min\left\{f(x_0)-a,b-f(x_0)\right\}$, there exists
$\delta>0$ such that $x\in[x_0,x_0+\delta)$ implies
$f(x)\in\left(f(x_0)-\epsilon,f(x_0)+\epsilon\right)$, i.e.,
\[
f\left([x_0,x_0+\delta)\right)\subset\left(f(x_0)-\epsilon,f(x_0)+\epsilon\right)\subset
V.
\]
By Theorem 18.1(4), $f$ is continuous.
\end{proof}
\newpage
\begin{problem}[Munkres \S18, p.\,112, \#13]
Let $A\subset X$; let $f\colon A\to Y$ be continuous; let $Y$ be
Hausdorff. Show that if $f$ may be extended to a continuous
function $g\colon\clsr A\to Y$, then $g$ is uniquely determined
by $f$.
\end{problem}
\begin{proof}
We shall proceed by contradiction. Suppose that $g_1$ and $g_2$
are distinct continuous extensions of $f$ to the closure of $A$,
i.e., $g_1(x)\neq g_2(x)$ for some $x\in\clsr A\setminus
A$. Recall from Problem 2.7 (Munkres \S13, p.\,101, \#13) that
$Y$ is Hausdorff if and only the diagonal
$\Delta=\left\{\,y\times y\;\middle|\;y\in Y\,\right\}$ is closed
in $Y\times Y$. Now, consider the product map $G=g_1\times
g_2\colon X\to Y\times Y$. This map is continuous by Theorem
18.4 so, by Theorem 18.1(2), $G(\clsr
A)\subset\clsr{G(A)}$. However, since $g_1=g_2$ on $A$ we have
that $G(A)\subset\Delta$ so, by Lemma B (from Prof.\,McClure's
lectures), we have that
\[
G(\clsr{A})\subset\clsr{G(A)}\subset\Delta.
\]
But by assumption $g_1(x)\times g_2(x)\notin\Delta$. This is a
contradiction. Therefore, $g_1=g_2$ on $\clsr A$, i.e., the
extension of $f$ to a continuous function $g$ on $\clsr A$ is
unique.
\end{proof}
\newpage
\begin{problem}[Munkres \S19, p.\,118, \#2]
Prove Theorem 19.3.
\end{problem}
\begin{proof}
Recall the exact statement of Theorem 19.3 from Munkres \S19,
p.\,116:
\begin{theorem*}
Let $A_\alpha$ be a subspace of $X_\alpha$, for each $\alpha\in
J$. Then $\prod A_\alpha$ is a subspace of $\prod X_\alpha$ if
both products are given the box topology, or if both products are
given the product topology.
\end{theorem*}
Our goal is to show that the  product topology on $\prod
A_\alpha$ corresponds to subspace topology on $\prod
A_\alpha$ as a subset of $\prod X_\alpha$ with the
product topology. That is, we will show that $U$ is open in
$\prod A_\alpha$ if and only if it can be expressed in the form
$V\cap\prod A_\alpha$ where $V$ is open in $\prod X_\alpha$ with
the product topology (cf.\,Munkres \S16, p.\,88 and Munkres \S19,
pp.\,114-116 for the relevant definitions).

$\implies$ By Lemma 13.3(2), it is enough to consider basic open
sets in $\prod A_\alpha$. Recall from Theorem 19.2 that $U$ is a
basic open set in $\prod A_\alpha$ if it is of the form $U=\prod
U_\alpha$ where $U_\alpha$ is a basic open set in $A_\alpha$ and
is equal to $A_\alpha$ for all but finitely many $\alpha$. Then,
by Lemma 16.1, $U_\alpha$ is of the form $U_\alpha=V_\alpha\cap
A_\alpha$ for $V_\alpha$ open in $X_\alpha$ and
$V_\alpha=X_\alpha$ for all but finitely many $\alpha$. Then, by
Theorem 19.2, the set $V=\prod V_\alpha$ is a basic open set in
$\prod X_\alpha$. We claim that
\[
U=V\cap\prod A_\alpha.
\]
But this follows from the remarks made by Munkres in \S19,
p.\,144, namely that
\[
V\cap\prod A_\alpha=\prod V_\alpha\cap\prod
A_\alpha=\prod\left(V_\alpha\cap A_\alpha\right)=\prod U_\alpha.
\]
This fact is one which Munkres does not prove, but which
nevertheless can be effortlessly proven:
\begin{lemma}
\[\prod\left(A_\alpha\cap B_\alpha\right)=\prod
  A_\alpha\cap\prod B_\alpha.\]
\end{lemma}
\begin{proof}
\renewcommand\qedsymbol{$\clubsuit$}
The proof is just a matter of chasing definitions so we will be a
little informal. The point $\mathbf{x}\in\prod\left(A_\alpha\cap
  B_\alpha\right)$ if and only if
$\pi_\alpha(\mathbf{x})=x_\alpha\in A_\alpha\cap B_\alpha$ for
all $\alpha$ if and only if $x_\alpha\in A_\alpha$ and
$x_\alpha\in B_\alpha$ for all $\alpha$ if and only if
$\mathbf{x}\in\prod A_\alpha$ and $\mathbf{x}\in\prod B_\alpha$
if and only if $\mathbf{x}\in\prod A_\alpha\cap\prod B_\alpha$.
 \end{proof}

$\impliedby$ Conversely, suppose $V\cap\prod A_\alpha$ for some
basic open set $V=\prod V_\alpha$ where $V_\alpha=X_\alpha$ for
all but finitely many $\alpha$). Then, by Lemma 6, we have
\[
V\cap\prod A_\alpha=\prod\left(V_\alpha\cap A_\alpha\right)
\]
where $U_\alpha=V_\alpha\cap A_\alpha$, with $U_\alpha=A_\alpha$
for all but finitely many $\alpha$, is a basic open set in
$A_\alpha$ with the subspace topology. Hence, $\prod U_\alpha$ is
a basic open set in the product topology on $\prod A_\alpha$. We
have shown that a basic open set in the product topology on
$\prod A_\alpha$ corresponds to a basic open set in the subspace
topology on $\prod A_\alpha$. This is true for every $x_0\in
X'$. Therefore, by Lemma 13.3(2), the box (or product) topology
on $\prod A_\alpha$ is equivalent to the subspace topology on
$\prod A_\alpha$.
\end{proof}
\newpage
\begin{problem}[Munkres \S19, p.\,118, \#3]
Prove Theorem 19.4.
\end{problem}
\begin{proof}
Recall the exact statement of Theorem 19.4 from Munkres \S19,
p.\,116:
\begin{theorem*}
If each space $X_\alpha$ is a Hausdorff space, then $\prod
X_\alpha$ is a Hausdorff space in both the box and product
topologies.
\end{theorem*}
To show that $\prod X_\alpha$ equipped with the box (or the
product) topology we will proceed in the following way: let
$\mathbf{x},\mathbf{y}\in\prod X_\alpha$, it is sufficient,
although not necessary, to show that there exists basic open sets
$U=\prod U_\alpha$ and $V=\prod V_\alpha$ neighborhoods of
$\mathbf{x}$ and $\mathbf{y}$, respectively, such that $U\cap
V=\emptyset$.

We will first demonstrate this for the product topology on $\prod
X_\alpha$. Since $X_\alpha$ is Hausdorff for each $\alpha$, there
exists basic open sets $U_\alpha$ and $V_\alpha$ neighborhoods of
$x_\alpha$ and $y_\alpha$, respectively, such that $U_\alpha\cap
V_\alpha=\emptyset$. For a finite collection of $\alpha$'s, say
$A$, let $U_\alpha'=U_\alpha$ and $V_\alpha'=V_\alpha$ for all
$\alpha\in A$ and $U_\alpha'=X_\alpha$, $V_\alpha'=X_\alpha$
otherwise. Then $U=\prod U_\alpha'$ and $V=\prod V_\alpha'$, by
Theorem 19.2, are open in $\prod X_\alpha$ with the product
topology and, by Lemma 6,
\[
U\cap V=\prod U_\alpha'\cap \prod V_\alpha'
=\prod_{\alpha\in A}\left(U_\alpha\cap
  V_\alpha\right)\times\prod_{\alpha\notin A}X_\alpha
=\prod_{\alpha\in A}\emptyset\times\prod_{\alpha\notin A}X_\alpha
=\emptyset.
\]
Thus $\prod X_\alpha$ with the product topology is Hausdorff.
\end{proof}
\newpage
\begin{problem}[Munkres \S19, p.\,118, \#6]
Let $\mathbf{x}_1,\mathbf{x}_2,...$ be a sequence of the points
of the product space $\prod X_\alpha$. Show that this sequence
converges to the point $\mathbf{x}$ if and only if the sequence
$\pi_\alpha(\mathbf{x}_1),\pi_\alpha(\mathbf{x}_2),...$ converges
to $\pi_\alpha(\mathbf{x})$ for each $\alpha$. Is this fact true
if one uses the box topology instead of the product topology?
\end{problem}
\begin{proof}
$\implies$ Suppose that the sequence
$\left\{\mathbf{x}_n\right\}$ converges to $\mathbf{x}\in\prod
X_\alpha$ then, from Munkres \S17, p.\,98, for every neighborhood
$U$ of $\mathbf{x}$, there is a positive integer $N$ such that
$\mathbf{x}_n\in U$ for $n\geq N$. We claim that
\begin{lemma}
$\pi_\beta\colon\prod X_\alpha\to X_\beta$ is an open map on
$\prod X_\alpha$ with the product or the box topology.
\end{lemma}
\begin{proof}
\renewcommand\qedsymbol{$\clubsuit$}
This is \emph{forced} by the basis definition for the product
topology (cf.\,Theorem 19.2, Munkres \S19, p.\,114). That is, the
$U$ is a basic open set if and only if it is of the form $\prod
U_\alpha$ where $U_\alpha$ is open in $X_\alpha$ (and
$U_\alpha=X_\alpha$ for all but finitely many $\alpha$ in the
case of the product topology). Then clearly $\pi_\beta(\prod
U_\alpha)=U_\beta$ is open in $X_\alpha$. This extends to
arbitrary open sets in $\prod X_\alpha$ equipped with the product
or box topology.
\end{proof}
\noindent Now, since $\pi_\alpha$ is an open map $\pi_\alpha(U)$
is a neighborhood of $\pi_\alpha(\mathbf{x})$ such that
$\pi_\alpha(\mathbf{x}_n)\in\pi_\alpha(U)$ for all $n\geq N$ for
all $\alpha$. Thus, the sequence
$\left\{\pi_\alpha(\mathbf{x}_n)\right\}$ converges to
$\pi_\alpha(\mathbf{x})$ in $X_\alpha$ for all $\alpha$.

$\impliedby$ Conversely, suppose that the sequence
$\left\{\pi_\alpha(\mathbf{x})\right\}$ converges to
$\pi_\alpha(\mathbf{x})$ for all $\alpha$. Then, for every
neighborhood $U_\alpha$ of $\pi_\alpha(\mathbf{x})$ there exists
a positive integer $N$ such that $n\geq N$ implies
$\pi_\alpha(\mathbf{x}_n)\in U_\alpha$ for all $n\geq N$ for all
$\alpha$. Let $U$ be a basic neighborhood of $\mathbf{x}$. Then,
by Theorem 19.2, $U=\prod U_\alpha$ where $U_\alpha$ is an
open set in $X_\alpha$ and $U_\alpha=X_\alpha$ for all but
finitely many $\alpha$. Let
$B=\left\{\beta_1,...,\beta_k\right\}$ be the collection of
$\alpha$'s for which $U_\alpha\neq X_\alpha$. Then, since
$\pi_{\beta_i}(\mathbf{x}_n)\to\pi_{\beta_i}(\mathbf{x})$
converges and $\pi_{\beta_i}(U)$ is a neighborhood of
$\pi_{\beta_i}(\mathbf{x})$, there exists a positive integer
$N_i$ such that $\pi_{\beta_i}(\mathbf{x}_n)\in\pi_{\beta_i}(U)$
for all $n\geq N_i$ (note that we are not ignoring the case where
$U_\alpha=X_\alpha$ since in that case the sequence
$\left\{\pi_\alpha(\mathbf{x})\right\}\in X_\alpha$ for all $n$
and everything works out nicely). Take
$N=\max\left\{N_1,...,N_k\right\}$. Then we claim that
$\mathbf{x}_n\in U$ for every $n\geq N$. But this follows from
construction for $\mathbf{x}_n\in U$ if and only if
$\pi_\alpha(\mathbf{x}_n)\in U_\alpha$ (cf.\,Munkres \S19,
p.\,113) for all $\alpha$ if $n\geq N_i$, but $n\geq N\geq N_i$
for all $i$ so $\pi_\alpha(\mathbf{x}_n)\in U_\alpha$. Thus, the
sequence $\left\{\mathbf{x}_n\right\}$ converges to $\mathbf{x}$
in the product topology on $\prod X_\alpha$.

The argument $\mathbf{x}_n\to\mathbf{x}$ $\implies$
$\pi_\alpha(\mathbf{x}_n)\to\pi_\alpha(\mathbf{x})$ carries over
to the box topology on $\prod X_\alpha$ since $\pi_\alpha$ is an
open map. That is, for every open neighborhood $U$ of
$\mathbf{x}$ there exists a positive integer $N$ such that
$\mathbf{x}_n\in U$ for all $n\geq N$. Then, $\pi_\alpha(U)$ is a
neighborhood of $\pi_\alpha(\mathbf{x})$ with
$\mathbf{x}_n\in\pi_\alpha(U)$ for all $n\geq N$ for all
$\alpha$. Thus, the sequence
$\left\{\pi_\alpha(\mathbf{x}_n)\right\}$ converges to
$\pi_\alpha(\mathbf{x})$ in $X_\alpha$.

The reverse implication, however, does not hold. It is not too
difficult to see why the reverse implication may not hold since
we relied on the property that for a basic open set $U$ in $\prod
X_\alpha$, $\prod_\alpha(U)=X_\alpha$ for all but finitely many
$\alpha$ and from this we were able to pick a maximum positive
integer $N$ that made the sequence $\mathbf{x}$ converge in
$\prod X_\alpha$. Because in the product topology, we do not have
the condition that $\pi_\alpha(U)=X_\alpha$ for all but finitely
many $\alpha$, the set $\left\{N_\alpha\right\}$ of positive
integers does not necessarily posses a maximum element. More
concretely, consider $X=\RR^\omega$ in the box topology from
Example 2 in Munkres \S19, p.\,117. Consider the sequence
$\left\{x_n\right\}$ where $x_n=1/n$ in $\RR$. It is clear that
$x_n\to 0$, but we claim that the sequence
$\mathbf{x}_n=(x_n,x_n,...)\nrightarrow(0,0,...)=\mathbf{0}$ in
$\RR^\omega$. That is, we must show that there exist an open
neighborhood $U$ of $\mathbf{0}$ such that for every positive
integer $N$, $\mathbf{x}_n\notin U$ for some $n\geq N$. Then
consider the neighborhood $U=\prod_{k=1}^\infty U_k$ of
$\mathbf{0}$ for open neighborhoods $U_k=(-1/k,1/k)$ of
$0$. Thus, if $\mathbf{x}_n\in U$,
$\pi_k(\mathbf{x}_n)=1/n>1/N\in U_k$ for all $k$, but
$\pi_{N+1}(\mathbf{x}_N)=1/N\notin U_{N+1}$. Thus,
$\mathbf{x}_N\in U$ not so the sequence
$\left\{\mathbf{x}_n\right\}$ does not converge to
$\mathbf{0}$. In particular, we note that the sequence
$\left\{\mathbf{w}_n\right\}$ is not Cauchy (cf.\,Rudin \S3,
pp.\,52-54) so it does not converge.
\end{proof}
\newpage
\begin{problem}[Munkres \S19, p.\,118, \#7]
Let $\RR^\infty$ be the subset of $\RR^\omega$ consisting of all
sequences that are ``eventually zero,'' that is, all sequences
$(x_1,x_2,...)$ such that $x_i\neq 0$ for only finitely many
values of $i$. What is the closure of $\RR^\infty$ in
$\RR^\omega$ in the box and product topologies? Justify your
answer.
\end{problem}
\begin{proof}
We first consider the case when $\RR^\omega$ is equipped with the
product topology. We claim that
$\clsr{\RR^\infty}=\RR^\omega$. Let $\mathbf{x}\in\RR^\omega$. It
is enough to show that for every basic neighborhood $U=\prod
U_\alpha$ of $\mathbf{x}$, where $U_\alpha=\RR$ for all but
finitely many $\alpha$ (by Theorem 19.2), the intersection
$U\cap\RR^\omega \neq\emptyset$. Without loss of generality, we
may assume $U_i\neq\RR$ for $i\in\{1,..,n\}$. Then the point
$\mathbf{y}=(y_1,...,y_n,0,0,...)\in \RR^\infty$ where
$y_i\in U_i$ for $i\in\{1,..,n\}$. In particular,
$U\cap\RR^\infty\neq\emptyset$ so
$\clsr{\RR^\infty}=\RR^\omega$.

Now let us consider the case when $\RR^\omega$ is equipped with
the box toplogy. It this case, we claim that
$\clsr{\RR^\infty}=\RR^\infty$, in particular, we prove that
$\RR^\omega\setminus\RR^\infty$ is open. Let
$\mathbf{x}\in\RR^\omega\setminus\RR^\infty$. Then $\mathbf{x}$
is a point such that $x_n\neq 0$ for infinitely
many $n$. Since $\RR$ is Hausdorff, choose neighborhoods
$U_i$ of $x_i$ such that $0\notin U_i$ whenever
$x_i\neq 0$ (and whenever $x_i=0$ choose arbitrary neighborhoods
of $0$). Then the set $U=\prod U_i$ is a basic open set in
$\RR^\omega$ such that $\prod U_i\cap\RR^\infty=\emptyset$ since
every point $\mathbf{y}'\in U$ is zero for only finitely
many $y_n$. Therefore, $U\subset\RR^\omega\setminus\RR^\infty$ so
$\RR^\infty$ is closed, i.e., $\clsr{\RR^\infty}=\RR$.
\end{proof}
\newpage
\begin{problem}[Munkres \S20, p.\,126, \#3(b)]
Let $X$ be a metric space with metric $d$.
\begin{enumerate}[noitemsep]
\item[(b)] Let $X'$ denote a space having the same underlying set
  as $X$. Show that if $d\colon X'\times X'\to\RR$ is continuous,
  then the topology of $X'$ is finer than the topology of $X$.
\end{enumerate}
\end{problem}
\begin{proof}
Suppose that $d\colon X'\times X'\to\RR$ is continuous. Fix
$x_0\in X$ and $\epsilon>0$. Recall from by Problem 2.8 (Munkres
\S18, Ex.\,\#4) the map $\iota(x)=x\times x_0$ is an imbedding,
in particular it is continuous so by theorem 18.2(c), the
composite map $d_0=d\circ\iota\colon X\to\RR$ is
continuous. Then, we claim
\[
B_d(x_0,\epsilon)=\left\{\,x\in
  X\;\middle|\;d(x_0,y)<\epsilon\,\right\}=d_0^{-1}((-\epsilon,\epsilon)).
\]
To see this let $y\in B_d(x_0,\epsilon)$. Then
$d_0(y)=d\circ\iota(y)=d(x_0,y)<\epsilon$ so $y\in
d_0^{-1}((-\epsilon,\epsilon))$. To see the reverse containment
take $y\in d_0^{-1}((-\epsilon,\epsilon))$. Then, similarly, we
have that $d(x_0,y)=d_0(y)=d\circ\iota(y)<\epsilon$ so $y\in
B_d(x_0,\epsilon)$. Hence, by Theorem 13.3(2), the topology on
$X'$ is finer than the topology on $X$ induced by $d$.
\end{proof}
\newpage
\begin{problem}[Munkres \S20, p.\,127, \#4(b)]
Consider the product, uniform and box topologies on $\RR^\omega$
\begin{enumerate}[noitemsep]
\item[(b)] In which topologies do the following sequences converge?
\begin{align*}
\mathbf{w}_1&=(1,1,1,1,...),&\mathbf{x}_1&=(1,1,1,1,...),\\
\mathbf{w}_2&=(0,2,2,2,...),&\mathbf{x}_2&=\left(0,\tfrac{1}{2},\tfrac{1}{2},\tfrac{1}{2},...\right),\\
\mathbf{w}_3&=(0,0,3,3,...),&\mathbf{x}_3&=\left(0,0,\tfrac{1}{3},\tfrac{1}{3},...\right),\\
&\vdotswithin{=}&&\vdotswithin{=}\\
\mathbf{y}_1&=(1,0,0,0,...)&\mathbf{z}_1&=(1,1,0,0,...),\\
\mathbf{y}_2&=\left(\tfrac{1}{2},\tfrac{1}{2},0,0,...\right)&\mathbf{z}_2&=\left(\tfrac{1}{2},\tfrac{1}{2},0,0,...\right),\\
\mathbf{y}_3&=\left(\tfrac{1}{3},\tfrac{1}{3},\tfrac{1}{3},0,...\right)&\mathbf{z}_3&=\left(\tfrac{1}{3},\tfrac{1}{3},0,0,...\right),\\
&\vdotswithin{=}&&\vdotswithin{=}
\end{align*}
\end{enumerate}
\end{problem}
\begin{proof}
We were asked to work only on $\mathbf{w}$. We claim that
$\mathbf{w}_n\to\mathbf{0}$ in the product topology on $\RR^\omega$
and that $\mathbf{w}$ does not converge in the uniform topology
and the box topology.

To see that $\mathbf{w}_n\to\mathbf{0}$ in $\RR^\omega$ with the
product topology we appeal to Problem 3.5. That is,
$\mathbf{w}_n\to\mathbf{0}$ if and only if $w_n\to 0$ for all
$n$. But this is clear since for the sequence $w_k^{(n)}=0$ for
all $n\geq k$ so $w_k^{(n)}\in U$ for any neighborhood $U$ of $0$
for all $n\geq k$. Thus, by Problem 3.5, we have that
$\mathbf{w}_n\to\mathbf{0}$.

By Theorem 20.4, it is enough to prove that
$\mathbf{w}_n\nrightarrow\mathbf{0}$ in the uniform topology since
a counterexample in the uniform topology provides a
counterexample in the box topology on $\RR^\omega$. By the
definition of the uniform topology (cf.\,Munkres \S20,
p.\,124). Consider the ball of radius $1$,
$U=B_{\text{unif}}(\mathbf{0},1)$, in the uniform topology on
$\RR^\omega$, that is, the set
\[
U
=\left\{\,\mathbf{x}\in\RR^\omega\;\middle|\;
\clsr{\rho}(\mathbf{0},\mathbf{x})<1\,\right\}.
\]
Then
\begin{align*}
\clsr\rho (\mathbf{0},\mathbf{w}_n)
&=\sup\left\{\clsr d(0,0),...,\clsr d(0,0),\clsr d(0,n),\clsr
  d(0,n),...\right\}\\
&=\sup\{0,...,0,1,1,...\}\\
&=1
\end{align*}
for all $n$. Hence the sequence $\left\{\mathbf{w}_n\right\}$ is
never in $U$ for any $n$ so $\left\{\mathbf{w}\right\}$ does not
converge to $\mathbf{0}$.
\end{proof}
\newpage
\begin{problem}[A]
Given: $X$ a metric space, $A$ a countable subset of $X$, and
$\clsr A=X$. To prove: the topology of $X$ has a countable
basis.
\end{problem}
\begin{proof}
Let $d\colon X\times X\to\RR$ denote the metric on $X$. We claim
that the collection
\[
\mathcal{B}=\bigcup_{a\in A} B_a
\quad
\text{where}
\quad
B_a=
\left\{\,B_d\left(a,\tfrac{1}{k}\right)\;\middle|\;k\in \ZZ\,\right\}
\]
is a countable basis for the topology on $X$ induced by the
metric $d$. First $B_a$ is countable, by Theorem 7.1(3), since
there is an injection,
namely the map
$B_d\left(a,\frac{1}{k}\right)\overset{f}{\longmapsto} k$ (since if
$f\left(B_d\left(a,\frac{1}{k}\right)\right)=k=\ell=f\left(B_d\left(a,\frac{1}{\ell}\right)\right)$
then $B_d\left(a,\frac{1}{k}\right)=B_d\left(a,\frac{1}{\ell}\right)$), from
$B_a$ to $\ZZ_+$. Then $\mathcal{B}$ is countable by Theorem 7.5
since it is a countable union of countable sets.

Now we will prove that $\mathcal{B}$ is indeed a basis. Let $y\in
X$. Then, since $A$ is dense in $X$, for any
open ball $B_d(y,\epsilon)$, for $\epsilon>0$, there is
$B_d(y,\epsilon)\cap A\neq\emptyset$. By the Archimedean
property of $\RR$ (cf.\,Munkres \S4, Theorem 4.2), we may choose
$N>2/\epsilon$ so that we have
$B_d\left(y,\frac{1}{N}\right)\subset B_d(y,\epsilon)$. Then,
since $A$ is dense in $X$, $B_d\left(y,\frac{1}{N}\right)\cap
A\neq\emptyset$. Let $x\in B_d\left(y,\frac{1}{N}\right)\cap
A$. Then $y\in B_d\left(x,\frac{1}{N}\right)\subset
B_d(y,\epsilon)$ so by Lemma 13.2, $\mathcal{B}$ is a basis for
the topology of $X$. Hence, the topology of $X$ has a countable
basis.
\end{proof}
\newpage
\begin{problem}[B]
Given: $Y$ is an ordered set, $(a,b)$ and $(c,d)$ are disjoint
open intervals, and there are elements $x\in(a,b)$ and
$y\in(c,d)$ with $x<y$. To prove: every element of $(a,b)$ is less
than every element of $(c,d)$.
\end{problem}
\begin{proof}
First, it is evident that $a<b$ and $c<d$. Let $w\in(a,b)$ and
$z\in(c,d)$. Then $w<x$ or $w>x$ and $z<y$ or $z>y$ (there are
four cases in total). In most of the cases we shall proceed by
contradiction.
\begin{enumerate}[label=Case \arabic*:]
\item Suppose $w<x$ and $z<y$. If $z<w$, then the interval
  $(a,b)\cap (c,d)\neq\emptyset$ since $x\in (z,y)\subset (c,d)$ and
  $x\in (a,b)$ by hypothesis. This is a contradiction.
\item Suppose $w>x$ and $z<y$. If $z<w$, then the interval
  $(a,b)\cap (c,d)\neq\emptyset$ since $(z,y)\subset (c,d)$ and
  $w\in (z,y)$. This is a contradiction.
\item Suppose $w<x$ and $z>y$. Then by the transitivity of the
  order relation (cf.\,Munkres \S1, p.\,24), $w<x<y<z$ so
  $w<z$.
\item Suppose $w>x$ and $z>y$. If $z<w$, then
  $(a,b)\cap(c,d)\neq\emptyset$  since $y\in (x,w)\subset (a,b)$
  and $y\in(c,d)$ by hypothesis. This is a contradiction.
\end{enumerate}
In every case, we have that $w<z$.
\end{proof}
\newpage
\begin{problem}[C]
(This problem will be used when we discuss quotient spaces). Let
$S$ and $T$ be sets and let $f\colon S\to T$ be a function. Let
$A\subset S$.
\begin{enumerate}[noitemsep,label=(\roman*)]
\item Give an example to show that the equation
\begin{equation}
\label{eq:mcclure-1}
\tag{*}
f^{-1}(f(A))=A
\end{equation}
isn't always valid.
\item Define an equivalence relation $\sim$ on $S$ by $s\sim s'$
  if and only if $f(s)=f(s')$. Using this equivalence relation,
  describe the subsets $A$ of $S$ for which (\ref{eq:mcclure-1}) is
  true. Prove that your answer is correct.
\end{enumerate}
\end{problem}
\begin{proof}
(i) Problem 1.1 (Munkres \S2, Ex.\,1(b)) gives a hint as to what
sort of map $f$ might be. In particular, we are looking for a map
between sets $f\colon S\to T$ that is not surjective. Consider
the map $f\colon x\mapsto 0$ from $\RR\subset\RR$ to $\RR$. Then
$f(A)=\{0\}$ for any subset $A\subset\RR$, but
$f^{-1}(f(A))=f^{-1}(\{0\})=\RR\neq A$ if $A\subsetneq\RR$.
\\\\
(ii) First we shall demonstrate that $\sim$ is indeed an
equivalence relation. By the definition of equivalence class in
Munkres \S3, p.\,22, we must show that:
\begin{enumerate}[noitemsep,label=(\arabic*)]
\item $s\sim s$ for every $s$ in $A$.
\item If $s\sim s'$, then $s'\sim s$.
\item If $s\sim s'$ and $s'\sim s''$, then $s'\sim s''$.
\end{enumerate}
(1): Is clear since if $s=s$, $f(s)=f(s)$ so $s\sim s$. (2):
Suppose $s\sim s'$, then $f(s)=f(s')$ so $f(s')=f(s)$ thus,
$s'\sim s'$. (3): Lastly, if $s\sim s'$ and $s'\sim s''$ then
$f(s)=f(s')$ and $f(s')=f(s'')$ so $f(s)=f(s'')$ hence, $s\sim
s''$. We conclude that $\sim$ is indeed an equivalence relation
on $S$.

Now, we claim that the subsets $A$ of $S$ for which
(\ref{eq:mcclure-1}) is true are disjoint unions of equivalence
classes of points $s$ in $S$ under $\sim$. That is, we will show
that $s\in A$ if and only if $f^{-1}(f(s))\subset A$.

$\implies$ Suppose $s\in A$. Then, we have that $f^{-1}(f(A))=A$
then $f(s)\in f(A)$ so $f^{-1}(f(s))\subset f^{-1}(f(A))=A$. Thus
$f^{-1}(f(s))\subset A$.

$\impliedby$ Conversely, if $f^{-1}(f(s))\subset A$ then $A$
contains every point $s'\in S$ such that $f(s')=f(s)$, in
particular $f(s)=f(s)$ so $s\in A$.
\end{proof}

%%% Local Variables:
%%% mode: latex
%%% TeX-master: "../MA571-HW-Current"
%%% End:
