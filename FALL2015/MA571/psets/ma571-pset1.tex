\begin{problem}[Munkres \S2, 1(a,b).]
Let $f\colon A\to B$. Let $A_0\subset A$ and $B_0\subset B$.
\begin{enumerate}[noitemsep,label=(\alph*)]
\item Show that $A_0\subset f^{-1}(f(A_0))$ and that equality
  holds if $f$ is injective.
\item Show that $f(f^{-1}(B_0))\subset B_0$ and that equality
  holds if $f$ is surjective.
\end{enumerate}
\end{problem}
\begin{proof}
(a). First, we will show $A_0\subset f^{-1}(f(A_0))$. Let $x\in
A_0$. Then $f(x)\in f(A_0)$. By definition, $f^{-1}(f(A_0))$ is
the set of those points $x_0\in A$ such that $f(x_0)\in f(A_0)$
and in particular we see that the containment $A_0\subset
f^{-1}(f(A_0))$ holds. Thus, $x\in f^{-1}(f(A_0))$.

Now, let us suppose the map $f$ is injective. By our former
argument, we have that $A_0\subset f^{-1}(f(A_0))$ therefore, we
will show the reverse containment. If $y\in f(A_0)$, then
$f(x)=y$ for some $x\in A_0$. By the injectivity of $f$, if
$f(x_0)=y$ for some $x_0\in A$, then we must have that
$x_0=x$. In particular, $x_0\in A_0$. Thus $f^{-1}(f(A_0))\subset
A_0$ and equality $A_0=f^{-1}(f(A_0))$ holds.
\\\\
(b). First, we will show that $f(f^{-1}(B_0))\subset
B_0$. Consider the preimage $f^{-1}(B_0)$ of $B_0$. Let $x\in
f^{-1}(B_0)$. Then $f(x)=y$ for some $y\in B_0$. Since
$f(f^{-1}(B_0))$ is, by definition, the set of all points
$f(x)\in B$ where $x\in f^{-1}(B_0)$ and $f(x)=y$ for $y\in B_0$,
we have that $f(f^{-1}(B_0))\subset B_0$.

Now, let us suppose the map $f$ is surjective. Let $y\in B_0$,
then there exists $x\in A$ such that $f(x)=y$. Thus, $x\in
f^{-1}(B_0)$. Then $y=f(x)\in f(f^{-1}(B_0))$ (in particular
$B_0\subset f(f^{-1}(B_0))$) and we have equality
$B_0=f(f^{-1}(B_0))$.
\end{proof}
\newpage

\begin{problem}[Munkres, \S2, 2(g).]
Let $f\colon A\to B$ and let $A_i\subset A$ and $B_i\subset B$
for $i=0$ and $i=1$. Show that $f^{-1}$ preserves inclusion,
unions, intersections, and differences of sets:
\begin{enumerate}[noitemsep]
\item[(g)] $f(A_0\cap A_1)\subset f(A_0)\cap f(A_1)$; show that
  equality holds if $f$ is injective.
\end{enumerate}
\end{problem}
\begin{proof}[Proof of (g)]
The claim is evident if $A_0$ and $A_1$ are disjoint
subsets. Suppose $A_0\cap A_1\neq\emptyset$. Let $y\in f(A_0\cap
A_1)$. Then $y=f(x)$ for some $x\in A_0$, $x\in A_1$. Then
$f(x)\in f(A_0)$ and $f(x)\in f(A_1)$ so $y\in f(A_0)\cap
f(A_1)$. Thus, $f(A_0\cap A_1)\subset f(A_0)\cap f(A_1)$.

Now, suppose $f$ is injective. Then, if $f(x)=f(x')=y$ for some
$y\in B$, then $x=x'$. Let $y\in f(A_0)\cap f(A_1)$. Then
$y=f(x_0)$, $y=f(x_1)$ for some $x_0\in A_0$, $x_1\in A_1$. But,
by the injectivity of $f$, $x_0=x_1$ so $x_0\in A_0\cap
A_1$. Hence, $y\in f(A_0\cap A_1)$ and the equality $f(A_0\cap
A_1)=f(A_0)\cap f(A_1)$ holds.
\end{proof}
\newpage

\begin{problem}[Munkres, \S13, 3.]
Show that the collection $\mathcal{T}_c$ given in
Example 4 of \S12 is a topology on the set $X$. Is the collection
\[
\mathcal{T}_\infty=
\left\{\,U\;\middle|\;
\text{$X\smallsetminus U$ is infinite or empty or all of $X$}
\,\right\}
\]
a topology on $X$?
\end{problem}
\begin{proof}
Recall that $\mathcal{T}_c$ is the collection of all subsets $U$
of $X$ such that $X\smallsetminus U$ is either countable or is
all of $X$. Let us verify that $\mathcal{T}_c$ defines a topology
on $X$. First, $\emptyset\in\mathcal{T}_c$ since
$X\smallsetminus\emptyset=X$ and $X\in\mathcal{T}_c$ since
$X\smallsetminus X=\emptyset$ is countable. Second, let
$\left\{U_\alpha\right\}$, $\alpha\in A$, be an indexed
family of nonempty elements of $\mathcal{T}_c$, then
$X\smallsetminus U_\alpha$ is countable for all $\alpha$. Thus,
by DeMorgan's laws, we have that
\[
X\smallsetminus\bigcup U_\alpha=\bigcap X\smallsetminus U_\alpha
\]
is countable (this follows from Corollary 7.3, since
$\bigcap_\alpha X\smallsetminus U_\alpha$ is a subset of
$U_\beta$ for all $\beta\in A$, hence it is countable). Thus, the
union $\bigcup U_\alpha$ is in $\mathcal{T}_c$. Lastly, let
$U_1,...,U_n$ be nonempty elements of $\mathcal{T}_c$, then by
DeMorgan's laws, we have that
\[
X\smallsetminus\bigcap_{i=1}^n U_i=\bigcup_{i=1}^n (X\smallsetminus U_i)
\]
is countable by Theorem 7.5 since
$\bigcup_{i=1}^n(X\smallsetminus U_i)$ is a countable union of
countable sets. So the finite intersection
$\bigcap_{i=1}^nU_i\in\mathcal{T}_c$. Therefore, $\mathcal{T}_c$
satisfies all the properties to define a topology on $X$.
\\\\
Now, let us consider the collection of subsets of $X$,
$\mathcal{T}_\infty$, given above. We will show that arbitrary
unions of elements of $\mathcal{T}_\infty$ are, in general, not
in $\mathcal{T}_\infty$. Let $X=\ZZ_+$ and suppose that
$\mathcal{T}_\infty$ defines a topology on $X$. Consider the
collection of subsets
$\left\{\{i\}\right\}_{i=1}^\infty$. $\ZZ_+\smallsetminus\{n\}=\left\{1,...,n-1,n+1,...\right\}$
is infinite hence, $\{i\}\in\mathcal{T}_\infty$ for all
$i\in\{1,...\}$. However,
$\ZZ_+\smallsetminus\bigcup_{i=1}^\infty\{i\}=\{0\}$ is finite so
$\bigcup_{i=1}^\infty\{i\}\notin\mathcal{T}_\infty$, this is a
contradiction. Therefore, $\mathcal{T}_\infty$ does not define a
topology on $X$.
\end{proof}
\newpage

\begin{problem}[Munkres, \S13, 5.]
Show that if $\mathcal{A}$ is a basis for a topology on $X$, then
the topology generated by $\mathcal{A}$ equals the intersection
of all topologies on $X$ that contain $\mathcal{A}$. Prove the
same if $\mathcal{A}$ is a subbasis.
\end{problem}
\begin{proof}
Let $\mathcal{T}$ be the topology generated by $\mathcal{A}$ and
let $\mathcal{S}$ be the collection of all topologies
$\mathcal{T}'$ that contain $\mathcal{A}$. By Lemma 13.3, it
suffices to check that $\mathcal{T}=\bigcap\mathcal{T}'$. First
we will show that the intersection $\bigcap\mathcal{T}'$ indeed
defines a topology on $X$. To that end we shall prove the
following lemma:
\begin{lemma}
Let $X$ be a nonempty set and let $\left\{\mathcal{T}_\alpha\right\}$ be
an indexed collection of topologies on $X$. Then
$\bigcap\mathcal{T}_\alpha$ defines a topology on $X$.
\end{lemma}
\begin{proof}[Proof of Lemma 1]
\renewcommand\qedsymbol{$\vardiamondsuit$}
Let $\mathcal{T}=\bigcap\mathcal{T}_\alpha$. First, since
$\emptyset\in\mathcal{T}_\alpha$ and $X\in\mathcal{T}_\alpha$ for
all $\alpha\in A$, $\emptyset$ and $X$ are in
$\mathcal{T}$. Second, let $\left\{U_\beta\right\}$, $\beta\in
B$, be an indexed family of nonempty elements of
$\mathcal{T}$. Then, $U_\beta\in \mathcal{T}_\alpha$ for all
$\beta\in B$ for all $\alpha\in A$ so $\bigcup
U_\beta\in\mathcal{T}_\alpha$ for all $\alpha\in A$. Hence,
$\bigcup U_\beta\in\mathcal{T}$. Lastly, let $U_1,...,U_n$ be
nonempty elements of $\mathcal{T}$. Then,
$U_1,...,U_n\in\mathcal{T}_\alpha$ for all $\alpha\in A$ so
$\bigcap_{i=1}^n U_i\in\mathcal{T}_\alpha$ for all $\alpha\in A$
thus, $\bigcap_{i=1}^n U_i\in\mathcal{T}$. We see that, indeed,
$\mathcal{T}$ defines a topology on $X$.
\end{proof}
By the Lemma 1 above, it follows that $\bigcap\mathcal{T}'$ gives
a topology on $X$. Now, it is easy to see that
$\bigcap\mathcal{T}'\subset\mathcal{T}$ since
$\mathcal{T}\in\mathcal{S}$ is the coarsest topology containing
$\mathcal{A}$. Let us prove this fact:
\begin{lemma}
Let $X$ be a nonempty set. Let $\mathcal{A}$  be a basis for the
topology $\mathcal{T}$ on $X$. Then $\mathcal{T}$ is the coarsest
topology containing $\mathcal{A}$.
\end{lemma}
\begin{proof}[Proof of Lemma 2]
\renewcommand\qedsymbol{$\vardiamondsuit$}
This can be easily proven by contradiction for suppose
$\mathcal{T}$ is not the coarsest topology containing
$\mathcal{A}$. Let $\mathcal{C}$ be a strictly coarser topology
that contains $\mathcal{A}$. Then there exists some open set
$U\in\mathcal{T}$ not in $\mathcal{C}$. Thus, $\mathcal{C}$ is
not generated by $\mathcal{A}$.
\end{proof}
On the other hand we see by Lemma 13.1 that
$\mathcal{T}\subset\bigcap\mathcal{T}'$ since each
$\mathcal{T}'\in\mathcal{S}$ contains the basis $\mathcal{A}$ of
$\mathcal{T}$, hence contains the open sets of $\mathcal{T}$.
\\\\
T
\end{proof}
\newpage

\begin{problem}[Munkres, \S13, 8(b).]
\begin{enumerate}[noitemsep]
\item[(b)] Show that the collection
\[\mathcal{C}=\left\{\,[a,b)\;\middle|\;\text{$a<b$, $a$ and $b$ rational}\right\}\]
is a basis that generates a topology different from the lower
limit topology on $\RR$.
\end{enumerate}
\end{problem}
\begin{proof}[Proof of (b)]

\end{proof}
\newpage

\begin{problem}[Munkres, \S16, 1.]
Show that if $Y$ is a subspace of $X$, and $A$ is a subset of
$Y$, then the topology $A$ inherits as a subspace of $Y$ is the
same as the topology it inherits as a subspace of $X$.
\end{problem}
\begin{proof}
\end{proof}
\newpage

\begin{problem}[Munkres, \S16, 4.]
A map $f\colon X\to Y$ is said to be an \emph{open map} if for
every open set $U$ of $X$, the set $f(U)$ is open in $Y$. Show
that $\pi_1\colon X\times Y\to X$ and $\pi_2\colon X\times Y\to
Y$ are open maps.
\end{problem}
\begin{proof}
\end{proof}
\newpage

\begin{problem}[Munkres, \S16, 6.]
Show that the countable collection
\[
\left\{\,(a,b)\times(c,d)\;\middle|\;
\text{$a<b$ and $c<d$, and $a,b,c,d$ are rational}\,\right\}
\]
is a basis for $\RR^2$.
\end{problem}
\begin{proof}
\end{proof}
\newpage

\begin{problem}[Munkres, \S16, 9.]
Show that the dictionary order topology on the set $\RR\times\RR$
is the same as the product topology $\RR_d\times\RR$, where
$\RR_d$ denotes $\RR$ in the discrete topology. Compare this
topology with the standard topology on $\RR^2$.
\end{problem}
\begin{proof}
\end{proof}

%%% Local Variables:
%%% mode: latex
%%% TeX-master: "../MA571-HW-Current"
%%% End:
