\begin{problem}[Munkres \S23, Ex.\,3]
Let $\left\{A_\alpha\right\}$ be a collection of connected
subspaces of $X$; let $A$ be a connected subspace of $X$. Show
that if $A\cap A_\alpha\neq\emptyset$ for all $\alpha$, then
$A\cup\left(\bigcup A_\alpha\right)$ is connected.
\end{problem}
\begin{proof}
We shall aim to prove this result by using Theorem 23.3 from
Munkres. Define the collection $\left\{B_\alpha\right\}$ by
setting $B_\alpha=A\cup A_\alpha$. Note that by Theorem 23.3,
$B_\alpha$ is connected for all $\alpha$, since $A\cap
A_\alpha\neq\emptyset$ and both $A$ and $A_\alpha$ are
connected. Next observe that the intersection $B_\alpha\cap
B_\beta\neq\emptyset$ for all $\alpha$ and $\beta$, in
particular, the subspace $A$ is contained in the intersection
since $A\subset B_\alpha$ and $A\subset B_\beta$ for all $\alpha$
and $\beta$. Therefore, $\left\{B_\alpha\right\}$ is a collection
of connected subspaces of $X$ that have a point in
common. Applying Theorem 23.3 one last time, we see that the
union
\[
\bigcup B_\alpha
=
\bigcup\left(A\cup A_\alpha\right)
=
A\cup\left(\bigcup A_\alpha\right)
\]
is connected.
\end{proof}
\newpage
\begin{problem}[Munkres \S23, Ex.\,6]
Let $A\subset X$. Show that if $C$ is a connected subspace of $X$
that intersects both $A$ and $X\setminus A$, then $C$ intersects
$\partial A$.
\end{problem}
\begin{proof}
We shall proceed by contradiction. Suppose that $C\cap\partial
A=\emptyset$, then we shall show that the pair $C\cap A$ and
$C\cap(X\setminus A)$ forms a separation of
$C$. Recall that by definition (see Munkres \S17, p.\,102) the
boundary $\partial A=\clsr A\cap\clsr{X\setminus A}$. Then we
claim that $\clsr A=\partial A\cup \Int A$:
\begin{lemma}
Let $X$ be a topological space and $A\subset X$. Then $\partial
A$ and $\Int{A}$ are disjoint and $\clsr{A}=\partial A\cup\Int
A$.
\end{lemma}
\begin{proof}[Proof of lemma]
\renewcommand\qedsymbol{$\clubsuit$}
The point $x\in\partial A$ if and only if $x\in\clsr A$ and
$x\in\clsr{X\setminus A}$. Thus, for every neighborhood $U$ of
$x$, the intersection $U\cap X\setminus A\neq\emptyset$, in
particular $U\nsubset A$ so $x$ is not an interior point of
$A$. Hence, we see that $\partial A\cap\Int A=\emptyset$. To
prove the last statement note that $\partial A\subset\clsr A$ and
$\Int A\subset A\subset\clsr A$ (cf.\,Munkres \S17, p.\,95), so
that $\partial A\cup\Int A\subset\clsr A$ hence, it suffices to
show the reverse inclusion, namely, $\clsr A\subset\partial
A\cup\Int A$. Let $x\in\clsr A$. If $x\in\Int A$, then clearly
$x\in\partial A\cup\Int A$. Suppose $x\notin\Int A$. Then, by
Theorem 17.5(a), for every neighborhood $U$ of $x$, the
intersection $U\cap A\neq\emptyset$ and $U\nsubset A$. Thus,
$U\cap (X\setminus A)\neq\emptyset$ so $x\in\clsr{X\setminus
  A}$. It follows that $x\in\clsr{A}\cap\clsr{X\setminus
  A}=\partial A$.
\end{proof}
\begin{lemma}
Let $X$ be a topological space and $A\subset X$. Then $\partial
A=\partial(X\setminus A)$.
\end{lemma}
\begin{proof}[Proof of lemma]
\renewcommand\qedsymbol{$\clubsuit$}
Replace $A$ by $X\setminus A$ in the definition of the boundary
of $A$. Then we have:
\begin{align*}
\partial(X\setminus A)&=\clsr{X\setminus A}\cap
                             \clsr{X\setminus(X\setminus A)}\\
&=\clsr{X\setminus A}\cap\clsr{A}\\
&=\clsr A\cap\clsr{X\setminus A}\\
&=\partial A.
\end{align*}
\end{proof}
Now, by Theorem 17.4, we have that $\clsr{C\cap A}=C\cap\clsr A$
and $\clsr{C\cap(X\setminus A)}=C\cap\clsr{X\setminus
  A}$. But by Lemma 13 and Lemma 14, the latter sets are
equivalent to $\clsr{C\cap A}=C\cap(\partial A\cup\Int
  A)$ and $\clsr{C\cap(X\setminus
    A)}=C\cap(\partial A\cup\Int(X\setminus
    A))$. But since $C\cap\partial A=\emptyset$ by
assumption, we have
\begin{align*}
\clsr{C\cap A}\cap(C\cap(X\setminus A))
&=(C\cap(\partial A\cup\Int A))
\cap(C\cap(X\setminus A))\\
&=((C\cap\partial A)
\cup
(C\cap\Int A))
\cap
(C\cap(X\setminus A))\\
&=(C\cap\Int A)\cap(C\cap(X\setminus A))\\
&=\emptyset
\end{align*}
since $C\cap\Int A\subset A$ and $C\cap(X\setminus
  A)\subset X\setminus A$. Similarly, we have that the intersection
$\clsr{C\cap(X\setminus A)}\cap(C\cap
  A)=\emptyset$. So by Lemma 23.1, $C\cap A$ and
$C\cap(X\setminus A)$ form a separation of $C$. This
contradicts the assumption that $C$ is connected. Therefore, we
conclude that $C\cap\partial A\neq\emptyset$.
\end{proof}
\newpage
\begin{problem}[Munkres \S23, Ex.\,7]
Is the space $\RR_\ell$ connected? Justify your answer.
\end{problem}
\begin{proof}
No. The space $\RR_\ell$ is not connected and we may exhibit an
explicit separation. Namely, consider the basis elements
$(-\infty,0)$ and $[0,\infty)$. Then
$\RR=(-\infty,0)\cup[0,\infty)$, hence $(-\infty,0)$ and
$[0,\infty)$ form a separation of $\RR$ with the lower limit
topology.
\\\\
Alternatively, one may note that
$\RR\setminus(-\infty,0)=[0,\infty)$ is open in $\RR_\ell$ so
$(-\infty,0)$ is both open and closed. Hence, by Munkres's
alternative formulation of connectedness (cf.\,Munkres \S23,
p.\,148 the italicized paragraph), $\RR_\ell$ is disconnected.
\end{proof}
\newpage
\begin{problem}[Munkres \S23, Ex.\,9]
Let $A$ be a proper subset of $X$, and let $B$ be a proper subset
of $Y$. If $X$ and $Y$ are connected, show that
\[
(X\times Y)\setminus(A\times B)
\]
is connected.
\end{problem}
\begin{proof}
Consider the family of embeddings $\left\{i_\alpha\right\}$
where $i_\alpha\colon X\hookrightarrow X\times Y$ maps
$x\mapsto x\times y_\alpha$ for $y_\alpha\notin B$, for all
$\alpha$. By Theorem 23.5, $i_\alpha(X)=X\times y_\alpha$ is
connected subspace of $X\times Y$. Moreover $X\times
y_\alpha\subset(X\times Y)\setminus(A\times B)$ so $X\times y_0$,
in particular, we have that is a connected subspace of $(X\times
Y)\setminus(A\times B)$. Similarly, consider the family of
embeddigs $\left\{j_\alpha\right\}$ where $j_\alpha\colon
Y\hookrightarrow X\times Y$ maps $y\mapsto x_\alpha\times y$ for
$x_\alpha\notin A$. We similarly have that
$j_\alpha(Y)=x_\alpha\times Y$ is a connected subspace of
$(X\times Y)\setminus(A\times B)$. Then we claim that
\[
(X\times Y)\setminus(A\times B)
=\bigcup (X\times y_\alpha)\cup(x_\beta\times Y).
\]
It is clear that the union on the right is a subset of $(X\times
Y)\setminus(A\times B)$ since each $X\times y_\alpha$ and
$x_\beta\times Y$ is a subset of $(X\times Y)\setminus(A\times
B)$. To see the reverse containment, take $x\times y$ in the
union $\bigcup (X\times y_\alpha)\cup(x_\beta\times Y)$. Then
$x\times y$ is in some $(X\times y_\alpha)\cup(x_\beta\times Y)$
so $x\times y\in X\times y_\alpha$ or $x\times y\in x_\beta\times
Y$. If $x\times y\in\bigcup X\times y_\alpha$, then
$y_\alpha\notin B$ so $x\times y\notin A\times B$, hence $x\times
y\in(X\times Y)\setminus(A\times B)$. If $x\times y\in\bigcup
x_\beta\times Y$ then $x\notin A$, hence $x\times y\notin
A\times B$ so $x\times y\in(X\times Y)\setminus(A\times
B)$. Thus, we have that $(X\times Y)\setminus(A\times
B)=\bigcup (X\times y_\alpha)\cup(x_\beta\times Y)$. Then, note
that by Theorem 23.3, since $X\cap  y_\alpha\cap x_\beta\cap
Y\neq\emptyset$, in particular, $x_\beta\times y_\alpha$ is in
the intersection, $(X\times y_\alpha)\cup(x_\beta\times Y)$ is
connected for all $\alpha$ and all $\beta$. Thus, the subspace
$(X\times Y)\setminus(A\times B)$ is connected.
\end{proof}
\newpage
\begin{problem}[Munkres \S24, Ex.\,1(ac)]
\begin{enumerate}[noitemsep]
\item[(a)] Show that no two of the spaces $(0,1)$, $(0,1]$ and
  $[0,1]$ are homeomorphic. [\emph{Hint:} What happens if you
  remove a point from each of these spaces?]
\item[(c)] Show $\RR^n$ and $\RR$ are not homeomorphic if $n>1$.
\end{enumerate}
\end{problem}
\begin{proof}
(a) Suppose $\phi\colon (0,1]\to (0,1)$ is a homeomorphism. We
claim that the restriction of $\phi$ to $(0,1)\subset(0,1]$ gives
a homeomorphism to $(0,1)\setminus\{\phi(1)\}$, more generally,
the following result holds:
\begin{lemma}
Suppose $\phi\colon X\to Y$ is a homeomorphism and $U\subset
X$. Then the restriction $\restr{\phi}{U}\colon U\to\phi(U)$ is a
homeomorphism.
\end{lemma}
\begin{proof}[Proof of lemma]
\renewcommand\qedsymbol{$\clubsuit$}
The restriction $\phi_U=\restr{\phi}{U}\colon U\to \phi(U)$ has a
canonical inverse, namely,
$\phi_U^{-1}=\restr{\phi^{-1}}{\phi(U)}\colon\phi(U)\to U$ since
$\phi$ is a bijection. By Theorem 18.2(d,e) both $\phi_U$ and
$\phi_U^{-1}$ are continuous hence,
$U\approx\phi(U)$.
\end{proof}
Now remove $1$ from $(0,1]$. Then, since $\phi(1)$ is bijective,
there exists $y\in(0,1)$ such that $\phi(1)=y$ with $0<y<1$. Then
$(0,1)\setminus\{y\}=(0,y)\cup(y,1)$ is disconnected, but
$(0,1]\setminus\{1\}=(0,1)$ is connected. This contradicts
Theorem 23.5 that the image of $(0,1]$ under a continuous map is
connected. The same argument shows that $(0,1)\not\approx[0,1]$
(in fact, if we allow ourselves results from \S26 and \S27
we have that $[0,1]$ is compact by 27.3 (Heine--Borel), but
$(0,1)$ is not compact, by 26.5 it follows that they are not
homeomorphic).

Similarly, if $[0,1]\approx(0,1]$ via $\phi$ then
$[0,1]\setminus\{0,1\}\approx(0,1]\setminus\{\phi(0),\phi(1)\}$.
\\\\
(b) From Example 4 of \S24, the punctured Euclidean space
$\RR\setminus\{\mathbf{0}\}$ is path-connected, in particular,
connected. But $\RR$ minus a point is disconnected. More
precisely, if $\RR^n\approx\RR$ via $\phi$, by Lemma 15,
$\RR^n\setminus\{0\}\approx\RR\setminus\{\phi(0)\}$, but
$\RR\setminus\{\phi(0)\}$ is disconnected, contradicting Theorem 23.5.
\end{proof}
\newpage
\begin{problem}[Munkres \S24, Ex.\,2]
Let $f\colon S^1\to\RR$ be a continuous map. Show there exists a
point $x$ of $S^1$ such that $f(x)=f(-x)$.
\end{problem}
\begin{proof}
Consider the map $g\colon S^1\to\RR$ given by
$g(x)=f(x)-f(-x)$. This map is continuous by Lemma 9(i) (proved on
Homework 4 which showed that if $f,g$ are continuous real valued maps on a
metric space $X$ then (i) $f+g$ and (ii) $fg$ are continuous;
moreover $S^1$ is naturally a metric space as a subspace of
$\RR^2$ which is how Munkres defines it in Example 5 on
\S24). Fix $x_0\in S^1$ and suppose, without loss of generality,
that $g(x_0)>0$ (for if $g(x_0)=0$ we are done, i.e,
$f(x_0)=f(-x_0)$ and if $g(x_0)<0$ we reverse the direction of
$<$ in the following argument). Then
\[
g(-x_0)=f(-x_0)-f(-(-x_0))=-f(x_0)+f(-x_0)=-g(x_0).
\]
Then $g(-x_0)=-g(x_0)<g(x_0)$ and by the Intermediate Value
Theorem (Theorem 24.3) there exists $y\in S$ such that $g(y)=0$,
i.e, $f(y)=f(-y)$.
\end{proof}
\newpage
\begin{problem}[Munkres \S25, Ex.\,2(b)]
\begin{enumerate}[noitemsep]
\item[(b)] Consider $\RR^\omega$ in the uniform topology. Show
  that $\mathbf{x}$ and $\mathbf{y}$ lie in the same component of
  $\RR^\omega$ if and only if the sequence
  \[
    \mathbf{x}-\mathbf{y}=(x_1-y_1,x_2-y_2,...)
  \]
  is bounded. [\emph{Hint:} It suffices to consider the case
  where $\mathbf{y}=\mathbf{0}$.]
\end{enumerate}
\end{problem}
\begin{proof}
\end{proof}
\newpage
\begin{problem}[Munkres \S25, Ex.\,4]
Let $X$ be locally path connected. Show that every connected open
set in $X$ is path connected.
\end{problem}
\begin{proof}
First we prove the following claim:
\begin{claim*}
If $U$ is an open subset of $X$, then it is locally
path-connected.
\end{claim*}
\begin{proof}[Proof of claim]
\renewcommand\qedsymbol{$\spadesuit$}
Let $x\in U$ and let $V\subset U$ be a neighborhood of $x$ then,
by Lemma 16.2, since $V$ is open in $X$ and $X$ is locally
path-connected, there exists path-connected neighborhood $W$ of
$x$ contained in $V$, hence contained in $U$. Thus, $U$ is
locally path-connected.
\end{proof}
Now, suppose $U$ is a connected open subset of $X$. Then $U$ has
one component. Moreover, by Theorem 25.5, since $U$ is locally
path-connected the components of $U$ and path-components are
equivalent. Thus, $U$ has exactly one path component, i.e, $U$ is
path-connected.
\end{proof}
\newpage
\begin{problem}[Munkres \S25, Ex.\,6]
A space $X$ is said to be \emph{weakly locally path connected at
  $x$} if for every neighborhood $U$ of $x$, there is a connected
subspace of $X$ contained in $U$ that contains a neighborhood of
$x$. Show that if $X$ is weakly locally connected at each of its
points, then $X$ is locally connected. [\emph{Hint:} Show that
components of open sets are open.]
\end{problem}
\begin{proof}
By Theorem 25.4, it suffices to show that for every open set $U$
of $X$, each path component of $U$ is open in $X$. Let $x\in
U$. Then, by Theorem 25.2, $x$ lies in some path component of
$U$, say $C$. Since $X$ is weakly locally path-connected, there
is a connected subspace, say $C_x$, contained in $U$ that
contains a neighborhood $V_x$ of $x$. Then by Theorem 25.2,
$C_x\subset C$. In particular, for every $x\in C$ we have a
neighborhood $V_x$ of $x$ contained in $C$. Thus, $C$ is open in
$X$.
\end{proof}
\newpage
\begin{problem}[A]
Let $X$ be a topological space. The quotient space
$(X\times[0,1])/(X\times 0)$ is called the \emph{cone} of $X$ and
denoted $CX$.
\\\\
Prove that if $X$ is homeomorphic to $Y$ then $CX$ is
homeomorphic to $CY$ (\emph{Hint:} There are maps in both
directions).
\end{problem}
\begin{proof}
Let $\phi\colon X\to Y$ be a homeomorphism and let $p$ and $q$
denote the quotient maps the pairs $(X\times[0,1],CX)$ and
$(Y\times[0,1],CY)$, respectively. Then we get a canonical
homeomorphism $\Phi\colon X\times[0,1]\to Y\times[0,1]$ given by
the map $(x,z)\mapsto (\phi(x),z)$. Note that $\Phi$ is
continuous, by Theorem 18.4, since $\phi$ and $\id_{[0,1]}$ are
continuous and its inverse is given by
$\Phi^{-1}=(\phi^{-1},\id_{[0,1]})$ (which is continuous by
18.4). Now, we claim that the map $\Phi^*\colon CX\to CY$ given
by $[(x,z)]\mapsto[\Phi(x,z)]=[(\phi(x),z)$ defines a
homeomorphism $CX\approx CY$.

First we will prove that $\Phi^*$ is well-defined. Fix an
equivalence class $[(x,z)]$ in $CX$ and choose two
representatives $(x_1,z_1)$ and $(x_2,z_2)$ of $[(x,z)]$ in
$X\times[0,1]$. Then, by the definition of the quotient space
(cf.\,Homework 4, Problem F), $(x_1,z_1)\sim(x_2,z_2)$ if and
only if $(x_1,z_1)=(x_2,z_2)$ or $z_1=z_2=0$, i.e,
$\{(x_1,z_1),(x_2,z_2)\}\subset X\times 0$. In the former case
$\Phi(x_1,z_1)=\Phi(x_2,z_2)=(\phi(x_1),z_1)$ and we see that
\[
\Phi^*([(x_1,z_1)]=[\Phi(x_1,z_1)]=[(\phi(x_1),z_1)]=[\Phi(x_2,z_2)]=\Phi^*([(x_2,z_2)])
\]
and in the latter $\Phi(x_1,0)=(\phi(x_1),0)$ and
$\Phi(x_2,0)=(\phi(x_2),0)$ so $(\phi(x_1),0)\sim(\phi(x_2),0)$, hence
\[
\Phi^*([(x_1,0)]=[\Phi(x_1,0)]=[(\phi(x_1),0)]=[\Phi(x_2,0)]=\Phi^*([(x_2,0)]).
\]
Thus $\Phi$ is well-defined.

Now we will show that $\Phi^*$ is a continuous bijection and with
a continuous inverse. To show bijectivity we construct an
explicit inverse, namely, define $(\Phi^*)^{-1}\colon CY\to CX$
by $[(y,z)]\mapsto[\Phi^{-1}(y,z)]=[\phi^{-1}(x),z]$. The map
$(\Phi^*)^{-1}$ is clearly well-defined (by a similar argument to
showing that $\Phi$ is well-defined) and we have that
\begin{align*}
\Phi^*\circ(\Phi^*)^{-1}([y,z])
&=\Phi^*([\Phi^{-1}(y,z)]
&
(\Phi^*)^{-1}\circ\Phi^*([x,z])
&=(\Phi^*)^{-1}([\Phi(x,z)])
\\
&=[\Phi(\Phi^{-1}(y,z)]
&
&=[\Phi^{-1}(\Phi(x,z))]
\\
&=[(y,z)]
&
&=[(x,z)]\\
&=\id_{CY}
&&=\id_{CX}.
\end{align*}
It is clear that $\Phi^*$ is continuous since, by Theorem Q.2,
$\Phi^*\circ p=q\circ\Phi$ is continuous. Let $U_\sim$ be open in
$CX$. Then $U=p^{-1}(U_\sim)$ is open in $X\times[0,1]$ then
$\Phi(U)$ is open in $Y\times[0,1]$ since $\Phi$ is a
homeomorphism. The same argument applies to showing that
$(\Phi^*)^{-1}$ is continuous in the reverse direction, that is,
consider the composition $(\Phi^*)^{-1}\circ q=p\circ\Phi^{-1}$
and apply Theorem Q.2.
\end{proof}

%%% Local Variables:
%%% mode: latex
%%% TeX-master: "../MA571-HW-Current"
%%% TeX-engine: default
%%% End:
