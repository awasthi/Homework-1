\begin{problem}[Munkres \S23, Ex.\,3]
Let $\left\{A_\alpha\right\}$ be a collection of connected
subspaces of $X$; let $A$ be a connected subspace of $X$. Show
that if $A\cap A_\alpha\neq\emptyset$ for all $\alpha$, then
$A\cup\left(\bigcup A_\alpha\right)$ is connected.
\end{problem}
\begin{proof}
We shall aim to prove this result by using Theorem 23.3 from
Munkres. Define the collection $\left\{B_\alpha\right\}$ by
setting $B_\alpha=A\cup A_\alpha$. Note that by Theorem 23.3,
$B_\alpha$ is connected for all $\alpha$, since $A\cap
A_\alpha\neq\emptyset$ and both $A$ and $A_\alpha$ are
connected. Next observe that the intersection $B_\alpha\cap
B_\beta\neq\emptyset$ for all $\alpha$ and $\beta$, in
particular, the subspace $A$ is contained in the intersection
since $A\subset B_\alpha$ and $A\subset B_\beta$ for all $\alpha$
and $\beta$. Therefore, $\left\{B_\alpha\right\}$ is a collection
of connected subspaces of $X$ that have a point in
common. Applying Theorem 23.3 one last time, we see that the
union
\[
\bigcup B_\alpha
=
\bigcup\left(A\cup A_\alpha\right)
=
A\cup\left(\bigcup A_\alpha\right)
\]
is connected.
\end{proof}
\newpage
\begin{problem}[Munkres \S23, Ex.\,6]
Let $A\subset X$. Show that if $C$ is a connected subspace of $X$
that intersects both $A$ and $X\setminus A$, then $C$ intersects
$\partial A$.
\end{problem}
\begin{proof}
We shall proceed by contradiction. Suppose that $C\cap\partial
A=\emptyset$, then we shall show that the pair $C\cap A$ and
$C\cap(X\setminus A)$ forms a separation of
$C$. Recall that by definition (see Munkres \S17, p.\,102) the
boundary $\partial A=\clsr A\cap\clsr{X\setminus A}$. Then we
claim that $\clsr A=\partial A\cup \Int A$:
\begin{lemma}
Let $X$ be a topological space and $A\subset X$. Then $\partial
A$ and $\Int{A}$ are disjoint and $\clsr{A}=\partial A\cup\Int
A$.
\end{lemma}
\begin{proof}[Proof of lemma]
\renewcommand\qedsymbol{$\clubsuit$}
The point $x\in\partial A$ if and only if $x\in\clsr A$ and
$x\in\clsr{X\setminus A}$. Thus, for every neighborhood $U$ of
$x$, the intersection $U\cap X\setminus A\neq\emptyset$, in
particular $U\nsubset A$ so $x$ is not an interior point of
$A$. Hence, we see that $\partial A\cap\Int A=\emptyset$. To
prove the last statement note that $\partial A\subset\clsr A$ and
$\Int A\subset A\subset\clsr A$ (cf.\,Munkres \S17, p.\,95), so
that $\partial A\cup\Int A\subset\clsr A$ hence, it suffices to
show the reverse inclusion, namely, $\clsr A\subset\partial
A\cup\Int A$. Let $x\in\clsr A$. If $x\in\Int A$, then clearly
$x\in\partial A\cup\Int A$. Suppose $x\notin\Int A$. Then, by
Theorem 17.5(a), for every neighborhood $U$ of $x$, the
intersection $U\cap A\neq\emptyset$ and $U\nsubset A$. Thus,
$U\cap (X\setminus A)\neq\emptyset$ so $x\in\clsr{X\setminus
  A}$. It follows that $x\in\clsr{A}\cap\clsr{X\setminus
  A}=\partial A$.
\end{proof}
\begin{lemma}
Let $X$ be a topological space and $A\subset X$. Then $\partial
A=\partial(X\setminus A)$.
\end{lemma}
\begin{proof}[Proof of lemma]
\renewcommand\qedsymbol{$\clubsuit$}
Replace $A$ by $X\setminus A$ in the definition of the boundary
of $A$. Then we have:
\begin{align*}
\partial(X\setminus A)&=\clsr{X\setminus A}\cap
                             \clsr{X\setminus(X\setminus A)}\\
&=\clsr{X\setminus A}\cap\clsr{A}\\
&=\clsr A\cap\clsr{X\setminus A}\\
&=\partial A.
\end{align*}
\end{proof}
Now, by Theorem 17.4, we have that $\clsr{C\cap A}=C\cap\clsr A$
and $\clsr{C\cap(X\setminus A)}=C\cap\clsr{X\setminus
  A}$. But by Lemma 13 and Lemma 14, the latter sets are
equivalent to $\clsr{C\cap A}=C\cap(\partial A\cup\Int
  A)$ and $\clsr{C\cap(X\setminus
    A)}=C\cap(\partial A\cup\Int(X\setminus
    A))$. But since $C\cap\partial A=\emptyset$ by
assumption, we have
\begin{align*}
\clsr{C\cap A}\cap(C\cap(X\setminus A))
&=(C\cap(\partial A\cup\Int A))
\cap(C\cap(X\setminus A))\\
&=((C\cap\partial A)
\cup
(C\cap\Int A))
\cap
(C\cap(X\setminus A))\\
&=(C\cap\Int A)\cap(C\cap(X\setminus A))\\
&=\emptyset
\end{align*}
since $C\cap\Int A\subset A$ and $C\cap(X\setminus
  A)\subset X\setminus A$. Similarly, we have that the intersection
$\clsr{C\cap(X\setminus A)}\cap(C\cap
  A)=\emptyset$. So by Lemma 23.1, $C\cap A$ and
$C\cap(X\setminus A)$ form a separation of $C$. This
contradicts the assumption that $C$ is connected. Therefore, we
conclude that $C\cap\partial A\neq\emptyset$.
\end{proof}
\newpage
\begin{problem}[Munkres \S23, Ex.\,7]
Is the space $\RR_\ell$ connected? Justify your answer.
\end{problem}
\begin{proof}
No. The space $\RR_\ell$ is not connected and we may exhibit an
explicit separation. Namely, consider the basis elements
$(-\infty,0)$ and $[0,\infty)$. Then
$\RR=(-\infty,0)\cup[0,\infty)$, hence $(-\infty,0)$ and
$[0,\infty)$ form a separation of $\RR$ with the lower limit
topology.
\\\\
Alternatively, one may note that
$\RR\setminus(-\infty,0)=[0,\infty)$ is open in $\RR_\ell$ so
$(-\infty,0)$ is both open and closed. Hence, by Munkres's
alternative formulation of connectedness (cf.\,Munkres \S23,
p.\,148 the italicized paragraph), $\RR_\ell$ is disconnected.
\end{proof}
\newpage
\begin{problem}[Munkres \S23, Ex.\,9]
Let $A$ be a proper subset of $X$, and let $B$ be a proper subset
of $Y$. If $X$ and $Y$ are connected, show that
\[
(X\times Y)\setminus(A\times B)
\]
is connected.
\end{problem}
\begin{proof}
% We shall proceed by contradiction. Let $C$ and $D$ form a
% separation of $(X\times Y)\setminus (A\times B)$. Now, consider
% the embedding $X\hookrightarrow X\times Y$ at a point $y_0\in
% Y\setminus B$, i.e, the map $x\mapsto x\times y_0$. Its image in
% $X\times Y$ is the subspace $X\times y_0$. Note that $X\times
% y_0\subset(X\times Y)\setminus(A\times B)$ since
% \[
% (X\times Y)\setminus(A\times B)=\left\{\,x\times y\in X\times
%  Y\;\middle|\;\text{$x\in X\setminus A$ and $y\in Y\setminus
%  B$}\,\right\},
% \]
% but $y_0\notin B$ so $x\times y_0\notin A\times B$ for all $x\in
% X$. By Problem 2.8 (Munkres \S18, Ex.\,4), the latter map is
% continuous. Moreover, by Theorem 18.2(e), the restriction of its
% codomain to $(X\times Y)\setminus (A\times B)$ yields a continuous
% injection $X\hookrightarrow (X\times Y)\setminus(A\times
% B)$. Then by Theorem 23.5, we have that $X\times y_0$ is a
% connected subspace of $(X\times Y)\setminus(A\times B)$. Thus, by
% Theorem 23.2, $X\times y_0\subset C$ or $X\times y_0\subset D$.
Consider the family of embeddings $\left\{\iota_\alpha\right\}$
where $i_\alpha\colon X\hookrightarrow X\times Y$ maps
$x\mapsto x\times y_\alpha$ for $y_\alpha\notin B$, for all
$\alpha$. By Theorem 23.2, $i_\alpha(X)=X\times y_\alpha$ is
connected subspace of $X\times Y$. Moreover $X\times
y_\alpha\subset(X\times Y)\setminus(A\times B)$ so $X\times y_0$,
in particular, we have that is a connected subspace of $(X\times
Y)\setminus(A\times B)$. Similarly, consider the family of
embeddigs $\left\{j_\alpha\right\}$ where $j_\alpha\colon
Y\hookrightarrow X\times Y$ maps $y\mapsto x_\alpha\times y$ for
$x_\alpha\notin A$. We similarly have that
$j_\alpha(Y)=x_\alpha\times Y$ is a connected subspace of
$(X\times Y)\setminus(A\times B)$. Then we claim that
\[
(X\times Y)\setminus(A\times B)
=\left(\bigcup X\times y_\alpha\right)\cup\left(\bigcup x_\alpha\times Y\right).
\]

\end{proof}
\newpage
\begin{problem}[Munkres \S24, Ex.\,1(ac)]
\begin{enumerate}[noitemsep]
\item[(a)] Show that no two of the spaces $(0,1)$, $(0,1]$ and
  $[0,1]$ are homeomorphic. [\emph{Hint:} What happens if you
  remove a point from each of these spaces?]
\item[(c)] Show $\RR^n$ and $\RR$ are not homeomorphic if $n>1$.
\end{enumerate}
\end{problem}
\begin{proof}
\end{proof}
\newpage
\begin{problem}[Munkres \S24, Ex.\,2]
Let $f\colon S^1\to\RR$ be a continuous map. Show there exists a
point $x$ of $S^1$ such that $f(x)=f(-x)$.
\end{problem}
\begin{proof}
\end{proof}
\newpage
\begin{problem}[Munkres \S25, Ex.\,2(b)]
\begin{enumerate}[noitemsep]
\item[(b)] Consider $\RR^\omega$ in the uniform topology. Show
  that $\mathbf{x}$ and $\mathbf{y}$ lie in the same component of
  $\RR^\omega$ if and only if the sequence
  \[
    \mathbf{x}-\mathbf{y}=(x_1-y_1,x_2-y_2,...)
  \]
  is bounded. [\emph{Hint:} It suffices to consider the case
  where $\mathbf{y}=\mathbf{0}$.]
\end{enumerate}
\end{problem}
\begin{proof}
\end{proof}
\newpage
\begin{problem}[Munkres \S25, Ex.\,4]
Let $X$ be locally path connected. Show that every connected open
set in $X$ is path connected.
\end{problem}
\begin{proof}
\end{proof}
\newpage
\begin{problem}[Munkres \S25, Ex.\,6]
A space $X$ is said to be \emph{weakly locally path connected at
  $x$} if for every neighborhood $U$ of $x$, there is a connected
subspace of $X$ contained in $U$ that contains a neighborhood of
$x$. Show that if $X$ is weakly locally connected at each of its
points, then $X$ is locally connected. [\emph{Hint:} H]
\end{problem}
\begin{proof}
\end{proof}
\newpage
\begin{problem}[A]
Let $X$ be a topological space. The quotient space
$(X\times[0,1])/(X\times 0)$ is called the \emph{cone} of $X$ and
denoted $CX$.
\\\\
Prove that if $X$ is homeomorphic to $Y$ then $CX$ is
homeomorphic to $CY$ (\emph{Hint:} There are maps in both
directions).
\end{problem}
\begin{proof}
\end{proof}
\newpage
\begin{problem}[Extra problem]
Notation: for positive integers $i,n,I,N$, let us write
$(i,n)\gg(I,N)$ if $i>I$ and $n>N$.
\begin{theorem}
A sequence $\left\{\mathbf{x}_n\right\}$ in $\RR^\omega$
converges to $\mathbf{0}$ in the box topology if and only if two
conditions hold:
\begin{enumerate}[noitemsep,label=(\roman*)]
\item for each $k$, $\lim_{n\to\infty} x_n^{(k)}=0$, and
\item there is a pair $(I,N)$ with $x_n^{(k)}=0$ whenever
  $(i,n)\gg(I,N)$.
\end{enumerate}
\end{theorem}
\end{problem}
\begin{proof}
\end{proof}

%%% Local Variables:
%%% mode: latex
%%% TeX-master: "../MA571-HW-Current"
%%% End:
