\begin{problem}[Munkres \S53, Ex.\,7(abcd)]
Let $G$ be a topological group with operation $\cdot$ and identity element
$x_0$. Let $\Omega(G,x_0)$ denote the set of all loops in $G$ based at
$x_0$. If $f,g\in\Omega(G,x_0)$, let us define a loop $f\otimes g$ by the
rule
\[
(f\otimes g)(s)=f(s)\cdot g(s).
\]
\begin{enumerate}[label=(\alph*)]
\item Show that this operation makes the set $\Omega(G,x_0)$ into a group.
\item Show that this operation induces a group operation $\otimes$ on
  $\pi_1(G,x_0)$.
\item Show that the two group operations $*$ and $\otimes$ on
  $\pi_1(G,x_0)$ are the same. [\emph{Hint:} Compute
  $(f*e_{x_0})\otimes(e_{x_0}*g)$.]
\item Show that $\pi_1(G,x_0)$ is Abelian.
\end{enumerate}
\end{problem}
\begin{proof}
For part (a) we need to show that the operation (0) $\otimes$ is associative,
(1) $\Omega(G,x_0)$ is closed under $\otimes$, (2) $\Omega(G,x_0)$ contains
an identity element $e$ and (3) for every $f\in\Omega(G,x_0)$ there
exists an element $\bar f\in\Omega(G,x_0)$ such that $f\otimes\bar f=\bar
f\otimes f=e$. We shall proceed in order: (0) Let
$f,g,h\in\Omega(G,x_0)$. Then $(f\otimes g)\otimes h=f\otimes(g\otimes f)$
since the multiplication $\cdot$ is associative in $G$, i.e., since given
$t\in I$ we have $(f(t)\cdot g(t))\cdot h(t)=f(t)\cdot (g(t)\cdot h(t))$,
in particular this holds for all $\in I$. (1) Let $f$ and $g$ be loops at
$x_0$ then $f\otimes g=f(s)\cdot g(s)$
\end{proof}
\newpage
\begin{problem}[(A)]
Prove Proposition F from the note on the Fundamental Group of the
Circle.
\end{problem}
\begin{proof}
\end{proof}
\newpage
\begin{problem}[(B)]
Prove Lemma G from the note on the Fundamental Group of the Circle. (Hint:
one way to do this is to use the fact, which you don’t have to prove, that
if $\sim$ is the equivalence relation on $[a,a+1]$ which identifies $a$ and
$a+1$ then the restriction of $p$ induces a homeomorphism
$[a,a+1]/{\sim}\to S^1$.)
\end{problem}
\begin{proof}
\end{proof}
\newpage
\begin{problem}[(C)]
Show that for every point $x\in S^n$ the space $S^n-x$ is homeomorphic to
$\RR^n$. You may use the fact, shown in Step 1 of the proof of Theorem
59.3, that $S^n$ with the \emph{north pole} removed is homeomorphic to
$\RR^n$. (Hint: linear algebra.)
\end{problem}
\begin{proof}
\end{proof}
\newpage
\begin{problem}[(D)]
Show that every loop in $S^n$ which is not onto is path-homotopic to a
constant path. (Hint: use Problem C).
\end{problem}
\begin{proof}
\end{proof}
\newpage
\begin{problem}[(E)]
Let $X$ be a topological space and let $A\subset X$ be a deformation
retract. In the space $X/A$, the set $A$ is a point (because it is an
equivalence class). Show that this point is a deformation retract of
$X/A$. (Hint: use p.\,289 \# 9.)
\end{problem}
\begin{proof}
\end{proof}

%%% Local Variables:
%%% mode: latex
%%% TeX-master: "../MA571-HW-Current"
%%% End:
