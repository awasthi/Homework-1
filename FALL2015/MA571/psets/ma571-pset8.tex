\begin{problem}[Munkres \S 46, Ex.\,6]
Show that the compact-open topology, $\mathcal{C}(X,Y)$ is
Hausdorff if $Y$ is Hausdorff, and regular if $Y$ is
regular. [\emph{Hint:} If $\overline U\subset V$, then
$\overline{S(C,U)}\subset S(C,V)$.]
\end{problem}
\begin{proof}
Suppose that $Y$ is Hausdorff. Let $f$ and $g$ be distinct
continuous functions from $X$ to $Y$. Then there exists a point
$x_0\in X$ such that $f(x_0)\neq g(x_0)$. Since $Y$ is Hausdorff
there exists disjoint neighborhoods $U$ and $V$ of $f(x_0)$ and
$g(x_0)$, respectively. Now, we claim that
\begin{claim*}
If $C\subset X$ is finite, $C$ is compact.
\end{claim*}
\begin{proof}
\renewcommand\qedsymbol{$\clubsuit$}
Write $C=\{x_1,...,x_n\}$. Let $\mathcal{A}$ be an open cover of
$C$. Then since $C\subset\bigcup_{U_\alpha\in\mathcal{A}}
U_\alpha$ we can choose $A_i$ containing $x_i$ for every $1\leq
i\leq n$. Thus, the subcollection $\left\{U_i\right\}_{i=1}^n$
covers $C$.
\end{proof}
Let $U'=S\left(\left\{x_0\right\},U\right)$ and
$V'=S\left(\left\{x_0\right\},V\right)$. Note that $U'$ and $V'$
are nonempty since $f\in U'$ and $g\in V'$. Moreover, their
intersection is empty for suppose $h\in U'\cap V'$, then
$h(x_0)\in U\cap V$, but $U\cap V=\emptyset$. Then, since $U'$
and $V'$ are subbasis elements for the compact-open topology on
$\mathcal{C}(X,Y)$ and they ``separate'' $f$ and $g$, it follows
that $\mathcal{C}(X,Y)$ is Hausdorff.

Now, suppose that $Y$ is regular. We shall proceed by the
hint and Lemma 31.1(b). Consider the subbasis element
$S(C,U)$. Since $Y$ is regular, there exists a neighborhood
$V\supset U$ such that $V\supset\overline{U}$. Let
$f\in\overline{S(C,U)}$. Then, we claim that $f\in S(C,V)$. For
suppose not, then there exists an element $x_0\in C$ such that
$f(x_0)\notin V$. Then, since $\overline{U}\subset V$, by
hypothesis, $f(x_0)\notin\overline{U}$. Consider the subbasic
neighborhood $S\left(\left\{x_0\right\},Y-\overline{U}\right)$ of
$f$. Then, $S\left(\left\{x_0\right\},Y-\overline{U}\right)\cap
S(C,U)$ is nonempty. Let $g$ be in the aforementioned
intersection. Then $g(x_0)\in g(C)\subset U$, but $g(x_0)\in
Y-\overline{U}$. This is a contradiction. It follows by Lemma
31.1(b) that $\mathcal{C}(X,Y)$ is regular.
\end{proof}
\newpage
\begin{problem}[Munkres \S 46, Ex.\,7]
Show that if $Y$ is locally compact Hausdorff, then composition
of maps
\[\mathcal{C}(X,Y)\times\mathcal{C}(Y,Z)\longrightarrow\mathcal{C}(X,Z)\]
is continuous, provided the compact-open topology is used
throughout. [\emph{Hint:} If $g\circ f\in S(C,U)$, find $V$ such
that $f(C)\subset V$ and $g\bigl(\overline{V}\bigr)\subset U$.]
\end{problem}
\begin{proof}
Let
$F\colon\mathcal{C}(X,Y)\times\mathcal{C}(Y,Z)\to\mathcal{C}(X,Z)$
given by $(f,g)\mapsto g\circ f$. Suppose $g\circ f\in
S(C,U)$. Then $g(f(C))\subset U$ and since is continuous, we have
that $g^{-1}(U)$ is an open set containing $f(C)$. Thus, by
theorem 29.2, for every $x\in f(C)$ there exists an open
neighborhood $V_x$ of $x$ such that $\overline{V_x}\subset
g^{-1}(U)$ is compact. Then the collection of all such open
neighborhoods, $\left\{V_x\right\}_{x\in f(C)}$, forms an open
cover of $f(C)$. Since $f(C)$ is compact, by Theorem 26.5 since
$C$ is compact and $f$ is continuous, then by Lemma 26.1 there
exists a finite subcollection, say $\left\{V_i\right\}_{i=1}^n$,
that covers $C$. Let $V=\bigcup_{i=1}^n V_i$. We claim that
$\overline{V}\subset U$ and is compact. More generally, we
have
\begin{lemma}[Munkres \S 26, Ex.\,3]
A finite union of compact subspaces of $X$ is compact.
\end{lemma}
\begin{proof}[Proof of lemma]
\renewcommand\qedsymbol{$\clubsuit$}
Suppose $C_1,...,C_n\subset X$ are compact and write
$C=\bigcup_{i=1}^n C_i$. Let
$\mathcal{A}=\left\{U_\alpha\right\}$ be an open cover of
$C$. Then $C_i\subset\bigcup U_\alpha$ so, since $C_i$ is
compact, there exists a finite subcollection
$\mathcal{A}_i=\left\{U_j^i\right\}_{j=1}^{n_i}$ that covers
$C_i$. Then $\mathcal{B}=\bigcup_{i=1}^n\mathcal{A}_i$ is a
finite subcollection of $\mathcal{A}$ that covers $C$, i.e., $C$
is compact.
\end{proof}
By Lemma 16, $\overline{V}$ is compact since, by induction on
Problem 2.2 (Munkres \S 17, Ex.\,6(b)), it is the union of
finitely many compact sets
$\overline{V}=\bigcup_{i=1}^n\overline{V_i}$. Moreover, by Lemma
5 (from HW \# 2\footnote{This states that if $A_\alpha\subset C$
  then $\bigcup   A_\alpha\subset C$.}) we have that
$f(C)\subset V\subset\overline{V}\subset g^{-1}(U)$. At last,
tying these results together, we have
\[
F\left(S(C,V)\times\left(\overline{V},U\right)\right)\subset S(C,U),
\]
since $f'\in S(C,V)$ if $f'(C)\subset V$ and $g'\in
S\left(\overline{V},U\right)$ if
$g'\left(\overline{V}\right)\subset U$ so $g'(f'(C))\subset
g'\left(\overline{V}\right)\subset U$ so $g'\circ f'\in
S(C,U)$. It follows, by Theorem 18.1(4), that $F$ is continuous.
\end{proof}
\newpage
\begin{problem}[Munkres \S 46, Ex.\,8]
Let $\mathcal{C}'(X,Y)$ denote the set $\mathcal{C}(X,Y)$ in some
topology $\mathcal{T}$. Show that if the evaluation map
\[
e\colon X\times\mathcal{C}'(X,Y)\longrightarrow Y
\]
is continuous, then $\mathcal{T}$ contains the compact-open
topology. [\emph{Hint:} The induced map
$E\colon\mathcal{C}'(X,Y)\to\mathcal{C}(X,Y)$ is continuous.]
\end{problem}
\begin{proof}
\end{proof}
\newpage
\begin{problem}[(A)]
\begin{definition}
Definition. If $X$ is a locally compact Hausdorff space then the
space $Y$ given by Theorem 29.1 is called the \emph{one-point
  compactification} of $X$.
\end{definition}

Let $X$ be a compact Hausdorff space and let $W$ be an open
subset of $X$ (so $W$ is locally compact by Corollary 29.3) with
$W\neq X$. Prove that the one-point compactification of $W$ is
homeomorphic to the quotient space $X/(X-W)$.
\end{problem}
\begin{proof}
\end{proof}
\newpage
\begin{problem}[(B)]
Let $X$ be a compact Hausdorff space, let $Y$ be a topological
space, and let $p\colon X\to Y$ be a closed surjective continuous
map. Prove that $Y$ is Hausdorff. [\emph{Hint:} one ingredient in the
proof is p. 171 \# 5.]
\\\\
Note: combining this with HW 4 Problem E and HW 6 Problem A gives
a necessary and sufficient condition for a quotient of a compact
Hausdorff space to be Hausdorff.
\end{problem}
\begin{proof}
\end{proof}
\newpage
\begin{problem}[(C)]
Let $S^2\subset\RR^3$ be the subspace
\[
\left\{\,(x,y,z)\;\middle|\; x^2+y^2+z^2=1\,\right\}.
\]
Prove that $S^2$ is a $2$-manifold. (The definition of
$m$-manifold, where $m$ is a positive whole number, is given at
the top of page 225.)
\end{problem}
\begin{proof}
\end{proof}
\newpage
\begin{problem}[(D)]
Prove that the union of the $x$ and $y$-axes in $\RR^2$ is not a
$1$-manifold.
\end{problem}
\begin{proof}
\end{proof}

%%% Local Variables:
%%% mode: latex
%%% TeX-master: "../MA571-HW-Current"
%%% End:
