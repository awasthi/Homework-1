\begin{problem}[Munkres \S 46, Ex.\,6]
Show that the compact-open topology, $\mathcal{C}(X,Y)$ is
Hausdorff if $Y$ is Hausdorff, and regular if $Y$ is
regular. [\emph{Hint:} If $\over/line U\subset V$, then
$\overline{S(C,U)}\subset S(C,V)$.]
\end{problem}
\begin{proof}
\end{proof}
\newpage
\begin{problem}[Munkres \S 46, Ex.\,7]
Show that if $Y$ is locally compact Hausdorff, then composition
of maps
\[\mathcal{C}(X,Y)\times\mathcal{C}(Y,Z)\longrightarrow\mathcal{C}(X,Z)\]
is continuous, provided the compact-open topology is used
throughout. [\emph{Hint:} If $g\circ f\in S(C,U)$, find $V$ such
that $f(C)\subset V$ and $g\bigl(\overline{V}\bigr)\subset U$.]
\end{problem}
\begin{proof}
\end{proof}
\newpage
\begin{problem}[Munkres \S 46, Ex.\,8]
Let $\mathcal{C}'(X,Y)$ denote the set $\mathcal{C}(X,Y)$ in some
topology $\mathcal{T}$. Show that if the evaluation map
\[
e\colon X\times\mathcal{C}'(X,Y)\longrightarrow Y
\]
is continuous, then $\mathcal{T}$ contains the compact-open
topology. [\emph{Hint:} The induced map
$E\colon\mathcal{C}'(X,Y)\to\mathcal{C}(X,Y)$ is continuous.]
\end{problem}
\begin{proof}
\end{proof}
\newpage
\begin{problem}[(A)]
\begin{definition}
Definition. If $X$ is a locally compact Hausdorff space then the
space $Y$ given by Theorem 29.1 is called the \emph{one-point
  compactification} of $X$.
\end{definition}

Let $X$ be a compact Hausdorff space and let $W$ be an open
subset of $X$ (so $W$ is locally compact by Corollary 29.3) with
$W\neq X$. Prove that the one-point compactification of $W$ is
homeomorphic to the quotient space $X/(X-W)$.
\end{problem}
\begin{proof}
\end{proof}
\newpage
\begin{problem}[(B)]
Let $X$ be a compact Hausdorff space, let $Y$ be a topological
space, and let $p\colon X\to Y$ be a closed surjective continuous
map. Prove that $Y$ is Hausdorff. [\emph{Hint:} one ingredient in the
proof is p. 171 \# 5.]
\\\\
Note: combining this with HW 4 Problem E and HW 6 Problem A gives
a necessary and sufficient condition for a quotient of a compact
Hausdorff space to be Hausdorff.
\end{problem}
\begin{proof}
\end{proof}
\newpage
\begin{problem}[(C)]
Let $S^2\subset\RR^3$ be the subspace
\[
\left\{\,(x,y,z)\;\middle|\; x^2+y^2+z^2=1\,\right\}.
\]
Prove that $S^2$ is a $2$-manifold. (The definition of
$m$-manifold, where $m$ is a positive whole number, is given at
the top of page 225.)
\end{problem}
\begin{proof}
\end{proof}
\newpage
\begin{problem}[(D)]
Prove that the union of the $x$ and $y$-axes in $\RR^2$ is not a
$1$-manifold.
\end{problem}
\begin{proof}
\end{proof}

%%% Local Variables:
%%% mode: latex
%%% TeX-master: "../MA571-HW-Current"
%%% End:
