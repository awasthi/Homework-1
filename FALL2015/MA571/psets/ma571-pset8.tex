\begin{problem}[Munkres \S 46, Ex.\,6]
Show that the compact-open topology, $\mathcal{C}(X,Y)$ is
Hausdorff if $Y$ is Hausdorff, and regular if $Y$ is
regular. [\emph{Hint:} If $\overline U\subset V$, then
$\overline{S(C,U)}\subset S(C,V)$.]
\end{problem}
\begin{proof}
Suppose that $Y$ is Hausdorff. Let $f$ and $g$ be distinct
continuous functions from $X$ to $Y$. Then there exists a point
$x_0\in X$ such that $f(x_0)\neq g(x_0)$. Since $Y$ is Hausdorff
there exists disjoint neighborhoods $U$ and $V$ of $f(x_0)$ and
$g(x_0)$, respectively. Now, we claim that
\begin{claim*}
If $C\subset X$ is finite, $C$ is compact.
\end{claim*}
\begin{proof}
\renewcommand\qedsymbol{$\clubsuit$}
Write $C=\{x_1,...,x_n\}$. Let $\mathcal{A}$ be an open cover of
$C$. Then since $C\subset\bigcup_{U_\alpha\in\mathcal{A}}
U_\alpha$ we can choose $A_i$ containing $x_i$ for every $1\leq
i\leq n$. Thus, the subcollection $\left\{U_i\right\}_{i=1}^n$
covers $C$.
\end{proof}
Let $U'=S\left(\left\{x_0\right\},U\right)$ and
$V'=S\left(\left\{x_0\right\},V\right)$. Note that $U'$ and $V'$
are nonempty since $f\in U'$ and $g\in V'$. Moreover, their
intersection is empty for suppose $h\in U'\cap V'$, then
$h(x_0)\in U\cap V$, but $U\cap V=\emptyset$. Then, since $U'$
and $V'$ are subbasis elements for the compact-open topology on
$\mathcal{C}(X,Y)$ and they ``separate'' $f$ and $g$, it follows
that $\mathcal{C}(X,Y)$ is Hausdorff.

Now, suppose that $Y$ is regular. We shall proceed by the
hint and Lemma 31.1(b). Consider the subbasis element
$S(C,U)$. Since $Y$ is regular, there exists a neighborhood
$V\supset U$ such that $V\supset\overline{U}$. Let
$f\in\overline{S(C,U)}$. Then, we claim that $f\in S(C,V)$. For
suppose not, then there exists an element $x_0\in C$ such that
$f(x_0)\notin V$. Then, since $\overline{U}\subset V$, by
hypothesis, $f(x_0)\notin\overline{U}$. Consider the subbasic
neighborhood $S\left(\left\{x_0\right\},Y-\overline{U}\right)$ of
$f$. Then, $S\left(\left\{x_0\right\},Y-\overline{U}\right)\cap
S(C,U)$ is nonempty. Let $g$ be in the aforementioned
intersection. Then $g(x_0)\in g(C)\subset U$, but $g(x_0)\in
Y-\overline{U}$. This is a contradiction. It follows by Lemma
31.1(b) that $\mathcal{C}(X,Y)$ is regular.
\end{proof}
\newpage
\begin{problem}[Munkres \S 46, Ex.\,7]
Show that if $Y$ is locally compact Hausdorff, then composition
of maps
\[\mathcal{C}(X,Y)\times\mathcal{C}(Y,Z)\longrightarrow\mathcal{C}(X,Z)\]
is continuous, provided the compact-open topology is used
throughout. [\emph{Hint:} If $g\circ f\in S(C,U)$, find $V$ such
that $f(C)\subset V$ and $g\bigl(\overline{V}\bigr)\subset U$.]
\end{problem}
\begin{proof}
Let
$F\colon\mathcal{C}(X,Y)\times\mathcal{C}(Y,Z)\to\mathcal{C}(X,Z)$
given by $(f,g)\mapsto g\circ f$. Suppose $g\circ f\in
S(C,U)$. Then $g(f(C))\subset U$ and since is continuous, we have
that $g^{-1}(U)$ is an open set containing $f(C)$. Thus, by
theorem 29.2, for every $x\in f(C)$ there exists an open
neighborhood $V_x$ of $x$ such that $\overline{V_x}\subset
g^{-1}(U)$ is compact. Then the collection of all such open
neighborhoods, $\left\{V_x\right\}_{x\in f(C)}$, forms an open
cover of $f(C)$. Since $f(C)$ is compact, by Theorem 26.5 since
$C$ is compact and $f$ is continuous, then by Lemma 26.1 there
exists a finite subcollection, say $\left\{V_i\right\}_{i=1}^n$,
that covers $C$. Let $V=\bigcup_{i=1}^n V_i$. We claim that
$\overline{V}\subset U$ and is compact. More generally, we
have
\begin{lemma}[Munkres \S 26, Ex.\,3]
A finite union of compact subspaces of $X$ is compact.
\end{lemma}
\begin{proof}[Proof of lemma]
\renewcommand\qedsymbol{$\clubsuit$}
Suppose $C_1,...,C_n\subset X$ are compact and write
$C=\bigcup_{i=1}^n C_i$. Let
$\mathcal{A}=\left\{U_\alpha\right\}$ be an open cover of
$C$. Then $C_i\subset\bigcup U_\alpha$ so, since $C_i$ is
compact, there exists a finite subcollection
$\mathcal{A}_i=\left\{U_j^i\right\}_{j=1}^{n_i}$ that covers
$C_i$. Then $\mathcal{B}=\bigcup_{i=1}^n\mathcal{A}_i$ is a
finite subcollection of $\mathcal{A}$ that covers $C$, i.e., $C$
is compact.
\end{proof}
By Lemma 16, $\overline{V}$ is compact since, by induction on
Problem 2.2 (Munkres \S 17, Ex.\,6(b)), it is the union of
finitely many compact sets
$\overline{V}=\bigcup_{i=1}^n\overline{V_i}$. Moreover, by Lemma
5 (from HW \# 2\footnote{This states that if $A_\alpha\subset C$
  then $\bigcup   A_\alpha\subset C$.}) we have that
$f(C)\subset V\subset\overline{V}\subset g^{-1}(U)$. At last,
tying these results together, we have
\[
F\left(S(C,V)\times\left(\overline{V},U\right)\right)\subset S(C,U),
\]
since $f'\in S(C,V)$ if $f'(C)\subset V$ and $g'\in
S\left(\overline{V},U\right)$ if
$g'\left(\overline{V}\right)\subset U$ so $g'(f'(C))\subset
g'\left(\overline{V}\right)\subset U$ so $g'\circ f'\in
S(C,U)$. It follows, by Theorem 18.1(4), that $F$ is continuous.
\end{proof}
\newpage
\begin{problem}[Munkres \S 46, Ex.\,8]
Let $\mathcal{C}'(X,Y)$ denote the set $\mathcal{C}(X,Y)$ in some
topology $\mathcal{T}$. Show that if the evaluation map
\[
e\colon X\times\mathcal{C}'(X,Y)\longrightarrow Y
\]
is continuous, then $\mathcal{T}$ contains the compact-open
topology. [\emph{Hint:} The induced map
$E\colon\mathcal{C}'(X,Y)\to\mathcal{C}(X,Y)$ is continuous.]
\end{problem}
\begin{proof}
Suppose that the evaluation map $e\colon
X\times\mathcal{C}'(X,Y)\longrightarrow Y$ is continuous. Then,
by Theorem 46.11 the induced map
$E\colon\mathcal{C}'(X,Y)\to\mathcal{C}(X,Y)$ in
\[
X\times\mathcal{C}'(X,Y)\xrightarrow{\;(\id_X,E)\;}
X\times\mathcal{C}(X,Y)\xrightarrow{\;e'\;}
Y
\]
is continuous. In fact, it is easy to see that the induced map
$E$ is the identity map on $\mathcal{C}(X,Y)$ for
$e(x,f)=f(x)=f'(x)=e'(f',x)=e'(E(f),x)$ for all $x$ so
$f=f'$. Now, let $S(C,U)$ be a subbasic open set in
$\mathcal{C}(X,U)$. Then $E^{-1}(S(C,U))=S(C,U)$ is open in
$\mathcal{C}'(X,Y)$. Thus $\mathcal{T}$ is finer than the
compact-open topology.
\end{proof}
\newpage
\begin{problem}[(A)]
\begin{definition}
Definition. If $X$ is a locally compact Hausdorff space then the
space $Y$ given by Theorem 29.1 is called the \emph{one-point
  compactification} of $X$.
\end{definition}

Let $X$ be a compact Hausdorff space and let $W$ be an open
subset of $X$ (so $W$ is locally compact by Corollary 29.3) with
$W\neq X$. Prove that the one-point compactification of $W$ is
homeomorphic to the quotient space $X/(X-W)$.
\end{problem}
\begin{proof}
Let $W_\infty$ denote the one-point compactification of $W$ and
define the map $p\colon X\to W_\infty$ by
\[
p(x)=\begin{cases}x,&x\in W\\\infty,&x\in X-W.\end{cases}
\]
We claim that $p$ is continuous. It suffices to show that the preimage of a
basic open set in $W_\infty$ is open in $X$. Suppose $U$ is a type 1 open
subset of $W_\infty$, that is, $U$ does not contain the point at
infinity. Then $U\subset W$ so is open in $X$ by Theorem 16.2. Suppose that
$U$ is a type 2 open subset of $W_\infty$. Then $C=W_\infty-U$ is a compact
subset of $W_\infty$. Moreover $C\subset W$ so $C$ is a compact subset of
$X$, that is to say, if $\{U_\alpha\}$ is an open cover of $C$ in $X$, then
$\{U_\alpha\cap W\}$ is an open cover of $C$ in $W$ and since $C$ is
compact in $W$, there exists a finite subcollection $\{U_i\cap W\}_{i=1}^n$
in $Y$ that covers $C$ hence, the collection $\{U_i\}_{i=1}^n$ is a finite
subcollection in $X$ that covers $C$. It follows by Theorem 26.3 that $C$
is closed so $p^{-1}(U)=X-C$ is open in $X$. Thus, $p$ is continuous. By
Theorem Q.3, it follows that the induced map $\bar p\colon X/(X-W)\to
W_\infty$ is continuous. Moreover, $p$ preserves the equivalence relation:
Suppose $x\sim y$ then either $x=y\in W$ or $x,y\in X-W$; in the former we
have $p(x)=x=y=p(y)$; in the latter we have $p(x)=\infty=p(y)$.

By Theorem 26.6, since the quotient $X/(X-W)$ is compact and $W_\infty$ is
Hauusdorff, it suffices to show that $\bar p$ is bijective. It is clear
that $\bar p$ is surjective since $p$ is surjective
($p(X)=p(W\cup(X-W))=p(W)\cup
p(X-W)=W\cup\left\{\infty\right\}=W_\infty$). To see that $\bar p$ is
injective suppose $p([x])=p([y])$. Then $p([x])=\infty$ or
$p([x])\neq\infty$. If $p([x])\neq\infty$, then $p([x])=x=y=p([x])$ or
$p([x])=\infty=p([y])$. In either case, $x\sim y$ so $[x]=[y]$. Thus, $\bar
p$ is bijective. It follows that $\bar p$ is a homeomorphism.
\end{proof}
\newpage
\begin{problem}[(B)]
Let $X$ be a compact Hausdorff space, let $Y$ be a topological
space, and let $p\colon X\to Y$ be a closed surjective continuous
map. Prove that $Y$ is Hausdorff. [\emph{Hint:} one ingredient in the
proof is p. 171 \# 5.]
\\\\
Note: combining this with HW 4 Problem E and HW 6 Problem A gives
a necessary and sufficient condition for a quotient of a compact
Hausdorff space to be Hausdorff.
\end{problem}
\begin{proof}
Let $x$ and $y$ be distinct points in $Y$. Since $p$ is surjective, there
exist $x_0$ and $y_0$ in $X$ such that $p(x_0)=x$ and $p(y_0)=y$. Then,
since $X$ is Hausdorff, by Theorem 17.8, $x_0$ and $y_0$ are closed in $X$
so $x$ and $y$ are closed in $Y$. Then $p^{-1}(x)$ and $p^{-1}(y)$ are
closed since
\[
X-p^{-1}(x)=p^{-1}(Y-x)
\]
which is open in $X$ since $Y-x$ is open in $Y$ and $p$ is
continuous. Moreover, $p^{-1}(x)$ and $p^{-1}(y)$ are clearly disjoint for
otherwise $p(z)=x=y$, but $x\neq y$. Now, by Theorem 32.3, $X$ is normal
since it is a compact Hausdorff space (alternatively we may appeal to
Theorem 26.3 and Munkres \S 26, Ex.\,5 as suggested in the hint) so there
exist disjoint open sets $U$ and $V$ containing $p^{-1}(x)$ and
$p^{-1}(y)$, respectively. Then $X-U$ and $X-V$ are closed so $p(X-U)$ and
$p(X-V)$ are closed in $Y$. Then, we claim $U'=Y-p(X-U)$ and $V'=Y-p(X-V)$
are disjoint neighborhoods of $x$ and $y$, respectively. It is clear that
$U'$ and $V'$ are open, since their complements are closed. Moreover,
$U'\ni x$ and $V'\ni y$ since $Y-U'=p(X-U)$ does not contain $x$ and
$Y-V'=p(X-V)$ does not contain $y$. Lastly, $U'\cap V'=\emptyset$ for
otherwise there is $z\in U'\cap V'$ so $z\notin p(X-U)$ and $z\notin
p(X-V)$ so $z\in Y-(p(X-U)\cup p(X-V))$, but $p(X-U)\cup p(X-V)\supset
p((X-U)\cup p(X-V))=p(X)$ so $z\in\emptyset$, this is a
contradiction. Thus, $Y$ is Hausdorff.
\end{proof}
\newpage
\begin{problem}[(C)]
Let $S^2\subset\RR^3$ be the subspace
\[
\left\{\,(x,y,z)\;\middle|\; x^2+y^2+z^2=1\,\right\}.
\]
Prove that $S^2$ is a $2$-manifold. (The definition of
$m$-manifold, where $m$ is a positive whole number, is given at
the top of page 225.)
\end{problem}
\begin{proof}

\end{proof}
\newpage
\begin{problem}[(D)]
Prove that the union of the $x$ and $y$-axes in $\RR^2$ is not a
$1$-manifold.
\end{problem}
\begin{proof}
Let $X$ denote the union of he $x$ and $y$-axes, that is,
\[
X=\left\{\,(x,0)\,\right\}\cup\left\{\,(0,x)\,\right\}
\]
for $x\in\RR$. Suppose $X$ is a $1$-manifold. Then around every
open subset $U$ of $X$, there exists a homeomorphism
$\phi\colon U\to V$ for some $V$ open in $\RR$. Without loss of
generality, we may assume $V=(a,b)$ for some real numbers
$a<b$. Now, consider the open neighborhood
$U=B((0,0),\epsilon)\cap X=((-\epsilon,\epsilon)\times
0)\cup(0\times(-\epsilon,\epsilon))$. Since $X$ is a
$1$-manifold, there exists a homeomorphism $\phi\colon U\to
(a,b)$. Then, by Lemma A, $\phi(U-0\times 0)\approx
(a,b)-\phi(0\times 0)$. However,
\[
U-0\times 0=((-\epsilon,0)\times 0)\cup((0,\epsilon)\times 0)\cup(0\times(-\epsilon,0))\cup(0\times(0,\epsilon))
\]
is a union of four disjoint open subsets of $X$, therefore, $U$
consists of four connected components. However,
$(a,b)-\phi(0\times 0)=(a,\phi(0\times 0))\cup(\phi(0\times
0),b)$ consists of only two connected components. This is a
contradiction. It follows that $X$ is not a $1$-manifold.
\end{proof}

%%% Local Variables:
%%% mode: latex
%%% TeX-master: "../MA571-HW-Current"
%%% End:
