\begin{problem}
Let $X$ be a Hausdorff space and let $A$ be a compact subset of
$X$. Prove from the definitions that $A$ is closed.
\end{problem}
\begin{proof}
This is Theorem 26.3 from Munkers \S26, p.\,165; we shall paraphrase it.
\\\\
We show that $X-A$ is open. To that end we will show that, given a point
$x_0\in X-A$, there is neighborhood $U$ of $x_0$ disjoint from $A$. For
each point $a\in A$, by the Hausdorff property of $X$, choose disjoint
neighborhoods $U_a$ and $V_a$ of $x_0$ and $a$, respectively. Then the
collection $\left\{\,V_a\;\middle|\;a\in A\,\right\}$ forms an open cover
of $A$ so, by Lemma 26.1, only finitely many of the $V_a$'s cover $A$, say
$V_{a_1},...,V_{a_n}$. Define $U\coloneqq U_{a_1}\cap\cdots\cap
U_{a_n}$. We claim that $U$ is a neighborhood of $x_0$ disjoint from
$A$. First, it is clear that $U$ is a neighborhood of $x_0$ since each
$U_a$ contains $x_0$ and $U$ is an intersection of finitely many of
these. Second, note that if $z\in U\cap A$ then $z\in U_{a_i}$ for all $i$
and $z\in V_{a_j}$ for some $j\in\{1,...,n\}$, but $U_{a_j}\cap
V_{a_j}=\emptyset$. Therefore, $U\cap A=\emptyset$. By Lemma C, it follows
that $X-A$ is open.
\end{proof}
\begin{problem}
Let $X$ be a Hausdorff space and let $A$ and $B$ be disjoint
compact subsets of $X$. Prove that there are open sets $U$ and
$V$ such that $U$ and $V$ are disjoint, $A\subset U$ and
$B\subset V$.
\end{problem}
\begin{proof}
This is Ex.\,5 from Munkres \S26, p.\,171.
\\\\
Suppose $A$ and $B$ are disjoint compact subspaces of $X$. Since $X$ is
Hausdorff, by Theorem 26.4, for every $x\in B$ there exists disjoint open
sets $U_x$ and $V_x$ where $U_x\supset A$ and $V_x$ is a neighborhood of
$x$. Then the collection $\left\{\,V_x\;\middle|\;x\in B\right\}$ is an
open cover of $B$ so by Lemma 26.1, only finitely many of the $V_x$'s cover
$B$, say $V_{x_1},...,V_{x_n}$. Define $U\coloneqq U_{x_1}\cap\cdots\cap
U_{x_n}$ and $V\coloneqq V_{x_1}\cup\cdots\cup V_{x_n}$. We claim that $U$
and $V$ are disjoint neighborhood containing $A$ and $B$, respectively. It
is clear that $U$ and $V$ are open since $U$ is a finite intersection of
open sets and $V$ is a union of open sets and that they contain $A$ and
$B$, respectively, since each of the $U_x$'s contain $A$ and
$V_{x_1},...,V_{x_n}$ is an open cover of $B$. Lastly, $U$ and $V$ are
disjoint since intersection distributes over union, i.e., we have
\[
U\cap V=
\left(\bigcap_{i=1}^nU_{x_i}\right)\cap\left(\bigcup_{j=1}^n
  V_{x_j}\right)=
\bigcup\left(\bigcap_{i=1}^nU_{x_i}\cap V_{x_j}\right)=\emptyset
\]
since $U_{x_i}\cap V_{x_i}=\emptyset$ so
$\left(\bigcap_{i=1}^nU_{x_j}\right)\cap V_{x_i}=\emptyset$.
\end{proof}
\begin{problem}
Prove the Tube Lemma: Let $X$ and $Y$ be topological spaces with
$Y$ compact, let $x_0\in X$, and let $N$ be an open set of
$X\times Y$ containing $x_0\times Y$, then there is an open set
$W$ of $X$ containing $x_0$ with $W\times Y\subset
N$.
\end{problem}
\begin{proof}
This is Lemma 26.8 from Munkres \S26, p.\,168, but is proved in \emph{Step
  1} in the process of showing Theorem 26.7; we paraphrase the proof here.
\\\\
Let $x_0\in X$, and let $N$ be an open set of $X\times Y$ containing
$x_0\times Y$. Cover $x_0\times Y$ by basic open sets $U\times V$ lying in
$N$. Note that $x_0\times Y$ is compact, since it is an imbedding of $Y$
given by the map $y\mapsto (x_0,y)$ from $Y$ into $X\times Y$ therefore, by
Lemma 26.1, only finitely many of the $U\times V$'s, say $U_1\times
V_1,...,U_n\times V_n$, cover $x_0\times Y$. Define $W\coloneqq
U_1\cap\cdots\cap U_n$. We claim that $W$ is a neighborhood of $x_0$ such
that $W\times Y\subset N$. First, it is clear that $W$ is a neighborhood of
$x_0$ since it is the finite intersection of open sets and each $U_i\times
V_i$ intersects $x_0\times Y$ hence contains a point of the form $(x_0,y)$
so $U_i=\pi_1(U_i\times V_i)$ contains $x_0$. Lastly, $W\times Y\subset N$
since $W\times Y\subset\bigcup_{i=1}^n U_i\times V_i$. To see this let
$(x,y)\in W\times Y$ and consider the point $(x_0,y)\in x_0\times Y$. Since
$(x_0,y)$ is in $U_i\times V_i$ for some $i$, we have $y\in V_i$. But $x\in
U_j$ for every $j$ since $x\in W$. Thus $(x,y)\in U_i\times V_i$ as
desired. It follows that, $W$ is a neighborhood of $x_0$ with $W\times
Y\subset N$ as desired.
\end{proof}
\begin{problem}
Show that if $Y$ is compact, then the projection map $X\times
X\to X$ is a closed map.
\end{problem}
\begin{proof}
We shall proceed by the tube lemma, i.e, Theorem 26.8. Let $C$ be a closed
subset of $X\times Y$ then $N=(X\times Y)-C$ is open. Choose $x_0\in
X-\pi_1(C)$. Then $x_0\times Y$ is contained in $N$ so by the tube lemma,
there exists a neighborhood $W$ of $x_0$ such that $W\times
Y\subset N$. In particular, $W\subset X-\pi_1(C)$ otherwise if
$x\in W\cap\pi_1(C)$ then $x\times Y\subset N$ and $(x,y)\in C$
for some $y\in Y$, but $N\cap C=\emptyset$. It follows by Lemma C
that $X-\pi_1(C)$ is open so $\pi_1(C)$ is closed. Since $C$ was
chosen arbitrarily we see that $\pi_1$ is a closed map.
\end{proof}
\begin{problem}
Let $X$ be a compact space and suppose we are given a nested
sequence of subsets $C_1\supset C_2\supset\cdots$ with all $C_i$
closed. Let $U$ be an open set containing $\bigcap C_i$. Prove
that there is an $i_0$ with $C_{i_0}\subset U$.
\end{problem}
\begin{proof}
Consider the family of open sets $U_i\coloneqq X-C_i$. Since $U$
is open $X-U$ is closed so by Theorem 26.2 is compact. We claim
that $U_i$ forms an open cover of $X-U$. To see note that by De
Morgan's laws
\[
\bigcup_{i\in\NN} U_i=\bigcup_{i\in\NN} X-C_i=X-\bigcap_{i\in\NN}
C_i\supset X-U
\]
since $\bigcap_{i\in\NN} C_i\subset U$.
\end{proof}
\begin{problem}
Let $X$ be a compact space, and suppose there is a finite family
of continuous functions $f_i\colon X\to\RR$, $i=1,...,n$ with the
following property: given $x\neq y$ in $X$ there is an $i$ such
that $f_i(x)\neq f_i(y)$. Prove that $X$ is homeomorphic to a
subspace of $\RR^n$.
\end{problem}
\begin{proof}
Consider the map $f\colon X\to\RR^n$ defined by
$f\coloneqq(f_1,...,f_n)$. This map is continuous by Theorem 18.4
since each component $f_i$ is continuous. We claim that $X\approx
f(X)$. To prove this it suffices to show that $f$ is injective so
that its restriction to $f(X)$ will be surjective and lastly
invoke Theorem 26.6. Suppose $f(x)=f(y)$ but $x\neq y$. Then
$f_i(x)\neq f_i(y)$ for some $i$, but this implies that $f(x)\neq
f(y)$. This is a contradiction therefore, $x=y$. It follows that
$f$ is a bijection from a compact space $X$ into
$f(X)\subset\RR^n$ so by Theorem 26.6, we have $X\approx f(X)$.
\end{proof}
\begin{problem}
Let $X$ be a compact metric space and let $\mathcal{U}$ be a
covering of $X$ by open sets. Prove that there is an
$\varepsilon>0$ such that, for each set $S\subset X$ with
diameter $<\varepsilon$, there is a $U\in\mathcal{U}$ with
$S\subset U$. (This fact is known as the ``Lebesgue number lemma.'')
\end{problem}
\begin{proof}
\end{proof}
\begin{problem}
Let $S^1$ denote the circle $\left\{\,x^2+y^2=1\,\right\}$ in
$\RR^2$. Define an equivalence relation on $S^1$ by
\[\text{$(x,y)\sim (x',y')$ $\iff$ $(x,y)=(x',y')$ or $(x,y)=(-x',-y')$}\]
(you do not have to prove that this is an equivalence
relation). Prove that the quotient space $S^1/{\sim}$ is
homeomorphic to $S^1$.
\\\\
One way to do this is by using complex numbers.
\end{problem}
\begin{proof}
\end{proof}
\begin{problem}
Let $X$ be a nonempty compact Hausdorff space and let $f\colon
X\to X$ be a continuous function. Suppose $f$ is $1$-$1$. Prove
that there is a nonempty closed set $A$ with $f(A)=A$. (The
hypothesis that $f$ is $1$-$1$ is not actually needed, but it
makes the proof a little easier.)
\end{problem}
\begin{proof}
\end{proof}
\begin{problem}
Let $\sim$ be the equivalence relation on $\RR^2$ defined by
$(x,y)\sim(x',y')$ if and only if there is a nonzero $t$ with
$(x,y)=(tx',ty')$. Prove that the quotient space $\RR^2/{\sim}$
is compact but not Hausdorff.
\end{problem}
\begin{proof}
\end{proof}
\begin{problem}
Let $X$ be a locally compact Hausdorff space. Explain how to
construct the one-point compactification of $X$ and prove that
the space you construct is really compact (you do not have to
prove anything else for this problem.)
\end{problem}
\begin{proof}
\end{proof}
\begin{problem}
Show that if $\prod_{n=1}^\infty X_n$ is locally compact (and
each $X_n$ is nonempty), then each $X_n$ is locally compact and
$X_n$ is compact for all but finitely many $n$.
\end{problem}
\begin{proof}
\end{proof}
\begin{problem}
Let $X$ be a locally compact Hausdorff space, let $Y$ be any
space, and let the function space $\mathcal{C}(X,Y)$ have the
compact-open topology. Prove that the map
\[
e\colon X\times\mathcal{C}(X,Y)\to Y
\]
define by the equation $e(x,f)=f(x)$ is continuous.
\end{problem}
\begin{proof}
\end{proof}
\begin{problem}
Let $I$ be the unit interval, and let $Y$ be a path-connected
space. Prove that any two maps from $I$ to $Y$ are homotopic.
\end{problem}
\begin{proof}
\end{proof}
\begin{problem}
Let $X$ be a topological space and $f\colon[0,1]\to X$ any
continuous function. Define $\bar f$ by $\bar f(t)=f(1-t)$. Prove
that $f*\bar f$ is path-homotopic to the constant path at $f(0)$.
\end{problem}
\begin{proof}
\end{proof}
\begin{problem}
LEt $X$ be a path-connected topological space and let $x_0,x_1\in
X$. Recall that any path $\alpha$ from $x_0$ to $x_1$  gives an
isomorphism $\hat\alpha$ from $\pi_1(X,x_0)$ to $\pi_1(X,x_1)$
(you do not have to prove this.)
\\\\
Suppose that for every pair of paths $\alpha$ and $\beta$ from
$x_0$ to $x_1$ the isomorphisms $\hat\alpha$ and $\hat\beta$  are
the same. Prove that $\pi_1(X,x_0)$ is Abelian.
\end{problem}
\begin{proof}
\end{proof}

%%% Local Variables:
%%% mode: latex
%%% TeX-master: "../MA571-MID-Current"
%%% End:
