\begin{problem}
Let $X$ be a Hausdorff space and let $A$ be a compact subset of
$X$. Prove from the definitions that $A$ is closed.
\end{problem}
\begin{proof}
This is Theorem 26.3 from Munkres, p.\,165.
\\\\
We shall prove that $X-A$ is open, so that $A$ is closed. Let $x_0\in
X-A$. We show there is a neighborhood of $x_0$ disjoint from $A$. For each
point $a\in A$, let us choose disjoint neighborhoods $U_a$ and $V_a$ of the
points $x_0$ and $a$, respectively (using the Hausdorff condition). The
collection
\[
\left\{\,V_a\;\middle|\;a\in A\,\right\}
\]
is a covering of $Y$ by sets open in $X$; therefore, finitely many of them
$V_{a_1},...,V_{a_n}$ cover $A$. The open set $V=V_{a_1}\cup\cdots\cup
V_{a_n}$ contains $Y$, and is disjoint from the open set
$U=U_{a_1}\cap\cdots\cap U_{a_n}$ formed by taking the intersection of the
corresponding neighbrohoods of $x_0$. For if $z$ a point of $V$, then $z\in
V_{a_i}$ for some $i$, hence $z\notin U_{a_i}$ so $z\notin U$. Then $U$ is
a neighborhood of $x_0$ disjoint from $Y$, as desired.
\end{proof}
\begin{problem}
Let $X$ be a Hausdorff space and let $A$ and $B$ be disjoint
compact subsets of $X$. Prove that there are open sets $U$ and
$V$ such that $U$ and $V$ are disjoint, $A\subset U$ and
$B\subset V$.
\end{problem}
\begin{proof}
Suppose that $A$ and $B$ are disjoint compact subsets of $X$. By Theorem
26.4 for every $x\in B$ there exists disjoint open sets $U_x\supset A$ and
$V_x$ a neighborhood of $x$.
\end{proof}
\begin{problem}
Prove the Tube Lemma: Let $X$ and $Y$ be topological spaces with
$Y$ compact, let $x_0\in X$, and let $N$ be an open set of
$X\times Y$ containing $x_0\times Y$, then there is an open set
$W$ of $X$ containing $x_0$ with $W\times Y\subset
N$.
\end{problem}
\begin{proof}
\end{proof}
\begin{problem}
Show that if $Y$ is compact, then the projection map $X\times
X\to X$ is a closed map.
\end{problem}
\begin{proof}
\end{proof}
\begin{problem}
Let $X$ be a compact space and suppose we are given a nested
sequence of subsets $C_1\supset C_2\supset\cdots$ with all $C_i$
closed. Let $U$ be an open set containing $\bigcap C_i$. Prove
that there is an $i_0$ with $C_{i_0}\subset U$.
\end{problem}
\begin{proof}
\end{proof}
\begin{problem}
Let $X$ be a compact space, and suppose there is a finite family
of continuous functions $f_i\colon X\to\RR$, $i=1,...,n$ with the
following property: given $x\neq y$ in $X$ there is an $i$ such
that $f_i(x)\neq f_i(y)$. Prove that $X$ is homeomorphic to a
subspace of $\RR^n$.
\end{problem}
\begin{proof}
\end{proof}
\begin{problem}
Let $X$ be a compact metric space and let $\mathcal{U}$ be a
covering of $X$ by open sets. Prove that there is an
$\varepsilon>0$ such that, for each set $S\subset X$ with
diameter $<\varepsilon$, there is a $U\in\mathcal{U}$ with
$S\subset U$. (This fact is known as the ``Lebesgue number lemma.'')
\end{problem}
\begin{proof}
\end{proof}
\begin{problem}
Let $S^1$ denote the circle $\left\{\,x^2+y^2=1\,\right\}$ in
$\RR^2$. Define an equivalence relation on $S^1$ by
\[\text{$(x,y)\sim (x',y')$ $\iff$ $(x,y)=(x',y')$ or $(x,y)=(-x',-y')$}\]
(you do not have to prove that this is an equivalence
relation). Prove that the quotient space $S^1/{\sim}$ is
homeomorphic to $S^1$.
\\\\
One way to do this is by using complex numbers.
\end{problem}
\begin{proof}
\end{proof}
\begin{problem}
Let $X$ be a nonempty compact Hausdorff space and let $f\colon
X\to X$ be a continuous function. Suppose $f$ is $1$-$1$. Prove
that there is a nonempty closed set $A$ with $f(A)=A$. (The
hypothesis that $f$ is $1$-$1$ is not actually needed, but it
makes the proof a little easier.)
\end{problem}
\begin{proof}
\end{proof}
\begin{problem}
Let $\sim$ be the equivalence relation on $\RR^2$ defined by
$(x,y)\sim(x',y')$ if and only if there is a nonzero $t$ with
$(x,y)=(tx',ty')$. Prove that the quotient space $\RR^2/{\sim}$
is compact but not Hausdorff.
\end{problem}
\begin{proof}
\end{proof}
\begin{problem}
Let $X$ be a locally compact Hausdorff space. Explain how to
construct the one-point compactification of $X$ and prove that
the space you construct is really compact (you do not have to
prove anything else for this problem.)
\end{problem}
\begin{proof}
\end{proof}
\begin{problem}
Show that if $\prod_{n=1}^\infty X_n$ is locally compact (and
each $X_n$ is nonempty), then each $X_n$ is locally compact and
$X_n$ is compact for all but finitely many $n$.
\end{problem}
\begin{proof}
\end{proof}
\begin{problem}
Let $X$ be a locally compact Hausdorff space, let $Y$ be any
space, and let the function space $\mathcal{C}(X,Y)$ have the
compact-open topology. Prove that the map
\[
e\colon X\times\mathcal{C}(X,Y)\to Y
\]
define by the equation $e(x,f)=f(x)$ is continuous.
\end{problem}
\begin{proof}
\end{proof}
\begin{problem}
Let $I$ be the unit interval, and let $Y$ be a path-connected
space. Prove that any two maps from $I$ to $Y$ are homotopic.
\end{problem}
\begin{proof}
\end{proof}
\begin{problem}
Let $X$ be a topological space and $f\colon[0,1]\to X$ any
continuous function. Define $\bar f$ by $\bar f(t)=f(1-t)$. Prove
that $f*\bar f$ is path-homotopic to the constant path at $f(0)$.
\end{problem}
\begin{proof}
\end{proof}
\begin{problem}
LEt $X$ be a path-connected topological space and let $x_0,x_1\in
X$. Recall that any path $\alpha$ from $x_0$ to $x_1$  gives an
isomorphism $\hat\alpha$ from $\pi_1(X,x_0)$ to $\pi_1(X,x_1)$
(you do not have to prove this.)
\\\\
Suppose that for every pair of paths $\alpha$ and $\beta$ from
$x_0$ to $x_1$ the isomorphisms $\hat\alpha$ and $\hat\beta$  are
the same. Prove that $\pi_1(X,x_0)$ is Abelian.
\end{problem}
\begin{proof}
\end{proof}

%%% Local Variables:
%%% mode: latex
%%% TeX-master: "../MA571-MID-Current"
%%% End:
