\begin{problem}[Munkres \S58, Ex.9(a,b,c)]
We define the \emph{degree} of a continuous map $h\colon S^1\to S^1$ as
follows:
\\\\
Let $b_0$ be the point $(0,1)$ of $S^1$; choose a generator $\gamma$ for
the infinite cyclic group $\pi_1(S^1,b_0)$. If $x_0$ is any point of $S^1$,
choose a path $\alpha$ in $S^1$ from $b_0$ to $x_0$ and define
$\gamma(x_0)\coloneqq\hat\alpha(\gamma)$. Then $\gamma(x_0)$ generates
$\pi_1(S^1,x_0)$. The element $\gamma(x_0)$ is independent of the choice of
the path $\alpha$, since the fundamental group of $S^1$ is Abelian.

Now given $h\colon S^1\to S^1$, choose $x_0\in S^1$ and let
$h(x_0)=x_1$. Consider the homomorphism
\[
h_*\colon\pi_1(S^1,x_0)\longrightarrow\pi_1(S^1,x_1).
\]
Since both groups are infinite cyclic, we have
\begin{equation}
\label{eq:1}
\tag{*}
h_*(\gamma(x_0))=d\cdot\gamma(x_1)
\end{equation}
for some integer $d$, if the group is written additively. The integer $d$
is called the \emph{degree} of $h$ and is denoted by $\deg h$.

The degree of $h$ is independent of the choice of the generator $\gamma$;
choosing the other generator woul merely change the sign of both sides of
(\ref{eq:1}).
\begin{enumerate}
\item[(e)] Show that if $h,k\colon S^1\to S^1$ have the same degree, they
  are homotopic.
\end{enumerate}
\end{problem}
\begin{proof}
\end{proof}
\newpage
\begin{problem}[Munkres \S69, Ex.\,1]
Check the details of Example 1.
\end{problem}
\begin{proof}
The following is the statement of Example 1 as found in the book:
\begin{example*}[1]
Consider the group $P$ of bijections of the set $\{0,1,2\}$ with
itself. For $i=1,2$, define an element $\pi_1$ of $P$ by setting
$\pi_i(i)=i-1$ and $\pi_i(i-1)=i$ and $\pi_i(j)=j$ otherwise. Then $\pi_i$
generates a subgroup $G_i$ of $P$ of order $2$. The group $G_1$and $G_2$
generate $P$, as you can check. But $P$ is not their free product. The
reduced words $(\pi_1,\pi_2,\pi_1)$ and $(\pi_2,\pi_1,\pi_2)$, for
instance, represent the same element of $P$.
\end{example*}
\end{proof}
\newpage
\begin{problem}[Munkres \S69, Ex.\,2(a,b,c)]
Let $G=G_1*G_2$, where $G_1$ and $G_2$ are nontrivial groups.
\begin{enumerate}[label=(\alph*)]
\item Show $G$ is not Abelian.
\item If $x\in G$, define the \emph{length} of $x$ to be the length of the
  unique reduced word in the elements of $G_1$ and $G_2$ that represents
  $x$. Show that if $x$ has even length (at least $2$), then $x$ does not
  have finite order. Show that if $x$ has odd length (at least $3$), then
  $x$ is conjugate to an element of shorter length.
\item Show that the only elements of $G$ that have finite order are the
  elements of $G_1$ and $G_2$ that have finite order, and their
  conjugates.
\end{enumerate}
\end{problem}
\begin{proof}
\end{proof}
\newpage
\begin{problem}[Munkres \S69, Ex.\,3]
Let $G=G_1*G_2$. Given $c\in G$, let $cG_1c^{-1}$ denrote the set of all
elements of the form $cxc^{-1}$, for $x\in G_1$. It is a subgroup of $G$;
show that the intersection with $G_2$ is the identity alone.
\end{problem}
\begin{proof}
\end{proof}
\newpage
\begin{problem}
\end{problem}
\begin{proof}
\end{proof}
\newpage
\begin{problem}
\end{problem}
\begin{proof}
\end{proof}
\newpage
\begin{problem}
\end{problem}
\begin{proof}
\end{proof}
\newpage
\begin{problem}
\end{problem}
\begin{proof}
\end{proof}

%%% Local Variables:
%%% mode: latex
%%% TeX-master: "../MA571-HW-Current"
%%% End:
