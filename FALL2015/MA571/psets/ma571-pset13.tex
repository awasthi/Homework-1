\begin{problem}[Munkres \S68, Ex.\,1]
Check the details of Example 1.
\end{problem}
\begin{proof}
The following is the statement of Example 1 as found in the book:
\begin{example*}[1]
Consider the group $P$ of bijections of the set $\{0,1,2\}$ with
itself. For $i=1,2$, define an element $\pi_1$ of $P$ by setting
$\pi_i(i)=i-1$ and $\pi_i(i-1)=i$ and $\pi_i(j)=j$ otherwise. Then $\pi_i$
generates a subgroup $G_i$ of $P$ of order $2$. The group $G_1$ and $G_2$
generate $P$, as you can check. But $P$ is not their free product. The
reduced words $(\pi_1,\pi_2,\pi_1)$ and $(\pi_2,\pi_1,\pi_2)$, for
instance, represent the same element of $P$.
\end{example*}
We need to check two claims (i) that $G_1$ and $G_2$, as defined above,
generate $P$ and (ii) that $P\neq G_1*G_2$, i.e., show that
$(\pi_1,\pi_2,\pi_1)=(\pi_2,\pi_1,\pi_2)$. Let us deal with (i) first. We
show that $\langle G_1,G_2 \rangle=P$. Our strategy is the following, by
the pigeon-hole principle, it suffices to show that $\langle  G_1,G_2
\rangle\subset P$ and that $|\langle G_1,G_2\rangle|=|P|$. Since
$G_1,G_2<P$, i.e., $G_1$ and $G_2$ are subgroups of $P$, the group
generated by $G_1$ and $G_2$ will be a subgroup of $P$ hence, $\langle
G_1,G_2 \rangle\subset P$. The group $P$ is a well-known group, namely (up
to group isomorphism) $S_3$, and we shall not waste time any time showing
that $|P|=|\{0,1,2\}|=3!=6$, but instead we proceed to showing that
$|\langle G_1,G_2 \rangle|=6$. From the definitions of $G_1$ and $G_2$, we
have at least $3$ in $\langle  G_1,G_2 \rangle$, these are the elements
$1$, $\pi_1$ and $\pi_2$ (the latter two have order $2$, e.g.,
\[
\pi_i^2(j)=
\pi_i\left(
\begin{cases}
i-1&\text{if $j=i$}\\
i&\text{if $j=i-1$}\\
j&\text{otherwise}
\end{cases}
\right)
=
\begin{cases}
i&\text{if $j=i$}\\
i-1&\text{if $j=i-1$}\\
j&\text{otherwise}
\end{cases}
\]
which is the identity on $\{0,1,2\}$.) So the elements
$1,\pi_1,\pi_2,\pi_1\pi_2,\pi_2\pi_1,\pi_1\pi_2\pi_1\in\langle
G_1,G_2\rangle$ and all finite strings $\pi_1\pi_2\cdots\pi_i$,
$\pi_2\pi_1\cdots\pi_i$ for that matter. But as a consequence of Lagrange's
theorem, the size of $\langle G_1,G_2\rangle$ must not exceed the size of
$P$ so that we are done when we show that the elements $\pi_1\pi_2$,
$\pi_2\pi_1$ and $\pi_1\pi_2\pi_1$ are distinct elements. First, observe that
\begin{align*}
\pi_2\pi_1(j)&=
\pi_2\left(
\begin{cases}
1&\text{if $j=0$}\\
0&\text{if $j=1$}\\
2&\text{if $j=2$}
\end{cases}
\right)
&
\pi_1\pi_2(j)&=
\pi_1\left(
\begin{cases}
0&\text{if $j=0$}\\
2&\text{if $j=1$}\\
1&\text{if $j=2$}
\end{cases}
\right)
\\
&=
\begin{cases}
2&\text{if $j=0$}\\
0&\text{if $j=1$}\\
1&\text{if $j=2$}
\end{cases}
&&=
\begin{cases}
1&\text{if $j=0$}\\
2&\text{if $j=1$}\\
0&\text{if $j=2$}
\end{cases}
\end{align*}
and, using the computations above,
\[
\pi_1\pi_2\pi_1(j)=
\pi_1\left(
\begin{cases}
2&\text{if $j=0$}\\
0&\text{if $j=1$}\\
1&\text{if $j=2$}
\end{cases}
\right)\\
=
\begin{cases}
2&\text{if $j=0$}\\
1&\text{if $j=1$}\\
0&\text{if $j=2$}
\end{cases}.
\]
Note that none of these elements are equivalent to any of $1$, $\pi_1$ or
$\pi_2$ and are certainly not equal to each other. Moreover, there are six
of these elements and there are no more elements in $P$ since
$|P|=6$. Thus, $\langle G_1,G_2 \rangle=P$.

Lastly, we show that $P\neq G_1*G_2$ since
\[
(\pi_1,\pi_2,\pi_1)=
\pi_1\pi_2\pi_1(j)=
\begin{cases}
2&\text{if $j=0$}\\
1&\text{if $j=1$}\\
0&\text{if $j=2$}
\end{cases}
\]
and
\[
(\pi_2,\pi_1,\pi_2)=
\pi_2\pi_1\pi_2(j)=
\pi_1\left(
\begin{cases}
1&\text{if $j=0$}\\
2&\text{if $j=1$}\\
0&\text{if $j=2$}
\end{cases}
\right)
=
\begin{cases}
2&\text{if $j=0$}\\
1&\text{if $j=1$}\\
0&\text{if $j=2$}
\end{cases}
\]
would imply that $(\pi_1,\pi_2,\pi_1)=(\pi_2,\pi_1,\pi_2)$ in the free
product $G_1*G_2$, but $\pi_1\neq\pi_2$.
\end{proof}
\newpage
\begin{problem}[Munkres \S68, Ex.\,2(a,b,c)]
Let $G=G_1*G_2$, where $G_1$ and $G_2$ are nontrivial groups.
\begin{enumerate}[label=(\alph*)]
\item Show $G$ is not Abelian.
\item If $x\in G$, define the \emph{length} of $x$ to be the length of the
  unique reduced word in the elements of $G_1$ and $G_2$ that represents
  $x$. Show that if $x$ has even length (at least $2$), then $x$ does not
  have finite order. Show that if $x$ has odd length (at least $3$), then
  $x$ is conjugate to an element of shorter length.
\item Show that the only elements of $G$ that have finite order are the
  elements of $G_1$ and $G_2$ that have finite order, and their
  conjugates.
\end{enumerate}
\end{problem}
\begin{proof}
(i) Suppose $G$ is Abelian. Take an element $x\in G_1$ and $y\in G_2$. Then
$(x,y)=(y,x)$. By the definition of a free product (Munkres \S68,
pp.\,413-414) this implies that the word $(x^{-1},y^{-1},x,y)=1$ which
implies that $y^{-1}x=1$, but $y^{-1}\notin G_1$.
\\\\
(ii) Let $x\in G$ be a word of even length. Then $x=(y_1,y_2,...,y_{2k})$
for $k\in\NN$ where the right hand-side is irreducible, i.e., either
$y_i\in G_1$ if $2\mid i$ and $y_j\in G_2$ if $2\nmid j$ or vice-versa
since two consecutive ``letters'' in a word must be from distinct groups or
else we can reduce the word further. Then
$x^2=(y_1,y_2,...,y_{2k},y_1,y_2,...,y_{2k})$ is again irreducible since
$y_{2k}\in G_1$ and $y_1\in G_2$ or vice-versa. It follows by induction
that $x^n\neq 1$ for any finite positive integer $n$.

Now, suppose that $x\in G$ has odd length. Then $x=(y_1,y_2,...,y_{2k+1})$
for $k\in\NN$ where the right hand-side is irreducible. Without loss of
generality, we may assume that $y_1,y_{2k+1}\in G_1$. Then, setting
$y_{2k+1}'\coloneqq y_{2k+1}y_1$, we have
\[
y_1^{-1}xy_1=y_1^{-1}(y_1,y_2,...,y_{2k+1})y_1=(y_2,y_3,...,y_{2k+1}y_1)=(y_2,y_3,...,y_{2k+1}')
\]
which has length $2k$. Thus, $x$ is conjugate to a word of shorter
length.
\\\\
(iii) Suppose that $x\in G$ has finite order. By part (i) the length of $x$
cannot be even. Moreover, if $x$ is of finite order, i.e., if $x^n=1$ for
some positive integer $n$, and $y$ is conjugate to $x$, i.e., there exist
$g\in G$ such that $y=g^{-1}xg$, then
\[
y^n=(g^{-1}xg)^n=(g^{-1}xg)(g^{-1}xg)\cdots(g^{-1}xg)=g^{-1}x^ng=1
\]
so $y$ is of finite order. It remains to show that if $x$ has finite order
then $x$ is a conjugate of an element $y$ of $G_i$, where $i=1,2$. Let
$2k+1$ be the length of $x$. By part (ii), $x$ is conjugate to an element
$y'$ of shorter length. Since $x$ has finite order $y$ has finite order so
by part (i) $y'$ must be of odd length. If $y'$ is of length $1$ we are
done. If not, then $y'$ is conjugate to a word $y''$ of shorter
length with finite order. Since the length of $x$ is finite, this process
must terminate at a word $y$ of length $1$ with finite order.
\end{proof}
\newpage
\begin{problem}[Munkres \S68, Ex.\,3]
Let $G=G_1*G_2$. Given $c\in G$, let $cG_1c^{-1}$ denote the set of all
elements of the form $cxc^{-1}$, for $x\in G_1$. It is a subgroup of $G$;
show that the intersection with $G_2$ is the identity alone.
\end{problem}
\begin{proof}
Let $x\in cG_1c^{-1}\cap G_2$. Then $x\in G_2$ and $x=cyc^{-1}$ for some
$y\in G_1$ or $c=xcy^{-1}$. Now, we break up $c$ into the following cases:
$c=y_1\cdots y_k$ where $y_1\in G_1$ and $y_k\in G_2$, $y_1\in G_2$ and
$y_k\in G_1$ or $y_1,y_k\in G_i$, where we assume, of course, that $c$ is
reduced. In the first case we have $c=y_1\cdots y_k=xy_1\cdots y_ky^{-1}$
which implies that
\[
1=(y_k^{-1}\cdots y_1^{-1})(xy_1\cdots y_ky^{-1})=
y_k^{-1}\cdots y_1^{-1}xy_1\cdots y_ky^{-1}
\]
this implies that $y_1^{-1}x=1$ or $y_1^{-1}xy_1=1$. If $y_1^{-1}x=1$, then
$x\in G_1$ which is a contradiction. Thus, $y^{-1}xy_1=1$ which implies
that $x=1$. For the second case, $c=y_1\cdots y_k$ and $xy_1\cdots
y_ky^{-1}$ so by the uniqueness of representation $xy_1=y_1$ and
$y_ky^{-1}=y_k$ so $x=y=1$. For the last case we may suppose that
$y_1,y_k\in G_1$. Then, again by uniqueness of representation, $c=y_1\cdots
y_k=xy_1\cdots y_ky^{-1}$, but
\end{proof}
\newpage
\begin{problem}[A]
\begin{enumerate}[label=(\roman*)]
\item Do the case of p.\,367 \# 9(e) where $h$ and $k$ take $b_0$ to
  $b_0$. (The proof is similar to the proof of Lemma 55.3, (3) $\implies$
  (1), that I gave in class).
\item Let $G$ be a path-connected topological group and let $a\in G$. Prove
  that the map $\varphi\colon G\to G$ defined by $\varphi(g)\coloneqq ag$
  is homotopic to the identity map.
\item Use part (ii) to complete the proof of p.\,367 \# 9(e).
\end{enumerate}
\end{problem}
\begin{proof}
(i)
\\\\
(ii)
\\\\
(iii) Recall the statement of Ex.\,9 on p.\,367: Show that if $h,k\colon
S^1\to S^1$ have the same degree, they are homotopic.
\end{proof}
\newpage
\begin{problem}[B]
Let $q\colon S^2\to P^2$ be the quotient map, where $P^2$ is the
projective plane. Let $x_0=q(1,0,0)$ and let
\[f(s)=q(\cos(\pi s),\sin(\pi s),0)\]
for $0\leq s\leq 1$. Then $f\colon I\to P^2$ is a loop at
$x_0$. Prove that $[f]*[f]=[e_{x_0}]$.
\end{problem}
\begin{proof}
\end{proof}
\newpage
\begin{problem}[C]
Let $Y$ be the following subset of $\RR^2$: $Y=\left\{\,(s,t)\in
  I\times I\;\middle|\;\text{$s\in\{0,1\}$ or
    $t\in\{0,1\}$}\,\right\}$ (that is, $Y$ is the boundary of
the square $I\times I$). Give $Y$ the equivalence relation $\sim$
that identifies the top and the bottom edges and the left and the
right edges: specifically, $\sim$ is the equivalence relation
associated to the partition of $Y$ into the following sets:
\begin{itemize}
\item for each $s\notin\{0,1\}$, the set $\{(s,0),(s,1)\}$,
\item for each $t\notin\{0,1\}$, the set $\{(t,0),(t,1)\}$,
\item the set $\{0,1\}\times\{0,1\}$.
\end{itemize}
Prove that $Y/\sim$ is a wedge of two circles.
\end{problem}
\begin{proof}
\end{proof}
\newpage
\begin{problem}[Optional Problem]
Let $B^2$ denote the unit disk
$\left\{\,(x,y)\in\RR^2\;\middle|\;\,x^2+y^2\leq 1\right\}$ and
let $S^1$ denote the unit circle. Let $\mathbf{a}\in B^2-S^1$. In
this problem we will show that there is a homeomorphism $h\colon
B^2\to B^2$a which takes $(0,0)$ to $\mathbf{a}$ and fixes $S^1$.
\begin{enumerate}[label=(\roman*)]
\item Let $h\colon B^2\to B^2$ be the function defined as
  follows: note that every point in $B^2$ is of the form
\end{enumerate}
\end{problem}
\begin{proof}
\end{proof}
% \newpage
% \begin{problem}
% \end{problem}
% \begin{proof}
% \end{proof}

%%% Local Variables:
%%% mode: latex
%%% TeX-master: "../MA571-HW-Current"
%%% End:
