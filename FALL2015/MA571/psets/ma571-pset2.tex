\begin{problem}[Munkres \S17, p.\,100, 2]
Show that if $A$ is closed in $Y$ and $Y$ is closed in $X$, then
$A$ is closed in $X$.
\end{problem}
\begin{proof}
Let $C$ denote the closure of $A$ in $X$ then, by Theorem 17.4,
$A=\bar A=C\cap Y$ is the closure of $A$ in $Y$. Thus, $A$ is
closed in $X$ since it is the intersection of two closed subsets
of $X$.
\end{proof}
\newpage
\begin{problem}[Munkres \S17, p.\,100, 3]
Show that if $A$ is closed in $X$ and $B$ is closed in $Y$, then
$A\times B$ is closed in $X\times Y$.
\end{problem}
\begin{proof}
Before proceeding we will prove the following set theoretic
result (which was adapted from Exercises 2(n) and 2(o) from \S1,
p.\,14 of Munkres):

\begin{lemma}
For sets $A$, $B$, $C$ and $D$ we the following equalities hold:
\begin{enumerate}[noitemsep,label=(\alph*)]
\item $(A\times B)\cap (C\times D)=(A\cap C)\times (B\cap D)$.
\item $A\times (B\setminus C)=(A\times B)\setminus (A\times C)$.
\item $(A\setminus C)\times B=(A\times B)\setminus (C\times B)$.
\end{enumerate}
that is, the Cartesian product distributes over taking
complements.
\end{lemma}
\begin{proof}[Proof of Lemma 4]
\renewcommand\qedsymbol{$\vardiamondsuit$}
(a) The equality follows (rather straightforwardly) from the
definition of the Cartesian product and the complement of a set
for $x\times y\in (A\times B)\cap (C\times D)$ if and only if
$x\times y\in A\times B$ and $x\times y\in C\times D$ if and only
if $x\in A$ and $x\in C$ and $y\in B$ and $y\in D$ if and only if
$x\in A\cap C$ and $y\in B\cap D$ if and only if $x\times y\in
(A\cap C)\times (B\cap D)$.
\\\\
(b) The point $x\times y\in A\times (B\setminus C)$ if and only
if $x\in A$ and $y\in B\setminus C$ if and only if $x\in A$ and
$y\in B$ and $y\notin C$ if and only if $x\times y\in A\times B$
and $x\times y\notin A\times C$ if and only if $x\times y\in
(A\times B)\setminus (A\times C)$.
\\\\
(c) The very same argument as part (b) can be used, taking $B$ to
be a subset of $A$ and replacing (where appropriate) $A$ by
$A\setminus B$ and $B\setminus C$ by $C$, to prove that
\[
(A\setminus B)\times C=(A\times C)\setminus (B\times C).\qedhere
\]
\end{proof}
Now, since $A$ is closed in $X$ and $B$ is closed in $Y$, their
complements, $X\setminus A$ and $Y\setminus B$ are, by
definition, open in $X$ and $Y$, respectively. Then, the sets
\[
(X\setminus A)\times Y
\quad
\text{and}
\quad
X\times (Y\setminus B)
\]
are open since they are basic open sets in the product topology
on $X\times Y$. So, applying Lemma 4(b) and (c), their
complements
\[
(X\times Y)\setminus (X\setminus A)\times Y=A\times Y
\quad
\text{and}
\quad
(X\times Y)\setminus X\times (Y\setminus B)=X\times B
\]
are closed in $X\times Y$. At last, we have that
\[
(A\times Y)\cap (X\times B)
\]
is the intersection of closed sets, hence, by Theorem 17.1(b), is
closed. By Lemma 4(a),
\[
(A\times Y)\cap (X\times B)=(A\cap X)\times (Y\cap B)=A\times B
\]
so $A\times B$ is closed in $X\times Y$.
\end{proof}
\newpage
\begin{problem}[Munkres \S17, p.\,101, 6(b)]
Let $A$, $B$ and $A_\alpha$ denote subsets of a space $X$. Prove
the following:
\begin{enumerate}[noitemsep]
\item[(b)] $\clsr{A\cup B}=\bar A\cup\bar B$.
\end{enumerate}
\end{problem}
\begin{proof}
By definition, the closure of a set is the intersection of all
closed sets which contain it therefore, $\overline{A\cup
  B}\subset\bar A\cup\bar B$ since $\bar A\cup\bar B$ is a
closed set, by Theorem 17.1(a), which contains $A\cup B$. To see
the reverse containment note that $\bar A\subset\clsr{A\cup B}$
since $\clsr{A\cup B}$ is a closed set which contains
$A$. Similarly $\bar B\subset\clsr{A\cup B}$ so $\bar A\cup \bar
B\subset\clsr{A\cup B}$. Therefore, $\clsr{A\cup B}=\bar
A\cup\bar B$ holds.

Naturally this results extends, by induction, to the case of
finite unions of sets.
\end{proof}

\newpage
\begin{problem}[Munkres \S17, p.\,101, 6(c)]
Let $A$, $B$ and $A_\alpha$ denote subsets of a space $X$. Prove
the following:
\begin{enumerate}[noitemsep]
\item[(b)] $\clsr{\bigcup A_\alpha}\supset\bigcup\clsr{A_\alpha}$.
\end{enumerate}
\end{problem}
\begin{proof}
Let $C$ denote the set $\clsr{\bigcup A_\alpha}$. It is clear, by
the definition of the closure of a set, that $\bar
A_\alpha\subset C$ for every $\alpha$ since $C$ is a closed set
which contains $A_\alpha$, so $\bigcup_\alpha \bar
A_\alpha\subset C$.

The reverse is not true in general; in fact, as Theorem 17.1(3)
suggests, an arbitrary union of closed sets is not even
necessarily closed. For a concrete example consider the family
$A_r=\{r\}$ for $r\in\QQ$. The closure of a point $r$ in $\RR$ is
itself since its complement, $\RR\setminus\{r\}$, is the
union of the open intervals $(-\infty,r)$ and $(r,\infty)$; in
particular, $\{r\}$ is the ``smallest'' closed set containing
$\{r\}$. Hence, we see that the union
\[
\bigcup_{r\in\QQ}\bar A_r=\QQ,
\]
but, by Example 6, $\bar\QQ=\RR$.
\end{proof}
\newpage
\begin{problem}[Munkres \S17, p.\,101, 7]
Criticize the following ``proof'' that $\clsr{\bigcup
  A_\alpha}\subset\bigcup\bar A_\alpha$: if
$\left\{A_\alpha\right\}$ is a collection of sets in $X$ and if
$x\in\clsr{\bigcup A_\alpha}$, then every neighborhood $U$ of $x$
intersects $\bigcup A_\alpha$. Thus $U$ must intersect some
$A_\alpha$, so $x$ must belong to the closure of some
$A_\alpha$. Therefore, $x\in\bigcup\bar A_\alpha$.
\end{problem}
\begin{proof}[Critique]
The main argument, that ``$x$ must belong to the closure of some
$A_\alpha$'', is what is wrong here. The point $x$ may belong to
the closure of multiple $A_\alpha$'s, in fact uncountably many of
them, so that one would have to prove that if $x$ belongs the
closure of some family $A_\beta$ of set, then $x$ must belong to
the union of their closures. This takes us right back to what we
are trying to prove.
\end{proof}
\newpage
\begin{problem}[Munkres \S17, p.\,101, 9]
Let $A\subset X$ and $B\subset Y$. Show that in the space
$X\times Y$,
\[
\clsr{A\times B}=\bar A\times\bar B.
\]
\end{problem}
\begin{proof}
\end{proof}
\newpage
\begin{problem}[Munkres \S17, p.\,101, 10]
Show that every order topology is Hausdorff.
\end{problem}
\begin{proof}
\end{proof}
\newpage
\begin{problem}[Munkres \S17, p.\,101, 13]
Show that $X$ is Hausdorff if and only if the \emph{diagonal}
$\Delta=\left\{\,x\times x\;\middle|\;x\in X\,\right\}$ is closed
in $X\times X$.
\end{problem}
\begin{proof}
\end{proof}
\newpage
\begin{problem}[Munkres \S18, p.\,111, 4]
Given $x_0\in X$ and $y_0\in Y$, show that the maps $f\colon X\to
X\times Y$ and $g\colon Y\to X\times Y$ defined by
\[
f(x)=x\times y_0\quad\text{and}\quad g(y)=x_0\times y
\]
are imbeddings.
\end{problem}
\begin{proof}
\end{proof}
\newpage
\begin{problem}[Munkres \S18, p.\,111-112, 8(a,b)]
Let $Y$ be an ordered set in the order topology. Let
$f,g\colon X\to Y$ be continuous.
\begin{enumerate}[noitemsep,label=(\alph*)]
\item Show that the set
  $\left\{\,x\;\middle|\;f(x)\leq g(x)\,\right\}$ is closed in $X$.
\item Let $h\colon X\to Y$ be the
  function \[h(x)=\min\{f(x),g(x)\}.\] Show that $h$ is
  continuous. [\emph{Hint:} Use the pasting lemma.]
\end{enumerate}
\end{problem}
\begin{proof}
\end{proof}
\newpage
\begin{problem}
Given: $X$ is a topological space with open sets $U_1,...,U_n$
such that $\bar U_i=X$ for all $i$. Prove that the closure of
$U_1\cap\cdots\cap U_n$ is $X$.
\end{problem}
\begin{proof}
\end{proof}

%%% Local Variables:
%%% mode: latex
%%% TeX-master: "../MA571-HW-Current"
%%% End:
