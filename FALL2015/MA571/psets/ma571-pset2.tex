\begin{problem}[Munkres \S17, p.\,100, 2]
Show that if $A$ is closed in $Y$ and $Y$ is closed in $X$, then
$A$ is closed in $X$.
\end{problem}
\begin{proof}
Let $C$ denote the closure of $A$ in $X$ then, by Theorem 17.4,
$A=\bar A=C\cap Y$ is the closure of $A$ in $Y$. Thus, $A$ is
closed in $X$ since it is the intersection of two closed subsets
of $X$.
\end{proof}
\newpage
\begin{problem}[Munkres \S17, p.\,100, 3]
Show that if $A$ is closed in $X$ and $B$ is closed in $Y$, then
$A\times B$ is closed in $X\times Y$.
\end{problem}
\begin{proof}
Before proceeding we will prove the following set theoretic
result (which was adapted from Exercises 2(n) and 2(o) from \S1,
p.\,14 of Munkres):

\begin{lemma}
For sets $A$, $B$, $C$ and $D$ we the following equalities hold:
\begin{enumerate}[noitemsep,label=(\alph*)]
\item $(A\times B)\cap (C\times D)=(A\cap C)\times (B\cap D)$.
\item $A\times (B\setminus C)=(A\times B)\setminus (A\times C)$.
\item $(A\setminus C)\times B=(A\times B)\setminus (C\times B)$.
\end{enumerate}
that is, the Cartesian product distributes over taking
complements.
\end{lemma}
\begin{proof}[Proof of Lemma 4]
\renewcommand\qedsymbol{$\clubsuit$}
(a) The equality follows (rather straightforwardly) from the
definition of the Cartesian product and the complement of a set
for $x\times y\in (A\times B)\cap (C\times D)$ if and only if
$x\times y\in A\times B$ and $x\times y\in C\times D$ if and only
if $x\in A$ and $x\in C$ and $y\in B$ and $y\in D$ if and only if
$x\in A\cap C$ and $y\in B\cap D$ if and only if $x\times y\in
(A\cap C)\times (B\cap D)$.
\\\\
(b) The point $x\times y\in A\times (B\setminus C)$ if and only
if $x\in A$ and $y\in B\setminus C$ if and only if $x\in A$ and
$y\in B$ and $y\notin C$ if and only if $x\times y\in A\times B$
and $x\times y\notin A\times C$ if and only if $x\times y\in
(A\times B)\setminus (A\times C)$.
\\\\
(c) The very same argument as part (b) can be used, taking $B$ to
be a subset of $A$ and replacing (where appropriate) $A$ by
$A\setminus B$ and $B\setminus C$ by $C$, to prove that
\[
(A\setminus B)\times C=(A\times C)\setminus (B\times C).\qedhere
\]
\end{proof}
Now, since $A$ is closed in $X$ and $B$ is closed in $Y$, their
complements, $X\setminus A$ and $Y\setminus B$ are, by
definition, open in $X$ and $Y$, respectively. Then, the sets
\[
(X\setminus A)\times Y
\quad
\text{and}
\quad
X\times (Y\setminus B)
\]
are open since they are basic open sets in the product topology
on $X\times Y$. So, applying Lemma 4(b) and (c), their
complements
\[
(X\times Y)\setminus (X\setminus A)\times Y=A\times Y
\quad
\text{and}
\quad
(X\times Y)\setminus X\times (Y\setminus B)=X\times B
\]
are closed in $X\times Y$. At last, we have that
\[
(A\times Y)\cap (X\times B)
\]
is the intersection of closed sets, hence, by Theorem 17.1(b), is
closed. By Lemma 4(a),
\[
(A\times Y)\cap (X\times B)=(A\cap X)\times (Y\cap B)=A\times B
\]
so $A\times B$ is closed in $X\times Y$.
\end{proof}
\newpage
\begin{problem}[Munkres \S17, p.\,101, 6(b)]
Let $A$, $B$ and $A_\alpha$ denote subsets of a space $X$. Prove
the following:
\begin{enumerate}[noitemsep]
\item[(b)] $\clsr{A\cup B}=\bar A\cup\bar B$.
\end{enumerate}
\end{problem}
\begin{proof}
By definition, the closure of a set is the intersection of all
closed sets which contain it therefore, $\overline{A\cup
  B}\subset\bar A\cup\bar B$ since $\bar A\cup\bar B$ is a
closed set, by Theorem 17.1(a), which contains $A\cup B$. To see
the reverse containment note that $\bar A\subset\clsr{A\cup B}$
since $\clsr{A\cup B}$ is a closed set which contains
$A$. Similarly $\bar B\subset\clsr{A\cup B}$ so $\bar A\cup \bar
B\subset\clsr{A\cup B}$. Therefore, $\clsr{A\cup B}=\bar
A\cup\bar B$ holds.

Naturally this results extends, by induction, to the case of
finite unions of sets.
\end{proof}

\newpage
\begin{problem}[Munkres \S17, p.\,101, 6(c)]
Let $A$, $B$ and $A_\alpha$ denote subsets of a space $X$. Prove
the following:
\begin{enumerate}[noitemsep]
\item[(b)] $\clsr{\bigcup A_\alpha}\supset\bigcup\clsr{A_\alpha}$.
\end{enumerate}
\end{problem}
\begin{proof}
Let $C$ denote the set $\clsr{\bigcup A_\alpha}$. It is clear, by
the definition of the closure of a set, that $\bar
A_\alpha\subset C$ for every $\alpha$ since $C$ is a closed set
which contains $A_\alpha$, so $\bigcup_\alpha \bar
A_\alpha\subset C$.

The reverse is not true in general; in fact, as Theorem 17.1(3)
suggests, an arbitrary union of closed sets is not even
necessarily closed. For a concrete example consider the family
$A_r=\{r\}$ for $r\in\QQ$. The closure of a point $r$ in $\RR$ is
itself since its complement, $\RR\setminus\{r\}$, is the
union of the open intervals $(-\infty,r)$ and $(r,\infty)$; in
particular, $\{r\}$ is the ``smallest'' closed set containing
$\{r\}$. Hence, we see that the union
\[
\bigcup_{r\in\QQ}\bar A_r=\QQ,
\]
but, by Example 6, $\bar\QQ=\RR$.
\end{proof}
\newpage
\begin{problem}[Munkres \S17, p.\,101, 7]
Criticize the following ``proof'' that $\clsr{\bigcup
  A_\alpha}\subset\bigcup\bar A_\alpha$: if
$\left\{A_\alpha\right\}$ is a collection of sets in $X$ and if
$x\in\clsr{\bigcup A_\alpha}$, then every neighborhood $U$ of $x$
intersects $\bigcup A_\alpha$. Thus $U$ must intersect some
$A_\alpha$, so $x$ must belong to the closure of some
$A_\alpha$. Therefore, $x\in\bigcup\bar A_\alpha$.
\end{problem}
\begin{proof}[Critique]
The main argument, that ``$x$ must belong to the closure of some
$A_\alpha$'', is what is wrong here. The point $x$ may belong to
the closure of multiple $A_\alpha$'s, in fact uncountably many of
them, so that one would have to prove that if $x$ belongs the
closure of some family $A_\beta$ of set, then $x$ must belong to
the union of their closures. This takes us right back to what we
are trying to prove.

Like in Problem 2.4, consider the sets $A_r=\{r\}$ for
$r\in\QQ$. For any open interval about a point $x\in\RR$, say
$B_\delta(x)$ for $\delta>0$, we may

Then, by the density of the irrational numbers in
$\RR$, between any two sets
\end{proof}
\newpage
\begin{problem}[Munkres \S17, p.\,101, 9]
Let $A\subset X$ and $B\subset Y$. Show that in the space
$X\times Y$,
\[
\clsr{A\times B}=\bar A\times\bar B.
\]
\end{problem}
\begin{proof}
By Problem 2.2, $\bar A\times\bar B$ is a closed set which
contains $A\times B$ so it must contain the closure of $A\times
B$, i.e., $\clsr{A\times B}\subset\bar A\times\bar B$. To see the
reverse containment, take a point $x\times y\in\bar A\times\bar
B$. Then, by Theorem 17.5(a), for every neighborhood $U$ of $x$
and every neighborhood $V$ of $y$, the intersections $U\cap A$
and $V\cap B$ are nonempty. Thus, by Lemma 5(a), the set
\[
(V\times U)\cap(A\times B)=(V\cap A)\times (U\cap B)
\]
is nonempty. Then, since $U\times V$ is an arbitrary basis
element containing $x\times y$, by Theorem 17.5(b) $x\times
y\in\clsr{A\times B}$. Thus, $\clsr{A\times B}=\bar A\times\bar
B$.
\end{proof}
\newpage
\begin{problem}[Munkres \S17, p.\,101, 10]
Show that every order topology is Hausdorff.
\end{problem}
\begin{proof}
Let $(X,<)$ denote a nonempty set equipped with a simple order
relation. Then by the definition on Munkres \S14, p.\,84, a basis
for the order topology on $X$ are sets of the following types:
\begin{enumerate}[noitemsep,label=(\arabic*)]
\item All open intervals $(a,b)$ in $X$.
\item All intervals of the form $[a_0,b)$, where $a_0$ is the
  smallest element (if any) of $X$.
\item All intervals of the form $(a,b_0]$, where $b_0$ is the
  largest element (if any) of $X$.
\end{enumerate}
Let $a$ and $b$ be two distinct points in $X$; we may assume,
without loss of generality, that $a<b$. Then, we must show that
there exists neighborhoods $U$ and $V$ of $x$ and $y$,
respectively, such that $U\cap V=\emptyset$.

If $X$ set with finite cardinality the order topology on $X$ will
coincide with the discrete topology so that we may take $\{a\}$
and $\{b\}$ to be neighborhoods of $a$ and $b$. Then,
$\{a\}\cap\{b\}=\emptyset$ so $X$ is Hausdorff.

Now, suppose $X$ is not of finite cardinality.  Define the sets
\[
C=(a,b),
\quad
A=\left\{\,x\in X\;\middle|\;x<a\right\}
\quad
\text{and}
\quad
B=\left\{\,x\in X\;\middle|\;x>b\right\}.
\]
Then at least one of $A$, $B$ or $C$ is nonempty and has infinite
cardinality.

Suppose $A$ is nonempty, but $B$ and $C$ are empty. Take any
element $x\in A$, then $(x,b)$ is a neighborhood of $a$ and $b$
must be a largest element so $(a,b_0]=C\cup\{b\}=\{b\}$ is a
neighborhood of $b$ satisfying
$(x,b)\cap\{b\}=\emptyset$. Similarly, if $B$ is nonempty, but
$A$ and $C$ are empty, $\{a\}$ and $(a,x)$ for some $x\in B$ are
neighborhoods of $a$ and $b$, respectively, with
$\{a\}\cap(a,x)=\emptyset$.

If $C$ is nonempty but $A$ and $B$ are empty, $a$ must be a
smallest element and $b$ must be a largest element. Then, since
$X$ is not finite, there exist at least two distinct elements $x$
and $y$ in $C$ with $x<y$ so $[a,x)$ and $(y,b]$ are
neighborhoods of $a$ and $b$, respectively, with
$[a,x)\cap(y,b]=\emptyset$.

Now, suppose at least two of $A$, $B$ and $C$ are nonempty. If
$C$ is empty, but $A$ and $B$ are nonempty. Then the intervals
$(x,b)=(x,a]$ and $(a,y)=[b,y)$ are neighborhoods of $a$ and $b$
respectively with $(x,b)\cap (a,y)=\emptyset$. If $A$ is empty,
but $B$ and $C$ are nonempty, then $a$ is a smallest
element. Then there exists at least two distinct elements $x$
and $y$ with $x<y$ in $C$ so that $[a,x)$ and $(y,b)$ are
neighborhoods of $a$ and $b$, respectively, with $[a,x)\cap
(y,b)=\emptyset$. Similarly, if $B$ is empty, but $A$ and $C$ are
nonempty, for any $x<y$ in $C$, $(a,x)$ and $(y,b]$ are
neighborhoods of $a$ and $b$, respectively, with $(a,x)\cap
(y,b]$.

Lastly, if $A$, $B$ and $C$ are nonempty we win! Then, for any
$x\in A$, $y\in B$ and $z,w\in C$ with $z<w$ the intervals
$(x,z)$ and $(w,y)$ are neighborhoods of $a$ and $b$,
respectively, with $(x,z)\cap (w,y)=\emptyset$.

In every case, $X$ satisfies the Hausdorff property.
\end{proof}
\begin{remarks*}
Perhaps there is a better way to approach this problem. The
demonstration is thorough and covers every case, but we still
desire a more elegant proof.
\end{remarks*}
\newpage
\begin{problem}[Munkres \S17, p.\,101, 13]
Show that $X$ is Hausdorff if and only if the \emph{diagonal}
$\Delta=\left\{\,x\times x\;\middle|\;x\in X\,\right\}$ is closed
in $X\times X$.
\end{problem}
\begin{proof}
$\implies$ Suppose $X$ is Hausdorff. The diagonal $\Delta$ is
closed, by definition, if and only if its complement, $(X\times
X)\setminus\Delta$, is open in $X\times X$. Let $x\times
y\in(X\times X)\setminus\Delta$. Since $X$ is Hausdorff, there
exists neighborhoods $U$ and $V$ of $x$ and $y$, respectively,
such that $U\cap V=\emptyset$. Thus, $U\times V$ is a
neighborhood of $x\times y$ contained in $(X\times X)\setminus
\Delta$. By the definition on Munkres \S13 p.\,78, since for
every point $x\times y\in(X\times X)\setminus\Delta$ we may find
a basis element $U\times V\subset (X\times X)\setminus\Delta$
containing $x\times y$, it follows that $(X\times
X)\setminus\Delta$ is open. Thus, $\Delta$ is closed.

$\impliedby$ Suppose $\Delta$ is closed. Then the complement
of $\Delta$ is open in $X\times X$. Thus, for every $x\times y$
in the complement of $\Delta$, we may find a basis element
$U\times V\subset(X\times X)\setminus\Delta$ containing $x\times
y$. Thus, $U$ and $V$ are neighborhoods of $x$ and $y$,
respectively, such that $U\cap V=\emptyset$ (for otherwise
$z\times z\in U\times V$ but $U\times V$ is in the complement of
$\Delta$). Thus, $X$ is Hausdorff.
\end{proof}
\newpage
\begin{problem}[Munkres \S18, p.\,111, 4]
Given $x_0\in X$ and $y_0\in Y$, show that the maps $f\colon X\to
X\times Y$ and $g\colon Y\to X\times Y$ defined by
\[
f(x)=x\times y_0\quad\text{and}\quad g(y)=x_0\times y
\]
are imbeddings.
\end{problem}
\begin{proof}
Let $Z=\im f$. To show that $f\colon X\to X\times Y$ is an
imbedding, we will show that the map $f'\colon X\to Z$, which is
obtained by restricting the codomain of $f$ is a continuous
injection with a continuous inverse $g$. First we shall show
injectivity. To see that $f$ is continuous we note that $f$ can
be written as the tuple $f'(x)=(f_1,f_2)$ where $f_1=\id_X$ and
$f_2$ is the constant map $x\mapsto y_0$ for all $x\in X$. The
maps $f_1$ and $f_2$ are continuous (by Theorem 18.2(a) and (b))
so, by Theorem 18.4, $f$ is continuous. To prove that $f$ is
bijective it suffices to exhibit an inverse. We claim that the
map $F=\restr{\pi_X}{Z}$ is an inverse (continuity follows of $F$
from Theorem 18.2(d) and the fact that projections are continuous
as discussed on \S15 pp.\,87-88). But this claim is clear since
\begin{align*}
F\circ f(x)
&=F(f(x))&
f\circ F(x\times y_0)
&=f(F(x\times y_0))\\
&=F(x\times y_0)&
&=f(x)\\
&=x&&=x\times y_0\\
&=\id_X(x)&
&=\id_Z(x\times y_0).
\end{align*}
Thus, $f$ is an imbedding.

The proof that $g$ is an imbedding is analogous (it is sufficient
to replace $f$ by $g$, $F$ by $G$, $x\times y_0$ by $x_0\times
y$, $x\mapsto y_0$ by $y\mapsto x_0$, $\pi_X$ by $\pi_Y$, and
$\id_X$ by $\id_Y$ in the argument above). So as not to be
penalized for not providing the proof for $g$ we copy and paste,
making the appropriate replacements, here:
\\\\
Let $Z=\im g$. To show that $g\colon Y\to X\times Y$ is an
imbedding, we will show that the map $g'\colon Y\to Z$, which is
obtained by restricting the codomain of $g$ is a continuous
injection with a continuous inverse $g$. First we shall show
injectivity. To see that $g$ is continuous we note that $g$ can
be written as the tuple $g'(x)=(g_1,g_2)$ where $g_1=\id_Y$ and
$g_2$ is the constant map $y\mapsto x_0$ for all $y\in Y$. The
maps $g_1$ and $g_2$ are continuous (by Theorem 18.2(a) and (b))
so, by Theorem 18.4, $g$ is continuous. To prove that $g$ is
bijective it suffices to exhibit an inverse. We claim that the
map $G=\restr{\pi_Y}{Z}$ is an inverse (continuity of $G$ follows
from Theorem 18.2(d) and the fact that projections are continuous
as discussed on \S15 pp.\,87-88). But this claim is clear since
\begin{align*}
G\circ G(x)
&=G(g(x))&
g\circ G(x_0\times y)
&=g(G(x_0\times y))\\
&=G(x_0\times y)&
&=g(y)\\
&=y&&=x_0\times y\\
&=\id_Y(y)&
&=\id_Z(x_0\times y).
\end{align*}
Thus, $g$ is an imbedding.
\end{proof}
\newpage
\begin{problem}[Munkres \S18, p.\,111-112, 8(a,b)]
Let $Y$ be an ordered set in the order topology. Let
$f,g\colon X\to Y$ be continuous.
\begin{enumerate}[noitemsep,label=(\alph*)]
\item Show that the set
  $\left\{\,x\;\middle|\;f(x)\leq g(x)\,\right\}$ is closed in $X$.
\item Let $h\colon X\to Y$ be the
  function \[h(x)=\min\{f(x),g(x)\}.\] Show that $h$ is
  continuous. [\emph{Hint:} Use the pasting lemma.]
\end{enumerate}
\end{problem}
\begin{proof}

\end{proof}
\newpage
\begin{problem}
Given: $X$ is a topological space with open sets $U_1,...,U_n$
such that $\bar U_i=X$ for all $i$. Prove that the closure of
$U_1\cap\cdots\cap U_n$ is $X$.
\end{problem}
\begin{proof}
\end{proof}

%%% Local Variables:
%%% mode: latex
%%% TeX-master: "../MA571-HW-Current"
%%% End:
