\begin{problem}
Let $A\subset X$ and $B\subset Y$. Show that the space $X\times Y$,
\[\clsr{A\times B}=\clsr A\times\clsr B.\]
\end{problem}
\begin{proof}
Before we proceed, we need to prove the following nontrivial facts:
\begin{claim}[Munkres \S17, Ex.\,3]
If $A$ is closed in $X$ and $B$ is closed in $Y$, then $A\times
B$ is closed in $X\times Y$.
\end{claim}
\begin{proof}[Proof of claim]
\renewcommand\qedsymbol{$\clubsuit$}
We will show that the complement of $A\times B$ is open in
$X\times Y$. Let $(x,y)\in (X\times Y)\setminus(A\times B)$. Then
$x\notin A$ and $y\notin B$. Since $A$ and $B$ are closed in $X$
and $Y$, respectively, there exist neighborhoods $U$ and $V$ of
$x$ and $y$, respectively, such that $U\subset X\setminus A$ and
$V\subset Y\setminus B$. Then $U\times V\subset (X\times
Y)\setminus(A\times B)$ is a neighborhood of $(x,y)$ so, by Lemma
C, $(X\times Y)\setminus(A\times B)$ is open. Thus, $A\times B$
is closed.
\end{proof}
% \begin{claim}[Munkres \S17, Ex.\,6(a)]
% If $A\subset B$, then $\clsr A\subset\clsr B$.
% \end{claim}
% \begin{proof}[Proof of claim]
% \renewcommand\qedsymbol{$\clubsuit$}
% By Theorem 17.5(a), $x\in\clsr A$ if and only if for every
% neighborhood $U\ni x$, $U\cap A\neq\emptyset$. In particular,
% since $A\subset B$, $U\cap B\neq\emptyset$ for every neighborhood
% $U\ni x$ for every $x\in\clsr A$. Thus, $x\in\clsr B$ and we have
% the following containment $\clsr A\subset\clsr B$.
% \end{proof}

Since $A\subset\clsr A$ and $B\subset\clsr B$ then $A\times
B\subset\clsr A\times\clsr B$. Then by Lemma B
$\clsr{A\times B}\subset\clsr{\clsr{A}\times\clsr{B}}$, but by
Claim 1 $\clsr{\clsr{A}\times\clsr{B}}=\clsr A\times\clsr B$ so
$\clsr{A\times B}\subset\clsr A\times\clsr B$. To see the reverse
containment, take an element $(x,y)\in \clsr A\times\clsr B$ then
for $x\in\clsr A$ and $y\in\clsr B$. Thus, by Theorem 17.5(a) for
every neighborhood $U\ni x$ and $V\ni y$, we have $U\cap
A\neq\emptyset$ and $V\cap B\neq\emptyset$. Thus, $U\times V\cap
A\times B\neq\emptyset$ so by Theorem 17.5(b), since $U\times V$
is a basis element for the topology on $X\times Y$,
$(x,y)\in\clsr{A\times B}$. Thus, $\clsr{A\times B}\supset\clsr
A\times\clsr B$ and the equality $\clsr{A\times B}=\clsr
A\times\clsr B$ holds.
\end{proof}
\begin{problem}
Let $X$ be a topological space and let $A$ be a dense subset of
$X$. Let $Y$ be a Hausdorff space and let $g,h\colon X\to Y$ be
continuous functions which agree on $A$. Prove that $g=h$.
\end{problem}
\begin{proof}
Suppose, towards a contradiction, that $g\neq h$. Then $g(x)\neq
h(x)$ for some $x\in X\setminus A$. Since $Y$ is Hausdorff, there
exists neighborhoods $U\ni g(x)$ and $V\ni h(x)$ with $U\cap
V=\emptyset$. Since $g$ and $h$ are continuous, $g^{-1}(U)$ and
$h^{-1}(V)$ are neighborhoods of $x$. In particular,
$g^{-1}(U)\cap h^{-1}(U)$ is a nonempty neighborhood of
$x$. Since $\clsr A=X$, by Theorem 17.5(a), $\left(g^{-1}(U)\cap
h^{-1}(V)\right)\cap A\neq\emptyset$. Let $x_0\in\left(g^{-1}(U)\cap
h^{-1}(V)\right)\cap A$. Then $g(x_0)=h(x_0)\in U\cap V$. This
contradicts the fact that $U$ and $V$ were chosen to be
disjoint.
\end{proof}
\begin{problem}
Let $X$ and $Y$ be topological spaces and let $f\colon X\to Y$
be a continuous function. Let $G_f$ (called the \emph{graph} of
$f$) be the subspace $\left\{\,x\times f(x)\;\middle|\;x\in
  X\,\right\}$ of $X\times Y$. Prove that if $Y$ is Hausdorff then
$G_f$ is closed.
\end{problem}
\begin{proof}
We will show that the complement of $G_f$ in $X\times Y$ is
open. Let $(x,y)\in(X\times Y)\setminus G_f$. Since $Y$ is
Hausdorff, choose neighborhoods $U$ and $V$ of $y$ and $f(x)$
respectively, such that $f^{-1}(U)\cap V=\emptyset$. Then
$f^{-1}(U)\times V\ni(x,y)$ is contained in the complement of
$G_f$ so, by Lemma C, $G_f$ is open.
\end{proof}
\begin{problem}
Let $X$ be a topological space and let $f,g\colon X\to\RR$ be
continuous. Define $h\colon X\to\RR$ by
\[
h(x)=\min\left\{(f(x),g(x)\right\}.
\]
Use the pasting lemma to prove that $h$ is continuous. (You will
not get full credit for any other method.)
\end{problem}
\begin{proof}
Define the sets
\[
A=\left\{\,x\in X\;\middle|\;f(x)\leq g(x)\,\right\}
\quad\text{and}\quad
B=\left\{\,x\in X\;\middle|\;f(x)\geq g(x)\,\right\}.
\]
Note $X=A\cup B$ and $f(x)=g(x)$ for every $x\in A\cap
B$. Moreover, we have that
\[
h(x)
=\min\{f(x),g(x)\}
=\begin{cases}
f(x)&\text{if $x\in A$}\\
g(x)&\text{if $x\in B$}
\end{cases}.
\]
Thus, by the pasting lemma, $h$ is continuous if we can show that
$A$ and $B$ are closed in $X$.

We will prove that the complement of $A$ in $X$ is open; the
proof of $B$ is similar. Let $x\in X\setminus A$. Then
$f(x)>g(x)$. Thus we have the following result
\begin{lemma}
Let $x,y\in X$ with the order topology. Then there exists a
neighborhood $U\ni x$, $V\ni y$ with $U\cap V=\emptyset$ and
$x'<y'$ for all $x'\in U$, $y'\in V$.
\end{lemma}
\begin{proof}[Proof of lemma]
\renewcommand\qedsymbol{$\clubsuit$}
We break the demonstration into the following cases:
\\\\
Case 1: Suppose there exists $z\in X$ with $x<z<y$, i.e.,
$z\in(x,y)$. Let $U$ be the ray $U=(-\infty,z)$ and $V$ be the
ray $V=(z,\infty)$. Then $U\cap V=\emptyset$ and for every $x'\in
V$, $y'\in U$ $x'<c<y'$, in particular, $x'<y'$.
\\\\
Case 2: Suppose that there does not exists $z\in X$ with $x<z<y$,
i.e., $(x,y)=\emptyset$. Let $U$ be the ray $U=(-\infty,x)$ and
$V$ be the ray $V=(y,\infty)$. Then $U\cap V=\emptyset$ and for
every $x'\in U$, $y'\in V$ we have $x'<x<y<y'$, in particular,
$x'<y'$.
\end{proof}

By Lemma 2, choose $U\ni g(x)$ and $V\ni f(x)$ as above. Then
$g^{-1}(U)\cap f^{-1}(V)$ is a neighborhood of $x$ with
$g(x)<f(x)$ for all. Hence $g^{-1}(U)\cap f^{-1}(V)\subset
X\setminus A$ and, by Lemma C, $X\setminus A$ is open. Thus, $A$
is closed.

Having satisfied the conditions of the pasting lemma (Theorem 18.3), it follows
that $h$ is continuous.
\end{proof}
\begin{problem}
Let $X$ and $Y$ be topological spaces and let $f\colon X\to Y$ be
a function with the property that
\[
f(\clsr A)\subset\clsr{f(A)}
\]
for all subsets $A$ of $X$. Prove that $f$ is continuous.
\end{problem}
\begin{proof}
Suppose that $f$ has the property given above. Then we claim that:
\begin{claim}
For every closed set $B$ of $Y$, $f^{-1}(B)$, $f^{-1}(B)$ is
closed in $X$.
\end{claim}
\begin{proof}[Proof of claim]
\renewcommand\qedsymbol{$\clubsuit$}
Let $B$ be closed in $Y$. We will show that
$\clsr{f^{-1}(B)}=f^{-1}(B)$. To that end, it suffices to show
that $\clsr{f^{-1}(B)}\subset f^{-1}(B)$ since the containment
$\clsr{f^{-1}(B)}\supset f^{-1}(B)$ is immediate (from the
definition of the closure). By Munkres \S2 Ex.\,1(b), we have
that $f(f^{-1}(B))\subset B$ so if $x\in\clsr{f^{-1}(B)}$ then
$f(x)\in B$ since, by our assumption on $f$ together with Lemma
C, we have
\[
f\bigl(\clsr{f^{-1}(B)}\bigr)\subset\clsr{f(f^{-1}(B))}\subset B.
\]
Thus, $x\in f^{-1}(B)$ so $\clsr{f^{-1}(B)}\subset f^{-1}(B)$ as
desired.
\end{proof}
Let $U$ be open in $Y$. Then $Y\setminus U$ is closed in
$Y$. Then, by Claim 3, $f^{-1}(Y\setminus U)=X\setminus
f^{-1}(U)$ is closed in $X$ so $X\setminus (X\setminus
f^{-1}(U))=f^{-1}(U)$ is open in $X$. Thus, $f$ is continuous.
\end{proof}
\begin{problem}
Let $X$ and $Y$ be topological spaces and let $f\colon X\to Y$ be
a continuous function. Prove that
\[
f(\clsr A)\subset\clsr{f(A)}
\]
for all subsets $A$ of $X$.
\end{problem}
\begin{proof}
Suppose $f$ is continuous. Then, for every $U$ open in $Y$,
$f^{-1}(U)$ is open in $X$. Let $A\subset X$ and consider
$\clsr{f(A)}$. Then, $f^{-1}\bigl(Y\setminus\clsr{f(A)}\bigr)=X\setminus
f^{-1}\bigl(\clsr{f(A)}\bigr)$ is open in $X$ so its complement
$f^{-1}\bigl(\clsr{f(A)}\bigr)$ is closed in $X$. Moreover, by
Munkres \S2 Ex.\,1(a), we we have $A\subset f^{-1}(f(A))$ and
since, by Theorem 17.6, $\clsr{f(A)}=f(A)\cup f(A)'$ we have that
\[
A\subset f^{-1}\bigl(\clsr{f(A)}\bigr)=f^{-1}(f(A)\cup
f(A)')=f^{-1}(f(A))\cup f^{-1}(f(A)').
\]
In particular, by Lemma C, $\clsr A\subset
f^{-1}\bigl(\clsr{f(A)}\bigr)$ so, by Munkres \S2 Ex.\,1(b), we have
\[
f\bigl(\clsr A\bigr)\subset
f\bigl(f^{-1}\bigl(\clsr{f(A)}\bigr)\bigr)
\subset
\clsr{f(A)},
\]
as desired.
\end{proof}
\begin{problem}
Let $X$ be any topological space and let $Y$ be a Hausdorff
space. Let $f,g\colon X\to Y$ be continuous functions. Prove that
the set $\left\{\,x\in X\;\middle|\;f(x)=g(x)\,\right\}$ is
closed.
\end{problem}
\begin{proof}
By Munkres \S17 Ex.\,13, $Y$ is Hausdorff if and only if
$\Delta_Y=\left\{\,(y,y)\;\middle|\;y\in Y\,\right\}$ is closed in
$Y\times Y$. By Theorem 18.4, the map $F=(f,g)\colon X\to Y\times
Y$ is continuous since $f$ and $g$ are continuous. We claim that
$F^{-1}(\Delta_Y)=\left\{\,x\in
  X\;\middle|\;f(x)=g(x)\,\right\}$.

It is clear that if $f(x)=g(x)=y$ then $F(x)=(y,y)\in\Delta_Y$ so
$F^{-1}(\Delta_Y)\supset\left\{\,x\in
  X\;\middle|\;f(x)=g(x)\,\right\}$. Now suppose $x\in
F^{-1}(\Delta_Y)$ then $F(x)=(f(x),g(x))=(y,y)\in\Delta_Y$ so
$f(x)=g(x)=y$ so $x\in\left\{\,x\in
  X\;\middle|\;f(x)=g(x)\,\right\}$. Thus, $F^{-1}(\Delta_Y)=\left\{\,x\in
  X\;\middle|\;f(x)=g(x)\,\right\}$ so, by Theorem 18.1(3), it
follows that $\left\{\,x\in X\;\middle|\;f(x)=g(x)\,\right\}$ is
closed in $X$.
\end{proof}
\begin{problem}
Let $X$ be a topological space and $A$ a subset of $X$. Suppose that
\[
A\subset\clsr{X\setminus\clsr A}.
\]
Prove that $\clsr{A}$ does not contain any nonempty open set.
\end{problem}
\begin{proof}
Suppose, seeking a contradiction, that $\Int A\neq\emptyset$. Then there
exists $x\in\Int A\subset A$ and a neighborhood $U\ni x$ with
$U\subset A$. Then $U\subset\clsr{X\setminus\clsr A}$. In
particular, $x\in\clsr{X\setminus\clsr A}$ so
$U\cap\clsr{X\setminus\clsr A}\neq\emptyset$. But $U\subset
A\subset\clsr A$ so $U\cap \bigl(X\setminus\clsr
A\bigr)=\emptyset$. This is a contradiction since
$x\in\clsr{X\setminus\clsr A}$. Thus, $\Int A=\emptyset$.
\end{proof}
\begin{problem}
Let $X$ be a topological space with a countable basis. Prove that
every open cover of $X$ has a countable subcover.
\end{problem}
\begin{proof}
\end{proof}
\begin{problem}
Let $X_\alpha$ be an infinite family of topological spaces.
\begin{enumerate}[noitemsep,label=(\alph*)]
\item Define the product topology on $\prod X_\alpha$.
\item For each $\alpha$, let $A_\alpha$ be a subspace of
  $X_\alpha$. Prove that $\clsr{\prod
    A_\alpha}=\prod\clsr{A_\alpha}$.
\end{enumerate}
\end{problem}
\begin{proof}
(a)From Munkres \S19, p.\,114:
\begin{definition*}
Let $\mathcal{S}_\beta$ denote the collection
\[
\mathcal{S}_\beta=\left\{\,\pi_\beta^{-1}(U_\beta)\;\middle|\;\text{$U_\beta$
  open in $X_\beta$}\,\right\},
\]
and let $\mathcal{S}$ denote the union of these collections,
\[
\mathcal{S}=\bigcup\mathcal{S}_\beta.
\]
The topology generated by the subbasis $\mathcal{S}$ is called
the \emph{product topology}.
\end{definition*}
Alternatively, we have the theorem:
\begin{theorem*}[Munkres, Thm.\,19.2]
Suppose the topology on each space $X_\alpha$ is given by a basis
$\mathcal{B}_\alpha$. The collection of all sets of the form
\[
\prod B_\alpha,
\]
where $B_\alpha\in\mathcal{B}_\alpha$ for finitely many indicel
$\alpha$ and $B_\alpha=X_\alpha$ for all the remaining indices is
a basis for the product topology on $\prod X_\alpha$.
\end{theorem*}

(b) (cf.\,Munkres \S19, Theorem 19.5) Let
$\mathbf{x}=(x_\alpha)\in\prod\clsr A_\alpha$; we show that
$\mathbf{x}\in\clsr{\prod A_\alpha}$. Let $U=\prod U_\alpha\ni x$ be a
basis element. Since $x_\alpha\in\clsr A_\alpha$, there exists
$y_\alpha\in U_\alpha\cap A_\alpha$ for each $\alpha$. Then
$\mathbf{y}=(y_\alpha)$ belongs to both $U$ and $\prod
A_\alpha$. Since $U$ is arbitrary, it follows that
$\mathbf{x}\in\clsr{\prod A_\alpha}$.

Conversely, suppose that $\mathbf{x}=(x_\alpha)\in\clsr{\prod
  A_\alpha}$; we show that $x_\beta\in\clsr A_\beta$ for any
index $\beta$. Let $V_\beta\ni x_\beta$ be an arbitrary
neighborhood in $X_\beta$. Since $\pi_\beta^{-1}(V_\beta)$ is
open in $\prod X_\alpha$, it contains a point
$\mathbf{y}=(y_\alpha)$ of $\prod A_\alpha$. Then $y_\beta\in
V_\beta\cap A_\beta$. It follows that $x_\beta\in\clsr A_\beta$.
\end{proof}
\begin{problem}
Suppose that we are given an indexing set $A$, and for each
$\alpha\in A$ a topological space $X_\alpha$. Suppose also that
for each $\alpha\in A$ we are given a point $b_\alpha\in
X_\alpha$. Let $Y=\prod X_\alpha$ with the product topology. Let
$\pi_\alpha\colon Y\to X_\alpha$ be the projection. Prove that
the set
\[
S=\left\{\,y\in Y\;\middle|\;\text{$\pi_\alpha(y)=b_\alpha$ except for
    finitely many $\alpha$}\,\right\}
\]
is dense in $Y$ (that is, its closure is $Y$).
\end{problem}
\begin{proof}
We want to show that $\clsr S=X$ therefore, we will show that for
every open subset $U$ of $X$, $U\cap S\neq\emptyset$. By Theorem
17.5(b), it suffices to show this for basis elements. Let
$\mathcal{B}_\alpha$ be a basis for $X_\alpha$ and $U=\prod
U_\alpha$ be a basis element in the product topology on $\prod
X_\alpha$. Then, by Theorem 19.2, $U_\alpha\in\mathcal{B}_\alpha$
for finitely many indices $\alpha$ and $U_\alpha=X_\alpha$ for
all the remaining indices. Hence, at least one $X_\alpha\ni
b_\alpha$ so $U\cap S\neq\emptyset$. Since $U$ was arbitrary, we
conclude that $\clsr S=X$.
\end{proof}
\begin{problem}
Let $X$ be the Cartesian product
$\RR^\omega=\prod_{i=1}^\infty\RR$ with the box topology (recall
that a basis for this topology consists of all sets of the form
$\prod_{i=1}^\infty U_i$, where each $U_i$ is open in $\RR$). Let
$f\colon\RR\to X$ be the function which takes $t$ to
$(t,t,t,...)$. Prove that $f$ is not continuous.
\end{problem}
\begin{proof}
(cf.\,Example 2 in Munkres \S19) It suffices to show that the
preimage of a basis element $U$ in the box topology is not open
in $\RR$. Let
\[
U=\prod\left(-\frac{1}{n},\frac{1}{n}\right).
\]
Suppose that $f$ is continuous. Then $f^{-1}(U)$ is open. Then by
18.1(4), for some $\delta>0$, $(-\delta,\delta)\ni 0\subset
f^{-1}(U)$, $f((-\delta,\delta))=\prod (-\delta,\delta)\subset
B$. But, by the Archimedean principle, there exists $n\in\ZZ_+$
such that $1/n<\delta$ so $(-\delta,\delta)\nsubset(-1/N,1/N)$
for any $N\geq n$. This is a contradiction. Therefore, $f$ is not
continuous on $\RR^\omega$ with the box topology.
\end{proof}
\begin{problem}
Prove that the countable product $\RR^\omega$ (with the product
topology) has the following property: there is a countable family
$\mathcal{F}$ of neighborhoods of the point
$\mathbf{0}=(0,0,0,...)$ such that for every neighborhood $V$ of
$\mathbf{0}$ there is a $U\in\mathcal{F}$ with $U\subset V$.
\\\\
Note: the book proves that $\RR^\omega$ is a metric space, but
you may not use this in your proof. Use the definition of the
product topology.
\end{problem}
\begin{proof}
Define $\mathcal{F}$ to be the collection of all sets
$U_{k,\ell}=\prod U_n$ where $U_n=(-1/k,1/k)$ for $1\leq
n\leq\ell$ and $U_n=\RR$ otherwise. Then we want to show that for
every neighborhood $V$ of $\mathbf{0}$, there exists
$U\in\mathcal{F}$ with $U\subset V$. By Theorem 17.5(b) it
suffices to prove this for basis elements containing
$\mathbf{0}$. Hence, let $V=\prod V_n$ be a basis element
containing $\mathbf{0}$. Then, by Theorem 19.2, $V_n$ is a basis
element for the standard topology on $\RR$ containing $0$, i.e,
$V_n=(a_n,b_n)$ for $a_n<0<b_n$, for finitely many $n$ and
$V_n=\RR$ otherwise. Without loss of generality, we may assume
that $V=(a_1,b_1)\times\cdots(a_N,b_N)\times\RR\times\cdots$. Let
$\delta=\min\left\{|a_1|,b_1,...,|a_N|,b_N\right\}$. Then by the
Archimedean principle, there exists a positive integer $m$ such
that $1/m<\delta$. Thus, $U_{m,N}\subset V$.
\end{proof}
\begin{problem}
Let $X$ be the two-point set $\{0,1\}$ with the discrete
topology. Let $Y$ be a countable product of copies of $X$, thus
an element of $Y$ is a sequence of $0$'s and $1$'s. For each
$n\geq 1$, let $y_0\in Y$ be the element $(1,...,1,0,..)$, with
$n$ $1$'s at the beginning and all other entries $0$. Let $y\in
Y$ be the element with all $1$s. Prove that the set
$\left\{y_n\right\}_{n\geq 1}\cup\{y\}$ is closed. Give a clear
explanation. Do not use a metric.
\end{problem}
\begin{proof}
Let $A=\left\{y_n\right\}_{n\geq 1}\cup\{y\}$. We will show that
the complement of $A$ in $Y$ is open. By Lemma C, it suffices to
find a basis element $U\ni\mathbf{x}$ with $U\cap
A=\emptyset$. Let $\mathbf{x}\in  Y\setminus A$. Then
$\mathbf{x}$ is a sequence of $0$'s and $1$'s where, say the
first $n$ terms, are not all $1$. Let $k$, for $1\leq k\leq n$,
be the first zero to appear in the sequence $\mathbf{x}$ and
$\ell$, for $\ell>k$, be the first one to appear right
after. Then the product $U=\prod U_n$ where
\[
U_n=
\begin{cases}
\{0\}&\text{if $n=k$,}\\
\{1\}&\text{if $n=\ell$,}\\
X&\text{otherwise},
\end{cases}
\]
is a basis element containing $\mathbf{x}$, but $U\cap
A=\emptyset$ for otherwise there is a sequence $\mathbf{y}\in A$
with $y_k=0$, but $y_\ell=1$ which is impossible since $\ell>k$
and $A$ consists of sequences $\mathbf{y}$ with the property that
if $y_N=1$ then $y_n=1$ for all $n\leq N$. Thus, $Y\setminus A$
is open so $A$ is closed.
\end{proof}
\begin{problem}
Let $X$ be the two-point set $\{0,1\}$ with the discrete
topology. Let $Y$ be a countable product of copies of $X$; thus
an element of $Y$ is a sequence of $0$'s and $1$'s. Let $A$ be
the subset of $Y$ consisting of sequences with only a finite
number of $1$'s. Is $A$ closed? Prove or disprove.
\end{problem}
\begin{proof}
$A$ is not closed. Consider the point $\mathbf{1}=(1,1,...)\notin
A$. But for every basis element $U=\prod U_n\ni\mathbf{1}$ where
$U_n=X_n$ except for finitely many $n$'s, $U\cap A\neq\emptyset$.
\end{proof}
\begin{problem}
Let $Y$ be a topological space.Let $X$ be a set and let $f\colon
X\to Y$ be a function. Give $X$ the topology in which the open
sets are the sets $f^{-1}(V)$ with $V$ open in $Y$ (you do not
have to verify that this is a topology). Let $a\in X$ and let $B$
be a closed set in $X$ not containing $a$. Prove that $f(a)$ is
not in the closure of $f(B)$.
\end{problem}
\begin{proof}
Suppose $B$ is closed in $X$ and $a\in X\setminus B$. Then
$X\setminus B$ is open in $X$ so $X\setminus B=f^{-1}(V)$ for
some $V$ open in $Y$. Then $f(X\setminus B)\subset V\ni f(a)$
with $V\cap f(B)=\emptyset$ (otherwise $f(b)\in V$ for some $b\in
B$, but the preimage of $V$ lies in the complement of $B$). By
Theorem 17.5(a), $f(a)\notin\clsr{f(B)}$.
\end{proof}
\begin{problem}
Let $f\colon X\to Y$ be a function that takes closed sets to
closed sets. Let $y\in Y$ and let $U$ be an open set containing
$f^{-1}(y)$. Prove that there is an open set $V$ containing $y$
such that $f^{-1}(V)$ is contained in $U$.
\end{problem}
\begin{proof}
Since $U$ is open in $X$, $X\setminus U$ is closed in $X$. Since
$f$ is a closed mapping, $f(X\setminus U)$ is closed in $Y$ so
$Y\setminus f(X\setminus U)$ is open in $Y$. Moreover, $y\in
Y\setminus f(X\setminus U)$ since $y\notin f(X\setminus U)$. Let
$V\ni y$ open in $Y$. Then we claim that $f^{-1}(V)\subset
U$. Otherwise, there exists $x\in f^{-1}(V)\cap(X\setminus U)$ so
$f(x)\in V\cap f(X\setminus U)$, but this contradicts that
$V\subset X\setminus f(X\setminus U)$.
\end{proof}
\begin{problem}
Let $X$ be a topological space with an equivalence relation
$\sim$. Suppose that the quotient space $X/{\sim}$ is
Hausdorff. Prove that the set $S=\left\{\,x\times y\in X\times
  X\;\middle|\;x\sim y\,\right\}$ is a closed subset of $X\times
X$.
\end{problem}
\begin{proof}
Recall that a space $Y$ is Hausdorff if and only if $\Delta_Y$ is
closed in $Y\times Y$. Therefore, $X/{\sim}$ is Hausdorff implies
$\Delta_{X/{\sim}}$ is closed in $X/{\sim}\times X/{\sim}$. Now
consider the map $P=(p,p)\colon X\to X/{\sim}\times
X/{\sim}$ where $p\colon X\to X/{\sim}$ is the quotient map on
$X$. $p$ is continuous by the definition of the quotient topology
so by Theorem 18.4, the composite map $P$ is continuous since it
is continuous in each factor. Hence, we have that
\begin{align*}
P^{-1}(\Delta_{X/{\sim}})
&=
\left\{\,(x,y)\in X\times X\;\middle|\;P(x,y)\in\Delta_{X/{\sim}}\,\right\}\\
&=
\left\{\,(x,y)\in X\times X\;\middle|\;p(x)=p(y)\,\right\}\\
&=
\left\{\,(x,y)\in X\times X\;\middle|\;x\sim y\,\right\}\\
&=S,
\end{align*}
so by Theorem 18.1(3), $S$ is closed in $X$.
\end{proof}
\begin{problem}
Let $p\colon X\to Y$ be a quotient map. Let us say that a subset
$S$ of $X$ is \emph{saturated} if it has the form $p^{-1}(T)$ for
some subset $T$ of $Y$. Suppose that for every $y\in Y$ and every
open neighborhood $U$ of $p^{-1}(y)$ there is a saturated open
set $V$ with $p^{-1}(y)\subset V\subset U$. Prove that $p$ takes
closed sets to closed sets.
\end{problem}
\begin{proof}
Suppose that $W\neq X$ is closed so $X\setminus W$ is open. If
$p(W)=Y$ we are done. Suppose $p(W)\neq Y$. Then there exists
some $y\in Y\setminus p(W)$ so $p^{-1}(y)\subset X\setminus
W$. Then, for some open $V\in Y$, $p^{-1}(y)\subset
p^{-1}(V)\subset X\setminus W$. Thus, $y\in V\subset p(X\setminus
W)$, but $p(X\setminus W)\subset Y\setminus p(W)$ since $y\in
p(X\setminus W)$ if and only if $y=p(x)$ for $x\notin W$, but
$y\in Y\setminus p(W)$ if and only if $y\neq p(x)$ for $x\in
W$. Thus, $Y\setminus p(W)$ is open so $p(W)$ is closed.
\end{proof}
\begin{problem}
Let $X$ be a topological space, let $D$ be a connected subset of
$X$, and let $\left\{E_\alpha\right\}$ be a collection of
connected subsets of $X$.
\\\\
Prove that if $D\cap E_\alpha\neq\emptyset$ for all $\alpha$,
then $D\cup\left(\bigcup E_\alpha\right)$ is connected.
\end{problem}
\begin{proof}
Consider the collection $\left\{D_\alpha\right\}$ where
$D_\alpha=D\cup E_\alpha$. By Theorem 23.3, $D\cup E_\alpha$ is
connected so every $D_\alpha$ is connected. Moreover
$D_\alpha\cap D_\beta\supset D\neq\emptyset$ so by Theorem 23.3,
\[
\bigcup D_\alpha=\bigcup D\cup E_\alpha=D\cup\left(\bigcup E_\alpha\right)
\]
is connected.
\end{proof}
\begin{problem}
Let $X$ and $Y$ be connected. Prove that $X\times Y$ is connected.
\end{problem}
\begin{proof}
Seeking a contradiction, suppose $C,D$ is a separation of
$X\times Y$. Fix an $y_0\in Y$. Then the map
$X\hookrightarrow X\times Y$ given by $x\mapsto (x,y_0)$ is
continuous (by Theorem 18.4) so by Theorem 23.5 its image,
$X\times y_0$, is connected. Similarly, the maps $y\mapsto
(x,y)$ for fixed $x\in X$ are continuous and hence their images,
$x\times Y$ are connected. Since $X\times y_0$ is connected, by
Theorem 23.2, $X\times y_0\subset C$ or $D$. Without loss of
generality, suppose $X\times y_0\subset C$. Then, since $x\times
Y\cap X\times y_0\ni (x,y_0)\neq\emptyset$ then $x\times Y\subset
C$ for all $x$. Thus,
\[
X\times y_0\cup\left(\bigcup_{x\in X}x\times Y\right)=X\times
Y\subset C
\]
implies that $D=\emptyset$. This contradicts the assumption
that $C,D$ is a separation of $X\times Y$.
\end{proof}
\begin{problem}
For any space $X$, let us say that two points are ``inseparable''
if there is no separation $X=U\cup V$ into disjoint open sets
such that $x\in U$ and $y\in V$.
\\\\
Write $x\sim y$ if $x$ and $y$
are inseparable. Then $\sim$ is an equivalence relation (you
don't have to prove this).
\\\\
Now suppose that $X$ is locally connected (this means that for
every point $x$ and every open neighborhood $U$ of $x$, there is
a connected open neighborhood $V$ of $x$ contained in $U$).
\\\\
Prove that each equivalence class of the relation $\sim$ is connected.
\end{problem}
\begin{proof}
Let $x$ and $C_x$ be the equivalence class of $x$. Then we claim
that $C_x$ is both closed and open.

To see that $C_x$ is closed we will prove that for every
$y\in\clsr C_x$ is in $C_x$. Suppose not, then there exists some
neighborhood $U$ of $y$ and $V$ of $x$ such that $U\cap
V=\emptyset$ and $X=U\cup V$. This contradicts Theorem 17.5(a)
that $y\in\clsr C_x$ if and only if for every neighborhood $U$ of
$y$, $U\cap C_x\neq\emptyset$.

To see that $C_x$ is open we will prove that its complement,
$X\setminus C_x$ is closed. Let $y\in\clsr{X\setminus C_x}$. Then
if $y\notin X\setminus C_x$, $y\in C_x$. Thus for any $z\in
X\setminus C_x$ there exists a neighborhood $U\ni z$ and $V\ni y$
with $U\cap V=\emptyset$ and $X=U\cup V$.
\end{proof}
\begin{problem}
Let $X$ be a topological space. Let $A\subset X$ be
connected. Prove $\clsr A$ is connected.
\end{problem}
\begin{proof}
Seeking a contradiction, suppose $C,D$ is a separation of $\clsr
A$. Then, by Theorem 23.2, $A\subset C$ or $A\subset D$. Suppose,
without loss of generality, that $A\subset C$. Let $B$ denote the
closure of $A$ in $\clsr A$ with the subspace topology. Then by
Theorem 17.4, $B=\clsr A\cap\clsr A=\clsr A$. Let $x\in
B\setminus C$. Then, for every neighborhood $U\ni x\subset D$,
$U\cap A\neq\emptyset$. But $D\cap C=\emptyset$. This is a
contradiction. Thus, $\clsr A$ is connected.
\end{proof}
\begin{problem}
Let $X_1,X_2,...$ be topological spaces. Suppose
$\prod_{n=1}^\infty X_n$ is locally connected. Prove that all but
finitely many $X_n$ are connected.
\end{problem}
\begin{proof}
Let $U=\prod U_n$ be a basis element. Then, by Theorem 19.2,
$U_n=X_n$ except for finitely many $X_n$. Let $C$ be a component
of $U$, then by Theorem 25.3 $C$ is open in $X$. Let $V\subset C$
be a basic open set.
\end{proof}
\begin{problem}
Let $X$ be a connected space and let $f\colon X\to Y$ be a
function which is continuous and onto. Prove that $Y$ is
connected. (This is a theorem in Munkres---prove it from the
definitions).
\end{problem}
\begin{proof}
\end{proof}
\begin{problem}
Given:
\begin{enumerate}[noitemsep,label=(\roman*)]
\item $p\colon X\to Y$ is a quotient map.
\item $Y$ is connected.
\item For every $y\in Y$, the set $p^{-1}(y)$ is connected.
\end{enumerate}
Prove that $X$ is connected.
\end{problem}
\begin{proof}
\end{proof}
\begin{problem}
Let $A$ be a subset of $\RR^2$ which is homeomorphic to the open
unit interval $(0,1)$. Prove that $A$ does not contain a nonempty
set which is open in $\RR^2$.
\end{problem}
\begin{proof}
\end{proof}
\begin{problem}
Let $X$ be a connected space. Let $\mathcal{U}$ be an open
covering of $X$ and let $U$ be a nonempty set in
$\mathcal{U}$. Say that a set $V$ in $\mathcal{U}$ is
\emph{reachable from $U$} if there is a sequence
$U=U_1,U_2,...,U_n=V$ of sets in $\mathcal{U}$ such that $U_i\cap
U_{i+1}\neq\emptyset$ for each $i$ from $1$ to $n-1$. Prove that
every nonempty $V$ in $\mathcal{U}$ is reachable from $U$.
\end{problem}
\begin{proof}
\end{proof}
\begin{problem}
Suppose that $X$ is connected and every point of $X$ has a
path-connected open neighborhood. Prove that $X$ is
path-connected.
\end{problem}
\begin{proof}
\end{proof}
\begin{problem}
Let $X$ be a topological space and let $f,g\colon X\to[0,1]$ be
continuous functions. Suppose that $X$ is connected and $f$ is
onto. Prove that there must be a point $x\in X$ with
$f(x)=g(x)$.
\end{problem}
\begin{proof}
\end{proof}

%%% Local Variables:
%%% mode: latex
%%% TeX-master: "../MA571-Midterm-Current"
%%% End:
