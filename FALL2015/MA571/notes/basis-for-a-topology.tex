\chapter{Topological Spaces and Continuous Functions}

Definitions and important results taken from Munkres's
\emph{Topology}, Chapter 2 \S 12, starting at page 75. Unless
otherwise stated, all definitions, theorems, corollaries, lemmas
etc. are taken from Munkres's \emph{Topology} as are most of the
problems in this set of notes.
\section{Topological Spaces}
\begin{definition}
  A \emph{topology} on a set $X$ is a collection $\mathcal{T}$ of
  subsets of $X$ having the following properties:
  \begin{enumerate}[(1)]
  \item $\emptyset$ and $X$ are in $\mathcal{T}$.
  \item The union of elements of any subcollection of
    $\mathcal{T}$ is in $\mathcal{T}$.
  \item The intersection of the elements of any finite
    subcollection of $\mathcal{T}$ is in
    $\mathcal{T}$.
  \end{enumerate}
  A set $X$ for which a topology $\mathcal{T}$ has been specified
  is called a \emph{topological space}.
\end{definition}

If $X$ is a topological space with topology $\mathcal{T}$, we say
that a subset $X$ of $U$ is an \emph{open set} if $U$ belongs to
the collection $\mathcal{T}$.

\begin{example}
  If $X$ is any set, the collection of \emph{all} subsets of $X$
  is a topology on $X$; it is called the \emph{discrete
    topology}. The collection consisting of $X$ and $\emptyset$
  only is also a topology on $X$; we shall call it the
  \emph{indiscrete topology}.
\end{example}
\begin{example}
  Let $X$ be a set; let $\mathcal{T}_f$ be the collection of
  subsets of $U$ such that $X-U$ either is finite or is all of
  $X$. Then $\mathcal{T}_f$ is a topology on $X$, called the
  \emph{finite complement topology}. Both $X$ and $\emptyset$ are
  in $\mathcal{T}_f$, since $X-X$ is finite and $X-\emptyset$ is
  all of $X$. If $\{U_\alpha\}$ is an indexed family of nonempty
  elements of $\mathcal{T}_f$, to show $\bigcup U_\alpha$ is in
  $\mathcal{T}_f$, we compute
  \[X-\bigcup U_\alpha=\bigcup \left(X-U_\alpha\right).\]
  The latter set is finite because each set $X-U_\alpha$ is
  finite. If $U_1,...,U_n$ are nonempty elements of
  $\mathcal{T}_f$, to show that $\bigcup U_i$ is in
  $\mathcal{T}_f$, we compute
  \[X-\bigcup_{i=1}^nU_i=\bigcup_{i=1}^n\left(X-U_i\right).\]
  The latter set is a finite union of finite sets, and therefore,
  finite.
\end{example}
\begin{definition}
  Suppose that $\mathcal{T}$ and $\mathcal{T}'$ are two
  topologies on a given set. If $\mathcal{T}'\supset\mathcal{T}$,
  we say that $\mathcal{T}'$ is \emph{finer} that $\mathcal{T}$;
  if $\mathcal{T}'$ properly contains $\mathcal{T}$, we say that
  $\mathcal{T}$ \emph{is strictly finer that} $\mathcal{T}'$. We
  also say that $\mathcal{T}$ is \emph{coarser} that
  $\mathcal{T}'$, or \emph{strictly coarser}, in these two
  respective situations. We say $\mathcal{T}$ is
  \emph{comparable} with $\mathcal{T}'$ either
  $\mathcal{T}'\supset\mathcal{T}$ or
  $\mathcal{T}\supset\mathcal{T}$.
\end{definition}
\begin{remarks}
  Other terminology is sometimes used for this concept. If
  $\mathcal{T}'\supset\mathcal{T}$, some mathematicians would say
  $\mathcal{T}'$ is \emph{larger} that $\mathcal{T}$, and
  $\mathcal{T}$ is \emph{smaller} that $\mathcal{T}'$.
\end{remarks}
\section{Basis for a Topology}
\begin{definition}
  If $X$ is a set, a \emph{basis} for a topology on $X$ is a
  collection $\mathcal{B}$ of subsets of $X$ (called \emph{basis
    elements}) such that
  \begin{enumerate}[(1)]
  \item For each $x\in X$, there is at least one basis element
    $B$ containing $x$.
  \item If $x$ belongs to the intersection of two basis elements
    $B_1$ and $B_2$, then there is a basis element $B_3$
    containing $x$ such that $B_3\subset B_1\cap B_2$.
  \end{enumerate}
  If $\mathcal{B}$ satisfies these two conditions, then we define
  the \emph{topology $\mathcal{T}$ generated by $\mathcal{B}$} as
  follows: A subset $U$ of $X$ is said to be open in $X$ (that
  is, to be an element of $\mathcal{T}$) if for each $x\in U$,
  there is a basis element $B\in\mathcal{B}$ such that $x\in B$
  and $B\subset U$. Note that each basis element is itself an
  element of $\mathcal{T}$.
\end{definition}

%%% Local Variables:
%%% mode: latex
%%% TeX-master: "../MA571-HW-Sets-FALL14"
%%% End:
