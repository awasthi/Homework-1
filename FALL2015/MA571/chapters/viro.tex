\chapter{Топология}
\section{Топология в множестве}
Пусть $X$ -- некоторое множество. Рассмотрим набор $\Omega$ его
поднмножеств, для которого:
\begin{enumerate}[noitemsep,label=(\arabic*)]
\item объединение любого смейства множеств, принадлежащих
  совокупности $\Omega$, также принадлежит совокупности $\Omega$;
\item пересечение любого конечного смейства множеств, принадлежит
  совокупности $\Omega$, также принадлежит совопукности $\Omega$;
\item пустое множество $\emptyset$ и всё множество $X$
  принадлежат $\Omega$.
\end{enumerate}
В таком случае -- $\Omega$ есть \emph{топологическая структура}
или просто \emph{топология} в множестве $X$.



%%% Local Variables:
%%% mode: latex
%%% TeX-master: "../Topology-Gamelin-Green-Probs"
%%% End:
