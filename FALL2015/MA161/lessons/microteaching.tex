\section{Indeterminate Forms and L'Hospital's Rule}
Limit of the form
\[
\lim_{x\to a}\frac{f(x)}{g(x)}
\]
where both $f(x)\to 0$ and $g(x)\to 0$ as $x\to a$ is called an
\emph{indeterminate form of type $\frac{0}{0}$}.
\begin{theorem}[L'Hospital's Rule]
Suppose $f$ and $g$ are differentiable and $g'(x)\neq 0$ on an
open interval $I$ that contains $a$ (except possibly at
$a$). Suppose that
\[
\lim_{x\to a}f(x)=0\qquad\text{and}\qquad\lim_{x\to a}g(x)=0
\]
or that
\[
\lim_{x\to a}f(x)=\pm\infty\qquad\text{and}\qquad\lim_{x\to a}g(x)=\pm\infty.
\]
(In other words, we have an indeterminate form of type
$\frac{0}{0}$ or $\frac{\infty}{\infty}$.) Then
\[
\lim_{x\to a}\frac{f(x)}{g(x)}=\lim_{x\to a}\frac{f'(x)}{g'(x)}
\]
if the limit on the right side exists (or is $\infty$ or $-\infty$).
\end{theorem}
\subsection{Indeterminate Products}
Limit of the form
\[
\lim_{x\to a}[f(x)g(x)]
\]
where $f(x)\to 0$ and $g(x)\to \pm\infty$ as $x\to a$ is called an
\emph{indeterminate form of type $0\cdot\infty$}. We can deal
with it by writing the product $fg$ as a quotient:
\[
fg=\frac{f}{1/g}\qquad\text{or}\qquad fg=\frac{g}{1/f}.
\]
\subsection{Indeterminate Differences}
Limit of the form
\[
\lim_{x\to a}[f(x)-g(x)]
\]
where $f(x)\to \infty$ and $g(x)\to\infty$ as $x\to a$ is called an
\emph{indeterminate form of type $\infty-\infty$}. Try to convert
the difference into a quotient (e.g., by using a common
denominator, or rationalization, or factoring out a common
factor) so that we have an indeterminate form of type
$\frac{0}{0}$ or $\frac{\infty}{\infty}$.
\subsection{Indeterminate Powers}
Several indeterminate forms arise from the limit
\[
\lim_{x\to a}[f(x)]^{g(x)}.
\]
\begin{enumerate}[noitemsep,label=\arabic*.]
\item $\lim_{x\to a}f(x)=0$ and $\lim_{x\to a}g(x)=0$ type $0^0$.
\item $\lim_{x\to a}f(x)=\infty$ and $\lim_{x\to a}g(x)=0$ type
  $\infty^0$.
\item $\lim_{x\to a}f(x)=1$ and $\lim_{x\to a}g(x)=\pm\infty$
  type $1^\infty$.
\end{enumerate}
Each of these three cases can be treated by taking the natural
logarithm: let $y=[f(x)]^{g(x)}$, then
\[
\ln y=g(x)\ln f(x)
\]
or by writing the function as an exponential:
\[
[f(x)]^{g(x)}=e^{g(x)\ln f(x)}
\]
In either method we are led to the indeterminate product $g(x)\ln
f(x)$, which is of type $0\cdot\infty$.
\subsection{Exercises}
\begin{exercise*}[\S4.4, \#11]
\[
\lim_{x\to(\pi/2)^+}\frac{\cos x}{1-\sin x}.
\]
\end{exercise*}
\begin{proof}[Solution]
Put $f(x)=\cos x$ and $g(x)=1-\sin x$ and note that
\[
\lim_{x\to(\pi/2)^+}\cos x=0
\quad\text{and}\quad
\lim_{x\to(\pi/2)^+}1-\sin x=0.\]
So we have
\begin{itemize}[noitemsep]
\item Classify: type $\frac{0}{0}$.
\item Check conditions for using l'Hospital's rules are
  satisfied: both $f'$ and $g'$ exist and they are
  \[
    f'(x)=-\sin x\qquad\text{and}\qquad g'(x)=-\cos x.
  \]
Moreover $g'(x)\neq 0$ on $(0,\pi)$, in particular, $g'(0)=-1$
and $g'(\pi)=1$.
\item Use l'Hospital's rule:
\[
\lim_{x\to(\pi/2)^+}\frac{\cos x}{1-\sin
  x}=\lim_{x\to(\pi/2)^+}\frac{-\sin x}{-\cos
  x}=\lim_{x\to(\pi/2)^+}\tan x=0.
\]
\end{itemize}
\end{proof}
\begin{exercise*}[\S4.4, \#12]
\[
\lim_{x\to 0}\frac{\sin 4x}{\tan 5x}.
\]
\end{exercise*}
\begin{proof}[Solution]
Put $f(x)=\sin 4x$ and $g(x)=\tan 5x$ and note that
\[
\lim_{x\to 0}f(x)=0
\quad\text{and}\quad
\lim_{x\to 0}g(x)=0.\]
So we have
\begin{itemize}[noitemsep]
\item Classify: type $\frac{0}{0}$.
\item Check conditions for using l'Hospital's rules are
  satisfied: both $f'$ and $g'$ exist and they are
  \[
    f'(x)=4\cos 4x\qquad\text{and}\qquad g'(x)=5\sec^2 5x.
  \]
Moreover $g'(x)\neq 0$ on $(-\pi/2,\pi/2)$.
\item Use l'Hospital's rule:
\[
\lim_{x\to 0}\frac{\sin 4x}{\tan 5x}
=\lim_{x\to 0}\frac{4\cos 4x}{5\sec^2 5x}
=\frac{4}{5}\lim_{x\to 0}\frac{\cos 4x}{\sec^2 5x}
=\frac{4}{5}
\]
\end{itemize}
\end{proof}
\begin{exercise*}[\S4.4, \#25]
\[
\lim_{x\to 0}\frac{e^x-1-x}{x^2}.
\]
\end{exercise*}
\begin{proof}[Solution]
Put $f(x)=e^x-1-x$ and $g(x)=x^2$ and note that
\[
\lim_{x\to 0}f(x)=0
\quad\text{and}\quad
\lim_{x\to 0}g(x)=0.\]
So we have
\begin{itemize}[noitemsep]
\item Classify: type $\frac{0}{0}$.
\item Check conditions for using l'Hospital's rules are
  satisfied: both $f'$ and $g'$ exist and they are
  \[
    f'(x)=e^x-1\qquad\text{and}\qquad g'(x)=2x.
  \]
Moreover $g'(x)\neq 0$ on $(-\pi/2,\pi/2)$.
\item $f'(x)/g'(x)$ is type $\frac{0}{0}$ so we apply
  L'Hospital's Rule again.
\item Both $f''$ and $g''$ exist and they are
  \[
    f'(x)=e^x\qquad\text{and}\qquad g''(x)=2
  \]
\item Use l'Hospital's rule:
\[
\lim_{x\to 0}\frac{e^x-1-x}{x^2}=\lim_{x\to 0}\frac{e^x}{2}=\frac{1}{2}.
\]
\end{itemize}
\end{proof}
\begin{exercise*}[\S4.4, \#30]
\[
\lim_{x\to\infty}\frac{(\ln x)^2}{x}.
\]
\end{exercise*}
\begin{proof}[Solution]
Put $f(x)=(\ln x)^2$ and $g(x)=x$ and note that
\[
\lim_{x\to 0}f(x)=\infty
\quad\text{and}\quad
\lim_{x\to 0}g(x)=\infty.\]
So we have
\begin{itemize}[noitemsep]
\item Classify: type $\frac{\infty}{\infty}$.
\item Check conditions for using l'Hospital's rules are
  satisfied: both $f'$ and $g'$ exist and they are
  \[
    f'(x)=\frac{2}{x}\ln x\qquad\text{and}\qquad g'(x)=1.
  \]
\item Quotient form, apply l'Hospital's Rule again:
\[
 F(x)=2\ln x\qquad\text{and}\qquad G(x)=x,
\]
then
\[
 F'(x)=2/x\qquad\text{and}\qquad G'(x)=1.
\]
\item Use l'Hospital's rule:
\[
\lim_{x\to\infty}\frac{(\ln
  x)^2}{x}=\lim_{x\to\infty}\frac{2/x}{1}=2\lim_{x\to\infty}\frac{1}{x}=2\cdot
0=0.
\]
\end{itemize}
\end{proof}
\begin{exercise*}[\S4.4, \#33]
\[
\lim_{x\to 1}\frac{x+\sin x}{x+\cos x}.
\]
\end{exercise*}
\begin{proof}[Solution]
Put $f(x)=x+\sin x$ and $g(x)=x+\cos x$ and note that
\[
\lim_{x\to 0}f(x)=
\quad\text{and}\quad
\lim_{x\to 0}g(x)=\infty.\]
So we have
\begin{itemize}[noitemsep]
\item Classify: type $\frac{\infty}{\infty}$.
\item Check conditions for using l'Hospital's rules are
  satisfied: both $f'$ and $g'$ exist and they are
  \[
    f'(x)=\frac{2}{x}\ln x\qquad\text{and}\qquad g'(x)=1.
  \]
\item Quotient form, apply l'Hospital's Rule again:
\[
 F(x)=2\ln x\qquad\text{and}\qquad G(x)=x,
\]
then
\[
 F'(x)=2/x\qquad\text{and}\qquad G'(x)=1.
\]
\item Use l'Hospital's rule:
\[
\lim_{x\to\infty}\frac{(\ln
  x)^2}{x}=\lim_{x\to\infty}\frac{2/x}{1}=2\lim_{x\to\infty}\frac{1}{x}=2\cdot
0=0.
\]
\end{proof}
\begin{exercise*}[\S4.4, \#43]
\[
\lim_{x\to 0}
\]
\end{exercise*}
\begin{proof}[Solution]
\end{proof}
\begin{exercise*}[\S4.4., \#50]
\[\lim_{x\to 0}\csc x\sec 5x.\]
\end{exercise*}
\begin{proof}[Solution]
\end{proof}
\begin{exercise*}[\S4.4, \#57]
\[\lim_{x\to 0}(1-2x)^{1/x}.\]
\end{exercise*}
\begin{proof}[Solution]
\end{proof}
\begin{exercise*}[\S4.4, \#61]
\[\lim_{x\to\infty} x^{1/x}.\]
\end{exercise*}
\begin{proof}[Solution]
\end{proof}

%%% Local Variables:
%%% mode: latex
%%% TeX-master: "../Lesson-Plan-Current"
%%% End:
