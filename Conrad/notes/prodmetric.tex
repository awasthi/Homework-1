\section{Product topology}
The aim of this handout is to address two points: metrizability of finite
products of metric spaces, and the abstract characterization of the product
topology in terms of the universal mapping properties among topological
spaces. This latter issue is related to explaining0 why the definition of
the product topology is not merely \emph{ad hoc} but in a sense the
``right'' definition. In particular, when you study topology more
systematically and encounter the problem of topologizing infinite products
of topological spaces, if you think in terms of the universal property to
be discussed below then you will be inexorably led to the right
definition of the product topology for a product of infinitely many
topological spaces (it is not what one would naively expect to be, based on
experience with the case of finite products).

\subsection{Metrics of finite products}
Let \(X_1,\dotsc,X_d\) be metrizable topological spaces. The product set
\[
  X=\prod_{i=1}^d X_i
\]
admits a natural product topology. It is natural to ask if, upon choosing
metrics \(\rho_j\) inducing the given topology on each \(X_j\), we can
define a metric \(\rho\) on \(X\) in terms of \(\rho_j\) such that \(\rho\)
induces the product topology on \(X\). The basic idea is to find a metric
which describes the idea of ``coordinate-wise closedness'', but several
natural candidates leap out, none of which are evidently better than others
\begin{align*}
  \rho^{\text{max}}\bigl((x_1,\dotsc,x_d),(x_1',\dotsc,x_d')\bigr)
  &=\max_{\mathclap{1\leq j\leq d}}\rho_j(x_j,x_j')\\
  \rho^{\text{Euc}}\bigl((x_1,\dotsc,x_d),(x_1',\dotsc,x_d')\bigr)
  &=\sqrt{\sum_{i=1}^d\rho_j(x_j,x_j')^2}\\
  \rho_1\bigl((x_1,\dotsc,x_d),(x_1',\dotsc,x_d')\bigr)
  &=\sum_{i=1}^d\rho_j(x_j,x_j')\\
  \rho_p\bigl((x_1,\dotsc,x_d),(x_1',\dotsc,x_d')\bigr)
  &=\left[\sum_{j=1}^d\rho_j(x_j,x_j')^p\right]^{1/p},&p\geq 1.
\end{align*}

%%% Local Variables:
%%% mode: latex
%%% TeX-master: "../Diffgeo"
%%% End:
