\chapter{What is Representation Theory?}
Groups arise in nature as ``sets of symmetries (of an object), which are
closed under composition and under taking inverses''. For example, the
\emph{symmetric group \(S_n\)} is the group of all permutations
(symmetries) of \(\{1,\dotsc,n\}\); the \emph{alternating group \(A_n\)} is
the set of all symmetries preserving the parity of the number of ordered
pairs; the \emph{dihedral group \(D_{2n}\)} is the group of symmetries of
the regular \(n\)-gon in the plane. The \emph{orthogonal group
  \(\Orth(3)\)} is the group of distance-preserving transformations of
Euclidean space which fix the origin. There is also the group of \emph{all}
distance preserving transformations, which includes the translations along
with \(\Orth(3)\).\footnote{This group is isomorphic to the
  \emph{semi-direct product \(\Orth(3)\ltimes\bbR^3\)}.}

The official definition is of course more abstract, a group is a set \(G\)
with a binary operation \(*\) which is associative, has a unit element
\(e\) and for which inverses exist. Associativity allows a convenient abuse
of notation, where we write \(gh\) for \(g*h\); we have \(ghk=(gh)k=g(hk)\)
and parentheses are unnecessary. I will often write \(1\) for \(e\), but
this is dangerous on rare occasions, such that when studying the group
\(\bbZ\) under addition; in that case, \(e=0\).

The abstract definition notwithstanding, the interesting situation involves
a group ``acting'' on a set. Formally, an action of a group \(G\) on a set
\(X\) is an ``action map'' \(a\colon G\times X\to X\) which is
\emph{compatible with the group law}, in the sense that
\begin{align*}
  a\bigl(h,a(g,x)\bigr)&=a(hg,x)\\
  a(e,x)&=x.
\end{align*}
This justifies the abuse of notation \(a(g,x)=gx\), for we have
\(h(gx)=(hg)x\).

From this point of view, geometry asks, ``Given a geometric object \(X\),
what is its group of symetries?'' Representation theory reverses the
quostion to ``Given a group \(G\), what objects \(X\) does it act on?'' and
attempts to answer this question by classifying such \(X\) up to
isomorphism.

Before restricting to the linear case, our main concern, let us remember
another way to describe an action of \(G\) on \(X\). Every \(g\in G\)
defines a map \(a(g)\colon X\to X\) by \(x\mapsto gx\). This map is a
bijection, with inverse map \(a(g^{-1})\): indeed, \(\bigl(a(g^{-1})\circ
a(g)\bigr)(x)=g^{-1}gx=ex=x\) from the properties of the action. Hence
\(a(g)\) belongs to the set \(\Sym X\) of bijective self-maps of
\(X\). This set forms a group under composition, and the properties of an
action imply that
\begin{proposition}
  An action of \(G\) on \(X\) ``is the same as'' a group homomorphism
  \(\alpha\colon G\to\Sym X\).
\end{proposition}

The formulation of Prop.\@ 1.1 leads to the following observation. For any
action \(a\) of \(H\) on \(X\) and group homomorphism
\(\varphi\colon G\to H\), there is defined a \emph{restricted} or
\emph{pulled-back} action \(\varphi^*a\) of \(G\) on \(X\), as
\(\varphi^*a=a\circ\varphi\). In the original definition, the action sends
\((g,x)\) to \(\varphi(g)(x)\).

\begin{example}[Tautological action of \(\Sym X\) on \(X\)]
  This is the obvious action, call it \(T\), sending \(f,x\) to \(f(x)\),
  where \(f\colon X\to X\) is a bijection and \(x\in X\). In this language,
  the action \(a\) of \(G\) on \(X\) is \(\alpha^*T\) with the homomorphism
  \(\alpha\) of the proposition -- the pull-back under \(\alpha\) of the
  tautological action.
\end{example}

\begin{example}[Linearity]
  The question of classifying all possible \(X\) with action of \(G\) is
  hopeless in such generality, but one should recall that, in first
  approximation, mathematics is linear. So we shall take our \(X\) to be
  a\emph{vector space} over some ground \emph{field}, and ask that the
  action of \(G\) be linear, as well, in other words, that it should
  preserve the vector space structure. Our interest is mostly confined to
  the case when the field of scalars is \(\bbC\), although we shall
  occasionally mention how the picture changes when other fields are
  studied.
\end{example}

\begin{definition}
  A linear representation \(\rho\) of \(G\) on a complex vector space \(V\)
  is a set-theoretic action on \(V\) which preserves the linear structure,
  i.e.,
  \begin{align*}
    \rho(g)(\bfv_1+\bfv_2)&=\rho(g)\bfv_1+\rho(g)\bfv_2,&&\text{for all
                                                          \(\bfv_1,\bfv_2\in
                                                          V\)},\\
    \rho(g)(k\bfv)&=k\rho(g)\bfv&&\text{for all \(k\in\bbC\), \(\bfv\in V\).}
  \end{align*}
  Unless otherwise mentioned, a \emph{representation} will mean a
  \emph{finite-dimensional complex representation}.
\end{definition}

\begin{example}[The general linear group]
  Let \(V\) be a complex vector space of dimension \(n<\infty\). After
  choosing a basis, we can identify it with \(\bbC^n\), although we shal
  lavoid doing so without good reason. Recall that the \emph{endomorphism
    algebra \(\End V\)} is the set of all linear maps (or \emph{operators})
  \(L\colon V\to V\), with the natural addition of linear maps and the
  composition as multiplication. If \(V\) has been identified with
  \(\bbC^n\), a linear map is uniquely representable by a matrix, and the
  addition of linear maps becomes the entrywise addition, while the
  composition becomes the matrix multiplication.

  Inside \(\End V\), there is contained the group \(\GL V\) of invertible
  linear operators; the group operation, of course, is composition.
\end{example}

%%% Local Variables:
%%% mode: latex
%%% TeX-master: "../RepTheo"
%%% End:
