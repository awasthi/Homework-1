\section{Transendence}
\subsection{Transendence bases}
Let \(K\) be an extension field of a field \(k\). Let \(S\) be a subset of
\(K\). We recall that \(S\) (or the elements of \(S\)) are said to be
algebraically independent over \(k\), if whenever we have arelation
\[
  0=\sum a_{(v)}M_{(v)}(S)=\sum a_{(v)}\prod_{x\in S}X^{v(X)}
\]
with coefficients \(a_{(v)}\in k\), almost all \(a_{(v)}=0\), then we must
necessarily have \(a_{(v)}=0\).

We can introduce an ordering among algebraically independent subsets of
\(K\), by ascending inclusion. These subsets are obviously inductively
ordered, and thus there exists maximal elements. If \(S\) is a subset of
\(K\) which is algebraically independent over \(k\), and if the Cardinality
of \(S\) is greatest among all such subsets, then we call this cardinality
the \emph{transendence degree or dimension} of \(K\) over \(k\). Actually,
we shall need to distinguish only between finite transendence degree or
infinite transendence degree.

%%% Local Variables:
%%% mode: latex
%%% TeX-master: "../Diffgeo"
%%% End:
