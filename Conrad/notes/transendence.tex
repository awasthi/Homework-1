\section{Transcendence}
\subsection{Transcendence bases}
Let \(K\) be an extension field of a field \(k\). Let \(S\) be a subset of
\(K\). We recall that \(S\) (or the elements of \(S\)) are said to be
algebraically independent over \(k\), if whenever we have arelation
\[
  0=\sum a_{(v)}M_{(v)}(S)=\sum a_{(v)}\prod_{x\in S}X^{v(X)}
\]
with coefficients \(a_{(v)}\in k\), almost all \(a_{(v)}=0\), then we must
necessarily have \(a_{(v)}=0\).

We can introduce an ordering among algebraically independent subsets of
\(K\), by ascending inclusion. These subsets are obviously inductively
ordered, and thus there exists maximal elements. If \(S\) is a subset of
\(K\) which is algebraically independent over \(k\), and if the cardinality
of \(S\) is greatest among all such subsets, then we call this cardinality
the \emph{transcendence degree or dimension} of \(K\) over \(k\). Actually,
we shall need to distinguish only between finite transcendence degree or
infinite transcendence degree. We observe that the notion of transcendence
degree bears to the notion of algebraic independence the same relation as
the notion of dimension bears to the notion of independence.

We frequently deal with familiar elements of \(K\), say a family
\(\{x_i:i\in I\}\), and say that such a family is algebraically independent
over \(k\) if its elements are distinct (in other words, \(x_i\neq x_j\) if
\(i\neq j\)) and if the set consisting of the elements in this family is
algebraically independent over \(k\).

A subset \(S\) of \(K\) which is algebraically independent over \(k\) and
is maximal with respect to the inclusion ordering will be called a
\emph{transcendence base} of \(K\) over \(k\). From the maximality, it is
clear that if \(S\) is a transcendence base of \(K\) over \(k\), then \(K\)
is algebraic over \(k(S)\).

\begin{theorem}
  Let \(K\) be an extension of a field \(k\). Any two transcendence bases of
  \(K\) over \(k\) have the same cardinality. If \(\Gamma\) is a set of
  generators of \(K\) over \(k\) (i.e., \(K=k(\Gamma)\)) and \(S\) is a
  subset of \(\Gamma\) which is algebraically independent over \(k\), then
  there exists a transcendence base \(\calB\) of \(K\) over \(k\) such that
  \(S\subseteq\calB\subseteq\Gamma\).
\end{theorem}
\begin{proof}
  We shall prove that if there exists one finite transcendence base, say
  \(\{x_1,\dotsc,x_m\}\), \(m\geq 1\), then any other transcendence must
  also have \(m\) elements. For this it will suffice to prove: If
  \(w_1,\dotsc,w_n\) are elements of \(K\) which are algebraically
  independent over \(k\) then \(n\geq m\) (for we can use symmetry). By
  assumption, there exists a nonzero polynomial \(f_1\) in \(m+1\)
  variables with coefficients in \(k\) such that
  \[
    f_1(w_1,x_1,\dotsc,x_m)=0.
  \]
  Furthermore, by hypothesis,  \(w_1\) occurs in \(f_1\), and some \(x_i\)
  also occurs in \(f_1\), say \(x_1\). Then \(x_1\) is algebraic over
  \(k(w_1,x_2,\dotsc,x_m)\). Suppose inductively that after a suitable
  renumbering of \(x_2,\dotsc,x_m\) we have found \(w_1,\dotsc,w_r\)
  \((r<n)\) such that \(K\) is algebraic over
  \[
    k(w_1,\dotsc,w_r,x_{r+1},\dotsc,x_m).
  \]
  Then there exists a nonzero polynomial \(f\) in \(m+1\) variables with
  coefficients in \(k\) such that
  \[
    f(w_{r+1},w_1,\dotsc,w_r,x_{r+1},\dotsc,x_m)=0,
  \]
  and such that \(w_{r+1}\) actually occurs in \(f\).
\end{proof}

  %%% Local Variables:
  %%% mode: latex
  %%% TeX-master: "../Diffgeo"
  %%% End:
