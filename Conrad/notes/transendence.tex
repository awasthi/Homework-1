\section{Transendence}
\subsection{Transendence bases}
Let \(K\) be an extension field of a field \(k\). Let \(S\) be a subset of
\(K\). We recall that \(S\) (or the elements of \(S\)) are said to be
algebraically independent over \(k\), if whenever we have arelation
\[
  0=\sum a_{(v)}M_{(v)}(S)=\sum a_{(v)}\prod_{x\in S}X^{v(X)}
\]
with coefficients \(a_{(v)}\in k\), almost all \(a_{(v)}=0\), then we must
necessarily have \(a_{(v)}=0\).

We can introduce an ordering among algebraically independent subsets of
\(K\), by ascending inclusion. These subsets are obviously inductively
ordered, and thus there exists maximal elements. If \(S\) is a subset of
\(K\) which is algebraically independent over \(k\), and if the cardinality
of \(S\) is greatest among all such subsets, then we call this cardinality
the \emph{transendence degree or dimension} of \(K\) over \(k\). Actually,
we shall need to distinguish only between finite transendence degree or
infinite transendence degree. We observe that the notion of transendence
degree bears to the notion of algebraic independence the same relation as
the notion of dimension bears to the notion of independence.

We frequently deal with familiar elements of \(K\), say a family
\(\{x_i:i\in I\}\), and say that such a family is algebraically independent
over \(k\) if its elements are distinct (in other words, \(x_i\neq x_j\) if
\(i\neq j\)) and if the set consisting of the elements in this family is
algebraically independent over \(k\).

A subset \(S\) of \(K\) which is algebraically independent over \(k\) and
is maximal with respect to the inclusion ordering will be called a
\emph{transendence base} of \(K\) over \(k\). From the maximality, it is
clear that if \(S\) is a transendence base of \(K\) over \(k\), then \(K\)
is algebraic over \(k(S)\).

%%% Local Variables:
%%% mode: latex
%%% TeX-master: "../Diffgeo"
%%% End:
