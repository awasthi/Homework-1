\chapter{Algebraic geometry}
\section{The Nötherian condition}
Let \(R\) be a commutative ring. In class we saw that the ascending chain
condition on ideals of \(R\) is equivalent to the condition that all ideals
in \(R\) are finitely generated. The aim of this handout is to prove the
equivalence of these two conditions with yet another ``finiteness''
condition, on the module theory of \(R\):
\begin{theorem}
  If all ideals of \(R\) are finitely generated then every submodule of a
  finitely generated \(R\)-module is also finitely generated.
\end{theorem}

In class we noted that the converse of this theorem is true (and
elementary), since ideals of \(R\) are precisely submodules of the module
\(M=R\cdot 1\). The subtlety with the implication in the theorem is that if
we try to induct on the number of generators of \(M\) then we run into the
problem that this does \emph{not} control the number of generators needed
for submodules of \(M\), in contrast with the case of p.i.d.s. (For
example, the ideal \((X,Y)\) in the ring \(R=k[X,Y]\) does not have a
single generator as an \(R\)-module, even though \(M=R\) does have a single
generator.) For this reason, we will not argue by induction on the number
of generators. We will argue by induction in another way.

\begin{proof}
  Let \(M\) be a finitely generated \(R\)-module, say with \(n\)
  generators. That is, there is a surjective map \(q\colon
  R^n\twoheadrightarrow M\). Hence, any submodule \(M'\subseteq M\) is the
  image of a submodule \(q^{-1}(M')\subseteq R^n\), so to show that \(M'\)
  is finitely generated it suffices to prove that \(q^{-1}(M')\) is
  finitely generated. In other words, we are reduced to proving finite
  generation for submodules of each \(R^n\). \emph{This} we will prove by
  induction on \(n\). The case \(n=1\) is precisely our hypothesis that all
  ideals of \(R\) are finitely generated!

  Now assume \(n>1\) and that the case \(n-1\) is known.
  \[
    (r_1,\dotsc,r_{n-1})\mapstoo (r_1,\dotsc,r_{n-1},0)
  \]
  so \(R^n/(R^{n-1})=R\) via projection to the \(n\)-th coordinate. For a
  submodule \(N\subseteq R^n\), we get a submodule \(N'=N\cap
  R^{n-1}\subseteq R^{n-1}\) and a quotient
  \[
    N''=\frac{N}{N'}\subseteq\frac{R^n}{R^{n-1}}\simeq R.
  \]
  Thus, by induction \(N'\) is finitely generated, and byt the base case
  \(N''\) is finitely generated (we have even identified with an ideal of
  \(R\)), so we have reached the following situation: we have a module
  \(N\) over \(R\) and a finitely generated submodule \(N'\) such that the
  quotient \(N/N'\) is also finitely generated. Then we claim that \(N\) is
  finitely generated.

  Explicitly, suppose \(\{e_1',\dotsc,e_m'\}\) is a generating set of
  \(N'\), and let \(\{e_1,\dotsc,e_r\}\) be a subset of \(N\) that lifts a
  generating set of \(N/N'\). Then \(\{e_1',\dotsc,e_m',e_1,\dotsc,e_r\}\)
  is a generating set of \(N\).
\end{proof}

If you think about it, this method of proof is essentially constructs a
finite chain of submodules that the successive quotients in the chain are
each identified with either an ideal of \(R\) or the quotient of \(R\) by
an ideal. If you look back at how one proves the existence of bases of
finite-dimensional vector spaces over a field or the structure theorem for
modules over a p.i.d., you'll see that those earlier arguments are
essentially special cases of the kind of analysis we have just carried out
(except  that in those cases we get more prcise structure theorems for the
modules, so we have to do some more work which is not relevant over general
Noetherian \(R\)).

%%% Local Variables:
%%% mode: latex
%%% TeX-master: "../Diffgeo"
%%% End:
