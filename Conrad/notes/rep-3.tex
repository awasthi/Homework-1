\chapter{Complete Reducibility and Unitarity}
In the homework, you find an example of a complex representation of the
group \(\bbZ\) which is not completely reducible, and also of a
representation of the cyclic group of prime order \(p\) over the finite
field \(\bbF_p\) which is not completely reducible. This underlines the
importance of the following complete reducibility theorem for finite
groups.

\begin{theorem}
  Every complex representation of a finite group is completely reducible.
\end{theorem}

The theorem is so important that we shall give two proofs. The first uses
inner products and so applies only to \(\bbR\) or \(\bbC\), but generalizes
to \emph{compact groups}. The more algebraic proof, on the other hand,
extends to any field of scalars \emph{whose characteristic does not divide
  the order of the group} (equivalently, the order of the group should not
be \(0\) in the field).

Beautiful as it is, the result would have limited value without some supply
of irreducible representations. It turns out that the following example
provides an adequate supply.

\section{The regular representation}
Let \(\bbC[G]\) be the vector space of complex functions on \(G\). It has a
basis \(\{\bfe_g:g\in G\}\), with \(\bfe_g\) representing the function
equual to \(1\) at \(g\) and \(0\) elsewhere. \(G\) acts on this basis as
follows:
\[
  \lambda(g)(\bfe_h)=\bfe_{gh}.
\]
This set theoretic action extends by linearity to the vector space:
\[
  \lambda(g)\Bigl(\sum\nolimits_{h\in G} v_h\cdot\bfe_h\Bigr)
  =\sum_{h\in G}v_h\cdot\lambda(g)\bfe_h
  =\sum_{h\in G}v_h\cdot\bfe_{gh}.
\]
On coordinates, the action is opposite to what you might expect: namely,
the \(h\)-coordinate of \(\lambda(g)(\bfv)\) is \(v_{g^{-1}h}\). The result
is the \emph{left regular representation} of \(G\). LAter we will decompose
\(\lambda\) into irreducibles, and we shall see that \emph{every}
irreducible isomorphism class of \(G\)representations occur in the
decomposition.

\begin{remark}
  If \(G\) acts on
\end{remark}

%%% Local Variables:
%%% mode: latex
%%% TeX-master: "../RepTheo"
%%% End:
