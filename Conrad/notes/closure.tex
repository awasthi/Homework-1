\section{Interior, closure and boundary}
We wish to develop some basic geometric concepts in metric spaces which
make precise intuitive ideas centered on the themes of ``interior'' and
``boundary'' of a subset of a metric space. One warning must be
given. Although there are a number of results proven in this handout, none
of it is particularly deep. If you carefully study the proofs, then you'll
see that none of this requires going much beyond the basic definitions. We
will certainly encounter some serious ideas and nontrivial proofs in due
course, but at this point the central aim is to acquire some linguistic
ability when discussing some basic geometric ideas in a metric space. Thus,
the main goal is to familiarize ourselves with some very convenient
geometric terminology in terms of which we can discuss more sophisticated
ideas later on.
\subsection{Interior and closure}
Let \(X\) be a metric space and \(A\subseteq X\) a subset. We define the
\emph{interior} of \(A\) to be the set
\[
  A^\circ=\Int A=\bigl\{\,a\in A:\text{there exists \(r>0\) such that
    \(B(a,)\subseteq A\)}\,\bigr\}
\]
consisting of points for which \(A\) is a ``neighborhood''. We define the
\emph{closure}  of \(A\) to be the set
\[
  \bar A=\Cls A=\bigl\{\,x\in X:\text{\(x=\lim_{n\to\infty} a_n\) with
    \(a_n\in A\) for all \(n\in\bbN\)}\,\bigr\}.
\]

In words, the interior consists of points in \(A\) for which all nearby
points of \(X\) are also in \(A\) whereas the closure allows for ``points
on the edge of \(A\)''. Note that obviously
\[
  A^\circ\subseteq\bar A.
\]
We will see shortly (after some examples) that \(A^\circ\) is the largest
open set inside of \(A\) --- that is, it is open and contains any open
lying inside of \(A\) (so in fact \(A\) is open if and only if
\(A=A^\circ\)) --- while \(\bar A\) is the smallest closed set containing
\(A\); i.e., \(\bar A\) is closed and lies inside any closed set containing
\(A\) (so  in fact \(A\) is closed if and only if \(\bar A=A\)).

\emph{Beware that we have to prove that the closure is actually closed!}
Just because we call  something the ``closure'' does not mean the concept
is automatically endowed with linguistically similarly sounding
properties. The proof won't be particularly deep, as we'll see.

\begin{example}
  Let's work out the interior and closure of the ``half-open'' square
  \[
    A=\bigl\{\,(x,y)\in\bbR^2:-1\leq x\leq
    1,-1<y<1\,\bigr\}=[-1,1]\times(-1,1)
  \]
  inside the metric space \(X=\bbR^2\) (the phrase ``half-open'' is purely
  intuitive; it has no precise meaning, but the picture should make it
  clear why we use this terminology). Intuitively, this is a square region
  whose horizontal edges are ``left out''. The interior of \(A\) should be
  \((-1,1)\times(-1,1)\) and the closure should be \([-1,1]\times[-1,1]\),
  as drawing a picture should convince you. Of course, we want to see that
  such conclusions really do follow from our precise definitions.

  First we check that \(A^\circ\) is correctly described. If \(-1<x<1\) and
  \(-1<y<1\) then for
  \[
    r=\min\bigl\{|-1-x|,|1-x|,|-1-y|,|1-y|\bigr\}>0
  \]
  it is easy to check that
  \(B\bigl((x,y),r\bigr)\subseteq(-1,1)\times(-1,1)\) (since a square box
  with side-length \(r\) contains the disc of radius \(r\) with the same
  center). Thus, \((-1,1)\times(-1,1)\subseteq A\) is an open subset of
  \(X=\bbR^2\). To check it is the full interior of \(A\), we just have to
  show that the ``missing points'' of the form \(\pm 1,y\) do not lie in
  the interior. But for any such point \(p=(\pm 1,y)\in A\), for any
  positive small \(r>0\) there is always a point in \(B(p,r)\) with the
  same \(y\)-coordinate but with the \(x\)-coordinate either slightly
  larger than \(1\) or slightly less than \(-1\). Such a point is not in
  \(A\). Thus, \(p\notin A^\circ\).

  Now we check that \(\bar A=[-1,1]\times[-1,1]\). Since convergence in
  \(\bbR^2\) forces convergence in coordinates, to see
  \[
    \bar A\subseteq[-1,1]\times[-1,1]
  \]
  it suffices to check that \([-1,1]\) is closed in \(\bbR\) (since
  certainly \(A\subseteq[-1,1]\times[-1,1]\)). But this is clear (either by
  using sequences or by explicitly showing its complement in \(\bbR\) to be
  open). To see that \(\bar A\) fills up all of \([-1,1]\times[-1,1]\), we
  have to show that each point in \([-1,1]\times[-1,1]\) can be obtained as
  a limit of a sequence in \(A\). We just have to deal with points not in
  \(A=[-1,1]\times(-1,1)\) since points in \(A\) are limits of constant
  sequences. That is, we're faced with studying points of the form \((x,\pm
  1)\) with \(x\in [-1,1]\). Such a point is a limit of a sequence
  \((x,q_n)\) with \(q_n\in(-1,1)\) having limit \(\pm 1\).
\end{example}
\begin{example}
  What happens if we work with the same set \(A\) but view it inside of a
  metric space \(X=A\) (with the Euclidean metric)? In this case,
  \(A^\circ=A\) and \(\bar A=A\)! Indeed, quite generally for any metric
  space \(X\) we have \(X^\circ=X\) and \(\bar X=X\). These are easy
  consequences of thee definitions. Likewise, the empty subset
  \(\emptyset\) in any metric space has interior and closure equal to the
  subset \(\emptyset\).

  The moral is that one has to always keep in mind what ambient metric
  space one is working in when forming interiors and closures! One could
  imagine that perhaps our notation for interior and closure should somehow
  incorporate a designation of the ambient metric space. But just as we
  freely use the same symbols ``\(+\)'' and ``\(0\)'' to denote the
  addition and additive identity in any vector space, even when working
  with several spaces at once, it would simply make life too cumbersome
  (and the notation too cluttered) to always write things like \(\Int_X A\)
  or \(\Cls_X A\). One just has to pay careful attention to what is going
  on so as to keep track of the ambient metric space with respect to which
  one is forming interiors and closures. The context will usually make it
  obvious what one is using as the ambient metric space, though if
  considering several ambient spaces at once it is sometimes helpful to use
  more precise notation such as \(\Int_XA\).
\end{example}
\begin{theorem}
  Let \(A\) be a subset of a metric space \(X\). Then \(A^\circ\) is open
  and is the largest open set of \(X\) inside of \(A\) (i.e., it contains
  all others).
\end{theorem}
\begin{proof}
  We first show that \(A^\circ\) is open. By definition, if \(x\in
  A^\circ\), then some \(B(x,r)\subseteq A\). But then since \(B(x,r)\) is
  itsefl an open set we see that any \(y\in B(x,r)\) has some
  \(B(y,s)\subseteq B(x,r)\subseteq A\), which forces \(y\in
  A^\circ\). That is, we have shown \(B(x,r)\subseteq A^\circ\), whence
  \(A^\circ\) is open.

  If \(U\subseteq A\) is an open set in \(X\), then for each \(u\in U\)
  there is some \(r>0\) such that \(B(u,r)\subseteq U\) whence
  \(B(u,r)\subseteq A\), so \(u\in A^\circ\). This is true for all \(u\in
  U\), so \(U\subseteq A^\circ\).
\end{proof}
\begin{corollary}
  A subset \(A\) in a metric space \(X\) is open if and only if
  \(A=A^\circ\).
\end{corollary}
\begin{proof}
  By the theorem, \(A^\circ\) is the unique largest open subset of \(X\)
  contained in \(A\). But obviously \(A\) is open if anf only if such a
  unique maximal  open subset of \(X\) lying in \(A\) is actually equal to
  \(A\). This establishes the corollary.
\end{proof}

We next want to show that the closure of a subset \(A\) in \(X\) is related
to closed subsets of \(X\) containing \(A\) in a manner very similar to the
way in which the interior of \(A\) is related to open subsets of \(X\)
which lie inside of \(A\). This goes along with the general idea that
openness and closedness are ``complementary'' points to view (recall that a
subset \(S\) in a metric space \(X\) is open (resp.\@ closed) if and only
if its complement \(X\setminus S\) is closed (resp.\@ open)). It is
actually more convenient for us to first show that closures and interiors
have complementary relationship, and to then use this to deduce our desired
properties of closure from already-established properties of interior.

\begin{theorem}
  \label{thm:metric-space-bar}
  Let \(A\) be a subset of a metric space \(X\). Then
  \(X\setminus\bar A=(X\setminus A)^\circ\) and
  \(X\setminus A^\circ=\overline{X\setminus A}\).
\end{theorem}
Before proving this theorem, we illustrate with an example. Consider
\(X=\bbR^2\) with the usual metric, and let \(A=[-1,1]\times(-1,1)\) be the
``half-open'' square as considered above. By drawing pictures of
\(X\setminus A\) and of the complements of \(\bar A\) and \(A^\circ\), you
should convince yourself intuitively that the assertions in this theorem
make sense in this case.

Now we prove Theorem \ref{thm:metric-space-bar}.

\begin{proof}
  We begin by proving \(X\setminus\bar A=(X\setminus A)^\circ\). If \(x\in
  X\) is not in \(\bar A\), there must exist some \(B(x,1/2^n)\) not
  meeting \(A\), for otherwise we'd have some \(x_n\in B(x,1/2^n)\cap A\)
  for all \(n\in\bbN\), so clearly \(x_n\to x\), contrary to the fact that
  \(x\notin\bar A\) is not a limit of a sequence of elements of \(A\). This
  shows
  \[
    X\setminus\bar A\subseteq(X\setminus A)^\circ.
  \]
  Conversely, if \(x\) is in the interior of \(X\setminus A\) then some
  \(B(x,r)\) lies in \(X\setminus A\) and hence is disjoint from \(A\). It
  follows that no sequnece in \(A\) can possibly converge to \(x\) because
  for \(\varepsilon=r>0\) we'd run into problems (i.e., there's nothing in
  \(A\) within a distance of less that \(\varepsilon\) from \(x\), since
  \(B(x,\varepsilon)\subseteq X\setminus A\)).

  Applying the \emph{general} equality
  \[
    X\setminus\bar A=(X\setminus A)^\circ
  \]
  for arbitrary subsets \(A\) to \(X\) to the subset \(X\setminus A\) in
  the role of \(A\), we get
  \[
    X\setminus\overline{X\setminus A}=A^\circ.
  \]
  Taking complements of both sides within \(X\) yields
  \[
    \overline{X\setminus A}=X\setminus A^\circ,
  \]
  as desired.
\end{proof}
\begin{corollary}
  Let \(A\) be a subset of a metric space \(X\). Then \(\bar A\) is closed
  and is contained inside of any closed subset of \(X\) which contains
  \(A\).
\end{corollary}
\begin{proof}
  Since the complement of \(\bar A\) is equal to \((X\setminus A)^\circ\),
  which we know to be open, it follows that \(\bar A\) is closed. If \(Z\)
  is any closed set containing \(A\), we want to prove that \(Z\) contains
  \(\bar A\) (so \(\bar A\) is ``minimal'' among closed sets containing
  \(A\)). But this is clear for several reasons. On the one hand, by
  definition every point \(x\in\bar A\) is the limit of a sequnece of
  elements in \(A\subseteq Z\), so by closedness of \(Z\) such limit points
  \(x\) are also in \(Z\). This shows \(\bar A\subseteq Z\). On the other
  hand, one can argue by noting that passage to complement takes \(Z\) to
  an open set \(X\setminus Z\) contained inside of \(X\setminus A\), so by
  maximality this open \(X\setminus Z\) must lie inside the interior of
  \(X\setminus A\), which we have seen is the complement \(X\setminus\bar
  A\) of \(\bar A\). Passage back to complements then gives
  \[
    \bar A=X\setminus(X\setminus\bar A)=X\setminus(X\setminus
    A)^\circ\subseteq X\setminus(X\setminus Z)\subseteq Z
  \]
  as desired.
\end{proof}
\begin{corollary}
  For subsets \(A_1,\dotsc,A_n\) in a metric space \(X\), the closure of
  \(A_1\cup\dotsb\cup A_n\) is equal to \(\bigcup_{i=1}^n \bar A_i\); that
  is, the formation of a finite union commutes with the formation of a
  closure.
\end{corollary}
\begin{proof}
  A closed set \(Z\) contains \(\bigcup_{i=1}^n A_i\) if and only if it
  contains each \(A_i\), and so if and only if it contains \(\bar A_i\) for
  every \(i\). Since \(\bigcup_{i=1}^n\bar A_i\) is a finite union of
  closed sets, it is closed. We conclude that this closed set is minimal
  among all closed sets containing \(\bigcup_{i=1}^n A_i\), so it is the
  closure of \(\bigcup_{i=1}^n A_i\).
\end{proof}

\subsection{Further aspects of interior and closure}
The ``interior'' and ``closure'' constructions have been seen to be
well-behaved with respect to the formation of complements within a metric
space. However, these notions are not well-behaved with respect
intersections within a metric space. Also, one cannot capture the closure
of a set just from knowing its interior. For example, a set can have empty
interior and yet the closure equal to the whole space: think about the
subset \(\bbQ\) in \(\bbR\).

Here is one mildly positive result.
\begin{theorem}
  The formation of closures is local in the sense that if \(U\) is open in
  a metric space \(X\) and \(A\) is an arbitrary subset of \(X\), then the
  closure of \(A\cap U\) in \(X\) meets \(U\) in \(\bar A\cap U\) (where
  \(\bar A\) denotes the closure of \(A\) in \(X\)). In particular, if
  \(Z\) is closed in \(X\) then \(U\cap\overline{Z\cap U}=Z\cap U\).

  Also if \(U\) is the interior of a closed set \(Z\) in \(X\), then
  \({\bar U}^\circ=U\).
\end{theorem}

After proving the theorem, we'll present an interesting example of an open
subset of a metric space which is \emph{not} equal to the interior of its
closure (and hence, by the second part of the theorem, cannot be expressed
as the interior of any closed set at all). It is probably not immediately
obvious to you how to find such open sets, since typical open sets one
writes down in \(\bbR\) or \(\bbR^2\) tend to be the interior of their
closures.

\begin{proof}
  Since \(\bar A\cap U\) is a closed set in \(U\) that contains
  \(A\cap U\), for the first part of the theorem we need to prove that
  every point \(x\in\bar A\cap U\) is a limit of a sequence of points
  \(x_n\in A\cap U\). Since \(x\in\bar A\) we can write
  \(x=\lim_{n\to\infty}x_n\) with \(x_n\in A\). By hypothesis \(x\in U\),
  so by the openness of \(U\) we must have some \(B(x,r)\subseteq U\), and
  so since \(x_n\to x\) by considering just sufficiently large \(n\) we
  have \(x_n\in U\). Thus, for large \(n\) the sequence \(\{x_n\}\) lies in
  \(A\cap U\) and converges to \(x\).

  Now we assume that \(U\) is the interior of a closed set \(Z\) and we
  wish to prove \(U\) is the interior of \(\bar U\). Since \(Z\) is a
  closed set containing \(U\), it also contains the closure of \(U\), and
  by openness of \(U\) the open subset \(U\) inside of \(\bar U\) must lie
  inside the interior of \(\bar U\). To summarize we have
  \[
    U\subseteq{\bar U}^\circ\subseteq Z^\circ=U,
  \]
  so equality is forced throughout.
\end{proof}

Let's give a counterexample to the equality \({\bar U}^\circ=U\) if one
only requires \(U\) to be an open subset of \(X\) (rather than even the
interior of a closed set). The basic problem is that the closure of \(U\)
can be quite a lot bigger than \(U\). In fact, we'll find a rather
``small'' open subset \(U\subseteq\bbR\) with closure equal to \(\bbR\)
(whose interior is \(\bbR\), and hence larger than \(U\)).

Let \(S\subseteq\bbQ\) denote the set of elements of the form \(q=a/10^n\)
with \(a\in\bbZ\) and \(n\geq 0\) (i.e., finite decimal expansions). We
define \(n(q)\geq 0\) to be the exponent of \(10\) in the denominator of
\(q\). In words, the base \(10\) decimal expansion of \(q\in S\) is finite
and (if \(q\notin 10\bbZ\)) begins on the right with a nonzero digit in the
\(10^{-n(q)}\)-slot. Define \(U\) to be the union of the intervals
\(B_{1/10^{n(q)+2}}(q)\) for \(q\in S\). This union is certainly open, as
it is the union of open intervals. This union \(U\) is certainly open, as
it is the union of open intervals. Try to draw a picture where \(U\)

%%% Local Variables:
%%% mode: latex
%%% TeX-master: "../Diffgeo"
%%% End:
