\section{Interior, closure and boundary}
We wish to develop some basic geometric concepts in metric spaces which
make precise intuitive ideas centered on the themes of ``interior'' and
``boundary'' of a subset of a metric space. One warning must be
given. Although there are a number of results proven in this handout, none
of it is particularly deep. If you carefully study the proofs, then you'll
see that none of this requires going much beyond the basic definitions. We
will certainly encounter some serious ideas and nontrivial proofs in due
course, but at this point the central aim is to acquire some linguistic
ability when discussing some basic geometric ideas in a metric space. Thus,
the main goal is to familiarize ourselves with some very convenient
geometric terminology in terms of which we can discuss more sophisticated
ideas later on.
\subsection{Interior and closure}
Let \(X\) be a metric space and \(A\subseteq X\) a subset. We define the
\emph{interior} of \(A\) to be the set
\[
  A^\circ=\Int A=\bigl\{\,a\in A:\text{there exists \(r>0\) such that
    \(B(a,)\subseteq A\)}\,\bigr\}
\]
consisting of points for which \(A\) is a ``neighborhood''. We define the
\emph{closure}  of \(A\) to be the set
\[
  \bar A=\Cls A=\bigl\{\,x\in X:\text{\(x=\lim_{n\to\infty} a_n\) with
    \(a_n\in A\) for all \(n\in\bbN\)}\,\bigr\}.
\]

In words, the interior consists of points in \(A\) for which all nearby
points of \(X\) are also in \(A\) whereas the closure allows for ``points
on the edge of \(A\)''. Note that obviously
\[
  A^\circ\subseteq\bar A.
\]
We will see shortly (after some examples) that \(A^\circ\) is the largest
open set inside of \(A\) --- that is, it is open and contains any open
lying inside of \(A\) (so in fact \(A\) is open if and only if
\(A=A^\circ\)) --- while \(\bar A\) is the smallest closed set containing
\(A\); i.e., \(\bar A\) is closed and lies inside any closed set containing
\(A\) (so  in fact \(A\) is closed if and only if \(\bar A=A\)).

\emph{Beware that we have to prove that the closure is actually closed!}
Just because we call  something the ``closure'' does not mean the concept
is automatically endowed with linguistically similarly sounding
properties. The proof won't be particularly deep, as we'll see.

\begin{example}
  Let's work out the interior and closure of the ``half-open'' square
  \[
    A=\bigl\{\,(x,y)\in\bbR^2:-1\leq x\leq
    1,-1<y<1\,\bigr\}=[-1,1]\times(-1,1)
  \]
  inside the metric space \(X=\bbR^2\) (the phrase ``half-open'' is purely
  intuitive; it has no precise meaning, but the picture should make it
  clear why we use this terminology). Intuitively, this is a square region
  whose horizontal edges are ``left out''. The interior of \(A\) should be
  \((-1,1)\times(-1,1)\) and the closure should be \([-1,1]\times[-1,1]\),
  as drawing a picture should convince you. Of course, we want to see that
  such conclusions really do follow from our precise definitions.

  First we check that \(A^\circ\) is correctly described. If \(-1<x<1\) and
  \(-1<y<1\) then for
  \[
    r=\min\bigl\{|-1-x|,|1-x|,|-1-y|,|1-y|\bigr\}>0
  \]
  it is easy to check that
  \(B\bigl((x,y),r\bigr)\subseteq(-1,1)\times(-1,1)\) (since a square box
  with side-length \(r\) contains the disc of radius \(r\) with the same
  center). Thus, \((-1,1)\times(-1,1)\subseteq A\) is an open subset of
  \(X=\bbR^2\). To check it is the full interior of \(A\), we just have to
  show that the ``missing points'' of the form \(\pm 1,y\) do not lie in
  the interior. But for any such point \(p=(\pm 1,y)\in A\), for any
  positive small \(r>0\) there is always a point in \(B(p,r)\) with the
  same \(y\)-coordinate but with the \(x\)-coordinate either slightly
  larger than \(1\) or slightly less than \(-1\). Such a point is not in
  \(A\). Thus, \(p\notin A^\circ\).

  Now we check that \(\bar A=[-1,1]\times[-1,1]\). Since convergence in
  \(\bbR^2\) forces convergence in coordinates, to see
  \[
    \bar A\subseteq[-1,1]\times[-1,1]
  \]
  it suffices to check that \([-1,1]\) is closed in \(\bbR\) (since
  certainly \(A\subseteq[-1,1]\times[-1,1]\)). But this is clear (either by
  using sequences or by explicitly showing its complement in \(\bbR\) to be
  open). To see that \(\bar A\) fills up all of \([-1,1]\times[-1,1]\), we
  have to show that each point in \([-1,1]\times[-1,1]\) can be obtained as
  a limit of a sequence in \(A\). We just have to deal with points not in
  \(A=[-1,1]\times(-1,1)\) since points in \(A\) are limits of constant
  sequences. That is, we're faced with studying points of the form \((x,\pm
  1)\) with \(x\in [-1,1]\). Such a point is a limit of a sequence
  \((x,q_n)\) with \(q_n\in(-1,1)\) having limit \(\pm 1\).
\end{example}
\begin{example}
  What happens if we work with the same set \(A\) but view it inside of a
  metric space \(X=A\) (with the Euclidean metric)? In this case,
  \(A^\circ=A\) and \(\bar A=A\)! Indeed, quite generally for any metric
  space \(X\) we have \(X^\circ=X\) and \(\bar X=X\). These are easy
  consequences of thee definitions. Likewise, the empty subset
  \(\emptyset\) in any  metric space has interior and closuer equal to the
  subset \(\emptyset\).

  The moral is that one has to always.
\end{example}

%%% Local Variables:
%%% mode: latex
%%% TeX-master: "../Diffgeo"
%%% End:
