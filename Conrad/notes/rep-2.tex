\chapter{Lecture 2}
In this section, we shall discuss the representations of a cyclic group,
and then proceed to define the important notions of irreducibility and
complete reducibility.
\section{Concrete realization of isomorphism classes}
We observed last time that every \(m\)-dimensional representation of a
group \(G\) was isomorphic to a representation on \(\bbC^m\). This leads to
a concrete realization of the set of \(m\)-dimensional isomorphism classes
of representations.

\begin{proposition}
  The set of \(m\)-dimensional isomorphism classes of \(G\)-representations
  is in bijection with the quotient
  \[
    \Hom\bigl(G,\GL(m,\bbC)\bigr)/\GL(m,\bbC)
  \]
  of the set of group homomorphisms to \(\GL(m)\) by the overall
  conjugation action on the latter.
\end{proposition}
\begin{proof}
  This feels rather tautological, but conjugation by \(\varphi\in\GL(m)\)
  sends a homomorphism \(\rho\) to the new homomorphism
  \(g\mapsto \varphi\circ\rho(g)\varphi^{-1}\). According to Def.\@ 1.2,
  this has exactly the effect of identifying isomorphic representations.
\end{proof}

\begin{remark}
  The proposition is not as useful as it looks (at least to us). It can be
  helpful in undesrtanding certain infinite discrete groups (such as the
  following example, \(\bbZ\)) in which case the set \(\Hom\) can have
  interesting geometric structures. However, for finite groups, the set of
  isomorphism classes is finite so its description above is not too
  enlightening.
\end{remark}

\section{Representation of \(\bbZ\)}
We shall classify all representations of the group \(\bbZ\), with its
additive structure. We must have \(\rho(0)=\Id\). Aside from that, we must
specify an invertible matrix \(\rho(n)\) for every \(n\in\bbZ\). However,
given \(\rho(1)\), we can recover \(\rho(n)\) as
\(\rho(1+\dotsb+1)=\rho(1)^n\). So there is no choice involved. Conversely,
for any invertible map \(\rho(1)\in\GL(m)\), we obtain a representation of
\(\bbZ\) this way.

Thus, \(m\)-dimensional isomorphism classes of representations of \(\bbZ\)
are in bijection with conjugacy classes in \(\GL(m)\). These can be
parametrized by the \emph{Jordan canonical form} (see the next
example). We will have \(m\) continuous parameters -- the eigenvalues,
which are nonzero complex numbers, and are defined up to reordering -- and
some discrete parameters whenever two or more eigenvalues coincide,
specifying the Jordan block sizes.

\section{The cyclic group of order \(n\)}
Let \(G=\{1,g,\dotsc,g^{n-1}\}\), with the relation \(g^n=1\). A
representation of \(G\) on \(V\) defines an invertible endomorphism
\(\rho(g)\in \GL(V)\). As before, \(\rho(1)=\Id\) and
\(\rho(g^k)=\rho(g)^k\), so all other images of \(\rho\) are determined by
the single operator \(\rho(g)\).

Choosing a basis of \(V\) allows us to convert \(\rho(g)\) into a matrix
\(A\), but we shall want to be careful with our choice. Recall that from
general theory that there exists a \emph{Jordan basis} in which \(\rho(g)\)
takes its block-diagonal \emph{Jordan normal form}
\[
  A=
  \begin{bmatrix}
    J_1&0&\cdots&0\\
    0&J_2&\cdots&0\\
    \vdots&\vdots&\ddots&\vdots\\
    0&0&\cdots&J_m
  \end{bmatrix}
\]
where the \emph{Jordan blocks \(J_k\)} take the form
\[
J_k=
\begin{bmatrix}
  \lambda&1&0&\cdots&0\\
  0&\lambda&1&\cdots&0\\
  \vdots&\vdots&\vdots&\ddots&\vdots\\
  0&0&0&\cdots&1\\
  0&0&0&\cdots&\lambda
\end{bmatrix}.
\]
However, we must impose the condition \(A^n=\Id\). But \(A^n\) itself will
be block-diagonal, with blocks \({J_k}^n\), so we must have
\({J_k}^n=1\). To compute that, let \(N\) be the Jordan matrix with
\(\lambda=0\); then we have \(J=\lambda\Id+N\), so
\[
J^n=\bigl(\lambda\Id+N\bigr)^n=\lambda^n\Id+\binom{n}{1}\lambda^{n-1}N+\binom{n}{2}\lambda^{n-2}N^2+\dotsb.
\]
Notice that for the above to be \(\Id\), since \(N^p\) is the matrix of
zeros and ones only, with the ones in index position \((i,j)\) with
\(i=j+k\) (a line parallel to the diagonal, \(k\) steps above it). So the
sum above can be \(\Id\) if and only if \(\lambda^n=1\) and \(N=0\). In
other words, \(J\) is a \(1\times 1\) black, and \(\rho(g)\) is
\emph{diagonal} in this basis. We conclude the following.

\begin{proposition}
  If \(V\) is a representation of the cyclic group \(G\) of order \(n\),
  there exists a basis in which the action of every group element is
  diagonal, with the \(n\)-th roots of unity on the diagonal.
\end{proposition}
\section{Finite Abelian groups}
The discussion for cyclic groups generalizes to any finite Abelian group
\(A\). (The resulting classification of representations is more or less
explicit, depending on whether we are willing to use the classification
theorem for finite Abelian groups.) We recall the following fact from
linear algebra:
\begin{proposition}
  Any family of commuting, separately diagonalizable \(m\times m\) matrices
  can be simultaneously diagonalized.
\end{proposition}

The proof of this can be found in most linear algebra books.

This implies that any representation of \(A\) is isomorphic to one where
every group element acts diagonally. Each diagonal entry then determines a
\emph{one-dimensional} representation of \(A\). So the classification
reads: \(m\)-dimensional isomorphism classes of representations of \(A\)
are in bijection with unordered \(m\)-tuples of \(1\)-dimensional
representations. Note that for \(1\)-dimensional representations, viewed as
homomorphisms \(\rho\colon A\to\bbC^\times\), there is no distinction
between identity and isomorphism (the conjugation action of
\(\GL(1,\bbC)\) on itself is trivial) .

To say more, we must invoke the classification of finite Abelian groups,
according to which \(A\) is isomorphic to a direct product of cyclic
groups. To specify a \(1\)-dimensional representation of \(A\) we must then
specify a root of unity of the appropriate order independently for each
generator.

\section{Subrepresentations and reducibility}
Let \(\rho\colon G\to\GL(V)\) be a representation of \(G\).

\begin{definition}
  A sub representation of \(V\) is a \(G\)-invariant subspace
  \(W\subseteq V\); i.e., for every \(\bfw\in W\), \(g\in G\) we have
  \(\rho(g)(\bfw)\in W\). \(W\) becomes a representation under the action
  \(\rho(g)\).
\end{definition}

Recall that, given a subspace \(W\subseteq V\), we can form the
\emph{quotient space \(V/W\)}, the set of \(W\)-cosets \(\bfv+V\). If \(W\)
was \(G\)-invariant, the \(G\)-action on \(V\) \emph{descends to} an action
on \(V/W\) by setting \(g(\bfv+W)=\rho(g)(\bfv)+W\). If we choose another
\(\bfv'\) in the same coset as \(\bfv\), then \(\bfv-\bfv'\in W\), so
\(\rho(g)(\bfv-\bfv')\in W\), and then the cosets \(\rho(\bfv)+W\) and
\(\rho(\bfv')+W\) agree.

\begin{definition}
  With this action, \(V/W\) is called the \emph{quotient representation} of
  \(V\) under \(W\).
\end{definition}

\begin{definition}
  The \emph{direct sum} of two representations \((\rho_1,V_1)\) and
  \((\rho_2,V_2)\) is the space \(V_1\oplus V_2\) with the block-diagonal
  action \(\rho_1\oplus\rho_2\) of \(G\).
\end{definition}

In the direct sum \(V_1\oplus V_2\), \(V_1\) is a subrepresentation and
\(V_2\) is isomorphic to the associated quotient representation. Of course
the roles of \(1\) and \(2\) can be interchanged. However, one should take
care that for an \emph{arbitrary} group, it need not be the case that any
representation \(V\) with subrepresentation \(W\) decomposes as
\(W\oplus W/V\). This will be proved for complex representations of
\emph{finite} groups.

\begin{definition}
  A representation is called \emph{irreducible} if it contains no proper
  invariant subspaces. It is called \emph{completely reducible} if it
  decomposes as a direct sum of irreducible subrepresentation.
\end{definition}

In particular, irreducible representations are completely reducible.

For example, \(1\)-dimensional representations of any group are
irreducible. Earlier, we thus proved that finite-dimensional complex
representations of a finite Abelian group are completely reducible: indeed,
we decomposed \(V\) into a direct sum of lines \(\bigoplus_{i=1}^nL_i\),
where \(n=\dim V\), along the vectors in the diagonal basis. Each line is
preserved by the action of the group. In the cyclic case, the possible
actions of \(C_n\) on the line correspond to the \(n\) eligible roots of
unity to specify for \(\rho(g)\).

\begin{proposition}
  Every complex representation of a finite Abelian group is completely
  reducible, and every irreducible representation is \(1\)-dimensional.
\end{proposition}

It will be our goal to establish an analogous proposition for every finite
group \(G\). The result is called the \emph{complete reducibility
  theorem}. For nonAbelian groups, we shall have to give up on the
\(1\)-dimensional requirement, but we shall still salvage a canonical
decomposition.

%%% Local Variables:
%%% mode: latex
%%% TeX-master: "../RepTheo"
%%% End:
