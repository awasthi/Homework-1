\chapter{Lecture 2}
In this section, we shall discuss the representations of a cyclic group,
and then proceed to define the important notions of irreducibility and
complete reducibility.
\section{Concrete realization of isomorphism classes}
We observed last time that every \(m\)-dimensional representation of a
group \(G\) was isomorphic to a representation on \(\bbC^m\). This leads to
a concrete realization of the set of \(m\)-dimensional isomorphism classes
of representations.

\begin{proposition}
  The set of \(m\)-dimensional isomorphism classes of \(G\)-representations
  is in bijection with the quotient
  \[
    \Hom\bigl(G,\GL(m,\bbC)\bigr)/\GL(m,\bbC)
  \]
  of the set of group homomorphisms to \(\GL(m)\) by the overall
  conjugation action on the latter.
\end{proposition}
\begin{proof}
  This feels rather tautological, but conjugation by \(\varphi\in\GL(m)\)
  sends a homomorphism \(\rho\) to the new homomorphism
  \(g\mapsto \varphi\circ\rho(g)\varphi^{-1}\). According to Def.\@ 1.2,
  this has exactly the effect of identifying isomorphic representations.
\end{proof}

\begin{remark}
  The proposition is not as useful as it looks (at least to us). It can be
  helpful in undesrtanding certain infinite discrete groups (such as the
  following example, \(\bbZ\)) in which case the set \(\Hom\) can have
  interesting geometric structures. However, for finite groups, the set of
  isomorphism classes is finite so its description above is not too
  enlightening.
\end{remark}

\section{Representation of \(\bbZ\)}
We shall classify all representations of the group \(\bbZ\), with its
additive structure. We must have \(\rho(0)=\Id\). Aside from that, we must
specify an invertible matrix \(\rho(n)\) for every \(n\in\bbZ\). However,
given \(\rho(1)\), we can recover \(\rho(n)\) as
\(\rho(1+\dotsb+1)=\rho(1)^n\). So there is no choice involved. Conversely,
for any invertible map \(\rho(1)\in\GL(m)\), we obtain a representation of
\(\bbZ\) this way.

Thus, \(m\)-dimensional isomorphism classes of representations of \(\bbZ\)
are in bijection with conjugacy classes in \(\GL(m)\). These can be
parametrized by the \emph{Jordan canonical form} (see the next
example). We will have \(m\) continuous parameters -- the eigenvalues,
which are nonzero complex numbers, and are defined up to reordering -- and
some discrete parameters whenever two or more eigenvalues coincide,
specifying the Jordan block sizes.

\section{The cyclic group of order \(n\)}
Let \(G=\{1,g,\dotsc,g^{n-1}\}\), with the relation \(g^n=1\). A
representation of \(G\) on \(V\) defines an invertible endomorphism
\(\rho(g)\in \GL(V)\). As before, \(\rho(1)=\Id\) and
\(\rho(g^k)=\rho(g)^k\), so all other images of \(\rho\) are determined by
the single operator \(\rho(g)\).

Choosing a basis of \(V\) allows us to convert \(\rho(g)\) into a matrix
\(A\), but we shall want to be careful with our choice. Recall that from
general theory that there exists a \emph{Jordan basis} in which \(\rho(g)\)
takes its block-diagonal \emph{Jordan normal form}
\[
  A=
  \begin{bmatrix}
    J_1&0&\cdots&0\\
    0&J_2&\cdots&0\\
    \vdots&\vdots&\ddots&\vdots\\
    0&0&\cdots&J_m
  \end{bmatrix}
\]
where the \emph{Jordan blocks \(J_k\)} take the form
\[
J_k=
\begin{bmatrix}
  \lambda&1&0&\cdots&0\\
  0&\lambda&1&\cdots&0\\
  \vdots&\vdots&\vdots&\ddots&\vdots\\
  0&0&0&\cdots&1\\
  0&0&0&\cdots&\lambda
\end{bmatrix}.
\]
However, we must impose the condition \(A^n=\Id\). But \(A^n\) itself will
be block-diagonal, with blocks \({J_k}^n\), so we must have
\({J_k}^n=1\). To compute that, let \(N\) be the Jordan matrix with
\(\lambda=0\); then we have \(J=\lambda\Id+N\), so
\[
J^n=\bigl(\lambda\Id+N\bigr)^n=\lambda^n\Id+\binom{n}{1}\lambda^{n-1}N+\binom{n}{2}\lambda^{n-2}N^2+\dotsb.
\]

%%% Local Variables:
%%% mode: latex
%%% TeX-master: "../RepTheo"
%%% End:
