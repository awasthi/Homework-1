\begin{problem}[Lee, Prob.\,3-1]
Suppose $(\widetilde M,\tilde g)$ is a Riemannian $m$-manifold,
$M\subset\widetilde M$ is an embedded $n$-dimensional submanifold, and $g$
is the induced Riemannian metric on $M$. For any point $p$ show that there
is a neighborhood $\widetilde U$ of $p$ in $\widetilde M$ and a smooth
orthonormal frame $(E_1,...,E_m)$ on $\widetilde U$ such that
$(E_1,...,E_m)$ form an orthonormal basis for $T_qM$ at each
$q\in\widetilde{U}\cap M$. Any such frame is called an adapted orthonormal
frame. [Hint: Apply the Gram--Schmidt algorithm to the coordinate frame
$\left\{\partial_i\right\}$ in slice coordinates.]
\end{problem}
\begin{proof}

\end{proof}
\newpage

\begin{problem}[Lee, Prob.\,3-2]
Suppose $g$ is a pseudo-Riemannian metric on an $n$-manifold $M$. For any
$p\in M$, show there is a smooth local frame $(E_1,...,E_n)$ defined in a
neighborhood of $p$ such that $g$ can be written locally in the form
(3.4). Conclude that the index of $g$ is constant on each component of
$M$.
\end{problem}
\begin{proof}
\end{proof}
\newpage

\begin{problem}[Lee, Prob.\,3-3]
Let $(M,g)$ be an oriented Riemannian manifold with volume element
$\diff V$. The divergence operator $\Div\colon\calT(M)\to C^\infty(M)$ is
defined by
\[
\diff(i_X\,\diff V)=(\Div X)\,\diff V,
\]
where $i_X$ denotes interior multiplication by $X$: for any $k$-form
$\omega$, $i_X\omega$ is the $(k-1)$-form defined by
\[
i_X\omega(V_1,...,V_{k_1})=\omega(X,V_1,...,V_{k-1}).
\]
\begin{enumerate}[label=(\alph*)]
\item Suppose $M$ is a compact, oriented Riemannian manifold with
  boundary. Prove the following divergence theorem for $X\in\calT(M)$:
\[
\int_M\Div X\,\diff V=\int_{\partial M}\langle X,N \rangle\,\diff\widetilde V.
\]
where $N$ is the outward unit normal to $\partial M$ and $\diff\widetilde V$ is
the Riemannian volume element of the induced metric on $\partial M$.
\item Show that the divergence operator satisfies the following product
  rule for a smooth function $u\in C^\infty(M)$:
\[
\Div(uX)=u\Div X+\langle\Grad u,X\rangle,
\]
and deduce the following ``integration by parts'' formula:
\[
\int_M\langle\Grad u,X\rangle\,\diff V=-\int_M u\Div X\,\diff V+\int{\partial M}
u\langle X,N\rangle\,\diff\widetilde V.
\]
\end{enumerate}
\end{problem}
\begin{proof}
\end{proof}
\newpage

\begin{problem}[Lee, Prob.\,3-4]
Let $(M,g)$ be a compact, connected, oriented Riemannian manifold with
boundary. For $u\in C^\infty(M)$, the Laplacian of $u$, denoted $\Lap u$,
is defined to be the function $\Lap u\coloneqq \Div(\Grad u)$. A function
$u\in C^\infty(M)$ is said to be harmonic if $\Lap u=0$.
\begin{enumerate}[label=(\alph*)]
\item Prove Green's identities:
\begin{align*}
\int_M u\Lap v\,\diff V+\int_M\langle\Grad u,\Grad v\rangle
&=\int_{\partial M}uNv\,\diff\widetilde V\\
\int_M\left(u\Lap v-v\Lap u\right)\,\diff V
&=\int_{\partial M}\left(uNv-vNu\right)\,\diff\widetilde V
\end{align*}
\item Show if $\partial M\neq\emptyset$, and $u$, $v$ are harmonic
  functions on $M$ whose restriction to $\partial M$ agree, then $u\equiv
  v$.
\item If $\partial M=\emptyset$ show that the only harmonic functions on
  $M$ are the constants.
\end{enumerate}
\end{problem}
\begin{proof}
\end{proof}
\newpage

\begin{problem}[Lee, Prob.\,3-5]
Let $M$ be a compact oriented Riemannian manifold (without boundary). A
real number $\lambda$ is called an eigenvalue of the Laplacian if
there exists a smooth function $u$ on $M$, not identically zero, such
that $\Lap u=\lambda u$. In this case, $u$ is called an eigenfunction
corresponding to $\lambda$.
\begin{enumerate}[label=(\alph*)]
\item Prove that $0$ is an eigenvalue of $\Lap$, and that all other
  eigenvalues are strictly negative.
\item If $u$ and $v$ are eigenfunctions corresponding to distinct
  eigenvalues, show that $\int_M uv\,\diff V=0$.
\end{enumerate}
\end{problem}
\begin{proof}
\end{proof}

%%% Local Variables:
%%% mode: latex
%%% TeX-master: "../MA661-HW-Current"
%%% End:
