\chapter{Preliminaries}
These set of notes are based off of Boothby's \emph{Differential Geometry}
book, chapters 1 through 6.
\\\\
\begin{definition}
A \emph{topological space} $M$ is a pair $(X,\calT)$, where $X$ is a set,
$\calT$ is a collection of subsets of $X$ such that
\begin{enumerate}[label=(\alph*)]
\item $\emptyset,X\in\calT$.
\item The union of any subcollection of $\calT$ is in $\calT$.
\[
\left\{U_\alpha\right\}\subset\calT
\quad\implies\quad
\bigcup_\alpha U_\alpha\in\calT.\]
\item Intersection of a finite subcollection of $\calT$ is in $\calT$.
\[\left\{U_1,...,U_k\right\}\subset\calT
\quad\implies\quad
\bigcap_{j=1}^k U_j\in\calT.\]
\end{enumerate}
$\calT$ is called the \emph{topology} of $M$. Elements of $\calT$ are
called the \emph{open sets} of $M$. By abuse of notation, we sometimes
refer to $X$ as $M$.
\end{definition}
\begin{definition}
\begin{enumerate}[label=(\alph*)]
\item A \emph{metric} on $X$ is a function $d\colon X\times X\to\bfR$ such
  that
  \begin{enumerate}[label=(\arabic*)]
  \item $d(x,y)\geq 0$ $\forall x,y\in X$ and $d(x,y)=0$ $\iff$ $x=y$.
  \item $d(x,y)=d(y,x)$.
  \item $d(x,y)+d(y,z)\geq d(x,z)$ (the triangle inequality).
  \end{enumerate}
\item $B_d(x,r)=\left\{\,y\in X\;\middle|\;d(x,y)<r\,\right\}$.
\item A topological space $M$ is a \emph{metric space} if the set of balls
  $B_d(x,r)$ form a \emph{basis} of $M$, i.e., any open set of $M$ can be
  written as a union of open balls $B_d(x,r)$ for some $x\in X$, $r>0$.
\end{enumerate}
\end{definition}

\begin{definition}
A topological space $X$ is \emph{Hausdorff} if for any $x_1\neq x_2$ in
$X$, there exist open sets $U_1\ni x_1$, $U_2\ni x_2$ such that $U_1\cap
U_2=\emptyset$.
\end{definition}

\begin{definition}
A \emph{topological manifold} $M$ of dimension $n$ is a topological space
such that
\begin{enumerate}[label=(\alph*)]
\item $M$ is Hausdorff.
\item locally Euclidean, i.e., $\forall x\in M$ there exists a neighborhood
  $U$ of $X$ which is homeomorphic to $V\subset\bfR^n$ (there exists a map
  $f\colon U_x\to V\subset\bfR^n$ such that $f$ is bijective, continuous
  and $f^{-1}$ is continuous).
\item $M$ has a countable basis of open sets.
\end{enumerate}
\end{definition}
\begin{theorem}[Boothby I.3.6]
A \emph{topological manifold} is metrizable (also locally connected,
locally compact, and normal).
\end{theorem}

\begin{definition}
\begin{enumerate}[label=(\alph*)]
\item A \emph{covering} of a topological manifold is a collection of open
  sets $\left\{U_\alpha\right\}$ such that any $x\in M$ is contained in
  some $U_\alpha$.
\item A manifold is \emph{compact} if every open cover contains a finite
  subcover.
\end{enumerate}
\end{definition}

\begin{definition}
\begin{enumerate}[label=(\arabic*)]
\item Half space
\[
\bfH^n=\left\{\,x\in\bfR^n\;\middle|\;x_n\geq 0\,\right\}.
\]
\item Manifold with boundary. (Similar to definition 4)
  \begin{enumerate}[label=(\alph*)]
  \item $M$ is Hausdorff.
  \item $M$ has a countable basis of open sets.
  \item For any $x\in M$, there exists $U$ open, $x\in U$ such that:
    \begin{enumerate}[label=(\roman*)]
    \item $\varphi\colon U\to V\subset\bfR^n$ is a homeomorphism, or
    \item $\varphi\colon U\to V\subset\bfH^n$ is a homeomorphism with $x$
      such that $\varphi(x)\in\partial\bfH^n$ referred to as \emph{boundary
      points}.
    \end{enumerate}
  \end{enumerate}
\item
\end{enumerate}
\end{definition}

\begin{example}[Unit Quaternions and Rotations in $\bfR^3$]
\[
f(v)=z\wedge z^{-1}\qquad
v=(v_1,v_2,v_3)=v_1i+v_2j+v_3k
\]
where $z=\cos(\alpha/2)+\sin(\alpha/2)\hat v$. $\hat v=v/\|v\|$. Quaternion
multplication:
\begin{align*}
ij&=k&ji&=-k\\
jk&=i&kj&=-i\\
ki&=j&ik&=-j.
\end{align*}
Can check that $z$ and $-z$ correspond to the same rotation.

Topologically, unit quaternions $\simeq
S^3=\left\{\,x\in\bfR^4\;\middle|\;\|x\|=1\right\}$ and rotations $\simeq
S^3/{\sim}$, $z\sim -z$
\[
\bfRP^3\approx\left(\bfR^4\minus\{0\}\right)/x\sim\lambda x.
\]
for all $\bfR^{n+1}\minus\{0\}$ can always find $\lambda$ such that
$\lambda x$ has norm $1$. There are precisely $2$ such $\lambda$ which
differ by a sign. Therefore, $\bfRP^n$ can be constructed by identifying
antipodal points of $S^n$ in $\bfR^{n+1}$.
\end{example}


%%% Local Variables:
%%% mode: latex
%%% TeX-master: "../MA562-Notes-Boothby"
%%% End:
