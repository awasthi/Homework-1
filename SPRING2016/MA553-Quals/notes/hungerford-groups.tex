\chapter{Course notes}
Taken from Hungerford's \emph{Algebra}. This first section will cover the
relevant group theory part.
\section{Group Theory}
\subsection{Semigroups, Monoids and Groups}
If $G$ is a nonempty subset, a \emph{binary operation} on $G$ is a function
$G\times G\to G$. There are several commonly noted notations for the image
of $(a,b)$ under the binary operation: $ab$ (multiplicative notation),
$a+b$ (additive notation), $a\cdot b$, $a*b$, etc. For convenience we shall
generally use multiplicative notation throughout this chapter and refer to
$ab$ as the \emph{product} of $a$ and $b$. A set may have several binary
operations defined on it (for example, addition and multiplication on
$\bbZ$ given by $(a,b)\mapsto a+b$ or $(a,b)\mapsto ab$ respectively).
\begin{definition}
A \emph{semigroup} is a nonempty set $G$ together with a binary operation
on $G$ which is
\begin{enumerate}[label=\textnormal{(\alph*)}]
\item associative: $a(bc)=(ab)c$ for all $a,b,c\in G$;

  \bigskip
  a \emph{monoid}  is a semigroup $G$ which contains a
\item two-sided identity element $e\in G$ such that $ae=ea=a$ for all $a\in
  G$.

  \bigskip
  A \emph{group} is a monoid $G$ such that
\item for every $a\in G$ there exists a (two-sided) \emph{inverse} element
  $a^{-1}\in G$ such that $aa^{-1}=a^{-1}a=e$.

  \bigskip
  A semigroup $G$ is said to be \emph{Abelian} or \emph{commutative} if its
  binary operation is
\item commutative: $ab=ba$ for all $a,b\in G$.
\end{enumerate}
\end{definition}
Our principal interests are groups, however semigroups and monoids are
convenient for stating certain certain theorems in the most
generality. Examples are given below. The \emph{order} of a group $G$ is
the cardinality of the set $G$. $G$ is said to be finite if $|G|$ is
finite (otherwise, it is said to be infinite).
\begin{theorem}[1.2]
If $G$ is a monoid, then the identity element $e$ is unique. If $G$ is a
group, then
\begin{enumerate}[label=\textnormal{(\alph*)}]
\item $a\in G$ and $aa=a$ $\implies$ $a=e$;
\item for all $a,b,c\in G$, $ab=ac$ $\implies$ $b=c$ and $ba=ca$ $\implies$
  $b=c$ (left and right cancellation);
\item for each $a\in G$, the inverse element $a^{-1}$ is unique;
\item for each $a\in G$, $\left( a^{-1} \right)^{-1}=a$;
\item for $a,b\in G$, $(ab)^{-1}=b^{-1}a^{-1}$;
\item for $a,b\in G$ the equation $ax=b$ and $ya=b$ have unique solutions
  in $G$: $x=a^{-1}b$ and $y=ba^{-1}$.
\end{enumerate}
\end{theorem}

\begin{proposition}[1.3]
Let $G$ be a semigroup. Then $G$ is a group if and only if the following
conditions hold:
\begin{enumerate}[label=\textnormal{(\roman*)}]
\item there exists an identity element $e\in G$ such that $ea=a$ for all
  $a\in G$ (left identity element);
\item for each $a\in G$, there exists an element $a^{-1}\in G$ such that
  $a^{-1}a=e$ (left inverse).
\end{enumerate}
\end{proposition}
\begin{proof}[Sketch of the proof]
The direction $\implies$ is trivial. $\impliedby$: By Theorem 1.2(i) is
true under the hypotheses. $G\neq\emptyset$ since $e\in G$. If $a\in G$,
then (ii) $(aa^{-1})(aa^{-1})=a(a^{-1}a)a^{-1}=aea^{-1}=aa^{-1}$ and hence
$aa^{-1}=e$ by Theorem 1.2(i). Thus $a^{-1}$ is a two-sided inverse of
$a$. Since $ae=a(a^{-1}a)=(aa^{-1})a=ea=a$ for every $a\in G$, $e$ is a
two-sided identity. Therefore, $G$ is a group by Definition 1.1.
\end{proof}
\begin{proposition}[1.4]
Let $G$ be a semigroup. Then $G$ is a group if and only if for all $a,b\in
G$ the equations $ax=b$, $ya=b$ have solutions in $G$.
\end{proposition}

%%% Local Variables:
%%% mode: latex
%%% TeX-master: "../MA553-HW-Current"
%%% End:
