\def\documentauthor{Carlos Salinas}
\def\documenttitle{MA553: Spring 2016 Homework}
% \def\hwnum{1}
\def\shorttitle{MA553 Homework}
\def\coursename{MA553}
\def\documentsubject{abstract algebra}
\def\authoremail{salinac@purdue.edu}

\documentclass[article,oneside,10pt]{memoir}
\usepackage{geometry}
\usepackage[dvipsnames]{xcolor}
\usepackage[
    breaklinks,
    bookmarks=true,
    colorlinks=true,
    pageanchor=false,
    linkcolor=black,
    anchorcolor=black,
    citecolor=black,
    filecolor=black,
    menucolor=black,
    runcolor=black,
    urlcolor=black,
    hyperindex=false,
    hyperfootnotes=true,
    pdftitle={\shorttitle},
    pdfauthor={\documentauthor},
    pdfkeywords={\documentsubject},
    pdfsubject={\coursename}
    ]{hyperref}

%% Math
\usepackage{amsmath}
\usepackage{amsthm}
\usepackage{amssymb}
\usepackage{mathtools}

%% Script math
\usepackage{eucal}
\usepackage{mathrsfs}

% Use symbols instead of numbers
\renewcommand*{\thefootnote}{\fnsymbol{footnote}}
\renewcommand{\theequation}{\hwnum.\arabic{equation}}

\usepackage[LAE,LFE,T2A,T1]{fontenc}
\usepackage[utf8]{inputenc}
\usepackage[farsi,french,german,spanish,russian,english]{babel}
\babeltags{fr=french,
           de=german,
           en=english,
           es=spanish,
           pa=farsi,
           ru=russian
           }
\def\spanishoptions{mexico}

\selectlanguage{english}

\newcommand{\textfa}[1]{\beginR\textpa{#1}\endR}

\usepackage{cmap}
\usepackage{CJKutf8}
\newcommand{\textkr}[1]{\begin{CJK}{UTF8}{mj}#1\end{CJK}}
\newcommand{\textjp}[1]{\begin{CJK}{UTF8}{min}#1\end{CJK}}
\newcommand{\textzh}[1]{\begin{CJK}{UTF8}{bsmi}#1\end{CJK}}

\usepackage{graphicx}
\graphicspath{{figures/}}

% Misc
\usepackage{microtype}
\usepackage{lineno}
\usepackage{multicol}
\usepackage[inline]{enumitem}
\usepackage{listings}
\usepackage{mleftright}
\mleftright

%% Theorems and definitions
%% remove parentheses
% \makeatletter
% \def\thmhead@plain#1#2#3{%
%   \thmname{#1}\thmnumber{\@ifnotempty{#1}{ }\@upn{#2}}%
%   \thmnote{ {\the\thm@notefont#3}}}
% \let\thmhead\thmhead@plain
% \makeatother

\theoremstyle{plain}
\newtheorem{theorem}{Theorem}
\newtheorem{proposition}[theorem]{Proposition}
\newtheorem{corollary}[theorem]{Corollary}
\newtheorem{claim}[theorem]{Claim}
\newtheorem{lemma}[theorem]{Lemma}
\newtheorem{axiom}[theorem]{Axiom}

\newtheorem*{corollary*}{Corollary}
\newtheorem*{claim*}{Claim}
\newtheorem*{lemma*}{Lemma}
\newtheorem*{proposition*}{Proposition}
\newtheorem*{theorem*}{Theorem}

\theoremstyle{definition}
\newtheorem{definition}{Definition}
\newtheorem{example}{Examples}
\newtheorem{examples}[example]{Examples}
\newtheorem{exercise}{Exercise}[chapter]
\newtheorem{problem}[exercise]{Problem}

\newtheorem*{example*}{Example}
\newtheorem*{exercise*}{Exercise}
\newtheorem*{problem*}{Problem}

%% Redefinitions & commands
\newcommand{\nsubset}{\ensuremath{\not\subset}}
\newcommand{\nsupset}{\ensuremath{\not\supset}}
\newcommand\minus{\ensuremath{\null\smallsetminus}}
\renewcommand\qedsymbol{\ensuremath{\null\hfill\blacksquare}}

%% Commands and operators
\DeclareMathOperator{\id}{id}
\DeclareMathOperator{\im}{im}
\DeclareMathOperator{\lcm}{lcm}
\DeclareMathOperator{\rad}{rad}
\DeclareMathOperator{\nil}{nil}

\DeclareMathOperator{\Aut}{Aut}
\DeclareMathOperator{\Gal}{Gal}
\DeclareMathOperator{\Quot}{Quot}

\DeclareMathOperator{\Id}{Id}
\DeclareMathOperator{\Img}{Im}
\DeclareMathOperator{\Ker}{Ker}
\DeclareMathOperator{\Coker}{Coker}

\newcommand{\GL}{\mathrm{GL}}
\newcommand{\SL}{\mathrm{SL}}

%% Symbols
\newcommand{\bbC}{\mathbb{C}}
\newcommand{\bbN}{\mathbb{N}}
\newcommand{\bbQ}{\mathbb{Q}}
\newcommand{\bbR}{\mathbb{R}}
\newcommand{\bbZ}{\mathbb{Z}}

\newcommand{\bfC}{\mathbf{C}}
\newcommand{\bfN}{\mathbf{N}}
\newcommand{\bfQ}{\mathbf{Q}}
\newcommand{\bfR}{\mathbf{R}}
\newcommand{\bfZ}{\mathbf{Z}}

\newcommand{\calA}{\mathcal{A}}
\newcommand{\calB}{\mathcal{B}}
\newcommand{\calC}{\mathcal{C}}
\newcommand{\calO}{\mathcal{O}}
\newcommand{\calU}{\mathcal{V}}
\newcommand{\calV}{\mathcal{U}}

\newcommand{\scrA}{\mathscr{A}}
\newcommand{\scrB}{\mathscr{B}}
\newcommand{\scrC}{\mathscr{C}}
\newcommand{\scrL}{\mathscr{L}}
\newcommand{\scrO}{\mathscr{O}}
\newcommand{\scrS}{\mathscr{S}}
\newcommand{\scrT}{\mathscr{T}}

\begin{document}
\author{\href{mailto:\authoremail}{\documentauthor}}
\title{\documenttitle}
\date{\today}
\maketitle
%% Ulrich homework
\chapter{Homework 1}
We'll discuss some important theorems and results as we solve some of these
problems.
\begin{problem}
Let $G$ be a group, $a\in G$ an element of finite order $m$, and $n$ a
positive integer. Prove that
\[
|a^n|=\frac{m}{\gcd(m,n)}.
\]
\end{problem}
\begin{proof}

\end{proof}

\begin{problem}
Let $G$ be a group, and let $a$, $b$ be elements of finite order $m$, $n$
respectively. Show that if $ba=ab$ and $\langle a\rangle\cap\langle
b\rangle=\{e\}$, then $|ab|=\lcm(m,n)$.
\end{problem}
\begin{proof}
\end{proof}

\begin{problem}
Let $G$ be a group and $H$, $K$ normal subgroups with $H\cap K=\{e\}$. Show
that
\begin{enumerate}[label=(\alph*)]
\item $hk=kh$ for every $h\in H$, $k\in K$.
\item $HK$ is a subgroup of $G$ with $HK\cong H\times K$.
\end{enumerate}
\end{problem}
\begin{proof}
\end{proof}

\begin{problem}
Show that $A_4$ has no subgroup of order $6$ (although $6\mid 12=|A_4|$).
\end{problem}
\begin{proof}
\end{proof}

%%% Local Variables:
%%% mode: latex
%%% TeX-master: "../MA553-HW-Current"
%%% End:

\section{Homework 2}
\begin{problem}
Let $G$ be the group of order $2^3\cdot 3$, $n\geq 2$. Show that $G$ has a
normal $2$-subgroup $\neq\left\{e\right\}$.
\end{problem}
\begin{proof}
\end{proof}

\begin{problem}
Let $G$ be a group of order $p^2q$, $p$ and $q$ primes. Show that the Sylow
$p$-Sylow subgroup or the $q$-Sylow subgroup of $G$ is normal in $G$.
\end{problem}
\begin{proof}
\end{proof}

\begin{problem}
Let $G$ be a subgroup of order $pqr$, $p<q<r$ primes. Show that the
$r$-Sylow subgroup of $G$ is normal in $G$.
\end{problem}
\begin{proof}
\end{proof}

\begin{problem}
Let $G$ be a group of order $n$ and let $\varphi\colon G\to S_n$ be given by
the action of $G$ on $G$ via translation.
\begin{enumerate}[label=(\alph*)]
\item For $a\in G$ determine the number and the lengths of the disjoint
  cycles of the permutation $\phi(a)$.
\item Show that $\varphi(G)\nsubset A_n$ if and only if $n$ is even and $G$
  has a cyclic $2$-Sylow subgroup.
\item If $n=2m$, $m$ odd, show that $G$ has a subgroup of index $2$.
\end{enumerate}
\end{problem}
\begin{proof}
\end{proof}

\begin{problem}
Show that the only simple groups $\neq\left\{e\right\}$ of order $<60$ are
the groups of prime order.
\end{problem}
\begin{proof}
\end{proof}

%%% Local Variables:
%%% mode: latex
%%% TeX-master: "../MA553-HW-Current"
%%% End:

\chapter{Homework 3}
\begin{problem}
Let $G$ be a finite group, $p$ a prime number, $N$ the intersection of all
$p$-Sylow subgroups of $G$. Show that $N$ is a normal $p$-subgroup of $G$
and that every normal $p$-subgroup of $G$ is contained in $N$.
\end{problem}
\begin{proof}
\end{proof}

\begin{problem}
Let $G$ be a group of order $231$ and let $H$ be an $11$-Sylow subgroup of
$G$. Show that $H\subset Z(G)$.
\end{problem}
\begin{proof}
\end{proof}

\begin{problem}
Let $G=\left\{e,a_1,a_2,a_3\right\}$ be a non-cyclic group of order $4$ and
define $\varphi\colon S_3\to\Aut(G)$ by $\varphi(\sigma)(e)=e$ and
$\varphi(\sigma)(a_1)=a_{\sigma(i)}$. Show that $\varphi$ is well-defined and an
isomorphism of groups.
\end{problem}
\begin{proof}
\end{proof}

\begin{problem}
Determine all groups of order $18$.
\end{problem}
\begin{proof}
\end{proof}

%%% Local Variables:
%%% mode: latex
%%% TeX-master: "../MA553-HW-Current"
%%% End:

% \chapter{Homework 4}

%%% Local Variables:
%%% mode: latex
%%% TeX-master: "../MA553-HW-Current"
%%% End:

\chapter{Homework 5}
\begin{problem}
Find all composition series and the composition factors of $D_6$.
\end{problem}
\begin{proof}
\end{proof}

\begin{problem}
Let $T$ be the subgroup of $\GL_n(\bfR)$ consisting of all upper triangular
invertible matrices. Show that $T$ is solvable.
\end{problem}
\begin{proof}
\end{proof}

\begin{problem}
Let $p\in\bfZ$ be a prime number. Show:
\begin{enumerate}[label=(\alph*)]
\item $(p-1)!\equiv -1\mod{p}$.
\item If $p\equiv 1\mod{4}$ then $x^2\equiv -1\mod{p}$ for some
  $x\in\bfZ$.
\end{enumerate}
\end{problem}
\begin{proof}
\end{proof}

\begin{problem}
\begin{enumerate}[label=(\alph*)]
\item Show that the following are equivalent for an odd prime number
  $p\in\bfZ$:
  \begin{enumerate}[label=(\roman*)]
  \item $p\equiv 1\mod 4$.
  \item $p=a^2+b^2$ for some $a$, $b$ in $\bfZ$.
  \item $p$ is not prime in $\bfZ[i]$.
  \end{enumerate}
\item Determine all prime ideals of $\bfZ[i]$.
\end{enumerate}
\end{problem}
\begin{proof}
\end{proof}

%%% Local Variables:
%%% mode: latex
%%% TeX-master: "../MA553-HW-Current"
%%% End:

\subsection{Homework 6}
\begin{problem}
Let $R$ be a domain. Show that $R$ is a UFD if and only if every nonzero
nonunit in $R$ is a product of irreducible elemnets and the intersection of
any two principal ideals is again principal.
\end{problem}
\begin{proof}
\end{proof}

\begin{problem}
Let $R$ be a PID and $p$ a prime ideal of $R[X]$. Show that $p$ is
principal or $p=(a,f)$ for some $a\in R$ and some monic $f\in R[X]$.
\end{problem}
\begin{proof}
\end{proof}

\begin{problem}
Let $k$ be a field and $n\geq 1$. Show that $Z^n+Y^3+X^2\in k(X,Y)[Z]$ is
irreducible.
\end{problem}
\begin{proof}
\end{proof}

\begin{problem}
Let $k$ be a field of characteristic zero and $n\geq 1$, $m\geq 2$. Show
that ${X_1}^n+\dotsb+{X_m}^n-1\in k[X_1,\dotsc,X_m]$ is irreducible.
\end{problem}
\begin{proof}
\end{proof}

\begin{problem}
Show that $X^{3^n}+2\in\bbQ(i)[X]$ is irreducible.
\end{problem}
\begin{proof}
\end{proof}

%%% Local Variables:
%%% mode: latex
%%% TeX-master: "../MA553-Quals"
%%% End:

\chapter{Homework 7}
\begin{problem}
Let $k\subset K$ and $k\subset L$ be finite field extensions contained in
some field. Show that:
\begin{enumerate}[label=(\alph*)]
\item $[KL:L]\leq [K:k]$.
\item $[KL:k]\leq [K:k][L:k]$.
\item $K\cap L=k$ if equality holds in (b).
\end{enumerate}
\end{problem}
\begin{proof}
\end{proof}

\begin{problem}
Let $k$ be a field of characteristic $\neq 2$ and $a,b$ elements of $k$ so
that $a$, $b$, $ab$ are not squares in $k$. Show that
$\bigl[k{(\sqrt{a},\sqrt{b})}:k\bigr]=4$.
\end{problem}
\begin{proof}
\end{proof}

\begin{problem}
Let $R$ be a UFD, but not a field, and write $K=\Quot(R)$. Show that $[\bar
K:k]=\infty$.
\end{problem}
\begin{proof}
\end{proof}

\begin{problem}
Let $k\in K$ be an algebraic field extension. Show that every
$k$-homomorphism $\delta\colon K\to K$ is an isomorphism.
\end{problem}
\begin{proof}
\end{proof}

\begin{problem}
Let $K$ be the splitting field of $x^6-4$ over $\bfQ$. Determine $K$ and
$[K:\bfQ]$.
\end{problem}
\begin{proof}
\end{proof}

%%% Local Variables:
%%% mode: latex
%%% TeX-master: "../MA553-HW-Current"
%%% End:

\end{document}

%%% Local Variables:
%%% mode: latex
%%% TeX-master: t
%%% End:
