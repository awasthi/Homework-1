\def\documentauthor{Carlos Salinas}
\def\documenttitle{MA553: Spring 2016 Homework}
% \def\hwnum{1}
\def\shorttitle{MA553 Homework}
\def\coursename{MA553}
\def\documentsubject{abstract algebra}
\def\authoremail{salinac@purdue.edu}

\documentclass[article,oneside,10pt]{memoir}
\usepackage{geometry}
\usepackage[dvipsnames]{xcolor}
\usepackage[
    breaklinks,
    bookmarks=true,
    colorlinks=true,
    pageanchor=false,
    linkcolor=black,
    anchorcolor=black,
    citecolor=black,
    filecolor=black,
    menucolor=black,
    runcolor=black,
    urlcolor=black,
    hyperindex=false,
    hyperfootnotes=true,
    pdftitle={\shorttitle},
    pdfauthor={\documentauthor},
    pdfkeywords={\documentsubject},
    pdfsubject={\coursename}
    ]{hyperref}

%% Math
\usepackage{amsmath}
\usepackage{amsthm}
\usepackage{amssymb}
\usepackage{mathtools}
\usepackage{eucal}
\usepackage{mathrsfs}
\usepackage[nointegrals]{wasysym}

%% Language
\usepackage{cmap}
\usepackage[LAE,LFE,T2A,T1]{fontenc}
\usepackage[utf8]{inputenc}
\usepackage[farsi,french,german,spanish,russian,english]{babel}
\babeltags{fr=french,
           de=german,
           en=english,
           es=spanish,
           pa=farsi,
           ru=russian
           }
\def\spanishoptions{mexico}

\selectlanguage{english}

\newcommand{\textfa}[1]{\beginR\textpa{#1}\endR}

\usepackage{CJKutf8}
\newcommand{\textkr}[1]{\begin{CJK}{UTF8}{mj}#1\end{CJK}}
\newcommand{\textjp}[1]{\begin{CJK}{UTF8}{min}#1\end{CJK}}
\newcommand{\textzh}[1]{\begin{CJK}{UTF8}{bsmi}#1\end{CJK}}

%% Misc
\usepackage{graphicx}
\graphicspath{{figures/}}

\usepackage{microtype}
\usepackage{lineno}
\usepackage{multicol}
\usepackage[inline]{enumitem}
\usepackage{listings}
\usepackage{mleftright}
\mleftright
\usepackage{carlos-variables}

% %% Unicode math and Polyglossia
% \usepackage{unicode-math}
% \usepackage{unicode-minionmath}

% \setmainfont[Ligatures=TeX]{Libertinus Serif}
% \setsansfont{Libertinus Sans}
% \setmonofont{Libertinus Mono}
% \setmathfont{Minion Math}
% \setmathfont[range={\mathfrak}]{XITS Math}
% \setmathfont[range={\mathcal},StylisticSet=1]{XITS Math}
% \setmathfont[range={\mathscr}]{XITS Math}
% \setmathfont[range={}]{Minion Math}

% \usepackage{polyglossia}

% \newfontfamily\cyrillicfont[Script=Cyrillic]{Libertinus Serif}
% \newfontfamily\cyrillicfontsf[Script=Cyrillic]{Libertinus Sans}

% \newfontfamily\farsifont[Script=Arabic,
%                          Scale=MatchUppercase]{Amiri}

% \setmainlanguage[variant=american]{english}
% \setotherlanguage{farsi}
% \setotherlanguage{french}
% \setotherlanguage[spelling=new,latesthyphen,babelshorthands]{german}
% \setotherlanguage{spanish}
% \setotherlanguage[spelling=modern,babelshorthands]{russian}

% \makeatletter
% \@Latintrue
% \makeatother

% \usepackage{xeCJK}
% \usepackage[overlap]{ruby}
% \renewcommand\rubysep{-0.2ex}
% \xeCJKDeclareSubCJKBlock{Kana}{"3040 -> "309F, "30A0 -> "30FF, "31F0 -> "31FF, "1B000 -> "1B0FF}
% \xeCJKDeclareSubCJKBlock{Hangul}{"1100 -> "11FF, "3130 -> "318F, "A960 -> "A97F, "AC00 -> "D7AF, "D7B0 -> "D7FF}

% \setCJKmainfont{HanaMinA}
% \setCJKmainfont[Kana]{HanaMinA}
% \setCJKmainfont[Hangul]{NanumMyeongjo}
% \setCJKsansfont[Hangul]{NanumGothic}

%% Theorems and definitions
%% remove parentheses
% \makeatletter
% \def\thmhead@plain#1#2#3{%
%   \thmname{#1}\thmnumber{\@ifnotempty{#1}{ }\@upn{#2}}%
%   \thmnote{ {\the\thm@notefont#3}}}
% \let\thmhead\thmhead@plain
% \makeatother

\theoremstyle{plain}
\newtheorem{theorem}{Theorem}
\newtheorem{proposition}[theorem]{Proposition}
\newtheorem{corollary}[theorem]{Corollary}
\newtheorem{claim}[theorem]{Claim}
\newtheorem{lemma}[theorem]{Lemma}
\newtheorem{axiom}[theorem]{Axiom}

\newtheorem*{corollary*}{Corollary}
\newtheorem*{claim*}{Claim}
\newtheorem*{lemma*}{Lemma}
\newtheorem*{proposition*}{Proposition}
\newtheorem*{theorem*}{Theorem}

\theoremstyle{definition}
\newtheorem{definition}{Definition}
\newtheorem{example}{Examples}
\newtheorem{examples}[example]{Example}
\newtheorem{exercise}{Exercise}[chapter]
\newtheorem{problem}[exercise]{Problem}

\newtheorem*{example*}{Example}
\newtheorem*{exercise*}{Exercise}
\newtheorem*{problem*}{Problem}

\begin{document}
%% Footnotes
\renewcommand*{\thefootnote}{\fnsymbol{footnote}}

\chapterstyle{veelo}
\pagestyle{ruled}
\author{\href{mailto:\authoremail}{\documentauthor}}
\title{\documenttitle}
\date{\today}
\maketitle
%% Ulrich homework
% \chapter{Course notes}
Taken from Hungerford's \emph{Algebra}. This first section will cover the
relevant group theory part.
\section{Group Theory}
\subsection{Semigroups, Monoids and Groups}
If $G$ is a nonempty subset, a \emph{binary operation} on $G$ is a function
$G\times G\to G$. There are several commonly noted notations for the image
of $(a,b)$ under the binary operation: $ab$ (multiplicative notation),
$a+b$ (additive notation), $a\cdot b$, $a*b$, etc. For convenience we shall
generally use multiplicative notation throughout this chapter and refer to
$ab$ as the \emph{product} of $a$ and $b$. A set may have several binary
operations defined on it (for example, addition and multiplication on
$\bbZ$ given by $(a,b)\mapsto a+b$ or $(a,b)\mapsto ab$ respectively).
\begin{definition}
A \emph{semigroup} is a nonempty set $G$ together with a binary operation
on $G$ which is
\begin{enumerate}[label=\textnormal{(\alph*)}]
\item associative: $a(bc)=(ab)c$ for all $a,b,c\in G$;

  \bigskip
  a \emph{monoid}  is a semigroup $G$ which contains a
\item two-sided identity element $e\in G$ such that $ae=ea=a$ for all $a\in
  G$.

  \bigskip
  A \emph{group} is a monoid $G$ such that
\item for every $a\in G$ there exists a (two-sided) \emph{inverse} element
  $a^{-1}\in G$ such that $aa^{-1}=a^{-1}a=e$.

  \bigskip
  A semigroup $G$ is said to be \emph{Abelian} or \emph{commutative} if its
  binary operation is
\item commutative: $ab=ba$ for all $a,b\in G$.
\end{enumerate}
\end{definition}
Our principal interests are groups, however semigroups and monoids are
convenient for stating certain certain theorems in the most
generality. Examples are given below. The \emph{order} of a group $G$ is
the cardinality of the set $G$. $G$ is said to be finite if $|G|$ is
finite (otherwise, it is said to be infinite).
\begin{theorem}[1.2]
If $G$ is a monoid, then the identity element $e$ is unique. If $G$ is a
group, then
\begin{enumerate}[label=\textnormal{(\alph*)}]
\item $a\in G$ and $aa=a$ $\implies$ $a=e$;
\item for all $a,b,c\in G$, $ab=ac$ $\implies$ $b=c$ and $ba=ca$ $\implies$
  $b=c$ (left and right cancellation);
\item for each $a\in G$, the inverse element $a^{-1}$ is unique;
\item for each $a\in G$, $\left( a^{-1} \right)^{-1}=a$;
\item for $a,b\in G$, $(ab)^{-1}=b^{-1}a^{-1}$;
\item for $a,b\in G$ the equation $ax=b$ and $ya=b$ have unique solutions
  in $G$: $x=a^{-1}b$ and $y=ba^{-1}$.
\end{enumerate}
\end{theorem}

\begin{proposition}[1.3]
Let $G$ be a semigroup. Then $G$ is a group if and only if the following
conditions hold:
\begin{enumerate}[label=\textnormal{(\roman*)}]
\item there exists an identity element $e\in G$ such that $ea=a$ for all
  $a\in G$ (left identity element);
\item for each $a\in G$, there exists an element $a^{-1}\in G$ such that
  $a^{-1}a=e$ (left inverse).
\end{enumerate}
\end{proposition}
\begin{proof}[Sketch of the proof]
The direction $\implies$ is trivial. $\impliedby$: By Theorem 1.2(i) is
true under the hypotheses. $G\neq\emptyset$ since $e\in G$. If $a\in G$,
then (ii) $(aa^{-1})(aa^{-1})=a(a^{-1}a)a^{-1}=aea^{-1}=aa^{-1}$ and hence
$aa^{-1}=e$ by Theorem 1.2(i). Thus $a^{-1}$ is a two-sided inverse of
$a$. Since $ae=a(a^{-1}a)=(aa^{-1})a=ea=a$ for every $a\in G$, $e$ is a
two-sided identity. Therefore, $G$ is a group by Definition 1.1.
\end{proof}
\begin{proposition}[1.4]
Let $G$ be a semigroup. Then $G$ is a group if and only if for all $a,b\in
G$ the equations $ax=b$, $ya=b$ have solutions in $G$.
\end{proposition}
\begin{example}
Let $S$ be a nonempty set and $A(S)$ the set of all bijections $S\to
S$. Under the operation of composition of functions, $f\circ g$, $A(S)$
is a group, since composition is associative, composition of bijections is
a bijection, $1_S$ is a bijection, and every bijection has an inverse. The
elements of $A(S)$ are called \emph{permutations} and $A(S)$ is called the
group of permutations on the set $S$. If $S=\{1,\dotsc,n\}$, then $A(S)$ is
called the \emph{symmetric group on $n$ letters} and denoted $S_n$.

Since an element $\sigma$ of $S_n$ is a function on the finite set
$S=\{1,\dotsc,n\}$, it can be described by listing the elements of $S$ on a
line and the image of each element under $\sigma$ directly below it:
$\left(\begin{smallmatrix}1&2&\cdots&n\\i_1&i_2&\cdots&i_n\end{smallmatrix}\right)$. The
product $\sigma\tau$ of two elements of $S_n$ is the composition function
\emph{$\tau$ followed by $\sigma$}; that is, the function on $S$ given by
$k\mapsto \sigma(\tau(k))$. For instance, let
$\sigma=\left(\begin{smallmatrix}
1&2&3&4\\
3&1&2&4
\end{smallmatrix}\right)$ and
$\tau=\left(
\begin{smallmatrix}
1&2&3&4\\
4&1&2&3
\end{smallmatrix}
\right)$ be elements of $S_4$. Then, under
$\sigma\tau\mapsto\sigma(\tau(1))=\sigma(4)=4$, etc.; thus
\[
\sigma\tau=
\begin{pmatrix}
1&2&3&4\\
4&3&1&2
\end{pmatrix}
\]
and
\[
\tau\sigma=
\begin{pmatrix}
1&2&3&4\\
2&4&1&3
\end{pmatrix}.
\]
This example shows that $S_n$ need not be Abelian.
\end{example}

%%% Local Variables:
%%% mode: latex
%%% TeX-master: "../MA553-HW-Current"
%%% End:

% \section{Ring Theory}

%%% Local Variables:
%%% mode: latex
%%% TeX-master: "../MA553-HW-Current"
%%% End:

% \section{Field Theory}

%%% Local Variables:
%%% mode: latex
%%% TeX-master: "../MA553-HW-Current"
%%% End:

\chapter{Homework 1}
We'll discuss some important theorems and results as we solve some of these
problems.
\begin{problem}
Let $G$ be a group, $a\in G$ an element of finite order $m$, and $n$ a
positive integer. Prove that
\[
|a^n|=\frac{m}{\gcd(m,n)}.
\]
\end{problem}
\begin{proof}

\end{proof}

\begin{problem}
Let $G$ be a group, and let $a$, $b$ be elements of finite order $m$, $n$
respectively. Show that if $ba=ab$ and $\langle a\rangle\cap\langle
b\rangle=\{e\}$, then $|ab|=\lcm(m,n)$.
\end{problem}
\begin{proof}
\end{proof}

\begin{problem}
Let $G$ be a group and $H$, $K$ normal subgroups with $H\cap K=\{e\}$. Show
that
\begin{enumerate}[label=(\alph*)]
\item $hk=kh$ for every $h\in H$, $k\in K$.
\item $HK$ is a subgroup of $G$ with $HK\cong H\times K$.
\end{enumerate}
\end{problem}
\begin{proof}
\end{proof}

\begin{problem}
Show that $A_4$ has no subgroup of order $6$ (although $6\mid 12=|A_4|$).
\end{problem}
\begin{proof}
\end{proof}

%%% Local Variables:
%%% mode: latex
%%% TeX-master: "../MA553-HW-Current"
%%% End:

\section{Homework 2}
\begin{problem}
Let $G$ be the group of order $2^3\cdot 3$, $n\geq 2$. Show that $G$ has a
normal $2$-subgroup $\neq\left\{e\right\}$.
\end{problem}
\begin{proof}
\end{proof}

\begin{problem}
Let $G$ be a group of order $p^2q$, $p$ and $q$ primes. Show that the Sylow
$p$-Sylow subgroup or the $q$-Sylow subgroup of $G$ is normal in $G$.
\end{problem}
\begin{proof}
\end{proof}

\begin{problem}
Let $G$ be a subgroup of order $pqr$, $p<q<r$ primes. Show that the
$r$-Sylow subgroup of $G$ is normal in $G$.
\end{problem}
\begin{proof}
\end{proof}

\begin{problem}
Let $G$ be a group of order $n$ and let $\varphi\colon G\to S_n$ be given by
the action of $G$ on $G$ via translation.
\begin{enumerate}[label=(\alph*)]
\item For $a\in G$ determine the number and the lengths of the disjoint
  cycles of the permutation $\phi(a)$.
\item Show that $\varphi(G)\nsubset A_n$ if and only if $n$ is even and $G$
  has a cyclic $2$-Sylow subgroup.
\item If $n=2m$, $m$ odd, show that $G$ has a subgroup of index $2$.
\end{enumerate}
\end{problem}
\begin{proof}
\end{proof}

\begin{problem}
Show that the only simple groups $\neq\left\{e\right\}$ of order $<60$ are
the groups of prime order.
\end{problem}
\begin{proof}
\end{proof}

%%% Local Variables:
%%% mode: latex
%%% TeX-master: "../MA553-HW-Current"
%%% End:

\chapter{Homework 3}
\begin{problem}
Let $G$ be a finite group, $p$ a prime number, $N$ the intersection of all
$p$-Sylow subgroups of $G$. Show that $N$ is a normal $p$-subgroup of $G$
and that every normal $p$-subgroup of $G$ is contained in $N$.
\end{problem}
\begin{proof}
\end{proof}

\begin{problem}
Let $G$ be a group of order $231$ and let $H$ be an $11$-Sylow subgroup of
$G$. Show that $H\subset Z(G)$.
\end{problem}
\begin{proof}
\end{proof}

\begin{problem}
Let $G=\left\{e,a_1,a_2,a_3\right\}$ be a non-cyclic group of order $4$ and
define $\varphi\colon S_3\to\Aut(G)$ by $\varphi(\sigma)(e)=e$ and
$\varphi(\sigma)(a_1)=a_{\sigma(i)}$. Show that $\varphi$ is well-defined and an
isomorphism of groups.
\end{problem}
\begin{proof}
\end{proof}

\begin{problem}
Determine all groups of order $18$.
\end{problem}
\begin{proof}
\end{proof}

%%% Local Variables:
%%% mode: latex
%%% TeX-master: "../MA553-HW-Current"
%%% End:

\chapter{Homework 4}

%%% Local Variables:
%%% mode: latex
%%% TeX-master: "../MA553-HW-Current"
%%% End:

\chapter{Homework 5}
\begin{problem}
Find all composition series and the composition factors of $D_6$.
\end{problem}
\begin{proof}
\end{proof}

\begin{problem}
Let $T$ be the subgroup of $\GL_n(\bfR)$ consisting of all upper triangular
invertible matrices. Show that $T$ is solvable.
\end{problem}
\begin{proof}
\end{proof}

\begin{problem}
Let $p\in\bfZ$ be a prime number. Show:
\begin{enumerate}[label=(\alph*)]
\item $(p-1)!\equiv -1\mod{p}$.
\item If $p\equiv 1\mod{4}$ then $x^2\equiv -1\mod{p}$ for some
  $x\in\bfZ$.
\end{enumerate}
\end{problem}
\begin{proof}
\end{proof}

\begin{problem}
\begin{enumerate}[label=(\alph*)]
\item Show that the following are equivalent for an odd prime number
  $p\in\bfZ$:
  \begin{enumerate}[label=(\roman*)]
  \item $p\equiv 1\mod 4$.
  \item $p=a^2+b^2$ for some $a$, $b$ in $\bfZ$.
  \item $p$ is not prime in $\bfZ[i]$.
  \end{enumerate}
\item Determine all prime ideals of $\bfZ[i]$.
\end{enumerate}
\end{problem}
\begin{proof}
\end{proof}

%%% Local Variables:
%%% mode: latex
%%% TeX-master: "../MA553-HW-Current"
%%% End:

\subsection{Homework 6}
\begin{problem}
Let $R$ be a domain. Show that $R$ is a UFD if and only if every nonzero
nonunit in $R$ is a product of irreducible elemnets and the intersection of
any two principal ideals is again principal.
\end{problem}
\begin{proof}
\end{proof}

\begin{problem}
Let $R$ be a PID and $p$ a prime ideal of $R[X]$. Show that $p$ is
principal or $p=(a,f)$ for some $a\in R$ and some monic $f\in R[X]$.
\end{problem}
\begin{proof}
\end{proof}

\begin{problem}
Let $k$ be a field and $n\geq 1$. Show that $Z^n+Y^3+X^2\in k(X,Y)[Z]$ is
irreducible.
\end{problem}
\begin{proof}
\end{proof}

\begin{problem}
Let $k$ be a field of characteristic zero and $n\geq 1$, $m\geq 2$. Show
that ${X_1}^n+\dotsb+{X_m}^n-1\in k[X_1,\dotsc,X_m]$ is irreducible.
\end{problem}
\begin{proof}
\end{proof}

\begin{problem}
Show that $X^{3^n}+2\in\bbQ(i)[X]$ is irreducible.
\end{problem}
\begin{proof}
\end{proof}

%%% Local Variables:
%%% mode: latex
%%% TeX-master: "../MA553-Quals"
%%% End:

\chapter{Homework 7}
\begin{problem}
Let $k\subset K$ and $k\subset L$ be finite field extensions contained in
some field. Show that:
\begin{enumerate}[label=(\alph*)]
\item $[KL:L]\leq [K:k]$.
\item $[KL:k]\leq [K:k][L:k]$.
\item $K\cap L=k$ if equality holds in (b).
\end{enumerate}
\end{problem}
\begin{proof}
\end{proof}

\begin{problem}
Let $k$ be a field of characteristic $\neq 2$ and $a,b$ elements of $k$ so
that $a$, $b$, $ab$ are not squares in $k$. Show that
$\bigl[k{(\sqrt{a},\sqrt{b})}:k\bigr]=4$.
\end{problem}
\begin{proof}
\end{proof}

\begin{problem}
Let $R$ be a UFD, but not a field, and write $K=\Quot(R)$. Show that $[\bar
K:k]=\infty$.
\end{problem}
\begin{proof}
\end{proof}

\begin{problem}
Let $k\in K$ be an algebraic field extension. Show that every
$k$-homomorphism $\delta\colon K\to K$ is an isomorphism.
\end{problem}
\begin{proof}
\end{proof}

\begin{problem}
Let $K$ be the splitting field of $x^6-4$ over $\bfQ$. Determine $K$ and
$[K:\bfQ]$.
\end{problem}
\begin{proof}
\end{proof}

%%% Local Variables:
%%% mode: latex
%%% TeX-master: "../MA553-HW-Current"
%%% End:

\chapter{Homework 8}
\begin{problem}
Let $k$ be a field, $f\in k[X]$ a polynomial of degree $n\geq 1$, and $K$
the splitting field of $f$ over $k$. Show that $[K:k]\mid n!$.
\end{problem}
\begin{proof}
\end{proof}

\begin{problem}
Let $k$ be a field and $n\geq 0$. Define a map $\Delta_n\colon k[X]\to
k[X]$ by $\Delta_n\left(\sum a_ix^i\right)\coloneqq\sum a_i\binom{i}{n}
x^{i-n}$. Show that
\begin{enumerate}[label=(\alph*)]
\item $\Delta_n$ is $k$-linear, and for $f,g\in k[X]$,
  $\Delta_n(fg)=\sum_{j=0}^n\Delta_j(f)\Delta_{n-j}(g)$.
\item $f^{(n)}=n!\Delta_n(f)$.
\item $f(x+a)=\sum\Delta_n(f)(a)x^n$.
\item $a\in k$ is a root of $f$ of multiplicity $n$ if and only if
  $\Delta_i(f)(a)=0$ for $0\leq i\leq n-1$ and $\Delta_n(f)(a)\neq 0$.
\end{enumerate}
\end{problem}
\begin{proof}
\end{proof}

\begin{problem}
Let $k\subset K$ be a finite field extension. Show that $k$ is perfect if
and only if $K$ is perfect.
\end{problem}
\begin{proof}
\end{proof}

\begin{problem}
Let $K$ be the splitting field of $x^p-x-1$ over $k=\bbZ/p\bbZ$. Show that
$k\subset K$ is normal, separable, of degree $p$.
\end{problem}
\begin{proof}
\end{proof}

\begin{problem}
Let $k$ be a field of characteristic $p>0$, and $k(X,Y)$ the field of
rational functions in two variables.
\begin{enumerate}[label=(\alph*)]
\item Show that $\bigl[k(X,Y):k(X^p,Y^p)\bigr]=p^2$.
\item Show that the extension $k(X^p,Y^p)\subset k(X,Y)$ is not simple.
\item Find infinitely many distinct fields $L$ with $k(X^p,Y^p)\subset
  L\subset k(X,Y)$.
\end{enumerate}
\end{problem}
\begin{proof}
\end{proof}

%%% Local Variables:
%%% mode: latex
%%% TeX-master: "../MA553-HW-Current"
%%% End:

\section{Homework 9}
\begin{problem}
Let $k\subset K$ be a finite extension of fields of characteristic
$p>0$. Show that if $p\nmid [K:k]$, then $k\subset K$ is separable.
\end{problem}
\begin{proof}
\end{proof}

\begin{problem}
Let $k\subset K$ be an algebraic extension of fields of characteristic
$p>0$, let $L$ be an algebraically closed field containing $K$, and let
$\delta\colon k\to L$ be an embedding. Show that $k\subset K$ is purely
inseparable if and only if there exists exactly one embedding $\tau\colon
K\to L$ extending $\delta$.
\end{problem}
\begin{proof}
\end{proof}

\begin{problem}
Let $k\subset K=k(\alpha,\beta)$ be an algebraic extension of fields of
characteristic $p>0$, where $\alpha$ is separable over $k$ and $\beta$ is
purely inseparable over $k$. Show that $K=k(\alpha+\beta)$.
\end{problem}
\begin{proof}
\end{proof}

\begin{problem}
Let $f(x)\in\bfF_q[X]$ be irreducible. Show that $f(X)\mid X^{q^n}-X$ if
and only if $\deg f(X)\mid n$.
\end{problem}
\begin{proof}
\end{proof}

\begin{problem}
Show that $\Aut_{\bfF_q}(\bar\bfF_q)$ is an infinite Abelian group
which is torsionfree (i.e., $\delta^n=\id$ implies $\delta=\id$ or $n=0$).
\end{problem}
\begin{proof}
\end{proof}

\begin{problem}
Show that in a finite field, every element can be written as a sum of two
perfect squares.
\end{problem}
\begin{proof}
\end{proof}

%%% Local Variables:
%%% mode: latex
%%% TeX-master: "../MA553-HW-Current"
%%% End:


%% References
\bibliographystyle{plainnat}
\bibliography{alg-bib}
\end{document}

%%% Local Variables:
%%% mode: latex
%%% TeX-master: t
%%% End:
