\section{Differentiation}
This portion of the notes corresponds to material covered before the final.

This section deals with questions of differentiability and culminates with
a couple of results tying together the Lebesgue integral with the
derivative à la the familiar fundamental theorem of calculus for Riemann
integrals.

\subsection{The indefinite integral}
If $f$ is a Riemann integrable function on an interval $[a,b]$ of $\bbR$,
then the familiar definition for its
\href{https://en.wikipedia.org/wiki/Antiderivative}{\emph{indefinite
    integral}} is
\begin{equation}
\label{eq:3:indefinite-integral}
\begin{aligned}
F(x)&=\int_a^xf(y)dy,&a\leq x\leq b.
\end{aligned}
\end{equation}
The
\href{https://en.wikipedia.org/wiki/Fundamental_theorem_of_calculus}{\emph{fundamental
    theorem of calculus}} then asserts that $F'=f$ if $f$ is continuous. In
this section, we study the analogue of this result for Lebesgue integrable
functions.

Since we want to generalize our results to $\bbR^n$, first we must find a
suitable notion of indefinite integral for multivariable functions. In two
dimensions we might, for instance, define the indefinite integral $F$ of
$f$ to be
\begin{equation}
\label{eq:3:2-dim-indefinite-integral}
F(x_1,x_2)\coloneqq \int_{a_1}^{x_1}\int_{a_2}^{x_2}f(y_1,y_2)\diff
y_2\diff y_1.
\end{equation}

As it turns out, it is better to abandon the notion that the indefinite
integral be a function of a point an instead let it be a function of
a set. Therefore, given a function $f$, integrable on some measurable
subset $A$ of $\bbR^n$, we define the \emph{indefinite integral of $f$} to
be the function
\begin{equation}
  \label{eq:3:lebesgue-indefinite-integral}
F(E)\coloneqq \int_E f,
\end{equation}
where $E$ is a measurable subset of $A$.

The function $F$ is an example of a
\href{https://en.wikipedia.org/wiki/Set_function}{\emph{set function}}, by
which we mean any real-valued function $F$ defined on a $\sigma$-algebra
$\Sigma$ of measurable sets such that
\begin{enumerate}[label=(\roman*),noitemsep]
\item $F(E)$ is finite for every $E\in\Sigma$.
\item F is
  \href{https://en.wikipedia.org/wiki/Measure_(mathematics)#Properties}{\emph{countably
      additive}}; i.e., if $E$ is the union of disjoint sets
  $E_k\in\Sigma$, $k=1,2,\dotsc$, then
  \begin{equation}
    \label{eq:3:countably-additive-set-function}
    F(E)\sum_{k\in\bbN} F(E_k).
  \end{equation}
\end{enumerate}

%%% Local Variables:
%%% mode: latex
%%% TeX-master: "../MA544-Quals"
%%% End:
