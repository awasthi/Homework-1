\section{Final Exam Review}
Material covered since exam 2.

\bigskip

If $f$ is a Riemann integrable function on an interval $[a,b]$ in $\bbR$,
then the familiar definition of its indefinite integral is
\[
F(x)=\int_a^x f(y)dy,\qquad a\leq x\leq b.
\]
The fundamental theorem of calculus asserts that $F'=f$ if $f$ is
continuous. We will study an analogue of this result for Lebesgue
integrable $f$ and higher dimensions.

We must first find an appropriate definition of the indefinite integral. In
two dimensions, for example, we might choose
\[
F(x_1,x_2)=\int_{a_1}^{x_1}\int_{a_2}^{x_2}f(y_1,y_2)dy_1dy_2.
\]
It turns out, however, to be better to abandon the notion that the
indefinite integral be a function of point and adopt the idea that it be a
function of set. Thus, given $f\in L(A)$, where $A$ is a measurable
subset of $\bbR^n$, we define the \emph{indefinite integral of $f$} to be
the function
\[
F(E)=\int_E f,
\]
where $E$ is any measurable subset of $A$.

$F$ is an example of a \emph{set function}, by which we mean any
real-valued function $F$ defined on a $\sigma$-algebra $\Sigma$ of
measurable sets such that
\begin{enumerate}[label=(\roman*)]
\item $F(E)$ is finite for every $E\in\Sigma$.
\item $F$ is \emph{countably additive}; that is, if $E=\bigcup_k E_k$ is a
  union of disjoint $E_k\in\Sigma$, then
\[
F(E)=\sum_k F(E_k).
\]
\end{enumerate}
By Theorem 5.5 and 5.24, the indefinite integral of $f\in L(A)$ satisfies
(i) and (ii) for the $\sigma$-algebra of measurable subsets of $A$.

Recall that the diameter of a set $E$ is the value
\[
\sup\left\{\,\|\bfx-\bfy\|:\bfx,\bfy\in E\,\right\}.
\]
A set function $F(E)$ is called \emph{continuous} if $F(E)$ tends to zero
as the diameter of $E$ tends to zero; i.e., $F(E)$ is continuous if, given
$\varepsilon>0$, there exists a $\delta>0$ such that $|F(E)|<\varepsilon$
whenever the diameter of $E$ is less than $\delta$. An example of a
function that is \emph{not} continuous can be obtained by setting $F(E)=1$
for any measurable set that contains the origin, and $F(E)=0$
otherwise.\footnote{Why is this function not continuous. Consider the
  following argument: Let $\varepsilon=1/2$ and let $B_k=
  B(\mathbf{0},1/k)$. Then as the diameter of $B_k$ goes to zero,
  $F(B_k)=1$ for all $k$ so $F(B_k)\to 1>1/2$.}

A set function $F$ is called \emph{absolutely continuous with respect to
  the Lebesgue measure}, or simply \emph{absolutely continuous} if $F(E)$
tends to zero as the measure of $E$ tends to zero. Thus, $F$ is absolutely
continuous if given a $\varepsilon>0$ there exists $\delta>0$ such that
$|F(E)|<\varepsilon$ whenever the measure of $E$ is less than $\delta$.

A set function that is absolutely continuous is clearly
continuous\footnote{Suppose $F$ is absolutely continuous. Then, given
  $\varepsilon>0$ there exists $\delta>0$ such that $|F(E)|<\varepsilon$
  whenever $|E|<\delta$.}, however, the converse is false, as shown in the
following example. Let $A$ be the unit square in $\bbR^2$, let $D$ be the
diagonal of $A$, and consider the $\sigma$-algebra of measurable subsets
$E$ of $A$ for which $E\cap D$ is linearly measurable. For such $E$, let
$F(E)$ be the linear measure of $E\cap D$. Then $F$ is a continuous set
function. However, it is not absolutely continuous since the sets $E$
containing a fixed segment of $D$ whose $\bbR^2$-measures are arbitrarily
small.

\begin{theorem}[7.1]
If $f\in L(A)$, its definite integral is absolutely continuous.
\end{theorem}
\begin{proof}
We may assume that $f\geq 0$ by considering $f^+$ and $f^-$. Fix $k$ and
write $f=g+h$, where $g=f$ whenever $f\leq k$ and $g=k$ otherwise. Given
$\varepsilon>0$, we may choose $k$ so large that $0\leq\int_A
h<\varepsilon/2$ and, \emph{a fortiori}, $0\leq\int_E f<\varepsilon/2$.
Since
\[
\int|f-C|\leq\int|f-f_{k_0}|+\int|f_{k_0}-C|<\frac{\varepsilon}{2}+\frac{\varepsilon}{2}=\varepsilon.
\]
we see that $f$ has property $\calA$.

To prove the lemma, let $f\in L(\bbR)$. Writing $f=f^+-f^-$, we may assume
that $f\geq 0$. Then
\[
\int|\chi_G-\chi_E|=|G\smallsetminus E|<\varepsilon.
\]
so we may assume that $f=\chi_G$ for open set $G$ of finite measure. Using
Theorem 1.11, write $G$ as the union of (partly open) disjoint intervals
$G=\bigcup I_k$. If we let $f_N$ be the characteristic function of
$\bigcup_{k=1}^N I_k$, we obtain
\[
\int|f-f_N|=\sum_{k=N+1}^\infty|I_k|\to 0
\]
since $\sum_k|I_k|=|G|<\infty$, i.e., the series converges. By (2), it is
enough to show that each $f_N$ has property $\calA$. But $f_N$ is the sum
$\chi_{I_k}$, $k=1,\dotsc,N$, so it suffices by (1) to show that the
characteristic function of any partly open interval $I$ has property
$\calA$. This is practically self-evident: if $I^*$ denotes an interval
that contains the closure of $I$ in its interior and that satisfies
$|I^*\smallsetminus I|<\varepsilon$, then for any continuous $C$, $0\leq
C\leq 1$, which is $1$ in $I$ and $0$ outside $I^*$, we have
\[
\int|chi_I-C|\leq|I^*-I|<\varepsilon.
\]
\end{proof}

\begin{theorem}[Simple Vitali lemma]
Let $E$ be a subset of $\bbR^n$ with $|E|_e<\infty$, and let $K$ be a
collection of cubes $Q$ covering $E$, then there exists a positive constant
$\beta$, depending only on $n$, and a finite number of disjoint cubes,
$Q_1,\dotsc,Q_N$ in $K$ such that
\[
\sum_{j=1}^N|Q_j|\geq\beta|E|_e
\]
\end{theorem}

As an application of Vitali's covering lemma, we will prove some basic
result concerning the differentiability of monotone functions on $\bbR$. If
$f(x)$ is a real-valued function defined and finite in a neighborhood of
$x_0$, consider the four \emph{Dini numbers} (or \emph{derivatives}),
\[
\begin{aligned}
D_1f(x_0)&=\limsup_{h\to 0+}\frac{f(x_0+h)-f(x_0)}{h},\\
D_2f(x_0)&=\limsup_{h\to 0+}\frac{f(x_0+h)-f(x_0)}{h},\\
D_3f(x_0)&=\limsup_{h\to 0-}\frac{f(x_0+h)-f(x_0)}{h},\\
D_4f(x_0)&=\limsup_{h\to 0-}\frac{f(x_0+h)-f(x_0)}{h}.
\end{aligned}
\]
Clearly, $D_2f\leq D_1f$ and $D_4f\leq D_3f$. If all four Dini numbers are
equal, that is, if $\lim_{h\to 0}[f(x_0+h)-f(x_0)]/h$ exist, finite or
infinite, we say that $f$ has a \emph{derivative at $x_0$} and call the
value the \emph{derivative $f'(x_0)$} at $x_0$. Thus, $-\infty\leq
f'(x_0)\leq\infty$ if $f'(x_0)$ exists.
\begin{theorem}[7.21]
Let $f$ be monotone increasing function and finite on an open interval
$(a,b)\subset\bbR$. Then $f$ has a measurable, nonnegative, finite
derivative $f'$ a.e.\@ in $(a,b)$. Moreover,
\[
0\leq\int_a^b f'\leq f(b-)-f(a+).
\]
\end{theorem}
\begin{proof}
We may assume that $(a,b)$ is finite; the general case follows from this by
passage to the limit: We will show that the set
$\left\{\,x\in(a,b):D_1f(x)>D_4f(x)\,\right\}$ has measure zero. A similar
argument will apply to any two Dini numbers of $f$. It is enough to show
that each set
\[
A_{r,s}\coloneqq\left\{\,x\in(a,b):D_1f(x)>r>s>D_4f(x)\,\right\}
\]
has measure zero since the original set is the union of these over rational
$r$ and $s$.
\end{proof}

%%% Local Variables:
%%% mode: latex
%%% TeX-master: "../MA544-Quals"
%%% End:
