\section{Final Exam Review}
Material covered since exam 2.

\bigskip

If $f$ is a Riemann integrable function on an interval $[a,b]$ in $\bfR$,
then the familiar definition of its indefinite integral is
\[
F(x)\coloneqq\int_a^x f(y)dy,\qquad a\leq x\leq b.
\]
The fundamental theorem of calculus asserts that $F'=f$ if $f$ is
continuous. We will study an analogue of this result for Lebesgue
integrable $f$ and higher dimensions.

We must first find an appropriate definition of the indefinite integral. In
two dimensions, for example, we might choose
\[
F(x_1,x_2)\coloneqq\int_{a_1}^{x_1}\int_{a_2}^{x_2}f(y_1,y_2)dy_1dy_2.
\]
It turns out, however, to be better to abandon the notion that the
indefinite integral be a function of point and adopt the idea that it be a
function of set. Thus, given $f\in L^1(A)$, where $A$ is a measurable
subset of $\bfR^n$, we define the \emph{indefinite integral of $f$} to be
the function
\[
F(E)\coloneqq\int_E f,
\]
where $E$ is any measurable subset of $A$.

$F$ is an example of a \emph{set function}, by which we mean any
real-valued function $F$ defined on a $\sigma$-algebra $\Sigma$ of
measurable sets such that
\begin{enumerate}[label=(\roman*)]
\item $F(E)$ is finite for every $E\in\Sigma$.
\item $F$ is \emph{countably additive}; that is, if $E=\bigcup_k E_k$ is a
  union of disjoint $E_k\in\Sigma$, then
\[
F(E)=\sum_k F(E_k).
\]
\end{enumerate}
By Theorem 5.5 and 5.24, the indefinite integral of $f\in L^1(A)$ satisfies
(i) and (ii) for the $\sigma$-algebra of measurable subsets of $A$.

Recall that the diameter of a set $E$ is the value
\[
\sup\left\{\,\|\bfx-\bfy\|:\bfx,\bfy\in E\,\right\}.
\]
A set function $F(E)$ is called \emph{continuous} if $F(E)$ tends to zero
as the diameter of $E$ tends to zero; i.e., $F(E)$ is continuous if, given
$\varepsilon>0$, there exists a $\delta>0$ such that $|F(E)|<\varepsilon$
whenever the diameter of $E$ is less than $\delta$. An example of a
function that is \emph{not} continuous can be obtained by setting $F(E)=1$
for any measurable set that contains the origin, and $F(E)=0$
otherwise.\footnote{Why is this function not continuous. Consider the
  following argument: Let $\varepsilon=1/2$ and let $B_k\coloneqq
  B(\mathbf{0},1/k)$. Then as the diameter of $B_k$ goes to zero,
  $F(B_k)=1$ for all $k$ so $F(B_k)\to 1>1/2$.}

A set function $F$ is called \emph{absolutely continuous with respect to
  the Lebesgue measure}, or simply \emph{absolutely continuous} if $F(E)$
tends to zero as the measure of $E$ tends to zero. Thus, $F$ is absolutely
continuous if given a $\varepsilon>0$ there exists $\delta>0$ such that
$|F(E)|<\varepsilon$ whenever the measure of $E$ is less than $\delta$.

A set function that is absolutely continuous is clearly
continuous\footnote{Suppose $F$ is absolutely continuous. Then, given
  $\varepsilon>0$ there exists $\delta>0$ such that $|F(E)|<\varepsilon$
  whenever $|E|<\delta$.}

%%% Local Variables:
%%% mode: latex
%%% TeX-master: "../MA544-Quals"
%%% End:
