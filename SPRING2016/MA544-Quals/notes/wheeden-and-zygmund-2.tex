\section{Exam 2 Review}
This is all of the material we covered before exam 2.

\bigskip

Let $f$ be defined on $E$, and let $\bfx_0$ be a limit point of $E$ in
$E$. Then $f$ is said to be \emph{upper semicontinuous at $\bfx_0$} if
\begin{equation}
  \label{eq:upper-semicontinuous}
\limsup_{\substack{\bfx\to\bfx_0\\\bfx\in E}}f(\bfx)\leq f(\bfx_0).
\end{equation}
Note that if $f(\bfx_0)=\infty$, then $f$ is usc at $\bfx_0$
automatically; otherwise, the statement that $f$ is usc at $\bfx_0$ means
that given any $M>f(\bfx_0)$, there exists $\delta>0$ such that $f(\bfx)<M$
for all $\bfx\in E$ that lie in the ball $B_\delta(\bfx_0)$.

Similarly, $f$ is said to be \emph{lower semicontinuous at $\bfx_0$} if
$-f$ is usc at $\bfx_0$.

\begin{theorem*}[4.14]
A function $f$ is usc relative to $E$ if and only if $\left\{\,\bfx\in
  E:f(\bfx)>a\,\right\}$ is relatively closed (equivalently, if
$\left\{\,\bfx\in E:f(\bfx)<a\,\right\}$ is relatively open) for all finite
$a$
\end{theorem*}
\begin{proof}[Proof of theorem 4.14]
Suppose that $f$ is usc relative to $E$. Given $a$, let $\bfx_0$ be a limit
point of $\left\{\,\bfx\in E:f(\bfx)>a\,\right\}$ in $E$. Then there exists
$\bfx_k\in E$ such that $\bfx_k\to\bfx_0$ and $f(\bfx_k)\geq a$. Since $f$
is usc at $\bfx_0$, we have $f(\bfx_0)\geq\limsup_{k\to\infty}
f(\bfx_k)$. Therefore, $f(\bfx_0)\geq a$, so $\bfx_0\in\left\{\,\bfx\in
  E:f(\bfx)>a\,\right\}$. Hence, $\left\{\,\bfx\in E:f(\bfx)>a\,\right\}$
is relatively closed.

Conversely, let $\bfx_0$ be a limit point of $E$ that is in $E$. If $f$ is
not usc at $\bfx_0$, then $f(\bfx_0)<\infty$, and there exists $M$ and
$\left\{ \bfx_k \right\}$ such that $f(\bfx_0)<M$, $\bfx_k\in E$,
$\bfx_k\to\bfx_0$, and $f(\bfx_k)\geq M$. Hence, $\left\{\,\bfx\in
  E:f(\bfx)>a\,\right\}$ is not relatively closed since it does not contain
all its limit points in $E$>
\end{proof}
\begin{theorem*}[4.17, Egorov's theorem]
Suppose that $\{f_k\}$ is a sequence of measurable functions that converge
a.e.\@ in a set $E$ of finite measure to a finite limit $f$. Then given
$\varepsilon>0$ there exits a closed subset $F$ of $E$ such that $|E\smallsetminus
F|<\varepsilon$ and $f_k\to f$ uniformly on $F$.
\end{theorem*}
A function $f$ defined on a measurable set $E$ has \emph{property $\calC$}
on $E$ if given $\varepsilon>0$, there is a closed set $F\subset E$ such
that
\begin{enumerate}[label=(\roman*)]
\item $|E\smallsetminus F|<\varepsilon$
\item $f$ is continuous relative to $F$.
\end{enumerate}
\begin{theorem*}[4.20, Lusin's theorem]
Let $f$ be defined and finite on a measurable set $E$. Then $f$ is
measurable if and only if it has property $\calC$ on $E$.
\end{theorem*}

We start with a nonnegative function $f$ defined on a measurable subset $E$
of $\bfR^n$. Let's
\begin{equation}
\label{eq:graph-and-region}
\begin{aligned}
\Gamma(f,E)&\coloneqq\left\{\,(\bfx,f(\bfx))\in\bfR^{n+1}:\text{$\bfx\in
    E$, $f(\bfx)<\infty$}\,\right\},\\
R(f,E)&\coloneqq\left\{\,(\bfx,y)\in\bfR^{n+1}:\text{$\bfx\in E$, $0\leq
    y\leq f(\bfx)$ if $f(\bfx)<\infty$ and $0\leq y<\infty$ if
    $f(\bfx)=\infty$}\,\right\}.
\end{aligned}
\end{equation}
$\Gamma(f,E)$ is called the \emph{graph of $f$ over $E$} and $R(f,E)$ the
\emph{region under $f$ over $E$}.

If $R(f,E)$ is measurable (as a subset of $\bfR^{n+1}$), its measure
$|R(f,E)|_{\bfR^{n+1}}$ is called the \emph{Lebesgue integral over $E$},
and we write
\begin{equation}
\label{eq:lebesgue-integral}
\int_E f(\bfx) d \bfx\coloneqq|R(f,E)|_{\bfR^{n+1}}.
\end{equation}
This is sometimes written as
\[
\int_E f
\]
or at times the lengthy notation
\[
\idotsint\limits_{E} f(x_1,\dotsc,x_n) d  x_1\cdots d  x_n
\]
is convenient.
\begin{theorem*}[5.1]
Let $f$ be a nonnegative function defined on a measurable set $E$. Then
$\int_E f$ exists if and only if $f$ is measurable.
\end{theorem*}
\begin{lemma*}[5.3]
If $f$ is a nonnegative measurable function on $E$, $0\leq |E|\leq\infty$,
then $|\Gamma(f,E)|=0$.
\end{lemma*}
\begin{theorem*}[5.5]
\begin{enumerate}[label=\textnormal{(\roman*)}]
\item If $f$ and $g$ are measurable and if $0\leq g\leq f$ on $E$, $\int_E
  g\leq\int_E f$. In particular, $\int_E\inf f\leq\int_E f$.
\item If $f$ is nonnegative and measurable on $E$ and if $\int_E f$ is
  finite, then $f<\infty$ a.e.\@ in $E$.
\item Let $E_1$ and $E_2$ be measurable and $E_1\subset E_2$. If $f$ is
  nonnegative and measurable on $E_2$, then $\int_{E_1} f\leq\int_{E_2}f$.
\end{enumerate}
\end{theorem*}
\begin{theorem*}[5.6, the monotone convergence theorem for nonnegative functions]
If $\{f_k\}$ is a sequence of nonnegative functions such that $f_k\nearrow
f$ on $E$, then
\[
\int_E f\to\int_E f.
\]
\end{theorem*}
\begin{proof}
By Theorem 4.12, $f$ is measurable since it is the limit of a sequence of
measurable functions. Since $R(f_k,E)\cup\Gamma(f,E)\nearrow R(f,E)$ and
$|\Gamma(f,E)|=0$, the result follows by Theorem 3.26 on the measure of a
monotone convergent sequences of measurable sets.
\end{proof}
\begin{theorem*}[5.9]
Let $f$ be nonnegative on $E$. If $|E|=0$, then $\int_E f=0$.
\end{theorem*}
\begin{theorem*}[5.10]
If $f$ and $g$ are nonnegative and measurable on $E$ and if $g\leq f$
a.e.\@ in $E$, then $\int_E g\leq\int_E f$.

In particular, if $f=g$ a.e.\@ in $E$, then $\int_E f=\int_E g$.
\end{theorem*}
\begin{theorem*}[5.11]
Let $f$ be nonnegative and measurable on $E$. Then $\int_E f=0$ if and only
if $f=0$ a.e.\@ in $E$.
\end{theorem*}
\begin{corollary*}[5.12, Chebyshev's inequality]
Let $f$ be nonnegative and measurable on $E$. If $a>0$, then
\[
\frac{1}{a}\int_E f\geq\left|\left\{\,\bfx\in E:f(\bfx)>a\,\right\}\right|.
\]
\end{corollary*}
\begin{theorem*}[5.13]
If $f$ is nonnegative and measurable, and if $c$ is any nonnegative
constant, then
\[
\int_E cf=c\int_E f.
\]
\end{theorem*}
\begin{theorem*}[5.14]
If $f$ and $g$ are nonnegative and measurable, then
\[
\int_E (f+g)=\int_E f+\int_E g.
\]
\end{theorem*}
\begin{corollary*}
Suppose that $f$ and $\varphi$ are measurable on $E$, $0\leq f\leq\varphi$,
and $\int_E f$ is finite. Then
\[
\int_E (\varphi-f)=\int_E\varphi-\int_E f.
\]
\end{corollary*}
\begin{theorem*}[5.16]
If $f_k$, $k=1,2,\dotsc$, are nonnegative and measurable, then
\[
\int_E\sum_{k=1}^\infty f_k=\sum_{k=1}^\infty\int_E f_k.
\]
\end{theorem*}
\begin{theorem*}[5.17, Fatou's lemma]
If $\{f_k\}$ is a sequence of nonnegative measurable functions on $E$, then
\[
  \int_E\liminf_{k\to\infty} f_k\leq\liminf_{k\to\infty}\int_E f_k.
\]
\end{theorem*}
\begin{proof}[Proof of Fatou's lemma]
\end{proof}
\begin{theorem*}[5.19, Lebesgue's dominated convergence theorem for
  nonnegative functions]
Let $\{f_k\}$ be a sequence of nonnegative measurable functions on $E$ such
that $f_k\to f$ a.e.\@ in $E$. If there exists a measurable function
$\varphi$ such that $f_k\leq\varphi$ a.e.\@ for all $k$ and if
$\int_E\varphi$ is finite, then
\[
\int_E f_k\longrightarrow\int_E f.
\]
\end{theorem*}
\begin{theorem*}[5.21]
Let $f$ be measurable in $E$. Then $f$ is integrable over $E$ if and only
if $|f|$ is.
\end{theorem*}
\begin{theorem*}[5.22]
If $f\in L(E)$, then $f$ is finite a.e.\@ in $E$.
\end{theorem*}
\begin{theorem*}[5.24]
If $\int_E f$ exists and $E=\bigcup_{k\in\bfN} E_k$ is the countable union
of disjoint measurable sets $E_k$, then
\[
\int_E f=\sum_{k\in\bfN}\int_{E_k}f.
\]
\end{theorem*}
\begin{theorem*}[5.25]
If $|E|=0$ or if $f=0$ a.e.\@ in $E$, then $\int_E f=0$.
\end{theorem*}
\begin{theorem*}[5.32, monotone convergence theorem]
Let $\{f_k\}$ be a sequence of measurable functions on $E$:
\begin{enumerate}[label=\textnormal{(\roman*)}]
\item If $f_k\nearrow f$ a.e.\@ on $E$ and there exists $\varphi\in L(E)$ such
  that $f_k\geq\varphi$ a.e.\@ on $E$ for all $k$, then $\int_E
  f_k\to\int_E f$.
\item If $f_k\searrow f$ a.e.\@ on $E$ and there exists $\varphi\in L(E)$ such
  that $f_k\leq\varphi$ a.e.\@ on $E$ for all $k$, then $\int_E
  f_k\to\int_E f$.
\end{enumerate}
\end{theorem*}
\begin{theorem*}[5.33, uniform convergence theorem]
Let $f_k\in L(E)$ for $k\in\bfN$ and let $\{f_k\}$ converge uniformly to
$f$ on $E$, $|E|<\infty$. Then $f\in L(E)$ and $\int_E f_k\to\int_E f$.
\end{theorem*}
\begin{theorem*}[5.34, Fatou's lemma]
Let $\{f_k\}$ be a sequence of measurable functions on $E$. If there exists
$\varphi\in L(E)$ such that $f_k\geq\varphi$ a.e.\@ on $E$ for all $k$,
then
\[
\int_E\liminf_{k\to\infty} f_k\leq\liminf_{k\to\infty}\int_E f_k.
\]
\end{theorem*}
\begin{corollary*}[5.35, reverse Fatou's lemma]
Let $\{f_k\}$ be a sequence of measurable functions on $E$. If there exits
$\varphi\in L(E)$ such that $f_k\leq\varphi$ a.e.\@ on $E$ for all $k$,
then
\[
\int_E\limsup_{k\to\infty} f_k\geq\limsup_{k\to\infty}\int_E f_k.
\]
\end{corollary*}
\begin{theorem*}[5.36, Lebesgue's dominated convergenge theorem]
Let $\{f_k\}$ be a sequence of measurable functions on $E$ such that
$f_k\to f$ a.e.\@ in $E$. If there exists $\varphi\in L(E)$ such that
$|f_k|\leq\varphi$  such that $|f_k|\leq\varphi$ a.e.\@ in $E$ for all
$k\in\bfN$, then $\int_E f_k\to\int_E f$.
\end{theorem*}
\begin{corollary*}[5.37, bounded convergence theeorem]
Let $\{f_k\}$ be a sequence of measurable functions on $E$ such  that
$f_k\to f$ a.e.\@ in $E$. If $|E|<\infty$ there is a finite constant $M$
such that $|f_k|\leq M$ a.e.\@ in $E$, then $\int_E f_k\to\int_E f$.
\end{corollary*}
\begin{theorem*}[6.1 Fubini's theorem]
Let $f(\bfx,\bfy)\in L(I)$, $I\coloneqq I_1\times I_2$. Then
\begin{enumerate}[label=\textnormal{(\roman*)}]
\item For almost every $\bfx\in I_1$, $f(\bfx,\bfy)$ is measurable and
  integrable on $I_2$ as a function of $\bfy$;
\item As a function of $\bfx$, $\int_{I_2} f(\bfx,\bfy) d \bfy$ is
  measurable and integrable on $I_1$, and
\[
\iint_I f(\bfx,\bfy) d \bfx d \bfy=
\int_{I_1}\left[\int_{I_2}f(\bfx,\bfy) d \bfy\right]\! d \bfx.
\]
\end{enumerate}
\end{theorem*}
\begin{theorem*}[6.8]
Let $f(\bfx,\bfy)$ be a measurable function defined on a measurable subset
$E$ of $\bfR^{n+m}$, and let $E_\bfx\coloneqq\left\{ \,\bfy:(\bfx,\bfy)\in
  E \,\right\}$.
\begin{enumerate}[label=\textnormal{(\roman*)}]
\item For almost every $\bfx\in \bfR^n$, $f(\bfx,\bfy)$ is a measurable
  function of $\bfy$ on $E_\bfx$.
\item If $f(\bfx,\bfy)\in L(E)$, then for almost every $\bfx\in\bfR^n$,
  $f(\bfx,\bfy)$ is an integrable on $E_\bfx$ with respect to $\bfy$;
  moreover $\int_{E_\bfx}f(\bfx,\bfy) d \bfy$ is an integrable function
  of $\bfx$ and
\[
\iint_E f(\bfx,\bfy) d \bfx d \bfy=\int_{\bfR^n}\left[\int_{E_\bfx}f(\bfx,\bfy) d \bfy\right]\! d \bfx.
\]
\end{enumerate}
\end{theorem*}
\begin{theorem*}[6.10, Tonelli's theorem]
Let $f(\bfx,\bfy)$ be nonnegative and measurable on an interval
$I=I_1\times I_2$ of $\bfR^{n+m}$. Then, for almost every $\bfx\in I_1$,
$f(\bfx,\bfy)$ is a measurable function of $\bfy$ on $I_2$. Moreover, as a
function of $\bfx$, $\int_{I_2}f(\bfx,\bfy) d  \bfy$ is measurable on
$I_1$, and
\[
\iint_I f(\bfx,\bfy) d \bfx d \bfy=\int_{I_1}\left[\int_{I_2}f(\bfx,\bfy) d \bfy\right]\! d \bfx
\]
\end{theorem*}
If $f$ and $g$ are measurable in $\bfR^n$, their \emph{convolution
  $(f*g)(\bfx)$} is defined by
\[
(f*g)(\bfx)\coloneqq\int_{\bfR^n}f(\bfx-\bfy)g(\bfy) d \bfy,
\]
provided the integral exists.
\begin{theorem*}[6.14]
If $f\in L(\bfR^n)$ and $g\in L(\bfR^n)$, then $(f*g)(\bfx)$ exists for
almost every $\bfx\in\bfR^n$ and is measurable. Moreover, $f*\in
L(\bfR^n)$ and
\[
\begin{aligned}
\int_{\bfR^n}|f*g| d \bfx
\leq{}&\left(\int_{\bfR^n}|f| d \bfx\right)\left(\int_{\bfR^n}|g| d \bfx\right)\\
\int_{\bfR^n}(f*g)(\bfx) d \bfx
={}&\left(\int_{\bfR^n}f d \bfx\right)\left(\int_{\bfR^n}g d \bfx\right).
\end{aligned}
\]
\end{theorem*}
\begin{corollary*}[6.16]
If $f$ and $g$ are nonnegative and measurable on $\bfR^n$, then $f*g$ is
measurable on $\bfR^n$ and
\[
\int_{\bfR^n}(f*g) d \bfx=
\left(\int_{\bfR^n} f d \bfx\right)
\left(\int_{\bfR^n}g d \bfx\right).
\]
\end{corollary*}
\begin{theorem*}[6.17, Marcinkiewicz]
Let $F$ be a closed subset of a bounded open interval $(a,b)$, and let
$\delta(x)\coloneqq\delta(x,F)$ be the corresponding distance
function. Then, given $\lambda>0$, the integral
\[
M_\lambda(x)\coloneqq\int_a^b\frac{\delta(y)^\lambda}{|x-y|^{1+\lambda}} d  y
\]
is finite a.e.\@ in $F$. Moreover, $M_\lambda\in L(F)$ and
\[
\int_F M_\lambda d  x\leq 2\lambda^{-1}|G|,
\]
where $G\coloneqq(a,b)\smallsetminus F$.
\end{theorem*}

%%% Local Variables:
%%% mode: latex
%%% TeX-master: "../MA544-Quals"
%%% End:
