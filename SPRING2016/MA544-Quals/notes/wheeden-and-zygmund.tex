\chapter{Course Notes}
These notes roughly correspond to chapters 2 through 8 of Wheeden and
Zygmund's \emph{Measure and Integration}
\cite{wheeden-zygmund:measure-and-integral}.

\section{Functions of bounded variation and the Riemann--Stieltjes
  integral}
In this section, we introduce functions of bounded variation as well as the
definition of the Riemann integral. We conclude with a proof that the

\subsection{Functions of bounded variation}
Let $f\colon[a,b]\to\bbR$ be a real-valued function defined for all $a\leq
x\leq b$ and finite; let $\Gamma=\left\{x_0,\dotsc,x_m\right\}$ be a
\href{https://en.wikipedia.org/wiki/Partition_of_an_interval}{\emph{partition}}
of $[a,b]$, i.e., a collection of points $x_i$, $i=0,\dotsc,m$, satisfying
$x_0=a$ and $x_m=b$, and $x_{i-1}<x_i$ for $i=1,\dotsc,m$. To each
partition $\Gamma$, we associated a sum
\begin{equation}
\label{eq:bv:sum}
S_\Gamma\coloneqq S_\Gamma[f;a,b]\coloneqq\sum_{i=1}^m\left|f(x_i)-f(x_{i-1})\right|.
\end{equation}
The
\href{https://en.wikipedia.org/wiki/Bounded_variation#Formal_definition}{\emph{variation}}
(or \emph{total variation}) \emph{of $f$ over $[a,b]$} is defined as
\begin{equation}
  \label{eq:bv:variation}
V\coloneqq V[f;a,b]\coloneqq\sup_\Gamma S_\Gamma,
\end{equation}
where the supremum is taken over all partitions $\Gamma$ of $[a,b]$. If
$V<\infty$, $f$ is said to be of
\href{https://en.wikipedia.org/wiki/Bounded_variation}{\emph{bounded
    variation}} \emph{on $[a,b]$}; if $V=\infty$, $f$ is of \emph{unbounded
variation on $[a,b]$}.

Before going on to prove important properties about
\eqref{eq:bv:variation}, let us look at some common examples (and
nonexamples) of functions $f$ of bounded variation.

\begin{example}
Suppose $f$ is
\href{https://en.wikipedia.org/wiki/Monotonic_function}{\emph{monotone}} in
$[a,b]$. Then, for an arbitrary partition, $\Gamma=\{x_0,\dotsc,x_m\}$ of
$[a,b]$ we have
\begin{align*}
S_\Gamma&=\sum_{i=1}^m|f(x_{i-1})-f(x_i)|\\
        &=|f(a)-f(x_1)|+|f(x_2)-f(x_1)|+\dotsb+|f(x_{m-1})-f(x_{m-2})|+|f(x_m)-f(x_{m-1})|\\
        &=
\end{align*}
\end{example}

%%% Local Variables:
%%% mode: latex
%%% TeX-master: "../MA544-Quals"
%%% End:
