\chapter{Course Notes}
\thispagestyle{empty}
These notes roughly correspond to chapters 2 through 8 of Wheeden and
Zygmund's \emph{Measure and Integration}
\cite{wheeden-zygmund:measure-and-integral}.

This first portion corresponds to material covered before Exam 1.
\section{Functions of bounded variation and the Riemann--Stieltjes
  integral}
In this section, we introduce functions of bounded variation as well as the
definition of the Riemann integral. We conclude with a proof that the

\subsection{Functions of bounded variation}
Let $f\colon[a,b]\to\bbR$ be a real-valued function defined for all $a\leq
x\leq b$ and finite; let $\Gamma=\left\{x_0,\dotsc,x_m\right\}$ be a
\href{https://en.wikipedia.org/wiki/Partition_of_an_interval}{\emph{partition}}
of $[a,b]$, i.e., a collection of points $x_i$, $i=0,\dotsc,m$, satisfying
$x_0=a$ and $x_m=b$, and $x_{i-1}<x_i$ for $i=1,\dotsc,m$. To each
partition $\Gamma$, we associated a sum
\begin{equation}
\label{eq:bv:sum}
S_\Gamma\coloneqq S_\Gamma[f;a,b]\coloneqq\sum_{i=1}^m\left|f(x_i)-f(x_{i-1})\right|.
\end{equation}
The
\href{https://en.wikipedia.org/wiki/Bounded_variation#Formal_definition}{\emph{variation}}
(or \emph{total variation}) \emph{of $f$ over $[a,b]$} is defined as
\begin{equation}
  \label{eq:bv:variation}
V\coloneqq V[f;a,b]\coloneqq\sup_\Gamma S_\Gamma,
\end{equation}
where the supremum is taken over all partitions $\Gamma$ of $[a,b]$. If
$V<\infty$, $f$ is said to be of
\href{https://en.wikipedia.org/wiki/Bounded_variation}{\emph{bounded
    variation}} \emph{on $[a,b]$}; if $V=\infty$, $f$ is of \emph{unbounded
variation on $[a,b]$}.

Before going on to prove important properties about
\eqref{eq:bv:variation}, let us look at some common examples (and
nonexamples) of functions $f$ of bounded variation.

\begin{example}
Suppose $f$ is
\href{https://en.wikipedia.org/wiki/Monotonic_function}{\emph{monotone}} in
$[a,b]$. Then, clearly, each $S_\Gamma$ is equal to $|f(a)-f(b)|$ for every
partition $\Gamma$\footnote{Carlos: This may not be clear at
  a first glance, but, upon closer inspection, this is true by
  monotoncity. If $a<x<b$, we have
  $|f(b)-f(a)|=|f(b)-f(x)|+|f(x)-f(a)|$. This holds for an arbitrary
  partitions $\Gamma$.}, and therefore $V=|f(b)-f(a)|$.
\end{example}
\begin{example}
Suppose the graph of $f$ can be split into a finite number of monotone
arcs, i.e., suppose $[a,b]=\bigcup_{i=1}^k [a_{i-1},a_i]$ and $f$ is
monotone in each $[a_{i-1},a_i]$. Then
$V=\sum_{i-1}^k|f(a_i)-f(a_{i-1})|$. To see this, we use the result of
Example 1 above and the fact, yet to be proven, than $V=V[a,b]=\sum_{i=1}^k
V[a_i,a_{i-1}]$.
\end{example}

If $\Gamma=\{x_0,\dotsc,x_m\}$ is a partition of $[a,b]$, let $|\Gamma|$,
called the
\href{https://en.wikipedia.org/wiki/Partition_of_an_interval#Norm_of_a_partition}{\emph{norm}}
\emph{of $\Gamma$}, be defined as the length of the longest subinterval of
$\Gamma$
\begin{equation}
\label{eq:bv:partition-norm}
|\Gamma|\coloneqq\max_{i}(x_i-x_{i-1}).
\end{equation}
If $f$ is continuous on $[a,b]$ and $\{\Gamma_j\}$ is a sequence of
partitions of $[a,b]$ with $|\Gamma_j|\to 0$, we shall see that
$V=\lim_{j\to\infty} S_{\Gamma_j}$.

\begin{example}
Let $f$ be the
\href{https://en.wikipedia.org/wiki/Dirichlet_function}{\emph{Dirichlet
    function}}, defined by $f(x)=1$ for $x$ rational and $f(x)=0$ for $x$
irrational. Then, clearly, $V[a,b]=\infty$ for any interval of
$[a,b]$.\footnote{Carlos: By the density of $\bbQ$ in $\bbR$ (and by
  restriction, $[a,b]$, since $[a,b]$ is path-connected), for any
  positive integer $N$, we may choose a partition $\Gamma$ of $[a,b]$
  containing $N+1$ rational numbers so $S_\Gamma=N+1>N$.}
\end{example}

\begin{example}
A function that is continuous on an interval, however, need not be of
bounded variation on that interval. Take for example the following
construction: let $\{a_j\}$ and $\{d_j\}$, $j=1,2,\dotsc$, be monotone
decreasing sequences in $(0,1]$ with $a_1=\lim_{j\to\infty}
a_j=\lim_{j\to\infty}d_j=0$ and $\sum d_j=\infty$. Construct a continuous
function $f$ as follows. On each subinterval $[a_{j+1},a_j]$, the graph of
$f$ consists of the sides of the isosceles triangle with base
$[a_{j+1},a_j]$ and height $d_j$. Thus, $f(a_j)=0$, and if $m_j$ denotes
the midpoint of $[a_{j+1},a_j]$, then $f(m_j)=d_j$. If we define $f(0)=0$,
then $f$ is continuous on $[0,1]$. Taking $\Gamma_k$ to be the partition
defined by the points $0$, ${\{a_j\}}_{j=1}^{k+1}$ and ${\{m_j\}}_{j=1}^k$,
we see that $S_\Gamma=2\sum_{j=1}^k d_j$. Hence, $V[f;0,1]=\infty$.
\end{example}

%%% Local Variables:
%%% mode: latex
%%% TeX-master: "../MA544-Quals"
%%% End:
