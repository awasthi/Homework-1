\chapter{Notes}
Notes based off of Wheeden and Zygmund's \emph{Measure and Integral} book.
\section{Exam 1 Review}
This is all of the material we covered before exam 1.

\bigskip

Introductory material I should have known from 504.

If $\calF$ is a countable (i.e., finite or countably infinite), it will be
called a \emph{sequence of sets} and denoted
$\calF\coloneqq\left\{\,E_k:k=1,2,...\,\right\}$. The corresponding union
and intersection will be written $\bigcup_k E_k$ and $\bigcap_k E_k$. A
sequence $\{E_k\}$ of sets is said to \emph{increase} to $\bigcup_k E_k$ if
$E_k\subset E_{k+1}$ for all $k$ and to \emph{decrease} to $\bigcap_k E_k$
if $E_k\supset E_{k+1}$ for all $k$; we use the notation
$E_k\nearrow\bigcap_k E_k$ and $E_k\searrow\bigcap_k E_k$ to denote these
two possibilities. If $\left\{E_k\right\}_{k=1}^\infty$ is a sequence of
sets, we define
\begin{equation}
\label{eq:limsup-liminf-sets}
\limsup E_k=\bigcap_{j=1}^\infty\bigcup_{k=j}^\infty E_k,\qquad
\liminf E_k=\bigcup_{j=1}^\infty\bigcap_{k=j}^\infty E_k,
\end{equation}
noting that the sets $U_j\coloneqq\bigcup_{k=j}^\infty E_k$ and
$V_j\coloneqq\bigcap_{k=j}^\infty E_k$ satisfy $U_j\searrow\limsup E_k$ and
$V_j\nearrow\liminf E_k$. Then $\limsup E_k$ consists of those points of
$\bbR^n$ that belong to infinitely many $E_k$ and $\liminf E_k$ to those
that belong to all $E_k$ for $k\geq k_0$ (where $k_0$ may vary from point
to point). Thus $\liminf E_k\subset\limsup E_k$.

If $E_1$ and $E_2$ are two sets, we define $E_1\setminus E_2$ by
$E_1\setminus E_2\coloneqq E_1\cap\complement E_2$ and call it the
\emph{difference} of $E_1$ and $E_2$ or the \emph{relative complement} of
$E_2$ in $E_1$. We will often have occasion to use \emph{de Morgan laws},
which govern relations between complements, unions, and intersections;
these state that
\begin{equation}
\label{eq:de-morgan-laws}
\complement
\left(\bigcup_{E\in\calF}E\right)=
\bigcap_{E\in\calF}\complement E,\qquad
\complement
\left(\bigcap_{E\in\calF}E\right)=
\bigcup_{E\in\calF}\complement E,
\end{equation}
and are easily verified.

If $\bfx\in\bbR^n$, we say that a sequence $\{\bfx_k\}$ \emph{converges} to
$\bfx$, or that $\bfx$ is the \emph{limit point} of $\{\bfx_k\}$, if
$\|\bfx-\bfx_k\|\to 0$ as $k\to\infty$. We denote this by writing either
$\bfx=\lim_{k\to\infty}\bfx_k$ or $\bfx_k\to\bfx$ as $k\to\infty$. A point
$\bfx\in\bbR^n$ is called a \emph{limit point of a set $E$} if it is the
limit point of a sequence of distinct points of $E$. A point $\bfx\in E$ is
called a \emph{isolated point} of $E$ if it is not the limit point of any
sequence in $E$ (excluding the trivial sequence $\left\{\bfx_k\right\}$
where $\bfx_k=\bfx$ for all $k\in\bbN$). It follows that $\bfx$ is isolated
if and only if there is a $\delta>0$ such that $\|\bfx-\bfy\|>\delta$ for
every $\bfy\in E$, $\bfy\neq\bfx$.

For sequences $\left\{x_k\right\}$ in $\bbR$, we will write
$\lim_{k\to\infty} x_K=\infty$, or $x_k\to\infty$ as $k\to\infty$, if given
$M>0$ there is an integer $N$ such that $x_k\geq M$ whenever $k\geq M$.

A sequence $\left\{ \bfx_k \right\}$ in $\bbR^n$ is called a \emph{Cauchy
  sequence} if given $\varepsilon>0$ there exists an integer $N$ such that
$\|\bfx_k-\bfx_\ell\|<\varepsilon$ for all $k,\ell\geq N$. $\bbR^n$ is a
complete metric space, i.e., every Cauchy sequence in $\bbR^n$ converges to
a point of $\bbR^n$.

A set $E\subset E_1$ is said to be \emph{dense} in $E_1$ if for every
$\bfx_1\in E_1$ and $\varepsilon>0$ there is a point $\bfx\in E$ such that
$0<\|\bfx-\bfx_1\|<\varepsilon$. Thus, $E$ is dense in $E_1$ if every point
of $E_1$ is a limit point of $E$. If $E=E_1$, we say $E$ is \emph{dense in
  itself}. As an example, the set of limit points of $\bbR^n$ each of whose
coordinates is a rational number is dense in $\bbR^n$. Since this set is
also countable, it follows that $\bbR^n$ is \emph{separable}, by which we
mean that $\bbR^n$ has a countable dense subset.

For nonempty subsets $E$ of $\bbR$, we use the standard notation $\sup E$
and $\inf E$ for the \emph{supremum} (\emph{least upper bound}) and
\emph{infimum} (\emph{greatest lower bound}) of $E$. In case $\sup E$
belong to $E$, it will be called $\max E$; similarly, $\inf E$ will be
called $\min E$ if it belongs to $E$.

If $\left\{ a_k \right\}_{k=1}^\infty$ is a sequence of points in $\bbR$,
let $b_j=\sup_{k\geq j} a_k$ and $c_j=\inf_{k\geq j} a_k$,
$j=1,2,\dotsc$. Then $-\infty\leq c_j\leq b_j\leq\infty$ and $\left\{ b_j
\right\}$ and $\left\{ c_j \right\}$ are monotone decreasing and
increasing, respectively; that is, $b_j\geq b_{j+1}$ and $c_j\leq
c_{j+1}$. Define $\limsup_{k\to\infty} a_k$ and $\liminf_{k\to\infty} a_k$
by
\begin{equation}
\label{eq:limsup-liminf-e-k}
\begin{aligned}
\limsup_{k\to\infty} a_j=
\lim_{j\to\infty}b_j=
\lim_{j\to\infty}\left\{\lim_{k\geq j} a_k\right\},\\
\liminf_{k\to\infty} a_k=
\lim_{j\to\infty} C_j=
\lim_{j\to\infty}\left\{\lim_{k\geq j} a_k\right\}.
\end{aligned}
\end{equation}
\begin{theorem}[1.4]
\begin{enumerate}[label=\textnormal{(\alph*)}]
\item $L\coloneqq\limsup_{k\to\infty} a_k$ if and only if (i) there is a
  subsequence $\{a_{k_j}\}$ of $\{a_k\}$ that   converges to $L$ and (ii)
  if $L'>L$, there is an integer $N$ such that $a_k<L'$ for $k\geq N$.
\item $\ell\coloneqq\liminf_{k\to\infty} a_k$ if and only if (i) there is a
  subsequence $\{a_{k_j}\}$ of $\{a_k\}$ that converges to $\ell$ and (ii)
  if $\ell'<\ell$, there is an integer $N$ such that $a_k>\ell'$ for $k\geq
  N$.
\end{enumerate}
\end{theorem}

Thus, when they are finite, $\limsup a_k$ and $\liminf a_k$ are the
largest and smallest limit points of $\{a_k\}$, respectively.

We can also use the metric on $\bbR$ to define the \emph{diameter of a set
  $E$} by letting
\begin{equation}
  \label{eq:diameter-of-set}
\diam E\coloneqq\left\{\,\|\bfx-\bfy\|:\bfx,\bfy\in E\,\right\}.
\end{equation}
If the diameter of $E$ is finite, $E$ is said to be
\emph{bounded}. Equivalently, $E$ is bounded if there is a finite constant
$M$ such  that $\|\bfx\|\leq M$ for all $\bfx\in E$. If $E_1$ and $E_2$ are
two sets, the \emph{distance between $E_1$ and $E_2$} is defined by
\begin{equation}
  \label{eq:distance-e-1-e-2}
d(E_1,E_2)\coloneqq\inf\left\{\,\|\bfx-\bfy\|:\bfx\in E_1,\bfy\in E_2\,\right\}.
\end{equation}

For $\bfx\in\bbR^n$ and $\delta>0$, the set
\begin{equation}
\label{eq:open-ball-r-n}
B(\bfx,\delta)\coloneqq\left\{\,\bfy:\|bfx-\bfy\|<\delta\,\right\}
\end{equation}
is called the \emph{open ball with center $\bfx$ and radius $\delta$}. A
point $\bfx$ of a set $E$ is called an \emph{interior point} of $E$ if
there exists $\delta>0$ such that $B(\bfx,\delta)\subset E$. The collection
of all interior points of $E$ is called the \emph{interior} of $E$ and
denoted $E^\circ$. A set $E$ is said to be \emph{open} if $E^\circ=E$; that
is, $E$ is open if for each $\bfx\in E$ there exists $\delta>0$ such that
$B(\bfx,\delta)\subset E$. The empty set $\emptyset$ is open by
convention. The whole space $\bbR^n$ is clearly open and $B(\bfx,\delta)$
is evidently open. We will generally denote open sets by the letter $G$.

A set $E$ is called \emph{closed} if $\complement E$ is open. Note that
$\emptyset$ and $\bbR^n$ are closed. Closed sets will generally be denoted
by the letter $F$. The union of a set $E$ and all its limit points is
called the \emph{closure} of $E$ and written $\bar E$. By the
\emph{boundary} of $E$, we mean $\partial E\coloneqq \bar E\minus
E^\circ$.
\begin{theorem}[1.5]
\begin{enumerate}[label=\textnormal{(\roman*)}]
\item $\overline{B(\bfx,\delta)}=\left\{\,\bfy:\|\bfx-\bfy\|\leq\delta\, \right\}$
\item $E$ is closed if and only if $E=\bar E$; that is, $E$ is closed if
  and only if it contains all of its limit points.
\item $\bar E$ is closed, and $\bar E$ is the smallest closed set
  containing $E$; that is, $F$ is closed and $E\subset F$, then $\bar
  E\subset F$.
\end{enumerate}
\end{theorem}

(The rest of this is a bunch of theorems that can be expressed in more
generality from a more topological perspective. At any rate, they are very
basic.)

Consider a collection $\{A\}$ of sets $A$. A set is said to be of \emph{type
$A_\delta$} if it can be written as a countable intersection of sets $A$
and of \emph{type $A_\sigma$} if it can be written as a countable union of
sets $A$. The most common uses of this notation are $G_\delta$ and
$F_\sigma$, where $\{G\}$ denotes open sets in $\bbR^n$ and $\{F\}$ closed
sets. Hence, $H$ is of \emph{type $G_\delta$} if
\begin{equation}
  \label{eq:G-delta}
H=\bigcap_k G_k,\qquad \text{$G_k$ open,}
\end{equation}
and is of \emph{type $F_\sigma$} if
\begin{equation}
  \label{eq:F-sigma}
H=\bigcap_k F_k,\qquad\text{$F_k$ closed.}
\end{equation}
The complement of a $G_\delta$ set is an $F_\sigma$ and vice-a-versa.

Another type of set that we have the occasion to use is the \emph{perfect
  set}, by which we mean a closed set $C$ each of whose points is a limit
point of $C$. Thus, a perfect set is a closed set that is dense in itself.

\begin{theorem}[1.9]
A perfect set is uncountable.
\end{theorem}

Other special sets that will be important are $n$-dimensional
intervals. When $n=1$ and $a<b$, we will use the usual notations
$[a,b]\coloneqq\left\{\,x:a\leq x\leq b\,\right\}$,
$(a,b)\coloneqq\left\{\,x:a<x<b\,\right\}$, etc. Whenever we use just the
word interval, we generally mean closed interval. An \emph{$n$-dimensional
  interval $I$} is a subset of $\bbR^n$ of the form
$I\coloneqq\left\{\,\bfx=(x_1,\dotsc,x_n):\text{$a_k\leq x_k\leq b_k$,
    $k=1,\dotsc,n$}\,\right\}$, where $a_k<b_k$, $k=1,\dotsc,n$.

%%% Local Variables:
%%% mode: latex
%%% TeX-master: "../MA544-Quals"
%%% End:
