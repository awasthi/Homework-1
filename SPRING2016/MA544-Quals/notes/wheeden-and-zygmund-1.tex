\chapter{Course Notes}
\thispagestyle{empty}
These notes roughly correspond to chapters 2 through 8 of Wheeden and
Zygmund's \emph{Measure and Integration}
\cite{wheeden-zygmund:measure-and-integral}.

This first portion corresponds to material covered before Exam 1.
\section{Preliminaries}
Here is some precursor material to the Lebesgue theory of integration.
\subsection{Points and sets in $\bbR^n$}
From this section, we need not say much only a few results and definitions
are important.

\bigskip

If $\calF$ is a
\href{https://en.wikipedia.org/wiki/Countable}{\emph{countable}} collection
of subsets of $\bbR^n$, it will be called a \emph{sequence of sets} and
denoted $\{E_k\}$ for $k\in\bbN$. The corresponding
\href{https://en.wikipedia.org/wiki/Union_(set_theory)}{\emph{union}} and
\href{https://en.wikipedia.org/wiki/Intersection_(set_theory)}{\emph{intersection}}
will be written $\bigcup_k E_k$ and $\bigcap_k E_k$. A sequence $\{E_k\}$
is said to \emph{increase} to $\bigcup_k E_k$ if $E_k\subset E_{k+1}$ for
all $k$ and to \emph{decrease} to $\bigcap_k E_k$ if $E_k\supset E_{k+1}$
for all $k$; we use the notation $E_k\nearrow \bigcup_k E_k$ and
$E_k\searrow\bigcap_k E_k$ to denote these two possibilities. If $\{E_k\}$
is a sequence of sets, we define
\begin{equation}
\label{eq:1:set-limsup-liminf}
\begin{aligned}
\limsup E_k
&\coloneqq\bigcap_{j=1}^\infty\left(\bigcup_{k=j}^\infty E_k\right),&
\liminf E_k&\coloneqq\bigcup_{j=1}^\infty\left(\bigcap_{k=j}^\infty E_k\right),
\end{aligned}
\end{equation}
noting that the subsets $U_j\coloneqq\bigcup_{k=j}^\infty E_k$ and
$V_j\coloneqq\bigcap_{k=j}^\infty E_k$ satisfy $U_j\searrow\limsup E_k$ and
$V_j\nearrow\liminf E_k$.\footnote{Carlos: Make note of this. It is often a
good strategy to decompose a set $E$ into the intersection or union of a
sequence $E_k$. Making appropriate manipulations, we often get $E_k\searrow
E$ or $E_k\nearrow E$ and make limiting arguments about properties of the
set, i.e., measure or the integral of some function whose domain is in $E$,
etc.}
\subsection{$\bbR^n$ as a metric space}
A student who has taken 504 or 571 will know most of the material under
this section. We include it here as a useful reference to some of the
more useful results of the properties of $\bbR^n$ as a metric space.

If $\bfx\in\bbR^n$, we say that a sequence $\{\bfx_k\}$
\href{https://en.wikipedia.org/wiki/Limit_(mathematics)}{\emph{converges}}
to $\bfx$, or that $\bfx$ is the \emph{limit} of $\{\bfx_k\}$, if
$|\bfx-\bfx_k|\to 0$ as $k\to\infty$. We denote this by writing either
$\lim_{k\to\infty}\bfx_k=\bfx$ or $\bfx_k\to\bfx$ as $k\to\infty$. A point
$\bfx\in\bbR^n$ is called a
\href{https://en.wikipedia.org/wiki/Limit_point}{\emph{limit point}}
\emph{of a set $E$} if it is the limit of a sequence of distinct points of
$E$. A point $\bfx\in E$ is called an
\href{https://en.wikipedia.org/wiki/Isolated_point}{\emph{isolated point}}
of $E$ if it is not the limit point of any sequence in $E$ (excluding the
trivial sequence $\{\bfx_k\}$ where $\bfx_k=\bfx$ for all $k$). It follows
that a point $\bfx$ is isolated if and only if there is a $\delta>0$ such
that $|\bfx-\bfy|>\delta$ for every $\bfy\in E$, $\bfy\neq\bfx$.

For sequences $\{x_k\}$ in $\bbR$, we will write
$\lim_{k\to\infty}x_k=\infty$, or $x_k\to\infty$ as $k\to\infty$, if given
$M>0$ there is an integer $N$ such that $x_k\geq M$ whenever $k\geq N$. A
similar definition holds for
$\lim_{k\to\infty}x_k=-\infty$.\footnote{Carlos: In fact, we can define it
  by saying that $\lim_{k\to\infty} x_k=-\infty$ if
  $\lim_{k\to\infty}-x_k=\infty$. }

A sequence $\{\bfx_k\}$ in $\bbR^n$ is called a
\href{https://en.wikipedia.org/wiki/Cauchy_sequence}{\emph{Cauchy
    sequence}} if given $\varepsilon>0$ there is an integer $N$ such that
$|\bfx_k-\bfx_\ell|<\varepsilon$ for all $k,\ell\geq N$. We say that a
metric space $(X,|\cdot|)$ is
\href{https://en.wikipedia.org/wiki/Complete_metric_space}{\emph{complete}}
\emph{with respect to the metric $|\cdot|$} if every Cauchy sequence in $X$
converges.

A set $E_0\subset E$ is said to be
\href{https://en.wikipedia.org/wiki/Dense_set}{\emph{dense}} in $E$ if for
every $\bfy\in E$ and $\varepsilon>0$, there is a point $\bfx\neq\bfy$ in
$E_0$ such that $0<|\bfy-\bfx|<\varepsilon$. Thus, $E_0$ is dense in $E$ if
every point of $E$ is a limit point of $E_0$. If $E_0=E$, we say $E$ is
\emph{dense in itself}. As an example $\bbQ^n\subset\bbR^n$ is dense in
$\bbR^n$. Since this set is also countable, it follows that $\bbR^n$ is
\href{https://en.wikipedia.org/wiki/Separable_space}{\emph{separable}}, by
which we mean that $\bbR^n$ has a countable dense subset.

For a nonempty subset $E$ of $\bbR^n$, we use the standard notation $\sup
E$ and $\inf E$ for the
\href{https://en.wikipedia.org/wiki/Infimum_and_supremum}{\emph{supremum}}
(\emph{least upper bound}) and $\inf E$ for the \emph{infimum}
(\emph{greatest lower bound}) of $E$. In case $\sup E$ is in $E$, it will
be called $\max E$; similarly, if $\inf E\in E$, $\infty E$ will be called
$\min E$.

If $\{a_k\}$ is a sequence of points in $\bbR$, let
$b_j\coloneqq\sup_{k\geq j} a_k$ and $c_j\coloneqq\inf_{k\geq j} a_k$,
$j\in\bbN$. Then $-\infty\leq c_j\leq b_j\leq\infty$, and $\{b_j\}$ and
$\{c_j\}$ are monotone decreasing and increasing, respectively; i.e.,
$b_j\geq b_{j+1}$ and $c_j\leq c_j+1$. Define $\limsup_{k\to\infty}a_k$ and
$\liminf_{k\to\infty}a_k$ by
\begin{equation}
\label{eq:limsup-liminf-definition}
\begin{aligned}
\limsup_{k\to\infty}a_k&\coloneqq\lim_{j\to\infty} b_j
=\lim_{j\to\infty}\left\{\sideset{}{_{k\geq j}}\sup a_k  \right\},\\
\liminf_{k\to\infty}a_k&\coloneqq\lim_{j\to\infty} c_j
=\lim_{j\to\infty}\left\{\sideset{}{_{k\geq j}}\inf a_k\right\}.
\end{aligned}
\end{equation}

\begin{theorem}[1.4]
\hfill
\begin{enumerate}[label=\textnormal{(\alph*)},noitemsep]
\item $L=\limsup_{k\to\infty} a_k$ if and only if \textnormal{(i)}, there
  is a subsequence $\{a_{k_j}\}$ of $\{a_k\}$ that converges to $L$ and
  \textnormal{(ii)} if $L'>L$, there is an integer $N$ such that $a_k<L'$
  for all $k\geq N$.
\item $\ell=\limsup_{k\to\infty} a_k$ if and only if \textnormal{(i)}, there
  is a subsequence $\{a_{k_j}\}$ of $\{a_k\}$ that converges to $\ell$ and
  \textnormal{(ii)} if $\ell'<\ell$, there is an integer $N$ such that
  $a_k>\ell'$ for all $k\geq N$.
\end{enumerate}
\end{theorem}
When they are finite, $\limsup a_k$ and $\liminf a_k$ are the largest and
smallest limit points of $\{a_k\}$, respectively. It's not too difficult to
show that $\{a_k\}$ converges to $a$, $-\infty\leq a\leq\infty$, if and
only if $\limsup a_k=\liminf a_k$.

We can also use the metric on $\bbR^n$ to define the
\href{https://en.wikipedia.org/wiki/Metric_space#Bounded_and_totally_bounded_spaces}{\emph{diameter}}
\emph{of a set $E$} by letting
\begin{equation}
\label{eq:1:diameter}
\diam(E)\coloneqq\sup\left\{\,|\bfx-\bfy|:\bfx,\bfy\in E\,\right\}
\end{equation}
If the diameter of $E$ is finite, $E$ is said to be
\href{https://en.wikipedia.org/wiki/Bounded_set}{\emph{bounded}}. Equivalently,
$E$ is bounded if there is a finite constant $M$ such that $|\bfx|\leq M$
for all $\bfx\in E$. If $E_1$ and $E_2$ are two sets, the
\href{https://en.wikipedia.org/wiki/Hausdorff_distance}{\emph{distance}}
\emph{between $E_1$ and $E_2$} is defined by
\begin{equation}
  \label{eq:1:hausdorff-distance}
d(E_1,E_2)\coloneqq\inf\left\{\,|\bfx-\bfy|:\text{$\bfx\in E_1$, $\bfy\in
    E_2 $}\,\right\}.
\end{equation}

\subsection{Open and closed sets in $\bbR^n$, and special sets}
For $\bfx\in\bbR^n$ and $\varepsilon>0$, the set
\begin{equation}
\label{eq:1:open-ball}
B_\varepsilon(\bfx)\coloneqq
B(\bfx,\varepsilon)\coloneqq\left\{\,\bfy:|\bfx-\bfy|<\varepsilon\,\right\}.
\end{equation}
A point $\bfx\in E$ is called an
\href{https://en.wikipedia.org/wiki/Interior_(topology)#Interior_point}{\emph{interior
    point}} of $E$ if there exists $\varepsilon>0$ such that
$B_\varepsilon(\bfx)\subset E$. The collection of all interior points of
$E$ is called the
\href{https://en.wikipedia.org/wiki/Interior_(topology)}{\emph{interior}}
\emph{of $E$} and denoted $E^\circ$. A set is said to be
\href{https://en.wikipedia.org/wiki/Open_set}{\emph{open}} if
$E=E^\circ$. The empty set $\emptyset$ is open by convention. The whole
space $\bbR^n$ is clearly open and it is easy to see that
$B_\varepsilon(\bfx)$ is open for any $\varepsilon>0$. We shall generally
denote open sets by the letter $G$.

A set $E$ is \href{https://en.wikipedia.org/wiki/Closed_set}{\emph{closed}}
if $\bbR^n\setminus E$ is open. Thus, $\emptyset$ and $\bbR^n$ are closed
(being the complements of each other). Closed sets will generally be
denoted by the letter $F$. The union of the set $E$ and all of its limit
points is called the
\href{https://en.wikipedia.org/wiki/Closure_(topology)}{\emph{closure}} of
$E$ and written $\bar E$. By the
\href{https://en.wikipedia.org/wiki/Boundary_(topology)}{\emph{boundary}}
of $E$, we mean the set $\partial E\coloneqq \bar E\setminus E^\circ$.

Now, consider a collection of sets $\calA=\{A\}$. A set is said to be of
\emph{type $A_\delta$} if it can be written as a countable intersection of
sets in $A$ and to be of \emph{type $A_\sigma$} if it can be written as a
countable union of sets in $A$. The most common usage of this notation is
$G_\delta$ and $F_\sigma$ sets where $\calG=\{G\}$ denotes the open sets in
$\bbR^n$ and $\calF=\{F\}$ the closed sets. Hence, $E$ is of type
\href{https://en.wikipedia.org/wiki/G-delta_set}{$G_\delta$} if
\begin{equation}
  \label{eq:1:g-delta-set}
E=\bigcap_k G_k,\text{ $G_k$ open,}
\end{equation}
and of type \href{https://en.wikipedia.org/wiki/F-sigma_set}{$F_\sigma$} if
\begin{equation}
  \label{eq:1:f-sigma}
E=\bigcup_k F_k,\text{ $F_k$ is closed.}
\end{equation}
The complement of a $G_\delta$ set is an $F_\sigma$ set and vice-versa.

Another type of special set we will have the occasion to use is a
\href{https://en.wikipedia.org/wiki/Perfect_set}{\emph{perfect set}}, by
which we mean a closed set $C$ each of whose points is a limit point of
$C$. Thus, a perfect set is a closed set that is dense in itself.

\begin{theorem}[1.9]
A perfect set is uncountable.
\end{theorem}

An $n$-dimensional interval $I$ is a subset of $\bbR^n$ of the form
\begin{equation}
\label{eq:1:n-interval}
I=\left\{\,(x_1,\dotsc,x_n):\text{$a_k\leq x_k\leq b_k$, for $k=1,\dotsc,n$}\,\right\}.
\end{equation}
An $n$-interval is closed and has edges parallel to the coordinate axes. If
the edge lengths $b_k-a_k$ are all equal, $I$ will be called an
\emph{$n$-dimensional cube} or simply an \emph{$n$-cube}. Cubes will
usually be denoted by the letter $Q$. Two intervals $I_1$ and $I_2$ are
said to be \emph{nonoverlapping} if their interiors are disjoint, i.e., if
the most they have in common is some part of their boundary. A set equal to
an $n$-interval minus some part of its boundary is called a \emph{partly
  open interval}. By definition, the \emph{volume $\vol I$


\section{Functions of bounded variation and the Riemann--Stieltjes
  integral}
In this section, we introduce functions of bounded variation as well as the
definition of the Riemann integral. We conclude with a proof that the

\subsection{Functions of bounded variation}
Let $f\colon[a,b]\to\bbR$ be a real-valued function defined for all $a\leq
x\leq b$ and finite; let $\Gamma=\left\{x_0,\dotsc,x_m\right\}$ be a
\href{https://en.wikipedia.org/wiki/Partition_of_an_interval}{\emph{partition}}
of $[a,b]$, i.e., a collection of points $x_i$, $i=0,\dotsc,m$, satisfying
$x_0=a$ and $x_m=b$, and $x_{i-1}<x_i$ for $i=1,\dotsc,m$. To each
partition $\Gamma$, we associated a sum
\begin{equation}
\label{eq:bv:sum}
S_\Gamma\coloneqq S_\Gamma[f;a,b]\coloneqq\sum_{i=1}^m\left|f(x_i)-f(x_{i-1})\right|.
\end{equation}
The
\href{https://en.wikipedia.org/wiki/Bounded_variation#Formal_definition}{\emph{variation}}
(or \emph{total variation}) \emph{of $f$ over $[a,b]$} is defined as
\begin{equation}
  \label{eq:bv:variation}
V\coloneqq V[f;a,b]\coloneqq\sup_\Gamma S_\Gamma,
\end{equation}
where the supremum is taken over all partitions $\Gamma$ of $[a,b]$. If
$V<\infty$, $f$ is said to be of
\href{https://en.wikipedia.org/wiki/Bounded_variation}{\emph{bounded
    variation}} \emph{on $[a,b]$}; if $V=\infty$, $f$ is of \emph{unbounded
variation on $[a,b]$}.

Before going on to prove important properties about
\eqref{eq:bv:variation}, let us look at some common examples (and
nonexamples) of functions $f$ of bounded variation.

\begin{example}
Suppose $f$ is
\href{https://en.wikipedia.org/wiki/Monotonic_function}{\emph{monotone}} in
$[a,b]$. Then, clearly, each $S_\Gamma$ is equal to $|f(a)-f(b)|$ for every
partition $\Gamma$\footnote{Carlos: This may not be clear at
  a first glance, but, upon closer inspection, this is true by
  monotoncity. If $a<x<b$, we have
  $|f(b)-f(a)|=|f(b)-f(x)|+|f(x)-f(a)|$. This holds for an arbitrary
  partitions $\Gamma$.}, and therefore $V=|f(b)-f(a)|$.
\end{example}
\begin{example}
Suppose the graph of $f$ can be split into a finite number of monotone
arcs, i.e., suppose $[a,b]=\bigcup_{i=1}^k [a_{i-1},a_i]$ and $f$ is
monotone in each $[a_{i-1},a_i]$. Then
$V=\sum_{i-1}^k|f(a_i)-f(a_{i-1})|$. To see this, we use the result of
Example 1 above and the fact, yet to be proven, than $V=V[a,b]=\sum_{i=1}^k
V[a_i,a_{i-1}]$.
\end{example}

If $\Gamma=\{x_0,\dotsc,x_m\}$ is a partition of $[a,b]$, let $|\Gamma|$,
called the
\href{https://en.wikipedia.org/wiki/Partition_of_an_interval#Norm_of_a_partition}{\emph{norm}}
\emph{of $\Gamma$}, be defined as the length of the longest subinterval of
$\Gamma$
\begin{equation}
\label{eq:bv:partition-norm}
|\Gamma|\coloneqq\max_{i}(x_i-x_{i-1}).
\end{equation}
If $f$ is continuous on $[a,b]$ and $\{\Gamma_j\}$ is a sequence of
partitions of $[a,b]$ with $|\Gamma_j|\to 0$, we shall see that
$V=\lim_{j\to\infty} S_{\Gamma_j}$.

\begin{example}
Let $f$ be the
\href{https://en.wikipedia.org/wiki/Dirichlet_function}{\emph{Dirichlet
    function}}, defined by $f(x)=1$ for $x$ rational and $f(x)=0$ for $x$
irrational. Then, clearly, $V[a,b]=\infty$ for any interval of
$[a,b]$.\footnote{Carlos: By the density of $\bbQ$ in $\bbR$ (and by
  restriction, $[a,b]$, since $[a,b]$ is path-connected), for any
  positive integer $N$, we may choose a partition $\Gamma$ of $[a,b]$
  containing $N+1$ rational numbers so $S_\Gamma=N+1>N$.}
\end{example}

\begin{example}
A function that is continuous on an interval, however, need not be of
bounded variation on that interval. Take for example the following
construction: let $\{a_j\}$ and $\{d_j\}$, $j=1,2,\dotsc$, be monotone
decreasing sequences in $(0,1]$ with $a_1=\lim_{j\to\infty}
a_j=\lim_{j\to\infty}d_j=0$ and $\sum d_j=\infty$. Construct a continuous
function $f$ as follows. On each subinterval $[a_{j+1},a_j]$, the graph of
$f$ consists of the sides of the isosceles triangle with base
$[a_{j+1},a_j]$ and height $d_j$. Thus, $f(a_j)=0$, and if $m_j$ denotes
the midpoint of $[a_{j+1},a_j]$, then $f(m_j)=d_j$. If we define $f(0)=0$,
then $f$ is continuous on $[0,1]$. Taking $\Gamma_k$ to be the partition
defined by the points $0$, ${\{a_j\}}_{j=1}^{k+1}$ and ${\{m_j\}}_{j=1}^k$,
we see that $S_\Gamma=2\sum_{j=1}^k d_j$. Hence, $V[f;0,1]=\infty$.
\end{example}

%%% Local Variables:
%%% mode: latex
%%% TeX-master: "../MA544-Quals"
%%% End:
