\section{Exam 2 Prep}
\begin{problem}
Define for $\bfx\in\bbR^n$,
\[
f(\bfx)\coloneqq
\begin{cases}
\left|\bfx\right|^{-(n+1)}&\text{if $x\neq 0$,}\\
0&\text{if $x=0$.}
\end{cases}
\]
Prove that $f$ is integrable outside any ball $B(0,\varepsilon)$, and that
there exists a constant $C>0$ such that
\[
\int_{\bbR^n\minus B(0,\varepsilon)}f(x)\diff x\leq\frac{C}{\varepsilon}.
\]
\end{problem}
\begin{proof}
What does it mean for a measurable function $f$ to be integrable over a set
$E\subset\bbR^n$, i.e., that $f$ belong to $L^1(E)$? It means that
\[
\int_E f(x)\diff x<\infty,
\]
or equivalently that the integral of the absolute value of $f$ be
finite.

Now, suppose $f$ is given as in the statement of the problem. It is enough
to prove the inequality
\begin{equation}
  \label{eq:given-inequality-1}
\int_{\bbR^n\minus B_\varepsilon(0)} f(x)\diff x<\frac{C}{\varepsilon}
\end{equation}
to prove that $f\in L^1\left(\bbR^n\minus B_\varepsilon(0)\right)$. Hence,
we proceed in this spirit. First, let us jot some estimates down. For any
$x\in B_\varepsilon(0)$, $|x|<\varepsilon$ so
\[
\int_{\bbR^n}
\]
\end{proof}

\begin{problem}
Let $\left\{f_k\right\}$ be a sequence of nonnegative measurable functions
on $\bbR^n$, and assume that $f_k$ converges pointwise almost everywhere to
a function $f$. If
\[
\int_{\bbR^n} f=\lim_{k\to\infty}\int_{\bbR^n} f_k<\infty,
\]
show that
\[
\int_E f=\lim_{k\to\infty}\int_E f_k
\]
for all measurable subsets $E$ of $\bbR^n$. Moreover, show that this is not
necessarily true if $\int_{\bbR^n} f=\lim_{k\to\infty} f_k=\infty$.
\end{problem}
\begin{proof}
\end{proof}

\begin{problem}
Assume that $E$ is a measurable set of $\bbR^n$, with $\lambda(E)<\infty$. Prove
that a nonnegative function $f$ defined on $E$ is integrable if and only if
\[
\sum_{k=0}^\infty\lambda\left(\left\{\,\bfx\in E:f(\bfx)\geq
    k\,\right\}\right)<\infty.
\]
\end{problem}
\begin{proof}
\end{proof}

\begin{problem}
Suppose that $E$ is a measurable subset of $\bbR^n$, with
$\lambda(E)<\infty$. If $f$ and $g$ are measurable functions on $E$, define
\[
\rho(f,g)=\int_E\frac{|f-g|}{1+|f-g|}.
\]
Prove that $\rho(f_k,g)\to 0$ as $k\to\infty$ if and only if $f_k$
converges to $f$ as $k\to\infty$.
\end{problem}
\begin{proof}
\end{proof}

\begin{problem}
Define the \emph{gamma function} $\Gamma\colon[0,\infty)\to\bbR$ by
\[
\Gamma(y)\coloneqq\int_0^\infty e^{-u}u^{y-1}\diff u,
\]
and the \emph{beta function} $\beta\colon[0,\infty)\times[0,\infty)\to\bbR$
by
\[
\beta(x,y)\coloneqq\int_0^1 t^{x-1}(1-t)^{y-1}\diff t.
\]
\begin{enumerate}[label=(\alph*)]
\item Prove that the definition of the gamma function is well-posed, i.e.,
  the function $u\mapsto e^{-u}u^{y-1}$ is in $L([0,\infty))$ for all
  $y\in[0,\infty)$.
\item Show that
\[
\beta(x,y)=\frac{\Gamma(x)\Gamma(y)}{\Gamma(x+y)}.
\]
\end{enumerate}
\end{problem}
\begin{proof}
\end{proof}

\begin{problem}
\end{problem}
\begin{proof}
\end{proof}

\begin{problem}
\end{problem}
\begin{proof}
\end{proof}

%%% Local Variables:
%%% mode: latex
%%% TeX-master: "../MA544-Quals"
%%% End:
