\section{Exam 2 Prep}
\begin{problem}
Define for $\bfx\in\bfR^n$,
\[
f(\bfx)\coloneqq
\begin{cases}
\left|\bfx\right|^{-(n+1)}&\text{if $\bfx\neq \mathbf{0}$,}\\
0&\text{if $\bfx=\mathbf{0}$.}
\end{cases}
\]
Prove that $f$ is integrable outside any ball $B(0,\varepsilon)$, and that
there exists a constant $C>0$ such that
\[
\int_{\bfR^n\minus B_\varepsilon(\mathbf{0})}f(\bfx)\diff\bfx\leq\frac{C}{\varepsilon}.
\]
\end{problem}
\begin{proof}
What does it mean for a measurable function $f$ to be integrable over a set
$E\subset\bfR^n$, i.e., that $f$ belong to $L^1(E)$? It means that
\[
\int_E f(\bfx)\diff\bfx<\infty,
\]
or equivalently that the integral of the absolute value of $f$ be
finite.

Put $E\coloneqq\bfR^n\minus B_\varepsilon(\mathbf{0})$ and
$E_i\coloneqq\bfR\minus B_\varepsilon(0)$ for $i=1,...,n$. Now, suppose $f$
is given as in the statement of the problem. It is enough to prove the
inequality
\begin{equation}
  \label{eq:given-inequality-1}
\int_E f(\bfx)\diff\bfx<\frac{C}{\varepsilon}
\end{equation}
to show that $f$ belongs to $L^1(E)$. Hence, we proceed in this
spirit. First, let us jot down some estimates on the integral. For any
$\bfx\in B_\varepsilon(\mathbf{0})$, $|\bfx|<\varepsilon$ so the integral
\begin{equation}
\label{eq:divergent-estimate-1}
\int_{B_\varepsilon(\mathbf{0})}f(\bfx)\diff\bfx=
\int_{B_\varepsilon(\mathbf{0})}\frac{\diff\bfx}{|\bfx|^{n+1}}\geq
\int_{B_\varepsilon(\mathbf{0})}\frac{\diff\bfx}{\varepsilon^{n+1}}=
\frac{\Vol B_\varepsilon(\mathbf{0})}{\varepsilon^{n+1}}
\end{equation}
for every $\varepsilon>0$, hence it diverges.

Now, taking the integral of $f$ directly we have
\begin{align*}
\int_E f(\bfx)\diff\bfx
&=\int_E\frac{\diff\bfx}{|\bfx|^{n+1}}\\
&=\int_E\frac{\diff\bfx}{\left(\sqrt{{x_1}^2+\cdots+{x_n}^2}\right)^{n+1}}\\
&=\idotsint_{E'}\frac{\diff x_1\cdots\diff x_n}
{\left(\sqrt{{x_1}^2+\cdots+{x_n}^2}\right)^{n+1}}\\
&=\idotsint_{E_1}
\left[\int_{E_1}\frac{\diff x_1}
{\left(\sqrt{{x_1}^2+\cdots+{x_n}^2}\right)^{n+1}}
\right]\diff x_2\cdots\diff x_n
\intertext{make the substitution,
  $\tan\theta=x_1/\sqrt{{x_2}^2+\cdots+{x_n}^2}$ and put
  $E_\theta\coloneqq(-\pi/2,\arctan(-\varepsilon))\cup(\arctan\varepsilon,\pi/2)$}
&=\int_{E_n}\cdots\int_{E_2}
\left[\int_{E_\theta}\frac{\cos^{n+1}\theta}{\left({x_2}^2+\cdots+{x_n}^2\right)^{(n+1)/2}}
({x_2}^2+\cdots+{x_n}^2)^{1/2}\sec^2\theta\diff\theta\right]\diff
  x_2\cdots\diff x_n\\
&=\int_{E_n}\cdots\int_{E_2}
\left[\int_{E_\theta}\frac{\cos^{n-1}\theta}{\left({x_2}^2+\cdots+{x_n}^2\right)^{n/2}}\diff\theta\right]\diff
  x_2\cdots\diff x_n\\
&=\int_{E_n}\cdots\int_{E_2}\frac{\diff x_2\cdots\diff
  x_n}{\left({x_2}^2+\cdots+{x_n}^2\right)^{n/2}}\left[\int_{E_\theta}\cos^{n-1}\theta\diff\diff\theta\right]
\end{align*}
where
\[
\int_{E_\theta}\cos^{n-1}\theta\diff\diff\theta<\infty.
\]
Proceeding in this fashion, we arrive at the desired inequality.

Here is the approach taken by Prof.\@ Danielli: Using spherical coordinates
$(x_1,...,x_n)\mapsto\left(\sqrt{{x_1}^2+\cdots+{x_n}^2},\vec\theta\right)$
\end{proof}

\begin{problem}
Let $\left\{f_k\right\}$ be a sequence of nonnegative measurable functions
on $\bfR^n$, and assume that $f_k$ converges pointwise almost everywhere to
a function $f$. If
\[
\int_{\bfR^n} f=\lim_{k\to\infty}\int_{\bfR^n} f_k<\infty,
\]
show that
\[
\int_E f=\lim_{k\to\infty}\int_E f_k
\]
for all measurable subsets $E$ of $\bfR^n$. Moreover, show that this is not
necessarily true if $\int_{\bfR^n} f=\lim_{k\to\infty} f_k=\infty$.
\end{problem}
\begin{proof}

\end{proof}

\begin{problem}
Assume that $E$ is a measurable set of $\bfR^n$, with $\lambda(E)<\infty$. Prove
that a nonnegative function $f$ defined on $E$ is integrable if and only if
\[
\sum_{k=0}^\infty\lambda\left(\left\{\,\bfx\in E:f(\bfx)\geq
    k\,\right\}\right)<\infty.
\]
\end{problem}
\begin{proof}
\end{proof}

\begin{problem}
Suppose that $E$ is a measurable subset of $\bfR^n$, with
$\lambda(E)<\infty$. If $f$ and $g$ are measurable functions on $E$, define
\[
\rho(f,g)=\int_E\frac{|f-g|}{1+|f-g|}.
\]
Prove that $\rho(f_k,g)\to 0$ as $k\to\infty$ if and only if $f_k$
converges to $f$ as $k\to\infty$.
\end{problem}
\begin{proof}
\end{proof}

\begin{problem}
Define the \emph{gamma function} $\Gamma\colon[0,\infty)\to\bfR$ by
\[
\Gamma(y)\coloneqq\int_0^\infty e^{-u}u^{y-1}\diff u,
\]
and the \emph{beta function} $\beta\colon[0,\infty)\times[0,\infty)\to\bfR$
by
\[
\beta(x,y)\coloneqq\int_0^1 t^{x-1}(1-t)^{y-1}\diff t.
\]
\begin{enumerate}[label=(\alph*)]
\item Prove that the definition of the gamma function is well-posed, i.e.,
  the function $u\mapsto e^{-u}u^{y-1}$ is in $L([0,\infty))$ for all
  $y\in[0,\infty)$.
\item Show that
\[
\beta(x,y)=\frac{\Gamma(x)\Gamma(y)}{\Gamma(x+y)}.
\]
\end{enumerate}
\end{problem}
\begin{proof}
\end{proof}

\begin{problem}
Let $f\in L^1(\bfR^n)$ and for $\mathbf{h}\in\bfR^n$ define
$f_{\mathbf{h}}\colon\bfR^n\to\bfR$ be $f_{\mathbf{h}}(x)\coloneqq
f(\bfx-\mathbf{h})$. Prove that
\[
\lim_{\mathbf{h}\to\mathbf{0}}\int_{\bfR^n}\left|f_{\mathbf{h}}-f\right|=0.
\]
\end{problem}
\begin{proof}
\end{proof}

\begin{problem}
\begin{enumerate}[label=(\alph*)]
\item If $f_k,g_k,f,g\in L^1(\bfR^n)$, $f_k\to f$ and $g_k\to g$ a.e.\@ in
  $\bfR^n$, $|f_k|\leq g_k$ and
\[
\int_{\bfR^n}g_k\to\int_{\bfR^n}g,
\]
prove that
\[
\int_{\bfR^n} f_k\to\int_{\bfR^n}f.
\]
\item Using part (a) show that if $f_k,f\in L^1(\bfR^n)$ and $f_k\to f$
  a.e.\@ in $\bfR^n$, then
\[
\int_{\bfR^n}|f_k-f|\to 0\qquad\text{as}\qquad k\to\infty
\]
if and only if
\[
\int_{\bfR^n}|f_k|\to\int_{\bfR^n}|f|\qquad\text{as}\qquad k\to\infty.
\]
\end{enumerate}
\end{problem}
\begin{proof}
\end{proof}

%%% Local Variables:
%%% mode: latex
%%% TeX-master: "../MA544-Quals"
%%% End:
