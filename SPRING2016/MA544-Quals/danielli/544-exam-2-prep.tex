\section{Exam 2 Prep}
\begin{problem}
Define for $\bfx\in\bfR^n$,
\[
f(\bfx)\coloneqq
\begin{cases}
\left|\bfx\right|^{-(n+1)}&\text{if $\bfx\neq \mathbf{0}$,}\\
0&\text{if $\bfx=\mathbf{0}$.}
\end{cases}
\]
Prove that $f$ is integrable outside any ball $B_\varepsilon(\mathbf{0})$,
and that there exists a constant $C>0$ such that
\[
\int_{\bfR^n\minus B_\varepsilon(\mathbf{0})}f(\bfx) d \bfx\leq\frac{C}{\varepsilon}.
\]
\end{problem}
\begin{proof}
Recall that a real-valued function $f\colon\bfR^n\to\bfR$ is (Lebesgue)
integrable over a subset $E$ of $\bfR^n$ (or, alternatively, $f$ belongs to
$L^1(E)$) if
\[
\int_E f(\bfx) d \bfx<\infty.
\]

Put $E\coloneqq\bfR^n\minus B_\varepsilon(\mathbf{0})$. Then, to show that
$f$ belongs to $L^1(E)$ it suffices to prove the inequality
\begin{equation}
\label{eq:eq:inequality-1}
\int_E f(\bfx) d \bfx<\frac{C}{\varepsilon}
\end{equation}
for some appropriate constant $C$. We proceed by directly computing the
Lebesgue integral of $f$ and employing Tonelli's theorem:
\begingroup
\allowdisplaybreaks
\begin{align*}
\int_Ef(\bfx) d \bfx
={}&\int_E\frac{ d \bfx}{|\bfx|^{n+1}}\\
={}&\idotsint_E\frac{ d  x_1\cdots d
  x_n}{\left({x_1}^2+\cdots+{x_n}^2\right)^{(n+1)/2}}
\intertext{let $E_i$ denote the projection of $E$ onto its $i$-th
  coordinate and make the trigonometric substitution
  $x_1=\sqrt{{x_2}^2+\cdots+{x_n}^2}\tan\theta$, $ d
  x_1=\sqrt{{x_2}^2+\cdots+{x_n}^2}\sec^2\theta d \theta$ with
  $\theta\in(-\pi/2,-\tan^{-1}(\varepsilon))\cup(\tan^{-1}(\varepsilon),\pi/2)$
  giving us the integral}
={}&\int_{E_n}\cdots\int_{E_2}\left[\frac{\cos^{n-1}\theta}{\left({x_2}^2+\cdots+{x_n}^2\right)^{n/2}} d \theta\right] d
  x_2\cdots d  x_n
\intertext{which, by Tonelli's theorem, is}
={}&\int_{E_n}\cdots\int_{E_2}
\frac{ d  x_2\cdots d  x_n}{\left({x_2}^2+\cdots+{x_n}^2\right)^{n/2}}
\left[\int_{E_\theta}\cos^{n-1}\theta d \theta\right]
\end{align*}
\endgroup
where the integral
\begin{equation}
\label{eq:finite-int-1}
\int_{E_\theta}\cos^{n-1}\theta d \theta<\infty.
\end{equation}
Proceeding in this manner, we eventually achieve the inequality
\begin{equation}
\label{eq:desired-inequality-1}
\begin{aligned}
\idotsint_Ef(\bfx) d \bfx
<{}&C'\int_{E_n}\frac{ d  x_n}{{x_n}^2}\\
={}&2C'\int_\varepsilon^\infty\frac{ d  x_n}{{x_n}^2}\\
={}&\frac{C}{\varepsilon}
\end{aligned}
\end{equation}
as desired.
\end{proof}

\begin{problem}
Let $\left\{f_k\right\}$ be a sequence of nonnegative measurable functions
on $\bfR^n$, and assume that $f_k$ converges pointwise almost everywhere to
a function $f$. If
\[
\int_{\bfR^n} f=\lim_{k\to\infty}\int_{\bfR^n} f_k<\infty,
\]
show that
\[
\int_E f=\lim_{k\to\infty}\int_E f_k
\]
for all measurable subsets $E$ of $\bfR^n$. Moreover, show that this is not
necessarily true if $\int_{\bfR^n} f=\lim_{k\to\infty} f_k=\infty$.
\end{problem}
\begin{proof}
This is probably some theorem I can't remember right now. But anyway, first
we shall establish that the limit $f$ of $\left\{f_k\right\}$ must be
nonnegative a.e. in $\bfR^n$. For assume otherwise. Then there exists a
collection of points $\bfx$ in $\bfR^n$ of nonzero $\bfR^n$-Lebesgue
measure such that $f(\bfx)<0$. But $f_k(\bfx)\geq 0$ for all
$k\in\bfN$. Set $0<\varepsilon<|f(\bfx)|$ then we have
\begin{equation}
\label{eq:estimate-contradiction-1}
|f(\bfx)-f_k(\bfx)|>|f(\bfx)|>\varepsilon
\end{equation}
for all $k$ which contradicts our assumption that $f_k\to f$ a.e.\@ on
$\bfR^n$. Therefore, the set of points $\bfx\in\bfR^n$ where $f(\bfx)<0$
must have measure zero.

Now, based on pointwise convergence a.e.\@ to $f$, given $\varepsilon>0$
for a.e.\@ $\bfx\in\bfR^n$ we have the following estimate
\begin{equation}
\label{eq:estimate-2}
|f(\bfx)-f_k(\bfx)|<\varepsilon
\end{equation}
for sufficiently large $k$; say $k$ greater than or equal to some index
$N\in\bfN$. Moreover, we are given convergence in $L^1(\bfR^n)$ of $f_k$ to
$f$
\begin{equation}
\label{eq:integral-estimate-1}
\int_{\bfR^n}f_k\to\int_{\bfR^n}f<\infty.
\end{equation}
By monotonicity of the Lebesgue integral (Theorem 5.5(iii)), this implies
that
\begin{equation}
\label{eq:estimate-monotonicity-2}
\int_E f\leq\int_{\bfR^n} f<\infty
\end{equation}
and
\begin{equation}
\label{eq:estimate-monotonicity-k-2}
\int_E f_k\leq\int_{\bfR^n}f_k<\infty
\end{equation}
for all $k\in\bfN$. By Theorem 5.5(ii), $f$ and the $f_k$'s are finite
a.e.\@ in $\bfR^n$ so for some sufficiently large real number $M$,
$|f|,|f_k|\leq M$ for a.e.\@ $\bfx\in\bfR^n$. In particular, for any
measurable subset $E$ of $\bfR^n$, $|f|,|f_k|\leq M$ for a.e.\@ $\bfx\in E$
so, by the bounded convergence theorem, we have the desired convergence
\begin{equation}
\label{eq:desired-convergence-2}
\int_E f_k\to\int_E f<\infty.
\end{equation}

However, if $f$ does not belong to $L^1(\bfR^n)$, i.e., its integral over
$\bfR^n$ is infinity, there is no guarantee that $f$ will be finite a.e.\@
in $\bfR^n$. This means that the bounded convergence theorem will fail to
ensure convergence in integral for any measurable subset $E$ of
$\bfR^n$. Let us demonstrate this with an example. Consider the sequence of
functions
\end{proof}

\begin{problem}
Assume that $E$ is a measurable set of $\bfR^n$, with
$|E|<\infty$. Prove that a nonnegative function $f$ defined
on $E$ is integrable if and only if
\[
\sum_{k=0}^\infty\left|\left\{\,\bfx\in E:f(\bfx)\geq
    k\,\right\}\right|<\infty.
\]
\end{problem}
\begin{proof}
If $f$ is integrable over a measurable subset $E$ of $\bfR^n$, then
\begin{equation}
\label{eq:integrability-3}
\int_E f(\bfx) d \bfx<\infty.
\end{equation}
Set $E_k\coloneqq\left\{\,\bfx\in E:k+1>f(\bfx)\geq k\,\right\}$ and
$F_k\coloneqq\left\{\,\bfx\in E:f(\bfx)\geq k\,\right\}$. Note the
following properties about the sets we have just defined: first, the
$E_k$'s are pairwise disjoint and the $F_k$'s are nested in the following
way $F_{k+1}\subset F_k$; second, $E=\bigcup_{k=1}^\infty E_k$ and
$E_k=F_k\minus F_{k+1}$. By Theorem 3.23, since the $E_k$'s are disjoint,
we have
\begin{equation}
  \label{eq:disjoint-measurable-sets-3}
|E|=\sum_{k=1}^\infty|E_k|<\infty.
\end{equation}
Now, since $k\chi_{E_k}(\bfx)\leq f(\bfx)\leq (k+1)\chi_{E_k}(\bfx)$ on
$E_k$, we have
\begin{equation}
\label{eq:estimates-E-k-3}
k|E_k|\leq\int_{E_k}f(\bfx) d \bfx\leq (k+1)|E_k|.
\end{equation}
Then we have the following upper and lower estimates on the integral of $f$
over $E$
\begin{equation}
\label{eq:upper-lower-estimates-3}
\sum_{k=0}^\infty k|E_k|\leq\int_E f(\bfx) d \bfx\leq\sum_{k=0}^\infty(k+1)|E_k|.
\end{equation}
But note that $|E_k|=|F_k\minus F_{k+1}|=|F_k|-|F_{k+1}|$ by Corollary 3.25
since the measures of $E_k$, $F_k$, and $F_{k+1}$ are all finite. Hence,
\eqref{eq:upper-lower-estimates-3} becomes
\begin{equation}
\label{eq:new-upper-lower-estimates-3}
\sum_{k=0}^\infty k\left(|F_k|-|F_{k+1}|\right)\leq
\int_E f(\bfx) d \bfx\leq
\sum_{k=0}^\infty (k+1)\left(|F_k|-|F_{k+1}|\right).
\end{equation}
A little manipulation of the series in the leftmost estimate gives us
\begin{equation}
\label{eq:leftmost-estimate-3}
\begin{aligned}
\sum_{k=0}^\infty k\left(|F_k|-|F_{k+1}|\right)
={}&\sum_{k=1}^\infty k|F_k|-\sum_{k=1}^\infty k|F_{k+1}|\\
={}&|F_1|+\sum_{k=2}^\infty k|F_k|-\sum_{k=1}^\infty k|F_{k+1}|\\
={}&|F_1|+\sum_{k=1}^\infty(k+1)|F_{k+1}|-\sum_{k=1}^\infty k|F_{k+1}\\
={}&|F_1|+\sum_{k=1}^\infty |F_{k+1}|\\
={}&\sum_{k=1}^\infty|F_{k+1}|
\end{aligned}
\end{equation}
and
\begin{equation}
\label{eq:rightmost-estimate-3}
\begin{aligned}
\sum_{k=0}^\infty(k+1)\left(|F_k|-|F_{k+1}|\right)
={}&\sum_{k=0}^\infty(k+1)|F_k|-\sum_{k=0}^\infty(k+1)|F_{k+1}|\\
={}&|F_0|+\sum_{k=1}^\infty(k+1)|F_k|-\sum_{k=0}^\infty(k+1)|F_{k+1}|\\
={}&|F_0|+\sum_{k=0}^\infty(k+2)|F_{k+1}|-\sum_{k=0}^\infty(k+1)|F_{k+1}|\\
={}&|F_0|+\sum_{k=0}^\infty|F_{k+1}|\\
={}&\sum_{k=0}^\infty|F_k|.
\end{aligned}
\end{equation}
Thus, from \eqref{eq:leftmost-estimate-3} and
\eqref{eq:rightmost-estimate-3}
\begin{equation}
\label{eq:final-upper-lower-estimates-3}
\sum_{k=1}^\infty|F_k|\leq\int_E f(\bfx) d \bfx\leq\sum_{k=0}^\infty|F_k|
\end{equation}
so the integral $\int_E f$ converges if and only if the sum
$\sum_{k=0}^\infty|F_k|$ converges.
\end{proof}
\begin{problem}
Suppose that $E$ is a measurable subset of $\bfR^n$, with
$|E|<\infty$. If $f$ and $g$ are measurable functions on
$E$, define
\[
\rho(f,g)\coloneqq\int_E\frac{|f-g|}{1+|f-g|}.
\]
Prove that $\rho(f_k,f)\to 0$ as $k\to\infty$ if and only if $f_k$
converges to $f$ as $k\to\infty$.
\end{problem}
\begin{proof}
$\implies$: First note that $\rho$ is strictly greater than or equal to
zero since it is the integral of a nonnegative function. Suppose that
$\rho(f_k,f)\to 0$ as $k\to\infty$. Then, given $\varepsilon>0$ there exist
an sufficiently large index $N$ such that for every $k\geq N$ we have
\begin{equation}
\label{eq:hypothesis-4}
\rho(f_k,g)=\int_E\frac{|f_k-f|}{1+|f_k-f|}<\varepsilon.
\end{equation}
By Theorem 5.11, this means that the map
\[
\frac{|f_k-f|}{1+|f_k-f|}
\]
is zero a.e.\@ in $E$ which happens if $|f_k-f|=0$ a.e.\@ in $E$.

$\impliedby$: Suppose that $f_k\to f$ as $k\to\infty$.

\bigskip

I don't know how to solve this. This is the intended solution:

$\implies$: Given $\varepsilon>0$, $\rho(f_k,f)\to 0$ implies that
\[
\int_{\left\{\,x\in
    E:|f_k(x)-f(x)|>\varepsilon\,\right\}}\frac{|f_k-f|}{1+|f_k-f} d
x\longrightarrow 0.
\]
Observe that the function $\Phi\colon\bfR^+\to\bfR$ given by
$\Phi(x)\coloneqq x/(1+x)$ is increasing on $\bfR^+$ and $0<\Psi(x)<1$,
hence
\[
\begin{aligned}
  \int_{\left\{\,x\in
      E:|f_k(x)-f(x)|>\varepsilon\,\right\}}\frac{|f_k-f|}{1+|f_k-f|} d
  x
&\geq\int_{\left\{\,x\in
    E:|f_k(x)-f(x)|>\varepsilon\,\right\}}\frac{\varepsilon}{1+\varepsilon} d
x\\
={}&\frac{\varepsilon}{1+\varepsilon}
\left|\left\{\,x\in E:|f_k(x)-f(x)|>\varepsilon\,\right\}\right|.
\end{aligned}
\]
Therefore,
\[
\left|\left\{\,x\in E:|f_k(x)-f(x)|>\varepsilon\,\right\}\right|
\leq\frac{1+\varepsilon}{\varepsilon}
\int_{\left\{\,x\in
    E:|f_k(x)-f(x)|>\varepsilon\,\right\}}\frac{|f_k-f|}{1+|f_k-f|} d  x
\longrightarrow 0
\]
as $k\to\infty$.

$\impliedby$: Conversely, given $\delta>0$, we have
\[
\begin{aligned}
\rho(f_k,f)
={}&\int_{\left\{\,x\in E:|f_k(x)-f(x)|>\delta\,\right\}}\frac{|f_k-f|}{1+|f_k-f|} d  x\\
&\phantom{{}={}}+\int_{\left\{\,x\in
    E:|f_k(x)-f(x)|\leq\delta\,\right\}}\frac{|f_k-f|}{1+|f_k-f|} d  x\\
\leq{}&\left|\left\{\,x\in
    E:|f_k(x)-f(x)|>\delta\,\right\}\right|+\frac{\delta}{1+\delta}|E|.
\end{aligned}
\]
Since $|E|<\infty$ and $\delta/(1+\delta)\searrow 0$, then for any
$\varepsilon>0$, there exists $\delta'>0$ such that
\[
\frac{\delta'}{1+\delta'}|E|<\frac{\varepsilon}{2}.
\]
If $f_k\to f$ as $k\to\infty$ in measure, then for the above $\delta'$
there is an index $N>0$ such that $k\geq N$ implies
\[
\left|\left\{\,x\in E:|f_k(x)-f(x)|>\delta'\,\right\}\right|<\frac{\varepsilon}{2}.
\]
Therefore, $f_k\to f$ in measure implies $\rho(f_k,f)\to 0$ as $k\to\infty$.
\end{proof}

\begin{problem}
Define the \emph{gamma function} $\Gamma\colon\bfR^+\to\bfR$ by
\[
\Gamma(y)\coloneqq\int_0^\infty e^{-u}u^{y-1} d  u,
\]
and the \emph{beta function} $\beta\colon\bfR^+\times\bfR^+\to\bfR$
by
\[
\beta(x,y)\coloneqq\int_0^1 t^{x-1}(1-t)^{y-1} d  t.
\]
\begin{enumerate}[label=(\alph*)]
\item Prove that the definition of the gamma function is well-posed, i.e.,
the function $u\mapsto e^{-u}u^{y-1}$ is in $L(\bfR^+)$ for all
$y\in\bfR^+$.
\item Show that
\[
\beta(x,y)=\frac{\Gamma(x)\Gamma(y)}{\Gamma(x+y)}.
\]
\end{enumerate}
\end{problem}
\begin{proof}
(a) Fix $y\in\bfR^+$. Then we must show that $\Gamma(y)<\infty$. First,
since $(0,1)$ and $[1,\infty)$ are disjoint measurable subsets of $\bfR$,
by Theorem 5.7 we can split the integral $\Gamma(y)$ into
\begin{equation}
\label{eq:split-integral-5}
\Gamma(y)=\underbrace{\int_0^1 e^{-u}u^{y-1} d  u}_{I_1}
+\underbrace{\int_1^\infty e^{-u}u^{y-1} d  u}_{I_2}.
\end{equation}
We will show, separately, that $I_1$ and $I_2$ are finite.

To see that $I_1$ is finite, note that
\begin{equation}
\label{eq:estimate-1-5}
\begin{aligned}
e^{-u}u^{y-1}={}&e^{-u}e^{(y-1)\log u}\\
={}&e^{-u+(y-1)\log u}\\
\leq{}&e^{(y-1)\log u}\\
={}&u^{y-1}
\end{aligned}
\end{equation}
since $0<u<1$
\begin{equation}
\label{eq:estimate-i-1-5}
\begin{aligned}
I_1={}&\int_0^1 e^{-u}u^{y-1} d  u\\
\leq{}&\int_0^1 u^{y-1} d  u\\
={}&\left[\frac{u^y}{y}\right]_0^1\\
={}&\frac{1}{y}\\
<{}&\infty.
\end{aligned}
\end{equation}

To see that $I_2$ is finite, note that
\begin{equation}
\label{eq:estimate-2-5}
e
\end{equation}

\textbf{Intended solution:}
\\\\
(b)
\end{proof}

\begin{problem}
Let $f\in L^1(\bfR^n)$ and for $\mathbf{h}\in\bfR^n$ define
$f_{\mathbf{h}}\colon\bfR^n\to\bfR$ be $f_{\mathbf{h}}(\bfx)\coloneqq
f(\bfx-\mathbf{h})$. Prove that
\[
\lim_{\mathbf{h}\to\mathbf{0}}\int_{\bfR^n}\left|f_{\mathbf{h}}-f\right|=0.
\]
\end{problem}
\begin{proof}
Note that by the triangle inequality, we have the following estimate on the
integral
\begin{equation}
\label{eq:minokwski-estimate-6}
\int_{\bfR^n}|f_{\mathbf{h}}(\bfx)-f(\bfx)| d \bfx\leq
\end{equation}

\end{proof}

\begin{problem}
\begin{enumerate}[label=(\alph*)]
\item If $f_k,g_k,f,g\in L^1(\bfR^n)$, $f_k\to f$ and $g_k\to g$ a.e.\@ in
  $\bfR^n$, $|f_k|\leq g_k$ and
\[
\int_{\bfR^n}g_k\to\int_{\bfR^n}g,
\]
prove that
\[
\int_{\bfR^n} f_k\to\int_{\bfR^n}f.
\]
\item Using part (a) show that if $f_k,f\in L^1(\bfR^n)$ and $f_k\to f$
  a.e.\@ in $\bfR^n$, then
\[
\int_{\bfR^n}|f_k-f|\to 0\qquad\text{as}\qquad k\to\infty
\]
if and only if
\[
\int_{\bfR^n}|f_k|\to\int_{\bfR^n}|f|\qquad\text{as}\qquad k\to\infty.
\]
\end{enumerate}
\end{problem}
\begin{proof}
(a) Since $f_k\to f$ and $g_k\to g$ a.e.\@ and $|f_k|\leq g_k$, then by
Fatou's theorem,
\begin{align*}
\int_{\bfR^n}(g-f)
={}&\int_{\bfR^n}\liminf_{k\to\infty} g_k-f_k
\leq\liminf_{k\to\infty}\int_{\bfR^n}g_k-f_k,\\
\int_{\bfR^n}g+f&\int_{\bfR^n}\liminf_{k\to\infty}g_k+f_k\leq\liminf_{k\to\infty}\int_{\bfR^n}g_k+f_k.
\end{align*}
Since $f_k,g_k,f,g\in L^1(\bfR^n)$ and $\int_{\bfR^n}g_k\to\int_{\bfR^n}g$,
then using the similar argument as problem 2, we have
\[
  \begin{aligned}
   \int_{\bfR^n}f\geq\limsup_{k\to\infty}\int_{\bfR^n}f_k,\\
   \int_{\bfR^n}f\leq\liminf_{k\to\infty}\int_{\bfR^n} f_k.
  \end{aligned}
\]
Therefore, $\int_{\bfR^n}f_k\to\int_{\bfR^n} f$.
\\\\
(b) $\implies$: This direction is obvious by the inequality
\[
\left|\int_{\bfR^n}|f_k|-|f|\right|\leq\int_{\bfR^n}\left||f_k|-|f|\right|\leq\int_{\bfR^n}|f_k-f|.
\]

$\impliedby$: Let $g_k\coloneqq |f_k|+|f|$ and $g\coloneqq 2|f|$. Since
$f_k,f\in L^1(\bfR^n)$ and $f_k\to f$ a.e., then $g_k,g\in L^1(\bfR^n)$ and
$g_k\to g$ a.e.\@ in $\bfR^n$. By the assumption, $\int_{\bfR^n}
g_k\to\int_{\bfR^n}g$.

Let $\tilde f_k\coloneqq|f_k-f|$. Then $\tilde f_k\to 0$ a.e.\@ in $\bfR^n$
and $\tilde f_k\leq g_k$. Applying part (a) to $\tilde f_k$ we have
\[
\lim_{k\to\infty}\int_{\bfR^n}\tilde f_k=\lim_{k\to\infty}\int_{\bfR^n}|f_k-f|=0.
\]
\end{proof}
%%% Local Variables:
%%% mode: latex
%%% TeX-master: "../MA544-Quals"
%%% End:
