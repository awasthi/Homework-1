\section{Exam 2 Prep}
\begin{problem}
Define for $\bfx\in\bfR^n$,
\[
f(\bfx)\coloneqq
\begin{cases}
\left|\bfx\right|^{-(n+1)}&\text{if $\bfx\neq \mathbf{0}$,}\\
0&\text{if $\bfx=\mathbf{0}$.}
\end{cases}
\]
Prove that $f$ is integrable outside any ball $B_\varepsilon(\mathbf{0})$,
and that there exists a constant $C>0$ such that
\[
\int_{\bfR^n\minus B_\varepsilon(\mathbf{0})}f(\bfx)\diff\bfx\leq\frac{C}{\varepsilon}.
\]
\end{problem}
\begin{proof}
Recall that a real-valued function $f\colon\bfR^n\to\bfR$ is (Lebesgue)
integrable over a subset $E$ of $\bfR^n$ (or, alternatively, $f$ belongs to
$L^1(E)$) if
\[
\int_E f(\bfx)\diff\bfx<\infty.
\]

Put $E\coloneqq\bfR^n\minus B_\varepsilon(\mathbf{0})$. Then, to show that
$f$ belongs to $L^1(E)$ it suffices to prove the inequality
\begin{equation}
\label{eq:eq:inequality-1}
\int_E f(\bfx)\diff\bfx<\frac{C}{\varepsilon}
\end{equation}
for some appropriate constant $C$. We proceed by directly computing the
Lebesgue integral of $f$ and employing Tonelli's theorem:
\begingroup
\allowdisplaybreaks
\begin{align*}
\int_Ef(\bfx)\diff\bfx
&=\int_E\frac{\diff\bfx}{|\bfx|^{n+1}}\\
&=\idotsint_E\frac{\diff x_1\cdots\diff
  x_n}{\left({x_1}^2+\cdots+{x_n}^2\right)^{(n+1)/2}}
\intertext{let $E_i$ denote the projection of $E$ onto its $i$-th
  coordinate and make the trigonometric substitution
  $x_1=\sqrt{{x_2}^2+\cdots+{x_n}^2}\tan\theta$, $\diff
  x_1=\sqrt{{x_2}^2+\cdots+{x_n}^2}\sec^2\theta\diff\theta$ with
  $\theta\in(-\pi/2,-\tan^{-1}(\varepsilon))\cup(\tan^{-1}(\varepsilon),\pi/2)$
  giving us the integral}
&=\int_{E_n}\cdots\int_{E_2}\left[\frac{\cos^{n-1}\theta}{\left({x_2}^2+\cdots+{x_n}^2\right)^{n/2}}\diff\theta\right]\!\diff
  x_2\cdots\diff x_n
\intertext{which, by Tonelli's theorem, is}
&=\int_{E_n}\cdots\int_{E_2}
\frac{\diff x_2\cdots\diff x_n}{\left({x_2}^2+\cdots+{x_n}^2\right)^{n/2}}
\left[\int_{E_\theta}\cos^{n-1}\theta\diff\theta\right]
\end{align*}
\endgroup
where the integral
\begin{equation}
\label{eq:finite-int-1}
\int_{E_\theta}\cos^{n-1}\theta\diff\theta<\infty.
\end{equation}
Proceeding in this manner, we eventually achieve the inequality
\begin{equation}
\label{eq:desired-inequality-1}
\begin{aligned}
\idotsint_Ef(\bfx)\diff\bfx
&<C'\int_{E_n}\frac{\diff x_n}{{x_n}^2}\\
&=2C'\int_\varepsilon^\infty\frac{\diff x_n}{{x_n}^2}\\
&=\frac{C}{\varepsilon}
\end{aligned}
\end{equation}
as desired.
\end{proof}

\begin{problem}
Let $\left\{f_k\right\}$ be a sequence of nonnegative measurable functions
on $\bfR^n$, and assume that $f_k$ converges pointwise almost everywhere to
a function $f$. If
\[
\int_{\bfR^n} f=\lim_{k\to\infty}\int_{\bfR^n} f_k<\infty,
\]
show that
\[
\int_E f=\lim_{k\to\infty}\int_E f_k
\]
for all measurable subsets $E$ of $\bfR^n$. Moreover, show that this is not
necessarily true if $\int_{\bfR^n} f=\lim_{k\to\infty} f_k=\infty$.
\end{problem}
\begin{proof}
This is probably some theorem I can't remember right now. But anyway, first
we shall establish that the limit $f$ of $\left\{f_k\right\}$ must be
nonnegative a.e. in $\bfR^n$. For assume otherwise. Then there exists a
collection of points $\bfx$ in $\bfR^n$ of nonzero $\bfR^n$-Lebesgue
measure such that $f(\bfx)<0$. But $f_k(\bfx)\geq 0$ for all
$k\in\bfN$. Set $0<\varepsilon<|f(\bfx)|$ then we have
\begin{equation}
\label{eq:estimate-contradiction-1}
|f(\bfx)-f_k(\bfx)|>|f(\bfx)|>\varepsilon
\end{equation}
for all $k$ which contradicts our assumption that $f_k\to f$ a.e.\@ on
$\bfR^n$. Therefore, the set of points $\bfx\in\bfR^n$ where $f(\bfx)<0$
must have measure zero.

Now, based on pointwise convergence a.e.\@ to $f$, given $\varepsilon>0$
for a.e.\@ $\bfx\in\bfR^n$ we have the following estimate
\begin{equation}
\label{eq:estimate-2}
|f(\bfx)-f_k(\bfx)|<\varepsilon
\end{equation}
for sufficiently large $k$; say $k$ greater than or equal to some index
$N\in\bfN$. Moreover, we are given convergence in $L^1(\bfR^n)$ of $f_k$ to
$f$
\begin{equation}
\label{eq:integral-estimate-1}
\int_{\bfR^n}f_k\to\int_{\bfR^n}f<\infty.
\end{equation}
By monotonicity of the Lebesgue integral (Theorem 5.5(iii)), this implies
that
\begin{equation}
\label{eq:estimate-monotonicity-2}
\int_E f\leq\int_{\bfR^n} f<\infty
\end{equation}
and
\begin{equation}
\label{eq:estimate-monotonicity-k-2}
\int_E f_k\leq\int_{\bfR^n}f_k<\infty
\end{equation}
for all $k\in\bfN$. By Theorem 5.5(ii), $f$ and the $f_k$'s are finite
a.e.\@ in $\bfR^n$ so for some sufficiently large real number $M$,
$|f|,|f_k|\leq M$ for a.e.\@ $\bfx\in\bfR^n$. In particular, for any
measurable subset $E$ of $\bfR^n$, $|f|,|f_k|\leq M$ for a.e.\@ $\bfx\in E$
so, by the bounded convergence theorem, we have the desired convergence
\begin{equation}
\label{eq:desired-convergence-2}
\int_E f_k\to\int_E f<\infty.
\end{equation}

However, if $f$ does not belong to $L^1(\bfR^n)$, i.e., its integral over
$\bfR^n$ is infinity, there is no guarantee that $f$ will be finite a.e.\@
in $\bfR^n$. This means that the bounded convergence theorem will fail to
ensure convergence in integral for any measurable subset $E$ of
$\bfR^n$. Let us demonstrate this with an example. Consider the sequence of
functions
\end{proof}

\begin{problem}
Assume that $E$ is a measurable set of $\bfR^n$, with
$|E|<\infty$. Prove that a nonnegative function $f$ defined
on $E$ is integrable if and only if
\[
\sum_{k=0}^\infty\left|\left\{\,\bfx\in E:f(\bfx)\geq
    k\,\right\}\right|<\infty.
\]
\end{problem}
\begin{proof}
If $f$ is integrable over a measurable subset $E$ of $\bfR^n$, then
\begin{equation}
\label{eq:integrability-3}
\int_E f(\bfx)\diff\bfx<\infty.
\end{equation}
Set $E_k\coloneqq\left\{\,\bfx\in E:k+1>f(\bfx)\geq k\,\right\}$ and
$F_k\coloneqq\left\{\,\bfx\in E:f(\bfx)\geq k\,\right\}$. Note the
following properties about the sets we have just defined: first, the
$E_k$'s are pairwise disjoint and the $F_k$'s are nested in the following
way $F_{k+1}\subset F_k$; second, $E=\bigcup_{k=1}^\infty E_k$ and
$E_k=F_k\minus F_{k+1}$. By Theorem 3.23, since the $E_k$'s are disjoint,
we have
\begin{equation}
  \label{eq:disjoint-measurable-sets-3}
|E|=\sum_{k=1}^\infty|E_k|<\infty.
\end{equation}
Now, since $k\chi_{E_k}(\bfx)\leq f(\bfx)\leq (k+1)\chi_{E_k}(\bfx)$ on
$E_k$, we have
\begin{equation}
\label{eq:estimates-E-k-3}
k|E_k|\leq\int_{E_k}f(\bfx)\diff\bfx\leq (k+1)|E_k|.
\end{equation}
Then we have the following upper and lower estimates on the integral of $f$
over $E$
\begin{equation}
\label{eq:upper-lower-estimates-3}
\sum_{k=0}^\infty k|E_k|\leq\int_E f(\bfx)\diff\bfx\leq\sum_{k=0}^\infty(k+1)|E_k|.
\end{equation}
But note that $|E_k|=|F_k\minus F_{k+1}|=|F_k|-|F_{k+1}|$ by Corollary 3.25
since the measures of $E_k$, $F_k$, and $F_{k+1}$ are all finite. Hence,
\eqref{eq:upper-lower-estimates-3} becomes
\begin{equation}
\label{eq:new-upper-lower-estimates-3}
\sum_{k=0}^\infty k\left(|F_k|-|F_{k+1}|\right)\leq
\int_E f(\bfx)\diff\bfx\leq
\sum_{k=0}^\infty (k+1)\left(|F_k|-|F_{k+1}|\right).
\end{equation}
A little manipulation of the series in the leftmost estimate gives us
\begin{equation}
\label{eq:leftmost-estimate-3}
\begin{aligned}
\sum_{k=0}^\infty k\left(|F_k|-|F_{k+1}|\right)
&=\sum_{k=1}^\infty k|F_k|-\sum_{k=1}^\infty k|F_{k+1}|\\
&=|F_1|+\sum_{k=2}^\infty k|F_k|-\sum_{k=1}^\infty k|F_{k+1}|\\
&=|F_1|+\sum_{k=1}^\infty(k+1)|F_{k+1}|-\sum_{k=1}^\infty k|F_{k+1}\\
&=|F_1|+\sum_{k=1}^\infty |F_{k+1}|\\
&=\sum_{k=1}^\infty|F_{k+1}|
\end{aligned}
\end{equation}
and
\begin{equation}
\label{eq:rightmost-estimate-3}
\begin{aligned}
\sum_{k=0}^\infty(k+1)\left(|F_k|-|F_{k+1}|\right)
&=\sum_{k=0}^\infty(k+1)|F_k|-\sum_{k=0}^\infty(k+1)|F_{k+1}|\\
&=|F_0|+\sum_{k=1}^\infty(k+1)|F_k|-\sum_{k=0}^\infty(k+1)|F_{k+1}|\\
&=|F_0|+\sum_{k=0}^\infty(k+2)|F_{k+1}|-\sum_{k=0}^\infty(k+1)|F_{k+1}|\\
&=|F_0|+\sum_{k=0}^\infty|F_{k+1}|\\
&=\sum_{k=0}^\infty|F_k|.
\end{aligned}
\end{equation}
Thus, from \eqref{eq:leftmost-estimate-3} and
\eqref{eq:rightmost-estimate-3}
\begin{equation}
\label{eq:final-upper-lower-estimates-3}
\sum_{k=1}^\infty|F_k|\leq\int_E f(\bfx)\diff\bfx\leq\sum_{k=0}^\infty|F_k|
\end{equation}
so the integral $\int_E f$ converges if and only if the sum
$\sum_{k=0}^\infty|F_k|$ converges.
\end{proof}
\begin{problem}
Suppose that $E$ is a measurable subset of $\bfR^n$, with
$|E|<\infty$. If $f$ and $g$ are measurable functions on
$E$, define
\[
\rho(f,g)\coloneqq\int_E\frac{|f-g|}{1+|f-g|}.
\]
Prove that $\rho(f_k,f)\to 0$ as $k\to\infty$ if and only if $f_k$
converges to $f$ as $k\to\infty$.
\end{problem}
\begin{proof}
$\implies$: First note that $\rho$ is strictly greater than or equal to
zero since it is the integral of a nonnegative function. Suppose that
$\rho(f_k,f)\to 0$ as $k\to\infty$. Then, given $\varepsilon>0$ there exist
an sufficiently large index $N$ such that for every $k\geq N$ we have
\begin{equation}
\label{eq:hypothesis-4}
\rho(f_k,g)=\int_E\frac{|f_k-f|}{1+|f_k-f|}<\varepsilon.
\end{equation}
By Theorem 5.11, this means that the map
\[
\frac{|f_k-f|}{1+|f_k-f|}
\]
is zero a.e.\@ in $E$ which happens if $|f_k-f|=0$ a.e.\@ in $E$.

$\impliedby$: Suppose that $f_k\to f$ as $k\to\infty$.

\bigskip

I don't know how to solve this. This is the intended solution:

$\implies$: Given $\varepsilon>0$, $\rho(f_k,f)\to 0$ implies that
\[
\int_{\left\{\,x\in
    E:|f_k(x)-f(x)|>\varepsilon\,\right\}}\frac{|f_k-f|}{1+|f_k-f}\diff
x\longrightarrow 0.
\]
Observe that the function $\Phi\colon\bfR^+\to\bfR$ given by
$\Phi(x)\coloneqq x/(1+x)$ is increasing on $\bfR^+$ and $0<\Psi(x)<1$,
hence
\[
\begin{aligned}
  \int_{\left\{\,x\in
      E:|f_k(x)-f(x)|>\varepsilon\,\right\}}\frac{|f_k-f|}{1+|f_k-f|}\diff
  x
&\geq\int_{\left\{\,x\in
    E:|f_k(x)-f(x)|>\varepsilon\,\right\}}\frac{\varepsilon}{1+\varepsilon}\diff
x\\
&=\frac{\varepsilon}{1+\varepsilon}
\left|\left\{\,x\in E:|f_k(x)-f(x)|>\varepsilon\,\right\}\right|.
\end{aligned}
\]
Therefore,
\[
\left|\left\{\,x\in E:|f_k(x)-f(x)|>\varepsilon\,\right\}\right|
\leq\frac{1+\varepsilon}{\varepsilon}
\int_{\left\{\,x\in
    E:|f_k(x)-f(x)|>\varepsilon\,\right\}}\frac{|f_k-f|}{1+|f_k-f|}\diff x
\longrightarrow 0
\]
as $k\to\infty$.

$\impliedby$: Conversely, given $\delta>0$, we have
\[
\begin{aligned}
\rho(f_k,f)
&=\int_{\left\{\,x\in E:|f_k(x)-f(x)|>\delta\,\right\}}\frac{|f_k-f|}{1+|f_k-f|}\diff x\\
&\phantom{{}={}}+\int_{\left\{\,x\in
    E:|f_k(x)-f(x)|\leq\delta\,\right\}}\frac{|f_k-f|}{1+|f_k-f|}\diff x\\
&\leq\left|\left\{\,x\in
    E:|f_k(x)-f(x)|>\delta\,\right\}\right|+\frac{\delta}{1+\delta}|E|.
\end{aligned}
\]
Since $|E|<\infty$ and $\delta/(1+\delta)\searrow 0$, then for any
$\varepsilon>0$, there exists $\delta'>0$ such that
\[
\frac{\delta'}{1+\delta'}|E|<\frac{\varepsilon}{2}.
\]
If $f_k\to f$ as $k\to\infty$ in measure, then for the above $\delta'$
there is an index $N>0$ such that $k\geq N$ implies
\[
\left|\left\{\,x\in E:|f_k(x)-f(x)|>\delta'\,\right\}\right|<\frac{\varepsilon}{2}.
\]
Therefore, $f_k\to f$ in measure implies $\rho(f_k,f)\to 0$ as $k\to\infty$.
\end{proof}

\begin{problem}
Define the \emph{gamma function} $\Gamma\colon\bfR^+\to\bfR$ by
\[
\Gamma(y)\coloneqq\int_0^\infty e^{-u}u^{y-1}\diff u,
\]
and the \emph{beta function} $\beta\colon\bfR^+\times\bfR^+\to\bfR$
by
\[
\beta(x,y)\coloneqq\int_0^1 t^{x-1}(1-t)^{y-1}\diff t.
\]
\begin{enumerate}[label=(\alph*)]
\item Prove that the definition of the gamma function is well-posed, i.e.,
the function $u\mapsto e^{-u}u^{y-1}$ is in $L(\bfR^+)$ for all
$y\in\bfR^+$.
\item Show that
\[
\beta(x,y)=\frac{\Gamma(x)\Gamma(y)}{\Gamma(x+y)}.
\]
\end{enumerate}
\end{problem}
\begin{proof}
(a) Fix $y\in\bfR^+$. Then we must show that $\Gamma(y)<\infty$. First,
since $(0,1)$ and $[1,\infty)$ are disjoint measurable subsets of $\bfR$,
by Theorem 5.7 we can split the integral $\Gamma(y)$ into
\begin{equation}
\label{eq:split-integral-5}
\Gamma(y)=\underbrace{\int_0^1 e^{-u}u^{y-1}\diff u}_{I_1}
+\underbrace{\int_1^\infty e^{-u}u^{y-1}\diff u}_{I_2}.
\end{equation}
We will show, separately, that $I_1$ and $I_2$ are finite.

To see that $I_1$ is finite, note that
\begin{equation}
\label{eq:estimate-1-5}
\begin{aligned}
e^{-u}u^{y-1}&=e^{-u}e^{(y-1)\log u}\\
&=e^{-u+(y-1)\log u}\\
&\leq e^{(y-1)\log u}\\
&=u^{y-1}
\end{aligned}
\end{equation}
since $0<u<1$
\begin{equation}
\label{eq:estimate-i-1-5}
\begin{aligned}
I_1&=\int_0^1 e^{-u}u^{y-1}\diff u\\
&\leq\int_0^1 u^{y-1}\diff u\\
&=\left[\frac{u^y}{y}\right]_0^1\\
&=\frac{1}{y}\\
&<\infty.
\end{aligned}
\end{equation}

To see that $I_2$ is finite, note that
\begin{equation}
\label{eq:estimate-2-5}
e
\end{equation}

\textbf{Intended solution:}
\\\\
(b)
\end{proof}

\begin{problem}
Let $f\in L^1(\bfR^n)$ and for $\mathbf{h}\in\bfR^n$ define
$f_{\mathbf{h}}\colon\bfR^n\to\bfR$ be $f_{\mathbf{h}}(\bfx)\coloneqq
f(\bfx-\mathbf{h})$. Prove that
\[
\lim_{\mathbf{h}\to\mathbf{0}}\int_{\bfR^n}\left|f_{\mathbf{h}}-f\right|=0.
\]
\end{problem}
\begin{proof}
Note that by the triangle inequality, we have the following estimate on the
integral
\begin{equation}
\label{eq:minokwski-estimate-6}
\int_{\bfR^n}|f_{\mathbf{h}}(\bfx)-f(\bfx)|\diff\bfx\leq
\end{equation}

\end{proof}

\begin{problem}
\begin{enumerate}[label=(\alph*)]
\item If $f_k,g_k,f,g\in L^1(\bfR^n)$, $f_k\to f$ and $g_k\to g$ a.e.\@ in
  $\bfR^n$, $|f_k|\leq g_k$ and
\[
\int_{\bfR^n}g_k\to\int_{\bfR^n}g,
\]
prove that
\[
\int_{\bfR^n} f_k\to\int_{\bfR^n}f.
\]
\item Using part (a) show that if $f_k,f\in L^1(\bfR^n)$ and $f_k\to f$
  a.e.\@ in $\bfR^n$, then
\[
\int_{\bfR^n}|f_k-f|\to 0\qquad\text{as}\qquad k\to\infty
\]
if and only if
\[
\int_{\bfR^n}|f_k|\to\int_{\bfR^n}|f|\qquad\text{as}\qquad k\to\infty.
\]
\end{enumerate}
\end{problem}
\begin{proof}
(a) Since $f_k\to f$ and $g_k\to g$ a.e.\@ and $|f_k|\leq g_k$, then by
Fatou's theorem,
\begin{align*}
\int_{\bfR^n}(g-f)
&=\int_{\bfR^n}\liminf_{k\to\infty} g_k-f_k
\leq\liminf_{k\to\infty}\int_{\bfR^n}g_k-f_k,\\
\int_{\bfR^n}g+f&\int_{\bfR^n}\liminf_{k\to\infty}g_k+f_k\leq\liminf_{k\to\infty}\int_{\bfR^n}g_k+f_k.
\end{align*}
Since $f_k,g_k,f,g\in L^1(\bfR^n)$ and $\int_{\bfR^n}g_k\to\int_{\bfR^n}g$,
then using the similar argument as problem 2, we have
\[
  \begin{aligned}
   \int_{\bfR^n}f\geq\limsup_{k\to\infty}\int_{\bfR^n}f_k,\\
   \int_{\bfR^n}f\leq\liminf_{k\to\infty}\int_{\bfR^n} f_k.
  \end{aligned}
\]
Therefore, $\int_{\bfR^n}f_k\to\int_{\bfR^n} f$.
\\\\
(b) $\implies$: This direction is obvious by the inequality
\[
\left|\int_{\bfR^n}|f_k|-|f|\right|\leq\int_{\bfR^n}\left||f_k|-|f|\right|\leq\int_{\bfR^n}|f_k-f|.
\]

$\impliedby$: Let $g_k\coloneqq |f_k|+|f|$ and $g\coloneqq 2|f|$. Since
$f_k,f\in L^1(\bfR^n)$ and $f_k\to f$ a.e., then $g_k,g\in L^1(\bfR^n)$ and
$g_k\to g$ a.e.\@ in $\bfR^n$. By the assumption, $\int_{\bfR^n}
g_k\to\int_{\bfR^n}g$.

Let $\tilde f_k\coloneqq|f_k-f|$. Then $\tilde f_k\to 0$ a.e.\@ in $\bfR^n$
and $\tilde f_k\leq g_k$. Applying part (a) to $\tilde f_k$ we have
\[
\lim_{k\to\infty}\int_{\bfR^n}\tilde f_k=\lim_{k\to\infty}\int_{\bfR^n}|f_k-f|=0.
\]
\end{proof}
\subsection{Review of concepts}
To conclude this review sheet, here are some important lemmas, theorems,
and corollaries from the book:

Let $f$ be defined on $E$, and let $\bfx_0$ be a limit point of $E$ in
$E$. Then $f$ is said to be \emph{upper semicontinuous at $\bfx_0$} if
\begin{equation}
  \label{eq:upper-semicontinuous}
\limsup_{\substack{\bfx\to\bfx_0\\\bfx\in E}}f(\bfx)\leq f(\bfx_0).
\end{equation}
Note that if $f(\bfx_0)=\infty$, then $f$ is usc at $\bfx_0$
automatically; otherwise, the statement that $f$ is usc at $\bfx_0$ means
that given any $M>f(\bfx_0)$, there exists $\delta>0$ such that $f(\bfx)<M$
for all $\bfx\in E$ that lie in the ball $B_\delta(\bfx_0)$.

Similarly, $f$ is said to be \emph{lower semicontinuous at $\bfx_0$} if
$-f$ is usc at $\bfx_0$.

\begin{theorem*}[4.14]
A function $f$ is usc relative to $E$ if and only if $\left\{\,\bfx\in
  E:f(\bfx)>a\,\right\}$ is relatively closed (equivalently, if
$\left\{\,\bfx\in E:f(\bfx)<a\,\right\}$ is relatively open) for all finite
$a$
\end{theorem*}
\begin{proof}[Proof of theorem 4.14]
Suppose that $f$ is usc relative to $E$. Given $a$, let $\bfx_0$ be a limit
point of $\left\{\,\bfx\in E:f(\bfx)>a\,\right\}$ in $E$. Then there exists
$\bfx_k\in E$ such that $\bfx_k\to\bfx_0$ and $f(\bfx_k)\geq a$. Since $f$
is usc at $\bfx_0$, we have $f(\bfx_0)\geq\limsup_{k\to\infty}
f(\bfx_k)$. Therefore, $f(\bfx_0)\geq a$, so $\bfx_0\in\left\{\,\bfx\in
  E:f(\bfx)>a\,\right\}$. Hence, $\left\{\,\bfx\in E:f(\bfx)>a\,\right\}$
is relatively closed.

Conversely, let $\bfx_0$ be a limit point of $E$ that is in $E$. If $f$ is
not usc at $\bfx_0$, then $f(\bfx_0)<\infty$, and there exists $M$ and
$\left\{ \bfx_k \right\}$ such that $f(\bfx_0)<M$, $\bfx_k\in E$,
$\bfx_k\to\bfx_0$, and $f(\bfx_k)\geq M$. Hence, $\left\{\,\bfx\in
  E:f(\bfx)>a\,\right\}$ is not relatively closed since it does not contain
all its limit points in $E$>
\end{proof}
\begin{theorem*}[4.17, Egorov's theorem]
Suppose that $\{f_k\}$ is a sequence of measurable functions that converge
a.e.\@ in a set $E$ of finite measure to a finite limit $f$. Then given
$\varepsilon>0$ there exits a closed subset $F$ of $E$ such that $|E\minus
F|<\varepsilon$ and $f_k\to f$ uniformly on $F$.
\end{theorem*}
A function $f$ defined on a measurable set $E$ has \emph{property $\calC$}
on $E$ if given $\varepsilon>0$, there is a closed set $F\subset E$ such
that
\begin{enumerate}[label=(\roman*)]
\item $|E\minus F|<\varepsilon$
\item $f$ is continuous relative to $F$.
\end{enumerate}
\begin{theorem*}[4.20, Lusin's theorem]
Let $f$ be defined and finite on a measurable set $E$. Then $f$ is
measurable if and only if it has property $\calC$ on $E$.
\end{theorem*}

We start with a nonnegative function $f$ defined on a measurable subset $E$
of $\bfR^n$. Let's
\begin{equation}
\label{eq:graph-and-region}
\begin{aligned}
\Gamma(f,E)&\coloneqq\left\{\,(\bfx,f(\bfx))\in\bfR^{n+1}:\text{$\bfx\in
    E$, $f(\bfx)<\infty$}\,\right\},\\
R(f,E)&\coloneqq\left\{\,(\bfx,y)\in\bfR^{n+1}:\text{$\bfx\in E$, $0\leq
    y\leq f(\bfx)$ if $f(\bfx)<\infty$and $0\leq y<\infty$ if
    $f(\bfx)=\infty$}\,\right\}.
\end{aligned}
\end{equation}
$\Gamma(f,E)$ is called the \emph{graph of $f$ over $E$} and $R(f,E)$ the
\emph{region under $f$ over $E$}.

If $R(f,E)$ is measurable (as a subset of $\bfR^{n+1}$), its measure
$|R(f,E)|_{\bfR^{n+1}}$ is called the \emph{Lebesgue integral over $E$},
and we write
\begin{equation}
\label{eq:lebesgue-integral}
\int_E f(\bfx)\diff\bfx\coloneqq|R(f,E)|_{\bfR^{n+1}}.
\end{equation}
This is sometimes written as
\[
\int_E f
\]
or at times the lengthy notation
\[
\idotsint\limits_{E} f(x_1,\dotsc,x_n)\diff x_1\cdots\diff x_n
\]
is convenient.
\begin{theorem*}[5.1]
Let $f$ be a nonnegative function defined on a measurable set $E$. Then
$\int_E f$ exists if and only if $f$ is measurable.
\end{theorem*}
\begin{lemma*}[5.3]
If $f$ is a nonnegative measurable function on $E$, $0\leq |E|\leq\infty$,
then $|\Gamma(f,E)|=0$.
\end{lemma*}
\begin{theorem*}[5.5]
\begin{enumerate}[label=\textnormal{(\roman*)}]
\item If $f$ and $g$ are measurable and if $0\leq g\leq f$ on $E$, $\int_E
  g\leq\int_E f$. In particular, $\int_E\inf f\leq\int_E f$.
\item If $f$ is nonnegative and measurable on $E$ and if $\int_E f$ is
  finite, then $f<\infty$ a.e.\@ in $E$.
\item Let $E_1$ and $E_2$ be measurable and $E_1\subset E_2$. If $f$ is
  nonnegative and measurable on $E_2$, then $\int_{E_1} f\leq\int_{E_2}f$.
\end{enumerate}
\end{theorem*}
\begin{theorem*}[5.6, the monotone convergence theorem for nonnegative functions]
If $\{f_k\}$ is a sequence of nonnegative functions such that $f_k\nearrow
f$ on $E$, then
\[
\int_E f\to\int_E f.
\]
\end{theorem*}
\begin{proof}
By Theorem 4.12, $f$ is measurable since it is the limit of a sequence of
measurable functions. Since $R(f_k,E)\cup\Gamma(f,E)\nearrow R(f,E)$ and
$|\Gamma(f,E)|=0$, the result follows by Theorem 3.26 on the measure of a
monotone convergent sequences of measurable sets.
\end{proof}
\begin{theorem*}[5.9]
Let $f$ be nonnegative on $E$. If $|E|=0$, then $\int_E f=0$.
\end{theorem*}
\begin{theorem*}[5.10]
If $f$ and $g$ are nonnegative and measurable on $E$ and if $g\leq f$
a.e.\@ in $E$, then $\int_E g\leq\int_E f$.

In particular, if $f=g$ a.e.\@ in $E$, then $\int_E f=\int_E g$.
\end{theorem*}
\begin{theorem*}[5.11]
Let $f$ be nonnegative and measurable on $E$. Then $\int_E f=0$ if and only
if $f=0$ a.e.\@ in $E$.
\end{theorem*}
\begin{corollary*}[5.12, Chebyshev's inequality]
Let $f$ be nonnegative and measurable on $E$. If $a>0$, then
\[
\frac{1}{a}\int_E f\geq\left|\left\{\,\bfx\in E:f(\bfx)>a\,\right\}\right|.
\]
\end{corollary*}
\begin{theorem*}[5.13]
If $f$ is nonnegative and measurable, and if $c$ is any nonnegative
constant, then
\[
\int_E cf=c\int_E f.
\]
\end{theorem*}
\begin{theorem*}[5.14]
If $f$ and $g$ are nonnegative and measurable, then
\[
\int_E (f+g)=\int_E f+\int_E g.
\]
\end{theorem*}
\begin{corollary*}
Suppose that $f$ and $\varphi$ are measurable on $E$, $0\leq f\leq\varphi$,
and $\int_E f$ is finite. Then
\[
\int_E (\varphi-f)=\int_E\varphi-\int_E f.
\]
\end{corollary*}
\begin{theorem*}[5.16]
If $f_k$, $k=1,2,\dotsc$, are nonnegative and measurable, then
\[
\int_E\sum_{k=1}^\infty f_k=\sum_{k=1}^\infty\int_E f_k.
\]
\end{theorem*}
\begin{theorem*}[5.17, Fatou's lemma]
If $\{f_k\}$ is a sequence of nonnegative measurable functions on $E$, then
\[
  \int_E\liminf_{k\to\infty} f_k\leq\liminf_{k\to\infty}\int_E f_k.
\]
\end{theorem*}
\begin{proof}[Proof of Fatou's lemma]
\end{proof}
\begin{theorem*}[5.19, Lebesgue's dominated convergence theorem for
  nonnegative functions]
Let $\{f_k\}$ be a sequence of nonnegative measurable functions on $E$ such
that $f_k\to f$ a.e.\@ in $E$. If there exists a measurable function
$\varphi$ such that $f_k\leq\varphi$ a.e.\@ for all $k$ and if
$\int_E\varphi$ is finite, then
\[
\int_E f_k\longrightarrow\int_E f.
\]
\end{theorem*}
\begin{theorem*}[5.21]
Let $f$ be measurable in $E$. Then $f$ is integrable over $E$ if and only
if $|f|$ is.
\end{theorem*}
\begin{theorem*}[5.22]
If $f\in L^1(E)$, then $f$ is finite a.e.\@ in $E$.
\end{theorem*}
\begin{theorem*}[5.24]
If $\int_E f$ exists and $E=\bigcup_{k\in\bfN} E_k$ is the countable union
of disjoint measurable sets $E_k$, then
\[
\int_E f=\sum_{k\in\bfN}\int_{E_k}f.
\]
\end{theorem*}
\begin{theorem*}[5.25]
If $|E|=0$ or if $f=0$ a.e.\@ in $E$, then $\int_E f=0$.
\end{theorem*}
\begin{theorem*}[5.32, monotone convergence theorem]
Let $\{f_k\}$ be a sequence of measurable functions on $E$:
\begin{enumerate}[label=\textnormal{(\roman*)}]
\item If $f_k\nearrow f$ a.e.\@ on $E$ and there exists $\varphi\in L^1(E)$ such
  that $f_k\geq\varphi$ a.e.\@ on $E$ for all $k$, then $\int_E
  f_k\to\int_E f$.
\item If $f_k\searrow f$ a.e.\@ on $E$ and there exists $\varphi\in L^1(E)$ such
  that $f_k\leq\varphi$ a.e.\@ on $E$ for all $k$, then $\int_E
  f_k\to\int_E f$.
\end{enumerate}
\end{theorem*}
\begin{theorem*}[5.33, uniform convergence theorem]
Let $f_k\in L^1(E)$ for $k\in\bfN$ and let $\{f_k\}$ converge uniformly to
$f$ on $E$, $|E|<\infty$. Then $f\in L^1(E)$ and $\int_E f_k\to\int_E f$.
\end{theorem*}
\begin{theorem*}[5.34, Fatou's lemma]
Let $\{f_k\}$ be a sequence of measurable functions on $E$. If there exists
$\varphi\in L^1(E)$ such that $f_k\geq\varphi$ a.e.\@ on $E$ for all $k$,
then
\[
\int_E\liminf_{k\to\infty} f_k\leq\liminf_{k\to\infty}\int_E f_k.
\]
\end{theorem*}
\begin{corollary*}[5.35, reverse Fatou's lemma]
Let $\{f_k\}$ be a sequence of measurable functions on $E$. If there exits
$\varphi\in L^1(E)$ such that $f_k\leq\varphi$ a.e.\@ on $E$ for all $k$,
then
\[
\int_E\limsup_{k\to\infty} f_k\geq\limsup_{k\to\infty}\int_E f_k.
\]
\end{corollary*}
\begin{theorem*}[5.36, Lebesgue's dominated convergenge theorem]
Let $\{f_k\}$ be a sequence of measurable functions on $E$ such that
$f_k\to f$ a.e.\@ in $E$. If there exists $\varphi\in L^1(E)$ such that
$|f_k|\leq\varphi$  such that $|f_k|\leq\varphi$ a.e.\@ in $E$ for all
$k\in\bfN$, then $\int_E f_k\to\int_E f$.
\end{theorem*}
\begin{corollary*}[5.37, bounded convergence theeorem]
Let $\{f_k\}$ be a sequence of measurable functions on $E$ such  that
$f_k\to f$ a.e.\@ in $E$. If $|E|<\infty$ there is a finite constant $M$
such that $|f_k|\leq M$ a.e.\@ in $E$, then $\int_E f_k\to\int_E f$.
\end{corollary*}
\begin{theorem*}[6.1 Fubini's theorem]
Let $f(\bfx,\bfy)\in L^1(I)$, $I\coloneqq I_1\times I_2$. Then
\begin{enumerate}[label=\textnormal{(\roman*)}]
\item For almost every $\bfx\in I_1$, $f(\bfx,\bfy)$ is measurable and
  integrable on $I_2$ as a function of $\bfy$;
\item As a function of $\bfx$, $\int_{I_2} f(\bfx,\bfy)\diff\bfy$ is
  measurable and integrable on $I_1$, and
\[
\iint_I f(\bfx,\bfy)\diff\bfx\diff\bfy=
\int_{I_1}\left[\int_{I_2}f(\bfx,\bfy)\diff\bfy\right]\!\diff\bfx.
\]
\end{enumerate}
\end{theorem*}
\begin{theorem*}[6.8]
Let $f(\bfx,\bfy)$ be a measurable function defined on a measurable subset
$E$ of $\bfR^{n+m}$, and let $E_\bfx\coloneqq\left\{ \,\bfy:(\bfx,\bfy)\in
  E \,\right\}$.
\begin{enumerate}[label=\textnormal{(\roman*)}]
\item For almost every $\bfx\in \bfR^n$, $f(\bfx,\bfy)$ is a measurable
  function of $\bfy$ on $E_\bfx$.
\item If $f(\bfx,\bfy)\in L^1(E)$, then for almost every $\bfx\in\bfR^n$,
  $f(\bfx,\bfy)$ is an integrable on $E_\bfx$ with respect to $\bfy$;
  moreover $\int_{E_\bfx}f(\bfx,\bfy)\diff\bfy$ is an integrable function
  of $\bfx$ and
\[
\iint_E f(\bfx,\bfy)\diff\bfx\diff\bfy=\int_{\bfR^n}\left[\int_{E_\bfx}f(\bfx,\bfy)\diff\bfy\right]\!\diff\bfx.
\]
\end{enumerate}
\end{theorem*}
\begin{theorem*}[6.10, Tonelli's theorem]
Let $f(\bfx,\bfy)$ be nonnegative and measurable on an interval
$I=I_1\times I_2$ of $\bfR^{n+m}$. Then, for almost every $\bfx\in I_1$,
$f(\bfx,\bfy)$ is a measurable function of $\bfy$ on $I_2$. Moreover, as a
function of $\bfx$, $\int_{I_2}f(\bfx,\bfy)\diff \bfy$ is measurable on
$I_1$, and
\[
\iint_I f(\bfx,\bfy)\diff\bfx\diff\bfy=\int_{I_1}\left[\int_{I_2}f(\bfx,\bfy)\diff\bfy\right]\!\diff\bfx
\]
\end{theorem*}
If $f$ and $g$ are measurable in $\bfR^n$, their \emph{convolution
  $(f*g)(\bfx)$} is defined by
\[
(f*g)(\bfx)\coloneqq\int_{\bfR^n}f(\bfx-\bfy)g(\bfy)\diff\bfy,
\]
provided the integral exists.
\begin{theorem*}[6.14]
If $f\in L^1(\bfR^n)$ and $g\in L^1(\bfR^n)$, then $(f*g)(\bfx)$ exists for
almost every $\bfx\in\bfR^n$ and is measurable. Moreover, $f*\in
L^1(\bfR^n)$ and
\[
\begin{aligned}
\int_{\bfR^n}|f*g|\diff\bfx
&\leq\left(\int_{\bfR^n}|f|\diff\bfx\right)\left(\int_{\bfR^n}|g|\diff\bfx\right)\\
\int_{\bfR^n}(f*g)(\bfx)\diff\bfx
&=\left(\int_{\bfR^n}f\diff\bfx\right)\left(\int_{\bfR^n}g\diff\bfx\right).
\end{aligned}
\]
\end{theorem*}
\begin{corollary*}[6.16]
If $f$ and $g$ are nonnegative and measurable on $\bfR^n$, then $f*g$ is
measurable on $\bfR^n$ and
\[
\int_{\bfR^n}(f*g)\diff\bfx=
\left(\int_{\bfR^n} f\diff\bfx\right)
\left(\int_{\bfR^n}g\diff\bfx\right).
\]
\end{corollary*}
\begin{theorem*}[6.17, Marcinkiewicz]
Let $F$ be a closed subset of a bounded open interval $(a,b)$, and let
$\delta(x)\coloneqq\delta(x,F)$ be the corresponding distance
function. Then, given $\lambda>0$, the integral
\[
M_\lambda(x)\coloneqq\int_a^b\frac{\delta(y)^\lambda}{|x-y|^{1+\lambda}}\diff y
\]
is finite a.e.\@ in $F$. Moreover, $M_\lambda\in L^1(F)$ and
\[
\int_F M_\lambda\diff x\leq 2\lambda^{-1}|G|,
\]
where $G\coloneqq(a,b)\minus F$.
\end{theorem*}

%%% Local Variables:
%%% mode: latex
%%% TeX-master: "../MA544-Quals"
%%% End:
