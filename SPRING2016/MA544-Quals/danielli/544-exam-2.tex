\section{MA 544 - Midterm 2}
\begin{problem}
Assume that $f\in L^1(\bfR^n)$. Show that for every $\varepsilon>0$ there
exists a ball $B$, centered at the origin, such that
\[
\int_{\bfR^n\minus B}|f|<\varepsilon.
\]
\end{problem}
\begin{proof}
Recall that $f\in L^1(\bfR^n)$ if and only if $|f|\in
L^1(\bfR^n)$. Let $B_k\coloneqq B(\mathbf{0},k)$ for $k\in\bfN$ and
$\chi_{B_k}$ be the indicator function associated with $B_k$. Then, the
sequence of maps $\left\{|f_k|\right\}$ defined $f_k\coloneqq f\chi_{B_k}$
converge pointwise to $|f_k|$. Since $|f|\in L^1(\bfR^n)$, by the monotone
convergence theorem, we have
\begin{equation}
\label{eq:monotonicity-2-1}
\int_{\bfR^n} |f_k|=\int_{B_k}|f|\longrightarrow\int_{\bfR^n}|f|.
\end{equation}
But this means, exactly, that for every $\varepsilon>0$ there exists
sufficiently large $N\in\bfN$ such that
\begin{equation}
\label{eq:desired-inequality-2-1}
\begin{aligned}
\varepsilon&>\left|\int_{\bfR^n}|f_k|-\int_{\bfR^n}|f|\right|\\
&=-\int_{\bfR^n}|f_k|+\int_{\bfR^n}|f|\\
&=-\int_{\bfR^n}|f|+\int_{\bfR^n}|f|\\
&=-\int_{B_k}|f|+\int_{\bfR^n}|f|\\
&=\int_{\bfR^n\minus B_k}|f|
\end{aligned}
\end{equation}
as desired.
\end{proof}

\begin{problem}
Let $f\in L^1(E)$, and let $\{E_j\}$ be a countable collection of pairwise
disjoint measurable subsets of $E$, such that $E=\bigcup_{j=1}^\infty
E_j$. Prove that
\[
\int_E f=\sum_{j=1}^\infty\int_{E_j}f.
\]
\end{problem}
\begin{proof}
First, since the $E_j$'s are pairwise disjoint, by Theorem 3.23, we have
\begin{equation}
\label{eq:disjoint-measure-2-2}
|E|=\sum_{j=1}^\infty|E_j|.
\end{equation}
Let $\chi_{E_j}$ be the characteristic function of the subset $E_j$ of
$E$ and define $f_j\coloneqq f\chi_{E_j}$ for $j\in\bfN$. Note that, since
both $f$ and $\chi_{E_j}$ are measurable on $E$, $f_j$ is
measurable on $E$ and $\sum_{j=1}^\infty f_j=f$. Moreover, since
$E_j\subset E$, by monotonicity of the integral we have
\begin{equation}
\label{eq:monotonicity-2-2}
\int_{E} f=
\int_{E_j} f+\int_{E\minus E_j}f=
\int_E f_j+\int_{E\minus E_j}f.
\end{equation}
Hence, because the $E_j$'s are disjoint $(E\minus E_k)\minus
E_\ell=(E\minus E_\ell)\minus E_k$ so
\begin{equation}
\label{eq:desired-sum-2}
\int_E f=\sum_{j=1}^\infty\int_E f_j=\sum_{j=1}^\infty\int_{E_j}f
\end{equation}
as desired.
\end{proof}

\begin{problem}
Let $\{f_k\}$ be a family in $L^1(E)$ satisfying the following property:
For any $\varepsilon>0$ there exits $\delta>0$ such that $|A|<\delta$
implies
\[
\int_A |f_k|<\varepsilon
\]
for all $k\in\bfN$. Assume $|E|<\infty$, and $f_k(x)\to f(x)$ as
$k\to\infty$ for a.e.\@ $x\in E$. Show that
\[
\lim_{k\to\infty}\int_E f_k=\int_E f.
\]
(\emph{Hint:} Use Egorov's theorem.)
\end{problem}
\begin{proof}
Let $\varepsilon>0$ be given. Then, by the hypothesis, there exists
$\delta>0$ such that
such that $|A|<\delta$
implies
\begin{equation}
  \label{eq:hypothesis-2-3}
\int_A |f_k|<\varepsilon
\end{equation}
for all $k\in\bfN$. By Egorov's theorem, there exists a closed subset $F$
of $E$ such that $|E\minus F|<\delta$ and $f_k\to f$ uniformly on
$F$. Then, by the uniform convergence theorem,
\begin{equation}
\label{eq:uniform-convergence-2-3}
\int_F f_k\to \int_F f
\end{equation}
as $k\to\infty$. But by hypothesis, we have
\begin{equation}
\label{eq:need-to-show-2-}
\int_{E\minus F} |f_k|<\varepsilon.
\end{equation}
Letting $\varepsilon\to 0$, we achieved the desired convergence.
\end{proof}

\begin{problem}
Let $I\coloneqq[0,1]$, $f\in L^1(I)$, and define $g(x)\coloneqq\int_x^1
t^{-1}f(t)\diff t$ for $x\in I$. Prove that $g\in L^1(I)$ and
\[
\int_I g=\int_I f.
\]
\end{problem}
\begin{proof}
By Lusin's theorem, there exists a closed subset $F$ of $I$ with $|I\minus
F|<\varepsilon$ such that the restriction of $f$ to $F\coloneqq I\minus E$
is continuous. Now, since $F$ is closed in $I$ and $I$ is compact, it
follows that $I$ is compact. Hence, by the Stone--Weierstraß approximation
theorem, there exist a sequence of polynomials $\left\{ p_k \right\}$ such
that $p_k\to f$ uniformly on $F$. Then, by the uniform convergence theorem,
we have
\begin{equation}
  \label{eq:uniform-convergence-2-4}
\int_F p_k\to \int_F f
\end{equation}
so
\begin{equation}
  \label{eq:uniform-convergence-2-2-4}
\begin{aligned}
\int_F\left[\int_x^1t^{-1}p_k(t)\diff t\right]\!\diff x
&=\int_F\left[\int_x^1 at^{-1}+q_k(t)\diff t\right]\!\diff x\\
&=\int_F q_k'(x)-a\log(x)\diff x\\
&<\infty
\end{aligned}
\end{equation}
for all $k$ and converges uniformly to $g$ so $g\in L^1(I)$. I don't know
how to show that in fact $\int_I g=\int_I f$. Perhaps you show that the
places where they differ is a set of measure zero.
\end{proof}

%%% Local Variables:
%%% mode: latex
%%% TeX-master: "../MA544-Quals"
%%% End:
