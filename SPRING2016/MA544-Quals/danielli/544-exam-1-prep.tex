\chapter{MA 544 Spring 2016}
\section{Exam 1 Prep}
\begin{problem}
Let $E\subset\bfR^n$ be a measurable set, $r\in\bfR$ and define the set
$rE=\left\{\,r\bfx:\bfx\in E\,\right\}$. Prove that $rE$ is
measurable, and that $|rE|=|r|^n|E|$.
\end{problem}
\begin{proof}
Define a linear map $T\colon\bfR^n\to\bfR^n$ by $\bfx\mapsto r\bfx$. Using
the standard basis for $\bfR^n$, this map has the matrix presentation
\begin{equation}
\label{eq:matrix-presentation}
T\bfx=
\begin{bmatrix}
r&\\
&\ddots\\
&&r
\end{bmatrix}\bfx
\end{equation}
which has determinant $\det T=r^n$. By 3.35, we have
$|E|=|T(E)|=r^n|E|=|rE|$.
\end{proof}

\begin{problem}
Let $\left\{ E_k \right\}$, $k\in\bfN$ be a collection of measurable
sets. Define the set
\[
\liminf_{k\to\infty} E_k
=\bigcup_{k=1}^\infty\left(\bigcap_{n=k}^\infty E_n\right).
\]
Show that
\[
\left|\liminf_{k\to\infty} E_k\right|\leq\liminf_{k\to\infty}\left|E_k\right|.
\]
\end{problem}
\begin{proof}
If the $\liminf\left|E_k\right|=\infty$ the inequality holds
trivially. Hence, we may, without loss of generality, assume that
$\liminf\left|E_k\right|<\infty$. By 3.20, the set $\liminf E_k$ is
measurable and we have
\begin{equation}
  \label{eq:limsup-rewrite}
\left|\liminf_{k\to\infty} E_k\right|
=\left|\bigcup_{k=1}^\infty F_k\right|,
\end{equation}
where $F_k=\bigcap_{n=k}^\infty E_n$. Now, note that the collection
of sets $F_k'=\bigcup_{\ell=1}^k F_\ell$ forms an increasing
sequence of measurable sets $F_k'\nearrow F'$, where
$F'=\bigcup_{k=1}^\infty F_k=\liminf E_k$. Then, by 3.26 (i), we have
\begin{equation}
  \label{eq:monotone-increasing-limit}
\lim_{k\to\infty}\left|F_k'\right|
=\left|F'\right|
=\left|\liminf_{k\to\infty} E_k\right|.
\end{equation}
Hence, it suffices to show that $\left|F_k'\right|\leq\left|E_k\right|$ for
all $k$, but this follows by monotonicity of the outer measure, 3.3, since
$F_k'\subset E_k$. Thus, we have the desired inequality
\begin{equation}
  \label{eq:meas-liminf-liminf-meas}
\left|\liminf_{k\to\infty} E_k\right|
\leq\liminf_{k\to\infty}\left|E_k\right|.
\end{equation}
\end{proof}

\begin{problem}
Consider the function
\[
F(x)=
\begin{cases}
|B(\mathbf{0},x)|&x>0\\
0&x=0
\end{cases}.
\]
Here
$B(\mathbf{0},r)=\left\{\,\bfy\in\bfR^n:|\bfy|<r\,\right\}$. Prove
that $F$ is monotonic increasing and continuous.
\end{problem}
\begin{proof}
That $F$ is increasing is immediate from the monotonicity of the outer
measure since for $x<x'$ we have $B(\mathbf{0},x)\subset B(\mathbf{0},x')$
so, by 3.2, we have
\[
F(x)\left|B(\mathbf{0},x)\right|\leq\left|B(\mathbf{0},x')\right|=F(x')
\]
as desired.

To see that $F$ is continuous, we will prove the following lemma
\begin{lemma}
\label{lem:one}
For any $x>0$, $xB(\mathbf{0},1)=B(\mathbf{0},x)$.
\end{lemma}
\begin{proof}[Proof of lemma]
\renewcommand\qedsymbol{$\clubsuit$}
If $\bfy\in xB(\mathbf{0},1)$ then $\bfy=x\bfy'$ for $\bfy'\in
B(\mathbf{0},1)$. Thus, $|\bfy'|=|\bfy|/x<1$ so $|\bfy|<x$ implies that
$\bfy\in B(\mathbf{0},x)$. Hence, we have the containment
$xB(\mathbf{0},1)\subset B(\mathbf{0},x)$.

On the other hand, if $\bfy\in B(\mathbf{0},x)$ then $|\bfy|<x$ so
$\left|\bfy/x\right|<1$. Hence, $\bfy/x\in B(\mathbf{0},1)$ so
$x(\bfy/x)=\bfy\in B(\mathbf{0},x)$. Thus, $B(\mathbf{0},x)\subset
xB(\mathbf{0},)$ and equality holds.
\end{proof}
In light of Lemma \ref{lem:one} and 3.35, for $x>0$, we have
\begin{equation}
\label{eq:scale-map}
F(x)=\left|B(\mathbf{0},x)\right|
=\left|xB(\mathbf{0},1)\right|
=x^n\left|B(\mathbf{0},1)\right|.
\end{equation}
It is clear that $F$ is continuous on the interval $[0,\infty)$ since $F$
is a polynomial in $x$.
\end{proof}

\begin{problem}
Let $f\colon\bfR\to\bfR$ be a function. Let $C$ be the set of all points
at which $f$ is continuous. Show that $C$ is a set of type $G_\delta$.
\end{problem}
\begin{proof}
From the topological definition of continuity, $f$ is continuous at $x\in
C$ if and only if for every neighborhood $U$ of $f(x)$, the preimage
$f^{-1}(U)$ is a neighborhood of $x$. Now,
\end{proof}
Let $x\in C$. Then, by the definition of continuity, for every natural
number $n>0$ there exists $\delta>0$ such that $|x-x'|<\delta$ implies
\begin{equation}
\label{eq:continuity-2n}
\left|f(x)-f(x')\right|<\frac{1}{2n}.
\end{equation}
Let $x'',x'\in B(x,\delta)$. Then, by the triangle inequality, we have
\begin{equation}
\label{eq:n-estimates}
\begin{aligned}
|f(x')-f(x)''|={}&\left|f(x')-f(x)-(f(x'')-f(x))\right|\\
              \leq{}&\left|f(x')-f(x)\right|+\left|f(x'')-f(x)\right|\\
              <{}&\frac{1}{2n}+\frac{1}{2n}\\
              ={}&\frac{1}{n}.
\end{aligned}
\end{equation}
In view of these estimates, define the set
\begin{equation}
\label{eq:a-n}
A_n=
\left\{\,
x\in\bfR :\text{there exists $\delta>0$ such that $x',x''\in B(x,\delta)$ implies $\left|f(x')-f(x'')\right|<\frac{1}{n}$ }
\,\right\}.
\end{equation}
Good Lord, that was a long definition! We claim that
$C=\bigcap_{n=1}^\infty A_n$ and that $A_n$ is open for all $n$.

First, let us show that $C=\bigcap_{n=1}^\infty A_n$. Let $x\in C$. Then for
every $n>0$, there exists $\delta>0$ such that $|x-x'|<\delta$ implies
$|f(x)-f(x')|<1/n$. Thus, $x\in A_n$ for all $n$ so $x\in\bigcap A_n$. On
the other hand, if $x\in\bigcap A_n$ for every $n>0$, there exists
$\delta>0$ such that $|x-x'|<\delta$ implies $|f(x)-f(x')|<1/n$. Fix
$\varepsilon>0$. By the Archimedean principle, there exists $N>0$ such that
$\varepsilon>1/N$. Then, since $x\in A_N$ it follows that for some
$\delta'>0$, $|x-x'|<\delta'$ implies
$\left|f(x)-f(x')\right|<1/N<\varepsilon$. Thus, $x\in C$ and we conclude
that $C=\bigcap_{n=1}^\infty A_n$.

Lastly, we show that $A_n$ is open. Let $x\in A_n$. Then there exists
$\delta>0$ such that $|x-x'|<\delta$ implies $|f(x)-f(x')|<1/n$. In
particular, this means that $B(x,\delta)\subset A_n$ for any $x'\in
B(x,\delta)$ satisfies $|f(x)-f(x')|<1/n$. Thus, $A_n$ is open and we
conclude that $C=\bigcap_{n=1}^\infty A_n$ is a $G_\delta$ set.
\begin{problem}
Let $f\colon\bfR\to\bfR$ be a function. Is it true that if the sets
$\left\{\,f=r\,\right\}$ are measurable for all $r\in\bfR$, then $f$ is
measurable?
\end{problem}
\begin{proof}
No. Recall that, by definition, or 4.1, $f$ is measurable if and only if
$\left\{\,f>a\,\right\}$ for all $a\in\bfR$.
\end{proof}

\begin{problem}
Let $\left\{f_k\right\}_{k=1}^\infty$ be a sequence of measurable functions
on $\bfR$. Prove that the set
$\left\{\,x:\text{$\lim_{k\to\infty} f_k(x)$ exists}\,\right\}$
is measurable.
\end{problem}
\begin{proof}
The idea here should be to rewrite
\begin{equation}
  \label{eq:measurable-lim-set}
E=\left\{\,x:\text{$\lim_{k\to\infty} f_k(x)$ exists}\,\right\}
\end{equation}
as a countable union/intersection of measurable sets. Let $x\in E$. By the
Cauchy criterion, for every $N>0$ there exists a positive integer
$M$ such that $m,n\geq M$ implies
$\left|f_n(x)-f_m(x)\right|<1/N$. With this in mind, define
\begin{equation}
  \label{eq:countable-lim-set}
E_N=
\left\{\,
x:\text{there exists $M$ such that $m,n\geq M$ implies $\left|f_n(x)-f_m(x)\right|<\frac{1}{N}$}
\,\right\}.
\end{equation}
Then, like for Problem 1.4, it is not too hard to see that the $E_n$'s are
open and that $E=\bigcap_{n=1}^\infty E_n$. Thus, $E$ is a $G_\delta$ set
and therefore measurable.
\end{proof}

\begin{problem}
A real valued function $f$ on an interval $[a,b]$ is said to be
\emph{absolutely continuous} if for every $\varepsilon>0$, there exists a
$\delta>0$ such that for every finite disjoint collection
$\left\{(a_k,b_k)\right\}_{k=1}^N$ of open intervals in $(a,b)$ satisfying
$\sum_{k=1}^Nb_k-a_k<\delta$, one has
$\sum_{k=1}^N\left|f(b_k)-f(a_k)\right|<\varepsilon$. Show that an
absolutely continuous function on $[a,b]$ is of bounded variation on
$[a,b]$.
\end{problem}
\begin{proof}
Suppose $f\colon[a,b]\to\bfR$ is absolutely continuous. Then for fixed
$\varepsilon=1$, there exists a $\delta>0$ such that for every finite
disjoint collection $\left\{ (a_kb_k) \right\}_{k=1}^N$ of open intervals
in $(a,b)$ satisfying $\sum_{k=1}^Nb_k-a_k<\delta$, we have
$\sum_{k=1}^N\left|f(b_k)-f(a_k)\right|<\varepsilon$. Let
$\Gamma=\left\{x_k\right\}_{k=1}^N$ be a partition of $[a,b]$ into closed
intervals such that $x_{k+1}-x_k<\delta$, then by absolute continuity we have
\begin{equation}
\label{eq:absolute-continuity-variation}
\begin{aligned}
V[f;\Gamma]
={}&\sum_{k=1}^N\left|f(x_{k+1})-f(x_k)\right|\\
<{}&1.
\end{aligned}
\end{equation}
Thus, $f$ is b.v.\@ on $[a,b]$.
\end{proof}

\begin{problem}
Let $f$ be a continuous function from $[a,b]$ into $\bfR$. Let
$\chi_{\{c\}}$ be the characteristic function of a singleton
$\left\{c\right\}$, i.e., $\chi_{\{c\}}(x)=0$ if $x\neq c$ and
$\chi_{\{c\}}(c)=1$. Show that
\[
\int_a^b f d \chi_{\{c\}}=
\begin{cases}
0&\text{if $c\in(a,b)$}\\
-f(a)&\text{if $c=a$}\\
f(a)&\text{if $c=b$}
\end{cases}.
\]
\end{problem}
\begin{proof}
\end{proof}

%%% Local Variables:
%%% mode: latex
%%% TeX-master: "../MA544-Quals"
%%% End:
