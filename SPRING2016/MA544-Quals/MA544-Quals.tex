\def\documentauthor{Carlos Salinas}
\def\documenttitle{MA544: Qual Problems}
% \def\hwnum{1}
\def\shorttitle{MA544 Quals}
\def\coursename{MA544}
\def\documentsubject{measure theory}
\def\authoremail{salinac@purdue.edu}

\documentclass[article,oneside,10pt]{memoir}
\usepackage{geometry}
\usepackage[dvipsnames]{xcolor}
\usepackage[
    breaklinks,
    bookmarks=true,
    colorlinks=true,
    pageanchor=false,
    linkcolor=black,
    anchorcolor=black,
    citecolor=black,
    filecolor=black,
    menucolor=black,
    runcolor=black,
    urlcolor=black,
    % linkcolor=violet!85!black,
    % citecolor=YellowOrange!85!black,
    % urlcolor=Aquamarine!85!black,
    hyperindex=false,
    hyperfootnotes=true,
    pdftitle={\shorttitle},
    pdfauthor={\documentauthor},
    pdfkeywords={\documentsubject},
    pdfsubject={\coursename}
    ]{hyperref}
\usepackage[numbers]{natbib}

%% Math
\usepackage{amsmath}
\usepackage{amsfonts}
\usepackage{amssymb}
\usepackage{amsthm}
\usepackage{mathtools}
\usepackage{eucal}
\usepackage{mathrsfs}
\usepackage[nointegrals]{wasysym}

%% Language
\usepackage{cmap}
\usepackage{selnolig}
\usepackage[LAE,LFE,T2A,T1]{fontenc}
\usepackage[utf8]{inputenc}
\usepackage[farsi,french,german,spanish,russian,english]{babel}
\babeltags{fr=french,
           de=german,
           en=english,
           es=spanish,
           pa=farsi,
           ru=russian
           }
\def\spanishoptions{mexico}

\selectlanguage{english}

\newcommand{\textfa}[1]{\beginR\textpa{#1}\endR}

\usepackage{CJKutf8}
\newcommand{\textkr}[1]{\begin{CJK}{UTF8}{mj}#1\end{CJK}}
\newcommand{\textjp}[1]{\begin{CJK}{UTF8}{min}#1\end{CJK}}
\newcommand{\textzh}[1]{\begin{CJK}{UTF8}{bsmi}#1\end{CJK}}

%% Misc
\usepackage{graphicx}
\graphicspath{{figures/}}

\usepackage{microtype}
\usepackage{lineno}
\usepackage{multicol}
\usepackage[inline]{enumitem}
\usepackage{listings}
\usepackage{mleftright}
\mleftright
\usepackage{carlos-variables}

% %% Unicode math and Polyglossia
% \usepackage{unicode-math}
% \usepackage{unicode-minionmath}

% \setmainfont[Ligatures=TeX]{Libertinus Serif}
% \setsansfont{Libertinus Sans}
% \setmonofont{Libertinus Mono}
% \setmathfont{Minion Math}
% \setmathfont[range={\mathfrak}]{XITS Math}
% \setmathfont[range={\mathcal},StylisticSet=1]{XITS Math}
% \setmathfont[range={\mathscr}]{XITS Math}
% \setmathfont[range={}]{Minion Math}

% \usepackage{polyglossia}

% \newfontfamily\cyrillicfont[Script=Cyrillic]{Libertinus Serif}
% \newfontfamily\cyrillicfontsf[Script=Cyrillic]{Libertinus Sans}

% \newfontfamily\farsifont[Script=Arabic,
%                          Scale=MatchUppercase]{Amiri}

% \setmainlanguage[variant=american]{english}
% \setotherlanguage{farsi}
% \setotherlanguage{french}
% \setotherlanguage[spelling=new,latesthyphen,babelshorthands]{german}
% \setotherlanguage{spanish}
% \setotherlanguage[spelling=modern,babelshorthands]{russian}

% \makeatletter
% \@Latintrue
% \makeatother

% \usepackage{xeCJK}
% \usepackage[overlap]{ruby}
% \renewcommand\rubysep{-0.2ex}
% \xeCJKDeclareSubCJKBlock{Kana}{"3040 -> "309F, "30A0 -> "30FF, "31F0 -> "31FF, "1B000 -> "1B0FF}
% \xeCJKDeclareSubCJKBlock{Hangul}{"1100 -> "11FF, "3130 -> "318F, "A960 -> "A97F, "AC00 -> "D7AF, "D7B0 -> "D7FF}

% \setCJKmainfont{HanaMinA}
% \setCJKmainfont[Kana]{HanaMinA}
% \setCJKmainfont[Hangul]{NanumMyeongjo}
% \setCJKsansfont[Hangul]{NanumGothic}

%% Theorems and definitions
%% remove parentheses
% \makeatletter
% \def\thmhead@plain#1#2#3{%
%   \thmname{#1}\thmnumber{\@ifnotempty{#1}{ }\@upn{#2}}%
%   \thmnote{ {\the\thm@notefont#3}}}
% \let\thmhead\thmhead@plain
% \makeatother

\theoremstyle{plain}
\newtheorem{theorem}{Theorem}
\newtheorem{proposition}[theorem]{Proposition}
\newtheorem{corollary}[theorem]{Corollary}
\newtheorem{claim}[theorem]{Claim}
\newtheorem{lemma}[theorem]{Lemma}
\newtheorem{axiom}[theorem]{Axiom}

\newtheorem*{corollary*}{Corollary}
\newtheorem*{claim*}{Claim}
\newtheorem*{lemma*}{Lemma}
\newtheorem*{proposition*}{Proposition}
\newtheorem*{theorem*}{Theorem}

\theoremstyle{definition}
\newtheorem{definition}{Definition}
\newtheorem{example}{Examples}
\newtheorem{examples}[example]{Example}
\newtheorem{exercise}{Exercise}[chapter]
\newtheorem{problem}[exercise]{Problem}

\newtheorem*{example*}{Example}
\newtheorem*{exercise*}{Exercise}
\newtheorem*{problem*}{Problem}

\begin{document}
\chapterstyle{veelo}
\pagestyle{ruled}
\author{\href{mailto:\authoremail}{\documentauthor}}
\title{\documenttitle}
\date{\today}
\maketitle
\tableofcontents

%% Footnotes
\renewcommand{\thefootnote}{\fnsymbol{footnote}}

\chapter{Notes}
Notes based off of Wheeden and Zygmund's \emph{Measure and Integral} book.
\section{Exam 1 Review}
This is all of the material we covered before exam 1.

\bigskip

%%% Local Variables:
%%% mode: latex
%%% TeX-master: "../MA544-Quals"
%%% End:

\section{The Lebesgue integral}
This portion corresponds to material covered before the second exam.

%%% Local Variables:
%%% mode: latex
%%% TeX-master: "../MA544-Quals"
%%% End:

\section{Differentiation}
This portion of the notes corresponds to material covered before the final.

This section deals with questions of differentiability and culminates with
a couple of results tying together the Lebesgue integral with the
derivative à la the familiar fundamental theorem of calculus for Riemann
integrals.

\subsection{The indefinite integral}
If $f$ is a Riemann integrable function on an interval $[a,b]$ of $\bbR$,
then the familiar definition for its
\href{https://en.wikipedia.org/wiki/Antiderivative}{\emph{indefinite
    integral}} is
\begin{equation}
\label{eq:3:indefinite-integral}
\begin{aligned}
F(x)&=\int_a^xf(y)dy,&a\leq x\leq b.
\end{aligned}
\end{equation}
The
\href{https://en.wikipedia.org/wiki/Fundamental_theorem_of_calculus}{\emph{fundamental
    theorem of calculus}} then asserts that $F'=f$ if $f$ is continuous. In
this section, we study the analogue of this result for Lebesgue integrable
functions.

Since we want to generalize our results to $\bbR^n$, first we must find a
suitable notion of indefinite integral for multivariable functions. In two
dimensions we might, for instance, define the indefinite integral $F$ of
$f$ to be
\begin{equation}
\label{eq:3:2-dim-indefinite-integral}
F(x_1,x_2)\coloneqq \int_{a_1}^{x_1}\int_{a_2}^{x_2}f(y_1,y_2)\diff
y_2\diff y_1.
\end{equation}

As it turns out, it is better to abandon the notion that the indefinite
integral be a function of a point an instead let it be a function of
a set. Therefore, given a function $f$, integrable on some measurable
subset $A$ of $\bbR^n$, we define the \emph{indefinite integral of $f$} to
be the function
\begin{equation}
  \label{eq:3:lebesgue-indefinite-integral}
F(E)\coloneqq \int_E f,
\end{equation}
where $E$ is a measurable subset of $A$.

The function $F$ is an example of a
\href{https://en.wikipedia.org/wiki/Set_function}{\emph{set function}}, by
which we mean any real-valued function $F$ defined on a $\sigma$-algebra
$\Sigma$ of measurable sets such that
\begin{enumerate}[label=(\roman*),noitemsep]
\item $F(E)$ is finite for every $E\in\Sigma$.
\item F is
  \href{https://en.wikipedia.org/wiki/Measure_(mathematics)#Properties}{\emph{countably
      additive}}; i.e., if $E$ is the union of disjoint sets
  $E_k\in\Sigma$, $k=1,2,\dotsc$, then
  \begin{equation}
    \label{eq:3:countably-additive-set-function}
    F(E)\sum_{k\in\bbN} F(E_k).
  \end{equation}
\end{enumerate}

\section{$L^p$ Classes}
Let's take a small detour to ch.\@ 5 of
\cite{wheeden-zygmund:measure-and-integral} to talk about $L^p$ spaces.
\subsection{The relation between the Riemann--Stieltjes integral and the
  Lebesgue integral, and the $L^p$ spaces, $0<p<\infty$}
As it turns out, there is a remarkably simple and useful representation of
the Lebesgue integral (over measurable subsets of $\bbR^n$) in terms of the
Riemann--Stiltjes integrals (over measurable subset of $\bbR$). In order to
establish this relationship, we will need to study the function
\begin{equation}
  \label{eq:eq:3:dist-function}
\omega(\alpha)\coloneqq\omega_{f,E}(\alpha)\coloneqq
\left|\left\{\,\bfx\in E:f(\bfx)>\alpha\,\right\}\right|,
\end{equation}
where $f$ is a measurable function on $E$ and $-\infty<\alpha<\infty$. We
call $\omega_{f,E}$ (or simply $\omega$) the
\href{https://en.wikipedia.org/wiki/Probability_density_function#Formal_definition}{\emph{distribution
    function}} \emph{of $f$ on $E$}.

The function $\omega$ is clearly not affected by changing $f$ in a set of
measure zero, and is decreasing. As $\alpha\nearrow\infty$, we have
\[
\left\{\,\bfx\in E:f(\bfx)>\alpha\,\right\}\searrow
\left\{\,\bfx\in E:f(\bfx)=\infty\,\right\}.
\]
hence, assuming that $f$ is finite a.e.\@ in $E$, by Theorem 3.62(ii),
$\lim_{\alpha\to\infty}\omega=0$, unless
$\omega(\alpha)\equiv\infty$. Similarly, we have
$\lim_{\alpha\to-\infty}\omega=|E|$. For now, let us assume that the
measure of $E$ is finite; this will ensure that $\omega$ is bounded.

In the following results, we assume that $f$ is a measurable function that
is finite a.e.\@ in $E$, $|E|<\infty$, and write
\[
\begin{aligned}
\omega(\alpha)&=\omega_{f,E}(\alpha),&\left\{\,f>a\,\right\}&=\left\{\,\bfx\in
  E:f(\bfx)>\alpha\,\right\},
\end{aligned}
\]
etc.
\begin{lemma}[5.38]
If $\alpha<\beta$, then $\left|\left\{\,\alpha\leq
    f\leq\beta\,\right\}\right|=\omega(\alpha)-\omega(\beta)$.
\end{lemma}
\begin{proof}
For $\alpha<\beta$, we have
$\left\{\,f>\beta\,\right\}\subset\left\{\,f>\alpha\,\right\}$ and
$\left\{\,\gamma<f\leq\beta\,\right\}=\left\{
  \,f>\alpha\,\right\}\setminus\left\{ \,f>\beta\,\right\}$. Since
$\left|\left\{\,f>\beta\,\right\}\right|<\infty$, the lemma follows from
Corollary 3.25.
\end{proof}

Given $\alpha$, let
\[
  \begin{aligned}
    \omega(\alpha{+})&\coloneqq\lim_{\varepsilon\searrow
      0}\omega(\alpha+\varepsilon)\quad&
    \omega(\alpha{-})&\coloneqq\lim_{\varepsilon\searrow 0}\omega(\alpha-\varepsilon).
  \end{aligned}
\]
denote the limits of $\omega$ from the right and left at $\alpha$.

\begin{lemma}[5.39]
\hfill
\begin{enumerate}[label=\textnormal{(\alph*)},noitemsep]
\item $\omega(\alpha+)=\omega(\alpha)$; i.e., $\omega$ is continuous from
  the right.
\item $\omega(\alpha-)=\left|\left\{\,f\geq\alpha\,\right\}\right|$.
\end{enumerate}
\end{lemma}

\begin{corollary}[5.40]
\hfill
\begin{enumerate}[label=\textnormal{(\alph*)},noitemsep]
\item
  $\omega(\alpha-)-\omega(\alpha)=\left|\left\{\,f=\alpha\,\right\}\right|$;
  in particular, $\omega$ is continuous at $\alpha$ if and only if
  $\left|\{\,f=\alpha\,\}\right|=0$.
\item $\omega$ is constant in an open interval $(\alpha,\beta)$ if and only
  if $\left|\{\,\alpha<f<\beta\,\}\right|=0$, that is, if and only if $f$
  takes almost no values between $\alpha$ and $\beta$.
\end{enumerate}
\end{corollary}

The rest of this section establishes the relations between the Lebesgue and
Riemann--Stieltjes integrals. As always, we assume $f$ is measurable and
finite a.e.\@ in $E$, $|E|<\infty$ and $\omega=\omega_{E,f}$.

\begin{theorem}[5.41]
If $a\leq f(\bfx)\leq b$ ($a$ and $b$ are finite) for all $\bfx\in E$, then
\[
\int_E f=-\int_a^b\alpha\diff\omega(\alpha).
\]
\end{theorem}
\begin{proof}
The Lebesgue integral on the left-hand side exists since $f$ is bounded and
$|E|<\infty$. The Riemann--Stieltjes integral on the right-hand side exists
by Theorem 2.24. To show that they are equal, let us partition the interval
the interval $[a,b]$ by $a=\alpha_0<\alpha_1<\dotsb<\alpha_k=b$ and let
$E_j=\left\{,\alpha_{j-1}<f\leq\alpha_j\,\right\}$. The $E_j$ are disjoint
and $E=\bigcup_{j=1}^k E_j$. Hence, $\int_E f=\sum_{j=1}^k\int_{E_j}f$ and,
therefore
\[
\sum_{j=1}^\infty \alpha_{j-1}|E_j|\leq\int_E f\leq\sum_{j=1}^k\alpha_j|E_j|.
\]
By Lemma 5.38,
$|E_j|=\omega(\alpha_j)\omega(\alpha_j)=-[\omega(\alpha_j)-\omega(\alpha_{j-1})]$. Hence,
the sums are Riemann--Stieltjes sums for
$-\int_a^b\alpha\diff\omega(\alpha)$. Since the sums must converge to
$-\int_a^b\alpha\diff\omega(\alpha)$ as the norm of the partition tends to
zero, the conclusion follows.
\end{proof}

We can extend the conclusion of Theorem 5.41 to the case when $f$ is not
bounded as follows.
\begin{theorem}[5.42]
Let $f$ be any measurable function on $E$, and let
$E_{ab}\coloneqq\left\{\,\bfx\in E:a<f(\bfx)<b\,\right\}$ ($a$ and $b$
finite). Then,
\[
\int_{E_{ab}}f=-\int_a^b\alpha\diff\omega(\alpha).
\]
\end{theorem}
\begin{proof}[Sketch of proof]
Take $\omega_{ab}(\alpha)\coloneqq\left|\left\{\,\bfx\in
    E_{ab}:f(\bfx)>\alpha\,\right\}\right|$. By Theorem 5.41, we have
\[
\int_{E_{ab}}f=-\int_a^b\alpha\diff\omega_{ab}(\alpha).
\]
Taking the limit of Riemann--Stieltjes sums that approximate the integrals,
it suffices to show that
$\omega_{ab}(\alpha)-\omega_{ab}(\beta)=\omega(\alpha)-\omega(\beta)$. Then
The expression on the right-hand side of the equation above, is seen to be
$\int_a^b\alpha\diff\omega(\alpha)$.
\end{proof}

\begin{theorem}[5.43]
If either $\int_E f$ or $\int_{-\infty}^\infty\alpha\diff\omega(\alpha)$
exist and is finite, then the other exists and is finite, and
\[
\int_E f=-\int_{-\infty}^\infty\alpha\diff\omega(\alpha).
\]
\end{theorem}

Two measurable functions $f$ and $g$ are said to be
\href{https://en.wikipedia.org/wiki/Equidistributed_sequence#Sequences_equidistributed_with_respect_to_an_arbitrary_measure}{\emph{equimeasurable}},
or \emph{equidistributed}, if
\[
\omega_{f,E}(\alpha)=\omega_{g,E}(\alpha)
\]
for all $\alpha$.

We may intuitively think of equimeasurable functions as being
\emph{rearrangements} of each other. For such functions, we have
\[
\begin{aligned}
\left|\{\,a<f\leq b\,\}\right|&=
\left|\{\,a<g\leq b\,\}\right|&
\left|\{\,f=a\,\}\right|&=
\left|\{\,g=a\,\}\right|,
\end{aligned}
\]
etc. We also gave the following immediate corollary of Theorem 5.43.

\begin{corollary}[5.44]
If $f$ and $g$ are equimeasurable on $E$ and $f\in L(E)$, then $g\in
L(E)$ and
\[
\int_E f=\int_E g.
\]
\end{corollary}

The method used to derive Theorem 5.41 through 5.43 illustrates a basic
difference between the Lebesgue and the Riemann integral. The Riemann
integral is defined by a limiting process whose initial step involves
partitioning the domain of $f$. On the other hand, the Lebesgue integral
can be obtained from a process that partitions the \emph{range} of $f$. In
order to define the process more clearly, let $f$ be a nonnegative
measurable function that is finite a.e.\@ in $E$, $|E|<\infty$. Let
$\Gamma=\{\,0=\alpha_0<\alpha_1<\cdots\,\}$ be a partition of the positive
ordinate axis by a countable number of points $\alpha_k\to\infty$, and let
$|\Gamma|=\sup_k(\alpha_{k+1}-\alpha_k)$. Set
$E_k\coloneqq\left\{\,\alpha_k\leq f<\alpha_{k+1}\,\right\}$ and
$Z\coloneqq\left\{\,f=\infty\,\right\}$. Then the $E_k$ are measurable and
disjoint, $|Z|=0$ and $E=\left(\bigcup E_k\right)\cup Z$, so that
$|E|=\sum_k|E_k|$. Let
\[
  \begin{aligned}
    S_\Gamma&\coloneqq\sum_{k\in\bbN}\alpha_k|E_k|,&
    S_\Gamma&\coloneqq\sum_{k\in\bbN}\alpha_{k+1}|E_k|.
  \end{aligned}
\]

\section{$L^p$ Classes}
Let's talk about $L^p$ classes now and some important results about
$L^p$ spaces.
\subsection{Definition of $L^p$}
If $E$ is a measurable subset of $\bbR^n$ and satisfies $0<p<\infty$, then
\href{https://en.wikipedia.org/wiki/Lp_space}{$L^p(E)$} denotes the
collection of measurable $f$ for which $\int_E|f|^p$ is finite, i.e.,
\begin{equation}
  \label{eq:3:lp-space}
L^p(E)\coloneqq\left\{\,f:\int_E |f|^p<\infty\,\right\}
\end{equation}
for $0<p<\infty$. Here, $f$ may be complex-valued, in which case, if
$f=f_1+if_2$ for measurable real-valued $f_1$ and $f_2$, we have
$|f|^2={f_1}^2+{f_2}^2$, so that
\[
|f_1|,|f_2|\leq|f|\leq |f_1|+|f_2|.
\]
It follows that $f\in L^p(E)$ if and only if both $f_1,f_2\in L^p(E)$.

We shall write
\[
\|f\|_{p,E}\coloneqq\left(\int_E|f|^p\right)^{1/p},
\]
for $0<p<\infty$. Thus, $L^p(E)$ is the set of measurable $f$ for which
$\|f\|_{p,E}$ is finite. Whenever it is clear from context, we will omit
$E$ in $L^p(E)$ and $\|f\|_{p,E}$, and instead write $L^p$ and
$\|f\|_p$. Also note that $L=L^1$.

In order to define $L^\infty(E)$, let $f$ be real-valued and measurable on
a set $E$ of positive measure. Define the
\href{https://en.wikipedia.org/wiki/Essential_supremum_and_essential_infimum}{\emph{essential
  supremum}} of $f$ on $E$ to be
\begin{equation}
  \label{eq:3:essential-supremum}
\esssup_E f\coloneqq\inf\left\{\,\alpha:\left|\left\{\,\bfx\in
      E:f(\bfx)>\alpha\,\right\}\right|=0\,\right\}.
\end{equation}
In words, this the essential supremum of $f$ is the least upper bound of
$f$ outside of a set of measure zero. It can be restated as such: $\esssup
f$ is the smallest number $M$, $-\infty\leq M\leq\infty$, such that
$f(\bfx)\leq M$ almost everywhere in $E$.

In the definition of $\esssup f$, we have made the explicit assumption that
the measure of $E$ is nonzero. Otherwise, $\esssup f=-\infty$ which can
result in awkward or incorrect statements of results involving $L^p$
spaces. Therefore, we shall adopt the convention that $\esssup f=0$ if
$|E|=0$.

A real or complex-valued measurable $f$ is said to be \emph{essentially
  bounded}, or simply \emph{bounded} almost everywhere on $E$ if $\esssup
|f|$ is finite. The class of all functions that are essentially bounded on
$E$ is denoted by $L^\infty(E)$. Clearly, $f\in L^\infty(E)$ if and only if
its real and imaginary parts belong to $L^\infty(E)$. We shall use the
notation $\|f\|_\infty$ synonymously with $\esssup f$.

The following theorem gives some good motivation for the use the notation
$\|f\|_\infty$, at least in the case $|E|<\infty$.
\begin{theorem}[8.1]
If $|E|<\infty$, then $\|f\|_\infty=\lim_{p\to\infty}\|f\|_p$.
\end{theorem}
\begin{proof}[Sketch of proof]
We may assume that $|E|>0$, for otherwise we have a trivial statement,
i.e., $\|f\|_p=0$ for all $p$ and by convention $\|f\|_\infty=0$ so clearly
$\|f\|_p\to\|f\|_\infty$ as $p\to\infty$. Set $M\coloneqq\|f\|_\infty$. If $M'<M$,
\end{proof}

%%% Local Variables:
%%% mode: latex
%%% TeX-master: "../MA544-Quals"
%%% End:

\section{Danielli}
\subsection{Danielli: Practice Exams Spring 2016}
\setcounter{exercise}{0}
\setcounter{equation}{0}

\subsubsection{Exam 1 Practice}
\begin{problem}
  Let \(E\subseteq\bbR^n\) be a measurable set, \(r\in\bbR\) and define the
  set \(rE=\left\{\,rx : x\in E\,\right\}\). Prove that \(rE\) is
  measurable, and that \(|rE|=|r|^n|E|\).
\end{problem}
\begin{solution}
  Define a map a linear map \(T\colon\bbR^n\to\bbR^n\) by
  \(T(x)=rx\). Since a the image of a measurable set \(E\) under linear map
  is measurable and \(m(T(E))=|{\det T}|m(E)=|r|^nm(E)\), it suffices to
  show that \(T(E)=rE\).

  Let \(y\in T(E)\) then \(y=rx\) for some \(x\in E\). Thus, \(y\in
  rE\). Let \(y\in rE\). Then, \(y=rx=T(x)\) for some \(x\in E\). Thus,
  \(y\in T(E)\). It follows that \(m(rE)=|r|^nm(E)\).
\end{solution}

\begin{problem}
  Let \(\{E_n\}\), \(n\in\bbN\) be a collection of measurable sets. Define
  the set
  \[
    \liminf_{n\to\infty} E_n
    =\bigcup_{n=1}^\infty\left(\bigcap_{k=n}^\infty E_k\right).
  \]
  Show that
  \[
    m\left(\liminf_{n\to\infty}
      E_n\right)\leq\liminf_{n\to\infty}m(E_n).
  \]
\end{problem}
\begin{solution}
  Here's a quick and dirty way of proving this: let \(\indicate_{E_n}\) be the
  characteristic function of \(E_n\). Then, by Fatou's lemma,
  \begin{equation}
    \label{eq:ex-1-prep:fatou}
    \int\liminf_{n\to\infty}\indicate_{E_n}(x)\diff x
    \leq\liminf_{n\to\infty}\int\indicate_{E_n}(x)\diff x.
  \end{equation}
  By definition of the characteristic function, it is easy to see that the
  right hand-side of the Equation \eqref{eq:ex-1-prep:fatou} is
  \[
    \liminf_{k\to\infty}m(E_k).
  \]
  But what about the left-hand side of \eqref{eq:ex-1-prep:fatou}? We claim
  that
  \[
    \liminf_{n\to\infty}\indicate_{E_n}=\indicate_{E}
  \]
  where \(E=\liminf_{n\to\infty} E_n\).
  \begin{quote}
    \begin{proof}[Proof of claim]
      Suppose \(x\in E\). We must show that
      \(\liminf_{n\to\infty}\indicate_{E_n}(x)=1\). By definition
      \[
        \liminf_{n\to\infty}\indicate_{E_n}=%
        \lim_{n\to\infty}\left[\inf_{k\geq n}\indicate_{E_k}\right].
      \]
      Now \(x\in E\) if and only if \(x\in \bigcap_{k=N}^\infty E_k\) for
      some \(N\in\bbN\). Then for \(k\geq N\)
      \[
        \inf_{k\geq n}\indicate_{E_k}(x)=1
      \]
      so \(\liminf_{n\to\infty}\indicate_{E_n}(x)=1\).

      On the other hand, if \(x\notin E\) then
      \(x\notin\bigcap_{k=n}^\infty E_k\) for all \(n\in\bbN\). Thus, for
      all \(n\in\bbN\),
      \[
        \inf_{k\geq n}\indicate_{E_k}(x)=0
      \]
      so \(\liminf_{n\to\infty}\indicate_{E_k}=0\).
    \end{proof}
  \end{quote}
  Having established this equivalence, we have
  \[
    m\Bigl(\liminf_{n\to\infty}E_n\Bigr)=%
    \int\liminf_{n\to\infty}\indicate_{E_n}(x)\diff x\leq%
    \liminf_{n\to\infty}\int\indicate_{E_n}(x)\diff x=%
    \liminf_{n\to\infty}m(E_n).
  \]
\end{solution}

\begin{problem}
  Consider the function
  \[
    F(x)=
    \begin{cases}
      m\bigl(B(\mathbf{0},x)\bigr)&x>0,\\
      0&x=0.
    \end{cases}
  \]
  Here \(B(\mathbf{0},r)=\left\{\, y \in\bbR^n:| y |<r\,\right\}\). Prove
  that \(F\) is monotonic increasing and continuous.
\end{problem}

\begin{solution}
  Let \(T\colon\bbR^n\times[0,x)\to\bbR^n\) be the linear map given by
  \(T(x,r)=rx\). By Problem 1, we know that
  \(T\bigl(B(\mathbf{0},1),r\bigr)=B(\mathbf{0},r)\) and consequently,
  \(m\bigl(B(\mathbf{0},1)\bigr)=|r|^nm\bigl(B(\mathbf{0},1)\bigr)\). Interpreting
  \(B(\mathbf{0},0)=\emptyset\), we have
  \(F(x)=|r|^nm\bigl(B(\mathbf{0},1)\bigr)\) and it is easy to see that
  \(F\) is both monotonically increasing and continuous since it is a
  polynomial in \(r\).
\end{solution}

\begin{problem}
  Let \(f\colon\bbR\to\bbR\) be a function. Let \(C\) be the set of all
  points at which \(f\) is continuous. Show that \(C\) is a set of type
  \(G_\delta\).
\end{problem}
\begin{solution}
  Let \(C\) be the subset of \(\bbR\) where \(f\) is continuous, i.e., the
  set
  \[
    C=\bigl\{\,x\in\bbR:\text{given \(\varepsilon>0\) there exist
        \(\delta>0\) such that \(|f(x)-f(y)|<\varepsilon\) whenever
      \(|x-y|<\delta\)}\,\bigr\}.
  \]
  In light of the latter equality, for each \(n\in\bbN\) define the
  following family of subsets of \(C\),
  \[
    G_n=\Bigl\{\,x\in\bbR:\text{there exists \(\delta_n>0\) such that
        \(|f(x)-f(y)|<\frac{1}{n}\) whenever \(|x-y|<\delta_n\)}\,\Bigr\}.
  \]
  We claim that (i) the \(G_n\) are open and (ii)
  \(C=\bigcap_{n\in\bbN}G_n\).

  The proof of (i) is easy: let \(x\in G_n\) then there exists
  \(\delta_n>0\) such that
  \[
    |f(x)-f(y)|<\frac{1}{n}.
  \]
  Then \(B(x,\delta_n)\subseteq G_n\) since \(x'\in B(x,\delta_n)\) implies
  that \(|x-x'|<\delta\) so
  \[
    |f(x)-f(x')|<\frac{1}{n}.
  \]

  The proof of (ii) is also straight forward: let \(x\in C\) then given
  \(\varepsilon>0\) there exists \(\delta>0\) such that
  \[
    |f(x)-f(y)|<\varepsilon
  \]
  whenever \(|x-y|<\delta\). In particular, if \(\varepsilon=1/n\) then
  there exists \(\delta_n\) such that \(|x-y|<\delta_n\) implies
  \[
    |f(x)-f(y)|<\frac{1}{n}
  \]
  for ever \(n\in\bbN\). Thus, \(x\in\bigcap_{n\in\bbN} G_n\). On the other
  hand, if \(x\in\bigcap_{ni\in\bbN} G_n\), then \(x\in G_n\) for all
  \(n\in\bbN\). Thus, given \(\varepsilon>0\), by the Archimedean property
  of the real numbers, there exists a positive integer \(N\) such that
  \(1/N<\varepsilon\) and hence for \(\delta=\delta_N>0\) we have
  \[
    |f(x)-f(y)|<\frac{1}{N}
  \]
  whenever \(|x-y|<\delta_N\). Thus, \(x\in C\).

  It follows that \(C=\bigcap_{n\in\bbN} G_n\) and hence is a \(G_\delta\)
  set.
\end{solution}

\begin{problem}
  Let \(f\colon\bbR\to\bbR\) be a function. Is it true that if the sets
  \(\left\{\,f=r\,\right\}\) are measurable for all \(r\in\bbR\), then
  \(f\) is measurable?
\end{problem}
\begin{solution}
  The statement is false and, of course, the counterexample involves
  existence of nonmeasurable sets. Let \(V\subseteq[0,1]\) be a Vitali set
  and consider the function \(f\colon\bbR\to\bbR\) given by the rule
  \[
    f(x)=
    \begin{cases}
      x&\text{if \(x\in V\),}\\
      -x&\text{if \(x\in\bbR\setminus V\).}
    \end{cases}
  \]
  Then, \(\left\{\,f=r\,\right\}\) is measurable for all \(r\in\bbR\) since
  the set either consists of a single point or is the empty set. However,
  \(\left\{\,f\geq 0\,\right\}=V\) is not measurable.
\end{solution}

\begin{problem}
  Let \(\{f_k\}\) be a sequence of measurable functions on \(\bbR\). Prove
  that the set
  \[
    \Bigl\{\,x:\text{\(\lim_{k\to\infty} f_k(x)\) exists}\,\Bigr\}
  \]
  is measurable.
\end{problem}
\begin{solution}
  Suppose \({\{f_n\}}\), \(n\in\bbN\), is a sequence of measurable
  functions and let
  \[
    E=\Bigl\{\,x: \text{\(\lim_{n\to\infty} f_n(x)\) exists} \,\Bigr\}.
  \]
  Then, by general properties of the limit supremum and the limit infimum,
  we know that \(\lim_{n\to\infty}f_n(x)\) exists if and only if
  \[
    \limsup_{n\to\infty}f_n(x)=\liminf_{n\to\infty}f_n(x).
  \]
  Both of these functions are measurable so the set
  \[
    E=\Bigl\{\,x:\limsup_{n\to\infty}f_n(x)=\liminf_{n\to\infty}f_n(x)\,\Bigr\}.
  \]
  is measurable.
\end{solution}

\begin{problem}
  A real valued function \(f\) on an interval \([a,b]\) is said to be
  \emph{absolutely continuous} if for every \(\varepsilon>0\), there exists
  a \(\delta>0\) such that for every finite disjoint collection
  \(\left\{(a_k,b_k)\right\}_{k=1}^N\) of open intervals in \((a,b)\)
  satisfying \(\sum_{k=1}^Nb_k-a_k<\delta\), one has
  \(\sum_{k=1}^N\left|f(b_k)-f(a_k)\right|<\varepsilon\). Show that an
  absolutely continuous function on \([a,b]\) is of bounded variation on
  \([a,b]\).
\end{problem}
\begin{solution}
  Let \(\varepsilon=1\) then, since \(f\colon[a,b]\to\bbR\) is absolutely
  continuous, there exists \(\delta>0\) such that
  \(|f(x)-f(y)|<\varepsilon\) whenever \(y-x<\delta\) (assuming
  \(x<y\)). Partition the closed interval \([a,b]\) into subintervals
  \(\left\{[a_n,b_n]:1\leq n\leq N\right\}\) of length less than or equal
  to \(\delta\). Then
  \[
    \var(f;[a_n,b_n])\leq 1.
  \]
  Thus,
  \[
    \var(f;[a,b])\leq N
  \]
  for every partition \(\Gamma\) of \([a,b]\).
\end{solution}

\begin{problem}
  Let \(f\) be a continuous function from \([a,b]\) into \(\bbR\). Let
  \(\indicate_{\{c\}}\) be the characteristic function of a singleton
  \(\left\{c\right\}\), that is, \(\indicate_{\{c\}}(x)=0\) if \(x\neq c\) and
  \(\indicate_{\{c\}}(c)=1\). Show that
  \[
    \int_a^b f \diff\indicate_{\{c\}}=
    \begin{cases}
      0&\text{if \(c\in(a,b)\),}\\
      -f(a)&\text{if \(c=a\),}\\
      f(b)&\text{if \(c=b\).}
    \end{cases}
  \]
\end{problem}
\begin{solution}
  There are three cases to consider (1) \(c\in(a,b)\), (2) \(c=a\) and (3)
  \(c=b\). Cases (2) and (3) can be handled easily: if \(c=a\) then the
  Rieman--Stieltjes integral of \(f\) with respect to \(\indicate_{\{c\}}\) is
  the supremum over all sums
  \[
    \sum_{n=1}^N f(\xi_n)\bigl[ \indicate_{\{c\}}(x_n)-\indicate_{\{c\}}(x_{n-1})\bigr]
  \]
  where \(x_0=a\) and \(x_N=b\) for all partitions
  \(\Gamma=\{x_0,\dotsc,x_N\}\) of \([a,b]\). Thus, the sum
  \[
    \sum_{n=1}^N
    f(\xi_n)\bigl[\indicate_{\{c\}}(x_n)-\indicate_{\{c\}}(x_{n-1})\bigr]
    =\begin{cases}
      -f(\xi_0)&\text{if \(c=a\),}\\
      f(\xi_N)&\text{if \(c=b\).}
    \end{cases}
  \]
  Letting \(\Delta(\Gamma)\to 0\), \(\xi_0\to a\) and \(\xi_N\to b\) giving
  us
  \[
    \int_a^b f \diff\indicate_{\{c\}}=
    \begin{cases}
      -f(a)&\text{if \(c=a\),}\\
      f(b)&\text{if \(c=b\).}
    \end{cases}
  \]

  It remains to show that
  \[
    \int_a^b f\diff\indicate_{\{c\}}=0
  \]
  if \(c\in(a,b)\). To that end, note that if \(\Gamma_c\) is a partition
  containing the point \(c\), say, \(x_m=c\) for some \(1\leq m\leq N\),
  the partial sums yield
  \[
    \sum_{n=1}^N
    f(\xi_n)\bigl[\indicate_{\{c\}}(x_n)-\indicate_{\{c\}}(x_{n-1})\bigr]
    =f(\xi_{m+1})-f(\xi_m).
  \]
  Letting \(\Delta(\Gamma_c)\to 0\), since \(f\) is continuous,
  \(f(\xi_{m+1})\to f(\xi_m)\). Thus,
  \[
    \int_a^b f\diff\indicate_{\{c\}}=0.
  \]
\end{solution}

%%% Local Variables:
%%% mode: latex
%%% TeX-master: "../MA544-Quals"
%%% End:

\subsubsection{Exam 1}
\setcounter{exercise}{0}
\setcounter{equation}{0}

I lost this exam. These are the questions I could recall explicitly. For
the first problem, we were asked to show that the Dichlet function
\(\indicate_\bbQ(x)\) is not Riemann integrable and prove something about
\(\bbQ\). For the second question, we were asked to show that the measure
of countable union of disjoint measurable sets \(\{E_n:n\in\bbN\}\), is
equal to the sum of their individual measures (or something to that
effect).
\begin{problem}
\end{problem}
% \begin{solution}
% \end{solution}

\begin{problem}
\end{problem}
% \begin{solution}
% \end{solution}

\begin{problem}
\hfill
\begin{enumerate}[label=(\roman*),noitemsep]
\item Show that if \(B_r=\left\{\,x\in\bbR^n:|x|<r\,\right\}\), then there
  exists a constant \(C\) such that \(|B_r|=Cr^n\).
  \\\\
  (\emph{Hint}: Think of \(B_r\) as \(\left\{\,rx:x\in B_1\,\right\}\).)
\item Let \(E\subseteq\bbR^n\) be a measurable set and let
  \(\varphi_E\colon\bbR^n\to\bbR\) be defined
  \(\varphi_E(x)=\bigl|E\cap B_{|x|}\bigr|\). Use part (i) to prove that
  \(\varphi_E\) is continuous.
\end{enumerate}
\end{problem}
\begin{solution}
\end{solution}

\begin{problem}
  Assume that \(f\colon[a,b]\to\bbR\) is of bounded variation on
  \([a,b]\). Prove that \(f\) is measurable.
\end{problem}
\begin{solution}
\end{solution}

%%% Local Variables:
%%% mode: latex
%%% TeX-master: "../MA544-Quals"
%%% End:

\section{Exam 2 Prep}
\begin{problem}
Define for $\bfx\in\bbR^n$,
\[
f(\bfx)\coloneqq
\begin{cases}
\left|\bfx\right|^{-(n+1)}&\text{if $x\neq 0$,}\\
0&\text{if $x=0$.}
\end{cases}
\]
Prove that $f$ is integrable outside any ball $B(0,\varepsilon)$, and that
there exists a constant $C>0$ such that
\[
\int_{\bbR^n\minus B(0,\varepsilon)}f(x)\diff x\leq\frac{C}{\varepsilon}.
\]
\end{problem}
\begin{proof}
What does it mean for a measurable function $f$ to be integrable over a set
$E\subset\bbR^n$, i.e., that $f$ belong to $L^1(E)$? It means that
\[
\int_E f(x)\diff x<\infty,
\]
or equivalently that the integral of the absolute value of $f$ be
finite.

Now, suppose $f$ is given as in the statement of the problem. It is enough
to prove the inequality
\begin{equation}
  \label{eq:given-inequality-1}
\int_{\bbR^n\minus B_\varepsilon(0)} f(x)\diff x<\frac{C}{\varepsilon}
\end{equation}
to prove that $f\in L^1\left(\bbR^n\minus B_\varepsilon(0)\right)$. Hence,
we proceed in this spirit. First, let us jot some estimates down. For any
$x\in B_\varepsilon(0)$, $|x|<\varepsilon$ so
\[
\int_{\bbR^n}
\]
\end{proof}

\begin{problem}
Let $\left\{f_k\right\}$ be a sequence of nonnegative measurable functions
on $\bbR^n$, and assume that $f_k$ converges pointwise almost everywhere to
a function $f$. If
\[
\int_{\bbR^n} f=\lim_{k\to\infty}\int_{\bbR^n} f_k<\infty,
\]
show that
\[
\int_E f=\lim_{k\to\infty}\int_E f_k
\]
for all measurable subsets $E$ of $\bbR^n$. Moreover, show that this is not
necessarily true if $\int_{\bbR^n} f=\lim_{k\to\infty} f_k=\infty$.
\end{problem}
\begin{proof}
\end{proof}

\begin{problem}
Assume that $E$ is a measurable set of $\bbR^n$, with $\lambda(E)<\infty$. Prove
that a nonnegative function $f$ defined on $E$ is integrable if and only if
\[
\sum_{k=0}^\infty\lambda\left(\left\{\,\bfx\in E:f(\bfx)\geq
    k\,\right\}\right)<\infty.
\]
\end{problem}
\begin{proof}
\end{proof}

\begin{problem}
Suppose that $E$ is a measurable subset of $\bbR^n$, with
$\lambda(E)<\infty$. If $f$ and $g$ are measurable functions on $E$, define
\[
\rho(f,g)=\int_E\frac{|f-g|}{1+|f-g|}.
\]
Prove that $\rho(f_k,g)\to 0$ as $k\to\infty$ if and only if $f_k$
converges to $f$ as $k\to\infty$.
\end{problem}
\begin{proof}
\end{proof}

\begin{problem}
Define the \emph{gamma function} $\Gamma\colon[0,\infty)\to\bbR$ by
\[
\Gamma(y)\coloneqq\int_0^\infty e^{-u}u^{y-1}\diff u,
\]
and the \emph{beta function} $\beta\colon[0,\infty)\times[0,\infty)\to\bbR$
by
\[
\beta(x,y)\coloneqq\int_0^1 t^{x-1}(1-t)^{y-1}\diff t.
\]
\begin{enumerate}[label=(\alph*)]
\item Prove that the definition of the gamma function is well-posed, i.e.,
  the function $u\mapsto e^{-u}u^{y-1}$ is in $L([0,\infty))$ for all
  $y\in[0,\infty)$.
\item Show that
\[
\beta(x,y)=\frac{\Gamma(x)\Gamma(y)}{\Gamma(x+y)}.
\]
\end{enumerate}
\end{problem}
\begin{proof}
\end{proof}

\begin{problem}
\end{problem}
\begin{proof}
\end{proof}

\begin{problem}
\end{problem}
\begin{proof}
\end{proof}

%%% Local Variables:
%%% mode: latex
%%% TeX-master: "../MA544-Quals"
%%% End:

\section{MA 544 - Midterm 2}
\begin{problem}
Assume that $f\in L(\bbR^n)$. Show that for every $\varepsilon>0$ there
exists a ball $B$, centered at the origin, such that
\[
\int_{\bbR^n\setminus B}|f|<\varepsilon.
\]
\end{problem}
\begin{proof}
Recall that $f\in L(\bbR^n)$ if and only if $|f|\in
L(\bbR^n)$. Let $B_k= B(\mathbf{0},k)$ for $k\in\bbN$ and
$\chi_{B_k}$ be the indicator function associated with $B_k$. Then, the
sequence of maps $\left\{|f_k|\right\}$ defined $f_k= f\chi_{B_k}$
converge pointwise to $|f_k|$. Since $|f|\in L(\bbR^n)$, by the monotone
convergence theorem, we have
\begin{equation}
\label{eq:monotonicity-2-1}
\int_{\bbR^n} |f_k|=\int_{B_k}|f|\longrightarrow\int_{\bbR^n}|f|.
\end{equation}
But this means, exactly, that for every $\varepsilon>0$ there exists
sufficiently large $N\in\bbN$ such that
\begin{equation}
\label{eq:desired-inequality-2-1}
\begin{aligned}
\varepsilon&>\left|\int_{\bbR^n}|f_k|-\int_{\bbR^n}|f|\right|\\
={}&-\int_{\bbR^n}|f_k|+\int_{\bbR^n}|f|\\
={}&-\int_{\bbR^n}|f|+\int_{\bbR^n}|f|\\
={}&-\int_{B_k}|f|+\int_{\bbR^n}|f|\\
={}&\int_{\bbR^n\setminus B_k}|f|
\end{aligned}
\end{equation}
as desired.
\end{proof}

\begin{problem}
Let $f\in L(E)$, and let $\{E_j\}$ be a countable collection of pairwise
disjoint measurable subsets of $E$, such that $E=\bigcup_{j=1}^\infty
E_j$. Prove that
\[
\int_E f=\sum_{j=1}^\infty\int_{E_j}f.
\]
\end{problem}
\begin{proof}
First, since the $E_j$'s are pairwise disjoint, by Theorem 3.23, we have
\begin{equation}
\label{eq:disjoint-measure-2-2}
|E|=\sum_{j=1}^\infty|E_j|.
\end{equation}
Let $\chi_{E_j}$ be the characteristic function of the subset $E_j$ of
$E$ and define $f_j= f\chi_{E_j}$ for $j\in\bbN$. Note that, since
both $f$ and $\chi_{E_j}$ are measurable on $E$, $f_j$ is
measurable on $E$ and $\sum_{j=1}^\infty f_j=f$. Moreover, since
$E_j\subset E$, by monotonicity of the integral we have
\begin{equation}
\label{eq:monotonicity-2-2}
\int_{E} f=
\int_{E_j} f+\int_{E\setminus E_j}f=
\int_E f_j+\int_{E\setminus E_j}f.
\end{equation}
Hence, because the $E_j$'s are disjoint $(E\setminus E_k)\setminus
E_\ell=(E\setminus E_\ell)\setminus E_k$ so
\begin{equation}
\label{eq:desired-sum-2}
\int_E f=\sum_{j=1}^\infty\int_E f_j=\sum_{j=1}^\infty\int_{E_j}f
\end{equation}
as desired.
\end{proof}

\begin{problem}
Let $\{f_k\}$ be a family in $L(E)$ satisfying the following property:
For any $\varepsilon>0$ there exits $\delta>0$ such that $|A|<\delta$
implies
\[
\int_A |f_k|<\varepsilon
\]
for all $k\in\bbN$. Assume $|E|<\infty$, and $f_k(x)\to f(x)$ as
$k\to\infty$ for a.e.\@ $x\in E$. Show that
\[
\lim_{k\to\infty}\int_E f_k=\int_E f.
\]
(\emph{Hint:} Use Egorov's theorem.)
\end{problem}
\begin{proof}
Let $\varepsilon>0$ be given. Then, by the hypothesis, there exists
$\delta>0$ such that
such that $|A|<\delta$
implies
\begin{equation}
  \label{eq:hypothesis-2-3}
\int_A |f_k|<\varepsilon
\end{equation}
for all $k\in\bbN$. By Egorov's theorem, there exists a closed subset $F$
of $E$ such that $|E\setminus F|<\delta$ and $f_k\to f$ uniformly on
$F$. Then, by the uniform convergence theorem,
\begin{equation}
\label{eq:uniform-convergence-2-3}
\int_F f_k\longrightarrow \int_F f
\end{equation}
as $k\to\infty$. But by hypothesis, we have
\begin{equation}
\label{eq:need-to-show-2-}
\int_{E\setminus F} |f_k|<\varepsilon.
\end{equation}
Letting $\varepsilon\to 0$, we achieved the desired convergence.
\end{proof}

\begin{problem}
Let $I=[0,1]$, $f\in L(I)$, and define $g(x)=\int_x^1
t^{-1}f(t) d  t$ for $x\in I$. Prove that $g\in L(I)$ and
\[
\int_I g=\int_I f.
\]
\end{problem}
\begin{proof}
By Lusin's theorem, there exists a closed subset $F$ of $I$ with $|I\setminus
F|<\varepsilon$ such that the restriction of $f$ to $F= I\setminus E$
is continuous. Now, since $F$ is closed in $I$ and $I$ is compact, it
follows that $I$ is compact. Hence, by the Stone--Weierstraß approximation
theorem, there exist a sequence of polynomials $\left\{ p_k \right\}$ such
that $p_k\to f$ uniformly on $F$. Then, by the uniform convergence theorem,
we have
\begin{equation}
  \label{eq:uniform-convergence-2-4}
\int_F p_k\longrightarrow \int_F f
\end{equation}
so
\begin{equation}
  \label{eq:uniform-convergence-2-2-4}
\begin{aligned}
\int_F\left[\int_x^1t^{-1}p_k(t) d  t\right] d  x
={}&\int_F\left[\int_x^1 at^{-1}+q_k(t) d  t\right] d  x\\
={}&\int_F q_k'(x)-a\log(x) d  x\\
<{}&\infty
\end{aligned}
\end{equation}
for all $k$ and converges uniformly to $g$ so $g\in L(I)$. I don't know
how to show that in fact $\int_I g=\int_I f$. Perhaps you show that the
places where they differ is a set of measure zero.
\end{proof}

%%% Local Variables:
%%% mode: latex
%%% TeX-master: "../MA544-Quals"
%%% End:

\subsection{Final Exam}

%% Local Variables:
%%% mode: latex
%%% TeX-master: "../MA544-Quals"
%%% End:


%% References
\bibliographystyle{plainnat}
\bibliography{anal-bib}
\end{document}

%%% Local Variables:
%%% mode: latex
%%% TeX-master: t
%%% End:
