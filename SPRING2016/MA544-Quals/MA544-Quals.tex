\def\documentauthor{Carlos Salinas}
\def\documenttitle{MA544: Qual Problems}
% \def\hwnum{1}
\def\shorttitle{MA544 Quals}
\def\coursename{MA544}
\def\documentsubject{measure theory}
\def\authoremail{salinac@purdue.edu}

\documentclass[article,oneside,10pt]{memoir}
\usepackage{geometry}
\usepackage[dvipsnames]{xcolor}
\usepackage[
    breaklinks,
    bookmarks=true,
    colorlinks=true,
    pageanchor=false,
    linkcolor=black,
    anchorcolor=black,
    citecolor=black,
    filecolor=black,
    menucolor=black,
    runcolor=black,
    urlcolor=black,
    hyperindex=false,
    hyperfootnotes=true,
    pdftitle={\shorttitle},
    pdfauthor={\documentauthor},
    pdfkeywords={\documentsubject},
    pdfsubject={\coursename}
    ]{hyperref}

% Use symbols instead of numbers
\renewcommand*{\thefootnote}{\fnsymbol{footnote}}

%% Math
\usepackage{amsthm}
\usepackage{amssymb}
\usepackage{mathtools}
% \usepackage{unicode-math}

%% PDFTeX specific
\usepackage[mathcal]{euscript}
\usepackage{mathrsfs}

\usepackage[LAE,LFE,T2A,T1]{fontenc}
\usepackage[utf8]{inputenc}
\usepackage[farsi,french,german,spanish,russian,english]{babel}
\babeltags{fr=french,
           de=german,
           en=english,
           es=spanish,
           pa=farsi,
           ru=russian
           }
\def\spanishoptions{mexico}

\selectlanguage{english}

\newcommand{\textfa}[1]{\beginR\textpa{#1}\endR}

\usepackage{cmap}
\usepackage{CJKutf8}
\newcommand{\textkr}[1]{\begin{CJK}{UTF8}{mj}#1\end{CJK}}
\newcommand{\textjp}[1]{\begin{CJK}{UTF8}{min}#1\end{CJK}}
\newcommand{\textzh}[1]{\begin{CJK}{UTF8}{bsmi}#1\end{CJK}}

\usepackage{graphicx}
\graphicspath{{figures/}}

% Misc
\usepackage{microtype}
\usepackage{multicol}
\usepackage[inline]{enumitem}
\usepackage{listings}
\usepackage{mleftright}
\mleftright

%% Theorems and definitions
%% remove parentheses
% \makeatletter
% \def\thmhead@plain#1#2#3{%
%   \thmname{#1}\thmnumber{\@ifnotempty{#1}{ }\@upn{#2}}%
%   \thmnote{ {\the\thm@notefont#3}}}
% \let\thmhead\thmhead@plain
% \makeatother

\theoremstyle{plain}
\newtheorem{theorem}{Theorem}
\newtheorem{proposition}[theorem]{Proposition}
\newtheorem{corollary}[theorem]{Corollary}
\newtheorem{claim}[theorem]{Claim}
\newtheorem{lemma}[theorem]{Lemma}
\newtheorem{axiom}[theorem]{Axiom}

\newtheorem*{corollary*}{Corollary}
\newtheorem*{claim*}{Claim}
\newtheorem*{lemma*}{Lemma}
\newtheorem*{proposition*}{Proposition}
\newtheorem*{theorem*}{Theorem}

\theoremstyle{definition}
\newtheorem{definition}{Definition}
\newtheorem{example}{Examples}
\newtheorem{examples}[example]{Examples}
\newtheorem{exercise}{Exercise}[chapter]
\newtheorem{problem}[exercise]{Problem}

\newtheorem*{example*}{Example}
\newtheorem*{exercise*}{Exercise}
\newtheorem*{problem*}{Problem}

%% Redefinitions & commands
\newcommand{\nsubset}{\ensuremath{\not\subset}}
\newcommand{\nsupset}{\ensuremath{\not\supset}}
\newcommand\minus{\ensuremath{\null\smallsetminus}}
\renewcommand\qedsymbol{\ensuremath{\null\hfill\blacksquare}}

%% Commands and operators
\DeclareMathOperator{\id}{id}
\DeclareMathOperator{\im}{im}
\DeclareMathOperator{\diff}{d}

\DeclareMathOperator{\Id}{Id}
\DeclareMathOperator{\Img}{Im}
\DeclareMathOperator{\Int}{Int}
\DeclareMathOperator{\Cl}{Cl}
\DeclareMathOperator{\BV}{BV}

%% Symbols
\newcommand{\bbC}{\mathbb{C}}
\newcommand{\bbN}{\mathbb{N}}
\newcommand{\bbQ}{\mathbb{Q}}
\newcommand{\bbR}{\mathbb{R}}
\newcommand{\bbZ}{\mathbb{Z}}

\newcommand{\bfC}{\mathbf{C}}
\newcommand{\bfN}{\mathbf{N}}
\newcommand{\bfQ}{\mathbf{Q}}
\newcommand{\bfR}{\mathbf{R}}
\newcommand{\bfZ}{\mathbf{Z}}

\newcommand{\bfu}{\mathbf{u}}
\newcommand{\bfv}{\mathbf{v}}
\newcommand{\bfw}{\mathbf{w}}
\newcommand{\bfx}{\mathbf{x}}
\newcommand{\bfy}{\mathbf{y}}
\newcommand{\bfz}{\mathbf{z}}

\newcommand{\dsC}{\mathds{C}}
\newcommand{\dsN}{\mathds{N}}
\newcommand{\dsQ}{\mathds{Q}}
\newcommand{\dsR}{\mathds{R}}
\newcommand{\dsZ}{\mathds{Z}}

\newcommand{\calA}{\mathcal{A}}
\newcommand{\calB}{\mathcal{B}}
\newcommand{\calC}{\mathcal{C}}
\newcommand{\calO}{\mathcal{O}}
\newcommand{\calU}{\mathcal{V}}
\newcommand{\calV}{\mathcal{U}}

\newcommand{\scrA}{\mathscr{A}}
\newcommand{\scrB}{\mathscr{B}}
\newcommand{\scrC}{\mathscr{C}}
\newcommand{\scrL}{\mathscr{L}}
\newcommand{\scrO}{\mathscr{O}}
\newcommand{\scrS}{\mathscr{S}}
\newcommand{\scrT}{\mathscr{T}}

\begin{document}
\author{\href{mailto:\authoremail}{\documentauthor}}
\title{\documenttitle}
\date{\today}
\maketitle

\section{Danielli}
\subsection{Danielli: Practice Exams Spring 2016}
\setcounter{exercise}{0}
\setcounter{equation}{0}

\subsubsection{Exam 1 Practice}
\begin{problem}
  Let $E\subset\bbR^n$ be a measurable set, $r\in\bbR$ and define the set
  $rE=\left\{\,rx : x\in E\,\right\}$. Prove that $rE$ is measurable, and
  that $|rE|=|r|^n|E|$.
\end{problem}
\begin{solution}
  Define a map a linear map $T\colon\bbR^n\to\bbR^n$ by $T(x)=rx$. Since a
  the image of a measurable set $E$ under linear map is measurable and
  $m(T(E))=|{\det T}|m(E)=|r|^nm(E)$, it suffices to show that $T(E)=rE$.

  Let $y\in T(E)$ then $y=rx$ for some $x\in E$. Thus, $y\in rE$. Let $y\in
  rE$. Then, $y=rx=T(x)$ for some $x\in E$. Thus, $y\in T(E)$. It follows
  that $m(rE)=|r|^nm(E)$.
\end{solution}

\begin{problem}
  Let $\{E_k\}$, $k\in\bbN$ be a collection of measurable sets. Define the
  set
  \[
    \liminf_{k\to\infty} E_k
    =\bigcup_{k=1}^\infty\left(\bigcap_{n=k}^\infty E_n\right).
  \]
  Show that
  \[
    m\left(\liminf_{k\to\infty}
      E_k\right)\leq\liminf_{k\to\infty}m(E_k).
  \]
\end{problem}
% Following the style of \cite[Ch.\@ 1, p.\@ 2]{wheeden-zygmund},
  % particularly, the sets defined after the introduction of equation (1.1),
  % set
  % \begin{equation}
  %   \label{eq:prep:1:2}
  %   V_k=\bigcap_{\ell=k}^\infty E_\ell.
  % \end{equation}
  % Note that the collection of sets $\{V_k\}$ forms an increasing sequence,
  % that is, if $x\in V_k$ then, by \eqref{eq:prep:1:2}, $x $ is in the
  % intersection $E_k\cap\bigl(\bigcap_{\ell=k+1}E_\ell\bigr)$, but, by
  % \eqref{eq:prep:1:2}, $\bigcap_{\ell=k+1}E_\ell=V_{k+1}$ thus, $x $ is
  % in $V_{k+1}$ so $V_{k+1}\supset V_k$. Hence, we have
  % $V_k\nearrow\liminf E_k$.

  % Now, consider the sequence $\{|V_k|\}$ formed by the Lebesgue measure of
  % the $V_k$. By Theorem 3.26 from \cite[Ch.\@ 3, p.\@ 51]{wheeden-zygmund},
  % since $V_k\nearrow\liminf E_k$,
  % \begin{equation}
  %   \label{eq:prep:1:3}
  %   \lim_{k\to\infty}|V_k|=
  %   \lim_{k\to\infty}\left|\bigcap_{\ell=k}^\infty E_\ell\right|=
  %   \left|\liminf_{k\to\infty} E_k\right|.
  % \end{equation}
  % But note that, by the monotonicity of the Lebesgue measure, we have
  % \begin{equation}
  %   \label{eq:prep:1:4}
  %   \left|\bigcap_{\ell=k}^\infty E_\ell\right|\leq |E_k|,
  % \end{equation}
  % so, by properties of the $\liminf$, in particular, by Theorem 19(v) from
  % \cite[Ch.\@ 1, p.\@ 23]{royden}, we have
  % \begin{equation}
  %   \label{eq:prep:1:5}
  %   \limsup_{k\to\infty}|V_k|\leq\liminf_{k\to\infty}|E_k|.
  % \end{equation}
  % Hence, by \eqref{eq:prep:1:3} and Proposition 19 (iv), since the sequence
  % $\{|V_k|\}$ converges and is equal to the measure of $\liminf E_k$, by
  % \eqref{eq:prep:1:5}, we have
  % \begin{equation}
  %   \label{eq:prep:1:6}
  %   \left|\liminf_{k\to\infty} E_k\right|\leq\liminf_{k\to\infty}|E_k|
  % \end{equation}
  % as was to be shown.
\begin{solution}
  Here's a quick and dirty way of proving this: let $\chi_{E_n}$ be the
  characteristic function of $E_n$. Then, by Fatou's lemma,
  \begin{equation}
    \label{eq:ex-1-prep:fatou}
    \int\liminf_{n\to\infty}\chi_{E_n}(x)\diff x
    \leq\liminf_{n\to\infty}\int\chi_{E_n}(x)\diff x.
  \end{equation}
  By definition of the characteristic function, it is easy to see that the
  right hand-side of the Equation \eqref{eq:ex-1-prep:fatou} is
  \[
    \liminf_{k\to\infty}m(E_k).
  \]
  But what about the left-hand side of \eqref{eq:ex-1-prep:fatou}? We claim
  that
  \[
    \liminf_{n\to\infty}\chi_{E_n}=\chi_{E}
  \]
  where $E=\liminf_{n\to\infty} E_n$.
  \begin{quote}
    \begin{proof}[Proof of claim]
      Let $x\in E$. We must show that
      $\liminf_{n\to\infty}\chi_{E_n}(x)=1$. By definition
      \[
        \liminf_{n\to\infty}\chi_{E_n}=%
        \lim_{n\to\infty}\left[\inf_{k\geq n}\chi_{E_k}\right].
      \]
      Now
    \end{proof}
  \end{quote}

  Define
  \[
    V_n=\bigcap_{k=n}^\infty E_k.
  \]
  Note that ${\{V_n\}}_{n=1}^\infty$ forms an increasing sequence of
\end{solution}

\begin{problem}
  Consider the function
  \[
    F(x)=
    \begin{cases}
      |B(\mathbf{0},x)|&x>0\\
      0&x=0
    \end{cases}.
  \]
  Here $B(\mathbf{0},r)=\left\{\, y \in\bbR^n:| y |<r\,\right\}$. Prove
  that $F$ is monotonic increasing and continuous.
\end{problem}
\begin{solution}
  Define the linear map $T\colon[0,\infty)\times\bbR^n\to\bbR^n$ by
  $T(r) x = rx $. We claim that $B(\mathbf{0},r)=T(r,B(\mathbf{0},1))$. To
  reduce notation, set $B_1= B(\mathbf{0},1)$ and $B_r= B(\mathbf{0},r)$.
  \begin{quote}
    \begin{proof}[Proof of claim]
      Let $x\in B_r$. Then $|x |<r$ so $|x |/r<1$. Thus, $|x |/r\in B_1$ so
      it is in the image of $B_1$ under the map $T(r,\cdot)$.

      On the other hand, suppose $x\in T(r,B_1)$. Then $x =r y $ for some
      $ y \in B_1$. Then, since $| y |<1$, $|x |=r| y |<r$ so $x\in B_r$.
    \end{proof}
  \end{quote}

  From the claim, we see that $F(x)=|T(x,B(\mathbf{0},1))|$ which, by
  Problem 1, is nothing more that the polynomial $|B_1|x^n$. It is clear,
  from this equivalence, that $F$ is monotonically increasing: Take
  $x,y\in[0,\infty)$ such that $x<y$, then $x^n<y^n$ so
  \begin{equation}
    \label{eq:prep:1:7}
    F(x)=|B_1|x^n<|B_1|y^n=F(y).
  \end{equation}
  Thus, $F$ is monotonically increasing.

  In the argument above, since $F(x)=|B_1|x^n$ is a polynomial in
  $[0,\infty)$ (and polynomials are continuous on $\bbR$) $F$ is continuous
  on $[0,\infty)$.
\end{solution}

\begin{problem}
  Let $f\colon\bbR\to\bbR$ be a function. Let $C$ be the set of all points
  at which $f$ is continuous. Show that $C$ is a set of type $G_\delta$.
\end{problem}
\begin{solution}
  (Without much motivation) let us consider the collection of sets
  $\{E_k\}$ defined by
  \begin{equation}
    \label{eq:prep:1:8}
    E_k=\left\{\,x\in\bbR:
      \text{there exists $\delta>0$ such that $y,z\in B(x,\delta)$ implies $\left|f(y)-f(z)\right|<\frac{1}{k}$}\,\right\}.
  \end{equation}
  We claim that $C=\bigcap_{k=1}^\infty E_k$ and that each $E_k$ is open.
  \begin{solution}[Proof of claim]
    First, we demonstrate equality. $\subset$: Suppose $x\in C$. Then, by
    the definition of continuity, for every $\varepsilon>0$, there exists a
    $\delta>0$ such that $y\in B(x,\delta)$ implies
    $|f(x)-f(y)|<\delta$. In particular, for every $k$, there exists
    $\delta>0$ such that for $y\in B(x,\delta)$ the inequality
    $|f(x)-f(y)|<1/k$ holds. Thus, $x$ is in $\bigcap_{k=1}^\infty E_k$.

    $\supset$: On the other hand, suppose that
    $x\in\bigcap_{k=1}^\infty E_k$. Then, given $\varepsilon>0$, by the
    Archimedean property, there exists a positive integer $N$ such that
    $1/N<\varepsilon$. Then, since $x\in\bigcap_{k=1}^\infty E_k$,
    $x\in E_N$ so
    \begin{equation}
      \label{eq:prep:1:9}
      |f(x)-f(y)|<\frac{1}{N}<\varepsilon.
    \end{equation}
    Thus, $x$ is in $C$ and $C=\bigcap_{k=1}^\infty E_k$.

    All that remains to be shown is that the $E_k$ are open. But this is
    clear by the way we defined $E_k$ in \eqref{eq:prep:1:8}: Let
    $x\in E_k$, then there exists $\delta>0$ such that for any
    $y,z\in B(x,\delta)$, $|f(y)-f(z)|<1/k$; Let $x'\in B(x,\delta)$ and
    set $\delta'=\min\{|(x+\delta)-x'|,|(x-\delta)-x|\}$. Then, since
    $B(x',\delta')\subset B(x,\delta)$, for every $y,z\in B(x',\delta')$,
    we have $|f(y)-f(z)|<1/k$. Hence, $x'\in E_k$ for any
    $x'\in B(x,\delta)$ so $B(x,\delta)\subset E_k$.
  \end{solution}
  Since $C$ can be expressed as the countable intersection of open sets
  $E_k$, it follows that $C$ is a $G_\delta$ set.
\end{solution}
\begin{problem}
  Let $f\colon\bbR\to\bbR$ be a function. Is it true that if the sets
  $\left\{\,f=r\,\right\}$ are measurable for all $r\in\bbR$, then $f$ is
  measurable?
\end{problem}
\begin{solution}
  If $\left\{\,f=r\,\right\}$ are measurable for all $r\in\bbR$, it is not
  necessarily the case that $f$ is measurable. Consider the following
  construction: Let $E\subset(0,1)$ be an unmeasurable set.\footnote{It's
    construction does not concern us. The interested reader such direct
    their refer to Theorem 3.38 from \cite[Ch.\@ 3, p.\@
    57-58]{wheeden-zygmund} or Theorem 17 from \cite[Ch.\@ 2\S 7, p.\@
    48]{royden}.} Define a map $f\colon\bbR\to\bbR$ by
  \begin{equation}
    \label{eq:prep:1:11}
    f(x)=
    \begin{cases}
      x&\text{if $x\in\bbR\setminus((0,1)\setminus E)$},\\
      x+1&\text{if $x\in (0,1)\setminus E$.}
    \end{cases}
  \end{equation}
  By the definition, it is clear that $\left\{\,f=r\,\right\}$ is
  measurable and $\left|\left\{\,f=r\,\right\}\right|=0$ since
  $\{\,f=r\,\}$ contains at most two elements. However, the set
  $\left\{\,0<f<1\,\right\}=E$ is not measurable. Thus, $f$ is not
  measurable.
\end{solution}

\begin{problem}
  Let $\left\{f_k\right\}_{k=1}^\infty$ be a sequence of measurable
  functions on $\bbR$. Prove that the set
  $\left\{\,x:\text{$\lim_{k\to\infty} f_k(x)$ exists}\,\right\}$ is
  measurable.
\end{problem}
\begin{solution}
  By Theorem 4.12 from \cite[Ch.\@ 4, p.\@ 67]{wheeden-zygmund},
  $\liminf_{k\to\infty}f_k$ and $\limsup_{k\to\infty}f_k$ are
  measurable. By Theorem 4.7 from \cite[Ch.\@ 4, p.\@ 66]{wheeden-zygmund}
  \begin{equation}
    \label{eq:prep:1:12}
    \left\{\,\liminf_{k\to\infty} f_k<\limsup_{k\to\infty} f_k\,\right\}
  \end{equation}
  is measurable. Since
  \begin{equation}
    \label{eq:prep:1:13}
    \left\{\,\text{$\lim_{k\to\infty}f_k$ exists}\,\right\}=
    \left\{\,{\limsup_{k\to\infty}f_k=\liminf_{k\to\infty}f_k}\,\right\}=
    \bbR\setminus
    \left\{\,{\liminf_{k\to\infty} f_k<\limsup_{k\to\infty} f_k}\,\right\},
  \end{equation}
  by Theorem 3.17 from \cite[Ch.\@ 3, p.\@ 48]{wheeden-zygmund}, the set
  $\left\{\,\text{$\lim_{k\to\infty}f_k$ exists}\,\right\}$ is measurable.
  % In a fashion similar to that of Problem 4, consider the set collection
  % of sets $\{E_k\}$ given by
  % \begin{equation}
  %   \label{eq:prep:1:11}
  %       %   E_k= \left\{\, x\in\bbR:\text{there exists $N$ such that $m,n\geq N$
  %     implies $\left|f_n(x)-f_m(x)\right|<\frac{1}{k}$} \,\right\}.
  % \end{equation}
  % You can show that the $E_k$ are open and that
  % $\left\{\,x:\text{$\lim_{x\to\infty}f_k(x)$
  %   exists}\,\right\}=\bigcap_{k=1}^\infty E_k$. Then, since open sets
  % are measurable and, by Theorem 3.18 from \cite[Ch.\@ 3, p.\@
  % 48]{wheeden-zygmund}, the countable intersection of measurable sets is
  % measurable, $\left\{\,x:\text{$\lim_{x\to\infty}f_k(x)$
  %   exists}\,\right\}$ is measuable.
\end{solution}
\begin{problem}
  A real valued function $f$ on an interval $[a,b]$ is said to be
  \emph{absolutely continuous} if for every $\varepsilon>0$, there exists a
  $\delta>0$ such that for every finite disjoint collection
  $\left\{(a_k,b_k)\right\}_{k=1}^N$ of open intervals in $(a,b)$
  satisfying $\sum_{k=1}^Nb_k-a_k<\delta$, one has
  $\sum_{k=1}^N\left|f(b_k)-f(a_k)\right|<\varepsilon$. Show that an
  absolutely continuous function on $[a,b]$ is of bounded variation on
  $[a,b]$.
\end{problem}
\begin{solution}
  Suppose $f$ is absolutely continuous on $[a,b]$. Let $\varepsilon=
  1$. Then, there exists $\delta>0$ such that for every finite disjoint
  collection $\left\{(a_k,b_k)\right\}_{k=1}^N$ of open intervals in
  $(a,b)$ satisfying $\sum_{k=1}^Nb_k-a_k<\delta$, one has
  $\sum_{k=1}^N\left|f(b_k)-f(a_k)\right|<1$. Let
  $N=\lceil(b-a)/\delta\rceil$, that is, $N$ is the smallest integer
  greater than $(b-a)/\delta$, and consider the partition $\Gamma=\{x_k\}$
  where $x_k= a+k(b-a)/N$, for $k=0,\dotsc,N$. Then
  $x_k-x_{k-1}<(b-a)/N<\delta$ so, by Theorem 2.2(i) from \cite[Ch.\@ 2,
  p.\@ 19]{wheeden-zygmund}, we have $V[f;x_{k-1},x_k]<1$ for
  $k=0,\dotsc,N$. In follows by Theorem 2.2(ii) that
  \begin{equation}
    \label{eq:prep:1:14}
    V[f;a,b]=\sum_{k=1}^N V[f;x_{k-1},x_k]<N.
  \end{equation}
  Thus, $f$ is b.v.\@ on $[a,b]$.
\end{solution}

\begin{problem}
  Let $f$ be a continuous function from $[a,b]$ into $\bbR$. Let
  $\chi_{\{c\}}$ be the characteristic function of a singleton
  $\left\{c\right\}$, that is, $\chi_{\{c\}}(x)=0$ if $x\neq c$ and
  $\chi_{\{c\}}(c)=1$. Show that
  \[
    \int_a^b f d \chi_{\{c\}}=
    \begin{cases}
      0&\text{if $c\in(a,b)$,}\\
      -f(a)&\text{if $c=a$,}\\
      f(b)&\text{if $c=b$.}
    \end{cases}
  \]
\end{problem}
\begin{solution}
  The result follows quite easily from more sophisticated measure theoretic
  arguments. At this point, however, such language has not been discussed
  so we shall prove this using nothing but the definition of the
  Riemann--Stieltjes integral and properties thereof.

  Let us consider each case $c\in(a,b)$, $c=a$, and $c=b$ separately.

  Recall that the given a partition $\Gamma=\{x_0,\dotsc,x_m\}$ of $[a,b]$,
  the Riemann--Stieltjes sum of $f$ with respect to $\varphi$ is
  \begin{equation}
    \label{eq:prep:1:15}
    R_\Gamma=\sum_{k=1}^mf(\xi_k)[\varphi(x_k)-\varphi(x_{k-1})].
  \end{equation}
  The Riemann--Stieltjes integral is defined as the limit
  \begin{equation}
    \label{eq:prep:1:16}
    \int_a^b f\diff\varphi=\lim_{|\Gamma|\to 0} R_\Gamma
  \end{equation}
  if it exists.

  Suppose $c\in(a,b)$. Then, for any partition $\Gamma$ of $[a,b]$, either
  $c\in\Gamma$ or $c\notin\Gamma$. In the latter case, $R_\Gamma=0$. In the
  former case $c$ is one of the $x_k$, say $c=x_\ell$ for $0<\ell<m$. Then
  \begin{equation}
    \label{eq:prep:1:17}
    \begin{aligned}
      R_\Gamma&=\sum_{k=1}^mf(\xi_k)[\chi_{\{c\}}(x_k)-\chi_{\{c\}}(x_{k-1})]\\
      &=0+\dotsb+0+f(\xi_{\ell-1})-f(\xi_\ell)+0+\dotsb+0\\
      &=f(\xi_{\ell-1})-f(\xi_\ell).
    \end{aligned}
  \end{equation}
  Since $f$ is continuous, given $\varepsilon>0$ there exists $\delta>0$
  such that $|\xi_\ell-\xi_{\ell-1}|<\delta$ implies
  $|f(\xi_{\ell})-f(\xi_{\ell-1})|<\varepsilon$. It follows that the
  quantity in \eqref{eq:prep:1:17} approaches $0$ as $|\Gamma|$ approaches
  $0$. Therefore, $\int_a^b f\diff\chi_{\{c\}}=0$.

  Suppose $c=a$. Then, since any partition $\Gamma$ of $[a,b]$ must contain
  the point $a$, we have
  \begin{equation}
    \label{eq:prep:1:18}
    \begin{aligned}
      R_\Gamma
      &=\sum_{k=1}^mf(\chi_k)[\chi_{\{c\}}(x_k)-\chi_{\{c\}}(x_{k-1})]\\
      &
      \begin{aligned}
        =f(\xi_1)[\chi_{\{c\}}(x_1)-\chi_{\{c\}}(x_0)]&
        +f(\xi_2)[\chi_{\{c\}}(x_2)-\chi_{\{c\}}(x_1)]\\
        &+\dotsb+f(\xi_m)[\chi_{\{c\}}(x_m)-\chi_{\{c\}}(x_{m-1})]
      \end{aligned}\\
      &=-f(\xi_1)+0+\dotsb+0\\
      &=-f(\xi_1)
    \end{aligned}
  \end{equation}
  Taking the limit as $|\Gamma|\to 0$, $\xi_1\to a$ so, by continuity of
  $f$, $f(\xi_1)\to f(a)$. Thus, $\int_a^b f\diff\chi_{\{c\}}=-f(a)$.

  A similar argument to the one above shows that, if $c=b$, the
  Riemann--Stieltjes integral $\int_a^bf\diff\chi_{\{c\}}=f(b)$.
\end{solution}

%%% Local Variables:
%%% mode: latex
%%% TeX-master: "../MA544-Quals"
%%% End:

% \subsubsection{Exam 1}
\setcounter{exercise}{0}
\setcounter{equation}{0}

I lost this exam. These are the questions I could recall explicitly. For
the first problem, we were asked to show that the Dichlet function
\(\indicate_\bbQ(x)\) is not Riemann integrable and prove something about
\(\bbQ\). For the second question, we were asked to show that the measure
of countable union of disjoint measurable sets \(\{E_n:n\in\bbN\}\), is
equal to the sum of their individual measures (or something to that
effect).
\begin{problem}
\end{problem}
% \begin{solution}
% \end{solution}

\begin{problem}
\end{problem}
% \begin{solution}
% \end{solution}

\begin{problem}
\hfill
\begin{enumerate}[label=(\roman*),noitemsep]
\item Show that if \(B_r=\left\{\,x\in\bbR^n:|x|<r\,\right\}\), then there
  exists a constant \(C\) such that \(|B_r|=Cr^n\).
  \\\\
  (\emph{Hint}: Think of \(B_r\) as \(\left\{\,rx:x\in B_1\,\right\}\).)
\item Let \(E\subseteq\bbR^n\) be a measurable set and let
  \(\varphi_E\colon\bbR^n\to\bbR\) be defined
  \(\varphi_E(x)=\bigl|E\cap B_{|x|}\bigr|\). Use part (i) to prove that
  \(\varphi_E\) is continuous.
\end{enumerate}
\end{problem}
\begin{solution}
\end{solution}

\begin{problem}
  Assume that \(f\colon[a,b]\to\bbR\) is of bounded variation on
  \([a,b]\). Prove that \(f\) is measurable.
\end{problem}
\begin{solution}
\end{solution}

%%% Local Variables:
%%% mode: latex
%%% TeX-master: "../MA544-Quals"
%%% End:

\end{document}

%%% Local Variables:
%%% mode: latex
%%% TeX-master: t
%%% End:
