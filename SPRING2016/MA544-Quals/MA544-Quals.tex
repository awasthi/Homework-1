\def\documentauthor{Carlos Salinas}
\def\documenttitle{MA544: Qual Problems}
% \def\hwnum{1}
\def\shorttitle{MA544 Quals}
\def\coursename{MA544}
\def\documentsubject{measure theory}
\def\authoremail{salinac@purdue.edu}

\documentclass[article,oneside,10pt]{memoir}
\usepackage{geometry}
\usepackage[dvipsnames]{xcolor}
\usepackage[
    breaklinks,
    bookmarks=true,
    colorlinks=true,
    pageanchor=false,
    linkcolor=black,
    anchorcolor=black,
    citecolor=black,
    filecolor=black,
    menucolor=black,
    runcolor=black,
    urlcolor=black,
    hyperindex=false,
    hyperfootnotes=true,
    pdftitle={\shorttitle},
    pdfauthor={\documentauthor},
    pdfkeywords={\documentsubject},
    pdfsubject={\coursename}
    ]{hyperref}
\usepackage[numbers]{natbib}

%% Math
\usepackage{amsmath}
\usepackage{amsfonts}
\usepackage{amssymb}
\usepackage{amsthm}
\usepackage{mathtools}
% \usepackage{eucal}
% \usepackage{mathrsfs}
% \usepackage[nointegrals]{wasysym}

%% Language
\usepackage{cmap}
\usepackage[LAE,LFE,T2A,T1]{fontenc}
\usepackage[utf8]{inputenc}
\usepackage[farsi,french,german,spanish,russian,english]{babel}
\babeltags{fr=french,
           de=german,
           en=english,
           es=spanish,
           pa=farsi,
           ru=russian
           }
\def\spanishoptions{mexico}

\selectlanguage{english}

\newcommand{\textfa}[1]{\beginR\textpa{#1}\endR}

\usepackage{CJKutf8}
\newcommand{\textkr}[1]{\begin{CJK}{UTF8}{mj}#1\end{CJK}}
\newcommand{\textjp}[1]{\begin{CJK}{UTF8}{min}#1\end{CJK}}
\newcommand{\textzh}[1]{\begin{CJK}{UTF8}{bsmi}#1\end{CJK}}

%% Misc
\usepackage{graphicx}
\graphicspath{{figures/}}

\usepackage{microtype}
\usepackage{lineno}
\usepackage{multicol}
\usepackage[inline]{enumitem}
\usepackage{listings}
\usepackage{mleftright}
\mleftright
\usepackage{carlos-variables}

% %% Unicode math and Polyglossia
% \usepackage{unicode-math}
% \usepackage{unicode-minionmath}

% \setmainfont[Ligatures=TeX]{Libertinus Serif}
% \setsansfont{Libertinus Sans}
% \setmonofont{Libertinus Mono}
% \setmathfont{Minion Math}
% \setmathfont[range={\mathfrak}]{XITS Math}
% \setmathfont[range={\mathcal},StylisticSet=1]{XITS Math}
% \setmathfont[range={\mathscr}]{XITS Math}
% \setmathfont[range={}]{Minion Math}

% \usepackage{polyglossia}

% \newfontfamily\cyrillicfont[Script=Cyrillic]{Libertinus Serif}
% \newfontfamily\cyrillicfontsf[Script=Cyrillic]{Libertinus Sans}

% \newfontfamily\farsifont[Script=Arabic,
%                          Scale=MatchUppercase]{Amiri}

% \setmainlanguage[variant=american]{english}
% \setotherlanguage{farsi}
% \setotherlanguage{french}
% \setotherlanguage[spelling=new,latesthyphen,babelshorthands]{german}
% \setotherlanguage{spanish}
% \setotherlanguage[spelling=modern,babelshorthands]{russian}

% \makeatletter
% \@Latintrue
% \makeatother

% \usepackage{xeCJK}
% \usepackage[overlap]{ruby}
% \renewcommand\rubysep{-0.2ex}
% \xeCJKDeclareSubCJKBlock{Kana}{"3040 -> "309F, "30A0 -> "30FF, "31F0 -> "31FF, "1B000 -> "1B0FF}
% \xeCJKDeclareSubCJKBlock{Hangul}{"1100 -> "11FF, "3130 -> "318F, "A960 -> "A97F, "AC00 -> "D7AF, "D7B0 -> "D7FF}

% \setCJKmainfont{HanaMinA}
% \setCJKmainfont[Kana]{HanaMinA}
% \setCJKmainfont[Hangul]{NanumMyeongjo}
% \setCJKsansfont[Hangul]{NanumGothic}

%% Theorems and definitions
%% remove parentheses
% \makeatletter
% \def\thmhead@plain#1#2#3{%
%   \thmname{#1}\thmnumber{\@ifnotempty{#1}{ }\@upn{#2}}%
%   \thmnote{ {\the\thm@notefont#3}}}
% \let\thmhead\thmhead@plain
% \makeatother

\theoremstyle{plain}
\newtheorem{theorem}{Theorem}
\newtheorem{proposition}[theorem]{Proposition}
\newtheorem{corollary}[theorem]{Corollary}
\newtheorem{claim}[theorem]{Claim}
\newtheorem{lemma}[theorem]{Lemma}
\newtheorem{axiom}[theorem]{Axiom}

\newtheorem*{corollary*}{Corollary}
\newtheorem*{claim*}{Claim}
\newtheorem*{lemma*}{Lemma}
\newtheorem*{proposition*}{Proposition}
\newtheorem*{theorem*}{Theorem}

\theoremstyle{definition}
\newtheorem{definition}{Definition}
\newtheorem{example}{Examples}
\newtheorem{examples}[example]{Example}
\newtheorem{exercise}{Exercise}[chapter]
\newtheorem{problem}[exercise]{Problem}

\newtheorem*{example*}{Example}
\newtheorem*{exercise*}{Exercise}
\newtheorem*{problem*}{Problem}

\begin{document}
\author{\href{mailto:\authoremail}{\documentauthor}}
\title{\documenttitle}
\date{\today}
\maketitle
% \chapter{Notes}
Notes based off of Wheeden and Zygmund's \emph{Measure and Integral} book.
\section{Exam 1 Review}
This is all of the material we covered before exam 1.

\bigskip

Introductory material I should have known from 504.

If $\calF$ is a countable (i.e., finite or countably infinite), it will be
called a \emph{sequence of sets} and denoted
$\calF\coloneqq\left\{\,E_k:k=1,2,...\,\right\}$. The corresponding union
and intersection will be written $\bigcup_k E_k$ and $\bigcap_k E_k$. A
sequence $\{E_k\}$ of sets is said to \emph{increase} to $\bigcup_k E_k$ if
$E_k\subset E_{k+1}$ for all $k$ and to \emph{decrease} to $\bigcap_k E_k$
if $E_k\supset E_{k+1}$ for all $k$; we use the notation
$E_k\nearrow\bigcap_k E_k$ and $E_k\searrow\bigcap_k E_k$ to denote these
two possibilities. If $\left\{E_k\right\}_{k=1}^\infty$ is a sequence of
sets, we define
\begin{equation}
\label{eq:limsup-liminf-sets}
\limsup E_k=\bigcap_{j=1}^\infty\bigcup_{k=j}^\infty E_k,\qquad
\liminf E_k=\bigcup_{j=1}^\infty\bigcap_{k=j}^\infty E_k,
\end{equation}
noting that the sets $U_j\coloneqq\bigcup_{k=j}^\infty E_k$ and
$V_j\coloneqq\bigcap_{k=j}^\infty E_k$ satisfy $U_j\searrow\limsup E_k$ and
$V_j\nearrow\liminf E_k$. Then $\limsup E_k$ consists of those points of
$\bbR^n$ that belong to infinitely many $E_k$ and $\liminf E_k$ to those
that belong to all $E_k$ for $k\geq k_0$ (where $k_0$ may vary from point
to point). Thus $\liminf E_k\subset\limsup E_k$.

If $E_1$ and $E_2$ are two sets, we define $E_1\setminus E_2$ by
$E_1\setminus E_2\coloneqq E_1\cap\complement E_2$ and call it the
\emph{difference} of $E_1$ and $E_2$ or the \emph{relative complement} of
$E_2$ in $E_1$. We will often have occasion to use \emph{de Morgan laws},
which govern relations between complements, unions, and intersections;
these state that
\begin{equation}
\label{eq:de-morgan-laws}
\complement
\left(\bigcup_{E\in\calF}E\right)=
\bigcap_{E\in\calF}\complement E,\qquad
\complement
\left(\bigcap_{E\in\calF}E\right)=
\bigcup_{E\in\calF}\complement E,
\end{equation}
and are easily verified.

If $\bfx\in\bbR^n$, we say that a sequence $\{\bfx_k\}$ \emph{converges} to
$\bfx$, or that $\bfx$ is the \emph{limit point} of $\{\bfx_k\}$, if
$\|\bfx-\bfx_k\|\to 0$ as $k\to\infty$. We denote this by writing either
$\bfx=\lim_{k\to\infty}\bfx_k$ or $\bfx_k\to\bfx$ as $k\to\infty$. A point
$\bfx\in\bbR^n$ is called a \emph{limit point of a set $E$} if it is the
limit point of a sequence of distinct points of $E$. A point $\bfx\in E$ is
called a \emph{isolated point} of $E$ if it is not the limit point of any
sequence in $E$ (excluding the trivial sequence $\left\{\bfx_k\right\}$
where $\bfx_k=\bfx$ for all $k\in\bbN$). It follows that $\bfx$ is isolated
if and only if there is a $\delta>0$ such that $\|\bfx-\bfy\|>\delta$ for
every $\bfy\in E$, $\bfy\neq\bfx$.

For sequences $\left\{x_k\right\}$ in $\bbR$, we will write
$\lim_{k\to\infty} x_K=\infty$, or $x_k\to\infty$ as $k\to\infty$, if given
$M>0$ there is an integer $N$ such that $x_k\geq M$ whenever $k\geq M$.

A sequence $\left\{ \bfx_k \right\}$ in $\bbR^n$ is called a \emph{Cauchy
  sequence} if given $\varepsilon>0$ there exists an integer $N$ such that
$\|\bfx_k-\bfx_\ell\|<\varepsilon$ for all $k,\ell\geq N$. $\bbR^n$ is a
complete metric space, i.e., every Cauchy sequence in $\bbR^n$ converges to
a point of $\bbR^n$.

A set $E\subset E_1$ is said to be \emph{dense} in $E_1$ if for every
$\bfx_1\in E_1$ and $\varepsilon>0$ there is a point $\bfx\in E$ such that
$0<\|\bfx-\bfx_1\|<\varepsilon$. Thus, $E$ is dense in $E_1$ if every point
of $E_1$ is a limit point of $E$. If $E=E_1$, we say $E$ is \emph{dense in
  itself}. As an example, the set of limit points of $\bbR^n$ each of whose
coordinates is a rational number is dense in $\bbR^n$. Since this set is
also countable, it follows that $\bbR^n$ is \emph{separable}, by which we
mean that $\bbR^n$ has a countable dense subset.

For nonempty subsets $E$ of $\bbR$, we use the standard notation $\sup E$
and $\inf E$ for the \emph{supremum} (\emph{least upper bound}) and
\emph{infimum} (\emph{greatest lower bound}) of $E$. In case $\sup E$
belong to $E$, it will be called $\max E$; similarly, $\inf E$ will be
called $\min E$ if it belongs to $E$.

If $\left\{ a_k \right\}_{k=1}^\infty$ is a sequence of points in $\bbR$,
let $b_j=\sup_{k\geq j} a_k$ and $c_j=\inf_{k\geq j} a_k$,
$j=1,2,\dotsc$. Then $-\infty\leq c_j\leq b_j\leq\infty$ and $\left\{ b_j
\right\}$ and $\left\{ c_j \right\}$ are monotone decreasing and
increasing, respectively; that is, $b_j\geq b_{j+1}$ and $c_j\leq
c_{j+1}$. Define $\limsup_{k\to\infty} a_k$ and $\liminf_{k\to\infty} a_k$
by
\begin{equation}
\label{eq:limsup-liminf-e-k}
\begin{aligned}
\limsup_{k\to\infty} a_j=
\lim_{j\to\infty}b_j=
\lim_{j\to\infty}\left\{\lim_{k\geq j} a_k\right\},\\
\liminf_{k\to\infty} a_k=
\lim_{j\to\infty} C_j=
\lim_{j\to\infty}\left\{\lim_{k\geq j} a_k\right\}.
\end{aligned}
\end{equation}
\begin{theorem}[1.4]
\begin{enumerate}[label=\textnormal{(\alph*)}]
\item $L\coloneqq\limsup_{k\to\infty} a_k$ if and only if (i) there is a
  subsequence $\{a_{k_j}\}$ of $\{a_k\}$ that   converges to $L$ and (ii)
  if $L'>L$, there is an integer $N$ such that $a_k<L'$ for $k\geq N$.
\item $\ell\coloneqq\liminf_{k\to\infty} a_k$ if and only if (i) there is a
  subsequence $\{a_{k_j}\}$ of $\{a_k\}$ that converges to $\ell$ and (ii)
  if $\ell'<\ell$, there is an integer $N$ such that $a_k>\ell'$ for $k\geq
  N$.
\end{enumerate}
\end{theorem}

Thus, when they are finite, $\limsup a_k$ and $\liminf a_k$ are the
largest and smallest limit points of $\{a_k\}$, respectively.

We can also use the metric on $\bbR$ to define the \emph{diameter of a set
  $E$} by letting
\begin{equation}
  \label{eq:diameter-of-set}
\diam E\coloneqq\left\{\,\|\bfx-\bfy\|:\bfx,\bfy\in E\,\right\}.
\end{equation}
If the diameter of $E$ is finite, $E$ is said to be
\emph{bounded}. Equivalently, $E$ is bounded if there is a finite constant
$M$ such  that $\|\bfx\|\leq M$ for all $\bfx\in E$. If $E_1$ and $E_2$ are
two sets, the \emph{distance between $E_1$ and $E_2$} is defined by
\begin{equation}
  \label{eq:distance-e-1-e-2}
d(E_1,E_2)\coloneqq\inf\left\{\,\|\bfx-\bfy\|:\bfx\in E_1,\bfy\in E_2\,\right\}.
\end{equation}

For $\bfx\in\bbR^n$ and $\delta>0$, the set
\begin{equation}
\label{eq:open-ball-r-n}
B(\bfx,\delta)\coloneqq\left\{\,\bfy:\|bfx-\bfy\|<\delta\,\right\}
\end{equation}
is called the \emph{open ball with center $\bfx$ and radius $\delta$}. A
point $\bfx$ of a set $E$ is called an \emph{interior point} of $E$ if
there exists $\delta>0$ such that $B(\bfx,\delta)\subset E$. The collection
of all interior points of $E$ is called the \emph{interior} of $E$ and
denoted $E^\circ$. A set $E$ is said to be \emph{open} if $E^\circ=E$; that
is, $E$ is open if for each $\bfx\in E$ there exists $\delta>0$ such that
$B(\bfx,\delta)\subset E$. The empty set $\emptyset$ is open by
convention. The whole space $\bbR^n$ is clearly open and $B(\bfx,\delta)$
is evidently open. We will generally denote open sets by the letter $G$.

A set $E$ is called \emph{closed} if $\complement E$ is open. Note that
$\emptyset$ and $\bbR^n$ are closed. Closed sets will generally be denoted
by the letter $F$. The union of a set $E$ and all its limit points is
called the \emph{closure} of $E$ and written $\bar E$. By the
\emph{boundary} of $E$, we mean $\partial E\coloneqq \bar E\minus
E^\circ$.
\begin{theorem}[1.5]
\begin{enumerate}[label=\textnormal{(\roman*)}]
\item $\overline{B(\bfx,\delta)}=\left\{\,\bfy:\|\bfx-\bfy\|\leq\delta\, \right\}$
\item $E$ is closed if and only if $E=\bar E$; that is, $E$ is closed if
  and only if it contains all of its limit points.
\item $\bar E$ is closed, and $\bar E$ is the smallest closed set
  containing $E$; that is, $F$ is closed and $E\subset F$, then $\bar
  E\subset F$.
\end{enumerate}
\end{theorem}

(The rest of this is a bunch of theorems that can be expressed in more
generality from a more topological perspective. At any rate, they are very
basic.)

Consider a collection $\{A\}$ of sets $A$. A set is said to be of \emph{type
$A_\delta$} if it can be written as a countable intersection of sets $A$
and of \emph{type $A_\sigma$} if it can be written as a countable union of
sets $A$. The most common uses of this notation are $G_\delta$ and
$F_\sigma$, where $\{G\}$ denotes open sets in $\bbR^n$ and $\{F\}$ closed
sets. Hence, $H$ is of \emph{type $G_\delta$} if
\begin{equation}
  \label{eq:G-delta}
H=\bigcap_k G_k,\qquad \text{$G_k$ open,}
\end{equation}
and is of \emph{type $F_\sigma$} if
\begin{equation}
  \label{eq:F-sigma}
H=\bigcap_k F_k,\qquad\text{$F_k$ closed.}
\end{equation}
The complement of a $G_\delta$ set is an $F_\sigma$ and vice-a-versa.

Another type of set that we have the occasion to use is the \emph{perfect
  set}, by which we mean a closed set $C$ each of whose points is a limit
point of $C$. Thus, a perfect set is a closed set that is dense in itself.

\begin{theorem}[1.9]
A perfect set is uncountable.
\end{theorem}

Other special sets that will be important are $n$-dimensional
intervals. When $n=1$ and $a<b$, we will use the usual notations
$[a,b]\coloneqq\left\{\,x:a\leq x\leq b\,\right\}$,
$(a,b)\coloneqq\left\{\,x:a<x<b\,\right\}$, etc. Whenever we use just the
word interval, we generally mean closed interval. An \emph{$n$-dimensional
  interval $I$} is a subset of $\bbR^n$ of the form
$I\coloneqq\left\{\,\bfx=(x_1,\dotsc,x_n):\text{$a_k\leq x_k\leq b_k$,
    $k=1,\dotsc,n$}\,\right\}$, where $a_k<b_k$, $k=1,\dotsc,n$.

%%% Local Variables:
%%% mode: latex
%%% TeX-master: "../MA544-Quals"
%%% End:

% \section{Exam 2 Review}
This is all of the material we covered before exam 2.

\bigskip

Let $f$ be defined on $E$, and let $\bfx_0$ be a limit point of $E$ in
$E$. Then $f$ is said to be \emph{upper semicontinuous at $\bfx_0$} if
\begin{equation}
  \label{eq:upper-semicontinuous}
\limsup_{\substack{\bfx\to\bfx_0\\\bfx\in E}}f(\bfx)\leq f(\bfx_0).
\end{equation}
Note that if $f(\bfx_0)=\infty$, then $f$ is usc at $\bfx_0$
automatically; otherwise, the statement that $f$ is usc at $\bfx_0$ means
that given any $M>f(\bfx_0)$, there exists $\delta>0$ such that $f(\bfx)<M$
for all $\bfx\in E$ that lie in the ball $B_\delta(\bfx_0)$.

Similarly, $f$ is said to be \emph{lower semicontinuous at $\bfx_0$} if
$-f$ is usc at $\bfx_0$.

\begin{theorem*}[4.14]
A function $f$ is usc relative to $E$ if and only if $\left\{\,\bfx\in
  E:f(\bfx)>a\,\right\}$ is relatively closed (equivalently, if
$\left\{\,\bfx\in E:f(\bfx)<a\,\right\}$ is relatively open) for all finite
$a$
\end{theorem*}
\begin{proof}[Proof of theorem 4.14]
Suppose that $f$ is usc relative to $E$. Given $a$, let $\bfx_0$ be a limit
point of $\left\{\,\bfx\in E:f(\bfx)>a\,\right\}$ in $E$. Then there exists
$\bfx_k\in E$ such that $\bfx_k\to\bfx_0$ and $f(\bfx_k)\geq a$. Since $f$
is usc at $\bfx_0$, we have $f(\bfx_0)\geq\limsup_{k\to\infty}
f(\bfx_k)$. Therefore, $f(\bfx_0)\geq a$, so $\bfx_0\in\left\{\,\bfx\in
  E:f(\bfx)>a\,\right\}$. Hence, $\left\{\,\bfx\in E:f(\bfx)>a\,\right\}$
is relatively closed.

Conversely, let $\bfx_0$ be a limit point of $E$ that is in $E$. If $f$ is
not usc at $\bfx_0$, then $f(\bfx_0)<\infty$, and there exists $M$ and
$\left\{ \bfx_k \right\}$ such that $f(\bfx_0)<M$, $\bfx_k\in E$,
$\bfx_k\to\bfx_0$, and $f(\bfx_k)\geq M$. Hence, $\left\{\,\bfx\in
  E:f(\bfx)>a\,\right\}$ is not relatively closed since it does not contain
all its limit points in $E$>
\end{proof}
\begin{theorem*}[4.17, Egorov's theorem]
Suppose that $\{f_k\}$ is a sequence of measurable functions that converge
a.e.\@ in a set $E$ of finite measure to a finite limit $f$. Then given
$\varepsilon>0$ there exits a closed subset $F$ of $E$ such that $|E\smallsetminus
F|<\varepsilon$ and $f_k\to f$ uniformly on $F$.
\end{theorem*}
A function $f$ defined on a measurable set $E$ has \emph{property $\calC$}
on $E$ if given $\varepsilon>0$, there is a closed set $F\subset E$ such
that
\begin{enumerate}[label=(\roman*)]
\item $|E\smallsetminus F|<\varepsilon$
\item $f$ is continuous relative to $F$.
\end{enumerate}
\begin{theorem*}[4.20, Lusin's theorem]
Let $f$ be defined and finite on a measurable set $E$. Then $f$ is
measurable if and only if it has property $\calC$ on $E$.
\end{theorem*}

We start with a nonnegative function $f$ defined on a measurable subset $E$
of $\bfR^n$. Let's
\begin{equation}
\label{eq:graph-and-region}
\begin{aligned}
\Gamma(f,E)&\coloneqq\left\{\,(\bfx,f(\bfx))\in\bfR^{n+1}:\text{$\bfx\in
    E$, $f(\bfx)<\infty$}\,\right\},\\
R(f,E)&\coloneqq\left\{\,(\bfx,y)\in\bfR^{n+1}:\text{$\bfx\in E$, $0\leq
    y\leq f(\bfx)$ if $f(\bfx)<\infty$ and $0\leq y<\infty$ if
    $f(\bfx)=\infty$}\,\right\}.
\end{aligned}
\end{equation}
$\Gamma(f,E)$ is called the \emph{graph of $f$ over $E$} and $R(f,E)$ the
\emph{region under $f$ over $E$}.

If $R(f,E)$ is measurable (as a subset of $\bfR^{n+1}$), its measure
$|R(f,E)|_{\bfR^{n+1}}$ is called the \emph{Lebesgue integral over $E$},
and we write
\begin{equation}
\label{eq:lebesgue-integral}
\int_E f(\bfx) d \bfx\coloneqq|R(f,E)|_{\bfR^{n+1}}.
\end{equation}
This is sometimes written as
\[
\int_E f
\]
or at times the lengthy notation
\[
\idotsint\limits_{E} f(x_1,\dotsc,x_n) d  x_1\cdots d  x_n
\]
is convenient.
\begin{theorem*}[5.1]
Let $f$ be a nonnegative function defined on a measurable set $E$. Then
$\int_E f$ exists if and only if $f$ is measurable.
\end{theorem*}
\begin{lemma*}[5.3]
If $f$ is a nonnegative measurable function on $E$, $0\leq |E|\leq\infty$,
then $|\Gamma(f,E)|=0$.
\end{lemma*}
\begin{theorem*}[5.5]
\begin{enumerate}[label=\textnormal{(\roman*)}]
\item If $f$ and $g$ are measurable and if $0\leq g\leq f$ on $E$, $\int_E
  g\leq\int_E f$. In particular, $\int_E\inf f\leq\int_E f$.
\item If $f$ is nonnegative and measurable on $E$ and if $\int_E f$ is
  finite, then $f<\infty$ a.e.\@ in $E$.
\item Let $E_1$ and $E_2$ be measurable and $E_1\subset E_2$. If $f$ is
  nonnegative and measurable on $E_2$, then $\int_{E_1} f\leq\int_{E_2}f$.
\end{enumerate}
\end{theorem*}
\begin{theorem*}[5.6, the monotone convergence theorem for nonnegative functions]
If $\{f_k\}$ is a sequence of nonnegative functions such that $f_k\nearrow
f$ on $E$, then
\[
\int_E f\to\int_E f.
\]
\end{theorem*}
\begin{proof}
By Theorem 4.12, $f$ is measurable since it is the limit of a sequence of
measurable functions. Since $R(f_k,E)\cup\Gamma(f,E)\nearrow R(f,E)$ and
$|\Gamma(f,E)|=0$, the result follows by Theorem 3.26 on the measure of a
monotone convergent sequences of measurable sets.
\end{proof}
\begin{theorem*}[5.9]
Let $f$ be nonnegative on $E$. If $|E|=0$, then $\int_E f=0$.
\end{theorem*}
\begin{theorem*}[5.10]
If $f$ and $g$ are nonnegative and measurable on $E$ and if $g\leq f$
a.e.\@ in $E$, then $\int_E g\leq\int_E f$.

In particular, if $f=g$ a.e.\@ in $E$, then $\int_E f=\int_E g$.
\end{theorem*}
\begin{theorem*}[5.11]
Let $f$ be nonnegative and measurable on $E$. Then $\int_E f=0$ if and only
if $f=0$ a.e.\@ in $E$.
\end{theorem*}
\begin{corollary*}[5.12, Chebyshev's inequality]
Let $f$ be nonnegative and measurable on $E$. If $a>0$, then
\[
\frac{1}{a}\int_E f\geq\left|\left\{\,\bfx\in E:f(\bfx)>a\,\right\}\right|.
\]
\end{corollary*}
\begin{theorem*}[5.13]
If $f$ is nonnegative and measurable, and if $c$ is any nonnegative
constant, then
\[
\int_E cf=c\int_E f.
\]
\end{theorem*}
\begin{theorem*}[5.14]
If $f$ and $g$ are nonnegative and measurable, then
\[
\int_E (f+g)=\int_E f+\int_E g.
\]
\end{theorem*}
\begin{corollary*}
Suppose that $f$ and $\varphi$ are measurable on $E$, $0\leq f\leq\varphi$,
and $\int_E f$ is finite. Then
\[
\int_E (\varphi-f)=\int_E\varphi-\int_E f.
\]
\end{corollary*}
\begin{theorem*}[5.16]
If $f_k$, $k=1,2,\dotsc$, are nonnegative and measurable, then
\[
\int_E\sum_{k=1}^\infty f_k=\sum_{k=1}^\infty\int_E f_k.
\]
\end{theorem*}
\begin{theorem*}[5.17, Fatou's lemma]
If $\{f_k\}$ is a sequence of nonnegative measurable functions on $E$, then
\[
  \int_E\liminf_{k\to\infty} f_k\leq\liminf_{k\to\infty}\int_E f_k.
\]
\end{theorem*}
\begin{proof}[Proof of Fatou's lemma]
\end{proof}
\begin{theorem*}[5.19, Lebesgue's dominated convergence theorem for
  nonnegative functions]
Let $\{f_k\}$ be a sequence of nonnegative measurable functions on $E$ such
that $f_k\to f$ a.e.\@ in $E$. If there exists a measurable function
$\varphi$ such that $f_k\leq\varphi$ a.e.\@ for all $k$ and if
$\int_E\varphi$ is finite, then
\[
\int_E f_k\longrightarrow\int_E f.
\]
\end{theorem*}
\begin{theorem*}[5.21]
Let $f$ be measurable in $E$. Then $f$ is integrable over $E$ if and only
if $|f|$ is.
\end{theorem*}
\begin{theorem*}[5.22]
If $f\in L^1(E)$, then $f$ is finite a.e.\@ in $E$.
\end{theorem*}
\begin{theorem*}[5.24]
If $\int_E f$ exists and $E=\bigcup_{k\in\bfN} E_k$ is the countable union
of disjoint measurable sets $E_k$, then
\[
\int_E f=\sum_{k\in\bfN}\int_{E_k}f.
\]
\end{theorem*}
\begin{theorem*}[5.25]
If $|E|=0$ or if $f=0$ a.e.\@ in $E$, then $\int_E f=0$.
\end{theorem*}
\begin{theorem*}[5.32, monotone convergence theorem]
Let $\{f_k\}$ be a sequence of measurable functions on $E$:
\begin{enumerate}[label=\textnormal{(\roman*)}]
\item If $f_k\nearrow f$ a.e.\@ on $E$ and there exists $\varphi\in L^1(E)$ such
  that $f_k\geq\varphi$ a.e.\@ on $E$ for all $k$, then $\int_E
  f_k\to\int_E f$.
\item If $f_k\searrow f$ a.e.\@ on $E$ and there exists $\varphi\in L^1(E)$ such
  that $f_k\leq\varphi$ a.e.\@ on $E$ for all $k$, then $\int_E
  f_k\to\int_E f$.
\end{enumerate}
\end{theorem*}
\begin{theorem*}[5.33, uniform convergence theorem]
Let $f_k\in L^1(E)$ for $k\in\bfN$ and let $\{f_k\}$ converge uniformly to
$f$ on $E$, $|E|<\infty$. Then $f\in L^1(E)$ and $\int_E f_k\to\int_E f$.
\end{theorem*}
\begin{theorem*}[5.34, Fatou's lemma]
Let $\{f_k\}$ be a sequence of measurable functions on $E$. If there exists
$\varphi\in L^1(E)$ such that $f_k\geq\varphi$ a.e.\@ on $E$ for all $k$,
then
\[
\int_E\liminf_{k\to\infty} f_k\leq\liminf_{k\to\infty}\int_E f_k.
\]
\end{theorem*}
\begin{corollary*}[5.35, reverse Fatou's lemma]
Let $\{f_k\}$ be a sequence of measurable functions on $E$. If there exits
$\varphi\in L^1(E)$ such that $f_k\leq\varphi$ a.e.\@ on $E$ for all $k$,
then
\[
\int_E\limsup_{k\to\infty} f_k\geq\limsup_{k\to\infty}\int_E f_k.
\]
\end{corollary*}
\begin{theorem*}[5.36, Lebesgue's dominated convergenge theorem]
Let $\{f_k\}$ be a sequence of measurable functions on $E$ such that
$f_k\to f$ a.e.\@ in $E$. If there exists $\varphi\in L^1(E)$ such that
$|f_k|\leq\varphi$  such that $|f_k|\leq\varphi$ a.e.\@ in $E$ for all
$k\in\bfN$, then $\int_E f_k\to\int_E f$.
\end{theorem*}
\begin{corollary*}[5.37, bounded convergence theeorem]
Let $\{f_k\}$ be a sequence of measurable functions on $E$ such  that
$f_k\to f$ a.e.\@ in $E$. If $|E|<\infty$ there is a finite constant $M$
such that $|f_k|\leq M$ a.e.\@ in $E$, then $\int_E f_k\to\int_E f$.
\end{corollary*}
\begin{theorem*}[6.1 Fubini's theorem]
Let $f(\bfx,\bfy)\in L^1(I)$, $I\coloneqq I_1\times I_2$. Then
\begin{enumerate}[label=\textnormal{(\roman*)}]
\item For almost every $\bfx\in I_1$, $f(\bfx,\bfy)$ is measurable and
  integrable on $I_2$ as a function of $\bfy$;
\item As a function of $\bfx$, $\int_{I_2} f(\bfx,\bfy) d \bfy$ is
  measurable and integrable on $I_1$, and
\[
\iint_I f(\bfx,\bfy) d \bfx d \bfy=
\int_{I_1}\left[\int_{I_2}f(\bfx,\bfy) d \bfy\right]\! d \bfx.
\]
\end{enumerate}
\end{theorem*}
\begin{theorem*}[6.8]
Let $f(\bfx,\bfy)$ be a measurable function defined on a measurable subset
$E$ of $\bfR^{n+m}$, and let $E_\bfx\coloneqq\left\{ \,\bfy:(\bfx,\bfy)\in
  E \,\right\}$.
\begin{enumerate}[label=\textnormal{(\roman*)}]
\item For almost every $\bfx\in \bfR^n$, $f(\bfx,\bfy)$ is a measurable
  function of $\bfy$ on $E_\bfx$.
\item If $f(\bfx,\bfy)\in L^1(E)$, then for almost every $\bfx\in\bfR^n$,
  $f(\bfx,\bfy)$ is an integrable on $E_\bfx$ with respect to $\bfy$;
  moreover $\int_{E_\bfx}f(\bfx,\bfy) d \bfy$ is an integrable function
  of $\bfx$ and
\[
\iint_E f(\bfx,\bfy) d \bfx d \bfy=\int_{\bfR^n}\left[\int_{E_\bfx}f(\bfx,\bfy) d \bfy\right]\! d \bfx.
\]
\end{enumerate}
\end{theorem*}
\begin{theorem*}[6.10, Tonelli's theorem]
Let $f(\bfx,\bfy)$ be nonnegative and measurable on an interval
$I=I_1\times I_2$ of $\bfR^{n+m}$. Then, for almost every $\bfx\in I_1$,
$f(\bfx,\bfy)$ is a measurable function of $\bfy$ on $I_2$. Moreover, as a
function of $\bfx$, $\int_{I_2}f(\bfx,\bfy) d  \bfy$ is measurable on
$I_1$, and
\[
\iint_I f(\bfx,\bfy) d \bfx d \bfy=\int_{I_1}\left[\int_{I_2}f(\bfx,\bfy) d \bfy\right]\! d \bfx
\]
\end{theorem*}
If $f$ and $g$ are measurable in $\bfR^n$, their \emph{convolution
  $(f*g)(\bfx)$} is defined by
\[
(f*g)(\bfx)\coloneqq\int_{\bfR^n}f(\bfx-\bfy)g(\bfy) d \bfy,
\]
provided the integral exists.
\begin{theorem*}[6.14]
If $f\in L^1(\bfR^n)$ and $g\in L^1(\bfR^n)$, then $(f*g)(\bfx)$ exists for
almost every $\bfx\in\bfR^n$ and is measurable. Moreover, $f*\in
L^1(\bfR^n)$ and
\[
\begin{aligned}
\int_{\bfR^n}|f*g| d \bfx
\leq{}&\left(\int_{\bfR^n}|f| d \bfx\right)\left(\int_{\bfR^n}|g| d \bfx\right)\\
\int_{\bfR^n}(f*g)(\bfx) d \bfx
={}&\left(\int_{\bfR^n}f d \bfx\right)\left(\int_{\bfR^n}g d \bfx\right).
\end{aligned}
\]
\end{theorem*}
\begin{corollary*}[6.16]
If $f$ and $g$ are nonnegative and measurable on $\bfR^n$, then $f*g$ is
measurable on $\bfR^n$ and
\[
\int_{\bfR^n}(f*g) d \bfx=
\left(\int_{\bfR^n} f d \bfx\right)
\left(\int_{\bfR^n}g d \bfx\right).
\]
\end{corollary*}
\begin{theorem*}[6.17, Marcinkiewicz]
Let $F$ be a closed subset of a bounded open interval $(a,b)$, and let
$\delta(x)\coloneqq\delta(x,F)$ be the corresponding distance
function. Then, given $\lambda>0$, the integral
\[
M_\lambda(x)\coloneqq\int_a^b\frac{\delta(y)^\lambda}{|x-y|^{1+\lambda}} d  y
\]
is finite a.e.\@ in $F$. Moreover, $M_\lambda\in L^1(F)$ and
\[
\int_F M_\lambda d  x\leq 2\lambda^{-1}|G|,
\]
where $G\coloneqq(a,b)\smallsetminus F$.
\end{theorem*}

%%% Local Variables:
%%% mode: latex
%%% TeX-master: "../MA544-Quals"
%%% End:

% \section{Final Exam Review}
Material covered since exam 2.

\bigskip

If $f$ is a Riemann integrable function on an interval $[a,b]$ in $\bfR$,
then the familiar definition of its indefinite integral is
\[
F(x)=\int_a^x f(y)dy,\qquad a\leq x\leq b.
\]
The fundamental theorem of calculus asserts that $F'=f$ if $f$ is
continuous. We will study an analogue of this result for Lebesgue
integrable $f$ and higher dimensions.

We must first find an appropriate definition of the indefinite integral. In
two dimensions, for example, we might choose
\[
F(x_1,x_2)=\int_{a_1}^{x_1}\int_{a_2}^{x_2}f(y_1,y_2)dy_1dy_2.
\]
It turns out, however, to be better to abandon the notion that the
indefinite integral be a function of point and adopt the idea that it be a
function of set. Thus, given $f\in L(A)$, where $A$ is a measurable
subset of $\bfR^n$, we define the \emph{indefinite integral of $f$} to be
the function
\[
F(E)=\int_E f,
\]
where $E$ is any measurable subset of $A$.

$F$ is an example of a \emph{set function}, by which we mean any
real-valued function $F$ defined on a $\sigma$-algebra $\Sigma$ of
measurable sets such that
\begin{enumerate}[label=(\roman*)]
\item $F(E)$ is finite for every $E\in\Sigma$.
\item $F$ is \emph{countably additive}; that is, if $E=\bigcup_k E_k$ is a
  union of disjoint $E_k\in\Sigma$, then
\[
F(E)=\sum_k F(E_k).
\]
\end{enumerate}
By Theorem 5.5 and 5.24, the indefinite integral of $f\in L(A)$ satisfies
(i) and (ii) for the $\sigma$-algebra of measurable subsets of $A$.

Recall that the diameter of a set $E$ is the value
\[
\sup\left\{\,\|\bfx-\bfy\|:\bfx,\bfy\in E\,\right\}.
\]
A set function $F(E)$ is called \emph{continuous} if $F(E)$ tends to zero
as the diameter of $E$ tends to zero; i.e., $F(E)$ is continuous if, given
$\varepsilon>0$, there exists a $\delta>0$ such that $|F(E)|<\varepsilon$
whenever the diameter of $E$ is less than $\delta$. An example of a
function that is \emph{not} continuous can be obtained by setting $F(E)=1$
for any measurable set that contains the origin, and $F(E)=0$
otherwise.\footnote{Why is this function not continuous. Consider the
  following argument: Let $\varepsilon=1/2$ and let $B_k=
  B(\mathbf{0},1/k)$. Then as the diameter of $B_k$ goes to zero,
  $F(B_k)=1$ for all $k$ so $F(B_k)\to 1>1/2$.}

A set function $F$ is called \emph{absolutely continuous with respect to
  the Lebesgue measure}, or simply \emph{absolutely continuous} if $F(E)$
tends to zero as the measure of $E$ tends to zero. Thus, $F$ is absolutely
continuous if given a $\varepsilon>0$ there exists $\delta>0$ such that
$|F(E)|<\varepsilon$ whenever the measure of $E$ is less than $\delta$.

A set function that is absolutely continuous is clearly
continuous\footnote{Suppose $F$ is absolutely continuous. Then, given
  $\varepsilon>0$ there exists $\delta>0$ such that $|F(E)|<\varepsilon$
  whenever $|E|<\delta$.}, however, the converse is false, as shown in the
following example. Let $A$ be the unit square in $\bfR^2$, let $D$ be the
diagonal of $A$, and consider the $\sigma$-algebra of measurable subsets
$E$ of $A$ for which $E\cap D$ is linearly measurable. For such $E$, let
$F(E)$ be the linear measure of $E\cap D$. Then $F$ is a continuous set
function. However, it is not absolutely continuous since the sets $E$
containing a fixed segment of $D$ whose $\bfR^2$-measures are arbitrarily
small.

\begin{theorem}[7.1]
If $f\in L(A)$, its definite integral is absolutely continuous.
\end{theorem}
\begin{proof}
We may assume that $f\geq 0$ by considering $f^+$ and $f^-$. Fix $k$ and
write $f=g+h$, where $g=f$ whenever $f\leq k$ and $g=k$ otherwise. Given
$\varepsilon>0$, we may choose $k$ so large that $0\leq\int_A
h<\varepsilon/2$ and, \emph{a fortiori}, $0\leq\int_E f<\varepsilon/2$.
Since
\[
\int|f-C|\leq\int|f-f_{k_0}|+\int|f_{k_0}-C|<\frac{\varepsilon}{2}+\frac{\varepsilon}{2}=\varepsilon.
\]
we see that $f$ has property $\calA$.

To prove the lemma, let $f\in L(\bfR)$. Writing $f=f^+-f^-$, we may assume
that $f\geq 0$. Then
\[
\int|\chi_G-\chi_E|=|G\smallsetminus E|<\varepsilon.
\]
so we may assume that $f=\chi_G$ for open set $G$ of finite measure. Using
Theorem 1.11, write $G$ as the union of (partly open) disjoint intervals
$G=\bigcup I_k$. If we let $f_N$ be the characteristic function of
$\bigcup_{k=1}^N I_k$, we obtain
\[
\int|f-f_N|=\sum_{k=N+1}^\infty|I_k|\to 0
\]
since $\sum_k|I_k|=|G|<\infty$, i.e., the series converges. By (2), it is
enough to show that each $f_N$ has property $\calA$. But $f_N$ is the sum
$\chi_{I_k}$, $k=1,\dotsc,N$, so it suffices by (1) to show that the
characteristic function of any partly open interval $I$ has property
$\calA$. This is practically self-evident: if $I^*$ denotes an interval
that contains the closure of $I$ in its interior and that satisfies
$|I^*\smallsetminus I|<\varepsilon$, then for any continuous $C$, $0\leq
C\leq 1$, which is $1$ in $I$ and $0$ outside $I^*$, we have
\[
\int|chi_I-C|\leq|I^*-I|<\varepsilon.
\]
\end{proof}

\begin{theorem}[Simple Vitali lemma]
Let $E$ be a subset of $\bfR^n$ with $|E|_e<\infty$, and let $K$ be a
collection of cubes $Q$ covering $E$, then there exists a positive constant
$\beta$, depending only on $n$, and a finite number of disjoint cubes,
$Q_1,\dotsc,Q_N$ in $K$ such that
\[
\sum_{j=1}^N|Q_j|\geq\beta|E|_e
\]
\end{theorem}

%%% Local Variables:
%%% mode: latex
%%% TeX-master: "../MA544-Quals"
%%% End:

\chapter{Course Notes}
These notes roughly correspond to chapters 2 through 8 of Wheeden and
Zygmund's \emph{Measure and Integration}
\cite{wheeden-zygmund:measure-and-integral}.

\section{Functions of bounded variation and the Riemann--Stieltjes
  integral}
In this section, we introduce functions of bounded variation as well as the
definition of the Riemann integral. We conclude with a proof that the

\subsection{Functions of bounded variation}
Let $f\colon[a,b]\to\bbR$ be a real-valued function defined for all $a\leq
x\leq b$ and finite; let $\Gamma=\left\{x_0,\dotsc,x_m\right\}$ be a
\href{https://en.wikipedia.org/wiki/Partition_of_an_interval}{\emph{partition}}
of $[a,b]$, i.e., a collection of points $x_i$, $i=0,\dotsc,m$, satisfying
$x_0=a$ and $x_m=b$, and $x_{i-1}<x_i$ for $i=1,\dotsc,m$. To each
partition $\Gamma$, we associated a sum
\begin{equation}
\label{eq:bv:sum}
S_\Gamma\coloneqq S_\Gamma[f;a,b]\coloneqq\sum_{i=1}^m\left|f(x_i)-f(x_{i-1})\right|.
\end{equation}
The
\href{https://en.wikipedia.org/wiki/Bounded_variation#Formal_definition}{\emph{variation}}
(or \emph{total variation}) \emph{of $f$ over $[a,b]$} is defined as
\begin{equation}
  \label{eq:bv:variation}
V\coloneqq V[f;a,b]\coloneqq\sup_\Gamma S_\Gamma,
\end{equation}
where the supremum is taken over all partitions $\Gamma$ of $[a,b]$. If
$V<\infty$, $f$ is said to be of
\href{https://en.wikipedia.org/wiki/Bounded_variation}{\emph{bounded
    variation}} \emph{on $[a,b]$}; if $V=\infty$, $f$ is of \emph{unbounded
variation on $[a,b]$}.

Before going on to prove important properties about
\eqref{eq:bv:variation}, let us look at some common examples (and
nonexamples) of functions $f$ of bounded variation.

\begin{example}
Suppose $f$ is
\href{https://en.wikipedia.org/wiki/Monotonic_function}{\emph{monotone}} in
$[a,b]$. Then, for an arbitrary partition, $\Gamma=\{x_0,\dotsc,x_m\}$ of
$[a,b]$ we have
\begin{align*}
S_\Gamma&=\sum_{i=1}^m|f(x_{i-1})-f(x_i)|\\
        &=|f(a)-f(x_1)|+|f(x_2)-f(x_1)|+\dotsb\\
        &\phantom{{}={}}+|f(x_{m-1})-f(x_{m-2})|+|f(x_m)-f(x_{m-1})|\\
        &=
\end{align*}
\end{example}

%%% Local Variables:
%%% mode: latex
%%% TeX-master: "../MA544-Quals"
%%% End:

\section{Danielli}
\subsection{Danielli: Practice Exams Spring 2016}
\setcounter{exercise}{0}
\setcounter{equation}{0}

\subsubsection{Exam 1 Practice}
\begin{problem}
  Let $E\subset\bbR^n$ be a measurable set, $r\in\bbR$ and define the set
  $rE=\left\{\,rx : x\in E\,\right\}$. Prove that $rE$ is measurable, and
  that $|rE|=|r|^n|E|$.
\end{problem}
\begin{solution}
  Define a map a linear map $T\colon\bbR^n\to\bbR^n$ by $T(x)=rx$. Since a
  the image of a measurable set $E$ under linear map is measurable and
  $m(T(E))=|{\det T}|m(E)=|r|^nm(E)$, it suffices to show that $T(E)=rE$.

  Let $y\in T(E)$ then $y=rx$ for some $x\in E$. Thus, $y\in rE$. Let $y\in
  rE$. Then, $y=rx=T(x)$ for some $x\in E$. Thus, $y\in T(E)$. It follows
  that $m(rE)=|r|^nm(E)$.
\end{solution}

\begin{problem}
  Let $\{E_k\}$, $k\in\bbN$ be a collection of measurable sets. Define the
  set
  \[
    \liminf_{k\to\infty} E_k
    =\bigcup_{k=1}^\infty\left(\bigcap_{n=k}^\infty E_n\right).
  \]
  Show that
  \[
    m\left(\liminf_{k\to\infty}
      E_k\right)\leq\liminf_{k\to\infty}m(E_k).
  \]
\end{problem}
% Following the style of \cite[Ch.\@ 1, p.\@ 2]{wheeden-zygmund},
  % particularly, the sets defined after the introduction of equation (1.1),
  % set
  % \begin{equation}
  %   \label{eq:prep:1:2}
  %   V_k=\bigcap_{\ell=k}^\infty E_\ell.
  % \end{equation}
  % Note that the collection of sets $\{V_k\}$ forms an increasing sequence,
  % that is, if $x\in V_k$ then, by \eqref{eq:prep:1:2}, $x $ is in the
  % intersection $E_k\cap\bigl(\bigcap_{\ell=k+1}E_\ell\bigr)$, but, by
  % \eqref{eq:prep:1:2}, $\bigcap_{\ell=k+1}E_\ell=V_{k+1}$ thus, $x $ is
  % in $V_{k+1}$ so $V_{k+1}\supset V_k$. Hence, we have
  % $V_k\nearrow\liminf E_k$.

  % Now, consider the sequence $\{|V_k|\}$ formed by the Lebesgue measure of
  % the $V_k$. By Theorem 3.26 from \cite[Ch.\@ 3, p.\@ 51]{wheeden-zygmund},
  % since $V_k\nearrow\liminf E_k$,
  % \begin{equation}
  %   \label{eq:prep:1:3}
  %   \lim_{k\to\infty}|V_k|=
  %   \lim_{k\to\infty}\left|\bigcap_{\ell=k}^\infty E_\ell\right|=
  %   \left|\liminf_{k\to\infty} E_k\right|.
  % \end{equation}
  % But note that, by the monotonicity of the Lebesgue measure, we have
  % \begin{equation}
  %   \label{eq:prep:1:4}
  %   \left|\bigcap_{\ell=k}^\infty E_\ell\right|\leq |E_k|,
  % \end{equation}
  % so, by properties of the $\liminf$, in particular, by Theorem 19(v) from
  % \cite[Ch.\@ 1, p.\@ 23]{royden}, we have
  % \begin{equation}
  %   \label{eq:prep:1:5}
  %   \limsup_{k\to\infty}|V_k|\leq\liminf_{k\to\infty}|E_k|.
  % \end{equation}
  % Hence, by \eqref{eq:prep:1:3} and Proposition 19 (iv), since the sequence
  % $\{|V_k|\}$ converges and is equal to the measure of $\liminf E_k$, by
  % \eqref{eq:prep:1:5}, we have
  % \begin{equation}
  %   \label{eq:prep:1:6}
  %   \left|\liminf_{k\to\infty} E_k\right|\leq\liminf_{k\to\infty}|E_k|
  % \end{equation}
  % as was to be shown.
\begin{solution}
  Here's a quick and dirty way of proving this: let $\chi_{E_n}$ be the
  characteristic function of $E_n$. Then, by Fatou's lemma,
  \begin{equation}
    \label{eq:ex-1-prep:fatou}
    \int\liminf_{n\to\infty}\chi_{E_n}(x)\diff x
    \leq\liminf_{n\to\infty}\int\chi_{E_n}(x)\diff x.
  \end{equation}
  By definition of the characteristic function, it is easy to see that the
  right hand-side of the Equation \eqref{eq:ex-1-prep:fatou} is
  \[
    \liminf_{k\to\infty}m(E_k).
  \]
  But what about the left-hand side of \eqref{eq:ex-1-prep:fatou}? We claim
  that
  \[
    \liminf_{n\to\infty}\chi_{E_n}=\chi_{E}
  \]
  where $E=\liminf_{n\to\infty} E_n$.
  \begin{quote}
    \begin{proof}[Proof of claim]
      Let $x\in E$. We must show that
      $\liminf_{n\to\infty}\chi_{E_n}(x)=1$. By definition
      \[
        \liminf_{n\to\infty}\chi_{E_n}=%
        \lim_{n\to\infty}\left[\inf_{k\geq n}\chi_{E_k}\right].
      \]
      Now
    \end{proof}
  \end{quote}

  Define
  \[
    V_n=\bigcap_{k=n}^\infty E_k.
  \]
  Note that ${\{V_n\}}_{n=1}^\infty$ forms an increasing sequence of
\end{solution}

\begin{problem}
  Consider the function
  \[
    F(x)=
    \begin{cases}
      |B(\mathbf{0},x)|&x>0\\
      0&x=0
    \end{cases}.
  \]
  Here $B(\mathbf{0},r)=\left\{\, y \in\bbR^n:| y |<r\,\right\}$. Prove
  that $F$ is monotonic increasing and continuous.
\end{problem}
\begin{solution}
  Define the linear map $T\colon[0,\infty)\times\bbR^n\to\bbR^n$ by
  $T(r) x = rx $. We claim that $B(\mathbf{0},r)=T(r,B(\mathbf{0},1))$. To
  reduce notation, set $B_1= B(\mathbf{0},1)$ and $B_r= B(\mathbf{0},r)$.
  \begin{quote}
    \begin{proof}[Proof of claim]
      Let $x\in B_r$. Then $|x |<r$ so $|x |/r<1$. Thus, $|x |/r\in B_1$ so
      it is in the image of $B_1$ under the map $T(r,\cdot)$.

      On the other hand, suppose $x\in T(r,B_1)$. Then $x =r y $ for some
      $ y \in B_1$. Then, since $| y |<1$, $|x |=r| y |<r$ so $x\in B_r$.
    \end{proof}
  \end{quote}

  From the claim, we see that $F(x)=|T(x,B(\mathbf{0},1))|$ which, by
  Problem 1, is nothing more that the polynomial $|B_1|x^n$. It is clear,
  from this equivalence, that $F$ is monotonically increasing: Take
  $x,y\in[0,\infty)$ such that $x<y$, then $x^n<y^n$ so
  \begin{equation}
    \label{eq:prep:1:7}
    F(x)=|B_1|x^n<|B_1|y^n=F(y).
  \end{equation}
  Thus, $F$ is monotonically increasing.

  In the argument above, since $F(x)=|B_1|x^n$ is a polynomial in
  $[0,\infty)$ (and polynomials are continuous on $\bbR$) $F$ is continuous
  on $[0,\infty)$.
\end{solution}

\begin{problem}
  Let $f\colon\bbR\to\bbR$ be a function. Let $C$ be the set of all points
  at which $f$ is continuous. Show that $C$ is a set of type $G_\delta$.
\end{problem}
\begin{solution}
  (Without much motivation) let us consider the collection of sets
  $\{E_k\}$ defined by
  \begin{equation}
    \label{eq:prep:1:8}
    E_k=\left\{\,x\in\bbR:
      \text{there exists $\delta>0$ such that $y,z\in B(x,\delta)$ implies $\left|f(y)-f(z)\right|<\frac{1}{k}$}\,\right\}.
  \end{equation}
  We claim that $C=\bigcap_{k=1}^\infty E_k$ and that each $E_k$ is open.
  \begin{solution}[Proof of claim]
    First, we demonstrate equality. $\subset$: Suppose $x\in C$. Then, by
    the definition of continuity, for every $\varepsilon>0$, there exists a
    $\delta>0$ such that $y\in B(x,\delta)$ implies
    $|f(x)-f(y)|<\delta$. In particular, for every $k$, there exists
    $\delta>0$ such that for $y\in B(x,\delta)$ the inequality
    $|f(x)-f(y)|<1/k$ holds. Thus, $x$ is in $\bigcap_{k=1}^\infty E_k$.

    $\supset$: On the other hand, suppose that
    $x\in\bigcap_{k=1}^\infty E_k$. Then, given $\varepsilon>0$, by the
    Archimedean property, there exists a positive integer $N$ such that
    $1/N<\varepsilon$. Then, since $x\in\bigcap_{k=1}^\infty E_k$,
    $x\in E_N$ so
    \begin{equation}
      \label{eq:prep:1:9}
      |f(x)-f(y)|<\frac{1}{N}<\varepsilon.
    \end{equation}
    Thus, $x$ is in $C$ and $C=\bigcap_{k=1}^\infty E_k$.

    All that remains to be shown is that the $E_k$ are open. But this is
    clear by the way we defined $E_k$ in \eqref{eq:prep:1:8}: Let
    $x\in E_k$, then there exists $\delta>0$ such that for any
    $y,z\in B(x,\delta)$, $|f(y)-f(z)|<1/k$; Let $x'\in B(x,\delta)$ and
    set $\delta'=\min\{|(x+\delta)-x'|,|(x-\delta)-x|\}$. Then, since
    $B(x',\delta')\subset B(x,\delta)$, for every $y,z\in B(x',\delta')$,
    we have $|f(y)-f(z)|<1/k$. Hence, $x'\in E_k$ for any
    $x'\in B(x,\delta)$ so $B(x,\delta)\subset E_k$.
  \end{solution}
  Since $C$ can be expressed as the countable intersection of open sets
  $E_k$, it follows that $C$ is a $G_\delta$ set.
\end{solution}
\begin{problem}
  Let $f\colon\bbR\to\bbR$ be a function. Is it true that if the sets
  $\left\{\,f=r\,\right\}$ are measurable for all $r\in\bbR$, then $f$ is
  measurable?
\end{problem}
\begin{solution}
  If $\left\{\,f=r\,\right\}$ are measurable for all $r\in\bbR$, it is not
  necessarily the case that $f$ is measurable. Consider the following
  construction: Let $E\subset(0,1)$ be an unmeasurable set.\footnote{It's
    construction does not concern us. The interested reader such direct
    their refer to Theorem 3.38 from \cite[Ch.\@ 3, p.\@
    57-58]{wheeden-zygmund} or Theorem 17 from \cite[Ch.\@ 2\S 7, p.\@
    48]{royden}.} Define a map $f\colon\bbR\to\bbR$ by
  \begin{equation}
    \label{eq:prep:1:11}
    f(x)=
    \begin{cases}
      x&\text{if $x\in\bbR\setminus((0,1)\setminus E)$},\\
      x+1&\text{if $x\in (0,1)\setminus E$.}
    \end{cases}
  \end{equation}
  By the definition, it is clear that $\left\{\,f=r\,\right\}$ is
  measurable and $\left|\left\{\,f=r\,\right\}\right|=0$ since
  $\{\,f=r\,\}$ contains at most two elements. However, the set
  $\left\{\,0<f<1\,\right\}=E$ is not measurable. Thus, $f$ is not
  measurable.
\end{solution}

\begin{problem}
  Let $\left\{f_k\right\}_{k=1}^\infty$ be a sequence of measurable
  functions on $\bbR$. Prove that the set
  $\left\{\,x:\text{$\lim_{k\to\infty} f_k(x)$ exists}\,\right\}$ is
  measurable.
\end{problem}
\begin{solution}
  By Theorem 4.12 from \cite[Ch.\@ 4, p.\@ 67]{wheeden-zygmund},
  $\liminf_{k\to\infty}f_k$ and $\limsup_{k\to\infty}f_k$ are
  measurable. By Theorem 4.7 from \cite[Ch.\@ 4, p.\@ 66]{wheeden-zygmund}
  \begin{equation}
    \label{eq:prep:1:12}
    \left\{\,\liminf_{k\to\infty} f_k<\limsup_{k\to\infty} f_k\,\right\}
  \end{equation}
  is measurable. Since
  \begin{equation}
    \label{eq:prep:1:13}
    \left\{\,\text{$\lim_{k\to\infty}f_k$ exists}\,\right\}=
    \left\{\,{\limsup_{k\to\infty}f_k=\liminf_{k\to\infty}f_k}\,\right\}=
    \bbR\setminus
    \left\{\,{\liminf_{k\to\infty} f_k<\limsup_{k\to\infty} f_k}\,\right\},
  \end{equation}
  by Theorem 3.17 from \cite[Ch.\@ 3, p.\@ 48]{wheeden-zygmund}, the set
  $\left\{\,\text{$\lim_{k\to\infty}f_k$ exists}\,\right\}$ is measurable.
  % In a fashion similar to that of Problem 4, consider the set collection
  % of sets $\{E_k\}$ given by
  % \begin{equation}
  %   \label{eq:prep:1:11}
  %       %   E_k= \left\{\, x\in\bbR:\text{there exists $N$ such that $m,n\geq N$
  %     implies $\left|f_n(x)-f_m(x)\right|<\frac{1}{k}$} \,\right\}.
  % \end{equation}
  % You can show that the $E_k$ are open and that
  % $\left\{\,x:\text{$\lim_{x\to\infty}f_k(x)$
  %   exists}\,\right\}=\bigcap_{k=1}^\infty E_k$. Then, since open sets
  % are measurable and, by Theorem 3.18 from \cite[Ch.\@ 3, p.\@
  % 48]{wheeden-zygmund}, the countable intersection of measurable sets is
  % measurable, $\left\{\,x:\text{$\lim_{x\to\infty}f_k(x)$
  %   exists}\,\right\}$ is measuable.
\end{solution}
\begin{problem}
  A real valued function $f$ on an interval $[a,b]$ is said to be
  \emph{absolutely continuous} if for every $\varepsilon>0$, there exists a
  $\delta>0$ such that for every finite disjoint collection
  $\left\{(a_k,b_k)\right\}_{k=1}^N$ of open intervals in $(a,b)$
  satisfying $\sum_{k=1}^Nb_k-a_k<\delta$, one has
  $\sum_{k=1}^N\left|f(b_k)-f(a_k)\right|<\varepsilon$. Show that an
  absolutely continuous function on $[a,b]$ is of bounded variation on
  $[a,b]$.
\end{problem}
\begin{solution}
  Suppose $f$ is absolutely continuous on $[a,b]$. Let $\varepsilon=
  1$. Then, there exists $\delta>0$ such that for every finite disjoint
  collection $\left\{(a_k,b_k)\right\}_{k=1}^N$ of open intervals in
  $(a,b)$ satisfying $\sum_{k=1}^Nb_k-a_k<\delta$, one has
  $\sum_{k=1}^N\left|f(b_k)-f(a_k)\right|<1$. Let
  $N=\lceil(b-a)/\delta\rceil$, that is, $N$ is the smallest integer
  greater than $(b-a)/\delta$, and consider the partition $\Gamma=\{x_k\}$
  where $x_k= a+k(b-a)/N$, for $k=0,\dotsc,N$. Then
  $x_k-x_{k-1}<(b-a)/N<\delta$ so, by Theorem 2.2(i) from \cite[Ch.\@ 2,
  p.\@ 19]{wheeden-zygmund}, we have $V[f;x_{k-1},x_k]<1$ for
  $k=0,\dotsc,N$. In follows by Theorem 2.2(ii) that
  \begin{equation}
    \label{eq:prep:1:14}
    V[f;a,b]=\sum_{k=1}^N V[f;x_{k-1},x_k]<N.
  \end{equation}
  Thus, $f$ is b.v.\@ on $[a,b]$.
\end{solution}

\begin{problem}
  Let $f$ be a continuous function from $[a,b]$ into $\bbR$. Let
  $\chi_{\{c\}}$ be the characteristic function of a singleton
  $\left\{c\right\}$, that is, $\chi_{\{c\}}(x)=0$ if $x\neq c$ and
  $\chi_{\{c\}}(c)=1$. Show that
  \[
    \int_a^b f d \chi_{\{c\}}=
    \begin{cases}
      0&\text{if $c\in(a,b)$,}\\
      -f(a)&\text{if $c=a$,}\\
      f(b)&\text{if $c=b$.}
    \end{cases}
  \]
\end{problem}
\begin{solution}
  The result follows quite easily from more sophisticated measure theoretic
  arguments. At this point, however, such language has not been discussed
  so we shall prove this using nothing but the definition of the
  Riemann--Stieltjes integral and properties thereof.

  Let us consider each case $c\in(a,b)$, $c=a$, and $c=b$ separately.

  Recall that the given a partition $\Gamma=\{x_0,\dotsc,x_m\}$ of $[a,b]$,
  the Riemann--Stieltjes sum of $f$ with respect to $\varphi$ is
  \begin{equation}
    \label{eq:prep:1:15}
    R_\Gamma=\sum_{k=1}^mf(\xi_k)[\varphi(x_k)-\varphi(x_{k-1})].
  \end{equation}
  The Riemann--Stieltjes integral is defined as the limit
  \begin{equation}
    \label{eq:prep:1:16}
    \int_a^b f\diff\varphi=\lim_{|\Gamma|\to 0} R_\Gamma
  \end{equation}
  if it exists.

  Suppose $c\in(a,b)$. Then, for any partition $\Gamma$ of $[a,b]$, either
  $c\in\Gamma$ or $c\notin\Gamma$. In the latter case, $R_\Gamma=0$. In the
  former case $c$ is one of the $x_k$, say $c=x_\ell$ for $0<\ell<m$. Then
  \begin{equation}
    \label{eq:prep:1:17}
    \begin{aligned}
      R_\Gamma&=\sum_{k=1}^mf(\xi_k)[\chi_{\{c\}}(x_k)-\chi_{\{c\}}(x_{k-1})]\\
      &=0+\dotsb+0+f(\xi_{\ell-1})-f(\xi_\ell)+0+\dotsb+0\\
      &=f(\xi_{\ell-1})-f(\xi_\ell).
    \end{aligned}
  \end{equation}
  Since $f$ is continuous, given $\varepsilon>0$ there exists $\delta>0$
  such that $|\xi_\ell-\xi_{\ell-1}|<\delta$ implies
  $|f(\xi_{\ell})-f(\xi_{\ell-1})|<\varepsilon$. It follows that the
  quantity in \eqref{eq:prep:1:17} approaches $0$ as $|\Gamma|$ approaches
  $0$. Therefore, $\int_a^b f\diff\chi_{\{c\}}=0$.

  Suppose $c=a$. Then, since any partition $\Gamma$ of $[a,b]$ must contain
  the point $a$, we have
  \begin{equation}
    \label{eq:prep:1:18}
    \begin{aligned}
      R_\Gamma
      &=\sum_{k=1}^mf(\chi_k)[\chi_{\{c\}}(x_k)-\chi_{\{c\}}(x_{k-1})]\\
      &
      \begin{aligned}
        =f(\xi_1)[\chi_{\{c\}}(x_1)-\chi_{\{c\}}(x_0)]&
        +f(\xi_2)[\chi_{\{c\}}(x_2)-\chi_{\{c\}}(x_1)]\\
        &+\dotsb+f(\xi_m)[\chi_{\{c\}}(x_m)-\chi_{\{c\}}(x_{m-1})]
      \end{aligned}\\
      &=-f(\xi_1)+0+\dotsb+0\\
      &=-f(\xi_1)
    \end{aligned}
  \end{equation}
  Taking the limit as $|\Gamma|\to 0$, $\xi_1\to a$ so, by continuity of
  $f$, $f(\xi_1)\to f(a)$. Thus, $\int_a^b f\diff\chi_{\{c\}}=-f(a)$.

  A similar argument to the one above shows that, if $c=b$, the
  Riemann--Stieltjes integral $\int_a^bf\diff\chi_{\{c\}}=f(b)$.
\end{solution}

%%% Local Variables:
%%% mode: latex
%%% TeX-master: "../MA544-Quals"
%%% End:

\subsubsection{Exam 1}
\setcounter{exercise}{0}
\setcounter{equation}{0}

I lost this exam. These are the questions I could recall explicitly. For
the first problem, we were asked to show that the Dichlet function
\(\indicate_\bbQ(x)\) is not Riemann integrable and prove something about
\(\bbQ\). For the second question, we were asked to show that the measure
of countable union of disjoint measurable sets \(\{E_n:n\in\bbN\}\), is
equal to the sum of their individual measures (or something to that
effect).
\begin{problem}
\end{problem}
% \begin{solution}
% \end{solution}

\begin{problem}
\end{problem}
% \begin{solution}
% \end{solution}

\begin{problem}
\hfill
\begin{enumerate}[label=(\roman*),noitemsep]
\item Show that if \(B_r=\left\{\,x\in\bbR^n:|x|<r\,\right\}\), then there
  exists a constant \(C\) such that \(|B_r|=Cr^n\).
  \\\\
  (\emph{Hint}: Think of \(B_r\) as \(\left\{\,rx:x\in B_1\,\right\}\).)
\item Let \(E\subseteq\bbR^n\) be a measurable set and let
  \(\varphi_E\colon\bbR^n\to\bbR\) be defined
  \(\varphi_E(x)=\bigl|E\cap B_{|x|}\bigr|\). Use part (i) to prove that
  \(\varphi_E\) is continuous.
\end{enumerate}
\end{problem}
\begin{solution}
\end{solution}

\begin{problem}
  Assume that \(f\colon[a,b]\to\bbR\) is of bounded variation on
  \([a,b]\). Prove that \(f\) is measurable.
\end{problem}
\begin{solution}
\end{solution}

%%% Local Variables:
%%% mode: latex
%%% TeX-master: "../MA544-Quals"
%%% End:

\subsubsection{Exam 2 Practice Problems}
\setcounter{exercise}{0}
\setcounter{equation}{0}

\begin{problem}
  Define for \( x \in\bbR^n\),
  \[
    f(x)=
    \begin{cases}
      |x|^{-(n+1)}&\text{if \(x\neq\mathbf{0}\),}\\
      0&\text{if \(x=\mathbf{0}\).}
    \end{cases}
  \]
  Prove that \(f\) is integrable outside any ball
  \(B(\mathbf{0},\varepsilon)\), and that there exists a constant \(C>0\)
  such that
  \[
    \int\limits_{\bbR^n\setminus B(\mathbf{0},\varepsilon)}f(x)\diff x
    \leq\frac{C}{\varepsilon}.
  \]
\end{problem}
\begin{solution}
  Danielli gave a wonderful solution to this problem by using spherical
  coordinates to compute the integral. However, she did not justify the use
  of polar coordinates, or even make it clear what exactly the meaning of
  \(\rmd x\) and \(rmd \sigma\) mean in this context. Here, we shed some
  light into this method and prove the polar decomposition of the Lebesgue
  measure.

  First, note that, given \(\varepsilon>0\), \(f(x)\neq 0\) for any
  \(x\in\bbR^n\setminus B_r(\mathbf{0})\). Now, define the map
  \(\Phi\colon\bbR^n\setminus\{\mathbf{0}\}\to(0,\infty)\times S^{n-1}\) by
  the rule \(\Phi(x)=(\|x\|,x/\|x\|)\). This map is smooth with a smooth
  inverse \(\Phi^{-1}(r,y)=ry\) and Jacobian
  \(\partial\Phi(x_1,\dotsc,x_n)/\partial(r,x)=\)
\end{solution}

\begin{problem}
  Let \(\left\{f_k\right\}\) be a sequence of nonnegative measurable
  functions on \(\bbR^n\), and assume that \(f_k\) converges pointwise
  almost everywhere to a function \(f\). If
  \[
    \int_{\bbR^n} f=\lim_{k\to\infty}\int_{\bbR^n} f_k<\infty,
  \]
  show that
  \[
    \int_E f=\lim_{k\to\infty}\int_E f_k
  \]
  for all measurable subsets \(E\) of \(\bbR^n\). Moreover, show that this
  is not necessarily true if
  \(\int_{\bbR^n} f=\lim_{k\to\infty} f_k=\infty\).
\end{problem}
\begin{solution}
\end{solution}

\begin{problem}
  Assume that \(E\) is a measurable set of \(\bbR^n\), with
  \(|E|<\infty\). Prove that a nonnegative function \(f\) defined on \(E\)
  is integrable if and only if
  \[
    \sum_{k=0}^\infty\left|\left\{\, x \in E:f( x )\geq
        k\,\right\}\right|<\infty.
  \]
\end{problem}
\begin{solution}
\end{solution}

\begin{problem}
  Suppose that \(E\) is a measurable subset of \(\bbR^n\), with
  \(|E|<\infty\). If \(f\) and \(g\) are measurable functions on \(E\),
  define
  \[
    \rho(f,g)=\int_E\frac{|f-g|}{1+|f-g|}.
  \]
  Prove that \(\rho(f_k,f)\to 0\) as \(k\to\infty\) if and only if \(f_k\)
  converges to \(f\) as \(k\to\infty\).
\end{problem}
\begin{solution}
\end{solution}

\begin{problem}
  Define the \emph{gamma function} \(\Gamma\colon\bbR^+\to\bbR\) by
  \[
    \Gamma(y)=\int_0^\infty e^{-u}u^{y-1}\diff u,
  \]
  and the \emph{beta function} \(\beta\colon\bbR^+\times\bbR^+\to\bbR\) by
  \[
    \beta(x,y)=\int_0^1 t^{x-1}(1-t)^{y-1}\diff t.
  \]
  \begin{enumerate}[label=(\alph*)]
  \item Prove that the definition of the gamma function is well-posed,
    i.e., the function \(u\mapsto e^{-u}u^{y-1}\) is in \(L(\bbR^+)\) for
    all \(y\in\bbR^+\).
  \item Show that
    \[
      \beta(x,y)=\frac{\Gamma(x)\Gamma(y)}{\Gamma(x+y)}.
    \]
  \end{enumerate}
\end{problem}
\begin{solution}
\end{solution}

\begin{problem}
  Let \(f\in L(\bbR^n)\) and for \(\mathbf{h}\in\bbR^n\) define
  \(f_{\mathbf{h}}\colon\bbR^n\to\bbR\) be
  \(f_{\mathbf{h}}( x )= f( x -\mathbf{h})\). Prove that
  \[
    \lim_{\mathbf{h}\to\mathbf{0}}\int_{\bbR^n}\left|f_{\mathbf{h}}-f\right|=0.
  \]
\end{problem}
\begin{solution}
\end{solution}

\begin{problem}
\begin{enumerate}[label=(\alph*)]
\item If \(f_k,g_k,f,g\in L(\bbR^n)\), \(f_k\to f\) and \(g_k\to g\) a.e.\@
  in \(\bbR^n\), \(|f_k|\leq g_k\) and
  \[
    \int_{\bbR^n}g_k\longrightarrow\int_{\bbR^n}g,
  \]

  prove that
  \[
    \int_{\bbR^n} f_k\longrightarrow\int_{\bbR^n}f.
  \]
\item Using part (a) show that if \(f_k,f\in L(\bbR^n)\) and \(f_k\to f\)
  a.e.\@ in \(\bbR^n\), then
  \[
    \int_{\bbR^n}|f_k-f|\longrightarrow 0\qquad\text{as \(k\to\infty\)}
  \]
  if and only if
  \[
    \int_{\bbR^n}|f_k|\longrightarrow\int_{\bbR^n}|f|\qquad\text{as
      \(k\to\infty\)}.
  \]
\end{enumerate}
\end{problem}
\begin{solution}
\end{solution}

%%% Local Variables:
%%% mode: latex
%%% TeX-master: "../MA544-Quals"
%%% End:

\section{MA 544 - Midterm 2}
\begin{problem}
Assume that $f\in L^1(\bbR^n)$. Show that for every $\varepsilon>0$ there
exists a ball $B$, centered at the origin, such that
\[
\int_{\bbR^n\minus B}|f|<\varepsilon.
\]
\end{problem}
\begin{proof}
Recall that $f\in L^1(\bbR^n)$ if and only if $|f|\in
L^1(\bbR^n)$. Let $B_k\coloneqq B(\mathbf{0},k)$ for $k\in\bbN$ and
$\chi_{B_k}$ be the indicator function associated with $B_k$. Then, the
sequence of maps $\left\{|f_k|\right\}$ defined $f_k\coloneqq f\chi_{B_k}$
converge pointwise to $|f_k|$. Since $|f|\in L^1(\bbR^n)$, by the monotone
convergence theorem, we have
\begin{equation}
\label{eq:monotonicity-2-1}
\int_{\bbR^n} |f_k|=\int_{B_k}|f|\longrightarrow\int_{\bbR^n}|f|.
\end{equation}
But this means, exactly, that for every $\varepsilon>0$ there exists
sufficiently large $N\in\bbN$ such that
\begin{equation}
  \label{eq:desired-inequality-2-1}
\begin{aligned}
\varepsilon&>\left|\int_{\bbR^n}|f_k|-\int_{\bbR^n}|f|\right|\\
&=-\int_{\bbR^n}|f_k|+\int_{\bbR^n}|f|\\
&=-\int_{\bbR^n}|f|+\int_{\bbR^n}|f|\\
&=-\int_{B_k}|f|+\int_{\bbR^n}|f|\\
&=\int_{\bbR^n\minus B_k}|f|
\end{aligned}
\end{equation}
as desired.
\end{proof}

\begin{problem}
Let $f\in L^1(E)$, and let $\{E_j\}$ be a countable collection of pairwise
disjoint measurable subsets of $E$, such that $E=\bigcup_{j=1}^\infty
E_j$. Prove that
\[
\int_E f=\sum_{j=1}^\infty\int_{E_j}f.
\]
\end{problem}
\begin{proof}
First, since the $E_j$'s are pairwise disjoint, by Theorem 3.23, we have
\begin{equation}
\label{eq:disjoint-measure-2-2}
|E|=\sum_{j=1}^\infty|E_j|.
\end{equation}
Let $\chi_{E_j}$ be the characteristic function of the subset $E_j$ of
$E$ and define $f_j\coloneqq f\chi_{E_j}$ for $j\in\bbN$. Note that, since
both $f$ and $\chi_{E_j}$ are measurable on $E$, $f_j$ is
measurable on $E$ and $\sum_{j=1}^\infty f_j=f$. Moreover, since
$E_j\subset E$, by monotonicity of the integral we have
\begin{equation}
\label{eq:monotonicity-2-2}
\int_{E} f=
\int_{E_j} f+\int_{E\minus E_j}f=
\int_E f_j+\int_{E\minus E_j}f.
\end{equation}
Hence, because the $E_j$'s are disjoint $(E\minus E_k)\minus
E_\ell=(E\minus E_\ell)\minus E_k$ so
\begin{equation}
\label{eq:desired-sum-2}
\int_E f=\sum_{j=1}^\infty\int_E f_j=\sum_{j=1}^\infty\int_{E_j}f
\end{equation}
as desired.
\end{proof}

\begin{problem}
Let $\{f_k\}$ be a family in $L^1(E)$ satisfying the following property:
For any $\varepsilon>0$ there exits $\delta>0$ such that $|A|<\delta$
implies
\[
\int_A |f_k|<\varepsilon
\]
for all $k\in\bbN$. Assume $|E|<\infty$, and $f_k(x)\to f(x)$ as
$k\to\infty$ for a.e.\@ $x\in E$. Show that
\[
\lim_{k\to\infty}\int_E f_k=\int_E f.
\]
(\emph{Hint:} Use Egorov's theorem.)
\end{problem}
\begin{proof}

\end{proof}

\begin{problem}
Let $I\coloneqq[0,1]$, $f\in L^1(I)$, and define $g(x)\coloneqq\int_x^1
t^{-1}f(t)\diff t$ for $x\in I$. Prove that $g\in L^1(I)$ and
\[
\int_I g=\int_I f.
\]
\end{problem}
\begin{proof}
\end{proof}

%%% Local Variables:
%%% mode: latex
%%% TeX-master: "../MA544-Quals"
%%% End:

\subsection{Final Exam}

%% Local Variables:
%%% mode: latex
%%% TeX-master: "../MA544-Quals"
%%% End:


%% References
\bibliographystyle{plainnat}
\bibliography{anal-bib}
\end{document}

%%% Local Variables:
%%% mode: latex
%%% TeX-master: t
%%% End:
