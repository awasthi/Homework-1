% Section 2.2: 3,7,13. Also prove that the kernel of the natural map from
% the fundamental group to the first homology group is the commutator
% subgroup.
\begin{problem}[Hatcher {\S}2.2, Ex.\@ 3]
Let $f\colon S^n\to S^n$ be a map of degree zero. Show that there exists
points $x,y\in S^n$ with $f(x)=x$ and $f(y)=-y$. Use this to show that if
$F$ is a continuous vector field defined on the unit ball $D^n$ in $\bfR^n$
such that $F(x)\neq 0$ for all $x$, then there exists a point on $\partial
D$ where $F$ points radially outward and another point on $\partial D^n$
where  $F$ points radially inward.
\end{problem}
\begin{proof}
Since $\deg f=0\neq (-1)^n=\deg a$, then $f\nsimeq a$ and so must have a
fixed point $x\in S^n$. Now, consider the map $g= a\circ f$. Since
$\deg g=\deg a\circ f=(\deg a)(\deg f)=0$, $g$ must have a fixed point
$y\in S^n$. Since $g(y)=a\circ f(y)=y$, then $f(y)=-y$.

Suppose $F$ is a continuous nonzero vector field on $S^n$, i.e., a map
$S^n\to\bfR^n$ which assigns to each point $x\in S^n$ a tangent vector
$\bfv(x)$ at $x$. Then, the map $f\colon\partial D^n\to\bfR^n$ given by
$f(\bfv(x))=\bfv(x)/\|\bfv(x)\|$ is well defined and nowhere zero.
\end{proof}
\newpage

\begin{problem}[Hatcher {\S}2.2, Ex.\@ 7]
For an invertible linear transformation $f\colon\bfR^n\to\bfR^n$ show that
the induced map
$H_n\left(\bfR^n,\bfR^n\smallsetminus\{\mathbf{0}\}\right)\cong\widetilde
H_{n-1}\left(\bfR^n\smallsetminus\{\mathbf{0}\}\right)\cong\bfZ$ is $\id$ or $-\id$
according to whether the determinant of $f$ is positive or negative. [Use
Gaußian elimination to show that the matrix of $f$ can be joined by a path
of invertible matrices to a diagonal matrix with $\pm 1$'s on the
diagonal.]
\end{problem}
\begin{proof}
We show that $\Orth(n)$ is a deformation retraction of $\GL(n,\bfR)$
and prove the result there. This procedure is adapted from a hint in
\emph{\begin{russian}Элементарная топология\end{russian}}
by \begin{russian} Виро, Нецветаев и Харламов, стр.\@ 338, номер
  39.11. \end{russian} Suppose $f\colon\bfR^n\to\bfR^n$ is
an invertible linear transformation. Let
$\left\{\mathbf{f}_1,\dotsc,\mathbf{f}_n\right\}$ be the set of columns
vectors of the matrix representation $F$ of $f$. By
Gram--Schmidt orthogonalization construct the vectors
\begin{equation}
\label{eq:gram-schmidt-orth}
\begin{aligned}
\mathbf{e}_1&=\lambda_{11}\mathbf{f}_1\\
\mathbf{e}_2&=\lambda_{21}\mathbf{f}_1+\lambda_{22}\mathbf{f}_2\\
&\vdotswithin{{}={}}\\
\mathbf{e}_n&=\lambda_{n1}\mathbf{f}_1+\dotsb+\lambda_{nn}\mathbf{f}_n
\end{aligned}
\end{equation}
where the $\lambda_{kk}>0$ for $k=1,\dotsc,n$. Now set
\begin{equation}
\label{eq:vectors-as-function-of-t}
\mathbf{g}_k(t)=
t\left(\lambda_{n1}\mathbf{f}_1+\lambda_{n2}\mathbf{f}_2+\dotsb+\lambda_{kk-1}\mathbf{f}_{k-1}\right)+(t\lambda_{kk}+1-t)\mathbf{f}_k.
\end{equation}
Let $g(t,A)$ be the matrix whose columns are the vectors $\mathbf{g}_k(t)$
and define a homotopy $f_t\colon I\times\GL(n,\bfR)\to\GL(n,\bfR)$ by mapping
the pair $(t,A)\mapsto g(t,A)$. Continuity of $H$ follows from the fact
that $H$ it is multiplication in $\bfR^n$ followed by a linear
mapping. It's not hard to see that $f_t$ stays in $\GL(n,\bfR)$ for all $t$
and $f_1(A)$ is in $\Orth(n)$.

Last but not least, we show that $\Orth(n)$ consists of two connected
components and that membership of $f$ to one of these components is
determined by $\det f$. First note that
$\det(\Orth(n))=\{-1,1\}$ which is disconnected in
$\bfR$. Hence, $\Orth(n)$ is disconnected. Now, if
$f\in\Orth(n)$, either $\det f=1$ or $\det f=-1$. Without loss of
generality, we may assume $\det f=1$ since if $r$ is a reflection.

Constructing the homotopy is hard. I can't think of a way of doing it and I
don't have the time right now, so I'll skip this part. There are other ways
to prove this indirectly, but I'm afraid I'm not familiar with Lie groups
and I am not willing to state a bunch of results from that subject.

Now that we have established that either $f\simeq \id$ or $-\id$, the map
$f$ on $\bfR^n$ induces a map $f_*=\pm\id_*$ on the homology groups
$H_n\left(\bfR^n,\bfR^n\smallsetminus\{\mathbf{0}\}\right)$.
\end{proof}
\newpage

\begin{problem}[Hatcher {\S}2.2, Ex.\@ 13]
Let $X$ be the $2$-complex obtained from $S^1$ with its usual cell
structure by attaching two $2$-cells by maps of degrees $2$ and $3$,
respectively.
\begin{enumerate}[label=(\alph*)]
\item Compute the homology groups of all the subcomplexes $A\subset X$ and
  the corresponding quotient complexes $X/A$.
\item Show that $X\simeq S^2$ and that the only subcomplex $A\subset X$ for
  which the quotient map $X\to X/A$ is a homotopy equivalence is the
  trivial subcomplex, the $0$-cell.
\end{enumerate}
\end{problem}
\begin{proof}
(a) Write $X$ as the union $e^0\cup e^1$ of a $0$-cell and a $1$-cell. Let
$e^2_1$, $e^2_2$ be $2$-cells attached to $X$ via maps $f_1,f_2\colon
S^2\to X$ of degrees $2$ and $3$, respectively. We use Lemma 2.34 to
compute the cellular homology of the new CW complex $X'$, it then follows
from Theorem 2.35 that the cellular homology is isomorphic to the singular
homology of $X'$. First, we write down the cellular chain complex for
$X'=X^2$
\begin{equation}
\label{eq:cellular-chain-x}
\cdots\longrightarrow
H_3^\CW(X^3)\longrightarrow
H_2^\CW(X^2)\longrightarrow
H_1^\CW(X^1)\longrightarrow
H_0^\CW(X^0)\longrightarrow
0.
\end{equation}
Filling in some of the values for $H_n^\CW(X^n)$ we have the chain
\begin{equation}
\label{eq:cellular-chain-x-2}
\cdots\longrightarrow
0\longrightarrow
\bfZ\oplus\bfZ\longrightarrow
\bfZ\longrightarrow
\bfZ\longrightarrow
0.
\end{equation}
Now recall that by definition a subcomplex of $X'$, $A$, is a closed
subspace that is the union of cells in $X$. Since we have the following
inclusion $e^0\subset e^1\subset e_1^2,e_2^2$ this makes for the following
candidates $A_0= e^0$, $A_1= e^0\cup e^1$, $A_{12}=
e^0\cup e^1\cup e_1^2$,
$A_{22}= e^0\cup e^1\cup e_2^2$, $X'$. Let's compute the homology
of these spaces.
\begin{itemize}
\item Case $A_0$: The cellular homology of $A_0$ is easy enough since it
is a $0$-cell. It's homology will be that of a point $H_n^\CW(A_0)=\bfZ$
for $n=0$ and $H_n^\CW(A_0)=0$ otherwise.
\item Case $A_1$: The subcomplex $A_1$ is homeomorphic to a circle $S^1$ so
  its cellular homology is isomorphic to that of a circle, i.e.,
  $H_n^\CW(A_1)=\bfZ$ if $n=0,1$ and $H_n^\CW(A_1)=0$ otherwise.
\item Case $A_{21}$: The cellular homology of $A_{21}$ is more
  interesting since we have the attaching map of degree $2$. This map $f_1$
  induces a map on homology $f_{1*}=2$ giving us the chain complex
  \begin{equation}
  \label{eq:chain-complex-a-1-2}
  \cdots\longrightarrow
  0\longrightarrow
  \bfZ\overset{2}{\longrightarrow}
  \bfZ\longrightarrow
  \bfZ\longrightarrow
  0
  \end{equation}
\item Case $A_{22}$:
\item Case $X$:
\end{itemize}
That concludes this part of the problem.
\\\\
(b)
\end{proof}
\newpage

\begin{problem}
\end{problem}
\begin{proof}
\end{proof}

%%% Local Variables:
%%% mode: latex
%%% TeX-master: "../MA572-HW-Current"
%%% End:
