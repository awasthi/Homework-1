% Section 2.2: 3,7,13. Also prove that the kernel of the natural map from
% the fundamental group to the first homology group is the commutator
% subgroup.
\begin{problem}[Hatcher {\S}2.2, Ex.\@ 3]
Let $f\colon S^n\to S^n$ be a map of degree zero. Show that there exists
points $x,y\in S^n$ with $f(x)=x$ and $f(y)=-y$. Use this to show that if
$F$ is a continuous vector field defined on the unit ball $D^n$ in $\bfR^n$
such that $F(x)\neq 0$ for all $x$, then there exists a point on $\partial
D$ where $F$ points radially outward and another point on $\partial D^n$
where  $F$ points radially inward.
\end{problem}
\begin{proof}
\end{proof}
\newpage

\begin{problem}[Hatcher {\S}2.2, Ex.\@ 7]
For an invertible linear transformation $f\colon\bfR^n\to\bfR^n$ show that
the induced map $H_n\left(\bfR^n,\bfR^n\minus\{0\}\right)\cong\widetilde
H_{n-1}\left(\bfR^n\minus\{0\}\right)\cong\bfZ$ is $\Id$ or $-\Id$
according to whether the determinant of $f$ is positive or negative. [Use
Gaußian elimination to show that the matrix of $f$ can be joined by a path
of invertible matrices to a diagonal matrix with $\pm 1$'s on the
diagonal.]
\end{problem}
\begin{proof}
\end{proof}
\newpage


\begin{problem}[Hatcher {\S}2.2, Ex.\@ 13]
Let $X$ be the $2$-complex obtained from $S^1$ with its usual cell
structure by attaching two $2$-cells by maps of degrees $2$ and $3$,
respectively.
\begin{enumerate}[label=(\alph*)]
\item Compute the homology groups of all the subcomplexes $A\subset X$ and
  the corresponding quotient complexes $X/A$.
\item Show that $X\simeq S^2$ and that the only subcomplex $A\subset X$ for
  which the quotient map $X\to X/A$ is a homotopy equivalence is the
  trivial subcomplex, the $0$-cell.
\end{enumerate}
\end{problem}
\begin{proof}
\end{proof}
\newpage

\begin{problem}
\end{problem}
\begin{proof}
\end{proof}

%%% Local Variables:
%%% mode: latex
%%% TeX-master: "../MA572-HW-Current"
%%% End:
