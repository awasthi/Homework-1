\begin{problem}[Hatcher {\S}2.1, Ex.\,16]
\begin{enumerate}[label=(\alph*)]
\item Show that $H_0(X,A)=0$ iff $A$ meets each path-component of $X$.
\item Show that $H_1(X,A)=0$ iff $H_1(A)\to H_1(X)$ is surjective and each
  path-component of $X$ contains at most one path-component of $A$.
\end{enumerate}
\end{problem}
\begin{proof}
(a) Let $i\colon A\hookrightarrow X$ denote the inclusion map. Then, the
map $i$ can be extended to a chain map between chain complexes so, by
proposition 2.9, induces a homomorphism $i_*\colon H_*(A)\to H_*(X)$ on
homology. Similarly, the map $j_\#\colon C_*(X)\to C_*(X,A)$ induces a map
$j_*\colon H_*(X)\to H_*(X,A)$ so, by theorem 2.16, we have a long exact
sequence
\begin{equation}
\label{eq:relative-long-exact-sequence}
\cdots\xrightarrow{\;\;\partial_*\;\;}
H_0(A)\xrightarrow{\;\;i_*\;\;}
H_0(X)\xrightarrow{\;\;j_*\;\;}
H_0(X,A)\xrightarrow{\;\;0\;\;}
0
\end{equation}
on homology. Thus, we see that $H_0(X,A)=0$ if and only if $i_*$ is
injective which, by proposition 2.6, happens if and only if $A$ meets each
path-component of $X$.
\\\\
(b) Let us extend to the left the long exact sequence of homology groups in
(\ref{eq:relative-long-exact-sequence}) as follows
\begin{equation}
  \label{eq:relative-short-exact-sequence-2}
\cdots\xrightarrow{\;\partial_*\;}
H_1(A)\xrightarrow{\;i_*\;}
H_1(X)\xrightarrow{\;j_*\;}
H_1(X,A)\xrightarrow{\;\partial_*\;}
H_0(A)\xrightarrow{\;i_*\;}
H_0(X)\xrightarrow{\;j_*\;}
H_0(X,A)\xrightarrow{\;0\;}0.
\end{equation}
Hence, $H_1(X,A)=0$ if and only if $j_*=0$ and $\partial_*=0$ if and only
if $i_*$ is surjective and $i_*$ is injective on $H_0(A)\to H_0(X)$, i.e,
each path-component of $X$ contains at most one path-component of $A$.
\end{proof}
\newpage

\begin{problem}[Hatcher {\S}2.1, Ex.\,18]
Show that for the subspace $\bfQ\subset\bfR$, the relative homology group
$H_1(\bfR,\bfQ)$ is free abelian and find a basis.
\end{problem}
\begin{proof}
Consider the long exact sequence of homology groups
\begin{equation}
  \label{eq:relative-short-exact-sequence-3}
\cdots\xrightarrow{\;\partial_*\;}
H_1(\bfQ)\xrightarrow{\;i_*\;}
H_1(\bfR)\xrightarrow{\;j_*\;}
H_1(\bfR,\bfQ)\xrightarrow{\;\partial_*\;}
H_0(\bfQ)\xrightarrow{\;i_*\;}
H_0(\bfR)\xrightarrow{\;j_*\;}
H_0(\bfR,\bfQ)\xrightarrow{\;0\;}0.
\end{equation}
Since the space $\bfR$ is contractible, $H_*(\bfR)=0$ which implies that
the map $i_*=0$ and $j_*=0$ on $H_0(\bfQ)\to H_0(\bfR)$ and $H_1(\bfR)\to
H_1(\bfR,\bfQ)$, respectively. Hence, $\partial_*\colon H_1(\bfR,\bfQ)\to
H_0(\bfQ)$ is surjective. Thus, $H_1(\bfR,\bfQ)\cong H_0(\bfQ)$. Since,
$\bfQ$ is totally disconnected, i.e., every connected component and hence,
path-component of $\bfQ$ is a singleton set, we have
$H_0(\bfQ)\cong \bfZ[\bfQ]\cong H_1(\bfR,\bfQ)$.
\end{proof}

\newpage
\begin{problem}
Homotopy invariance of homology.
\end{problem}
\begin{proof}
The proof of this follows immediately from corollary 2.10 for if $f\colon
X\to Y$ and $g\colon Y\to X$ are maps with $g\circ f\simeq\id_X$ and
$f\circ g\simeq \id_Y$ then by corollary 2.10 we have $(g\circ
f)_*=\id_{H_*(Y)}$ and $(f\circ g)_*=\id_{H_*(X)}$, but $(f\circ
g)_*=f_*\circ g_*$ and $(g\circ f)_*=g_*\circ f_*$ so $g_*=f_*^{-1}$ and we
see that $H_*(X)\cong H_*(Y)$.
\end{proof}

%%% Local Variables:
%%% mode: latex
%%% TeX-master: "../MA572-HW-Current"
%%% End:
