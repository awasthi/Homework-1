\begin{problem}[Hatcher {\S}2.1, Ex.\,11]
Show that if $A$ is a retract of $X$ then the map $H_n(A)\to H_n(X)$
induced by the inclusion $A\subset X$ is injective.
\end{problem}
\begin{proof}
Suppose that $A$ is a retract of $X$. Then there exists a continuous map
$r\colon X\to A$ such that $r(X)=A$ and $\restrict{r}{A}=\id_A$. Let
$i\colon A\hookrightarrow X$ denote the inclusion map and
$i_*\colon H_n(A)\to H_n(X)$ denote the induced homomorphism on the
homology groups of $A$ and $X$; do the same for $r$, $r_*\colon H_n(X)\to
H_n(X)$. Then $r\circ i=\id_A$ which induces the endomorphism $(r\circ
i)_*=r_*\circ i_*=\id_{H_n(A)}$ on $H_n(A)$. Thus, the inclusion map
$i_*$ is injective (since it has a left inverse).
\end{proof}
\newpage

\begin{problem}[Hatcher \S2.1, Ex.\,12]
Show that chain homotopy of chain maps is an equivalence relation.
\end{problem}
\begin{proof}
Let $X$ and $Y$ be topological spaces and $f,g,h\colon X\to Y$ be
continuous maps. Then $f_\#,g_\#,h_\#\colon C_n(X)\to C_n(Y)$ denote the
induced chain maps. We show that chain homotopy of chain maps is an
equivalence relation:
\begin{enumerate}[label=(\roman*)]
\item Let $P$ be the $0$ homomorphsim. Then, we have
  \[
    \partial0+0\partial=0=f_\#-f_\#.
  \]
  Thus, $f_\#$ is chain homotopic to itself.
\item Suppose $f_\#$ is chain homotopic to $g_\#$. Then there exist a
  homomorphism $P\colon C_n(X)\to C_{n+1}(Y)$ such hat $\partial
  P+P\partial=g_\#-f_\#$. Put $Q\coloneqq -P$. Then, we have
  \[
    \partial(-P)+(-P)\partial=-(\partial P+P\partial)=-(g_\#-f_\#)=f_\#-g_\#.
  \]
  Thus, $g_\#$ is chain homotopic to $f_\#$.
\item Suppose that $f_\#$ is chain homotopic to $g_\#$ and $g_\#$ is chain
  homotopic to $h_\#$. Then there exists homomorphism $P\colon C_n(X)\to
  C_{n+1}(Y)$ and a homomorphism $Q\colon C_n(X)\to C_{n+1}(Y)$ such that
  $\partial P+P\partial=g_\#-f_\#$ and $\partial
  Q+Q\partial=h_\#-g_\#$. Put $R\coloneqq P+Q$. Then, we have
  \begin{align*}
    \partial (P+Q)+(P+Q)\partial
    &=\partial P+\partial Q+P\partial+Q\partial\\
    &=(\partial Q+Q\partial)+(\partial P+P\partial)\\
    &=(h_\#-g_\#)+(g_\#-f_\#)\\
    &=h_\#-f_\#.
  \end{align*}
  Thus, $f_\#$ is chain homotopic to $h_\#$.
\end{enumerate}
We conclude that `chain homotopy' is an equivalence relation.
\end{proof}
\newpage


%%% Local Variables:
%%% mode: latex
%%% TeX-master: "../MA572-HW-Current"
%%% End:
