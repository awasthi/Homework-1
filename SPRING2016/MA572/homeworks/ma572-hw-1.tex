\begin{problem}[Hatcher {\S}2.1, Ex.\,11]
Show that if $A$ is a retract of $X$ then the map $H_n(A)\to H_n(X)$
induced by the inclusion $A\subset X$ is injective.
\end{problem}
\begin{proof}
Suppose that $A$ is a retract of $X$. Then there exists a continuous map
$r\colon X\to A$ such that $r(X)=A$ and $\restrict{r}{A}=\id_A$. Let
$i\colon A\hookrightarrow X$ denote the inclusion map and
$i_*\colon H_n(A)\to H_n(X)$ denote the induced homomorphism on the
homology groups of $A$ and $X$; do the same for $r$, $r_*\colon H_n(X)\to
H_n(X)$. Then $r\circ i=\id_A$ which induces the endomorphism $(r\circ
i)_*=r_*\circ i_*=\id_{H_n(A)}$ on $H_n(A)$. Thus, the inclusion map
$i_*$ is injective (since it has a left inverse).
\end{proof}
\newpage

\begin{problem}[Hatcher \S2.1, Ex.\,12]
Show that chain homotopy of chain maps is an equivalence relation.
\end{problem}
\begin{proof}
Let $X$ and $Y$ be topological spaces and $f,g,h\colon X\to Y$ be
continuous maps. Then $f_\#,g_\#,h_\#\colon C_n(X)\to C_n(Y)$ denote the
induced chain maps. We show that chain homotopy of chain maps is an
equivalence relation:
\begin{enumerate}[label=(\roman*)]
\item Let $P$ be the $0$ homomorphsim. Then, we have
  \[
    \partial0+0\partial=0=f_\#-f_\#.
  \]
  Thus, $f_\#$ is chain homotopic to itself.
\item Suppose $f_\#$ is chain homotopic to $g_\#$. Then there exist a
  homomorphism $P\colon C_n(X)\to C_{n+1}(Y)$ such hat $\partial
  P+P\partial=g_\#-f_\#$. Put $Q\coloneqq -P$. Then, we have
  \[
    \partial(-P)+(-P)\partial=-(\partial P+P\partial)=-(g_\#-f_\#)=f_\#-g_\#.
  \]
  Thus, $g_\#$ is chain homotopic to $f_\#$.
\item Suppose that $f_\#$ is chain homotopic to $g_\#$ and $g_\#$ is chain
  homotopic to $h_\#$. Then there exists homomorphism $P\colon C_n(X)\to
  C_{n+1}(Y)$ and a homomorphism $Q\colon C_n(X)\to C_{n+1}(Y)$ such that
  $\partial P+P\partial=g_\#-f_\#$ and $\partial
  Q+Q\partial=h_\#-g_\#$. Put $R\coloneqq P+Q$. Then, we have
  \begin{align*}
    \partial (P+Q)+(P+Q)\partial
    &=\partial P+\partial Q+P\partial+Q\partial\\
    &=(\partial Q+Q\partial)+(\partial P+P\partial)\\
    &=(h_\#-g_\#)+(g_\#-f_\#)\\
    &=h_\#-f_\#.
  \end{align*}
  Thus, $f_\#$ is chain homotopic to $h_\#$.
\end{enumerate}
We conclude that `chain homotopy' is an equivalence relation.
\end{proof}
\newpage

\begin{problem}[Hatcher {\S}2.1, Ex.\,16]
\begin{enumerate}[label=(\alph*)]
\item Show that $H_0(X,A)=0$ iff $A$ meets each path-component of $X$.
\item Show that $H_1(X,A)=0$ iff $H_1(A)\to H_1(X)$ is surjective and each
  path-component of $X$ contains at most one path-component of $A$.
\end{enumerate}
\end{problem}
\begin{proof}
% (a) $\implies$ Suppose that the $0$th homology group of $X$ relative to $A$
% is trivial. Let $\left\{X_\alpha\right\}$ denote the set of path components of
% $X$ and $\left\{A_\beta\right\}$ denote the set of path components of
% $A$. Note that, by elementary point-set topology,
% $\left\{A_\beta\right\}=\left\{A\cap X_\alpha\right\}$, i.e., $A_\beta=A\cap X_\alpha$
% for some $\alpha$. Let
% \[
% \cdots\longrightarrow C_1(X)\longrightarrow C_0(X)\longrightarrow 0.
% \]
% be a singular chain complex of $X$, and
% \[
% \cdots\longrightarrow C_1(A)\longrightarrow C_0(A)\longrightarrow 0.
% \]
% be a singular chain complex of $A$. Recall that the path components of

% $\impliedby$ Conversely, suppose that $A$ meets each path component of $X$,
% i.e., there is a one-to-one correspondence between $\left\{A_\beta\right\}$
% and $\left\{X_\alpha\right\}$. Then, by proposition 2.7 we have
% $H_0(A)\cong \bigoplus_\alpha\bfZ\cong H_0(X)$. By theorem 2.16, the
% long exact sequence
% \begin{equation}
% \label{eq:long-exact-sequence-relative-homology}
% \cdots\longrightarrow H_0(A)\longrightarrow H_0(X)\longrightarrow
% H_0(X,A)\longrightarrow 0
% \end{equation}
% so $H_0(X,A)\cong H_0(X)/\im i_*$ where $i_*$ is the chain map induced by
% the inclusion $i\colon A\hookrightarrow X$. But $H_0(A)\cong H_0(X)$, in
% particular, $i_*$ is surjective. Hence, $H_0(X,A)=0$.

% (a) Let $\left\{X_\alpha\right\}$ and $\left\{A_\beta\right\}$ denote the
% path components of $X$ and $A$, respectively. Recall that the $0$th
% homology of $X$, respectively of $A$, is generated by the path components
% of $X$ (more precisely, representatives of these). Now, by proposition 2.16
% we have the long exact sequence
% \begin{equation}
% \label{eq:long-exact-sequence-relative-homology}
% \cdots\longrightarrow H_0(A)\longrightarrow H_0(X)\longrightarrow
% H_0(X,A)\longrightarrow 0.
% \end{equation}
% So $H_0(X,A)=0$ if and only if $i_*\colon H_0(A)\to H_0(X)$ is surjective
% (where $i_*$ is the map on homology induced by the inclusion $i\colon
% A\hookrightarrow X$). By proposition 2.6, we have that
% $H_0(X)=\bigoplus_\alpha H_0(X_\alpha)$ so that if $A$ intersects each
% $X_\alpha$, then
(a)
\\\\
(b) By (\ref{eq:long-exact-sequence-relative-homology}), in particular,
if we extend it a little
\[
\cdots\longrightarrow H_1(A)\longrightarrow H_1(X)\longrightarrow H_1(X,A)
\longrightarrow H_0(A)\longrightarrow H_0(X)\longrightarrow
H_0(X,A)\longrightarrow 0,
\]
we see that $H_1(X)=0$ if and only if the homomorphism $i_*^1\colon H_1(A)\to
H_1(X)$ is surjective and $i_*^0\colon H_0(A)\to H_0(X)$ is injective,
where $i_*^j\coloneqq\restrict{i_*}{H_i(A)}$ if and only if $A$ meets each
path component of $X$.
\end{proof}
\newpage

\begin{problem}[Hatcher {\S}2.1, Ex.\,17]
\begin{enumerate}[label=(\alph*)]
\item Compute the homology groups $H_n(X,A)$ when $X$ is $\bfS^2$ or
  $\bfS^1\times\bfS^1$ and $A$ is a finite set of points in $X$.
\item Compute the groups $H_n(X,A)$ and $H_n(X,B)$ for $X$ a closed
  orientable surface of genus two with $A$ and $B$ the circles shown. [What
  are $X/A$ and $X/B$?]
\end{enumerate}
\end{problem}
\begin{proof}
(a) Since $A$ is a finite collection of points in $\bfS^2$, let us enumerate
the set $A$ via $\left\{ a_1,...,a_n \right\}$ and denote by $A_k$ the
subset $\left\{a_1,...,a_k\right\}$ of $A$, where $k\leq n$. Now, by the
generalization of theorem 2.16 to triples, we have the long exact sequence
\begin{equation}
\label{eq:long-exact-sequence-triples}
\cdots\longrightarrow H_m(A_n,A_{n-1})\longrightarrow
H_m(\bfS^2,A_{n-1})\longrightarrow H_m(\bfS^2,A_n)
\longrightarrow H_{m-1}(A_n,A_{n-1})\longrightarrow\cdots.
\end{equation}
Exactness of (\ref{eq:long-exact-sequence-triples}) tells us that for
$m\geq 2$ we have $H(\bfS^2,A_{n-1})\cong H(\bfS^2,A_n)$ since
\[
H_m(A_n,A_{n-1})=0\longrightarrow H_m(\bfS^2,A_{n-1})\longrightarrow
H_m(\bfS^2,A_n)\longrightarrow 0=H_{m-1}(A_n,A_{n-1})
\]
is exact. Evidently, $H_m(A_n,A_{n-1})=0$ for $m>1$.\footnote{I will prove
  this if time permits.}
\\\\
(b)
\end{proof}

%%% Local Variables:
%%% mode: latex
%%% TeX-master: "../MA572-HW-Current"
%%% End:
