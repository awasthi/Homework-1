\chapter{Homology}
A summary of Hatcher's homology section from his \emph{Algebraic Topology}
book.
\section{Simplicial and Signular Homology}
Skip all this nonsense. I need to catch up.
\section{Computations and Applications}
\subsection{Degree}
For a map $f\colon S^n\to S^n$ with $n>0$, the induced map $f_*\colon
H_n(S^n)\to H_n(S^n)$ is a homomorphism from an infinite cyclic group to
itself and so must be of the form $f_*(\alpha)=df(\alpha)$ for some integer
$d$ depending only on $f$. This integer is called the \emph{degree} of $f$
and is denoted by $\deg f$. Here are some basic properties of the degree
\begin{enumerate}[label=(\arabic*)]
\item $\deg\id_{S^n}=1$ since $(\id_{S^n})_*=\id_{H_n(S^n)}$.
\item $\deg f=0$ if $f$ is not injective. For if we choose a point $x_0\in
  S^n\minus f(S^n)$ then $f$ can be factored as a composition $S^n\to
  S^n\minus\{x_0\}\hookrightarrow S^n$ and $H_n(S^n\minus\{x_0\})=0$ since
  $S^n\minus\{x_0\}$ is contractible.
\item If $f\simeq g$ then $\deg f=\deg g$ since $f_*=g_*$. The converse
  statement, that if $\deg f=\deg g$, is a fundamental theorem of Hopf from
  around 1925 which we prove in Corollary 4.25.
\item $\deg fg=\deg f\deg g$, since $(f\circ g)_*=f_*\circ g_*$. As a
  consequence, $\deg f=\pm 1$ if $f$ is a homotopy equivalence since
  $f\circ g\simeq \id_{S^n}$ implies that $\deg f\deg g=\deg\id_{S^n}=1$.
\item $\deg f=-1$ if $f$ is a reflection of $S^n$, fixing the points in
  some subsphere $S^{n-1}\subset S^n$ and interchanging the two
  complementary hemispheres. For we can give $S^n$ a $\Delta$-complex
  structure with these two hemispheres as its two $n$-simplices
  $\Delta_1^n$ and $\Delta_2^n$, and the $n$-chain $\Delta_1^n-\Delta_2^n$
  represents a generator of $H_n(S^n)$ as we saw in Example 2.23, so the
  reflection interchanging $\Delta_1^n$ and $\Delta_2^n$ sends this
  generator to its negative.
\item The antipodal map $a\colon S^n\to S^n$, $x\mapsto -x$, has degree
  $(-1)^{n+1}$ since it is the composition of $n+1$ reflections, each
  changing the sign of one coordinate in $\bfR^{n+1}$.
\item If $f\colon S^n\to S^n$ has no fixed points then $\deg
  f=(-1)^{n+1}$. For if $f(x)\neq x$ for any $x\in S^n$, then the line
  segment from $f(x)$ to $-x$, defined by $t\mapsto(1-t)f(x)-tx$ for $0\leq
  t\leq 1$, does not pass through the origin. Hence if $f$ has no fixed
  points, the formula $f_t(x)\coloneqq[(1-t)f(x)-tx]/\|(1-t)f(x)-tx\|$
  defines a homotopy from $f$ to the antipodal map. Note that the antipodal
  map has no fixed points, so the fact that maps without fixed points are
  homotopic to the antipodal point is sort of a converse statement.
\end{enumerate}
\begin{theorem}[2.8]
$S^n$ has a continuous field of nonzero tangent vectors if and only if $n$
is odd.
\end{theorem}
\begin{proof}
$\imp}lies$: Suppose that $x\maspto\bfv(x)$ is a tangent vector field on
$S^n$, assigning to a vector $x\in S^n$ the vector $\bfv(x)$ tangent to
$S^n$ at $x$. Regarding $\bfv(x)$ as a vector at the origin instead of at
$x$, tangency just means that $x$ and $\bfv(x)$ are orthogonal in
$\bfR^{n+1}$.

$\impliedby$:
\end{proof}

%%% Local Variables:
%%% mode: latex
%%% TeX-master: "../MA572-Notes"
%%% End:
