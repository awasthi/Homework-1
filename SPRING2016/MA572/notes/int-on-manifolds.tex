\chapter{Smooth Manifolds}
\section{Some results from Boothby}
\begin{definition}[3.1]
A \emph{manifold} $M$ of \emph{dimension} $n$, or \emph{$n$-manifold}, is a
topological space with the following properties:
\begin{enumerate}[label=(\roman*)]
\item $M$ is Hausdorff,
\item $M$ is locally Euclidean of dimension $n$, and
\item $M$ has a countable basis of open sets.
\end{enumerate}
\end{definition}

As a matter of notation $\dim M$ is used for the \emph{dimension} of $M$;
when $\dim M=0$, then $M$ is a countable space with the discrete
topology. It follows from the homeomorphism of $U$ and $U'$ that
\emph{locally Euclidean} is equivalent to the requirement that each point
$p$ have a neighborhood $U$ homeomorphic to an $n$-ball in $\bbR^n$. Thus,
a manifold of dimension $1$ is locally homeomorphic to an open interval,
etc.

\begin{theorem}[3.6]
A topological manifold $M$ is locally connected, locally compact, and a
union of a countable collection of compact subsets; furthermore, it is
normal and metrizable.
\end{theorem}

The notion of \emph{coordinates} plays an important role in manifold
theory, just as it does in the study of the geometry of $\bbE^n$. In
$\bbE^n$, however, it is possible to find a single system of coordinates
for the \emph{entire} space, that is, to establish a correspondence between
all $\bbE^n$ and $\bbR^n$. Built into the definition of an $n$-manifold $M$
is a correspondence of a neighborhood $U$ of each $p\in M$ and an open
subset $U'$ of $\bbR^n$. Letting $\varphi\colon U\to U'$ be this
correspondence, we call the pair $(U,\varphi)$ a \emph{coordinate
  neighborhood} and the numbers $x^1(q),\dotsc,x^n(q)$ given by
$\varphi(q)=(x^1(q),\dotsc,x^n(q))$, the \emph{coordinates} of $q\in M$.

$\bbH^n$ is the \emph{subspace} of $\bbR^n$ defined by
\[
\bbH^n\coloneqq\left\{\,(x^1,\dotsc,x^n)\in\bbR^n:x^n\geq 0\,\right\}.
\]
We shall defined a \emph{manifold with boundary} to be a Hausdorff space
$M$ with a countable basis of open sets which has the property that each
$p\in M$ is contained in an open set $U'$ of $\bbH^n\setminus\partial\bbH^n$
or to an open set of $U'$ of $\bbH^n$ with $\varphi(p)\in\partial\bbH^n$,
i.e., a boundary point of $\bbH^n$. In the second case $p$ is called a
\emph{boundary point} of $M$ and the collection of boundary points of $M$
is denoted by $\partial M$ and is called the \emph{boundary} of $M$.

\begin{theorem}[4.1]
Every compact, connected, orientable $2$-manifold is homeomorphic to a
sphere with handles added. Two such manifolds with the same member of
handles are homeomorphic and conversely, so that the number of handles
(called the genus) is the only topological invariant.
\end{theorem}

%%% Local Variables:
%%% mode: latex
%%% TeX-master: "../MA572-Notes"
%%% End:
