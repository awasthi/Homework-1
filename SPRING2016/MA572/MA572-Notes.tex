\def\documentauthor{Carlos Salinas}
\def\documenttitle{MA572 Hatcher Notes}
% \def\hwnum{1}
\def\shorttitle{MA572-Notes}
\def\coursename{MA544}
\def\documentsubject{algebraic topology}
\def\authoremail{salinac@purdue.edu}

\documentclass[article,oneside,10pt]{memoir}
\usepackage{geometry}
\usepackage[dvipsnames]{xcolor}
\usepackage[
    breaklinks,
    bookmarks=true,
    colorlinks=true,
    pageanchor=false,
    linkcolor=black,
    anchorcolor=black,
    citecolor=black,
    filecolor=black,
    menucolor=black,
    runcolor=black,
    urlcolor=black,
    hyperindex=false,
    hyperfootnotes=true,
    pdftitle={\shorttitle},
    pdfauthor={\documentauthor},
    pdfkeywords={\documentsubject},
    pdfsubject={\coursename}
    ]{hyperref}

%% Math
\usepackage{amsmath}
\usepackage{amsthm}
\usepackage{amssymb}
\usepackage{mathtools}
% \usepackage{eucal}
% \usepackage{mathrsfs}
% \usepackage[nointegrals]{wasysym}

%% Language
\usepackage{cmap}
\usepackage[LAE,LFE,T2A,T1]{fontenc}
\usepackage[utf8]{inputenc}
\usepackage[farsi,french,german,spanish,russian,english]{babel}
\babeltags{fr=french,
           de=german,
           en=english,
           es=spanish,
           pa=farsi,
           ru=russian
           }
\def\spanishoptions{mexico}

\selectlanguage{english}

\newcommand{\textfa}[1]{\beginR\textpa{#1}\endR}

\usepackage{CJKutf8}
\newcommand{\textkr}[1]{\begin{CJK}{UTF8}{mj}#1\end{CJK}}
\newcommand{\textjp}[1]{\begin{CJK}{UTF8}{min}#1\end{CJK}}
\newcommand{\textzh}[1]{\begin{CJK}{UTF8}{bsmi}#1\end{CJK}}

%% Misc
\usepackage{graphicx}
\graphicspath{{figures/}}

\usepackage{microtype}
\usepackage{lineno}
\usepackage{multicol}
\usepackage[inline]{enumitem}
\usepackage{listings}
\usepackage{mleftright}
\mleftright
\usepackage{carlos-variables}

% %% Unicode math and Polyglossia
% \usepackage{unicode-math}
% \usepackage{unicode-minionmath}

% \setmainfont[Ligatures=TeX]{Libertinus Serif}
% \setsansfont{Libertinus Sans}
% \setmonofont{Libertinus Mono}
% \setmathfont{Minion Math}
% \setmathfont[range={\mathfrak}]{XITS Math}
% \setmathfont[range={\mathcal},StylisticSet=1]{XITS Math}
% \setmathfont[range={\mathscr}]{XITS Math}
% \setmathfont[range={}]{Minion Math}

% \usepackage{polyglossia}

% \newfontfamily\cyrillicfont[Script=Cyrillic]{Libertinus Serif}
% \newfontfamily\cyrillicfontsf[Script=Cyrillic]{Libertinus Sans}

% \newfontfamily\farsifont[Script=Arabic,
%                          Scale=MatchUppercase]{Amiri}

% \setmainlanguage[variant=american]{english}
% \setotherlanguage{farsi}
% \setotherlanguage{french}
% \setotherlanguage[spelling=new,latesthyphen,babelshorthands]{german}
% \setotherlanguage{spanish}
% \setotherlanguage[spelling=modern,babelshorthands]{russian}

% \makeatletter
% \@Latintrue
% \makeatother

% \usepackage{xeCJK}
% \usepackage[overlap]{ruby}
% \renewcommand\rubysep{-0.2ex}
% \xeCJKDeclareSubCJKBlock{Kana}{"3040 -> "309F, "30A0 -> "30FF, "31F0 -> "31FF, "1B000 -> "1B0FF}
% \xeCJKDeclareSubCJKBlock{Hangul}{"1100 -> "11FF, "3130 -> "318F, "A960 -> "A97F, "AC00 -> "D7AF, "D7B0 -> "D7FF}

% \setCJKmainfont{HanaMinA}
% \setCJKmainfont[Kana]{HanaMinA}
% \setCJKmainfont[Hangul]{NanumMyeongjo}
% \setCJKsansfont[Hangul]{NanumGothic}

%% Theorems and definitions
%% remove parentheses
% \makeatletter
% \def\thmhead@plain#1#2#3{%
%   \thmname{#1}\thmnumber{\@ifnotempty{#1}{ }\@upn{#2}}%
%   \thmnote{ {\the\thm@notefont#3}}}
% \let\thmhead\thmhead@plain
% \makeatother

\theoremstyle{plain}
\newtheorem{theorem}{Theorem}
\newtheorem{proposition}[theorem]{Proposition}
\newtheorem{corollary}[theorem]{Corollary}
\newtheorem{claim}[theorem]{Claim}
\newtheorem{lemma}[theorem]{Lemma}
\newtheorem{axiom}[theorem]{Axiom}

\newtheorem*{corollary*}{Corollary}
\newtheorem*{claim*}{Claim}
\newtheorem*{lemma*}{Lemma}
\newtheorem*{proposition*}{Proposition}
\newtheorem*{theorem*}{Theorem}

\theoremstyle{definition}
\newtheorem{definition}{Definition}
\newtheorem{example}{Examples}
\newtheorem{examples}[example]{Examples}
\newtheorem{exercise}{Exercise}[chapter]
\newtheorem{problem}[exercise]{Problem}

\newtheorem*{example*}{Example}
\newtheorem*{exercise*}{Exercise}
\newtheorem*{problem*}{Problem}

\begin{document}
\author{\href{mailto:\authoremail}{\documentauthor}}
\title{\documenttitle}
\date{\today}
\maketitle
\chapter{Homology}
A summary of Hatcher's homology section from his \emph{Algebraic Topology}
book.
\section{Simplicial and Signular Homology}
Skip all this nonsense. I need to catch up.
\section{Computations and Applications}
\subsection{Degree}
For a map $f\colon S^n\to S^n$ with $n>0$, the induced map $f_*\colon
H_n(S^n)\to H_n(S^n)$ is a homomorphism from an infinite cyclic group to
itself and so must be of the form $f_*(\alpha)=df(\alpha)$ for some integer
$d$ depending only on $f$. This integer is called the \emph{degree} of $f$
and is denoted by $\deg f$. Here are some basic properties of the degree
\begin{enumerate}[label=(\arabic*)]
\item $\deg\id_{S^n}=1$ since $(\id_{S^n})_*=\id_{H_n(S^n)}$.
\item $\deg f=0$ if $f$ is not injective. For if we choose a point $x_0\in
  S^n\minus f(S^n)$ then $f$ can be factored as a composition $S^n\to
  S^n\minus\{x_0\}\hookrightarrow S^n$ and $H_n(S^n\minus\{x_0\})=0$ since
  $S^n\minus\{x_0\}$ is contractible.
\item If $f\simeq g$ then $\deg f=\deg g$ since $f_*=g_*$. The converse
  statement, that if $\deg f=\deg g$, is a fundamental theorem of Hopf from
  around 1925 which we prove in Corollary 4.25.
\item $\deg fg=\deg f\deg g$, since $(f\circ g)_*=f_*\circ g_*$. As a
  consequence, $\deg f=\pm 1$ if $f$ is a homotopy equivalence since
  $f\circ g\simeq \id_{S^n}$ implies that $\deg f\deg g=\deg\id_{S^n}=1$.
\item $\deg f=-1$ if $f$ is a reflection of $S^n$, fixing the points in
  some subsphere $S^{n-1}\subset S^n$ and interchanging the two
  complementary hemispheres. For we can give $S^n$ a $\Delta$-complex
  structure with these two hemispheres as its two $n$-simplices
  $\Delta_1^n$ and $\Delta_2^n$, and the $n$-chain $\Delta_1^n-\Delta_2^n$
  represents a generator of $H_n(S^n)$ as we saw in Example 2.23, so the
  reflection interchanging $\Delta_1^n$ and $\Delta_2^n$ sends this
  generator to its negative.
\item The antipodal map $a\colon S^n\to S^n$, $x\mapsto -x$, has degree
  $(-1)^{n+1}$ since it is the composition of $n+1$ reflections, each
  changing the sign of one coordinate in $\bfR^{n+1}$.
\item If $f\colon S^n\to S^n$ has no fixed points then $\deg
  f=(-1)^{n+1}$. For if $f(x)\neq x$ for any $x\in S^n$, then the line
  segment from $f(x)$ to $-x$, defined by $t\mapsto(1-t)f(x)-tx$ for $0\leq
  t\leq 1$, does not pass through the origin. Hence if $f$ has no fixed
  points, the formula $f_t(x)\coloneqq[(1-t)f(x)-tx]/\|(1-t)f(x)-tx\|$
  defines a homotopy from $f$ to the antipodal map. Note that the antipodal
  map has no fixed points, so the fact that maps without fixed points are
  homotopic to the antipodal point is sort of a converse statement.
\end{enumerate}
\begin{theorem}[2.8]
$S^n$ has a continuous field of nonzero tangent vectors if and only if $n$
is odd.
\end{theorem}
\begin{proof}
$\imp}lies$: Suppose that $x\maspto\bfv(x)$ is a tangent vector field on
$S^n$, assigning to a vector $x\in S^n$ the vector $\bfv(x)$ tangent to
$S^n$ at $x$. Regarding $\bfv(x)$ as a vector at the origin instead of at
$x$, tangency just means that $x$ and $\bfv(x)$ are orthogonal in
$\bfR^{n+1}$.

$\impliedby$:
\end{proof}

%%% Local Variables:
%%% mode: latex
%%% TeX-master: "../MA572-Notes"
%%% End:

\chapter{Smooth Manifolds}
\section{Some results from Boothby}
\begin{definition}[3.1]
A \emph{manifold} $M$ of \emph{dimension} $n$, or \emph{$n$-manifold}, is a
topological space with the following properties:
\begin{enumerate}[label=(\roman*)]
\item $M$ is Hausdorff,
\item $M$ is locally Euclidean of dimension $n$, and
\item $M$ has a countable basis of open sets.
\end{enumerate}
\end{definition}

As a matter of notation $\dim M$ is used for the \emph{dimension} of $M$;
when $\dim M=0$, then $M$ is a countable space with the discrete
topology. It follows from the homeomorphism of $U$ and $U'$ that
\emph{locally Euclidean} is equivalent to the requirement that each point
$p$ have a neighborhood $U$ homeomorphic to an $n$-ball in $\bfR^n$. Thus,
a manifold of dimension $1$ is locally homeomorphic to an open interval,
etc.

\begin{theorem}[3.6]
A topological manifold $M$ is locally connected, locally compact, and a
union of a countable collection of compact subsets; furthermore, it is
normal and metrizable.
\end{theorem}

The notion of \emph{coordinates} plays an important role in manifold
theory, just as it does in the study of the geometry of $\bfE^n$. In
$\bfE^n$, however, it is possible to find a single system of coordinates
for the \emph{entire} space, that is, to establish a correspondence between
all $\bfE^n$ and $\bfR^n$. Built into the definition of an $n$-manifold $M$
is a correspondence of a neighborhood $U$ of each $p\in M$ and an open
subset $U'$ of $\bfR^n$. Letting $\varphi\colon U\to U'$ be this
correspondence, we call the pair $(U,\varphi)$ a \emph{coordinate
  neighborhood} and the numbers $x^1(q),\dotsc,x^n(q)$ given by
$\varphi(q)=(x^1(q),\dotsc,x^n(q))$, the \emph{coordinates} of $q\in M$.

$\bfH^n$ is the \emph{subspace} of $\bfR^n$ defined by
\[
\bfH^n\coloneqq\left\{\,(x^1,\dotsc,x^n)\in\bfR^n:x^n\geq 0\,\right\}.
\]
We shall defined a \emph{manifold with boundary} to be a Hausdorff space
$M$ with a countable basis of open sets which has the property that each
$p\in M$ is contained in an open set $U'$ of $\bfH^n\setminus\partial\bfH^n$
or to an open set of $U'$ of $\bfH^n$ with $\varphi(p)\in\partial\bfH^n$,
i.e., a boundary point of $\bfH^n$. In the second case $p$ is called a
\emph{boundary point} of $M$ and the collection of boundary points of $M$
is denoted by $\partial M$ and is called the \emph{boundary} of $M$.

\begin{theorem}[4.1]
Every compact, connected, orientable $2$-manifold is homeomorphic to a
sphere with handles added. Two such manifolds with the same member of
handles are homeomorphic and conversely, so that the number of handles
(called the genus) is the only topological invariant.
\end{theorem}

%%% Local Variables:
%%% mode: latex
%%% TeX-master: "../MA572-Notes"
%%% End:

\end{document}

%%% Local Variables:
%%% mode: latex
%%% TeX-master: t
%%% End:
