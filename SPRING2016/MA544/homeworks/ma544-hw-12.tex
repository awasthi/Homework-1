%% #12 – Due Apr. 25 Chapter 8: # 2, 3, 4, 5, 6, 9, 11.

\begin{problem}[Wheeden \& Zygmund {\S}8, Ex.\@ 2]
Prove the converse of Hölder's inequality for $p=1$ and $\infty$. Show also
that for $1\leq p\leq\infty$, a real-valued measurable $f$ belongs to
$L^p(E)$ if $fg\in L^1(E)$ for every $g\in L^{p'}(E)$, $1/p+1/p'=1$. The
negation is also of interest: if $f\in L^p(E)$ then there exists $g\in
L^{p'}(E)$ such that $fg\notin L^1(E)$. (To verify the negation, construct
$g$ of the form $\sum a_kg_k$ satisfying $\int_E fg_k\to\infty$.)
\end{problem}
\begin{proof}
In this problem, we finish the proof of Theorem 8.8 for the case
$p=1,\infty$. Therefore, we must show that:
\begin{quote}
For $f$ a measurable real-valued function on $E$ and $p=1,\infty$. Then
\[
\|f\|_p=\sup\int_E fg,
\]
where the supremum is taken over every real-valued $g$ such that
$\|g\|_{p'}\leq 1$ and $\int_E fg$ exists.
\end{quote}
In both cases, $p=1$ and $p=\infty$, we may, without loss of generality,
assume $\|f\|_p\neq 0$; otherwise, by Hölder's inequality, $\|fg\|_1\leq
\|f\|_p\|g\|_{p'}=0$ implies $\|fg\|_1=0$ so, by Theorem 5.11, $fg=0$
almost everywhere on $E$ and therefore, $f=0$ almost everywhere on $E$.

Let us prove this for $p=1$. Recall that, by convention, if $p=1$ its
conjugate exponent, $p'$, is $\infty$ and vice versa. Suppose
$\|g\|_\infty\leq 1$ and the integral $\int_E fg$ exists. One direction is
trivial, namely, by Hölder's inequality
\begin{equation}
\label{eq:1:holders-application}
\int_E fg\leq \int_E |fg|\leq\|f\|_1\|g\|_1\leq \|f\|_1,
\end{equation}
for all $g$ with $\|g\|_\infty\leq 1$. Hence,
\[
\sup\int_E fg\leq\|f\|_1.
\]
To get the reverse inequality, consider $g\coloneqq\sgn f$. The function
$g$ is measurable since $g=f/|f|$ for all $f(\bfx)\neq 0$ and $g=0$
otherwise. Moreover, $g$ is in $L^\infty(E)$ since $\|g\|_\infty\leq 1$,
that is, $|g|\leq 1$ almost everywhere on $E$. Therefore
\begin{equation}
  \label{eq:1:reverse-ineq}
\|f\|_1=\int_E |f|=\int_E fg\leq\sup_{\|g'\|_\infty\leq 1}\int_E fg'.
\end{equation}
Thus, $\|f\|_1=\sup\int fg$ where the supremum is taken over all $g\in
L^\infty(E)$ with $\|g\|\leq 1$.

Now, consider the case where $p=\infty$. By Hölder's inequality, it is
clear that
\begin{equation}
\label{eq:1:holders-app-2}
\sup \int_E fg\leq \|f\|_\infty
\end{equation}
since $\int_E fg\leq \|f\|_\infty\|g\|_1$ for all $g\in L(E)$. To prove the
reverse inequality, we consider the cases $\|f\|_\infty<\infty$ and
$\|f\|_\infty=\infty$ separately.

Suppose $0<\|f\|_\infty<\infty$; we may, without loss of generality, assume
$\|f\|_\infty=1$ by normalizing $f$ by its essential supremum. Now, by
definition
\begin{equation}
\label{eq:1:essential-supremum-definition-result}
\|f\|_\infty=\inf\left\{\,\alpha:\left|\left\{\,\bfx\in
      E:f(\bfx)>\alpha\,\right\}\right|=0\,\right\}=1.
\end{equation}
Set $E_k\coloneqq\left\{\,\bfx\in
  E:\text{$f(\bfx)>1-1/k$}\,\right\}\cap B(\mathbf{0},k)$. Then
$E_k\nearrow\bigcup E_k$ and $\left|E\setminus\bigcup E_k\right|=0$ by
Equation \eqref{eq:1:essential-supremum-definition-result} and the
definition of the essential supremum. Therefore, $\int_E fg=\int_{\bigcup
  E_k}fg$. Moreover, $|E_k|<|B(\mathbf{0},k)|<\infty$ so we can define the
sequence of functions
\begin{equation}
  \label{eq:1:special-g}
g_k(\bfx)\coloneqq
\begin{cases}
{\displaystyle\frac{1}{|E_k|}}&\text{if $x\in E_k$}\\
0&\text{otherwise}
\end{cases}.
\end{equation}
Note that $\|g_k\|_1=1$ and
\[
\int_E
fg_k=\int_{E_k}fg_k\geq\int_{E_k}\left(1-\frac{1}{k}\right)g_k
=\left(1-\frac{1}{k}\right)\int_E g_k=1-\frac{1}{k}
\]
\end{proof}

% Therefore, suppose
% \begin{equation}
%   \label{eq:1:hypothesis}
% \|fg\|_1\leq\|f\|_1\|g\|_{\infty}
% \end{equation}
% for every $g\in L^\infty(E)$. We may, without loss of generality, assume
% that $0<\|g\|_\infty\leq 1$, for otherwise, we need only manipulate
% Equation \eqref{eq:1:hypothesis} to get the inequality to this case, e.g.,
% by dividing both sides of the inequality by $\|g\|_\infty$.

%  By the definition of the essential supremum,
% $|g|$ is bounded almost everywhere in $E$ by $\|g\|_\infty$. Then, by
% Theorem 8.8 and Equation \eqref{eq:1:hypothesis}, we have
% \[
% \|fg\|_1\leq \|f\|_1\|g\|_\infty\leq \|f\|_1=\sup_{\|g\|_\infty\leq 1}\int_Efg
% \]
\newpage

\begin{problem}[Wheeden \& Zygmund {\S}8, Ex.\@ 3]
Prove Theorems 8.12 and 8.13. Show that Minkowski’s inequality for series
fails when $p<1$.
\end{problem}
\begin{proof}
Recall the statement of Theorem 8.12
\begin{quote}
Suppose that $1\leq p\leq\infty$, $1/p+1/p'=1$, $a=\{a_k\}$, $b=\{b_k\}$,
and $ab=\{a_kb_k\}$. Then $\|ab\|_1\leq\|a\|_p\|b\|_{p'}$.
\end{quote}
\\\\
Recall the statement of Theorem 8.13
\begin{quote}
Suppose that $1\leq p\leq\infty$, $1/p+1/p'=1$, $a=\{a_k\}$, $b=\{b_k\}$,
and $ab=\{a_kb_k\}$. Then $\|a+b\|_p\leq\|a\|_p+\|b\|_p$.
\end{quote}
\\\\
To see that Minkowski's inequality fails for $p<1$, consider the sequences
$a=(0,1,0,\dotsc)$ and $b=(1,0,\dotsc)$. Then
\[
\begin{aligned}
\|a_k+b_k\|_p&=2^{1/p},\quad&\|a_k\|_p&=1,\quad&\|b_k\|_p&=1.
\end{aligned}
\]
Since $2^{1/p}>2$ for $p<1$, we have
\[
\|a_k+b_k\|_P\geq\|a_k\|_p+\|b_k\|_p.
\]
\end{proof}
\newpage

\begin{problem}[Wheeden \& Zygmund {\S}8, Ex.\@ 4]
Let $f$ and $g$ be real-valued and not identically $0$ (i.e., neither
function equals $0$ a.e.), and let $1<p<\infty$. Prove that equality holds
in the inequality $\left|\int fg\right|\leq\|f\|_p\|g\|_{p'}$ if and only if
$fg$ has constant sign a.e.\@ and $|f|^p$ is a multiple of $|g|^{p'}$ a.e.
\\\\
If $\|f+g\|_p=\|f\|_p+\|g\|_{p}$ and $g\neq 0$ in Minkowski's inequality,
show that $f$ is a multiple of $g$.
\\\\
Find analogues of these results for the spaces $\ell^p$.
\end{problem}
\begin{proof}

  % Take the f and g given in the theorem to be step functions defined in the
  % following way: ( a k x ∈ (k − 1, k] for k ∈ N f (x) = 0 otherwise ( b k x
  % ∈ (k − 1, k] for k ∈ N g(x) = .  0 otherwise Then it is clear that Z |f
  % g| = E ∞ X |a k b k |, k=1 kf k L p = ka k k ` p and kgk L p = kb k k ` p
  % .  Thus the theorem is proven by substitution into Theorem 8.6.
\end{proof}
\newpage

\begin{problem}[Wheeden \& Zygmund {\S}8, Ex.\@ 5]
For $0<p\leq\infty$ and $0<|E|<\infty$, define
\[
N_p[f]\coloneqq\left(\frac{1}{E}\int_E|f|^p\right)^{1/p},
\]
where $N_\infty[f]$ means $\|f\|_\infty$. Prove that if $p_1<p_2$, then
$N_{p_1}[f]\leq N_{p_2}[f]$. Prove also that if $1\leq p\leq \infty$, then
$N_p[f+g]\leq N_p[f]+N_p[g]$, $(1/|E|)\int_E|fg|\leq N_p[f]N_{p'}[g]$,
$1/p+1/p'=1$, and $\lim_{p\to\infty} N_p[f]=\|f\|_\infty$. Thus, $N_p$
behaves like $\|\cdot\|_p$ but has the advantage of being monotone in
$p$. Recall Exercise 28 of Chapter 5.
\end{problem}
\begin{proof}
\end{proof}
\newpage

\begin{problem}[Wheeden \& Zygmund {\S}8, Ex.\@ 6]
\begin{enumerate}[label=(\alph*)]
\item Let $1\leq p_i$, $r\leq\infty$ and $\sum_{i=1}^k1/p_i=1/r$. Prove the
  following generalization of Hölder's inequality:
\[
\|f_1\dotsm f_k\|_r\leq\|f_1\|_{p_1}\dotsm\|f_k\|_{p_k}.
\]
\item Let $1\leq p<r<q\leq\infty$ and define $\theta\in(0,1)$ by
  $1/r=\theta/p+(1-\theta)/q$. Prove the interpolation estimate
\[
\|f\|_r\leq{\|f\|_p}^\theta{\|f\|_q}^{1-\theta}.
\]
In particular, if $A\coloneqq\max\left\{\|f\|_p,\|f\|_q\right\}$, then
$\|f\|_r\leq A$.
\end{enumerate}
\end{problem}
\begin{proof}
(a) We will proceed by induction on $k$ the number of measurable $f_k$
whose $p_k$-norm is finite. When $k=2$, by applying Hölder's inequality on
$|fg|^r$ with $1/(p/r)+1/(p'/r)=1$ we have
\begin{align*}
{\|fg\|_r}^r
&=\left(\int_E|fg|^r\right)\\
&\leq\left(\int_E|f|^{r(p/r)}\right)^{r/p}\left(\int_E|g|^{r(p'/r)}\right)^{r/p'}\\
&={\|f\|_p}^r{\|g\|_{p'}}^r.
\end{align*}
Therefore,
\begin{equation}
  \label{eq:5:general-holder-2}
{\|fg\|_r}\leq\|f\|_p\|g\|_{p'}.
\end{equation}

Now, suppose Equation \eqref{eq:5:general-holder-2} holds for $j\leq n-1$
functions measurable functions $f_j\in L^{p_j}(E)$ where
$\sum_j1/p_j=r$. Suppose $\sum_{j=1}^n1/p_j=1/r$ with $f_j\in L^{p_j}(E)$
and consider
\[
{\|f_1f_2\dotsm f_n\|_r}^r=\int_E|f_1f_2\dotsm
f_n|^r.
\]
Set $g\coloneqq\prod_{j=1}^n f_j$ and
$p'\coloneqq\left(\sum_{j=2}^n1/p_j\right)^{-1}$, then, by
\eqref{eq:5:general-holder-2}, we have
\begin{align*}
\|f_1f_2\dotsm f_n\|_r
&=\|f_1g\|_r\\
&\leq\|f_1\|_{p_1}\|g\|_{p'}\\
&\leq\|f_1\|_{p_1}\|f_2\|_{p_2}\dotsm\|f_n\|_{p_n}
\end{align*}
as desired.
\\\\
(b)
\end{proof}
\newpage

\begin{problem}[Wheeden \& Zygmund {\S}8, Ex.\@ 9]
If $f$ is real-valued and measurable on $E$, $|E|>0$, define its essential
infimum on $E$ by
\[
\essinf f\coloneqq\sup\left\{\,\alpha:\left|\{\,x\in
    E:f(x)<\alpha\,\}\right|=0\,\right\}.
\]
If $f\geq 0$, show that $\essinf_E f=(\esssup 1/f)^{-1}$.
\end{problem}
\begin{proof}
\end{proof}
\newpage

\begin{problem}[Wheeden \& Zygmund {\S}8, Ex.\@ 11]
If $f_k\to f$ in $L^p$, $1\leq p<\infty$, $g_k\to g$ pointwise, and
$\|g_k\|_\infty<M$ for all $k$, prove that $f_kg_k\to fg$ in $L^p$.
\end{problem}
\begin{proof}
\end{proof}

%%% Local Variables:
%%% mode: latex
%%% TeX-master: "../MA544-HW-Current"
%%% End:
