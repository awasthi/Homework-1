\begin{problem}[Wheeden \& Zygmund \S2, Ex.\,1]
Let $f(x)=x\sin(1/x)$ for $0<x\leq 1$ and $f(0)=0$. Show that $f$ is
bounded and continuous on $[0,1]$, but that $V[f;0,1]=+\infty$.
\end{problem}
\begin{proof}
By properties of continuous functions, we have $f$ is continuous on $(0,1]$
since it is the product of continuous functions on $(0,1]$. To see that $f$
is continuous at $0$ is suffices to show that $f(0+)=f(0)=0$. To that end,
let $\left\{x_n\right\}\subset[0,1]$ be a sequence such that $x_n\to 0$ and
consider $\lim_{n\to\infty} f\left(x_n\right)$. Since $x_n\to 0$, for every
$\varepsilon>0$, there exists a natural number $N$ such that $n\geq N$
implies $|0-x_n|<\varepsilon$. Thus, for $n\geq N$ we have
\[
\left|0-f(x_n)\right|=\left|f(x_n)\right|=
\left|x_n\right|\left|\sin(1/x_n)\right|\leq
\varepsilon\left|\sin(1/\varepsilon)\right|\leq
\varepsilon.
\]
Thus, $f(x_n)\to 0$ and we see that $f(0+)=0$. Hence, $f$ is continuous on
$[0,1]$.


It is easy to see that $f$ is bounded since $|\sin(1/x)|\leq 1$ for all
$x\in(0,1]$. More explicitly, we have
\[
|f(x)|\leq |x\sin(1/x)|=|x|\cdot|\sin(1/x)|\leq 1\cdot 1.
\]
Thus, $|f(x)|\leq 1$ and we see that $f$ is bounded.

Moreover, $f$ is continuous on $(0,1]$ since it is the product of
continuous functions on $(0,1]$. To see that $f$ is continuous at $0$, it
suffices to show that $f(0+)=0$. To that end, we shall use the following
limiting argument: Let $\varepsilon>0$ and consider the limit (from the
right) of $f(\varepsilon)$ as $\varepsilon\to 0$. This is
\[
\lim_{\varepsilon\to 0}f(\varepsilon)
\lim_{\varepsilon\to 0}\varepsilon\sin(1/\varepsilon)\leq
\lim_{\varepsilon\to 0}\left|\varepsilon\right|\left|\sin(1/\varepsilon)\right|
\leq\lim_{\varepsilon\to 0}\left|\varepsilon\right|\cdot 1=0.
\]
Thus, $f(0+)=0$ and we see that $f$ is continuous on $[0,1]$.

Last but not least, we show that $f$ is BV. As of now, I've not been able
to find the correct partition to show that the variation of $f$ blows up,
but the informal idea is the following: For any $M$, we (should be able to)
find a partition $\Gamma$ of $[0,1]$ such that $\sin(1/x_i)=1$ for every
$x_i\in\Gamma$, so we have
\[
S_\Gamma=\sum_{i=0}^m x_i\sin(1/x_i)>M
\]
and hence, $V[f;a,b]=+\infty$. I read online and they show, for the very
same problem, that the partition
$\Gamma\coloneqq\left\{x_n\right\}=\left\{((2n+1)\pi/2)^{-1}\right\}$ as
$n\to\infty$ gives $\int_0^1f\diff\phi\to\infty$, but I cannot see how
$\Gamma$ is a partition of the interval $[0,1]$. Anyway, it is fairly clear
that if we have such a partition then
\[
\lim_{n\to\infty}\sum_{i=0}^{n-1}\frac{2}{(2n+1)\pi}
\sin\left(\frac{(2n+1)\pi}{2}\right)=
\lim_{n\to\infty}\sum_{i=0}^{n-1}\frac{2}{(2n+1)\pi}=+\infty,
\]
by the ratio test since
\[
\frac{2/((2n+1)\pi)}{2/((2(n+1)+1)\pi)}=\frac{2n+1}{2n+3}=\frac{n+1}{n+3/2}...
\]
wait! This doesn't work.
\end{proof}
\newpage

\begin{problem}[Wheeden \& Zygmund \S2, Ex.\,2]
Prove theorem (2.1).
\end{problem}
\begin{proof}
Recall the statement of theorem (2.1):
\begin{theorem*}[Wheeden \& Zygmund, 2.1]
\begin{enumerate}[label=(\alph*)]
\item If $f$ is of bounded variation on $[a,b]$, then $f$ is bounded on
  $[a,b]$.
\item Let $f$ and $g$ be of bounded variation on $[a,b]$. Then $cf$ (for
  any real constant $c$), $f+g$, and $fg$ are of bounded variation on
  $[a,b]$. Moreover, $f/g$ is of bounded variation on $[a,b]$ if there
  exists an $\varepsilon>0$ such that $|g(x)|\geq\varepsilon$ for
  $x\in[a,b]$.
\end{enumerate}
\end{theorem*}
\bigskip

(a) We shall proceed by contradiction. Suppose that $f$ is not bounded,
i.e., for every positive real number $M>0$, there exists $x\in[a,b]$ such
that $|f(x)|>M$. In particular, if $V$ is the variation of $f$, then
$|f(x_0)|>V+(f(a)+f(b))/2$ for some $x_0\in[a,b]$. Then, putting
$\Gamma=\left\{a,x_0,b\right\}\subset[a,b]$, we have
\begin{align*}
S_\Gamma&=\left|f(b)-f(x_0)\right|+\left|f(x_0)-f(a)\right|\\
&=\left|f(x_0)-f(b)|+|f(x_0)-f(a)\right|\\
&\geq\left|2f(x_0)-f(a)-f(b)\right|\\
&=\left|2\left(V+(f(a)+f(b))/2\right)-f(a)-f(b)\right|\\
&=\left|2V+f(a)+f(b)-f(a)-f(b)\right|\\
&=2V\\
&>V.
\end{align*}
This is a contradiction since $V$ is the supremum over all such sums.
\\\\
(b) We shall prove these in the order in which they are listed above.
\begin{enumerate}[label=(\roman*)]
\item The constant map $g(x)\coloneqq c$ for some real number $c$ is of
  BV on $[a,b]$ and this is easy to see: take any two partitions
    $\Gamma=\left\{x_0,...,x_m\right\}$, and
    $\Gamma'=\left\{y_0,...,y_n\right\}$ of $[a,b]$,  then
    \[
      S_\Gamma
      =\sum_{i=0}^m\left|g(x_i)-g(x_{i-1})\right|
      =\sum_{i=0}^m\left|ct-c\right|
      =0
      =\sum_{i=0}^m\left|c-c\right|
      =\sum_{i=0}^n\left|g(y_i)-g(y_{i-1})\right|
      =S_{\Gamma'}.
    \]
    It takes just a few more steps in logic to see that
    $V[g;a,b]=0$. Therefore, by (iii) $gf=cf$ is of BV.
\item This result follows quite effortlessly from Jordan's theorem, so we
  shall not trouble ourselves with picking partitions. By Jordan's theorem,
  there exist bounded increasing functions $f_1,f_2$, and $g_1,g_2$ such
  that $f=f_1-f_2$ and $g=g_1-g_2$. Now, since since $f_1,f_2,g_1,g_2$ are
  bounded and increasing, the sums $h_1=f_1+g_1$ and $h_2=f_2+g_2$ are
  bounded and increasing. Thus,
  \[
    f+g=f_1-f_2+g_1-g_2=(f_1+g_1)-(f_2+g_2)=h_1-h_2,
  \]
  so by Jordan's theorem $f+g$ is BV on $[a,b]$.
\item For this result, Jordan's theorem is not very helpful so we rely on
  the definition of BV. First, note that by the triangle inequality, for
  any $x<y$ in $[a,b]$, we have
  \begin{align}
    \label{eq:1}
    |f(x)g(x)-f(y)g(y)|&=|(f(x)g(x)-f(x)g(y))+(f(x)g(y)-f(y)g(y))|\nonumber\\
                       &\leq|f(x)||g(x)-g(y)|+|g(y)||f(x)-f(y)|\nonumber\\
                       &\leq M|g(x)-g(y)|+N|f(x)-f(y)|,
  \end{align}
  by part (a), where $|f|\leq M$ and $|g|\leq M$ for all $x\in[a,b]$. By
  (\ref{eq:1}), it follows that for any partition $\Gamma$ of $[a,b]$, we have
  \[
    S_\Gamma[fg;a,b]\leq
    MS_\Gamma[g;a,b]+NS_\Gamma[f;a,b].
  \]
  Thus, passing to the supremum, we see that
  \[
    V[fg;a,b]\leq MV[g;a,b]+NV[f;a,b]<+\infty,
  \]
  so $fg$ is BV on $[a,b]$.
\item Suppose $|g(x)|>\varepsilon$ for some $\varepsilon>0$ for all
  $x\in[a,b]$. Then, by the triangle inequality, the following estimate
  holds
  \begin{align}
    \label{eq:2}
    \left|\frac{f(x)}{g(x)}-\frac{f(y)}{g(y)}\right|
    &\leq\left|\frac{g(y)f(x)-g(x)f(y)}{g(x)g(y)}\right|\nonumber\\
    &=\frac{1}{|g(x)g(y)|}\left|g(y)f(x)-g(x)f(y)\right|\nonumber\\
    &<\frac{1}{\varepsilon^2}\left|g(y)f(x)-g(x)f(y)\right|\nonumber\\
    &<\frac{1}{\varepsilon^2}\left|g(y)f(x)-g(y)f(y)
                                  +g(y)f(y)-g(x)f(y)\right|\nonumber\\
    &=\frac{1}{\varepsilon^2}\left|(g(y)f(x)-g(y)f(y))
                                  -(g(x)f(y)-g(y)f(y))\right|\nonumber\\
    &\leq\frac{1}{\varepsilon^2}\left(|g(y)|\left|f(x)-f(y)\right|
                                +|f(y)|\left|g(x)-g(y)\right|\right)\nonumber\\
    &\leq\frac{1}{\varepsilon^2}\left(|g(y)|\left|f(x)-f(y)\right|
      +|f(y)|(\left|g(x)\right|-\left|g(y)\right|)\right)\nonumber\\
    &\leq\frac{1}{\varepsilon^2}\left(N|f(x)-f(y)|+M|g(x)-g(y)|\right).
  \end{align}
  Hence, for any partition $\Gamma$ of $[a,b]$, we have
  \[
    S_\Gamma[f/g;a,b]\leq \frac{1}{\varepsilon^2}
                          \left(NS_\Gamma[f;a,b]+MS_\Gamma[g;a,b]\right).
  \]
  Thus, passing to the supremum, we see that
  \[
  V[f/g;a,b]\leq\frac{1}{\varepsilon^2}(NV[f;a,b]+MV[g;a,b])<+\infty,
  \]
  so $f/g$ is BV on $[a,b]$.
\end{enumerate}
\end{proof}
\newpage

\begin{problem}[Wheeden \& Zygmund \S2, Ex.\,3]
If $[a',b']$ is a subinterval of $[a,b]$ show that $P[a',b']\leq P[a,b]$
and $N[a',b']\leq N[a,b]$.
\end{problem}
\begin{proof}
Let $f\colon[a,b]\to\bfR$. If $f$ is unbounded, then $V[f;a,b]=+\infty$
and, by theorem 2.6, the result holds trivially.

Suppose $f$ is BV on $[a,b]$. Then $V[f;a,b]<+\infty$. Hence, by theorem
2.2, we have
\begin{equation}
\label{eq:3}
V[f;a',b']\leq V[f;a,b].
\end{equation}
By theorem 2.6, we have
\begin{align*}
N[f;a',b']&=\tfrac{1}{2}\left(V[f;a',b']+f(b')-f(a')\right)&
P[f;a',b']&=\tfrac{1}{2}\left(V[f;a',b']-f(b')+f(a')\right)\\
\intertext{which, by theorem 2.2, are bounded by}
&\leq\tfrac{1}{2}\left(V[f;a,b]-f(b)+f(a)\right)&
&\leq\tfrac{1}{2}\left(V[f;a,b]-f(b)+f(a)\right)\\
&=N[f;a,b]&
&=P[f;a,b],
\end{align*}
as desired.
\end{proof}
\newpage

\begin{problem}[Wheeden \& Zygmund \S2, Ex.\,11]
Show that $\int_a^bf\diff\phi$ exists if and only if given $\varepsilon>0$
there exists $\delta>0$ such that
$\left|R_\Gamma-R_{\Gamma'}\right|<\varepsilon$ if
$|\Gamma|,|\Gamma'|<\delta$.
\end{problem}
\begin{proof}
$\implies$ Suppose that $I\coloneqq \int_a^bf\diff\phi$ exists. Then, for
every $\varepsilon>0$ there exits $\delta>0$ such that for any partition
$\Gamma''$ of $[a,b]$ with $|\Gamma''|<\delta/2$,
$|I-R_{\Gamma''}|<\varepsilon$. Let $\Gamma$ and $\Gamma'$ be a partitions
with $|\Gamma|,|\Gamma'|<\delta/2$. Then, for the given $\varepsilon$, we
have $|I-R_\Gamma|<\varepsilon$ and $|I-R_{\Gamma'}|<\varepsilon$ from
which we have the estimates
\begin{align*}
|R_\Gamma-R_{\Gamma'}|
&=\left|-(I-R_\Gamma)+(I-R_{\Gamma'})\right|\\
&\leq\left|-(I-R_\Gamma)\right|+\left|I-R_{\Gamma'}\right|\\
&=\left|I-R_\Gamma\right|+\left|I-R_{\Gamma'}\right|\\
&\leq\delta/2+\delta/2\\
&=\delta,
\end{align*}
as desired.

$\impliedby$ Conversely, suppose that given $\varepsilon>0$ there exists
$\delta>0$ such that for any two partitions $\Gamma,\Gamma'$ with
$|\Gamma|,|\Gamma'|<\delta$ we have
$|R_\Gamma-R_{\Gamma'}|<\varepsilon/2$. Put $I\coloneqq\int_a^b
f\diff\phi$. Then, we have the following estimates
\begin{align*}
|I-R_\Gamma|
&=\left|(I-R_{\Gamma'})-(R_{\Gamma}-R_{\Gamma'})\right|\\
&\leq\left|I-R_{\Gamma'}\right|+\left|R_{\Gamma}-R_{\Gamma'}\right|\\
&\leq\left|I-R_{\Gamma'}\right|+\varepsilon/2
\end{align*}

Maybe you do something with a common refinement? Take
$\Gamma''=\Gamma\cup\Gamma'$. Then
\end{proof}
\newpage

\begin{problem}[Wheeden \& Zygmund \S2, Ex.\,13]
Prove theorem (2.16).
\end{problem}
\begin{proof}
\begin{theorem*}[Wheeden \& Zygmund, 2.16]
\begin{enumerate}[label=(\roman*)]
\item If $\int_a^b f\diff\phi$ exists, then so do $\int_a^bcf\diff\phi$ and
  $\int_a^b f\diff(c\phi)$ for any constant $c$, and
\[
\int_a^bcf\diff\phi=\int_a^bf\diff(c\phi)=c\int_a^bf\diff\phi.
\]
\item If $\int_a^b f_1\diff\phi$ and $\int_a^bf_2\diff\phi$ both exist, so
  does $\int_a^b\left(f_1+f_2\right)\diff\phi$, and
\[
\int_a^b\left(f_1+f_2\right)\diff\phi=\int_a^bf_1\diff\phi+\int_a^bf_2\diff\phi.
\]
\item If $\int_a^bf\diff\phi_1$ and $\int_a^bf\diff\phi_2$ both exist, so
  does $\int_a^bf\diff\left(\phi_1+\phi_2\right)$, and
\[
\int_a^bf\diff\left(\phi_1+\phi_2\right)=\int_a^bf\diff\phi_1+\int_a^bf\diff\phi_2.
\]
\end{enumerate}
\end{theorem*}
(i) Suppose that $I\coloneqq \int_a^b f\diff\phi$ exists and let $c$ be a
constant. Then, for every $\varepsilon>0$ there exists $\delta>0$ such that
$|\Gamma|<\delta$ implies $|I-R_\Gamma|<\varepsilon/|c|$. Then, we have
\begin{equation}
\label{eq:4}
R_\Gamma[cf;a,b]=\sum_{i=1}^n cf(\xi_i)\left[\phi(x_i)-\phi(x_{i-1})\right]
=c\left(\sum_{i=1}^nf(\xi_i)\left[\phi(x_i)-\phi(x_{i-1})\right]\right)
=cR_\Gamma
\end{equation}
and
\begin{equation}
\label{eq:5}
R_\Gamma[f;c\phi;a,b]=\sum_{i=1}^n
f(\xi_i)\left[c\phi(x_i)-c\phi(x_{i-1})\right]
=c\left(\sum_{i=1}^nf(\xi_i)\left[\phi(x_i)-\phi(x_{i-1})\right]\right)
=cR_\Gamma
\end{equation}
for $\Gamma=\{x_0,...,x_n\}$.\footnote{The $R_\Gamma[f;c\phi;a,b]$ is just
made up notation. I can't think of what else to call it.} Hence, we have
the estimates
\begin{align*}
\left|cI-R_\Gamma[cf;a,b]\right|
&=\left|cI-cR_\Gamma\right|\\
&=\left|c(I-R_\Gamma)\right|\\
&=|c||I-R_\Gamma|\\
&\leq |c|\left(\varepsilon/|c|\right)\\
&=\varepsilon
\end{align*}
for $\delta$ as given. A similar argument (in fact, the same) works for
$R[f;c\phi;a,b]$. Thus, we have
\[
\int_a^bcf\diff\phi=\int_a^bf\diff(c\phi)=c\int_a^bf\diff\phi,
\]
as desired.
\\\\
(ii) Suppose that $I_1\coloneqq \int_a^b f_1\diff\phi$ and
$I_2\coloneqq\int_a^b f_2\diff\phi$ exits. Then, for every $\varepsilon>0$
there exists $\delta$ such that if $\Gamma$ is a partition of $[a,b]$ with
$|\Gamma|<\delta$ then $\left|I_1-R_\Gamma[f_1;a,b]\right|<\varepsilon/2$
and $\left|I_2-R_\Gamma[f_2;a,b]\right|<\varepsilon/2$. Now, note that
\begin{align}
\label{eq:6}
R_\Gamma[f_1+f_2;a,b]
&=\sum_{i=0}^m(f_1(\xi_i)+f_2(\xi_i))\left[\phi(x_i)-\phi(x_{i-1})\right]\nonumber\\
&=\sum_{i=0}^m\left(f_1(\xi_i)\left[\phi(x_i)-\phi(x_{i-1})\right]
  +f_2(\xi_i)\left[\phi(x_i)-\phi(x_{i-1})\right]\right)\nonumber\\
&=\sum_{i=0}^mf_1(\xi_i)\left[\phi(x_i)-\phi(x_{i-1})\right]
  +\sum_{i=0}^mf_2(\xi_i)\left[\phi(x_i)-\phi(x_{i-1})\right]\nonumber\\
&=R_\Gamma[f_1;a,b]+R_\Gamma[f_2;a,b].
\end{align}
Thus, by (6), we have the following estimates
\begin{align*}
\left|(I_1+I_2)-R_\Gamma[f_1+f_2;a,b]\right|
&=\left|(I_1+I_2)-R_\Gamma[f_1+f_2;a,b]\right|\\
&=\left|(I_1+I_2)-(R_\Gamma[f_1;a,b]+R_\Gamma[f_2;a,b])\right|\\
&=\left|(I_1-R_\Gamma[f_1;a,b])+(I_2-R_\Gamma[f_2;a,b])\right|\\
\intertext{which, by the triangle inequality, is}
&\leq\left|(I_1-R_\Gamma[f_1;a,b])\right|+\left|(I_2-R_\Gamma[f_2;a,b])\right|\\
&\leq\varepsilon/2+\varepsilon/2\\
&=\varepsilon
\end{align*}
or $\delta$ as given. Thus, $\int_a^b f_1+f_2\diff\phi$ exists and is equal
to $\int_a^bf_1\diff\phi+\int_a^bf_2\diff\phi$.
\\\\
(iii) Suppose $I_1\coloneqq\int_a^bf\diff\phi_1$ and
$I_2\coloneqq\int_a^bf\diff\phi_2$ exist then for every $\varepsilon>0$
there exists $\delta_1,\delta_2>0$ such that for every partition
$\Gamma_1,\Gamma_2$ of $[a,b]$ with $|\Gamma_1|<\delta_1$ and
$|\Gamma_2|<\delta_2$ we have
$|I_1-R_{\Gamma_1}[f;\phi_1;a,b]|<\varepsilon/2$ and
$|I_2-R_{\Gamma_2}[f;\phi_2;a,b]|<\varepsilon/2$. Put
$\delta\coloneqq\min\{\delta_1,\delta_2\}$. Now, note that
\begin{align}
  \label{eq:7}
R_{\Gamma}[f;\phi_1+\phi_2;a,b]
&=\sum_{i=0}^mf\left[(\phi_1(x_i)+\phi_2(x_i))
                   -(\phi_1(x_{i-1})+\phi_2(x_{i-1}))\right]\nonumber\\
&=\sum_{i=0}^mf\left[(\phi_1(x_i)-\phi_1(x_{i-1}))
                   +(\phi_2(x_i)-\phi_2(x_{i-1}))\right]\nonumber\\
&=\sum_{i=0}^mf\left[(\phi_1(x_i)-\phi_1(x_{i-1}))\right]
+\sum_{i=0}^mf\left[(\phi_1(x_i)-\phi_1(x_{i-1}))\right]\nonumber\\
&=R_\Gamma[f;\phi_1;a,b]+R_\Gamma[f;\phi_2;a,b].
\end{align}
Hence, we have the following estimates
\begin{align*}
\left|(I_1+I_2)-R_{\Gamma}[f;\phi_1+\phi_2;a,b]\right|
&=\left|(I_1+I_2)-(R_\Gamma[f;\phi_1;a,b]+R_\Gamma[f;\phi_2;a,b])\right|\\
&=\left|(I_1-R_\Gamma[f;\phi_1;a,b])+(I_2-R_\Gamma[f;\phi_2;a,b])\right|\\
\intertext{which, by the triangle inequality, is}
&\leq\left|I_1-R_\Gamma[f;\phi_1;a,b]\right|+
  \left|I_2-R_\Gamma[f;\phi_2;a,b]\right|\\
&<\varepsilon/2+\varepsilon/2\\
&=\varepsilon.
\end{align*}
Thus, $\int_a^bf\diff(\phi_1+\phi_2)$ exists and it is equal to the sum
$\int_a^bf\diff\phi_1+\int_a^bf\diff\phi_2$.
\end{proof}
\newpage

%%% Local Variables:
%%% mode: latex
%%% TeX-master: "../MA544-HW-Current"
%%% End:
