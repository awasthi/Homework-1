\begin{problem}[Wheeden \& Zygmund \S2, Ex.\,1]
Let $f(x)=x\sin(1/x)$ for $0<x\leq 1$ and $f(0)=0$. Show that $f$ is
bounded and continuous on $[0,1]$, but that $V[f;0,1]=+\infty$.
\end{problem}
\begin{proof}
Moreover, $f$ is continuous on $(0,1]$ since it is the product of
continuous functions on $(0,1]$. To see that $f$ is continuous at $0$ is
suffices to show that $f(0+)=f(0)=0$. To that end, let
$\left\{x_n\right\}\subset[0,1]$ be a sequence such that $x_n\to 0$ and
consider $\lim_{n\to\infty} f\left(x_n\right)$. Since $x_n\to 0$, for every
$\varepsilon>0$, there exists a natural number $N$ such that $n\geq N$
implies $|0-x_n|<\varepsilon$. Thus, for $n\geq N$ we have
\[
\left|0-f(x_n)\right|=\left|f(x_n)\right|=
\left|x_n\right|\left|\sin(1/x_n)\right|\leq
\varepsilon\left|\sin(1/\varepsilon)\right|\leq
\varepsilon.
\]
Thus, $f(x_n)\to 0$ and we see that $f(0+)=0$. Hence, $f$ is continuous on
$[0,1]$.


It is easy to see that $f$ is bounded since $|\sin(1/x)|\leq 1$ for all
$x\in(0,1]$. More explicitly, we have that
\[
|f(x)|\leq |x\sin(1/x)|=|x|\cdot|\sin(1/x)|\leq 1\cdot 1.
\]
Thus, $|f(x)|\leq 1$ and we see that $f$ is bounded.

Moreover, $f$ is continuous on $(0,1]$ since it is the product of
continuous functions on $(0,1]$. To see that $f$ is continuous at $0$, it
suffices to show that $f(0+)=0$. To that end, we shall use the following
limiting argument: Let $\varepsilon>0$ and consider the limit (from the
right) of $f(\varepsilon)$ as $\varepsilon\to 0$. This is
\[
\lim_{\varepsilon\to 0}f(\varepsilon)
\lim_{\varepsilon\to 0}\varepsilon\sin(1/\varepsilon)\leq
\lim_{\varepsilon\to 0}\left|\varepsilon\right|\left|\sin(1/\varepsilon)\right|
\leq\lim_{\varepsilon\to 0}\left|\varepsilon\right|\cdot 1=0.
\]
Thus, $f(0+)=0$ and we see that $f$ is continuous on $[0,1]$.

Last but not least, we show that $f$ is BV. Define the family of partitions
$\left\{\Gamma_n\right\}_{n=1}^\infty$ by $x_i\coloneqq$
\end{proof}
\newpage

\begin{problem}[Wheeden \& Zygmund \S2, Ex.\,2]
Prove theorem (2.1).
\end{problem}
\begin{proof}
Recall the statement of theorem (2.1):
\begin{theorem*}[Wheeden \& Zygmund, 2.1]
\begin{enumerate}[label=(\alph*)]
\item If $f$ is of bounded variation on $[a,b]$, then $f$ is bounded on
  $[a,b]$.
\item Let $f$ and $g$ be of bounded variation on $[a,b]$. Then $cf$ (for
  any real constant $c$), $f+g$, and $fg$ are of bounded variation on
  $[a,b]$. Moreover, $f/g$ is of bounded variation on $[a,b]$ if there
  exists an $\varepsilon>0$ such that $|g(x)|\geq\varepsilon$ for
  $x\in[a,b]$.
\end{enumerate}
\end{theorem*}
\bigskip

(a) We shall proceed by contradiction. Suppose that $f$ is not bounded,
i.e., for every positive real number $M>0$, there exists $x\in[a,b]$ such
that $|f(x)|>M$. In particular, if $V$ is the variation of $f$, then
$|f(x_0)|>V+(f(a)+f(b))/2$ for some $x_0\in[a,b]$. Then, putting
$\Gamma=\left\{a,x_0,b\right\}\subset[a,b]$, we have
\begin{align*}
S_\Gamma&=\left|f(b)-f(x_0)\right|+\left|f(x_0)-f(a)\right|\\
&=\left|f(x_0)-f(b)|+|f(x_0)-f(a)\right|\\
&\geq\left|2f(x_0)-f(a)-f(b)\right|\\
&=\left|2\left(V+(f(a)+f(b))/2\right)-f(a)-f(b)\right|\\
&=\left|2V+f(a)+f(b)-f(a)-f(b)\right|\\
&=2V\\
&>V.
\end{align*}
This is a contradiction since $V$ is the supremum over all such sums.
\\\\
(b) Let $f$ and $g$ be BV on $[a,b]$ and $c$ a real number. Then, for every
partition $\Gamma$ of $[a,b]$, we have $V[f;a,b]\geq S_\Gamma[f;a,b]$
\end{proof}
\newpage

\begin{problem}[Wheeden \& Zygmund \S2, Ex.\,3]
If $[a',b']$ is a subinterval of $[a,b]$ show that $P[a',b']\leq P[a,b]$
and $N[a',b']\leq N[a,b]$.
\end{problem}
\begin{proof}
\end{proof}
\newpage

\begin{problem}[Wheeden \& Zygmund \S2, Ex.\,11]
Show that $\int_a^bf\diff\phi$ exists if and only if given $\varepsilon>0$
there exists $\delta>0$ such that
$\left|R_\Gamma-R_{\Gamma'}\right|<\varepsilon$ if
$|\Gamma|,|\Gamma'|<\delta$.
\end{problem}
\begin{proof}
\end{proof}
\newpage

\begin{problem}[Wheeden \& Zygmund \S2, Ex.\,13]
Prove theorem (2.16).
\end{problem}
\begin{proof}
\begin{theorem*}[Wheeden \& Zygmund, 2.16]
\begin{enumerate}[label=(\roman*)]
\item If $\int_a^b f\diff\phi$ exists, then so do $\int_a^bcf\diff\phi$ and
  $\int_a^b f\diff(c\phi)$ for any constant $c$, and
\[
\int_a^bcf\diff\phi=\int_a^bf\diff(c\phi)=c\int_a^bf\diff\phi.
\]
\item If $\int_a^b f_1\diff\phi$ and $\int_a^bf_2\diff\phi$ both exist, so
  does $\int_a^b\left(f_1+f_2\right)\diff\phi$, and
\[
\int_a^b\left(f_1+f_2\right)\diff\phi=\int_a^bf_1\diff\phi+\int_a^bf_2\diff\phi.
\]
\item If $\int_a^bf\diff\phi_1$ and $\int_a^bf\diff\phi_2$ both exist, so
  does $\int_a^bf\diff\left(\phi_1+\phi_2\right)$, and
\[
\int_a^bf\diff\left(\phi_1+\phi_2\right)=\int_a^bf\diff\phi_1+\int_a^bf\diff\phi_2.
\]
\end{enumerate}
\end{theorem*}
\end{proof}
\newpage

%%% Local Variables:
%%% mode: latex
%%% TeX-master: "../MA544-HW-Current"
%%% End:
