\begin{problem}[Wheeden \& Zygmund \S2, Ex.\,1]
Let $f(x)=x\sin(1/x)$ for $0<x\leq 1$ and $f(0)=0$. Show that $f$ is
bounded and continuous on $[0,1]$, but that $V[f;0,1]=+\infty$.
\end{problem}
\begin{proof}
% It is straightforward to see that $f$ is bounded and continuous on
% $[0,1]$. To see that $f$ is continuous we will appeal to an
% $\varepsilon$-$\delta$ argument. Let $\varepsilon>0$, then for $\delta>$
% for any $x\in[0,1]$, any $y\in(\delta-x,x+\delta)$ we have
% \begin{align*}
% \left|f(x)-f(y)\right|&\leq|x\sin(1/x)-y\sin(1/y)|\\
% &=\left|(x-y)\left(\sin(1/x)-\sin(1/y)\right)\right|\\
% &=|x-y|\left|\sin(1/x)-\sin(1/y)\right|\\
% &\leq\delta\cdot|\sin(1/x)-\sin(1/y)\right|\\
% \intertext{now recall the Taylor expansion of $\sin$ about $0$,
%   $\sin(x)=\sum_{k=0}^\infty\frac{(-1)^k}{(2k+1)!}x^{2k+1}$, then}
% \end{align*}

\end{proof}
\newpage

\begin{problem}[Wheeden \& Zygmund \S2, Ex.\,2]
Prove theorem (2.1).
\end{problem}
\begin{proof}
Recall the statement of theorem (2.1):
\begin{theorem*}[Wheeden \& Zygmund, 2.1]
\begin{enumerate}[label=(\alph*)]
\item If $f$ is of bounded variation on $[a,b]$, then $f$ is bounded on
  $[a,b]$.
\item Let $f$ and $g$ be of bounded variation on $[a,b]$. Then $cf$ (for
  any real constant $c$), $f+g$, and $fg$ are of bounded variation on
  $[a,b]$. Moreover, $f/g$ is of bounded variation on $[a,b]$ if there
  exists an $\varepsilon>0$ such that $|g(x)|\geq\varepsilon$ for
  $x\in[a,b]$.
\end{enumerate}
\end{theorem*}
\end{proof}
\newpage

\begin{problem}[Wheeden \& Zygmund \S2, Ex.\,3]
If $[a',b']$ is a subinterval of $[a,b]$ show that $P[a',b']\leq P[a,b]$
and $N[a',b']\leq N[a,b]$.
\end{problem}
\begin{proof}
\end{proof}
\newpage

\begin{problem}[Wheeden \& Zygmund \S2, Ex.\,11]
Show that $\int_a^bf\diff\phi$ exists if and only if given $\varepsilon>0$
there exists $\delta>0$ such that
$\left|R_\Gamma-R_{\Gamma'}\right|<\varepsilon$ if
$|\Gamma|,|\Gamma'|<\delta$.
\end{problem}
\begin{proof}
\end{proof}
\newpage

\begin{problem}[Wheeden \& Zygmund \S2, Ex.\,13]
Prove theorem (2.16).
\end{problem}
\begin{proof}
\begin{theorem*}[Wheeden \& Zygmund, 2.16]
\begin{enumerate}[label=(\roman*)]
\item If $\int_a^b f\diff\phi$ exists, then so do $\int_a^bcf\diff\phi$ and
  $\int_a^b f\diff(c\phi)$ for any constant $c$, and
\[
\int_a^bcf\diff\phi=\int_a^bf\diff(c\phi)=c\int_a^bf\diff\phi.
\]
\item If $\int_a^b f_1\diff\phi$ and $\int_a^bf_2\diff\phi$ both exist, so
  does $\int_a^b\left(f_1+f_2\right)\diff\phi$, and
\[
\int_a^b\left(f_1+f_2\right)\diff\phi=\int_a^bf_1\diff\phi+\int_a^bf_2\diff\phi.
\]
\item If $\int_a^bf\diff\phi_1$ and $\int_a^bf\diff\phi_2$ both exist, so
  does $\int_a^bf\diff\left(\phi_1+\phi_2\right)$, and
\[
\int_a^bf\diff\left(\phi_1+\phi_2\right)=\int_a^bf\diff\phi_1+\int_a^bf\diff\phi_2.
\]
\end{enumerate}
\end{theorem*}
\end{proof}
\newpage

%%% Local Variables:
%%% mode: latex
%%% TeX-master: "../MA544-HW-Current"
%%% End:
