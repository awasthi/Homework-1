% #9 - Due Mar. 21 Read Section 6.2-3. Chapter 6: #  1, 3, 4 (6, 7, 9).

\begin{problem}[Wheeden \& Zygmund {\S}6, Ex.\@ 1]
\begin{enumerate}[label=(\alph*)]
\item Let $E$ be a measurable subset of $\bbR^2$ such that for almost every
  $x\in\bbR^1$, $\left\{\,y:(x,y)\in E\,\right\}$ has
  $\bbR^1$-measure zero. Show that $E$ has measure zero and that for almost
  every $y\in\bbR^1$, $\left\{\,x:(x,y)\in E\,\right\}$ has
  measure zero.
\item Let $f(x,y)$ be nonnegative and measurable in $\bbR^2$. Suppose that
  for almost every $x\in\bbR^1$, $f(x,y)$ is finite for almost every
  $y$. Show that for almost $y\in\bbR^1$, $f(x,y)$ is finite for almost
  every $x$.
\end{enumerate}
\end{problem}
\begin{proof}
(a) That $E$ has measure zero is a consequence of Fubini's theorem. Set
$E_x\coloneqq\left\{\,y:(x,y)\in E\,\right\}$ and
$E_y\coloneqq\left\{\,x:(x,y)\in E\,\right\}$ then, by Theorem 6.8, we have
\begin{equation}
\label{eq:consequence-of-fubini}
|E|=
\iint_{\bbR^2}\chi_E\diff x\diff y=
\int_\bbR\left[\int_{E_x}1\diff y\right]\diff x=
\int_\bbR\biggl[\int_{E_y}1\diff x\biggr]\diff y=
0.
\end{equation}
Hence, $E$ has measure zero. Moreover, we see that
$\int_\bbR\bigl[\int_{E_y}1\diff x\bigr]\diff y=0$ which means that for
a.e.\@ $y\in\bbR$, $E_y$ has $\bbR^1$-measure zero.
\\\\
(b) Let $E$ be the set of all pairs $(x,y)\in\bbR^2$ such that $f(x,y)$ is
not finite. By hypothesis, the set $E_x$ has $\bbR^1$-measure zero for
a.e.\@ $x$. Therefore, by part (a) the set $E_y$ has measure zero. Hence,
for a.e.\@ $y$, $f(x,y)$ is finite for a.e.\@ $x$.
\end{proof}
\newpage

\begin{problem}[Wheeden \& Zygmund {\S}6, Ex.\@ 3]
Let $f$ be measurable and finite a.e.\@ on $[0,1]$. If $f(x)-f(y)$ is
integrable over the square $0\leq x\leq 1$, $0\leq y\leq 1$, show that
$f\in L[0,1]$.
\end{problem}
\begin{proof}
Set $I\coloneqq[0,1]$. Suppose that $f(x)-f(y)\in L(I\times I)$. Then by
Fubini's theorem we have
\begin{equation}
  \label{eq:double-integral}
\iint_{I\times I}f(x)-f(y)\diff x\diff y=
\int_I\left[\int_I f(x)-f(y)\diff x\right]\diff y=
\int_I\left[\int_I f(x)-f(y)\diff y\right]\diff x<\infty.
\end{equation}
Hence, for a.e. $y\in\bbR$, $f(x)-f(y)$ is integrable so $f(x)$ is
integrable.
\end{proof}
\newpage

\begin{problem}[Wheeden \& Zygmund {\S}6, Ex.\@ 4]
Let $f$ be measurable and periodic with period $1$: $f(t+1)=f(t)$. Suppose
there is a finite $c$ such that
\[
\int_0^1\left|f(a+t)-f(b+t)\right|\diff t\leq c
\]
for all $a$ and $b$. Show that $f\in L[0,1]$. (Set $a=x$, $b=-x$, integrate
with respect to $x$, and make the change of variables $\xi=x+t$,
$\eta=-x+t$.)
\end{problem}
\begin{proof}
Following the hint, write
\begin{align*}
c&\geq\int_0^1\int_0^1|f(x+t)-f(-x+t)|\diff x\diff t
\intertext{making the change of variables $\xi=x+t$, $\eta=-x+t$ and
  appropriate modification to the bounds of integration, i.e.,
  $0\leq\xi\leq 2$, $-1\leq\eta\leq 1$ we have}
 &=\int_{-1}^1\int_0^2|f(\xi)-f(\eta)|(\det\Jac(\xi,\eta))\diff\xi\diff\eta
\shortintertext{by Fubini's theorem}
 &=\int_0^2\int_{-1}^1|f(\xi)-f(\eta)|(\det\Jac(\xi,\eta))\diff\eta\diff\xi
\intertext{where
$
\Jac(\xi,\eta)=
\left[
\begin{smallmatrix}
\partial x/\partial\xi&\partial x/\partial\eta\\
\partial t/\partial\xi&\partial t/\partial\eta
\end{smallmatrix}
\right]
=
\left[
\begin{smallmatrix}
1/2&-1/2\\
1/2&1/2
\end{smallmatrix}
\right]
$ is the Jacobian of the linear transformation which sends
the pair $(\xi,\eta)$ to $(1/2(\xi-\eta),1/2(\xi+\eta))$, hence we have}
&=\frac{1}{2}\int_0^2\int_{-1}^1|f(\xi)-f(\eta)|\diff\xi\diff\eta\\
&=\frac{1}{2}\int_0^2\int_{-1}^0|f(\xi)-f(\eta)|\diff\xi\diff\eta+
+\frac{1}{2}\int_0^2\int_0^1|f(\xi)-f(\eta)|\diff\xi\diff\eta
\intertext{
Here we use Theorem 3.35 to note that the translation $\eta\mapsto\eta+1$
and the fact that $f$ is periodic with period $1$ gives us
}
&=\int_0^2\int_0^1|f(\xi)-f(\eta)|\diff\xi\diff\eta
\intertext{similarly, we have}
&=2\int_0^1\int_0^1|f(\xi)-f(\eta)|\diff\xi\diff\eta.
\end{align*}
Hence, the inequality
\begin{equation}
  \label{eq:desired-inequality}
\int_0^1\int_0^1|f(\xi)-f(\eta)|\diff\xi\diff\eta\leq\frac{c}{2}
\end{equation}
holds so by Problem 9.2 (\S6, Ex.\@ 3), $|f|\in L[0,1]$ hence, $f\in
L[0,1]$.
\end{proof}
\newpage

\begin{problem}[Wheeden \& Zygmund {\S}6, Ex.\@ 6]
For $f\in L(\bbR^1)$, define the \emph{Fourier transform $\hat f$} of $f$
by
\[
\hat f(x)=\frac{1}{2\pi}\int_{-\infty}^\infty f(t)e^{-ixt}\diff t
\]
for $x\in\bbR^1$. (For complex-valued function $F=F_0+iF_1$ whose real and
imaginary parts $F_0$ and $F_1$ are integrable, we define $\int F=\int
F_0+i\int F_1$.) Show that if $f$ and $g$ belong to $L(\bbR^1)$, then
\[
\widehat{(f*g)}(x)=2\pi\hat f(x)\hat g(x).
\]
\end{problem}
\begin{proof}
By direct computation we have
\begin{align*}
\widehat{(f*g)}(x)
&=\frac{1}{2\pi}\int_{-\infty}^\infty\left[\int_{-\infty}^\infty
  f(s-t)g(t)\diff t\right]e^{-ixs}\diff  s
\intertext{now do this}
&=\frac{1}{2\pi}\int_{-\infty}^\infty\int_{-\infty}^\infty
  f(s-t)g(t)e^{-ixs}\diff t\diff s
\intertext{make the substitution $u=s-t$, then the above becomes}
&=\frac{1}{2\pi}\int_{-\infty}^\infty\int_{-\infty}^\infty
  f(u)g(t)e^{-ix(u+t)}\diff t\diff u\\
&=\frac{1}{2\pi}\int_{-\infty}^\infty\int_{-\infty}^\infty
  f(u)e^{-ixu}g(t)e^{-ixt}\diff t\diff u
\intertext{by Fubini's theorem, this is just}
&=2\pi\left(\frac{1}{2\pi}\int_{-\infty}^\infty f(u)e^{-ixu}\diff u\right)
  \left(\frac{1}{2\pi}\int_{-\infty}^\infty g(t)e^{-ixt}\diff t\right)\\
&=2\pi\hat f(x)\hat g(x)
\end{align*}
as desired.
\end{proof}
\newpage

\begin{problem}[Wheeden \& Zygmund {\S}6, Ex.\@ 7]
Let $F$ be a closed subset of $\bbR^1$ and let $\delta(x)=\delta(x,F)$ be
the corresponding distance function. If $\lambda>0$ and $f$ is nonnegative
and integrable over the complement of $F$, prove that the function
\[
\int_{\bbR^1}\frac{\delta^\lambda(y)f(y)}{\left|x-y\right|^{1+\lambda}}\diff
t
\]
is integrable over $F$ and so is finite a.e.\@ in $F$. (In case
$f=\chi_{(a,b)}$, this reduces to Theorem 6.17.)
\end{problem}
\begin{proof}
Set $G\coloneqq\bbR\minus F$. By assumption, we have
\begin{equation}
  \label{eq:hypothesis-of-integrability}
\int_Gf(x)\diff x<\infty.
\end{equation}
By Tonelli's theorem, since $\delta(y)=0$ for $y\in F$, we have
\begin{equation}
\label{eq:application-of-tonelli}
\begin{aligned}[t]
\int_F\left[\int_\bbR\frac{\delta^\lambda(y)f(y)}{|x-y|^{1+\lambda}}\diff
  y\right]\diff x
&=\int_F\left[\int_G\frac{\delta^\lambda(y)f(y)}{|x-y|^{1+\lambda}}\diff
  y\right]\diff x\\
&=\int_G\delta^\lambda(y)f(y)\left[\int_F\frac{\diff
    x}{|x-y|^{1+\lambda}}\right]\diff y.
\end{aligned}
\end{equation}
Now, by Marcinkiewwicz's theorem, we have
\begin{equation}
  \label{eq:application-of-marcinkiewicz}
\int_F\frac{\diff x}{|x-y|^{1+\lambda}}\leq
2\lambda^{-1}\delta(y)^{-\lambda}.
\end{equation}
Then, by \eqref{eq:hypothesis-of-integrability}, we have
\begin{equation}
\label{eq:application-of-fubini-2}
\begin{aligned}[t]
 \int_F\left[\int_\bbR\frac{\delta^{\lambda}(y)f(y)}{|x-y|^{1+\lambda}}\diff
y\right]\diff x
&\leq\int_G\delta^\lambda(y)f(y)\left[2\lambda^{-1}\delta(y)^{-\lambda}\right]\diff
y\\
&=2\lambda^{-1}\int_Gf(y)\diff y\\
&<\infty
\end{aligned}
\end{equation}
as desired.
\end{proof}
\newpage

\begin{problem}[Wheeden \& Zygmund {\S}6, Ex.\@ 9]
\begin{enumerate}[label=(\alph*)]
\item Show that $M_\lambda(x;F)=+\infty$ if $x\notin F$, $\lambda>0$.
\item Let $F=[c,d]$ be a closed subinterval of a bounded open interval
  $(a,b)\subset\bbR^1$, and let $M_\alpha$ be the corresponding
  Marcinkiewicz integral, $\lambda>0$. Show that $M_\lambda$ is finite for
  every $x\in(c,d)$ and that $M_\lambda(c)=M_\lambda(d)=\infty$. Show also
  that $\int M_\lambda\leq\lambda^{-1}|G|$, where $G=(a,b)-[c,d]$.
\end{enumerate}
\end{problem}
\begin{proof}
(a) Put $G\coloneqq(a,b)\minus F$. Since $\delta(y)=0$ for $y\in F$, by
Tonelli's theorem we have
\begin{equation}
  \label{eq:application-of-tonelli-2}
M_\lambda(x)=\int_G\frac{\delta^\lambda(y)}{|x-y|^{1+\lambda}}\diff y.
\end{equation}
If $x\notin F$, then since $G$ is open, there exists a sufficiently small
$\varepsilon>0$ such that $B_\varepsilon(x)\subset G$ and
$m\coloneqq\inf_{y\in B_\varepsilon(x)}\delta(y)>0$. Since
$\delta^\lambda(y)/|x-y|^{1+\lambda}$ is nonnegative, we have
\begin{align*}
\int_G\frac{\delta^\lambda(y)}{|x-y|^{1+\lambda}}\diff y
&\geq
  \int_{B_\varepsilon(x)}\frac{\delta^\lambda(y)}{|x-y|^{1+\lambda}}\diff
  y\\
&\geq m^\lambda\int_{|x-y|<\varepsilon}\frac{1}{|x-y|^{1+\lambda}}\diff y\\
&=2m^\lambda\int_0^\varepsilon\frac{1}{u^{1+\lambda}}\diff u\\
&=\left[2m^\lambda\lambda^{-1}u^{-\lambda}\right]_0^\varepsilon\\
&=\infty.
\end{align*}
\\\\
(b)
\end{proof}

%%% Local Variables:
%%% mode: latex
%%% TeX-master: "../MA544-HW-Current"
%%% End:
