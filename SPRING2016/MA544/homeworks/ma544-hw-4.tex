% #4 - Due Feb. 8 Read Sections 3.5, 4.1. Chapter 3: # 12, 13, 14, 15 (16,
% 18), (Chapter 4: # 1, 2

\begin{problem}[Wheeden \& Zygmund {\S}3, Ex.\@ 12]
If $E_1$ and $E_2$ are measurable sets in $\bbR^1$, show $E_1\times E_2$ is
a measurable subset of $\bbR^2$ and $\left|E_1\times
  E_2\right|=\left|E_1\right|\left|E_2\right|$. (Interpret $0\cdot\infty$
as $0$.) [\textsc{Hint:} Use a characterization of measurability.]
\end{problem}
\begin{proof}
By (3.28) (i) we may write $E_1$ and $E_2$ as the unions $H_1\cup Z_1$ and
$H_2\cup Z_2$, respectively, where $H_1$ and $H_2$ are $F_\sigma$ and $Z_1$
and $Z_2$ are measure zero. Now, by elementary set theory, the Cartesian
product $E_1\times E_2$ can then be written as
\begin{equation}
\label{eq:1}
E_1\times E_2=(H_1\cup Z_1)\times (H_2\cup Z_2)=
\underbrace{H_1\times H_2}_H\cup
\underbrace{H_1\times Z_2\cup Z_1\times H_2\cup Z_1\times Z_2}_Z.
\end{equation}
Hence, we win by (3.28) (i) if we can show that the Cartesian of two
$F_\sigma$ sets is an $F_\sigma$ set and if the Cartesian product of a
measurable set with a set of measure zero is measure zero.

First, we prove the former, since the argument to be made is little more
than elementary set theory. Let $F_1$ and $F_2$ be $F_\sigma$. Write
$F_1=\bigcup F_k'$ and $F_2=\bigcup F_k''$ where the $F_k'$'s and the
$F_k''$'s are closed subsets of $\bbR$. Then, $F_k'\times F_\ell''$ are
closed subsets of $\bbR^2$ by elementary topology. Moreover,
$F_k'\times F_\ell''\subset F_1\times F_2$ hence,
$\bigcup_{k,\ell} F_k'\times F_\ell''\subset F_1\times F_2$. Thus, it
suffices to show that
$\bigcup_{k,\ell} F_k'\times F_\ell''\supset F_1\times F_2$. Let
$(x,y)\in F_1\times F_2$. Then $x\in F_1$ and $y\in F_2$. But since
$F_1=\bigcup F_k'$ and $F_2=\bigcup F_k''$ then $x\in F_k'$ and
$x\in F_\ell''$ for some $k$, $\ell$. In other words,
$(x,y)\in F_k'\times F_\ell''$ so $(x,y)$ is in the union
$\bigcup_{k,\ell}F_k'\times F_\ell''$. Hence, we have
$F_1\times F_2=\bigcup_{k,\ell} F_k'\times F_\ell''$. We conclude that if
$F_1$ and $F_2$ are $F_\sigma$, then so is their Cartesian product
$F_1\times F_2$.

Let $E$ be a measurable set with $\left|E\right|<\infty$ and $Z$ a set of
measure zero. Then, for every $\varepsilon>0$ there exists a countable
collection of intervals $\left\{I_k\right\}$ containing $Z$ such that
$\sum\Vol\left(I_k\right)<\varepsilon$. Similarly, we can find a collection
$\left\{I_k'\right\}$ of intervals containing $E$ such that
$\sum\Vol\left(I_k'\right)<\left|E\right|+\varepsilon$. Then,
$\left\{I_k'\times I_\ell\right\}$ is a countable collection of
$2$-intervals containing $E\times Z$ with
\begin{align*}
\sum_{k,\ell}\Vol\left(I_k'\times I_\ell\right)
&=\sum_{k,\ell}\Vol\left(I_k'\right)\Vol\left(I_\ell\right)\\
&=\sum_k\sum_\ell\Vol\left(I_k'\right)\Vol\left(I_\ell\right)\\
&=\left(\sum_k\Vol\left(I_k'\right)\right)
\left(\sum_\ell\Vol\left(I_\ell\right)\right)\\
&=\left(\left|E\right|+\varepsilon\right)\varepsilon
\end{align*}
Letting $\varepsilon\to 0$, we have $E\times Z$ is measure zero. If
$\left|E\right|=\infty$, partition $E$ into disjoint finite measure subsets
of $\bbR$ by taking the following intersection
\[
E_k=E\cap\left(B(0,k)\setminus B(0,k-1)\right)
\]
for $k\in\bbN$.\footnote{In fact, it might be quicker from now on to quote
  the fact that $\bbR^n$ is $\sigma$-finite.} By our previous argument,
$E_k\times Z$ is measure zero so $\left\{E_k\times Z\right\}$ is a cover of
$E\times Z$ hence, by (3.24), we have
\begin{align*}
\left|E\times Z\right|
&=\left|\left(\bigcup_k E_k\right)\times Z\right|\\
&=\left|\bigcup_k E_k\times Z\right|\\
&=\sum_k \left|E_k\times Z\right|\\
&=0.
\end{align*}
Thus, $E\times Z$ is measure zero.

Hence, $E_1\times E_2$ is measurable with $\left|E_1\times
E_2\right|=\left|H_1\times H_2\right|$. It's left to show is
that $\left|H_1\times H_2\right|=\left|H_1\right|\left|H_2\right|$.

Let $H_1$ and $H_2$ be $F_\sigma$ sets of finite measure. Then, for every
$\varepsilon>0$, there exists a collection of intervals
$\left\{I_k\right\}$ and $\left\{I_k'\right\}$ covering $H_1$ and $H_2$
respectively such that
\begin{align*}
\sum_k\Vol\left(I_k\right)&<\left|H_1\right|+\varepsilon&
\sum_k\Vol\left(I_k'\right)&<\left|H_2\right|+\varepsilon.
\end{align*}
Then the collection $\left\{I_k\times I_\ell'\right\}$ is a
cover of $H_1\times H_2$ by $2$-intervals and we have
\end{proof}
\newpage

\begin{problem}[Wheeden \& Zygmund {\S}3, Ex.\@ 13]
Motivated by (3.7), define the \emph{inner measure} of $E$ by
$\left|E\right|_i=\sup\left|F\right|$, where the supremum is taken over all
closed subsets $F$ of $E$. Show that
\begin{enumerate}[label=(\roman*)]
\item $\left|E\right|_i\leq\left|E\right|_e$, and
\item if $\left|E\right|_e<+\infty$, then $E$ is measurable if and only if
  $\left|E\right|_i=\left|E\right|_e$.
\end{enumerate}
[Use (3.22).]
\end{problem}
\begin{proof}
(i) If the outer measure of $E$ is infinite, the inequality holds
trivially. Suppose $|E|_e<\infty$. Since closed sets are measurable and
their outer measure is equal to their Lebesgue measure, then we may replace
$\left|F\right|$ by $\left|F\right|_e$ to mirror the definition of the
outer-measure and, by the monotonicity of the outer measure, we have
\begin{equation}
\label{eq:monotonicity-outer-measure}
\left|F\right|=\left|F\right|_e\leq\left|E\right|_e.
\end{equation}
Taking the supremum on both sides of (\ref{eq:monotonicity-outer-measure}),
we obtain the desired inequality
\begin{equation}
\label{eq:inner-outer-measure}
\left|E\right|_i\leq\left|E\right|_e.
\end{equation}
\\\\
(ii) $\implies$ Suppose $E$ is measurable with
$\left|E\right|<\infty$. Then for every $\varepsilon>0$ there exists an
open set $G\supset E$ such that
$\left|G\right|_e<\left|E\right|_e+\varepsilon$.

$\impliedby$ Suppose that $\left|E\right|_i=\left|E\right|_e$. Then
\end{proof}
\newpage

\begin{problem}[Wheeden \& Zygmund {\S}3, Ex.\@ 14]
  Show that the conclusion of part (ii) of Exercise 13 is false if
  $\left|E\right|_e=+\infty$.
\end{problem}
\begin{proof}
\end{proof}
\newpage


\begin{problem}[Wheeden \& Zygmund {\S}3, Ex.\@ 15]
If $E$ is measurable and $A$ is any subset of $E$, show that
$\left|E\right|=\left|A\right|_i+\left|E-A\right|_e$. (See Exercise 13 for
the definition of $\left|A\right|_i$.)
\end{problem}
\begin{proof}
\end{proof}

%%% Local Variables:
%%% mode: latex
%%% TeX-master: "../MA544-HW-Current"
%%% End:
