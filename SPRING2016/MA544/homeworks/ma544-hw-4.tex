% #4 - Due Feb. 8 Read Sections 3.5, 4.1. Chapter 3: # 12, 13, 14, 15 (16,
% 18), (Chapter 4: # 1, 2

\begin{problem}[Wheeden \& Zygmund {\S}3, Ex.\@ 12]
If $E_1$ and $E_2$ are measurable sets in $\bfR^1$, show $E_1\times E_2$ is
a measurable subset of $\bfR^2$ and $\left|E_1\times
  E_2\right|=\left|E_1\right|\left|E_2\right|$. (Interpret $0\cdot\infty$
as $0$.) [\textsc{Hint:} Use a characterization of measurability.]
\end{problem}
\begin{proof}
By (3.28) (i) we may write $E_1$ and $E_2$ as the set difference $H_1\minus
Z_1$ and $H_2\minus Z_2$, respectively, where $H_1$ and $H_2$ are $G_\delta$
and $Z_1$ and $Z_2$ are measure zero. Now, by elementary set theory, the
Cartesian product $E_1\times E_2$ can then be written as
\begin{equation}
\label{eq:1}
E_1\times E_2
=\left(H_1\minus Z_1\right)\times\left(H_2\minus Z_2\right)
=\underbrace{\left(H_1\times H_2\right)}_H\minus
\underbrace{\left(Z_1\times H_2\minus H_1\times Z_2\minus Z_1\times Z_2\right)}_Z
\end{equation}
Hence, we win by (3.28) (i) if we can show that the Cartesian product of two
$G_\delta$ sets is an $G_\delta$ set and if the Cartesian product of a
measurable set with a set of measure zero is measure zero.

First, we prove the former, since the argument to be made is little more
than elementary set theory.
\begin{lemma}
\label{lem:product-of-g-delta}
The Cartesian product of $G_\delta$ sets is again $G_\delta$.
\end{lemma}
\begin{proof}[Proof of lemma \ref{lem:product-of-g-delta}]
\renewcommand\qedsymbol{$\clubsuit$}
Let $G_1$ and $G_2$ be $G_\delta$. Write
$G_1=\bigcap G_k'$ and $G_2=\bigcap G_k''$ where the $G_k'$'s and the
$G_k''$'s are open subsets of $\bfR$. Then, $G_k'\times G_\ell''$ are
open subsets of $\bfR^2$ by the definition of the product
topology. Moreover, $G_k'\times G_\ell''\subset G_1\times G_2$ hence,
$\bigcap_{k,\ell} G_k'\times G_\ell''\subset G_1\times G_2$. Thus, it
suffices to show that
$\bigcap_{k,\ell} G_k'\times G_\ell''\supset G_1\times G_2$. Let
$(x,y)\in G_1\times G_2$. Then $x\in G_1$ and $y\in G_2$. But since
$G_1=\bigcap G_k'$ and $G_2=\bigcap G_k''$ then $x\in G_k'$ and
$x\in G_\ell''$ for some $k$, $\ell$. In other words,
$(x,y)\in G_k'\times G_\ell''$ so $(x,y)$ is in the intersection
$\bigcap_{k,\ell}G_k'\times G_\ell''$. Hence, we have
$G_1\times G_2=\bigcap_{k,\ell} G_k'\times G_\ell''$. We conclude that if
$G_1$ and $G_2$ are $G_\delta$, then so is their Cartesian product
$G_1\times G_2$.
\end{proof}

\begin{lemma}
\label{lem:measure-of-product-of-measure-zero}
Let $E$ be measurable and $Z$ be measure zero. Then $E\times Z$ is measure
zero.
\end{lemma}
\begin{proof}[Proof of lemma \ref{lem:measure-of-product-of-measure-zero}]
\renewcommand\qedsymbol{$\clubsuit$}
Let $E$ be a measurable set with $\left|E\right|<\infty$ and $Z$ a set of
measure zero. Then, for every $\varepsilon>0$ there exists a countable
collection of intervals $\left\{I_k\right\}$ containing $Z$ such that
$\sum\Vol\left(I_k\right)<\varepsilon$. Similarly, we can find a collection
$\left\{I_k'\right\}$ of intervals containing $E$ such that
$\sum\Vol\left(I_k'\right)<\left|E\right|+\varepsilon$. Then,
$\left\{I_k'\times I_\ell\right\}$ is a countable collection of
$2$-intervals containing $E\times Z$ with
\begin{align*}
\sum_{k,\ell}\Vol\left(I_k'\times I_\ell\right)
&=\sum_{k,\ell}\Vol\left(I_k'\right)\Vol\left(I_\ell\right)\\
&=\sum_k\sum_\ell\Vol\left(I_k'\right)\Vol\left(I_\ell\right)\\
&=\left(\sum_k\Vol\left(I_k'\right)\right)
\left(\sum_\ell\Vol\left(I_\ell\right)\right)\\
&=\left(\left|E\right|+\varepsilon\right)\varepsilon
\end{align*}
Letting $\varepsilon\to 0$, we have $E\times Z$ is measure zero. If
$\left|E\right|=\infty$, partition $E$ into disjoint finite measure subsets
of $\bfR$ by taking the following intersection
\[
E_k=E\cap\left(B(0,k)\setminus B(0,k-1)\right)
\]
for $k\in\bfN$.\footnote{In fact, it might be quicker from now on to quote
  the fact that $\bfR^n$ is $\sigma$-finite.} By our previous argument,
$E_k\times Z$ is measure zero so $\left\{E_k\times Z\right\}$ is a cover of
$E\times Z$ hence, by (3.24), we have
\begin{align*}
\left|E\times Z\right|
&=\left|\left(\bigcup_k E_k\right)\times Z\right|\\
&=\left|\bigcup_k E_k\times Z\right|\\
&=\sum_k \left|E_k\times Z\right|\\
&=0.
\end{align*}
Thus, $E\times Z$ is measure zero.
\end{proof}

By lemma \ref{lem:product-of-g-delta} and
\ref{lem:measure-of-product-of-measure-zero}, $E_1\times E_2$ is measurable
with $\left|E_1\times E_2\right|=\left|H_1\times H_2\right|$. It's left to
show is that $\left|H_1\times H_2\right|=\left|H_1\right|\left|H_2\right|$.

\begin{lemma}
\label{lem:measure-of-product-of-g-delta}
If $G_1$ and $G_2$ are $G_\delta$ then $\left|G_1\times
  G_2\right|=\left|G_1\right|\left|G_2\right|$.
\end{lemma}
But before that, we need to prove the above for the case where $G_1$ and
$G_2$ are open sets.

\begin{lemma}
\label{lem:measure-of-product-of-open}
If $G_1$ and $G_2$ are open then $\left|G_1\times
  G_2\right|=\left|G_1\right|\left|G_2\right|$.
\end{lemma}
\begin{proof}[Proof of lemma \ref{lem:measure-of-product-of-open}]
\renewcommand\qedsymbol{$\clubsuit$}
Let $G_1$ and $G_2$ be open with
$\left|G_1\right|,\left|G_2\right|<\infty$. By (1.11), we may write $G_1$
and $G_2$ as the countable intersection of a collection of nonoverlapping
closed intervals $\left\{I_k\right\}$ and $\left\{I_k'\right\}$,
respectively. Therefore, we have
\[
\left|G_1\right|=\sum_k\Vol\left(I_k\right)\quad\text{and}\quad
\left|G_2\right|=\sum_k\Vol\left(I_k'\right).
\]
Moreover the collection $\left\{I_k\times I_\ell'\right\}$ is a cover of
$G_1\times G_2$ of nonoverlapping closed $2$-intervals,\footnote{They are closed
  because of elementary topology: the Cartesian product of two closed sets
  is again closed in the product topology; and they are nonoverlapping
  because if $(x,y)\in I_k\times I_\ell'\cap I_{k'}\times
  I_{\ell'}\neq\emptyset$ then $x\in I_k\cap I_{k'}$ and $y\in I_\ell\cap
  I_{\ell'}$ a contradiction.} so by (3.2) we have
\begin{align*}
\left|G_1\times G_2\right|
&=\sum_{k,\ell}\Vol\left(I_k\times I_\ell'\right)\\
&=\sum_{k,\ell}\Vol\left(I_k\right)\Vol\left(I_\ell'\right)\\
&=\left(\sum_k\Vol(I_k)\right)\left(\Vol(I_\ell')\right)\\
&=\left|G_1\right|\left|G_2\right|
\end{align*}
\end{proof}
\begin{proof}[Proof of lemma \ref{lem:measure-of-product-of-g-delta}]
\renewcommand\qedsymbol{$\clubsuit$}
Now that we have the result of lemma \ref{lem:measure-of-product-of-open}
we may easily proceed to the countable case. Let $G_1$ and $G_2$ be
$G_\delta$. Then by lemma \ref{lem:product-of-g-delta} $G_1\times G_2$ is
$G_\delta$ and we may write $G_1\times G_2$ as the intersection of a
countable collection of open sets $\left\{G_k'\right\}$. In particular, if
$\left\{G_k'\right\}$ is a collection of open sets covering $G_1\times G_2$
that intersects to $G_1\times G_2$ then the collection
$\left\{G_k''\right\}$, where $G_k''\coloneqq\bigcap_{\ell=1}^k G_\ell'$,
also intersects to $G_1$ and has the property that $G_{k+1}''\subset
G_k''$. Thus, we may as well assume that $\left\{H_k\right\}$ is decreasing
so, by (3.26), we have
\begin{align*}
\left|G_1\times G_2\right|
&=\lim_{k\to\infty}\left|H_k\right|,\\
\intertext{but $H_k$ is open in the product topology so $H_k=H_k'\times
  H_k''$ for open subsets $H_k',H_k''\subset\bfR$, giving us}
&=\lim_{k\to\infty}\left|H_k\times H_k''\right|,\\
\intertext{which, by lemma \ref{lem:measure-of-product-of-open}, is just}
&=\lim_{k\to\infty}\left|H_k'\right|\left|H_k''\right|\\
&=\left|E_1\right|\left|E_2\right|,
\end{align*}
since $H_k'\supset E_1$ and $H_k''\supset E_2$ are open so $\bigcap
H_k'\supset E_1$ and $\bigcap H_k''\supset E_2$ and their outer measure
approach the outer measure of $E_1$ and $E_2$ as $k\to\infty$.
\end{proof}
Putting together our results, by equation \ref{eq:1},
lemma \ref{lem:measure-of-product-of-measure-zero}, and
lemma \ref{lem:measure-of-product-of-g-delta}, we can express
$E_1\times E_2$ as a $G_\delta$ set $H$ minus a set of measure zero $Z$
and its measure is
\[
 \left|E_1\times E_2\right|=
\left|H_1\right|\left|H_2\right|=
\left|E_1\right|\left|E_2\right|,
\]
as desired.
\end{proof}
\newpage

\begin{problem}[Wheeden \& Zygmund {\S}3, Ex.\@ 13]
Motivated by (3.7), define the \emph{inner measure} of $E$ by
$\left|E\right|_i=\sup\left|F\right|$, where the supremum is taken over all
closed subsets $F$ of $E$. Show that
\begin{enumerate}[label=(\roman*)]
\item $\left|E\right|_i\leq\left|E\right|_e$, and
\item if $\left|E\right|_e<+\infty$, then $E$ is measurable if and only if
  $\left|E\right|_i=\left|E\right|_e$.
\end{enumerate}
[Use (3.22).]
\end{problem}
\begin{proof}
(i) If the outer measure of $E$ is infinite, the inequality holds
trivially. Suppose $|E|_e<\infty$. Since closed sets are measurable and
their outer measure is equal to their Lebesgue measure, then we may replace
$\left|F\right|$ by $\left|F\right|_e$ to mirror the definition of the
outer-measure and, by the monotonicity of the outer measure, we have
\begin{equation}
\label{eq:monotonicity-outer-measure}
\left|F\right|=\left|F\right|_e\leq\left|E\right|_e.
\end{equation}
Taking the supremum on both sides of (\ref{eq:monotonicity-outer-measure}),
we obtain the desired inequality
\begin{equation}
\label{eq:inner-outer-measure}
\left|E\right|_i\leq\left|E\right|_e.
\end{equation}
\\\\
(ii) $\implies$ Suppose $E$ is measurable with
$\left|E\right|<\infty$. By (3.22), given $\varepsilon>0$, there exists a
closed set $F\subset E$ such that $\left|E\minus
  F\right|_e<\varepsilon$. Since $F$ is measurable, by (3.31), we have
\begin{equation}
\label{eq:applied-caratheodory}
\left|E\minus F\right|_e=\left|E\right|_e-\left|F\right|.
\end{equation}
But $E$ is also measurable, so equation (\ref{eq:applied-caratheodory})
becomes
\begin{equation}
  \label{eq:becomes-this}
\left|E\minus F\right|_e+\left|F\right|=\left|E\right|<\varepsilon+\left|F\right|.
\end{equation}
Taking the supremum of (\ref{eq:becomes-this}) over all $F$, we gave
\[
\left|E\right|_e=\left|E\right|\leq \left|F\right|+\varepsilon=\left|E\right|_i+\varepsilon
\]
for all $\varepsilon>0$. By equation (\ref{eq:inner-outer-measure}), we
achieve equality of the inner and outer measure, i.e.,
$\left|E\right|_i=\left|E\right|_e$.

$\impliedby$ Conversely, suppose that
$\left|E\right|_i=\left|E\right|_e$. Then, given $\varepsilon>0$, by the
definition of outer measure, there exists an open set $G\supset E$ and, by
the definition of inner measure, closed set $F\subset E$ such that
\begin{equation}
  \label{eq:inner-outer-approximation}
\left|G\right|-\left|E\right|_e<\frac{\varepsilon}{2}\qquad\text{and}\qquad
\left|E\right|_i-\left|F\right|=
\left|E\right|_e-\left|F\right|<\frac{\varepsilon}{2}.
\end{equation}
Then
\[
\left|E\minus F\right|_e
<\left|G\minus F\right|_e
=\left|G\right|_e-\left|G\cap F\right|_e
=\left|G\right|_e-\left|F\right|_e
<2\left(\frac{\varepsilon}{2}\right)=\varepsilon.
\]
So by (3.22) $E$ is measurable.
\end{proof}
\newpage

% \begin{problem}[Wheeden \& Zygmund {\S}3, Ex.\@ 14]
%   Show that the conclusion of part (ii) of Exercise 13 is false if
%   $\left|E\right|_e=+\infty$.
% \end{problem}
% \begin{proof}
% \end{proof}
% \newpage

\begin{problem}[Wheeden \& Zygmund {\S}3, Ex.\@ 15]
If $E$ is measurable and $A$ is any subset of $E$, show that
$\left|E\right|=\left|A\right|_i+\left|E\minus A\right|_e$. (See Exercise 13 for
the definition of $\left|A\right|_i$.)
\end{problem}
\begin{proof}
If $A$ is measurable, by our previous problem,
$\left|A\right|_e=\left|A\right|=\left|A\right|_i$ so by (3.31), we have
\[
|E\minus A|_e=|E|_e-|A|=|E|-|A|_i,
\]
so $\left|E\right|=\left|A\right|_i+\left|E\minus A\right|_e$.

If $A$ is not measurable and $\left|E\right|<\infty$ then $|A|_e$ and
$\left|E\minus A\right|_e<\infty$ since $E\minus A\subset E$. Hence, we can
subtract the quantity $\left|E\minus A\right|_e$ from
$\left|E\right|_e=\left|E\right|$ and we get
\begin{align*}
\left|E\right|-\left|E\minus A\right|_e
&=\left|E\right|-\inf\left\{\,\left|G\right|:\text{$G\supset E\minus A$ is open}\,\right\}\\
&=\left|E\right|-\inf\left\{\,\left|G\right|:\text{$G\supset E\minus A$ is open}\,\right\}\\
&=\left|E\right|-\inf\left\{\,\left|G\right|:\text{$E\supset G\supset E\minus
  A$ is open}\,\right\}\\
&=\left|E\right|-\inf\left\{\,\left|E\right|-\left|F\right|:\text{$E\minus
  G\subset A$, $G\supset E\minus A$ is open}\,\right\}\\
\intertext{pulling the $|E|$ inside of the infimum we get}
&=-\inf\left\{\,-\left|F\right|:\text{$F\subset A$, where $F\coloneqq
  E\minus G$ and $G\supset A$ is open}\,\right\}\\
&=\sup\left\{\,\left|F\right|:\text{$F\subset A$, where $F\coloneqq
  E\minus G$ and $G\supset A$ is open}\,\right\}\\
&=\left|A\right|_i,
\end{align*}
so $\left|E\right|=\left|A\right|_i+\left|E\minus A\right|_e$ as desired.

More generally, if $E$ is measurable we can establish the following lower
bound on $|E|$: By definition of the inner measure, for every $F\subset A$
closed we have
\begin{equation}
\label{eq:inner-measure-lower-bound}
|E|=|F|+|E\minus F|\geq |F|+\left|E\minus A\right|_e,
\end{equation}
which follows from (3.31), since $F\subset A\subset E$ is measurable,
$E\minus A\subset E\minus F$, and the monotonicity of outer measure. Taking
the supremum on both sides of equation (\ref{eq:inner-measure-lower-bound})
over all closed $F\subset A$ we achieve the desired lower bound
\begin{equation}
  \label{eq:lower-bound}
|E|\geq |A|_i+|E\minus A|_e.
\end{equation}

Now, let $E$ be measurable. Then, we have
\begin{align}
\label{eq:upper-bound}
|A|_i+|E\minus A|_e
&\geq\sup\left\{\,|F\cap A|_i+|F\minus A|_e:\text{$F\subset E$ is closed, $|F|<\infty$}\,\right\}\nonumber\\
&=\sup\left\{\,|F|:\text{$F\subset E$ closed, $|F|<\infty$}\,\right\}\nonumber\\
&=|E|_i\nonumber\\
\intertext{which by our last problem is just}
&=|E|,
\end{align}
since $E$ is measurable. Having bounded $|E|$ by equations
(\ref{eq:lower-bound}) and (\ref{eq:upper-bound}) we achieve the equality
\[
|E|=|A|_i+|E\minus A|_e.\qedhere
\]
\end{proof}

%%% Local Variables:
%%% mode: latex
%%% TeX-master: "../MA544-HW-Current"
%%% End:
