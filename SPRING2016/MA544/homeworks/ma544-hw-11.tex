% # 11 – Due Apr. 18 Read Section 5.4. Chapter 7: # 11, 15; Chapter 5: # 8,
% 11, 12, 17.

\begin{problem}[Wheeden \& Zygmund {\S}7, Ex.\@ 11]
Prove the following result concerning changes of variable. Let $g(t)$ be
monotone increasing and absolutely continuous on $[\alpha,\beta]$ and let $f$
be integrable on $[a,b]$, $a= g(\alpha)$, $b=
g(\beta)$. Then $f(g(t))g'(t)$ is measurable and integrable on
$[\alpha,\beta]$, and
\[
\int_a^b f(x)d x=\int_\alpha^\beta f(g(t))g'(t)d t.
\]
(Consider the case when $f$ is the characteristic function of an interval,
an open set, etc.)
\end{problem}
\begin{proof}
As the parenthesized text suggests, we will prove the result in
stages by first showing that the equality holds for characteristic
functions, for simple functions, and lastly, we pass to the limit. First,
note that, by Theorem 5.21, $f$ is integrable if and only if $|f|$ is
integrable therefore, it suffices to prove the result for $f\geq 0$.

Suppose $f$ is the characteristic function of an open interval
$(c,d)\subset(a,b)$, then
\begin{equation}
\label{eq:11:char-fun-interval}
\begin{aligned}
\int_a^b f(g(t))g'(t)dt
&=\int_{E_1}f(g(t))g'(t)dt+\int_{E_2}f(g(t))g'(t)dt\\
&=\int_{E_1}g'(t)dt\\
&=g(\beta)-g(\alpha)\\
&=d-c\\
&=\int_a^b f(x)dx,
\end{aligned}
\end{equation}
where $E_1\coloneqq\left\{\,t\in[\alpha,\beta]:g(t)\in[c,d]\,\right\}$ and
$E_2\coloneqq\left\{\,t\in[\alpha,\beta]:g(t)\in[a,b]\setminus[c,d]\,\right\}$,
the latter being equivalent to the complement in $[a,b]$ of the former,
i.e., $E_2=[a,b]\setminus E_1$.

Now, let $G\subset[a,b]$ be open. Then $G$ can be written as the disjoint
union of intervals $\{I_k\}$.
\end{proof}
\newpage

\begin{problem}[Wheeden \& Zygmund {\S}7, Ex.\@ 15]
Theorem 7.43 shows that a convex function is the indefinite integral of a
monotone increasing function. Prove the converse: If
$\varphi(x)=\int_a^xf(t)dt+\varphi(a)$ in $(a,b)$ and $f$ is monotone
increasing, then $\varphi$ is convex in $(a,b)$. (Use Exercise 14.)
\end{problem}
\begin{proof}
\end{proof}
\newpage

\begin{problem}[Wheeden \& Zygmund {\S}5, Ex.\@ 8]
Prove (5.49).
\end{problem}
\begin{proof}
Recall the content of equation 5.49: For $f$ measurable, we have
\begin{equation}
  \label{eq:11:chebyshevs-inequality}
\omega(\alpha)\leq\frac{1}{\alpha^p}\int\limits_{\{\,f>\alpha\,\}}f^p,\quad\alpha>0.
\end{equation}
\end{proof}
\newpage

\begin{problem}[Wheeden \& Zygmund {\S}5, Ex.\@ 11]
For which $p$ does $1/x\in L^p(0,1)$? $L^p(1,\infty)$? $L^p(0,\infty)$?
\end{problem}
\begin{proof}
\end{proof}
\newpage

\begin{problem}[Wheeden \& Zygmund {\S}5, Ex.\@ 12]
Give an example of a bounded continuous $f$ on $(0,\infty)$ such that
$\lim_{x\to\infty}f(x)=0$ but $f\notin L^p(0,\infty)$ for any $p>0$.
\end{problem}
\begin{proof}
\end{proof}
\newpage

\begin{problem}[Wheeden \& Zygmund {\S}5, Ex.\@ 17]
If $f\geq 0$, show that $f\in L^p$ if and only if $\sum_{k=-\infty}^\infty
2^{kp}\omega(2^k)<\infty$. (Use Exercise 16.)
\end{problem}
\begin{proof}
\end{proof}

%%% Local Variables:
%%% mode: latex
%%% TeX-master: "../MA544-HW-Current"
%%% End:
