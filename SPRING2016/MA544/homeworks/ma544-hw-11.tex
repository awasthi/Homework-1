% # 11 – Due Apr. 18 Read Section 5.4. Chapter 7: # 11, 15; Chapter 5: # 8,
% 11, 12, 17.

\begin{problem}[Wheeden \& Zygmund {\S}7, Ex.\@ 11]
Prove the following result concerning changes of variable. Let $g(t)$ be
monotone increasing and absolutely continuous on $[\alpha,\beta]$ and let $f$
be integrable on $[a,b]$, $a= g(\alpha)$, $b=
g(\beta)$. Then $f(g(t))g'(t)$ is measurable and integrable on
$[\alpha,\beta]$, and
\[
\int_a^b f(x)d x=\int_\alpha^\beta f(g(t))g'(t)d t.
\]
(Consider the case when $f$ is the characteristic function of an interval,
an open set, etc.)
\end{problem}
\begin{proof}
Recall that, by Theorem 5.21, $f$ is integrable (or in $L^1$) on
$[\alpha,\beta]$ if and only if $|f|$ is integrable on
$[\alpha,\beta]$. Therefore, it suffices to prove the result for the case
$f\geq 0$. We split the proof of the result into a series of claims and
then proceed to show the more general result.
\begin{claim}
Let $g$ be as above and $G$ be an open subset of $[\alpha,\beta]$. Then
\[
|g(G)|=\int_G g'(t)dt.
\]
\end{claim}
\begin{proof}[Proof of claim 1]
\renewcommand\qedsymbol{$\clubsuit$}
Let $G$ be an open subset of $(a,b)$ then, by Theorem 1.10, $G$ can be
written as the countable union of disjoint open intervals $\{I_k\}$. By
Theorem 5.7, since $g'$ is nonnegative and measurable and
$\int_Gg'$ is finite (in particular, it is bounded above by $\int_a^bg'$),
we have
\begin{equation}
\label{eq:11:additive-measure}
\int_Gg'(t)dt=\sum_k\int_{I_k}g'(t)dt.
\end{equation}
But by Theorem 7.27, since $g$ is absolutely continuous on
$[\alpha,\beta]$, $g$ is b.v.\@ on $[\alpha,\beta]$ so by Theorem 7.30
\[
|g(I_k)|
=g(\beta_k)-g(\alpha_k)
=V[g;\alpha_k,\beta_k]
=\int_{\alpha_k}^{\beta_k}g'(t)dt
\]
where $\alpha_k$ is the left-most endpoint of $I_k$ and $\beta_k$ the
right-most. By Equation \eqref{eq:11:additive-measure}, on the right-hand
side, we have
\[
\int_{I_k}g'(t)dt=|g(I_k)|
\]
so, by Theorem 3.23, we have
\begin{equation}
\label{eq:11:measure-sum-integral-dev}
\int_G g'(t)dt
=\sum_k|g(I_k)|
=\left|g\left({\textstyle\bigcup_k I_k}\right)\right|
=|g(G)|
\end{equation}
as desired.
\end{proof}
\begin{claim}
Let $g$ be as above and $E$ be a $G_\delta$-subset of
$[\alpha,\beta]$. Then
\[
|g(E)|=\int_E g'(t)dt.
\]
\end{claim}
\begin{proof}[Proof of claim 2]
\renewcommand\qedsymbol{$\clubsuit$}
Suppose $E$ is a $G_\delta$-set, then $E$ is the countable intersection of
open subsets $\{G_k\}$ of $[\alpha,\beta]$. We may choose $G_k$'s such that
$G_k\searrow E$ (for example, taking our original collection of open
subsets $\{G_k\}$ and taking the finite intersection $\bigcap_{j=1}^k
G_j$). Hence, we have $\chi_{G_k}\searrow\chi_E$ and consequently
$\chi_{G_k}g'\searrow\chi_Eg'$. Thus, we have
\begin{equation}
\label{eq:11:g-delta-case}
\lim_{k\to\infty}\int_E\chi_{G_k}g'(t)dt
=\lim_{k\to\infty}|g(G_k)|
=|g(E)|
\end{equation}
by Claim 1 and Theorem 3.10. Thus, by the monotone convergence theorem
together with Equation \eqref{eq:11:g-delta-case}, we have
\begin{equation}
\label{eq:11:apply-mct}
|g(E)|=\lim_{k\to\infty}\int_E\chi_{G_k}g'(t)dt=\int_E\chi_{G_k}g'(t)dt
\end{equation}
as desired.
\end{proof}
\begin{claim}

\end{claim}
\end{proof}
\newpage

\begin{problem}[Wheeden \& Zygmund {\S}7, Ex.\@ 15]
Theorem 7.43 shows that a convex function is the indefinite integral of a
monotone increasing function. Prove the converse: If
$\varphi(x)=\int_a^xf(t)dt+\varphi(a)$ in $(a,b)$ and $f$ is monotone
increasing, then $\varphi$ is convex in $(a,b)$. (Use Exercise 14.)
\end{problem}
\begin{proof}
We will assume the result in Exercise 14. First we check that $\varphi$ is
continuous. Since $f$ is monotone increasing, $f$ is b.v.\@ on $[a,b]$ so
$f$ is bounded a.e.\@ on $(a,b)$ by a previous exercise. Thus, $f\in
L(a,b)$ so by Theorem 7.1, $\varphi$ is absolutely continuous and hence,
continuous.

Now, let $x_1,x_2\in(a,b)$ and, without loss of generality, assume
$x_1<x_2$. Then, we have
\begin{align*}
\varphi\left(\frac{x_1+x_2}{2}\right)
&=\int_a^{(x_1+x_2)/2}f(t)dt+\varphi(a)\\
&=\int_a^{x_1}f(t)dt+\int_{x_1}^{(x_1+x_2)/2}f(t)dt+\varphi(a)
\intertext{since $f$ is monotone increasing, we have
  $\int_{x_1}^{(x_1+x_2)/2}f(t)dt\leq\int_{(x_1+x_2)/2}^{x_2}f(t)dt$ so}
&=\int_a^{x_1}f(t)dt
+\frac{1}{2}\left[2\int_{x_1}^{(x_1+x_2)/2}f(t)dt\right]+\varphi(a)\\
&\leq\int_a^{x_1}f(t)dt
+\frac{1}{2}\left[\int_{x_1}^{(x_1+x_2)/2}f(t)dt
+\int_{(x_1+x_2)/2}^{x_2}f(t)dt\right]+\varphi(a)\\
&=\frac{1}{2}\left[\int_a^{x_1}f(t)dt+\varphi(a)\right]
+\frac{1}{2}\left[\int_a^{x_1}f(t)dt+\int_{x_1}^{(x_1+x_2)/2}f(t)dt+\int_{(x_1+x_2)/2}^{x_2}f(t)dt+\varphi(a)\right]\\
&=\frac{1}{2}\left[\int_a^{x_1}f(t)dt+\varphi(a)\right]
+\frac{1}{2}\left[\int_a^{x_2}f(t)dt+\varphi(a)\right]\\
&=\frac{\varphi(x_1)+\varphi(x_2)}{2}.
\end{align*}
Thus, by Exercise 14, $\varphi$ is convex.
\end{proof}
\newpage

\begin{problem}[Wheeden \& Zygmund {\S}5, Ex.\@ 8]
Prove (5.49).
\end{problem}
\begin{proof}
Recall the content of equation 5.49: For $f$ measurable, we have
\begin{equation}
  \label{eq:11:chebyshevs-inequality}
\omega(\alpha)\leq\frac{1}{\alpha^p}\int\limits_{\{\,f>\alpha\,\}}f^p,\quad\alpha>0.
\end{equation}
Consider the $L^p$-norm of $f$ raised to the $p$-th power
\begin{align*}
{\|f\|_p}^p&=\int|f(x)|^pdx
\intertext{since $f$ is measurable, $f$ is measurable so $\left\{\,f>\alpha\,\right\}$ is
             measurable hence, by the monotonicity of the Lebesgue
             integral, we have}
  &\geq\int_{\left\{\,f>\alpha\,\right\}}f^pdx\\
           &\geq\int_{\left\{\,f>\alpha\,\right\}}\alpha^pdx\\
           &=\alpha^p\left|\left\{\,f>\alpha\,\right\}\right|\\
           &=\alpha^p\omega(\alpha).
\end{align*}
Thus, we have
\[
\omega(\alpha)\leq\frac{1}{\alpha^p}\int_{\left\{\,f>\alpha\,\right\}}f^p
\]
as desired.
\end{proof}
\newpage

\begin{problem}[Wheeden \& Zygmund {\S}5, Ex.\@ 11]
For which $p$ does $1/x\in L^p(0,1)$? $L^p(1,\infty)$? $L^p(0,\infty)$?
\end{problem}
\begin{proof}
For the case $1/x\in L^p(0,1)$, this happens if and only if $\int_0^1
x^{-p}dx<\infty$ if and only if $p<1$.

In the second case $1/x\in L^p(1,\infty)$ if and only if $p>1$.

Lastly, we have $1/x\in L^p(0,\infty)$ if and only if $1/x\in L^p(0,1)$ and
$1/x\in L^p(1,\infty)$. By our previous arguments, this is
impossible. Thus, $1/x\notin L^p(0,\infty)$.
\end{proof}
\newpage

\begin{problem}[Wheeden \& Zygmund {\S}5, Ex.\@ 12]
Give an example of a bounded continuous $f$ on $(0,\infty)$ such that
$\lim_{x\to\infty}f(x)=0$ but $f\notin L^p(0,\infty)$ for any $p>0$.
\end{problem}
\begin{proof}
An example, given in class, is the following: Set
\begin{equation}
\label{eq:11:example}
f(x)\coloneqq
\begin{cases}
1&x\leq e\\
1/\ln x&x\geq e.
\end{cases}
\end{equation}
This function is bounded (above by $1$), continuous ($\lim_{x+\to
  e}f(x)=1=\lim_{x-\to e}f(x)$) and $\lim_{x\to\infty}f(x)=0$ Now, observe
that, for every $p>0$, we have $\ln(x)\leq x^{1/p}$ for $x$ larger than
some number $K$ depending on $p$. Thus,
\[
\int_K^\infty\frac{dx}{\ln x}\geq\int_K^\infty\frac{dx}{x}=\infty.
\]
so $f$ cannot be in $L^p(0,\infty)$ for any $p>0$.
\end{proof}
\newpage

\begin{problem}[Wheeden \& Zygmund {\S}5, Ex.\@ 17]
If $f\geq 0$ and $\omega(\alpha)\leq c(1+\alpha)^p$ for all $\alpha>0$,
show that $f\in L^r$, $0<r<p$.
\end{problem}
\begin{proof}
Assuming the results of Exercise 16, it suffices to show that
\begin{equation}
\label{eq:11:need-to-show}
\int_0^\infty\alpha^{r-1}\omega(\alpha)d\alpha\leq c\int_0^\infty\frac{\alpha^{r-1}}{(1+\alpha)^p}d\alpha<\infty
\end{equation}
for all $r\in(0,p)$. The integral is improper only near $\infty$, and
convergence there follows from the fact that
\[
\frac{\alpha^{r-1}}{(1+\alpha)^p}<\frac{\alpha^{r-1}}{\alpha^p}=\frac{1}{\alpha^{p-(r-1)}}
\]L
for sufficiently large $\alpha$. Since $r<p$, we have $p-(r-1)>1$, hence
\[
\int_{K_p}^\infty\frac{d\alpha}{\alpha^{p-(r-1)}}
\]
converges.
\end{proof}
\newpage
\begin{problem}[Wheeden \& Zygmund {\S}8, Thm.\@ 8.3]
 If $f,g\in L^p(E)$, $p>0$, then $f+g\in L^p(E)$ and $cf\in L^p(E)$ for any constant $c$.
\end{problem}
\begin{proof}
Suppose $f,gi\in L^p(E)$ and $c$ is any constant, then, by Minkowski's
inequality
\[
\|f+g\|_p\leq\|f\|_p+\|g\|_p<\infty
\]
and
\[
\|cf\|_p=\left(\int_E|cf|^p\right)^{1/p}=
\left(\int_E|c|^p|f|^p\right)^{1/p}=
|c|\left(\int_E|c|^p|f|^p\right)^{1/p}<\infty.
\]
Thus, $f+g,cf\in L^p(E)$
\end{proof}

%%% Local Variables:
%%% mode: latex
%%% TeX-master: "../MA544-HW-Current"
%%% End:
