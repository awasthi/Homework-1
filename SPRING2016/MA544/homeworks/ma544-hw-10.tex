% #9 – Due Apr. 11 Read Section 7.1-6. Chapter 7: # 1, 2, 12, 13, 6 (8, 9, 11, 15).

\begin{problem}[Wheeden \& Zygmund {\S}7, Ex.\@ 1]
Let $f$ be measurable in $\bfR^n$ and different from zero in some set of
positive measure. Show that there is a positive constant $c$ such that
$f^*(\bfx)\geq c\|\bfx\|^{-n}$ for $\|\bfx\|\geq 1$.
\end{problem}
\begin{proof}
Suppose that $f$ is measurable and nonzero on a subset $E$ of $\bfR^n$ with
positive measure. Assume $E$ is bounded. Since $f$ is measurable $|f|$ is
measurable so the set $E_a\coloneqq \left\{\,\bfx\in
  E:|f|(\bfx)>a\,\right\}$, for $a$ finite, is a measurable bounded subset
of $\bfR^n$. Let $\chi_a$ denote the characteristic function of
$E_a$. Then, by Chebyshev's inequality, we have
\begin{equation}
\label{eq:10:bound-by-chebyshev}
\begin{aligned}
{\chi_a}^*(\bfx)
&=\sup_Q\frac{1}{|Q|}\int_Q|\chi_a(\bfy)|d\bfy\\
&\leq\sup_Q\frac{1}{|Q|}\left[\frac{1}{a}\int_Q|f(\bfy)|d\bfy\right]\\
&=\frac{1}{a}f^*(\bfx).
\end{aligned}
\end{equation}
By the commentary on p.\@ 138, there exists constants $c_1$ and $c_2$ such
that
\begin{equation}
\label{eq:10:bound-by-characteristic}
c_1\frac{|E_a|}{\|\bfx\|^n}\leq{\chi_a}^*(\bfx)\leq c_2\frac{|E_a|}{\|\bfx\|^n}
\end{equation}
for all large $\|\bfx\|$. Putting \eqref{eq:10:bound-by-chebyshev} and
\eqref{eq:10:bound-by-characteristic} together, we obtain
\begin{equation}
  \label{eq:lower-bound-maximal-fun}
ac_1\frac{|E_a|}{\|\bfx\|^n}\leq f^*(\bfx).
\end{equation}
Setting $c\coloneqq ac_1|E|$, we have the desired lower bound
$c\|\bfx\|^{-n}\leq f^*(\bfx)$ (assuming $\|\bfx\|$ is large).
\end{proof}
\newpage

\begin{problem}[Wheeden \& Zygmund {\S}7, Ex.\@ 2]
Let $\varphi(\bfx),\bfx\in\bfR^n$, be a bounded measurable function such
that $\varphi(\bfx)=0$ for $\|\bfx\|\geq 1$ and $\int\varphi=1$. For
$\varepsilon>0$, let
$\varphi_\varepsilon(\bfx)=\varepsilon^{-n}\varphi(\bfx/\varepsilon)$. ($\varphi_\varepsilon$
is called an \emph{approximation to the identity}.) If $f\in L(\bfR^n)$,
show that
\[
\lim_{\varepsilon\to 0}(f*\varphi_\varepsilon)(\bfx)=f(\bfx)
\]
in the Lebesgue set of $f$. (Note that $\int\varphi_\varepsilon=1$,
$\varepsilon>0$, so that
\[
(f*\varphi_\varepsilon)(\bfx)-f(\bfx)=\int\left[f(\bfx-\bfy)-f(\bfx)\right]\varphi_\varepsilon(\bfy)d\bfy.
\]
Use Theorem 7.16.)
\end{problem}
\begin{proof}
First note that, making the change of variables $\bfu=\bfx/\varepsilon$
(with Jacobian $\Jac(\bfx,\bfu)=\varepsilon^n$), we
have
\begin{equation}
\label{eq:10:change-of-vars}
\begin{aligned}
\int\varphi_\varepsilon(\bfx)d\bfx
&=\int\varepsilon^{-n}\varphi(\bfx/\varepsilon)d\bfx\\
&=\int_{B(\mathbf{0},\varepsilon)}\varepsilon^{-n}\varphi(\bfx/\varepsilon)d\bfx\\
&=\int_{B(\mathbf{0},1)}\varphi(\bfu)d\bfu\\
&=\int\varphi(\bfx)d\bfu\\
&=1.
\end{aligned}
\end{equation}
Hence, by the hint and the definition of the convolution, we have
\begin{equation}
\label{eq:10:convolution-hint}
\begin{aligned}
\left|(f*\varphi_\varepsilon)(\bfx)-f(\bfx)\right|
&=\left|\int[f(\bfx-\bfy)-f(\bfx)]\varphi_\varepsilon(\bfx)d\bfx\right|\\
&=\left|\int_{B(\mathbf{0},\varepsilon)}[f(\bfx-\bfy)-f(\bfx)]\varphi_\varepsilon(\bfx)d\bfx\right|\\
&\leq\int_{B(\mathbf{0},\varepsilon)}\left|{[f(\bfx-\bfy)-f(\bfx)]\varphi_\varepsilon(\bfx)}\right|d\bfx.
\end{aligned}
\end{equation}
Now, since $\varphi$ is bounded, say by $M$, we have
\begin{align}
\label{eq:10:bound-estimates}
\varphi_\varepsilon(\bfy)=\varepsilon^{-n}\varphi(\bfx/\varepsilon)\leq M.
\end{align}
Then, we have an estimate on \eqref{eq:10:convolution-hint}
\begin{equation}
\label{eq:10:better-estimate}
\begin{aligned}
\int_{B(\mathbf{0},\varepsilon)}\left|{[f(\bfx-\bfy)-f(\bfx)]\varphi_\varepsilon(\bfx)}\right|d\bfx
&\leq\frac{M}{\varepsilon^n}\int_{B\mathbf{0},\varepsilon}\left|f(\bfx-\bfy)-f(\bfx)\right|\\
&\leq\frac{M}{\varepsilon^n}\int_{B(\mathbf{0},\varepsilon)}\left|f(\bfx-\bfy)-f(\bfx)\right|.
\end{aligned}
\end{equation}
Now, let $Q_\varepsilon$ be the larges cube centered at $\bfx$ contained in
$B(\mathbf{0},\bfx)$. Then, as we have previously shown, the volume of
$Q_\varepsilon$ is $C\varepsilon^n$ for some positive real number
$C$. Making a change of variables $\bfv=\bfx-\bfy$ gives us
\begin{equation}
\label{eq:10:change-of-var}
|(f*\varphi_\varepsilon)(\bfx)-f(\bfx)|
&\leq C\frac{M}{|Q_\varepsilon|}\int_{Q_\varepsilon+bfx}|f(\bfv)-f(\bfx)|d\bfvx,
\end{equation}
which goes to $0$ an $\varepsilon\to 0$ by Theorem 7.16 since $\bfx$ is a
point in the Lebesgue set of $f$.
\end{proof}
\newpage

\begin{problem}[Wheeden \& Zygmund {\S}7, Ex.\@ 6]
Show that if $\alpha>0$, then $x^\alpha$ is absolutely continuous on every
bounded subinterval of $[0,\infty)$.
\end{problem}
\begin{proof}
\end{proof}
\newpage


\begin{problem}[Wheeden \& Zygmund {\S}7, Ex.\@ 8]
Prove the following converse of Theorem 7.31: If $f$ is of bounded
variation on $[a,b]$, and if the function $V(x)=V[a,x]$ is absolutely
continuous on $[a,b]$, then $f$ is absolutely continuous on $[a,b]$.
\end{problem}
\begin{proof}
\end{proof}
\newpage

\begin{problem}[Wheeden \& Zygmund {\S}7, Ex.\@ 9]
If $f$ is of bounded variation on $[a,b]$, show that
\[
\int_a^b|f'|\leq V[a,b].
\]
Show that if equality holds in this inequality, then $f$ is absolutely
continuous on $[a,b]$. (For the second part, use Theorems 2.2(ii) and 7.24
to show that $V(x)$ is absolutely continuous and then use the result of
Exercise 8).
\end{problem}
\begin{proof}
\end{proof}
\newpage

\begin{problem}[Wheeden \& Zygmund {\S}7, Ex.\@ 12]
Use Jensen's inequality to prove that if $a,b\geq 0$, $p,q>1$,
$(1/p)+(1/q)=1$, then
\[
ab\leq\frac{a^p}{p}+\frac{b^p}{q}.
\]
More generally, show that
\[
a_1\dotsm a_N=\sum_{j=1}^N\frac{{a_j}^{p_j}}{p_j},
\]
where $a_j\geq 0$, $p_j>1$, $\sum_{j=1}^N(1/p_j)=1$. (Write
$a_j=e^{x_j/p_j}$ and use the convexity of $e^x$).
\end{problem}
\begin{proof}
\end{proof}
\newpage

\begin{problem}[Wheeden \& Zygmund {\S}7, Ex.\@ 13]
Prove Theorem 7.36.
\end{problem}
\begin{proof}
Recall the statement of Theorem 7.36
\begin{theorem*}
\begin{enumerate}[label=\textnormal{(\roman*)}]
\item If $\varphi_1$ and $\varphi_2$ are convex in $(a,b)$, then
  $\varphi_1+\varphi_2$ is convex in $(a,b)$.
\item If $\varphi$ is convex in $(a,b)$ and $c$ is a positive constant,
  then $c\varphi$ is convex in $(a,b)$.
\item If $\varphi_k$, $k=1,2,\dotsc$, are convex in $(a,b)$ and
  $\varphi_k\to\varphi$ in $(a,b)$, then $\varphi$ is convex in $(a,b)$.
\end{enumerate}
\end{theorem*}
\end{proof}

% \begin{problem}[Wheeden \& Zygmund {\S}7, Ex.\@ 11]
% Prove the following result concerning changes of variable. Let $g(t)$ be
% monotone increasing and absolutely continuous on $[\alpha,\beta]$ and let $f$
% be integrable on $[a,b]$, $a= g(\alpha)$, $b=
% g(\beta)$. Then $f(g(t))g'(t)$ is measurable and integrable on
% $[\alpha,\beta]$, and
% \[
% \int_a^b f(x)d x=\int_\alpha^\beta f(g(t))g'(t)d t.
% \]
% (Consider the case when $f$ is the characteristic function of an interval,
% an open set, etc.)
% \end{problem}
% \begin{proof}
% \end{proof

%%% Local Variables:
%%% mode: latex
%%% TeX-master: "../MA544-HW-Current"
%%% End:
