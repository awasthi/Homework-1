\begin{problem}[Wheeden \& Zygmund {\S}3, Ex.\,5]
Construct a subset of $[0,1]$ in the same manner as the Cantor set, except
that at the $k$th stage each interval removed has length $\delta 3^{-k}$,
$0<\delta<1$. Show that the resulting set is perfect, has measure
$1-\delta$, and contains no interval.
\end{problem}
\begin{proof}
Let $1>\delta>0$ be given. Subdivide the interval $[0,1]$ into thirds,
$\left[1,1/3\right]$, $\left[1/3,2/3\right]$, and $\left[2/3,1\right]$ and
remove the open subset $\left(1/2-\delta,1/2+\delta\right)$ from the middle
third $\left[1/3,2/3\right]$. Repeat ad infinitum
\end{proof}
\newpage

\begin{problem}[Wheeden \& Zygmund {\S}3, Ex.\,7]
Prove (3.15).
\end{problem}
\begin{proof}
\begin{lemma*}[Wheeden \& Zygmund (3.15)]
If $\left\{I_k\right\}_k^N$ is a finite collection of nonoverlapping
intervals, then $\bigcup I_k$ is measurable and $\left|\bigcup
  I_k\right|=\sum\left|I_k\right|$.
\end{lemma*}
\end{proof}
\newpage

\begin{problem}[Wheeden \& Zygmund {\S}3, Ex.\,9]
If $\left\{E_k\right\}_{k=1}^\infty$ is a sequence of sets with
$\sum\left|E_k\right|_e<+\infty$, show that $\limsup E_k$ (and
also $\liminf E_k$) has measure zero.
\end{problem}
\begin{proof}
\end{proof}
\newpage

\begin{problem}[Wheeden \& Zygmund {\S}3, Ex.\,12]
If $E_1$ and $E_2$ are measurable subsets of $\bfR^1$, show that $E_1\times
E_2$ is measurable subset of $\bfR^2$ and $\left|E_1\times
  E_2\right|_e=\left|E_1\right|\left|E_2\right|_e$. (Interpret
$0\cdot\infty$ as $0$) [\textsc{Hint:} Use a characterization of
measurability.]
\end{problem}
\begin{proof}
\end{proof}
\newpage

\begin{problem}[Wheeden \& Zygmund {\S}3, Ex.\,13]
Motivated by (3.7), define the \emph{inner measure of $E$} to by
$\left|E\right|_i\coloneqq\sup\left|F\right|$, where the supremum is taken
over all closed subsets $F$ of $E$. Show that
\begin{enumerate}[label=(\roman*)]
\item $\left|E\right|_i\leq\left|E\right|_e$, and
\item if $\left|E\right|_e<+\infty$, then $E$ is measurable if and only if
  $\left|E\right|_e=\left|E\right|_i$. [Use (3.22).]
\end{enumerate}

\end{problem}
\begin{proof}
\end{proof}
\newpage

\begin{problem}[Wheeden \& Zygmund {\S}3, Ex.\,14]
Show that the conclusion of part  (ii) of Exercise 13 (Problem) is false if
$\left|E\right|_e=+\infty$.
\end{problem}
\begin{proof}
\end{proof}
\newpage

\begin{problem}[Wheeden \& Zygmund {\S}3, Ex.\,8]
Show that the Borel $\sigma$-algebra $\calB$ in $\bfR^n$ is the smallest
$\sigma$-algebra containing the closed sets in $\bfR^n$.
\end{problem}
\begin{proof}
\end{proof}
\newpage

\begin{problem}[Wheeden \& Zygmund {\S}3, Ex.\,10]
If $E_1$ and $E_2$ are measurable, show that $\left|E_1\cup
  E_2\right|+\left|E_1\cap E_2\right|=\left|E_1\right|+\left|E_2\right|$.
\end{problem}
\begin{proof}
\end{proof}
\newpage

\begin{problem}[Wheeden \& Zygmund {\S}3, Ex.\,15]
If $E$ is measurable and $A$ is any subset of $E$, show that
$\left|E\right|=\left|A\right|_i+\left|E\minus A\right|_e$. [See Exercise
13 for the definition of $\left|A\right|_i$.]
\end{problem}
\begin{proof}
\end{proof}
\newpage

\begin{problem}[Wheeden \& Zygmund {\S}3, Ex.\,16]
Prove (3.34).
\end{problem}
\begin{proof}
\begin{lemma*}
$\left|P\right|=v(P)$.
\end{lemma*}
\end{proof}
\newpage

\begin{problem}[Wheeden \& Zygmund {\S}3, Ex.\,18]
Prove that outer measure is \emph{translation invariant}; that is, if
$E_{\textbf{h}}\coloneqq\left\{\,\mathbf{x}+\mathbf{h}\;\middle|\;\mathbf{x}\in
  E\,\right\}$ is the translate of $E$ by $\mathbf{h}$,
$\mathbf{h}\in\bfR^n$, show that
$\left|E_{\mathbf{h}}\right|_e=\left|E\right|_e$. If $E$ is measurable,
show that $E_{\mathbf{h}}$ is also measurable. [This fact was used in
proving (3.37).]
\end{problem}
\begin{proof}
\end{proof}

%%% Local Variables:
%%% mode: latex
%%% TeX-master: "../MA544-HW-Current"
%%% End:
