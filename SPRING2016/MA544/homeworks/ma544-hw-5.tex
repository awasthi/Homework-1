% #5 - Due Feb. 15 Read Sections 3.6, 4.1 Chapter 3: # 14, 16, 18, Chapter 4: # 1, 2.
\begin{problem}[Wheeden \& Zygmund {\S}3, Ex.\@ 14]
Show that the conclusion of part  (ii) of Exercise 13 (Problem) is false if
$\left|E\right|_e=+\infty$.
\end{problem}
\begin{proof}
Let $V\subset[0,1]$ denote the Vitali set defined in 3.38 and consider the
union $E\coloneqq V\cup(2,\infty)$. It is clear that the inner and outer
measure of $E$ is $\infty$. However, $E$ itself is unmeasurable since
otherwise $E\cap[0,1]=V\cap[0,1]=V$ would be measurable.
\end{proof}
\newpage

\begin{problem}[Wheeden \& Zygmund {\S}3, Ex.\@ 16]
Prove (3.34).
\end{problem}
\begin{proof}
\begin{lemma*}
$\left|P\right|=v(P)$.
\end{lemma*}
Let $\left\{\mathbf{e}_k\right\}_{k=1}^n$ be a set of orthogonal vectors
emenating from a point in $\bbR^n$. The closed parallelapiped corresponding
to $\left\{\mathbf{e}_k\right\}_{k=1}^n$ is the set
\begin{equation}
\label{eq:closed-parallelapiped}
P=
\left\{\,\bfx\;\middle|\;
\text{$\bfx=\sum_{k=1}^nt_k\mathbf{e}_k$, $0\leq t\leq 1$}\,\right\}.
\end{equation}
\end{proof}
\newpage

\begin{problem}[Wheeden \& Zygmund {\S}3, Ex.\@ 18]
Prove that outer measure is \emph{translation invariant}; that is, if
$E_{\textbf{h}}\coloneqq\left\{\,\bfx+\mathbf{h}\;\middle|\;\bfx\in
  E\,\right\}$ is the translate of $E$ by $\mathbf{h}$,
$\mathbf{h}\in\bbR^n$, show that
$\left|E_{\mathbf{h}}\right|_e=\left|E\right|_e$. If $E$ is measurable,
show that $E_{\mathbf{h}}$ is also measurable. [This fact was used in
proving (3.37).]
\end{problem}
\begin{proof}
First, note that $E_{\mathbf{h}}=T(E)$ where $T\colon\bbR^n\to\bbR^n$ is
the linear transformation $\bfx\mapsto\bfx+\mathbf{h}$. By 3.36, we know
that $\left|E_{\mathbf{h}}\right|_e\leq\delta |E|_e=|E|_e$ since
$\delta=\left|\det T\right|=1$.\footnote{This can be computed classically,
or if we view the translation map $T$ as the matrix
\[
\begin{bmatrix}
1&&\cdots&h_1\\
&1&\cdots&h_2\\
&&\ddots&\vdots\\
&&&1
\end{bmatrix}
\]
which clearly has determinant equal to $1$, every other term in the sum
$\sum_{\sigma\in S_n}(\sgn\sigma)a_{1,\sigma(i)}\cdots a_{n,\sigma(n)}$
besides the diagonal having at least one $a_{i,j}=0$.} Therefore, to prove
equality, it suffices to demonstrate the reverse inequality. First, let us
observe that if $G\subset\bbR^n$ is open, then $G_{\mathbf{h}}$ is open so
that if $F$ is a $G_\delta$ set, then $F_{\mathbf{h}}$ is $G_\delta$.
\end{proof}
\newpage

\begin{problem}[Wheeden \& Zygmund {\S}4, Ex.\@ 1]
Prove corollary (4.2) and theorem (4.8)
\end{problem}
\begin{proof}
\begin{corollary*}[Wheeden \& Zygmund, 4.2]
If $f$ is measurable, then $\left\{\,f>-\infty\,\right\}$,
$\left\{\,f<+\infty\,\right\}$, $\left\{\,f=+\infty\,\right\}$,
$\left\{\,a\leq f\leq b\,\right\}$, $\left\{\,f=a\,\right\}$, etc., are all
measurable. Moreover $f$ is measurable if and only if $\left\{\,a\leq
  f<+\infty\,\right\}$ is measurable for every finite $a$.
\end{corollary*}

\begin{theorem*}[Wheeden \& Zygmund, 4.8]
If $f$ is measurable and $\lambda$ is any real number, then $f+\lambda$ and
$\lambda f$ are measurable.
\end{theorem*}
\end{proof}
\newpage

\begin{problem}[Wheeden \& Zygmund {\S}4, Ex.\@ 2]
Let $f$ be a simple function, taking its distinct values on disjoint sets
$E_1,...,E_N$. Show that $f$ is measurable if and only if $E_1,...,E_N$ are
measurable.
\end{problem}
\begin{proof}
\end{proof}

%%% Local Variables:
%%% mode: latex
%%% TeX-master: "../MA544-HW-Current"
%%% End:
