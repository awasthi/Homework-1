% #5 - Due Feb. 15 Read Sections 3.6, 4.1 Chapter 3: # 14, 16, 18, Chapter 4: # 1, 2.
\begin{problem}[Wheeden \& Zygmund {\S}3, Ex.\@ 14]
Show that the conclusion of part  (ii) of Exercise 13 (Problem) is false if
$\left|E\right|_e=+\infty$.
\end{problem}
\begin{proof}
Let $V\subset[0,1]$ denote the Vitali set defined in 3.38 and consider the
union $E\coloneqq V\cup(2,\infty)$. It is clear that the inner and outer
measure of $E$ is $\infty$. However, $E$ itself is unmeasurable since
otherwise $E\cap[0,1]=V\cap[0,1]=V$ would be measurable.
\end{proof}
\newpage

\begin{problem}[Wheeden \& Zygmund {\S}3, Ex.\@ 16]
Prove (3.34).
\end{problem}
\begin{proof}
\begin{lemma*}
$\left|P\right|=v(P)$.
\end{lemma*}
The result is trivial by 3.36, but then again, it is used to prove 3.36.

Let $\left\{\mathbf{e}_k\right\}_{k=1}^n$ be a set of orthogonal vectors
emenating from a point in $\bbR^n$. The closed parallelapiped corresponding
to $\left\{\mathbf{e}_k\right\}_{k=1}^n$ is the set
\begin{equation}
\label{eq:closed-parallelapiped}
P=
\left\{\,\bfx\;\middle|\;
\text{$\bfx=\sum_{k=1}^nt_k\mathbf{e}_k$, $0\leq t\leq 1$}\,\right\}.
\end{equation}
Let's do it this way. Let $T\colon\bbR^n\to\bbR^n$ be the map which sends
the standard basis of $\bbR^n$ to
$\left\{\mathbf{e}_k\right\}_{k=1}^n$. This map has determinant not equal
to
\end{proof}
\newpage

\begin{problem}[Wheeden \& Zygmund {\S}3, Ex.\@ 18]
Prove that outer measure is \emph{translation invariant}; that is, if
$E_{\textbf{h}}\coloneqq\left\{\,\bfx+\mathbf{h}\;\middle|\;\bfx\in
  E\,\right\}$ is the translate of $E$ by $\mathbf{h}$,
$\mathbf{h}\in\bbR^n$, show that
$\left|E_{\mathbf{h}}\right|_e=\left|E\right|_e$. If $E$ is measurable,
show that $E_{\mathbf{h}}$ is also measurable. [This fact was used in
proving (3.37).]
\end{problem}
\begin{proof}
By 3.6, given $\varepsilon>0$, there exists an open set $G\supset E$ with
$|G|_e\leq|E|_e+\varepsilon$. Let $T\colon\bbR^n\to\bbR^n$ denote the
linear transformation $\bfx\mapsto\bfx+\mathbf{h}$,
$\mathbf{h}\in\bbR^n$. By 3.35 we have $|G|_e=|G|=|T(G)|=|T(G)|_e$ and
$T(G)$ is an open set containing $E_{\mathbf{h}}$. Hence, we have an upper
bound on the outer measure of $E_{\mathbf{h}}$ given by the inequality
\begin{equation}
\label{eq:upper-bound}
\left|E_{\mathbf{h}}\right|_e\leq|T(G)|_e=|G|_e\leq |E|_e+\varepsilon.
\end{equation}

On the other hand, by 3.6 there exists an open set $H\supset
E_{\mathbf{h}}$ with
$|H|_e\leq\left|E_{\mathbf{h}}\right|+\varepsilon$. Then by 3.35, we get
the inequality
\begin{equation}
\label{eq:lower-bound}
|E|_e\leq\left|T^{-1}(H)\right|_e=|H|_e\leq\left|E_{\mathbf{h}}\right|_e+\varepsilon.
\end{equation}
Putting \eqref{eq:upper-bound} and \eqref{eq:lower-bound} we have
\[
|E|_e-\varepsilon\leq\left|E_{\mathbf{h}}\right|_e\leq|E|_e+\varepsilon.
\]
Letting $\varepsilon\to 0$, we have
$|E|_e=\left|E_{\mathbf{h}}\right|_e$. It then follows that if $E$ is
measurable then $E_{\mathbf{h}}$ is measurable since $E_{\mathbf{h}}=T(E)$
and $T$ is a Lipschitz transformation and
$|E|=\left|E_{\mathbf{h}}\right|$.
\end{proof}
\newpage

\begin{problem}[Wheeden \& Zygmund {\S}4, Ex.\@ 1]
Prove corollary (4.2) and theorem (4.8)
\end{problem}
\begin{proof}
\begin{corollary*}[Wheeden \& Zygmund, 4.2]
If $f$ is measurable, then $\left\{\,f>-\infty\,\right\}$,
$\left\{\,f<+\infty\,\right\}$, $\left\{\,f=+\infty\,\right\}$,
$\left\{\,a\leq f\leq b\,\right\}$, $\left\{\,f=a\,\right\}$, etc., are all
measurable. Moreover $f$ is measurable if and only if
$\left\{\,a<f<+\infty\,\right\}$ is measurable for every finite $a$.
\end{corollary*}
Suppose that $f$ is measurable. By 4.1, we have $\left\{\,f\geq
  a\,\right\}$ and $\left\{\,f\leq a\,\right\}$ are measurable so
\begin{equation}
\label{eq:equals}
\left\{\,f=a\,\right\}=
\left\{\,f\geq a\,\right\}\cap\left\{\,f\leq a\,\right\}
\end{equation}
is measurable and for $b>a$
\begin{equation}
\label{eq:between}
\left\{\,a\leq f\leq b\,\right\}
=\left\{\,f\geq a\,\right\}\cap\left\{\,f\leq b\,\right\}.
\end{equation}
\begin{proof}[Proof of corollary 4.2]
\renewcommand\qedsymbol{$\clubsuit$}
Now, consider the sequence of measurable sets
$\left\{E_k\right\}_{k=0}^\infty$ where
$E_k\coloneqq\left\{\,f<a+k\,\right\}$. Then
$\left\{\,f<\infty\,\right\}=\bigcup_{k=0}^\infty E_k$ and since
$E_k\nearrow\left\{\,f<\infty\,\right\}$ (take $\bfx\in E_k$ then
$f(\bfx)<a+k$ so $f(\bfx)<a+k+1$ $\implies$ $\bfx\in E_{k+1}$), by 3.26, we
have $\left\{\,f<\infty\,\right\}$ is measurable.

Similarly for $\left\{\,f>-\infty\,\right\}$ we may consider the family
$\left\{E_k\right\}_{k=0}^\infty$ where
$E_k\coloneqq\left\{\,f>a-k\,\right\}$ (take $\bfx\in E_k$ then
$f(\bfx)>a-k$ so $f(\bfx)>a-k-1$ $\implies$ $\bfx\in E_{k+1}$) and taking
the limit as $k\to\infty$ we have $\left\{\,f>-\infty\,\right\}$ is
measurable.

Last but not least, since $\left\{\,f<\infty\,\right\}$ is measurable,
$\left\{\,f=\infty\,\right\}=\left\{\,f<\infty\,\right\}^\complement$ is
measurable.

Now, $\implies$ suppose $f$ is measurable. Then $\left\{\,a<
  f<b\,\right\}=\left\{\,a\leq f\leq
  b\,\right\}\cap\left\{\,f=a\,\right\}^\complement\cap\left\{\,f=b\,\right\}^\complement$
is measurable for all finite
$a<b$. Moreover, the family $\left\{E_k\right\}_{k=0}^\infty$ of sets
$\left\{E_k\right\}_{k=0}^\infty$ where $E_k\coloneqq\left\{\,a\leq
f<b+k\,\right\}$ is measurable for all $k$ so, by 3.26, $\left\{\,a\leq
f<\infty\,\right\}$ is measurable since $E_k\nearrow\left\{\,a\leq
f<\infty\,\right\}$.

$\impliedby$ On the other hand, suppose that $\left\{\,a\leq
  f<\infty\,\right\}$ is measurable for every finite $a$. Then, for fixed
$a\in\bbR$ the family $\left\{E_k\right\}_{k=0}^\infty$ where
$E_k\coloneqq\left\{a-k\leq f<\infty\right\}$ is measurable. By 3.26,
$\{\,f<\infty\,\}$ is measurable so
$\left\{\,f=\infty\,\right\}=\left\{\,f<\infty\,\right\}^\complement$ is
measurable. Thus,
\[
\left\{\,f>a\,\right\}=\left\{\,a<f<\infty\,\right\}\cup\left\{\,f=\infty\,\right\}
\]
is measurable so $f$ is measurable.
\end{proof}
\begin{theorem*}[Wheeden \& Zygmund, 4.8]
If $f$ is measurable and $\lambda$ is any real number, then $f+\lambda$ and
$\lambda f$ are measurable.
\end{theorem*}
\begin{proof}[Proof of theorem 4.8]
\renewcommand\qedsymbol{$\clubsuit$}
If $f$ is measurable, then $\left\{\,f>a\,\right\}$ is measurable for all
$a$ so $\left\{\,f>a-\lambda\,\right\}=\left\{\,f+\lambda>a\,\right\}$ is
measurable for all $a$. Hence, $f+\lambda$ is measurable.

If $\lambda\neq 0$, then $\left\{\,f>a/\lambda\,\right\}$ is measurable for
all $a$ so $\lambda f$ is measurable. If $\lambda=0$ then $\lambda f=0$ is
clearly measurable since $\left\{\,0>a\,\right\}=(a,0)$ is open for all
$a$ (possibly empty if $a\geq 0$, but still an open set).

Thus, $f+\lambda$ and $\lambda f$ are measurable.
\end{proof}
\end{proof}
\newpage

\begin{problem}[Wheeden \& Zygmund {\S}4, Ex.\@ 2]
Let $f$ be a simple function, taking its distinct values on disjoint sets
$E_1,...,E_N$. Show that $f$ is measurable if and only if $E_1,...,E_N$ are
measurable.
\end{problem}
\begin{proof}
$\implies$ Suppose $f$ is a simple function taking distinct values on
disjoint sets $E_1,...,E_N$. Then $f=\sum_{k=1}^N a_k\chi_{E_k}$. If $f$ is
measurable, $\left\{\,f>a\,\right\}$ is measurable for all finite $a$. In
particular, $\left\{\,f>a_k\,\right\}=E_k$ is measurable.

$\impliedby$ On the other hand, suppose that $E_k$ is measurable for all
$1\leq k\leq N$. Then $\chi_{E_k}$ is measurable and by Problem 5.4, the
sum $f=\sum_{k=1}^N a_k\chi_{E_k}$ is measurable.
\end{proof}

%%% Local Variables:
%%% mode: latex
%%% TeX-master: "../MA544-HW-Current"
%%% End:
