 % Chapter 5: # 2, 4, 6, 7, 21. Chapter 6: # 10, 11.
\begin{problem}[Wheeden \& Zygmund {\S}5, Ex.\@ 2]
Show that the conclusion of (5.32) are not true without the assumption that
$\varphi\in L(E)$. [In part (ii), for example, take $f_k=\chi_{(k,\infty)}$.]
\end{problem}
\begin{proof}
(ii) Following the hint, consider the family of decreasing functions
$f_k=\chi_{(k,\infty)}$ on $\bfR$ which converge pointwise to $0$. Since
$\int_\bfR f_k=\int_\bfR\chi_{(k,\infty)}=\left|(k,\infty)\right|=\infty$
for all $k$, the sequence of integrals $\int_\bfR f_k\to\infty$, but
$\int_\bfR 0=0$.
\\\\
For part (i) we may again consider the indicator function
$\chi_{(k,\infty)}$ and define $f_k\coloneqq -\chi_{(k,\infty)}$. Then
$f_k\nearrow 0$, but $\int_\bfR f_k=-\int_\bfR\chi_{(k,\infty)}=-\infty$
for all $k$.
\end{proof}
\newpage

\begin{problem}[Wheeden \& Zygmund {\S}5, Ex.\@ 4]
If $f\in L(0,1)$, show that $x^kf(x)\in L(0,1)$ for $k=1,2,...$, and
$\int_0^1 x^kf(x)\,dx\to 0$.
\end{problem}
\begin{proof}
Since $x^k$ is a polynomial and therefore continuous, $x^k$ is measurable
as a consequence of Theorem 4.3. Moreover, since $\bigl|x^k\bigr|\leq 1$
for all $k$, by Theorem 5.30, $x^kf\in L(0,1)$.

Now, by the Stone--Weistraß approximation theorem, given $\varepsilon>0$
there exists a polynomial $p$ such that
$|p-f|<\varepsilon$ for every $x\in[0,1]$. Hence, we have
\begin{align*}
\left|x^kf\right|&=\left|x^kf+x^kp-x^kp\right|\\
                 &=\left|x^k(f-p)\right|+\left|x^kp\right|\\
                 &=\left|x^k\right|\left(\varepsilon+|p|\right)
\end{align*}
which goes to $0$ as $k\to\infty$ a.e.\@ except at $x=1$. Thus, by
Lebesgue's dominated convergence theorem, since $x^kf\to 0$ a.e.\@ on
$[0,1]$ and $\left|x^kf\right|\leq|f|$ a.e.\@ in $[0,1]$ for all $k$, we
have $\int_0^1 x^kf\to 0$ as desired.
\end{proof}
\newpage

\begin{problem}[Wheeden \& Zygmund {\S}5, Ex.\@ 6]
Let $f(x,y)$, $0\leq x,y\leq 1$, satisfy the following conditions: for each
$x$, $f(x,y)$ is an integrable function of $y$, and
$\partial f(x,y)/\partial x$ is a bounded function of $(x,y)$. Show that
$\partial f(x,y)/\partial x$ is a measurable function of $y$ for each $x$
and
\[
\frac{d}{dx}\int_0^1f(x,y)\;dy=\int_0^1\frac{\partial}{\partial x}f(x,y)\;dy.
\]
\end{problem}
\begin{proof}
First we will show that $\partial f/\partial x$ is measurable as a function
of $y$. Since $\int_0^1 f(x,y)\;dy$ exists for every $x$, by Theorem 5.1
$f(x,y)$ is measurable as a function of $y$ so $\partial f/\partial x$ is
measurable by Theorem 4.12 since it is limit of the sequence
\begin{equation}
\label{eq:diff-sequence}
f_n(x,y)\coloneqq\frac{f(x+(1/n),y)-f(x,y)}{1/n}
\end{equation}
which is a sum of measurable functions of $y$.

To prove the second half we will show that the sequence
\begin{equation}
\label{eq:seq-derivatives-of-ints}
g_n\coloneqq\int_0^1 f_n(x)\;dy\longrightarrow
\frac{d}{dx}\int_0^1 f(x,y)\;dy.
\end{equation}
Since $\partial f/\partial x$ is bounded, there exists $M$ such that
$\left|\partial f/\partial x\right|<M$ so convergence of $f_n$ to $\partial
f/\partial x$ means that for every $\varepsilon>0$, there exists $N$ such
that $n\geq N$ implies $\left|f_n\right|<M+\varepsilon$. Then by the
bounded convergence theorem, $\int_0^1 f_n\to\int_0^1\to\partial f/\partial
x$. But by definition the limit as $n\to\infty$ of
\begin{align*}
\int_0^1f_n(x)\;dy
&=\int_0^1\frac{f(x+1/n,y)-f(x,y)}{1/n}\;dy\\
&=\frac{\int_0^1 f(x+1/n,y)\;dy-\int_0^1 f(x,y)\;dy}{1/n}
\end{align*}
is
\[
\frac{d}{dx}\int_0^1 f(x,y)\;dy.
\]
Hence, by the uniqueness of limit, we have
\[
\frac{d}{dx}\int_0^1f(x,y)\;dy=\int_0^1\frac{\partial}{\partial x}f(x,y)\;dy
\]
a.e.\@ on $[0,1]$.
\end{proof}
\newpage

\begin{problem}[Wheeden \& Zygmund {\S}5, Ex.\@ 7]
Give an example of an $f$ that is not integrable, but whose improper
Riemann integral exists and is finite.
\end{problem}
\begin{proof}
The following is a standard example of a function $f$ that is improperly
Riemann integrable, but not Lebesgue integrable. Set $f\coloneqq
\sin(x)/x$. Then the Riemann $\int_{-\infty}^\infty f\;dx=2\pi$ may be computed
fairly easily by contour integration noting that
$f=\Im\left(e^{ix}/x\right)$ and applying Jordan's lemma.

However, by Theorem 5.21, $f$ is Lebesgue integrable if and only if $|f|$
is Lebesgue integrable however, we show that
$\int_{\bfR}|f|\;dx=\infty$. To see this note that
\begin{align*}
\int_{\pi}^{(n+1)\pi}\left|\frac{\sin x}{x}\right|\;dx
&=\sum_{k=1}^n\int_{k\pi}^{(k+1)\pi}\left|\frac{\sin x}{x}\right|\;dx\\
\intertext{make the substitution $x=t+k\pi$, then}
&=\sum_{k=1}^n\int_0^\pi\left|\frac{\sin(t+k\pi)}{t+k\pi}\right|\;dt\\
&=\sum_{k=1}^n\int_0^\pi\left|\frac{\sin(t+k\pi)}{t+k\pi}\right|\;dt\\
\intertext{and since $t+k\pi\leq \pi+k\pi$ for $0\leq t\leq\pi$ we have}
&\geq\sum_{k=1}^n\int_0^\pi\left|\frac{\sin(t+k\pi)}{\pi+k\pi}\right|\;dt\\
&\geq\sum_{k=1}^n\frac{1}{\pi(k+1)}\int_0^\pi\sin(t+k\pi)\;dt\\
&\geq\sum_{k=1}^n\frac{2}{\pi(k+1)}
\end{align*}
which clearly diverges as $n\to\infty$ since the lower bound above is the
scaled harmonic series starting at $2$. Thus, $|f|$ is not integrable over
$\bfR$ so $f$ is not integrable over $\bfR$.
\end{proof}
\newpage

\begin{problem}[Wheeden \& Zygmund {\S}5, Ex.\@ 21]
If $\int_A f=0$ for every measurable subset A of a measurable set $E$, show
that $f=0$ a.e.\@ in $E$.
\end{problem}
\begin{proof}
Since $E$ is measurable, $\int_E f=0$ so $f$ is measurable. Write $E$ as
the union
\begin{equation}
\label{eq:rewrite-e}
E=\left\{\,f>0\,\right\}\cup\left\{f=0\right\}\cup\left\{\,f<0\,\right\}.
\end{equation}
Then
\begin{align*}
\int_{\left\{\,f>a\,\right\}}f\;dx&=\int_Ef^+\;dx&
\int_{\left\{\,f<0\,\right\}}f\;dx&=-\int_Ef^-\;dx\\
&=0&&=0.
\end{align*}
Hence, by Theorem 5.11, $f^+=0$ and $f^-=0$ a.e.\@ on $E$. Thus,
$f=f^+-f^-=0$ a.e.\@ on $E$
\end{proof}
\newpage

\begin{problem}[Wheeden \& Zygmund {\S}6, Ex.\@ 10]
 Let $V_n$ be the volume of the unit ball in $\bfR^n$. Show by using
 Fubini's theorem that
\[
V_n=2V_{n-1}\int_0^1\left(1-t^2\right)^{(n-1)/2}\;dt.
\]
(We also observe that by setting $w=t^2$, the integral is a multiple of a
classical $\beta$-function and so can be expressed in terms of the
$\Gamma$-function: $\Gamma(s)=\int_0^\infty e^{-t}t^{s-1}\;dt$, $s>0$.)
\end{problem}
\begin{proof}

\end{proof}
\newpage

\begin{problem}[Wheeden \& Zygmund {\S}6, Ex.\@ 11]
Use Fubini's theorem to prove that
\[
\int_{\bfR^n}e^{-|\bfx|^2}\;d\bfx=\pi^{n/2}.
\]
(For $n=1$, write $\left(\int_{-\infty}^\infty
e^{-x^2}\;dx\right)^2=\int_{-\infty}^\infty\int_{-\infty}^\infty e^{-x^2-y^2}\;dxdy$
and use polar. For $n>1$, use the formula $e^{-|\bfx|^2}=e^{-{x_1}^2}\cdots
e^{-{x_n}^2}$ and Fubini's theorem to reduce the case $n=1$.)
\end{problem}
\begin{proof}
Using the hint, by induction for $n=1$ we have
\begin{equation}
\label{eq:e-x^2-n-1}
\left(\int_{-\infty}^\infty e^{-x^2}\;dx\right)^2
=\int_{-\infty}^\infty\int_{-\infty}^\infty e^{-x^2-y^2}\;dxdy.
\end{equation}
Now we make a change of $x=r\cos\theta$, $y=r\sin\theta$, $0\leq
r\leq\infty$, $0\leq\theta\leq\pi$ and the determinant of the Jacobian is
$r$ so we have
\begin{align*}
\int_{-\infty}^\infty\int_{-\infty}^\infty e^{-x^2-y^2}\;dxdy
&=\int_{0}^{2\pi}\int_{0}^\infty re^{-r^2}\;drd\theta\\
&=2\pi\int_0^\infty re^{-r^2}\;dr\\
\intertext{by making the substitution, $u=r^2$}
&=\pi\int_0^\infty e^{-u}\;du\\
&=\pi
\end{align*}
so $\int_\bfR e^{-x^2}=\sqrt{\pi}$.

Assume by induction that $\int_{\bfR^k}e^{-|\bfx|^2}\;d\bfx=\pi^{k/2}$ for
all $k<n$. Then writing
\begin{equation}
\label{eq:e-x^n-boldface}
\int_{\bfR^n}e^{-|\bfx|^2}\;d\bfx=
e^{-|\bfx|^2}=\idotsint_{\bfR^n}e^{-{x_1}^2}\cdots e^{-{x_n}^2}\;dx_1\cdots dx_n
\end{equation}
Then by Fubini's theorem
\begin{align*}
\idotsint_{\bfR^n}e^{-{x_1}^2}\cdots e^{-{x_n}^2}dx_1\cdots dx_n
&=\int_\bfR\left(\idotsint_{\bfR^{n-1}}e^{-{x_1}^2}\cdots e^{-{x_n}^2}\right)dx_1\\
&=\int_\bfR\left(\idotsint_{\bfR^{n-1}}e^{-{x_1}^2}\cdots
  e^{-{x_n}^2}\right)dx_1\\
&=\int_\bfR\left(\idotsint_{\bfR^{n-1}}e^{-{x_2}^2}\cdots
  e^{-{x_n}^2}\right)e^{-{x_1}^2}\;dx_1\\
&=\pi^{(n-1)/2}\int_\bfR e^{-{x_1}^2}\;dx_1\\
&=\pi^{(n-1)/2}\pi^{1/2}\\
&=\pi^{n/2}
\end{align*}
as desired.
\end{proof}

%%% Local Variables:
%%% mode: latex
%%% TeX-master: "../MA544-HW-Current"
%%% End:
