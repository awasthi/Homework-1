% #7 - Due Feb. 29 Read Sections 5.1-2. Chapter 4: #9, 11, 15, 18 (ignore
% the second part concerning convergence in measure);
\begin{problem}[Wheeden \& Zygmund {\S}4, Ex.\@ 9]
\begin{enumerate}[label=(\alph*)]
\item Show that the limit of a decreasing (increasing) sequence of
  functions usc (lsc) at $\bfx_0$ is usc (lsc) at $\bfx_0$. In particular,
  the limit of a decreasing (increasing) sequence of functions continuous
  at $\bfx_0$ is usc (lsc) at $\bfx_0$.
\item Let $f$ be usc and less than $\infty$ on $[a,b]$. Show that there
  exists continuous $f_k$ on $[a,b]$ such that $f_k\searrow f$.
\end{enumerate}
\end{problem}
\begin{proof}
(a) Without loss of generality, assume that the sequence of $f_k$'s are
define in all of $\bfR^n$. Suppose
$\left\{f_k\colon\bfR^n\to\bfR\right\}_{k=1}^\infty$ is a sequence of
decreasing functions which are usc at $\bfx_0$ and put $f\coloneqq\lim
f_k$. Then, since the $f_k$'s are decreasing, we have
\begin{equation}
  \label{eq:decreasing-estimate-on-f}
f(\bfx)\leq f_k(\bfx)\qquad\text{for all $\bfx\in\bfR^n$.}
\end{equation}
Moreover, $f_k$ being usc at $\bfx_0$ means that for any sequence
$\bfx_k\to\bfx_0$, $\varlimsup_{\bfx\to\bfx_0} f_k(\bfx)\leq
f_k(\bfx_0)$. Thus, by \eqref{eq:decreasing-estimate-on-f} we have
\begin{equation}
  \label{eq:limit-estimate-on-f}
\varlimsup_{\bfx\to\bfx_0}f(\bfx)\leq\varlimsup_{\bfx\to\bfx_0}
f_k(\bfx)\leq f_k(\bfx_0).
\end{equation}
Now, let $k\to\infty$ and we have $\varlimsup_{\bfx\to\bfx_0} f(\bfx)\leq
f(\bfx_0)$. Thus, $f$ is usc at $\bfx_0$.
\\\\
(b) Mimicking the construction of the function associated to the cantor set
divide the interval into $2n$ closed intervals
$I_1\coloneqq[y_0,y_1],...,I_{2n}\coloneqq[y_{2n-1},y_{2n}]$ and define
\begin{equation}
  \label{eq:define-fn}
f_n(y_k)\coloneqq\max_{I_{k}\cup I_{k+1}} f(x)
\end{equation}
and $f_n(x)$ is linear on $I_n$. Then the function $f_n$ is clearly
continuous on $[a,b]$. Moreover, we claim that $f_n\to f$.
\end{proof}
\newpage

\begin{problem}[Wheeden \& Zygmund {\S}4, Ex.\@ 11]
Let $f$ be defined on $\bfR^n$ and let $B(\bfx)$ denote the open ball
$\left\{\,\bfy\;\middle|\;\left|\bfx-\bfy\right|<r\,\right\}$ with center
$\bfx$ and fixed radius $r$. Show that the function
$g(\bfx)=\sup\left\{\,f(\bfy)\;\middle|\;\bfy\in B(\bfx)\,\right\}$ is lsc
and the function $h(\bfx)=\inf\left\{\,f(\bfy)\;\middle|\;\bfy\in
B(\bfx)\,\right\}$ is usc on $\bfR^n$. Is the same true for the closed ball
$\left\{\,\bfy\;\middle|\;\left|\bfx-\bfy\right|\leq r\,\right\}$?
\end{problem}
\begin{proof}
% Fix $\bfx_0\in\bfR^n$ and let $\left\{\bfx_k\right\}$ be any sequence
% converging to $\bfx_0$. Then since $g(\bfx_0)$ is the supremum of $f(\bfx)$
% for all $\bfx\in B(\bfx_0)$, given $\varepsilon>0$ there exists $\bfy\in
% B(\bfx_0)$ such that $f(\bfy)>g(\bfx_0)-\varepsilon$. Let
% $\delta\coloneqq\frac{1}{2}(r-|\bfy-\bfx_0|)$ ($\delta$ is positive since
% $\bfy\in B(\bfx_0)$ hence $|\bfy-\bfx_0|<r$). Then $\bfx_k\to\bfx_0$ means
% that for our chosen $\delta$ there exists an index $N\in\bfN$ such that
% $\bfx_n\in B_\delta(\bfx_0)$ for every $n\geq N$. Now, note that for
% $\bfx_n\in B_\delta(\bfx_0)$ we have
% \begin{equation}
% \label{eq:ball-estimates-x-y}
% \begin{aligned}
% |\bfx_n-\bfy|&=|\bfx_n-\bfx_0-(\bfy-\bfx_0)|\\
%            &\leq|\bfx_n-\bfx_0|+|\bfy-\bfx_0|\\
%            &=\tfrac{1}{2}(r-|\bfy-\bfx_0|)+|\bfy-\bfx_0|\\
%            &=\tfrac{1}{2}r+\tfrac{1}{2}|\bfy-\bfx_0|\\
%            &<r
% \end{aligned}
% \end{equation}
% so $\bfy\in B(\bfx_n)$ hence, $g(\bfx_n)\geq f(\bfy)>g(\bfx_0)-\varepsilon$
% so $\varliminf_{\bfx_k\to\bfx_0} g(\bfx_n)>g(\bfx_0)-\varepsilon$ since
% $\varliminf g(\bfx_0)=\lim_{j\to\infty}\left\{\,\inf_{k\geq
%     j}g(\bfx_0)\,\right\}$ by definition. Now let $\varepsilon\to 0$.
Fix $\bfx_0\in\bfR^n$. By 4.14(ii), given a real number $m$ such that
$f(\bfx_0)>m$, it suffices to show that there exists $\delta>0$ such that
for every $\bfx\in B_\delta(\bfx_0)$ we have $g(\bfx)>m$. Since $g(\bfx_0)$
is the supremum of $f(\bfx)$ for all $\bfx\in B(\bfx_0)$, given
$\varepsilon>0$ such that $g(\bfx_0)-\varepsilon>m$ there exists $\bfy\in
B(\bfx_0)$ such that $f(\bfy)>g(\bfx_0)-\varepsilon$. Let
$\delta\coloneqq\frac{1}{2}(r-|\bfy-\bfx_0|)$ ($\delta$ is positive since
$\bfy\in B(\bfx_0)$ hence $|\bfy-\bfx_0|<r$). Then for any $\bfx\in
B_\delta(\bfx_0)$ we have
\begin{equation}
\label{eq:ball-estimates-x-y}
\begin{aligned}
|\bfx-\bfy|&=|\bfx-\bfx_0-(\bfy-\bfx_0)|\\
           &\leq|\bfx-\bfx_0|+|\bfy-\bfx_0|\\
           &=\tfrac{1}{2}(r-|\bfy-\bfx_0|)+|\bfy-\bfx_0|\\
           &=\tfrac{1}{2}r+\tfrac{1}{2}|\bfy-\bfx_0|\\
           &<r
\end{aligned}
\end{equation}
so $\bfy\in B(\bfx)$. Hence, $g(\bfx)\geq f(\bfy)>g(\bfx_0)-\varepsilon>m$,
i.e.,
$B_\delta(\bfx_0)\subset\left\{\,\bfx\;\middle|\;f(\bfx)>m\,\right\}$. Thus,
$g$ is lsc on $\bfR^n$.

The proof for $h$ is similar to $g$. In fact, note that for any set
$E\subset\bfR$ we have $\inf E=-\sup(-E)$ so that if we set $f'\coloneqq
-f$ and define $g'(\bfx)\coloneqq\sup\left\{\,f'(\bfy)\;\middle|\;\bfy\in
  B(\bfx)\,\right\}$. By the above, $g'$ is lsc in $\bfR^n$ so $h=-g'$ is
usc in $\bfR^n$.
\end{proof}
\newpage

\begin{problem}[Wheeden \& Zygmund {\S}4, Ex.\@ 15]
Let $\left\{f_k\right\}$ be a sequence of measurable functions defined on a
measurable set $E$ with $|E|<\infty$. If $|f_k(M)|\leq M<\infty$ for all
$k$ for each $\bfx\in E$, show that given $\varepsilon>0$, there is closed
$F\subset E$ and finite $M$ such that $|E\minus F|<\varepsilon$ and
$|f_k(\bfx)|\leq M$ for all $\bfx\in F$.
\end{problem}
\begin{proof}
Define $f\coloneqq\sup|f_k|$. Note that, since $|f_k|=f^++f^-$ and $f^+$
and $f^-$ are measurable, $|f_k|$ is measurable hence, by 4.11, $f$ is
measurable. Now, given $\varepsilon>0$ by Lusin's theorem $f$ has the
$\calC$-property on $E$, i.e., there exists a closed subset $F$ of $E$ such
that $|E\minus F|<\varepsilon/2$ and $f$ is continuous when restricted to
$F$. Take the $\delta>0$ such that
$\bigl|E\minus\overline{B_\delta(\mathbf{0})}\bigr|<\varepsilon/2$. Then
$F\cap\overline{B_\delta(\mathbf{0})}$ is closed and compact and we have
\begin{align*}
\left|E\minus(F\cap\overline{B_\delta(\mathbf{0})})\right|
&=\left|E\minus F\cup
  \bigl(E\minus\overline{B_\delta(\mathbf{0})}\bigr)\right|\\
&\leq|E\minus F|+\bigl|E\minus\overline{B_\delta(\mathbf{0})}\bigr|\\
&<\varepsilon.
\end{align*}
By Problem 6.2 (W\&Z, 4.7) $f$ achieves its maximum $M$ on
$F\cap\overline{B_\delta(\mathbf{0})}$. Thus, $|f_k|\leq M$ for all
$\bfx\in F\cap\overline{B_\delta(\mathbf{0})}$.
\end{proof}
\newpage

\begin{problem}[Wheeden \& Zygmund {\S}4, Ex.\@ 18]
If $f$ is measurable on $E$, define
$\omega_f(a)=\left|\left\{\,f>a\,\right\}\right|$ for
$-\infty<a<\infty$. If $f_k\nearrow f$, show that
$\omega_{f_k}\nearrow\omega_f$. If $f_k\to f$, show that
$\omega_{f_k}\to\omega_f$ at each point of continuity of $\omega_f$. [For
the second part, show that if $f_k\to f$, then
$\varlimsup_{k\to\infty}\omega_{f_k}(a)\leq\omega_f(a-\varepsilon)$ and
$\varliminf_{k\to\infty}\omega_{f_k}(a)\geq\omega_f(a+\varepsilon)$ for every
$\varepsilon>0$.]
\end{problem}
\begin{proof}
For fixed $a$ define $E_k\coloneqq\left\{\,f_k>a\,\right\}$. Then we have
$E_1\subset E_2\subset\cdots$ so $E_k\nearrow\bigcup E_k$. Now, it is clear
that $\left\{\,f>a\,\right\}\supset\bigcup E_k$ since $\bfx\in\bigcup E_k$
implies that $\bfx\in E_k$ for some $k$ so $f_k(\bfx)>a$ for all $K\geq
k$. In particular, $f(\bfx)>a$. Thus, $\bfx\in\left\{\,f>a\,\right\}$. On the
other hand, if $\bfx\in\left\{\,f>a\,\right\}$ then $f(\bfx)>a$ so
$f_k\nearrow f$ implies that for sufficiently large $N$ we have
$f_N(\bfx)>a$. Thus, $\bfx\in E_N$ so $\bfx\in\bigcup E_k$ and we have
\begin{equation}
\label{eq:equality-of-f-a}
\left\{\,f>a\,\right\}=\bigcup E_k.
\end{equation}
It follows by 3.26(i) that $\omega_{f_k}\nearrow\omega_f$ point-wise.
\end{proof}
\newpage

% (Chapter 5: # 1, 2, 3, 4).
\begin{problem}[Wheeden \& Zygmund {\S}5, Ex.\@ 1]
If $f$ is a simple measurable function (not necessarily positive) taking
values $a_j$ on $E_j$, $j=1,...,N$, show that $\int_E f=\sum_{j=1}^N
a_j\left|E_j\right|$. [Use (5.24)].
\end{problem}
\begin{proof}
Since $f$ is a simple measurable function $E_k\cap E_\ell=\emptyset$ for
$k\neq\ell$. Since $E\coloneqq\bigcup_{j=1}^N E_j$ is countable, by 5.24 we
have
\[
\int_E f=\sum_{j=1}^N\int_{E_j} f=\sum_{j=1}^N\int a_j\chi_{E_j}=\sum_{j=1}^N a_j|E_j|.
\]
\end{proof}
\newpage

\begin{problem}[Wheeden \& Zygmund {\S}5, Ex.\@ 3]
Let $\left\{f_k\right\}$ be a sequence of nonnegative measurable functions
defined on $E$. If $f_k\to f$ and $f_k\leq f$ a.e.\@ on $E$, show that
$\int_E f_k\to\int_E f$.
\end{problem}
\begin{proof}
By Fatou's lemma we have
\begin{equation}
  \label{eq:fatou-lower-bound}
\int_E f=\int_E\varliminf_{k\to\infty} f_k\leq\varliminf_{k\to\infty}\int_E f_k.
\end{equation}
Now, by 5.10 since $f_k\leq f$ a.e.\@ on $E$ we have $\int_E
f_k\leq\int_E f$ so
\begin{equation}
  \label{eq:upper-bound}
\varlimsup_{k\to\infty}\int f_k\leq\int_E f.
\end{equation}
But $\varliminf\int_E f_k\leq\varlimsup\int_E f_k$ so we must have
$\lim\int_E f_k\to\int_E f$ as $k\to\infty$.
\end{proof}

%%% Local Variables:
%%% mode: latex
%%% TeX-master: "../MA544-HW-Current"
%%% End:
