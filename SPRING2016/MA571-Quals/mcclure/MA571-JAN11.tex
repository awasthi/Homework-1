\section{MA 571: Qualifying Exam, January 2011}
\begin{problem}
Let $A$ be a subset of a topological space $X$ and let $B$ be a subset of
$A$. Prove that $\bar A\minus\bar B\subset\overline{A\minus B}$.
\end{problem}
\begin{proof}
\end{proof}

\begin{problem}
Let $G$ be a topological group (thas is, a group with a topology for which
the group operations are continuous) and let $H$ be a subgroup of
$G$. Suppose that $G$ is connected, that $H$ is a normal subgroup of $G$,
and that the subspace topology on $H$ is discrete. Prove that $gh=hg$ for
every $g\in G$, $h\in G$.
\end{problem}
\begin{proof}
\end{proof}

\begin{problem}
Let $X$ be the space with two points and the discrete topology. Let
$Y=\prod_{n=1}^\infty X$, with the product topology. What are the connected
components of $Y$? Prove that your answer is correct.
\end{problem}
\begin{proof}
\end{proof}

\begin{problem}
Let $X$ and $Y$ be topological spaces. Let $x_0\in X$ and let $C$ be a
compact subset of $Y$. Let $N$ be an open set in $X\times Y$ containing
$\left\{x_0\right\}\times C$. Prove that there is an open set $U$
containing $x_0$ and an open set $V$ containing $C$ such that $U\times
V\subset N$.
\end{problem}
\begin{proof}
\end{proof}

\begin{problem}
Let $X$ and $Y$ be homotopy-equivalent topological spaces. Suppose that $X$
is connected. Prove that $Y$ is connected.
\end{problem}
\begin{proof}
\end{proof}

\begin{problem}
Let $p\colon E\to B$ be a covering map. Let $e_0\in E$ and $b_0\in B$ with
$p(e_0)=b_0$. Let $Y$ be simply connected (in particular, $Y$ is
path-connected). Let $y_0\in Y$. Let $f\colon Y\to B$ be continuous, with
$f(y_0)=b_0$.

Prove that the following function $g\colon Y\to E$ is well-defined: Given
$y\in Y$, choose a path $\gamma$ from $y_0$ to $y$; let $\beta$ be the lift
of $f\circ\gamma$ to $E$ starting at $e_0$; now define $g(y)=\beta(1)$.

You may use the fact (without proving it) that the lift of a path homotopy
is again a path homotopy.
\end{problem}
\begin{proof}
\end{proof}

\begin{problem}
Let $S^2$ be the $2$-sphere, that is, the following subspace of $\bfR^3$:
\[
\left\{\,(x,y,z)\in\bfR^3:x^2+y^2+z^2=1\,\right\}.
\]
Let $x_0$ be the point $(0,0,1)\in S^2$.

Use the Seifert--van Kampen theorem to prove that $\pi_1(S^2,x_0)$ is the
trivial group. You may use either of the two versions of the Seifert--van
Kampen theorem given in Munkres's book. You will not get credit for any
other method.
\end{problem}
\begin{proof}
\end{proof}

%%% Local Variables:
%%% mode: latex
%%% TeX-master: "../MA571-Quals"
%%% End:
