\section{MA571: Qualifying Exam, January 2012}
\begin{problem}
Let $X$ be a topological space. Recall that a subset of $X$ is \emph{dense}
if its closure is $X$. Prove that the intersection of two dense open sets is
dense.
\end{problem}
\begin{proof}
Suppose $U$ and $V$ are open dense subsets of $X$. We will show that $U\cap
V$ is dense in $X$, i.e., $\overline{U\cap V}=X$. To that end, we will show
that for any point $x\in X$, for any neighborhood $W$ of $x$, $W\cap(U\cap
V)\neq\emptyset$. Therefore, let $x\in X$. Let $W$ be a neighborhood of
$x$. Then, since $U$ is dense in $X$, $W\cap U\neq\emptyset$. Let $y\in
W\cap U$. Then, since $U$ and $V$ are open, $U\cap V$ is open so $U\cap V$
is a neighborhood of $y$. Moreover, since $V$ is dense in $X$, $(W\cap
U)\cap V\neq\emptyset$. Now, since intersection is associative, $(W\cap
U)\cap V=W\cap(U\cap V)\neq\emptyset$. Thus, $x\in\overline{U\cap V}$ and
we have $\overline{U\cap V}=X$ as desired.
\end{proof}

\begin{problem}
Let $X$ be a set with two elements $\{a,b\}$. Give $X$ the
\emph{indiscrete} topology. Give $X\times\bfR$ the product topology. Let
$A\subset X\times\bfR$ be $(\{a\}\times[0,1])\cup(\{b\}\times(0,1))$. Prove
that $A$ is compact.

You may use the fact that a set is compact if every covering by
\emph{basic} open sets has a finite subcovering.
\end{problem}
\begin{proof}
Let $\calU$ be an open cover of $A$ by basic open sets. Then each
$U\in\calU$ is of the form $\{a,b\}\times V$ where $V$ is an open subset of
$\bfR$. Then, the $V$'s, i.e., $\pi_2(U)$ where $\pi_2\colon
X\times\bfR\to\bfR$ is an open map by previous work, form open cover of
$[0,1]$ (since $\bigcup_{U\in\calU} U\supset A$, we must have
$\bigcup_{U\in\calU}\pi_2(U)\supset[0,1]$). Now, since $[0,1]$ is compact
in $X$ there is a finite collection of the $V$'s, say
$\left\{V_1,\dotsc,V_n\right\}$, that cover $[0,1]$. Call $U_i$ the element of
$\calU$ such that $\pi_2(U_i)=V_i$. Then the $U_i$'s form a finite subcover
of $A$. Thus, $A$ is compact.
\end{proof}

\begin{problem}
Let $B^2$ be the disk
\[
\left\{\,(x,y)\in\bfR^2:x^2+y^2\leq 1\,\right\}.
\]
Let $S^1$ be the circle
\[
\left\{\,(x,y)\in\bfR^2:x^2+y^2=1\,\right\}.
\]
Prove that there is an equivalence relation $\sim$ such that $B^2$ is
homeomorphic to $(S^1\times[0,1])/{\sim}$. As port of your proof explain
how you are using one or more properties of the quotient topology.
\end{problem}
\begin{proof}
Such an equivalence relation is called the cone of $S^1$. We define it as
follows, let $(x,y,z),(x',y',z')\in S^1\times[0,1]$ then we say
$(x,y,z)\sim(x',y',z')$ if and only if $(x,y)=(x',y')$ or $z=z'=0$. We
shall take it on faith that $\sim$ is in fact an equivalence relation (we
may return to this if time permits).

By the UMP of the quotient space, we need to find a continuous surjection
$f\colon S^1\times[0,1]\to B^2$ that preserves the equivalence relation
$\sim$. So consider the map $f(x,y,z)\coloneqq(zx,zy)$. This map is
continuous by Theorem 18.4 since $\pi_1\circ f(x,y,z)=zx$ is multiplication
on $\bfR$ and similarly for $\pi_2\circ f(x,y,z)$. Moreover, this map
preserves the equivalence relation: let $(x,y,z)\sim(x',y',z')$ then
$(x,y,z)=(x',y',z')$ in which case
\[
f(x,y,z)=(zx,zy)=(z'x',z'y')=f(x',y',z')
\]
or $z=z'=0$ so
\[
f(x,y,0)=(0\cdot x,0\cdot y)=(0,0)=(0\cdot x',0\cdot y')=f(x',y',0).
\]
In either case, we have $f(x,y,z)=f(x',y',z')$. Thus, by the UMP of the
quotient space, the induced map $f'\colon (S^1\times[0,1])/{\sim}\to B^2$
is continuous.

Now, since $S^1\times[0,1]$ is closed and bounded, by Heine--Borel,
$S^1\times[0,1]$ is a compact subset of $\bfR^3$. Therefore,
$(S^1\times[0,1])/{\sim}$ is compact. Since $B^2\subset\bfR^2$ is
Hausdorff, it suffices to show, by Theorem 26.6, that $f$ is bijective.

It is eassy to see that $f$ is surjective since for any point
$(x,y)\neq(0,0)$ in $B^2$, $\sqrt{x^2+y^2}\leq 1$ so letting
$z=\sqrt{x^2+y^2}$, $x'=x/\sqrt{x^2+y^2}$, and $y'=y/\sqrt{x^2+y^2}$ we
have
\[
f(x',y',z)=\sqrt{x^2+y^2}
\left(\frac{x}{\sqrt{x^2+y^2}},
\frac{y}{\sqrt{x^2+y^2}}\right)=(x,y).
\]
And, trivially, if $(x,y)=0$, we have $\varphi(x,y,0)=0$ for any $(x,y)\in
S^1$.

To see that it is injective, merely note that, by the definition of $f$,
$f(x,y,z)=f(x',y',z')$ if and only if $(x,y,z)=(x',y',z')$ or $z=z'=0$
which precisely means that $(x,y,z)\sim(x',y',z')$. Thus, $f$ is
injective.

It follows that $(S^1\times[0,1])/{\sim}\approx B^2$.
\end{proof}

\begin{problem}
Let $X$ be a set with $2$ elements $\{a,b\}$. Give $X$ the \emph{discrete}
topology. Let $Y$ be any topological space. Recall that $\calC(X,Y)$
denotes the set of continuous functions from $X$ to $Y$, with the
compact-open topology. Prove that $\calC(X,Y)$ is homeomorphic to $Y\times
Y$ (with the product topology).
\end{problem}
\begin{proof}
Consider the map $F\colon\calC(X,Y)\to Y\times Y$ given by
$F(f)\coloneqq(f(a),f(b))$. This map is continuous by Theorem 18.4, since
$\pi_1(F)$ and $\pi_2(F)$ are, respectively, the evaluation of $f$ at $a$
and the evaluation of $f$ and $b$, both of which are continuous because
under the compact-open topology. This map is clearly surjective since for
any $(y_1,y_2)\in Y\times Y$ we may define the function $f(a)\coloneqq y_1$
and $f(b)\coloneqq y_2$ which is continuous sinced $X$ has the discrete
topology. Moreover, $F$ is injective since if $(f(a),f(b))=(g(a),g(b))$
then $f(x)=g(x)$ for all $x\in X$ hence, $f=g$. Therefore, to show that $F$
is a homeomorphism, it suffices to show that $F$ is an open map.

Now it suffices to find a continuous inverse. For any $(y_1,y_2)\in Y\times
Y$, define the map $g\colon Y\times Y\to\calC(X,Y)$.
\[
g(y_1,y_2)\coloneqq f
(y)=\begin{cases}
a&\text{if $y=y_1$}\\
b&\text{if $y=y_2$}.
\end{cases}
\]

\end{proof}

\begin{problem}
Let $X$ and $Y$ be homotopy-equivalent topological spaces. Suppose that $X$
is path-connected. Prove that $Y$ is path-connected.
\end{problem}
\begin{proof}
\end{proof}

\begin{problem}
Suppose that $X$ is a wedge of two circles: that is, $X$ is a Hausdorff
space which is a union of two subspaces $A_1$ and $A_2$ such that $A_1$ and
$A_2$ are each homeomorphic to $S^1$ and $A_1\cap A_2$ is a single point $p$.

Use the Seifert--van Kampen theorem to calculate $\pi_1(X,p)$. You should
state what deformation retractions you are using, but you do not have ttto
give formulas for them.
\end{problem}
\begin{proof}
\end{proof}

\begin{problem}
Let $p\colon E\to B$ be a covering map. Let $A$ be a connected space and
let $a\in A$. Prove that if two continuous functions $\alpha,\beta\colon
A\to E$ have a property that $\alpha(a)=\beta(a)$ and
$p\circ\alpha=p\circ\beta$ then $\alpha=\beta$.

For partial credit, you may assume that $p$ is the standard covering map
from $\bfR$ to $S^1$.
\end{problem}
\begin{proof}
\end{proof}

Here's an extra problem I felt like doing since I thought it might be on
the exam:
\begin{problem*}
\begin{theorem*}[Munkres, Theorem 18.4]
Let $f\colon A\to X\times Y$ be given by the equation
$f(a)\coloneqq(f_1(a),f_2(a))$. Then $f$ is continuous if and only if
$f_1\colon A\to X$ and $f_2\colon A\to Y$ are continuous.
\end{theorem*}
\end{problem*}
\begin{proof}
Let $\pi_1\colon X\times Y\to X$ and $\pi_2\colon X\times Y\to Y$ be
projections onto the 1st and 2nd factors, respectively. These maps are
continuous and open by previous work. Now, for every $a\in A$ we have
\[
\pi_1(f(a))=f_1(a)\qquad\text{and}\qquad
\pi_2(f(a))=f_2(a).
\]
Therefore, if $f$ is continuous, then $f_1$ and $f_2$ are the composites of
the continuous functions above therefore, are continuous.

Conversely, suppose that $f_1$ and $f_2$ are continuous. By Lemma C, it
suffices to show that for each basic open set $U\times V\subset X\times Y$,
the preimage $f^{-1}(U\times V)$ is open. But $a\in f^{-1}(U\times V)$ if
and only if $f(a)\in U\times V$, if and only if $f_1(a)\in U$ and
$f_2(a)\in V$. Thus, $f^{-1}(U\times V)=f^{-1}(U)\cap f^{-1}(V)$ which is
open in $A$ since $U$ is open in $X$ and $V$ is open in $Y$ and $f_1,f_2$
are continuous.
\end{proof}

%%% Local Variables:
%%% mode: latex
%%% TeX-master: "../MA571-Quals"
%%% End:
