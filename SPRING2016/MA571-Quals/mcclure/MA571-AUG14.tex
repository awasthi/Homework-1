\chapter{Past Qualifying Examinations}
\section{MA 571: Qualifying Exam, August 2014}
\begin{problem}
Let $X$ be a topological space, let $A$ be a subset of $X$, and let $U$ be
an open subset of $X$. Prove that $U\cap \bar A\subset\overline{U\cap A}$.
\end{problem}
\begin{proof}
Let $x\in U\cap\bar A$. Then $x\in U$ and $x\in\bar A$. This means that,
since $U$ is open, by Lemma C there exist an open neighborhood $V$ of $x$
such that $V\subset U$. Moreover, since $x\in\bar A$, $V'\cap
A\neq\emptyset$ for every open neighborhood $V'$ of $x$. In particular,
$V\cap A\neq\emptyset$. Thus, we have $V\cap U\neq\emptyset$ and $V\cap
A\neq\emptyset$ so $V\cap(U\cap A)\neq\emptyset$.
\end{proof}

\begin{problem}
Let $X$ be the following subspace of $\bbR^2$:
\[
((0,1]\times[0,1])\cup([2,3)\times[0,1]).
\]
Let $\sim$ be the equivalence relation on $X$ with $(1,t)\sim(2,t)$ (that
is $(s,t)\sim(s',t')\iff(s,t)=(s',t')$ or $t=t'$ and $\{s,s'\}=\{1,2\}$;
you do \emph{not} have to prove that this is an equivalence
relation). Prove that $X/{\sim}$ is homeomorphic to
$(0,2)\times[0,1]$. (\emph{Hint}: construct maps in both directions).
\end{problem}
\begin{proof}
We shall proceed by the hint. Let $q\colon X\to X/{\sim}$ denote the
quotinet map. Then, for $(x,y)\in X$, we define the map

We shall proceed by the hint. Let $q\colon X\to X/{\sim}$ denote the
quotient map. Then, for $x\in X$, we define the map
\[
h(s,t)\coloneqq
\begin{cases}
(s,t)&\text{if $(s,t)\in(0,1]\times[0,1]$}\\
(s-1,t)&\text{if $(s,t)\in(2,3]\times[0,1])$}
\end{cases}
\]
from $X\to(0,2)\times[0,1]$.

By the UMP of the quotient space (Theorem Q.3), if we can show that $h$ is
continuous and preserves the equivalence relation, the induced map on the
quotient space, $h'\colon X/{\sim}\to (0,2)\times[0,1]$ will be
continuous. To that end, we will use the pasting lemma. First, note that
$(0,1]\times[0,1]$ and $[2,3)\times[0,1]$ are closed subsets of $X$ since
$(0,1]\times[0,1]$ is the complement of $((1,\infty)\times (-2,2))\cap X$
which is open in $X$ (since $X$ inherits its topology from $\bbR^2$),
similarly, $[2,3)\times[0,1]$ is closed in $X$ since it is the complement
of $((-\infty,2)\times(-2,2))\cap X$ which is open in $X$ for the same
reasons. It is clear that the maps $x\mapsto x$ and $x\mapsto x-1$ are
continuous onto their image, since the latter is nothing more than the
inclusion map and the former is nothing more than subtraction, which is
continuous by Theorem 21.5. Thus, by the pasting lemma, $h$ is continuous.

Now we show that $h$ does in fact preserve the equivalence
relation. Suppose $(s,t)\sim(s',t')$. Then either $(s,t)=(s',t')$ or $t=t'$
and $s,s'\in\{1,2\}$. In the former case, we have $h(s,t)=h(s',t')$
(whether $(s,t),(s',t')\in(0,1]\times[0,1]$ or its complement). In the
latter case, we may, without loss of generality, assume that $(s,t)=(1,t)$
and $(s',t')=(2,t)$. Then $h(s,t)=(1,t)=(2-1,t)=h(s',t')$. Thus, by Theorem
Q.3, the induced map $h'\colon X/{\sim}\to(0,2)\times[0,1]$ is
continuous. Moreover, the map is bijective with inverse
\[
(h')^{-1}\coloneqq
\begin{cases}
[s,t]&\text{if $x\in (0,1]$}\\
[s+1,t]&\text{if $x\in [1,2)$}
\end{cases}.
\]
This is clearly an inverse as
\[
h'\circ (h')^{-1}=\id_{X/{\sim}}
\]
and
\[
(h')^{-1}\circ h'=\id_{(0,2)\times[0,1]}.
\]
Thus, by Theorem 26.6, $h'$ is a homeomorphism.
\end{proof}

\begin{problem}
Prove that there is an equivalence relation $\sim$ on the interval $[0,1]$
such that $[0,1]/{\sim}$ is homeomorphic to $[0,1]\times[0,1]$. As part of
your proof \emph{explain} how you are using one or more properties of the
quotient topology.
\end{problem}
\begin{proof}
First, it suffices to find a continuous surjective map $f\colon[0,1]\to
[0,1]\times[0,1]$ and quotient out by the preimage of every point
$x\in[0,1]\times[0,1]$. These maps are hard to describe in general, but
they exists (take for example a space-filling curve). Next, note that if
$C$ is a closed subset of $[0,1]$ then it is compact so $f(C)$ is
compact. But since $[0,1]\times[0,1]$ is compact Hausdorff, then
$f(C)\subset[0,1]\times[0,1]$ will be closed. It follows by that $f$ will
be a Munkres quotient map, so by Theorem Q.4, $f'\colon
[0,1]/{\sim}\to[0,1]\times[0,1]$ is a homeomorphism for some equivalence
relation $\sim$ on $[0,1]$.
\end{proof}

\begin{problem}
Let $D$ be the closed unit disk in $\bbR^2$, that is, the set
\[
\left\{\,(x,y):x^2+y^2\leq 1\,\right\}.
\]
Let $E$ be the open unit disk
\[
\left\{\,(x,y):x^2+y^2<1\,\right\}.
\]
Let $X$ be the one-point compactification of $E$, and let $f\colon D\to X$
be the map defined by
\[
f(x,y)=
\begin{cases}
(x,y)&\text{if $x^2+y^2<1$}\\
\infty&\text{if $x^2+y^2=1$.}
\end{cases}
\]
Prove that $f$ is continuous.
\end{problem}
\begin{proof}
By the section on the one-point-compactification, it suffices to check two
cases of open sets (1) all sets $U$ open in $E$, and (2) all sets of the
form $U=X\minus C$ containing the point at infinity, $\infty$, where $C$ is
compact. In the first case, it is clear that $f$ is continuous since it is
just the inclusion map and is in fact bijective on $E$. For the second
case, suppose that $U$ is a neighborhood of $\infty$. Then $Y-U$ is a
compact subset of $E$, hence closed since $X$ is a compact Hausdorff
space. But since $f$ is bijective, continuous on $E$, then $f^{-1}(X-U)$ is
a closed subset of $E$. Thus, by theorem 18.2, $f$ is continuous.
\end{proof}

\begin{problem}
Let $X$ and $Y$ be homotopy-equivalent topological spaces. Suppose that $X$
is path-connected. Prove that $Y$ is path-connected.
\end{problem}
\begin{proof}
% Suppose that $X$ is homotopy-equivalent to $Y$. Then there exists a
% continuous maps $f\colon X\to Y$ and $g\colon Y\to X$ such that $g\circ
% f\simeq\id_X$ and $g\circ f\simeq\id_Y$. Now, since $X$ is path-connected,
% its image is path-connected (as we will show shortly) thus, it suffices to
% show that for any point $y\in Y$, there exists a path $p\colon I\to Y$ from
% $p(0)=y$ to $p(1)\in f(X)$. Let $y\in Y\setminus f(X)$.
First we prove the following important result:
\begin{lemma}
\label{lem:path-connected-image}
Path-connectedness is a topological property, i.e., if $X$ is
path-connected and $f\colon X\to Y$ is a continuous map then, $f(X)$ is
path connected.
\end{lemma}
\begin{proof}
\renewcommand\qedsymbol{$\clubsuit$}
Since $X$ is path-connected, for any pair of points $x,x'\in X$ there
exists a continuous map $p\colon [0,1]\to X$ such that $p(0)=x$ and
$p(1)=x'$. Since composition of continuous maps is continuous, $f\circ
p\colon[0,1]\to Y$ is a path from $f(x)$ to $f(x')$. Since this property
holds for any $y\in f(X)$, it follows that $f(X)$ is path-connected.
\end{proof}
Now, suppose that $X$ is homotopy-equivalent to $Y$. Then there exists
continuous maps $f\colon X\to Y$ and $g\colon Y\to X$ such that $g\circ
f\simeq\id_X$ and $f\circ g\simeq\id_Y$. Now, since $X$ is path-connected,
by Lemma (\ref{lem:path-connected-image}) we have $f(X)$ is
path-connected. Thus, it suffices to show that for every point $y\in Y$
there exists a path $p\colon [0,1]\to Y$ from $y$ to some point $y'\in
f(X)$. Now, since $f\circ g\simeq\id_Y$, there exists a homotopy, say
$H\colon Y\times[0,1]\to Y$ such that $H(s,0)=f\circ g(s)$ and
$H(s,1)=s$. Consider the evaluation $H_y\coloneqq H(y,t)\circ H(y,t)$ where
the map $(y,t)\colon [0,1]\to Y\times[0,1]$ is the imbedding of $[0,1]$ at
$y$ (which is continuous by Theorem 18.4) thus, $H_y$ is
continuous. Moreover, $H_y(0)=f\circ g(y)\in f(Y)$ and
$H_y(1)=\id_Y(y)=y$ so $H_y$ is a path from $y$ to a point $f\circ g(y)$ in
$f(X)$. Since we can do this for any point $y\in Y$, it follows, since
path-connectedness is an equivalence relation, that $Y$ is path-connected.
\end{proof}
\begin{problem}
Let $a$ and $b$ denote the points $(-1,0)$ and $(1,0)$ in $\bbR^2$. Let
$x_0$ denote the origin $(0,0)$. Use the Seifert--van Kampen theorem to
calculate $\pi_1\left(\bbR^2-\{a,b\},x_0\right)$. You may not use any other
method.
\end{problem}
\begin{proof}
We'll use Theorem 70.2's version of the Seifert--van Kampen theorem. Define
\[
U\coloneqq\left(-\infty,\tfrac{1}{2}\right)\times\bbR
\quad\text{and}\quad
V\coloneqq\left(-\tfrac{1}{2},\infty\right)\times\bbR.
\]
Then $U\cap V=(-1/2,1/2)\times\bbR$ is clearly path-connected since it is a
convex set. Moreover, note that $U\simeq\bbR^2\minus\{x_0\}$ and
$V\simeq\bbR^2\minus\{x_0\}$ (in the case of $U$, first consider the
homeomorphism $(x,y)\mapsto(x+1,y)$ which sends $a$ to $(0,0)$ and then the
homotopy $(x,y)\mapsto\tfrac{1}{t}(x,ys)$).

Once we have established the above, since the fundamental group of a space
is invariant under homotopy-equivalence,
$\pi_1(U,x_0)\cong\pi_1\left(\bbR^2\minus\{x_0\},y_0\right)\cong\bbZ$ for some
arbitrary $y_0\neq x_0$ and similarly $\pi_1(V,x_0)\cong\bbZ$. Thus, by the
classical version of the Seifert--van Kampen theorem
\[
\pi_1\left(\bbR^2\minus\{a,b\},x_0\right)\cong
\frac{\bbZ*\bbZ}{N}
\]
where $N$ is the least normal subgroup
\end{proof}

\begin{problem}
Let $p\colon E\to B$ be a covering map with $B$ locally connected, and let
$x\in B$. Prove that $x$ has a neighborhood $W$ with the following
property: for every connected component $C$ of $p^{-1}(W)$, the map
$p\colon C\to W$ is a homeomorphism.
\end{problem}
\begin{proof}
Let $U$ be an evenly covered neighborhood of $x$. Then
$p^{-1}(U)=\bigcup_\alpha V_\alpha$ where the $V_\alpha$ are open in $E$
and  $V_\alpha\cap V_\beta=\emptyset$ whenever $\alpha\neq\beta$. For any
$\alpha$, let $C$ be a connected component of $p^{-1}(U)$ containing
$p^{-1}(x)\cap V_\alpha$ (the latter is a one point set since
$\left.p\right|_{V_\alpha}$ is a bijection). Then $C\subset V_\alpha$ for
at most one such $\alpha$ for otherwise $C\cap V_\beta\neq\emptyset$ for
some $\beta\neq\alpha$, so $C\cap V_\beta$ and $C\cap V_\alpha$ form a
separation of (note that $C\minus(C\cap V_\beta)=C\cap V_\alpha$ and
vice-versa thus, $C\cap V_\beta$ and $C\cap V_\alpha$ are open and closed
in the subspace topology on $C$, conversely) by Lemma 23.1.

Thus, $p(C)\subset U$ is connected by Theorem 23.5. Moreover, since
$V_\alpha\supset C$ is homeomorphic to $U$ by the restriction
$\left.p\right|_{V_\alpha}$, $p(C)$ is a connected component of $U$ as the
following lemma shows
\begin{lemma}
Suppose $C$ is a connected component of $X$ and $h\colon X\to Y$ is a
homeomorphism. Then $h(C)$ is a connected component of $Y$.
\end{lemma}
\begin{proof}[Proof of lemma]
\renewcommand\qedsymbol{$\clubsuit$}
Let $C$ be a connected component of $X$. By theorem 23.5, $h(C)$ is a
connected subset of $Y$, moreover, is open. By Theorem 25.1, $h(C)$ is
contained in a connected component of $Y$, say $D$. Hence, we must show
that $D\subset h(C)$. Now, since $h$ is a homeomorphism, $h^{-1}(D)$ is a
connected subset of $X$, by Theorem 23.5, so is contained in only one
component of $X$. But $h^{-1}(D)\cap C\neq\emptyset$ so $h^{-1}(D)\subset
C$. Thus, since $h$ is a set-bijection, $D\subset h(C)$.
\end{proof}
so by Theorem 25.3, $p(C)$ is open in $B$ since $B$ is locally
connected. Thus, the restriction $\left.p\right|_{C}$ is a homeomorphism
onto its image $W\coloneqq p(C)$, by Lemma A, which is a neighborhood of $x$.
\end{proof}

%%% Local Variables:
%%% mode: latex
%%% TeX-master: "../MA571-Quals"
%%% End:
