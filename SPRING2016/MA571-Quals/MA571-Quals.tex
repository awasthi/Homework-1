\def\documentauthor{Carlos Salinas}
\def\documenttitle{MA571: Qual Problems}
% \def\hwnum{1}
\def\shorttitle{MA571 Quals}
\def\coursename{MA571}
\def\documentsubject{point-set topology}
\def\authoremail{salinac@purdue.edu}

\documentclass[article,oneside,10pt]{memoir}
\usepackage{geometry}
\usepackage[dvipsnames]{xcolor}
\usepackage[
    breaklinks,
    bookmarks=true,
    colorlinks=true,
    pageanchor=false,
    linkcolor=black,
    anchorcolor=black,
    citecolor=black,
    filecolor=black,
    menucolor=black,
    runcolor=black,
    urlcolor=black,
    hyperindex=false,
    hyperfootnotes=true,
    pdftitle={\shorttitle},
    pdfauthor={\documentauthor},
    pdfkeywords={\documentsubject},
    pdfsubject={\coursename}
    ]{hyperref}

% Use symbols instead of numbers
\renewcommand*{\thefootnote}{\fnsymbol{footnote}}

%% Math
\usepackage{amsthm}
\usepackage{amssymb}
\usepackage{mathtools}
% \usepackage{unicode-math}

%% PDFTeX specific
\usepackage[mathcal]{euscript}
\usepackage{mathrsfs}

\usepackage[LAE,LFE,T2A,T1]{fontenc}
\usepackage[utf8]{inputenc}
\usepackage[farsi,french,german,spanish,russian,english]{babel}
\babeltags{fr=french,
           de=german,
           en=english,
           es=spanish,
           pa=farsi,
           ru=russian
           }
\def\spanishoptions{mexico}

\selectlanguage{english}

\newcommand{\textfa}[1]{\beginR\textpa{#1}\endR}

\usepackage{cmap}
\usepackage{CJKutf8}
\newcommand{\textkr}[1]{\begin{CJK}{UTF8}{mj}#1\end{CJK}}
\newcommand{\textjp}[1]{\begin{CJK}{UTF8}{min}#1\end{CJK}}
\newcommand{\textzh}[1]{\begin{CJK}{UTF8}{bsmi}#1\end{CJK}}

\usepackage{graphicx}
\graphicspath{{figures/}}

% Misc
\usepackage{microtype}
\usepackage{multicol}
\usepackage[inline]{enumitem}
\usepackage{listings}
\usepackage{mleftright}
\mleftright

%% Theorems and definitions
%% remove parentheses
% \makeatletter
% \def\thmhead@plain#1#2#3{%
%   \thmname{#1}\thmnumber{\@ifnotempty{#1}{ }\@upn{#2}}%
%   \thmnote{ {\the\thm@notefont#3}}}
% \let\thmhead\thmhead@plain
% \makeatother

\theoremstyle{plain}
\newtheorem{theorem}{Theorem}
\newtheorem{proposition}[theorem]{Proposition}
\newtheorem{corollary}[theorem]{Corollary}
\newtheorem{claim}[theorem]{Claim}
\newtheorem{lemma}[theorem]{Lemma}
\newtheorem{axiom}[theorem]{Axiom}

\newtheorem*{corollary*}{Corollary}
\newtheorem*{claim*}{Claim}
\newtheorem*{lemma*}{Lemma}
\newtheorem*{proposition*}{Proposition}
\newtheorem*{theorem*}{Theorem}

\theoremstyle{definition}
\newtheorem{definition}{Definition}
\newtheorem{example}{Examples}
\newtheorem{examples}[example]{Examples}
\newtheorem{exercise}{Exercise}[chapter]
\newtheorem{problem}[exercise]{Problem}

\newtheorem*{example*}{Example}
\newtheorem*{exercise*}{Exercise}
\newtheorem*{problem*}{Problem}

%% Redefinitions & commands
\newcommand{\nsubset}{\ensuremath{\not\subset}}
\newcommand{\nsupset}{\ensuremath{\not\supset}}
\newcommand\minus{\ensuremath{\null\smallsetminus}}
\renewcommand\qedsymbol{\ensuremath{\null\hfill\blacksquare}}

%% Commands and operators
\DeclareMathOperator{\id}{id}
\DeclareMathOperator{\im}{im}

\DeclareMathOperator{\Id}{Id}
\DeclareMathOperator{\Img}{Im}
\DeclareMathOperator{\Int}{Int}
\DeclareMathOperator{\Cl}{Cl}
\DeclareMathOperator{\Fix}{Fix}
\DeclareMathOperator{\Filter}{Filter}
\DeclareMathOperator{\Ker}{Ker}
\DeclareMathOperator{\Coker}{Coker}
\DeclareMathOperator{\Tr}{Tr}
\DeclareMathOperator{\Det}{Det}

\newcommand{\bbC}{\mathbb{C}}
\newcommand{\bbN}{\mathbb{N}}
\newcommand{\bbQ}{\mathbb{Q}}
\newcommand{\bbR}{\mathbb{R}}
\newcommand{\bbZ}{\mathbb{Z}}
\newcommand{\bfC}{\mathbf{C}}
\newcommand{\bfN}{\mathbf{N}}
\newcommand{\bfQ}{\mathbf{Q}}
\newcommand{\bfR}{\mathbf{R}}
\newcommand{\bfZ}{\mathbf{Z}}

\newcommand{\calA}{\mathcal{A}}
\newcommand{\calB}{\mathcal{B}}
\newcommand{\calC}{\mathcal{C}}
\newcommand{\calU}{\mathcal{U}}
\newcommand{\calV}{\mathcal{U}}

\newcommand{\scrL}{\mathscr{L}}
\newcommand{\scrO}{\mathscr{O}}
\newcommand{\scrS}{\mathscr{S}}

\begin{document}
\author{\href{mailto:\authoremail}{\documentauthor}}
\title{\documenttitle}
\date{\today}
\maketitle
\chapter{MA571 (Midterm 2015)}
\begin{problem}
Prove that a function to a product space is continuous if and only if its
components are.
\end{problem}
\begin{proof}
\end{proof}

\begin{problem}
Prove that a subspace is closed if and only if it contains all of its limit
points.
\end{problem}
\begin{proof}
\end{proof}

\begin{problem}
Prove that the projection maps for a product are open maps.
\end{problem}
\begin{proof}
\end{proof}

\begin{problem}
Prove that $\partial A=\emptyset$ if and only if $A$ is open and closed.
\end{problem}
\begin{proof}
\end{proof}

\begin{problem}
Prove that a metric space satisfies the 1st countability axiom.
\end{problem}
\begin{proof}
\end{proof}

\begin{problem}
Prove that $\bfR^\omega$ is not metrizable in the box topology.
\end{problem}
\begin{proof}
\end{proof}

\begin{problem}
Show that the diagonal map is not continuous in the box topology, but it is
in the product topology.
\end{problem}
\begin{proof}
\end{proof}

\begin{problem}
Prove the sequence lemma.
\end{problem}
\begin{proof}
\end{proof}

\begin{problem}
Give an example of a surjective map of spaces that is not a quotient map.
\end{problem}
\begin{proof}
\end{proof}

\begin{problem}
Prove that if $f_n$ is a sequence of functions $X\to\bfR$ considered as
elements of $X^{\bfR}$ with the product topology, then $f_n\to f$ if and
only if for each $x\in X$ the sequence $f_n(x)$ converges to the point
$f_n(x)$.
\end{problem}
\begin{proof}
\end{proof}

\begin{problem}
Prove that if $f_n$ is a sequence of functions $X\to\bfR$ considered as
elements of $X^{\bfR}$ with the topology induced by the uniform metric
$\bar\rho$, then $f_n\to f$ if and only if the sequence of functions
$f_n$ converges uniformly to the point $f$. (Recall that $f_n\colon X\to
Y$, with $Y$ a metric space, uniformly converges to $f$ if for any
$\varepsilon>0$ there exists an integer $N$ such that for all $n>N$ and
$x\in D$, $d_y(f_n(x),f(x))<\varepsilon$.)
\end{problem}
\begin{proof}
\end{proof}

\begin{problem}
Give an example of a surjective map of spaces that is not a quotient map.
\end{problem}
\begin{proof}
\end{proof}

\begin{problem}
\end{problem}
\begin{proof}
\end{proof}

\begin{problem}
\end{problem}
\begin{proof}
\end{proof}

\begin{problem}
\end{problem}
\begin{proof}
\end{proof}

\begin{problem}
\end{problem}
\begin{proof}
\end{proof}

\begin{problem}
\end{problem}
\begin{proof}
\end{proof}

\begin{problem}
\end{problem}
\begin{proof}
\end{proof}

\begin{problem}
\end{problem}
\begin{proof}
\end{proof}

\begin{problem}
\end{problem}
\begin{proof}
\end{proof}

\begin{problem}
\end{problem}
\begin{proof}
\end{proof}

\begin{problem}
\end{problem}
\begin{proof}
\end{proof}


%%% Local Variables:
%%% mode: latex
%%% TeX-master: "../MA571-Quals"
%%% End:

\chapter{MA571 (Final 2015)}
\begin{problem}
\end{problem}
\begin{proof}
\end{proof}

\begin{problem}
\end{problem}
\begin{proof}
\end{proof}

\begin{problem}
\end{problem}
\begin{proof}
\end{proof}

\begin{problem}
\end{problem}
\begin{proof}
\end{proof}

\begin{problem}
\end{problem}
\begin{proof}
\end{proof}

\begin{problem}
\end{problem}
\begin{proof}
\end{proof}

\begin{problem}
\end{problem}
\begin{proof}
\end{proof}

\begin{problem}
\end{problem}
\begin{proof}
\end{proof}

\begin{problem}
\end{problem}
\begin{proof}
\end{proof}

\begin{problem}
\end{problem}
\begin{proof}
\end{proof}

\begin{problem}
\end{problem}
\begin{proof}
\end{proof}

\begin{problem}
\end{problem}
\begin{proof}
\end{proof}


%%% Local Variables:
%%% mode: latex
%%% TeX-master: "../MA571-Quals"
%%% End:

\include{mcclure/MA571-JAN16}
\section{August, 2014}

%%% Local Variables:
%%% mode: latex
%%% TeX-master: "../MA571-Quals"
%%% End:

\section{MA 571: Qualifying Exam, January 2014}
\begin{problem}
Let $X$ be a topological space, let $A$ be a subset of $X$, and let $U$ be
an open subset of $X$. Prove that $U\cap\bar A\subset\overline{U\cap A}$.
\end{problem}
\begin{proof}
The proof is simple and we have shown this before in the August 2014
quals, it goes as follows: If $U\cap\bar A=\emptyset$, there is nothing to
show. Let $x\in U\cap\bar A$. Then $x\in U$ and $x\in\bar A$. Since $x\in
U$ and $U$ is open, by Lemma C, there exists a neighborhood $V$ of $x$ such
that $V\subset U$; in particular, note that $V\cap U\neq\emptyset$. But
$x\in\bar A$ so $V\cap A\neq\emptyset$. Thus, $V\cap(U\cap
A)\neq\emptyset$. Thus, $x\in\overline{U\cap A}$.
\end{proof}
\begin{problem}
Let $\sim$ be an equivalence relation on $\bbR^2$ defined by
$(x,y)\sim(x',y')$ if and only if there is a nonzero $t$ with
$(x,y)=(tx',ty')$. Prove that the quotient space $\bbR^2/{\sim}$ is compact
but not Hausdorff.
\end{problem}
\begin{proof}
To show that $\bbR^2/{\sim}$ is compact, we need to show that for every
open covering $\calA$ of $\bbR^2/{\sim}$, there is a finite subcover
$\calA'\subset\calA$. Let $q\colon\bbR^2\to\bbR^2/{\sim}$ denote the
quotient map. Then, since $q$ is continuous and onto $\bbR^2/{\sim}$, the
set $\left\{q^{-1}(A_\alpha)\right\}_{A_\alpha\in\calA}$ is an open cover
of $\bbR^2$. In particular, there exists at least one $A_\alpha$ such that
$q^{-1}(A_\alpha)$ is a neighborhood of $(0,0)$. By Lemma C, there exists a
basic open neighborhood, i.e., an open ball $B((0,0),\varepsilon)\subset
q^{-1}(A_\alpha)$ for $\varepsilon>0$. Now, for any point
$[(x,y)]\in\bbR^2$ pick a representative $(x,y)\in\bbR^2$. Then, by the
Archimedean principle, there exists a positive real numbers $t',t''>0$ such
that $t'x<\sqrt{\varepsilon}$ and $t''y<\sqrt{\varepsilon}$. Define
$t\coloneqq\min\{t',t''\}$. Then $tx<\sqrt{\varepsilon}$ and
$ty<\varepsilon$. Thus, $(tx,ty)\in A_\alpha$ (since
$t^2x^2+t^2y^2<\varepsilon$). Since we can do this for any point
$[x]\in\bbR^2/{\sim}$, it follows that
$A_\alpha\supset\bbR^2/{\sim}$. Thus,
$\calA'\coloneqq\left\{A_\alpha\right\}$ is a finite subset of $\calA$
which covers $\bbR^2/{\sim}$. Thus, $\bbR^2/{\sim}$ is compact.

To show that $\bbR^2/{\sim}$ is not compact, we will employ a very similar
strategy, that is, we will show that every neighborhood of the point
$[0,0]\in\bbR^2/{\sim}$, contains every point $[x,y]\in\bbR^2/{\sim}$. Let
$[x,y]\in\bbR^2/{\sim}$ and let $U$ be a neighborhood of $[0,0]$. Then
$q^{-1}(U)$ is an open neighborhood of $(0,0)$, i.e., there exists an open
ball $B((0,0),\varepsilon)\subset q^{-1}(U)$. But as we have just shown,
for sufficiently small values of $t>0$, $(tx,ty)\in
B((0,0),\varepsilon)\subset q^{-1}(U)$. Thus, $[x,y]\in U$. In particular,
for any open neighborhood $V$ of $[x,y]$, $V\cap U\neq\emptyset$. Thus,
$\bbR^2/{\sim}$ is not Hausdorff.
\end{proof}
\begin{problem}
Let $X$ and $Y$ be topological spaces. Let $x_0\in X$ and let $C$ be a
compact subset of $Y$. Let $N$ be an open set in $X\times Y$ containing
$\left\{x_0\right\}\times C$. Prove that there is an open set $U$
containing $x_0$ and an open set $V$ containing $C$ such that $U\times
V\subset N$.
\end{problem}
\begin{proof}
This is a classical theorem called the tube lemma. We shall prove first in
the style of Munkres and second in the style of McClure (if I can find the
proof or somehow reconstruct it).

Let $X$, $Y$, $x_0$, $N$, and $C$ be as above. Note that since $C$ is
compact and the injection $\iota_{x_0}\colon X\hookrightarrow X\times Y$
given by $\iota_{x_0}(y)\coloneqq(x_0,y)$ is continuous by Theorem 18.4 (since
its components, i.e., projections to $X$ and $Y$, are continuous these are
$\pi_1(\iota_{x_0})(x)=x_0$ and $\pi_1(\iota_{x_0})(y)=y$ a constant map
and identity map, respectively) so the image of $C$ under $\iota_{x_0}$,
$\{x_0\}\times C$, is compact by Theorem 23.5. For every point
$x\in\left\{x_0\right\}\times C$, let $U_x\times V_x$ be a basic open
neighborhood of $x$ contained in $N$ (this can be arranged by Lemma
C). Then the collection $\calA\coloneqq\left\{U_x\times
  V_x\right\}_{x\in\left\{x_0\right\}\times   Y}$ forms an open covering of
$\left\{x_0\right\}\times C$. Thus, there exists a finite subcover
$\left\{U_{x_i}\times V_{x_i}\right\}_{i=1}^n$ of $\calA$.

Define $W\coloneqq U_{x_1}\cap\cdots\cap U_{x_n}$. This set is clearly open
since it is a finite intersection of open sets and contains $x_0$ since
every $U_{x_i}\times V_{x_i}$ intersects $\left\{x_0\right\}\times
Y$. Define $W'\coloneqq\pi_2(N)\cap Y$. This set is open since it is a
finite intersection of open sets in $Y$. The $W\times W'\subset N$. This is
clear since every point $(x,y)\in W\times W'$ is in $N$ ($x\in
W\subset U_{x_i}$ for all $i$ which in turn is a subset of $\pi_1(N)$ and
$y\in W'=\pi_2(N)$). Lastly, $W\times W'\supset \{x_0\}\times C$ since
$x_0\in W$ and $W'=\pi_1(N)\supset C$. Thus, $W\times W'\subset N$
containing $\left\{x_0\right\}\times C$ as desired.
\end{proof}
\begin{problem}
Let $X$ be a locally compact Hausdorff space and let $A$ be a subset with
the property that $A\cap K$ is closed for every compact $K$. Prove that $A$
is closed.
\end{problem}
\begin{proof}
Here's what I have so far:

We will try to show that $\bar A\subset A$. Let $x\in\bar A$. Then, for
every neighborhood $U$ of $x$, $U\cap A\neq\emptyset$. Now, since $X$ is
locally compact, there exists a neighborhood $V$ of $x$ such that $\bar V$
is compact and is a subset of $U$. Since $X$ is Hausdorff, $\bar V$ is
compact so $\bar V\cap A$ is closed.
\end{proof}
\begin{problem}
Let $X$ and $Y$ be path-connected and let $h\colon X\to Y$ be a continuous
function which induces the trivial homomorphism of fundamental groups. Let
$x_0,x_1\in X$ and let $f$ and $g$ be paths from $x_0$ to $x_1$. Prove that
$h\circ f$ and $h\circ g$ are homotopic.
\end{problem}
\begin{proof}
Consider the path-product $\gamma\coloneqq f*\bar g$. $\gamma$ is a loop
based at $x_0$ since $\gamma(0)=f(0)=x_0$ and $\gamma(1)=\bar g(2-1)=\bar
g(1)=x_0$. Thus, $[\gamma]\in\pi_1(X,x_0)$. Now, since
$h_*\colon\pi_1(X,x_0)\to\pi_1(Y,h(x_0))$ induces the trivial homomorphism,
i.e., $h(\gamma)\simeq_p e_{x_0}$, there exists a homotopy $H\colon
[0,1]\times[0,1]\to Y$ such that $H(s,0)=h\circ\gamma(s)$ and
$H(s,1)=e_{x_0}(s)$. Now, since $Y$ is path-connected, there exists a path
$\delta\colon[0,1]\to Y$ from $h(x_0)$ to $h(x_1)$.
\end{proof}
\begin{problem}
Let $X$ be the quotient space obtained from an $8$-sided polygonal region
$P$ by pasting its edges together according to the labelling scheme
$aabbcdc^{-1}d^{-1}$.
\begin{enumerate}[noitemsep,label=(\roman*)]
\item Calculate $H_1(X)$.
\item Assuming $X$ is homeomorphic to one of the standard surfaces in the
  classification theorem, which surface is it?
\end{enumerate}
\end{problem}
\begin{proof}
\end{proof}
\begin{problem}
Let $p\colon E\to B$ be a covering map with $B$ locally connected, and let
$x\in B$. Prove that $x$ has a neighborhood $W$ with the following
property: for every connected component $C$ of $p^{-1}(W)$, the map
$p\colon C\to W$ is a homeomorphism.
\end{problem}
\begin{proof}
\end{proof}

%%% Local Variables:
%%% mode: latex
%%% TeX-master: "../MA571-Quals"
%%% End:

\section{MA571: Qualifying Exam, January 2012}
\begin{problem}
Let $X$ be a topological space. Recall that a subset of $X$ is \emph{dense}
if its closure is $X$. Prove that the intersection of two dense open sets is
dense.
\end{problem}
\begin{proof}
Suppose $U$ and $V$ are open dense subsets of $X$. We will show that $U\cap
V$ is dense in $X$, i.e., $\overline{U\cap V}=X$. To that end, we will show
that for any point $x\in X$, for any neighborhood $W$ of $x$, $W\cap(U\cap
V)\neq\emptyset$. Therefore, let $x\in X$. Let $W$ be a neighborhood of
$x$. Then, since $U$ is dense in $X$, $W\cap U\neq\emptyset$. Let $y\in
W\cap U$. Then, since $U$ and $V$ are open, $U\cap V$ is open so $U\cap V$
is a neighborhood of $y$. Moreover, since $V$ is dense in $X$, $(W\cap
U)\cap V\neq\emptyset$. Now, since intersection is associative, $(W\cap
U)\cap V=W\cap(U\cap V)\neq\emptyset$. Thus, $x\in\overline{U\cap V}$ and
we have $\overline{U\cap V}=X$ as desired.
\end{proof}

\begin{problem}
Let $X$ be a set with two elements $\{a,b\}$. Give $X$ the
\emph{indiscrete} topology. Give $X\times\bbR$ the product topology. Let
$A\subset X\times\bbR$ be $(\{a\}\times[0,1])\cup(\{b\}\times(0,1))$. Prove
that $A$ is compact.

You may use the fact that a set is compact if every covering by
\emph{basic} open sets has a finite subcovering.
\end{problem}
\begin{proof}
Let $\calU$ be an open cover of $A$ by basic open sets. Then each
$U\in\calU$ is of the form $\{a,b\}\times V$ where $V$ is an open subset of
$\bbR$. Then, the $V$'s, i.e., $\pi_2(U)$ where $\pi_2\colon
X\times\bbR\to\bbR$ is an open map by previous work, form open cover of
$[0,1]$ (since $\bigcup_{U\in\calU} U\supset A$, we must have
$\bigcup_{U\in\calU}\pi_2(U)\supset[0,1]$). Now, since $[0,1]$ is compact
in $X$ there is a finite collection of the $V$'s, say
$\left\{V_1,\dotsc,V_n\right\}$, that cover $[0,1]$. Call $U_i$ the element of
$\calU$ such that $\pi_2(U_i)=V_i$. Then the $U_i$'s form a finite subcover
of $A$. Thus, $A$ is compact.
\end{proof}

\begin{problem}
Let $B^2$ be the disk
\[
\left\{\,(x,y)\in\bbR^2:x^2+y^2\leq 1\,\right\}.
\]
Let $S^1$ be the circle
\[
\left\{\,(x,y)\in\bbR^2:x^2+y^2=1\,\right\}.
\]
Prove that there is an equivalence relation $\sim$ such that $B^2$ is
homeomorphic to $(S^1\times[0,1])/{\sim}$. As port of your proof explain
how you are using one or more properties of the quotient topology.
\end{problem}
\begin{proof}
Such an equivalence relation is called the cone of $S^1$. We define it as
follows, let $(x,y,z),(x',y',z')\in S^1\times[0,1]$ then we say
$(x,y,z)\sim(x',y',z')$ if and only if $(x,y)=(x',y')$ or $z=z'=0$. We
shall take it on faith that $\sim$ is in fact an equivalence relation (we
may return to this if time permits).

By the UMP of the quotient space, we need to find a continuous surjection
$f\colon S^1\times[0,1]\to B^2$ that preserves the equivalence relation
$\sim$. So consider the map $f(x,y,z)\coloneqq(zx,zy)$. This map is
continuous by Theorem 18.4 since $\pi_1\circ f(x,y,z)=zx$ is multiplication
on $\bbR$ and similarly for $\pi_2\circ f(x,y,z)$. Moreover, this map
preserves the equivalence relation: let $(x,y,z)\sim(x',y',z')$ then
$(x,y,z)=(x',y',z')$ in which case
\[
f(x,y,z)=(zx,zy)=(z'x',z'y')=f(x',y',z')
\]
or $z=z'=0$ so
\[
f(x,y,0)=(0\cdot x,0\cdot y)=(0,0)=(0\cdot x',0\cdot y')=f(x',y',0).
\]
In either case, we have $f(x,y,z)=f(x',y',z')$. Thus, by the UMP of the
quotient space, the induced map $f'\colon (S^1\times[0,1])/{\sim}\to B^2$
is continuous.

Now, since $S^1\times[0,1]$ is closed and bounded, by Heine--Borel,
$S^1\times[0,1]$ is a compact subset of $\bbR^3$. Therefore,
$(S^1\times[0,1])/{\sim}$ is compact. Since $B^2\subset\bbR^2$ is
Hausdorff, it suffices to show, by Theorem 26.6, that $f$ is bijective.

It is eassy to see that $f$ is surjective since for any point
$(x,y)\neq(0,0)$ in $B^2$, $\sqrt{x^2+y^2}\leq 1$ so letting
$z=\sqrt{x^2+y^2}$, $x'=x/\sqrt{x^2+y^2}$, and $y'=y/\sqrt{x^2+y^2}$ we
have
\[
f(x',y',z)=\sqrt{x^2+y^2}
\left(\frac{x}{\sqrt{x^2+y^2}},
\frac{y}{\sqrt{x^2+y^2}}\right)=(x,y).
\]
And, trivially, if $(x,y)=0$, we have $\varphi(x,y,0)=0$ for any $(x,y)\in
S^1$.

To see that it is injective, merely note that, by the definition of $f$,
$f(x,y,z)=f(x',y',z')$ if and only if $(x,y,z)=(x',y',z')$ or $z=z'=0$
which precisely means that $(x,y,z)\sim(x',y',z')$. Thus, $f$ is
injective.

It follows that $(S^1\times[0,1])/{\sim}\approx B^2$.
\end{proof}

\begin{problem}
Let $X$ be a set with $2$ elements $\{a,b\}$. Give $X$ the \emph{discrete}
topology. Let $Y$ be any topological space. Recall that $\calC(X,Y)$
denotes the set of continuous functions from $X$ to $Y$, with the
compact-open topology. Prove that $\calC(X,Y)$ is homeomorphic to $Y\times
Y$ (with the product topology).
\end{problem}
\begin{proof}
Consider the map $F\colon\calC(X,Y)\to Y\times Y$ given by
$F(f)\coloneqq(f(a),f(b))$. This map is continuous by Theorem 18.4, since
$\pi_1(F)$ and $\pi_2(F)$ are, respectively, the evaluation of $f$ at $a$
and the evaluation of $f$ and $b$, both of which are continuous because
under the compact-open topology. This map is clearly surjective since for
any $(y_1,y_2)\in Y\times Y$ we may define the function $f(a)\coloneqq y_1$
and $f(b)\coloneqq y_2$ which is continuous sinced $X$ has the discrete
topology. Moreover, $F$ is injective since if $(f(a),f(b))=(g(a),g(b))$
then $f(x)=g(x)$ for all $x\in X$ hence, $f=g$. Therefore, to show that $F$
is a homeomorphism, it suffices to show that $F$ is an open map.

Now it suffices to find a continuous inverse. For any $(y_1,y_2)\in Y\times
Y$, define the map $g\colon Y\times Y\to\calC(X,Y)$.
\[
g(y_1,y_2)\coloneqq f
(y)=\begin{cases}
a&\text{if $y=y_1$}\\
b&\text{if $y=y_2$}.
\end{cases}
\]

\end{proof}

\begin{problem}
Let $X$ and $Y$ be homotopy-equivalent topological spaces. Suppose that $X$
is path-connected. Prove that $Y$ is path-connected.
\end{problem}
\begin{proof}
\end{proof}

\begin{problem}
Suppose that $X$ is a wedge of two circles: that is, $X$ is a Hausdorff
space which is a union of two subspaces $A_1$ and $A_2$ such that $A_1$ and
$A_2$ are each homeomorphic to $S^1$ and $A_1\cap A_2$ is a single point $p$.

Use the Seifert--van Kampen theorem to calculate $\pi_1(X,p)$. You should
state what deformation retractions you are using, but you do not have ttto
give formulas for them.
\end{problem}
\begin{proof}
\end{proof}

\begin{problem}
Let $p\colon E\to B$ be a covering map. Let $A$ be a connected space and
let $a\in A$. Prove that if two continuous functions $\alpha,\beta\colon
A\to E$ have a property that $\alpha(a)=\beta(a)$ and
$p\circ\alpha=p\circ\beta$ then $\alpha=\beta$.

For partial credit, you may assume that $p$ is the standard covering map
from $\bbR$ to $S^1$.
\end{problem}
\begin{proof}
\end{proof}

Here's an extra problem I felt like doing since I thought it might be on
the exam:
\begin{problem*}
\begin{theorem*}[Munkres, Theorem 18.4]
Let $f\colon A\to X\times Y$ be given by the equation
$f(a)\coloneqq(f_1(a),f_2(a))$. Then $f$ is continuous if and only if
$f_1\colon A\to X$ and $f_2\colon A\to Y$ are continuous.
\end{theorem*}
\end{problem*}
\begin{proof}
Let $\pi_1\colon X\times Y\to X$ and $\pi_2\colon X\times Y\to Y$ be
projections onto the 1st and 2nd factors, respectively. These maps are
continuous and open by previous work. Now, for every $a\in A$ we have
\[
\pi_1(f(a))=f_1(a)\qquad\text{and}\qquad
\pi_2(f(a))=f_2(a).
\]
Therefore, if $f$ is continuous, then $f_1$ and $f_2$ are the composites of
the continuous functions above therefore, are continuous.

Conversely, suppose that $f_1$ and $f_2$ are continuous. By Lemma C, it
suffices to show that for each basic open set $U\times V\subset X\times Y$,
the preimage $f^{-1}(U\times V)$ is open. But $a\in f^{-1}(U\times V)$ if
and only if $f(a)\in U\times V$, if and only if $f_1(a)\in U$ and
$f_2(a)\in V$. Thus, $f^{-1}(U\times V)=f^{-1}(U)\cap f^{-1}(V)$ which is
open in $A$ since $U$ is open in $X$ and $V$ is open in $Y$ and $f_1,f_2$
are continuous.
\end{proof}

%%% Local Variables:
%%% mode: latex
%%% TeX-master: "../MA571-Quals"
%%% End:

\chapter{MA 571: Qualifying Exam, January 2011}
\begin{problem}
Let $A$ be a subset of a topological space $X$ and let $B$ be a subset of
$A$. Prove that $\bar A\minus\bar B\subset\overline{A\minus B}$.
\end{problem}
\begin{proof}
\end{proof}

\begin{problem}
Let $G$ be a topological group (thas is, a group with a topology for which
the group operations are continuous) and let $H$ be a subgroup of
$G$. Suppose that $G$ is connected, that $H$ is a normal subgroup of $G$,
and that the subspace topology on $H$ is discrete. Prove that $gh=hg$ for
every $g\in G$, $h\in G$.
\end{problem}
\begin{proof}
\end{proof}

\begin{problem}
Let $X$ be the space with two points and the discrete topology. Let
$Y=\prod_{n=1}^\infty X$, with the product topology. What are the connected
components of $Y$? Prove that your answer is correct.
\end{problem}
\begin{proof}
\end{proof}

\begin{problem}
Let $X$ and $Y$ be topological spaces. Let $x_0\in X$ and let $C$ be a
compact subset of $Y$. Let $N$ be an open set in $X\times Y$ containing
$\left\{x_0\right\}\times C$. Prove that there is an open set $U$
containing $x_0$ and an open set $V$ containing $C$ such that $U\times
V\subset N$.
\end{problem}
\begin{proof}
\end{proof}

\begin{problem}
Let $X$ and $Y$ be homotopy-equivalent topological spaces. Suppose that $X$
is connected. Prove that $Y$ is connected.
\end{problem}
\begin{proof}
\end{proof}

\begin{problem}
Let $p\colon E\to B$ be a covering map. Let $e_0\in E$ and $b_0\in B$ with
$p(e_0)=b_0$. Let $Y$ be simply connected (in particular, $Y$ is
path-connected). Let $y_0\in Y$. Let $f\colon Y\to B$ be continuous, with
$f(y_0)=b_0$.

Prove that the following function $g\colon Y\to E$ is well-defined: Given
$y\in Y$, choose a path $\gamma$ from $y_0$ to $y$; let $\beta$ be the lift
of $f\circ\gamma$ to $E$ starting at $e_0$; now define $g(y)=\beta(1)$.

You may use the fact (without proving it) that the lift of a path homotopy
is again a path homotopy.
\end{problem}
\begin{proof}
\end{proof}

\begin{problem}
Let $S^2$ be the $2$-sphere, that is, the following subspace of $\bfR^3$:
\[
\left\{\,(x,y,z)\in\bfR^3\;\middle|\;x^2+y^2+z^2=1\,\right\}.
\]
Let $x_0$ be the point $(0,0,1)\in S^2$.

Use the Seifert--van Kampen theorem to prove that $\pi_1(S^2,x_0)$ is the
trivial group. You may use either of the two versions of the Seifert--van
Kampen theorem given in Munkres's book. You will not get credit for any
other method.
\end{problem}
\begin{proof}
\end{proof}

%%% Local Variables:
%%% mode: latex
%%% TeX-master: "../MA571-Quals"
%%% End:

\end{document}

%%% Local Variables:
%%% mode: latex
%%% TeX-master: t
%%% End:
