\chapter{Homework 9 Solutions}
\begin{problem}[WebAssign HW 9, 1]
Evaluate the integral using integration by parts with the indicated choices
of $u$ and $\diff v$. (Use $C$ for the constant of integration.)
\[
\int 7x^2\ln x\diff x;\qquad
u=\ln x,\quad
\diff v=7x^2\diff x.
\]
\end{problem}
\begin{proof}[Solution]
Remember the formula for integration by parts? No? Well, here it is:
\begin{equation}
  \label{eq:integration-by-parts}
\int u\diff v=uv-\int v\diff u.
\end{equation}
Now, let us integrate the above expression making the substitutions which
were given to us:
\begin{align*}
\int u\diff v&=\int\ln 7x^2\diff x\\
             &=\tfrac{7}{3}x^3\ln x-\int\frac{1}{x}\cdot \tfrac{7}{3}x^3\diff
               x\\
             &=\tfrac{7}{3}x^3\ln x-\int \tfrac{7}{3}x^2 \diff x\\
             &=\boxed{\tfrac{7}{3}x^3\ln x-\tfrac{7}{9}x^3+C}
\end{align*}
\end{proof}

\begin{problem}[WebAssign HW 9, 2]
Evaluate the integral. (Use $C$ for the constant of integration.)
\[
\int 5x\cos 6x\diff x.
\]
\end{problem}
\begin{proof}[Solution]
Since the derivative of $5x$ is just $5$, let's use the substitution $u=5x$
and $\diff v=\cos 6x\diff x$. Then, substituting this into Equation
(\ref{eq:integration-by-parts}) we have:
\begin{align*}
  \int u\diff v&=\int 5x\cos 6x\diff x\\
               &=\tfrac{5}{6}x\sin 6x-\int \tfrac{5}{6}\sin 6x\diff x\\
               &=\boxed{\tfrac{25}{6}x\sin 6x+\tfrac{5}{36}\cos 6x+C.}
\end{align*}
\end{proof}

\begin{problem}[WebAssign HW 9, 3]
Evaluate the integral. (Use $C$ for the constant of integration.)
\[
\int \left( x^2+3x \right)\cos x\diff x.
\]
\end{problem}
\begin{proof}[Solution]
Again, since the third derivative of $x^2+3x$ disappears, let's make make
the substitution $u=x^2+3x$ and $\diff v=\cos x\diff x$. Then, we have
\begin{align}
\label{eq:hw-9-3}
\int u\diff v
&=\int \left(x^2+3\right)\cos x\diff x\nonumber\\
&=\left(x^2+3x\right)\sin x-\int (2x+3)\sin x\diff x.
\end{align}
Now what? That second integral looks pretty hard to integrate? We use
integration by parts on that. Since the derivative of $2x$ is $2$ let's
make the substitution $u=2x$ and $\diff v=\sin x\diff x$ and we have
\[
\int (2x+3)\sin x\diff x=
-(2x+3)\cos x-\int 2(-\cos x)\diff x
=-(2x+3)\cos x+2\sin x.
\]
Substituting this back into Equation (\ref{eq:hw-9-3}), we have
\begin{align*}
\int u\diff v
&=\left(x^2+3x\right)\sin x-\int 2x\sin x\diff x\\
&=\left(x^2+3x\right)\sin x-\left(-(2x+3)\cos x+2\sin x\right)\\
&=\left(x^2+3x\right)\sin x+(2x+3)\cos x-2\sin x+C\\
&=\boxed{\left(x^2+3x-2\right)\sin x+(2x+3)\cos x+C.}
\end{align*}
\end{proof}

\begin{problem}[WebAssign HW 9, 4]
Evaluate the integral. (Use $C$ for the constant of integration.)
\[
\int\sin^{-1}x\diff x
\]
\end{problem}
\begin{proof}[Solution]
If it were not for integration by parts, I wouldn't know how to even begin
to integrate this one. Since we know the derivative of $\sin^{-1}x$, let's
make the substitution $u=\sin^{-1}x$ and $\diff v=\diff x$. Then we have
\begin{align*}
\int u\diff v&=\int \sin^{-1}x\diff x\\
             &=x\sin^{-1}x-\int\frac{x}{\sqrt{1-x^2}}\diff x.
\end{align*}
Now, how in the world do we evaluate $\int\frac{x}{\sqrt{1-x^2}}\diff x$?
Integration by parts? That sounds hard. Why not make a substitution? What
substitution? Well, we have a power of $x$ in the denominator of the
integral
\[
\int\frac{x}{\sqrt{1-x^2}}\diff x,
\]
and we have a square root of a $x^2$ in the denominator, why not $w=1-x^2$?
If we do that, we have $\diff w=-2x\diff x$,i.e., $\diff x=\diff w/(-2x)$
and we have
\begingroup
\allowdisplaybreaks
\begin{align*}
\int\frac{x}{\sqrt{w}}\frac{\diff w}{-2x}
&=\int\frac{1}{-2\sqrt{w}}\diff w\\
&=-\frac{1}{2}\int w^{-1/2}\diff w\\
&=-\frac{1}{2}\left(\frac{w^{1/2}}{1/2}\right)\\
&=w^{1/2}\\
&=\sqrt{1-x^2.}
\end{align*}
\endgroup
Now, back to our original problem. Substituting what we just computed, we
have
\[
\int \sin^{-1}x\diff x=\boxed{x\sin^{-1}x-\sqrt{1-x^2}+C.}
\]
\end{proof}

% p468:1,3,7,10,17,27,29,32,37,62
\begin{problem}[WebAssign HW 9, 5]
Evaluate the integral.
\[
\int_1^3 17r^3\ln r\diff r.
\]
\end{problem}
\begin{proof}[Solution]
Hopefully, you have caught on the pattern here. We make a choice of $u$ and
$\diff v$ depending on which one we think is easier to differentiate or
integrate. In this case, the integral of $\ln r$ is hard, but the integral
of $17r^3$ is easy so let's make the substitution $u=\ln r$ and $\diff
v=17r^3\diff x$ and we get
\begin{align*}
  \int_1^3 u\diff v&=\int_1^3 17r^3\ln r\diff r\\
                   &=\left.\tfrac{17}{4}r^4\ln r\right|_1^3
                     -\int_1^3\frac{17r^4}{4r}\diff r\\
                   &=\left.\tfrac{17}{4}r^4\ln
                     r-\tfrac{17}{16}r^4\right|_1^3\\
                   &=\left.\frac{4\cdot 17r^4\ln
                     r-17r^4}{16}\right|_1^3\\
                   &=\left.\frac{17r^4(4\ln r-1)}{16}\right|_1^3\\
                   &=\frac{17\cdot 3^4(4\ln 3-1)}{16}-\frac{17\cdot (4\ln
                     1-1)}{16}\\
                   &=\frac{1377(4\ln 3-1)}{16}+\frac{17}{16}\\
                   &=\frac{1377\cdot 4\ln 3-1377+17}{16}\\
                   &=\tfrac{1377}{4}\ln 3-\tfrac{1360}{16}\\
                   &=\boxed{\tfrac{1377}{4}\ln 3-85.}
\end{align*}
\end{proof}

\begin{problem}[WebAssign HW 9, 6]
Evaluate the integral.
\[
\int_0^1 \frac{y}{e^{3y}}\diff y.
\]
\end{problem}
\begin{proof}[Solution]
By parts, let $u=y$ and $\diff v=e^{-3y}\diff y$. Then, we have
\begin{align*}
\int_0^1\frac{y}{e^{3y}}\diff y
&=\left.-\frac{ye^{-3y}}{3}\right|_0^1-\int-\frac{e^{-3y}}{3}\diff y\\
&=\left.-\frac{ye^{-3y}}{3}\right|_0^1
+\int\frac{e^{-3y}}{3}\diff y\\
&=\left.-\frac{ye^{-3y}}{3}-\frac{e^{-3y}}{9}\right|_0^1\\
&=\left.-\frac{3ye^{-3y}+e^{-3y}}{9}\right|_0^1\\
&=-\frac{3e^{-3}+e^{-3}}{9}+\frac{1}{9}\\
&=\boxed{\frac{1-4e^{-3}}{9}.}
\end{align*}
\end{proof}

\begin{problem}[WebAssign HW 9, 7]
Evaluate the integral.
\[
\int_1^5\frac{\ln^2 x}{x^3}\diff x.
\]
\end{problem}
\begin{proof}[Solution]
First, make the substitution $u=\ln^2 x$ and $\diff v=\diff x/x^3$. Then,
we have
\begin{align*}
\int u\diff v&=\int_1^5\frac{\ln^2 x}{x^3}\diff x\\
             &=\left.-\frac{\ln^2 x}{2x^2}\right|_1^5
               -\int_1^5-\frac{2\ln x}{2x^3}\diff x\\
             &=\left.-\frac{\ln^2 x}{2x^2}\right|_1^5
               +\int_1^5\frac{\ln x}{x^3}\diff x.
\end{align*}
Then we integrate $\int_1^5-\ln x/x^3\diff x$ by parts and we get
\begin{align*}
\int_1^5\frac{\ln x}{x^3}\diff x
&=\left.-\frac{\ln x}{2x^2}\right|_1^5
+\int_1^5\frac{1}{x}\frac{1}{2x^2}\diff x\\
&=\left.-\frac{\ln x}{2x^2}\right|_1^5
+\int_1^5\frac{1}{2x^3}\diff x\\
&=\left.-\frac{\ln x}{2x^2}-\frac{1}{4x^2}\right|_1^5\\
&=\left.-\frac{2\ln x+1}{4x^2}\right|_1^5.
\end{align*}
So our integral is
\begin{align*}
\left.-\frac{\ln^2 x}{2x^2}-\frac{2\ln x+1}{4x^2}\right|_1^5
&=\left.-\frac{2\ln^2 x+2\ln x+1}{4x^2}\right|_1^5\\
&=-\frac{2\ln^2 5+2\ln 5+1}{4\cdot 5^2}+\frac{1}{4}\\
&=-\frac{2\ln^2 5+2\ln 5-24}{4\cdot 5^2}\\
&=\boxed{\frac{6}{25}-\tfrac{1}{50}\ln^2 5-\tfrac{1}{50}\ln 5.}
\end{align*}
\end{proof}

\begin{problem}[WebAssign HW 9, 8]
First make a substitution and then use integration by parts to evaluate the
integral. (Use $C$ for the constant of integration.)
\[
\int 7\cos\sqrt{x}\diff x.
\]
\end{problem}
\begin{proof}
What's the most obvious substitution? Perhaps $w=\sqrt{x}$. Then, we get
\[\diff w=-\frac{\diff x}{2\sqrt{x}}=\frac{\diff x}{2w}.\]
So our integral turns into
\[
\int 14 w\cos w\diff w
\]
Then, by integration by parts, setting $u=14w$, we get
\begin{align*}
  \int u\diff v&=\int 14w\cos w\diff w\\
               &=14w\sin w-\int 14\sin w\diff w\\
               &=14w\sin w-\int 14\sin w\diff w\\
               &=14w\sin w+14\cos w+C.
\end{align*}
Substituting back, we get our answer
\[
\boxed{14\sqrt{x}\sin\sqrt{x}+14\cos\sqrt{x}+C.}
\]
\end{proof}

\begin{problem}[WebAssign HW 9, 9]
Use the method of cylindrical shells to find the volume $V$ generated by
rotating the region bounded by the given curves about the specified axis.
\[
y=4e^x,\quad y=4e^{-x},\quad x=1;\qquad\text{about the $y$-axis.}
\]
\end{problem}
\begin{proof}[Solution]
I give up on trying to plot these, (it's too much work for one person). By
the method of cylindrical shells, we want to find an expression for the
cross-sectional area (which has the form of a cylinder) in terms of our
variable perpendicular to the axis we are rotating about. In this case, we
are rotating about the $y$-axis, so we must solve for the cross-sectional
area in terms of $x$ like so:
\[
A(x)=2\pi x(4e^x-4e^{-x})=8\pi x(e^{x}-e^{-x}).
\]
Then, we integrate the area between where the curves $4e^x$, $4e^{-x}$ and
$x=1$ intersect (if you plot the curves, this will be the area of the
difference of $e^x$ and $e^{-x}$ from $0\leq x\leq 1$). Thus, the
volume is
\begin{align*}
V&=\int_0^1 8\pi x(e^x-e^{-x})\diff x\\
 &=8\pi\left(\left.x(e^x+e^{-x})\right|_0^1
   -\int_0^1 e^x+e^{-x}\diff x\right)\\
 &=8\pi\left(\left.x(e^x+e^{-x})\right|_0^1
   -\int_0^1 e^x+e^{-x}\diff x\right)\\
 &=8\pi\left(\left. x(e^x+e^{-x})-e^x+e^{-x}\right|_0^1\right)\\
 &=8\pi\left(e+e^{-1}-e+e^{-1}-(0-1+1)\right)\\
 &=\boxed{\frac{16\pi}{e}.}
\end{align*}
\end{proof}

%%% Local Variables:
%%% mode: latex
%%% TeX-master: "../MA166-HW-Current"
%%% End:
