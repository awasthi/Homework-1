\chapter{Homework 6 Solutions}
Sorry, no pictures. Unless I can get a hang of
\href{https://inkscape.org/}{Inkscape}'s graphing syntax, I won't be able
to generate all of these images with just
\href{http://www.texample.net/tikz/}{Ti$k$Z}; it is simply too much work. I
am willing to plot integrals, cross-sectional areas, etc., but no shells
and solids of revolution.

\begin{problem}[Stewart \S{6.2}, Exercise 2]
Find the volume of the solid obtained by rotating the region bounded by the
given curves about the specified line. Sketch the region, the solid, and a
typical washer.
\\\\
$y=1-x^2$, $y=0$; about the $x$-axis
\end{problem}
\begin{proof}[Solution]
  We first need to figure out an equation for the cross-sectional area
  $A(y)$ for our solid. Since we are rotating about the $x$-axis, our
  problem is most easily approached by solving for $A$ in terms of
  $x$. Since we are rotating, $A(y)=\pi y^2$ will be the area of a
  circle. Solving for $A(y)$ in terms of $x$, we have $A(x)=\pi
  (1-x^2)^2$.

  Next we need to find the points where the graph $y=1-x^2$ intersects the
  line $y=0$, i.e., where $1-x^2=0$. This is easy as, $-x^2=-1$ so $x^2=1$
  hence, $x=-1$ or $1$.

  Now we are ready to start calculating the volume of the solid of
  revolution: Since the graph of $y$ is symmetric about the $y$-axis, it
  suffices to compute the following integral from $0$ to $1$ and multiply
  by $2$. Hence, we have
  \begin{align*}
    2\pi\int_0^1\left(1-x^2\right)^2\diff x
    &=2\pi\int_0^11-2x^2+x^4\diff x\\
    &=2\pi\left(\left.x-\frac{2x^3}{3}+\frac{x^5}{5}\right|_0^1\right)\\
    &=2\pi\left(1-\frac{2}{3}+\frac{1}{5}-(0-0+0)\right)\\
    &=2\pi\left(\frac{6}{15}\right)\\
    &=\boxed{\frac{12\pi}{15}.}
  \end{align*}
\end{proof}

\begin{problem}[Stewart \S{6.2}, Exercise 3]
$y=\sqrt{x-1}$, $y=0$, $x=5$; about the $x$-axis.
\end{problem}
\begin{proof}[Solution]
Same procedure, set $A(x)=\pi y^2=\pi\left(\sqrt{x-1}\right)^2$. Find the
point $x$ where $y=\sqrt{x-1}$ intersects the lines $y=0$ and $x=5$. These
values are $x=1$ and $x=5$. Now we are ready to compute the integral:
\begin{align*}
\pi\int_1^5x-1\diff x
&=\pi\left(\left.\frac{x^2}{2}-x\right|_1^5\right)\\
&=\pi\left(\frac{25}{2}-5-\frac{1}{2}+1\right)\\
&=\boxed{8\pi.}
\end{align*}
\end{proof}

\begin{problem}[Stewart \S{6.2}, Exercise 8]
$y=\frac{1}{4}x^2$, $y=5-x^2$; about the $x$-axis.
\end{problem}
\begin{proof}[Solution]
First, let us find the points of intersection of the graphs
$y=\frac{1}{4}x^2$, $y=5-x^2$ as such
\begin{align*}
\frac{1}{4}x^2&=5-x^2\\
x^2&=20-4x^2\\
5x^2&=20\\
x^2&=4
\end{align*}
so $x=\pm 2$. Now, since $y=5-x^2$ is above $y=\frac{1}{4}x^2$ for all
$-2\leq x\leq 2$, the cross-section $A(y)=\pi y^2$ varies as the difference
$5-x^2-\frac{1}{4}x^2=5-\frac{5}{4}x^2$ so
\[
A(x)=\pi\left(5-\frac{5}{4}x^2\right)^2
\]
and we have
\begin{align*}
\pi\int_{-2}^2 \left(5-\frac{5}{4}x^2\right)^2\diff x
&=\pi\int_{-2}^225-\frac{5}{2}x^2+\frac{25}{16}x^4\diff x
\\
&=2\pi\int_0^225-\frac{5}{2}x^2+\frac{25}{16}x^4\diff x
\\
&=2\pi\left(\left.25x-\frac{5}{6}x^3+\frac{5}{16}x^5\right|_0^2\right)\\
&=2\pi\left(25\cdot 2-\frac{5\cdot 8}{6}+\frac{5\cdot 2^5}{16}\right)\\
&=2\pi\left(50-\frac{20}{3}+10\right)\\
&=2\pi\left(\frac{150-20+30}{3}\right)\\
&=\boxed{\frac{320\pi}{3}}
\end{align*}
\end{proof}

\begin{problem}[Stewart \S{6.2}, Exercise 9]
$y^2=x$, $x=2y$; about the $y$-axis.
\end{problem}
\begin{proof}[Solution]
Since we are rotating about the $y$-axis, we want to consider the
cross-sectional area $A$ perpendicular to the $y$-axis. Therefore, we want
to solve for $A$ in terms of $x$. Now, let us find the values of $y$ where
the equations $x=y^2$ and $x=2y$ intersect. These points are
\begin{align*}
y^2&=2y\\
y^2-2y&=0\\
y(y-2)&=0
\end{align*}
so $y=0$ or $y=2$. Now that we have our bounds, let us express the radius
of our cross-sectional area in terms of $y$, i.e., it will be the
difference $y^2-2y$ so that $A(y)=\pi (y(y-2))^2$ and we are ready to
integrate:
\begin{align*}
\pi\int_0^2y^2(y-2)^2\diff y
&=\pi\int_0^2y^2(y^2-2y+4)\diff y\\
&=\pi\int_0^2y^4-2y^3+4y^2\diff y\\
&=\pi\left(\left.\frac{y^5}{5}-\frac{y^4}{2}+\frac{4y^3}{3}\right|_0^2\right)\\
&=\pi\left(\frac{2^5}{5}-\frac{2^4}{4}+\frac{4\cdot 2^3}{3}\right)\\
&=\boxed{\frac{196\pi}{15}.}
\end{align*}
\end{proof}

\begin{problem}[Stewart \S{6.2}, Exercise 19]

\end{problem}
\begin{proof}[Solution]
\end{proof}

\begin{problem}[Stewart \S{6.2}, Exercise 21]

\end{problem}
\begin{proof}[Solution]
\end{proof}

\begin{problem}[Stewart \S{6.2}, Exercise 24]

\end{problem}
\begin{proof}[Solution]
\end{proof}

\begin{problem}[Stewart \S{6.2}, Exercise 26]

\end{problem}
\begin{proof}[Solution]
\end{proof}

\begin{problem}[Stewart \S{6.2}, Exercise 27]

\end{problem}
\begin{proof}[Solution]
\end{proof}

%%% Local Variables:
%%% mode: latex
%%% TeX-master: "../MA166-HW-Current"
%%% End:
