\chapter{Solutions to Homework}
These are solutions the assigned problems from Stewart's \emph{Calculus,
  Early Transcendentals, 7th ed.}\,from the
\href{https://www.math.purdue.edu/academic/files/courses/2016spring/MA16600/166assignr1.pdf}{assignment
  sheet}. I plan to keep updating the solutions sheet as the semester goes
on.
\begin{problem}[Stewart \S{12.1}, Exercise 6]
\begin{enumerate}[label=(\alph*)]
\item What does the equation $x=4$ represent in $\bbR^2$? What does it
  represent in $\bbR^3$? Illustrate with a sketches.
\item What does the equation $y=3$ represent in $\bbR^3$? What does $z=5$
  represent? What does the pair of equations $y=3$, $z=5$ represent? In
  other words, describe the set of points $(x,y,z)$ such that $y=3$ and
  $z=5$. Illustrate with a sketch.
\end{enumerate}
\end{problem}
\begin{proof}[Solution]
In such problems, your first instinct should be to ask yourself ``What is a
point on the equation I am given?.''

(a) Let us begin by making a table of values of $x=4$ in $\bbR^2$:
\begin{center}
\begin{tabular}{|c|c|}
\hline
$x$&$y$\\
\hline
4&0\\
4&1\\
4&-1\\
4&1000\\
\hline
\end{tabular}.
\end{center}
Recognize the pattern? In fact, for any value of $y$ that we choose, $x$
will always be $4$ so the equation represents a \ul{vertical line which
  perpendicular to the $x$-axis, and touching the $x$-axis at the point
  $(0,4)$.}

We can use the same method to figure out what this equation, $x=4$,
represents in $\bbR^3$:
\begin{center}
\begin{tabular}{|c|c|c|}
\hline
$x$&$y$&$z$\\
\hline
4&0&0\\
4&1&0\\
4&0&1\\
4&1&1\\
\hline
\end{tabular}.
\end{center}
If you know some linear algebra, we have enough
\href{https://en.wikipedia.org/wiki/Linear_independence}{linearly
  independent} vectors to span a \ul{plane}. If you don't know any
linear algebra, just note that if we choose any values for $y$ and $z$ the
point $(4,y,z)$ will still be on the equation $x=4$. Hence, the equation
$x=4$ represents a plane \ul{perpendicular to $x$-axis at the
  point $(4,0,0)$.}
\\\\
(b) Again, we may repeat the previous strategy to solve this problem
$y=3$,i.e., we make a table with points in the equation $y=3$
\begin{center}
\begin{tabular}{|c|c|c|}
\hline
$x$&$y$&$z$\\
\hline
$0$&$3$&$0$\\
$1$&$3$&$0$\\
$1$&$3$&$-1$\\
\hline
\end{tabular}.
\end{center}
Again, we see that no matter what value of $x$ and $z$ we pick, the point
$(x,3,z)$ will always be in the equation $y=3$. Hopefully, you can see why
this equation represents a \ul{plane perpendicular to the $y$-axis at the
  point $(0,3,0)$}.

Hopefully you are able to see the pattern now and can tell me all on your
own what the equation $z=5$ represents in $\bbR^3$. Right! It will be a
\ul{plane perpendicular to the $z$-axis and touching the $z$-axis at the
  point $(0,0,5)$.} In fact, we can make a categorical statement about what
$x=a$ and $y=b$ or $z=c$ represent in $\bbR^3$:
\begin{definition}
Let $a$ stand-in for one of the $x$, $y$, or $z$ variables and let $b$ be
any real number. Then the equation
\begin{equation}
  \label{eq:plane}
  a=b
\end{equation}
represents a plane perpendicular to the $a$-axis and touching the $a$-axis
at the point $(b,0,0)$ (or $(0,b,0)$, or $(0,0,b)$ as it may be).
\end{definition}

Now, it is not so clear what the pair of equations $y=3$ and $z=5$, but we
can make a table and take a look at some of the values these equations take
on
\begin{center}
\begin{tabular}{|c|c|c|}
\hline
$x$&$y$&$z$\\
\hline
$-1$&$3$&$5$\\
$0$&$3$&$5$\\
$-1$&$3$&$5$\\
$1000$&$3$&$5$\\
\hline
\end{tabular}.
\end{center}
So no matter what value of $x$ we pick, the point $(x,3,5)$ will be in the
pair of equations $y=3$ and $z=5$. Hence, we can see that this pair of
equations represents a \ul{line extending in the same direction $x$-axis
  which touches the point $(0,3,5)$.} That last bit is a bit arbitrary, we
can say it touches the point $(1,3,5)$ or even $(10000,3,5)$, there is not
much else we can say since the line never touches the any of the axes, but
it does cross the $yz$-plane at the point $(0,3,5)$, which is questionably
useful.

In fact, an easier way to say this is that the pair of equations $y=3$ and
$z=5$ represents the intersection of the planes $y=3$ and $z=5$. Finding
the line where these planes intersect is equivalent to finding the points
that the pair of equations satisfy.
\end{proof}
\begin{remarks*}
In all this discussion, I failed to address one thing and that is, for
example, why the point $(0,-3,5)$ is not on the pair of equations $y=3$ and
$z=5$. Well, that is simply because we have a restriction on $y$ and $z$,
i.e., $y$ must be $3$ and $z$ must be $5$ or else we are looking at
something all together different.
\end{remarks*}

\begin{problem}[Stewart \S{12.1}, Exercise 16]
Show that the equation represents a sphere, and find its center and radius.
\[
x^2+y^2+z^2-2x-4y+8z=15.
\]
\end{problem}
\begin{proof}[Solution]
You have no doubt seen the \emph{standard equation} for a sphere of
radius $r$ centered at $\left(x_0,y_0,z_0\right)$; in case you missed it
here it is
\begin{equation}
\label{eq:sphere}
\left(x-x_0\right)^2+\left(y-y_0\right)^2+\left(z-z_0\right)^2=r^2.
\end{equation}
What is scary is that an equation of the form
\[
Ax^2+By^2+Cz^2+Dxy+Exz+Fyz+Gx+Hy+Iz+J=0
\]
is called a \href{https://en.wikipedia.org/wiki/Conic_section}{conic
  section} and if we are given an equation like the one above, we cannot
always factor it into something nice. Enough of that, our equation is
clearly an equation of a sphere (you can tell this because it does not
contain any mixed terms, i.e., terms of the form $xy$, $xz$, or $yz$) so we
can use the method of
\href{https://en.wikipedia.org/wiki/Completing_the_square}{completing the
  square} to express the equation in standard form:
\begin{align*}
x^2+y^2+z^2-2x-4y+8z&=15\\
\left(x^2-2x\right)+\left(y^2-4y\right)+\left(z^2+8z\right)&=15\\
\left(x^2-2x+1\right)-1+\left(y^2-4y+2\right)-2+\left(z^2+8z+16\right)-16&=15\\
\left(x^2-2x+1\right)+\left(y^2-4y+2\right)+\left(z^2+8z+16\right)&=15+1+2+16\\
\left(x-1\right)^2+\left(x-2\right)^2+\left(x+4\right)^2&=34\\
\end{align*}
Now that we've gotten the equation down the standard form of a sphere, we
can read off the necessary values: So the equation
\[
x^2+y^2+z^2-2x-4y+8z=15
\]
represents a \ul{sphere centered at $(1,2,4)$ with radius $\sqrt{34}$}.
\end{proof}

% p790:6,16,31,33,37,38
\begin{problem}[Stewart \S{12.1}, Exercise 31]
Describe in words the region of $\bbR^3$ represented by the inequality
\[
x^2+y^2+z^2\leq 3.
\]
\end{problem}
\begin{proof}[Solution]
  Remember the equation of our good old friend the
  \hyperref[eq:sphere]{sphere}? Well, it's not quite a sphere that we have
  but something like it. The equation we are given is indeed that of a
  sphere, but the radius is allowed to change from $0$ to $\sqrt{3}$ so it
  is in fact a union of spheres of every radius between $0$ and
  $\sqrt{3}$. In other words, it is the \ul{solid sphere (also called a
    ball) of radius $3$}.
\end{proof}

\begin{problem}[Stewart \S{12.1}, Exercise 33]
Describe in words the region of $\bbR^3$ represented by the inequality
\[
x^2+z^2\leq 9.
\]
\end{problem}
\begin{proof}[Solution]
  This looks something like the circle of radius $3$ in the $xz$-plane. In
  fact, we again have all of the circles of radius $0$ to $3$ in the
  $xz$-plane and this gives us a \ul{solid circle of radius (also called a
    disk) $3$}.
\end{proof}

\begin{problem}[Stewart \S{12.1}, Exercise 37]

\end{problem}
\begin{proof}[Solution]
\end{proof}

\begin{problem}[Stewart \S{12.1}, Exercise 38]

\end{problem}
\begin{proof}[Solution]
\end{proof}

% +12.2 beg.-Ex. 2
% p798:3,5
\begin{problem}[Stewart \S{12.2}, Exercise 3]

\end{problem}
\begin{proof}[Solution]
\end{proof}

\begin{problem}[Stewart \S{12.2}, Exercise 5]

\end{problem}
\begin{proof}[Solution]
\end{proof}

%%% Local Variables:
%%% mode: latex
%%% TeX-master: "../MA166-HW-Current"
%%% End:
