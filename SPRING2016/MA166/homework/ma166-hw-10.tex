\chapter{Homework 10 Solutions}
\begin{problem}[WebAssign HW 10, 1]
Evaluate the integral. (Use $C$ for the constant of integration.)
\[
\int 2\sin^2 x\cos^3 x\diff x
\]
\end{problem}
\begin{proof}[Solution]
First, note that $\cos^2 x=1-\sin^2 x$ so the integral above can be written
as
\begin{align*}
\int 2\sin^2 x\cos^3 x\diff x
&=\int 2\sin^2 x\cos^2 x\cos x\diff x\\
&=\int 2\sin^2x(1-\sin^2 x)\cos x\diff x.
\end{align*}
Now, make the substitution $u=\sin x$. Then $\diff u=\cos x\diff x$ so our
integral is now
\begin{align*}
\int 2u^2(1-u^2)\cos x\frac{\diff u}{\cos x}
&=\int 2u^2(1-u^2)\diff u\\
&=\int 2u^2-2u^4\diff u\\
&=\tfrac{2}{3}u^3-\tfrac{2}{5}u^5+C\\
&=\boxed{\tfrac{2}{3}\sin^3 x-\tfrac{2}{5}\sin^5 x+C.}
\end{align*}
\end{proof}

\begin{problem}[WebAssign HW 10, 2]
Evaluate the integral
\[
\int_0^{\pi/2}5\cos^2\theta\diff\theta.
\]
\end{problem}
\begin{proof}[Solution]
For this problem, we can use the double-angle identity, i.e.,
\begin{equation}
  \label{eq:cosine-double-angle}
\cos 2\theta=2\cos^2\theta-1=1-2\sin^2 x
\end{equation}
Substituting Equation (\ref{eq:cosine-double-angle}) into our integral, we
have
\begin{align*}
\int_0^{\pi/2}5\cos^2\theta\diff\theta&=
\int_0^{\pi/2}\tfrac{5}{2}(1+\cos 2\theta)\diff\theta\\
&=\tfrac{5}{2}\int_0^{\pi/2}(1+\cos 2\theta)\diff\theta\\
&=\tfrac{5}{2}\left(\left.\theta+\tfrac{1}{2}\sin
  2\theta\right|_0^{\pi/2}\right)\\
&=\tfrac{5}{2}\left(\frac{\pi}{2}+0-(0+0)\right)\\
&=\boxed{\frac{5\pi}{4}.}
\end{align*}
\end{proof}

\begin{problem}[WebAssign HW 10, 3]
Evaluate the integral.
\[
\int_0^{\pi/2}5\sin^2 x\cos^2 x\diff x.
\]
\end{problem}
\begin{proof}[Solution]
Again, we just use a trigonometric identity, specifically
\begin{equation}
  \label{eq:sine-double-angle}
\sin 2 x =2\sin  x \cos x .
\end{equation}
Using Equation (\ref{eq:sine-double-angle}) our integral turns into
\begin{align*}
\int_0^{\pi/2}5\sin^2x\cos^2x\diff x
&=5\int_0^{\pi/2}\tfrac{1}{4}(4\sin^2 x\cos^2x)\diff x\\
&=\frac{5}{4}\int_0^{\pi/2}(2\sin x\cos x)^2\diff x\\
&=\frac{5}{4}\int_0^{\pi/2}(\sin 2x)^2\diff x\\
&=\frac{5}{4}\int_0^{\pi/2}\sin^2 2x\diff x\\
&=\frac{5}{4}\int_0^{\pi/2}\sin^2 2x\diff x\\
\shortintertext{now we use this identity $2\sin^2 x =1-\cos 2 x $
  to get}
&=\frac{5}{4}\int_0^{\pi/2}\frac{1}{2}(1-\cos 4 x )\diff x \\
&=\frac{5}{8}\int_0^{\pi/2}1-\cos 4 x \diff x \\
&=\frac{5}{8}\left(\left. x -\tfrac{1}{4}\sin
  4 x \right|_0^{\pi/2}\right)\\
&=\boxed{\tfrac{5\pi}{16}.}
\end{align*}
\end{proof}

\begin{problem}[WebAssign HW 10, 4]
Evaluate the integral. (Remember to use $\ln(|u|)$ where appropriate. Use
$C$ for constant of integration.)
\[
\int 9\cos^2 x\tan^3 x\diff x.
\]
\end{problem}
\begin{proof}[Solution]
Let's make the substitution $u=\cos x$ so $\diff u=-\sin x\diff x$ and we
have
\begingroup
\allowdisplaybreaks
\begin{align*}
\int 9\cos^2 x\tan^3 x\diff x
&=9\int \cos^2 x\tan^3 x\diff x\\
&=9\int\frac{\cos^2 x\sin^3 x}{\cos^3 x}\diff x\\
&=9\int\frac{\sin^3 x}{\cos x}\diff x\\
&=9\int\frac{\sin x(1-\cos^2 x)}{\cos x}\diff x\\
\shortintertext{now, for the substitution,}
&=9\int\frac{\sin x(1-u^2)}{u}\frac{\diff u}{-\sin x}\\
&=9\int\frac{u^2-1}{u}\diff u\\
&=9\int u-\frac{1}{u}\diff u\\
&=\tfrac{9}{2}u^2-9\ln|u|+C\\
\shortintertext{and back, we have}
&=\boxed{\tfrac{9}{2}\cos^2 x-9\ln|\cos x|+C.}
\end{align*}
\endgroup
\end{proof}

\begin{problem}[WebAssign HW 10, 5]
Evaluate the integral. (Use $C$ for the constant of integration.)
\[
\int 5\tan^2 x\diff x.
\]
\end{problem}
\begin{proof}[Solution]
The easiest one so far. Recall the identity
\begin{equation}
  \label{eq:tangent-identity}
\sec^2 x -\tan^2 x =1.
\end{equation}
Then, we gave
\[
\int 5\tan^2x\diff x=5\int(\sec^2 x -1)\diff x =
\boxed{5\tan x -5 x +C.}
\]
\end{proof}

% p476:1,7,11,17,23,24,35,61
\begin{problem}[WebAssign HW 10, 6]
\[
\int 8\left(\tan^2x+\tan^4x\right)\diff x.
\]
\end{problem}
\begin{proof}[Solution]
We'll make the substitution $u=\tan x$, so $\diff u=\sec^2x\diff
x$. Proceeding, we have
\begin{align*}
\int 8\left(\tan^2 x+\tan^4x\right)\diff x
&=8\int\tan^2 x(1+\tan^2 x)\diff x\\
\shortintertext{by Equation (\ref{eq:tangent-identity}), we have}
&=8\int\tan^2 x\sec^2 x\diff x\\
&=8\int u^2\diff u\\
&=\tfrac{8}{3}u^3+C\\
&=\boxed{\tfrac{8}{3}\tan^3 x+C.}
\end{align*}
\end{proof}

\begin{problem}[WebAssign HW 10, 7]
Evaluate the integral
\[
\int_{\pi/6}^{\pi/2}\cot^2 x\diff x
\]
\end{problem}
\begin{proof}[Solution]
Remember that, like Equation (\ref{eq:tangent-identity}), there is a
cotangent-identity
\begin{equation}
  \label{eq:cotangent-identity}
1+\cot^2 x=\csc^2 x.
\end{equation}
With the help of this equation, we have
\begin{align*}
\int_{\pi/6}^{\pi/2}\cot^2 x\diff x&=
\int_{\pi/6}^{\pi/2}(\csc^2-1)\diff x\\
&=\left.-\cot x-x\right|_{\pi/6}^{\pi/2}\\
&=0-\frac{\pi}{2}-\left(\sqrt{3}-\frac{\pi}{6}\right)\\
&=\boxed{\sqrt{3}-\frac{\pi}{3}.}
\end{align*}
\end{proof}

\begin{problem}[WebAssign HW 10, 8]
Find the volume $V$ obtained by rotating the region bounded by the given
curves about the specified axis.
\[
y=9\sin x,\quad
y=0,\quad
\frac{\pi}{2}\leq x\leq\pi;\qquad
\text{about the $x$-axis.}
\]
\end{problem}
\begin{proof}[Solution]
Since we are rotating about the $x$-axis, we need to find the
cross-sectional area in terms of $y$ in terms of $x$, hence we have
\[
A(x)=\pi y^2=81\pi\sin^2 x.
\]
Now, integrating from $\pi/2$ to $\pi$, we get
\begin{align*}
\int_{\pi/2}^\pi A(x)\diff x
&=81\pi\int_{\pi/2}^\pi \sin^2 x\\
\shortintertext{using the double-angle identity for cosine, Equation
  (\ref{eq:cosine-double-angle}), we get}
&=81\pi\int_{\pi/2}^\pi \tfrac{1}{2}(1-\cos 2x)\diff x\\
&=\tfrac{81\pi}{2}\left(\left.x-\tfrac{1}{2}\sin 2x\right|_{\pi/2}^\pi\right)\\
&=\tfrac{81\pi}{2}\left(\pi-0-(\tfrac{1}{2}\pi)\right)\\
&=\boxed{\frac{81\pi^2}{4}.}
\end{align*}
\end{proof}

%%% Local Variables:
%%% mode: latex
%%% TeX-master: "../MA166-HW-Current"
%%% End:
