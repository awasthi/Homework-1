\def\documentauthor{Carlos Salinas}
\def\documenttitle{MA166 Recitation Notes and Exercises}
% \def\hwnum{1}
\def\shorttitle{MA166-Notes}
\def\coursename{MA166}
\def\documentsubject{calculus ii}
\def\authoremail{salinac@purdue.edu}

\documentclass[article,oneside,10pt]{memoir}
\usepackage{geometry}
\usepackage[dvipsnames]{xcolor}
\usepackage[
    breaklinks,
    bookmarks=true,
    colorlinks=true,
    pageanchor=false,
    linkcolor=black,
    anchorcolor=black,
    citecolor=black,
    filecolor=black,
    menucolor=black,
    runcolor=black,
    urlcolor=black,
    hyperindex=false,
    hyperfootnotes=true,
    pdftitle={\shorttitle},
    pdfauthor={\documentauthor},
    pdfkeywords={\documentsubject},
    pdfsubject={\coursename}
    ]{hyperref}

% Use symbols instead of numbers
\renewcommand*{\thefootnote}{\fnsymbol{footnote}}

%% Math
\usepackage{amsthm}
\usepackage{amssymb}
\usepackage{mathtools}
% \usepackage{unicode-math}

%% PDFTeX specific
\usepackage[mathcal]{euscript}
\usepackage{mathrsfs}

\usepackage[LAE,LFE,T2A,T1]{fontenc}
\usepackage[utf8]{inputenc}
\usepackage[farsi,french,german,spanish,russian,english]{babel}
\babeltags{fr=french,
           de=german,
           en=english,
           es=spanish,
           pa=farsi,
           ru=russian
           }
\def\spanishoptions{mexico}

\selectlanguage{english}

\newcommand{\textfa}[1]{\beginR\textpa{#1}\endR}

\usepackage{cmap}
\usepackage{CJKutf8}
\newcommand{\textkr}[1]{\begin{CJK}{UTF8}{mj}#1\end{CJK}}
\newcommand{\textjp}[1]{\begin{CJK}{UTF8}{min}#1\end{CJK}}
\newcommand{\textzh}[1]{\begin{CJK}{UTF8}{bsmi}#1\end{CJK}}

\usepackage{graphicx}
\graphicspath{{figures/}}

% Misc
\usepackage{microtype}
\usepackage{multicol}
\usepackage[inline]{enumitem}
\usepackage{listings}
\usepackage{mleftright}
\mleftright

%% Theorems and definitions
%% remove parentheses
% \makeatletter
% \def\thmhead@plain#1#2#3{%
%   \thmname{#1}\thmnumber{\@ifnotempty{#1}{ }\@upn{#2}}%
%   \thmnote{ {\the\thm@notefont#3}}}
% \let\thmhead\thmhead@plain
% \makeatother

\theoremstyle{plain}
\newtheorem{theorem}{Theorem}
\newtheorem{proposition}[theorem]{Proposition}
\newtheorem{corollary}[theorem]{Corollary}
\newtheorem{claim}[theorem]{Claim}
\newtheorem{lemma}[theorem]{Lemma}
\newtheorem{axiom}[theorem]{Axiom}

\newtheorem*{corollary*}{Corollary}
\newtheorem*{claim*}{Claim}
\newtheorem*{lemma*}{Lemma}
\newtheorem*{proposition*}{Proposition}
\newtheorem*{theorem*}{Theorem}

\theoremstyle{definition}
\newtheorem{definition}{Definition}
\newtheorem{example}{Examples}
\newtheorem{examples}[example]{Examples}
\newtheorem{exercise}{Exercise}[chapter]
\newtheorem{problem}[exercise]{Problem}

\newtheorem*{example*}{Example}
\newtheorem*{exercise*}{Exercise}
\newtheorem*{problem*}{Problem}

%% Redefinitions & commands
\newcommand{\nle}{\ensuremath{\not<}}
\newcommand{\nge}{\ensuremath{\not>}}
\newcommand{\nsubset}{\ensuremath{\not\subset}}
\newcommand{\nsupset}{\ensuremath{\not\supset}}
\newcommand\minus{\ensuremath{\null\smallsetminus}}
\renewcommand\qedsymbol{\ensuremath{\null\hfill\blacksquare}}

%% Commands and operators
\DeclareMathOperator{\id}{id}
\DeclareMathOperator{\im}{im}

\newcommand{\bbC}{\mathbb{C}}
\newcommand{\bbN}{\mathbb{N}}
\newcommand{\bbQ}{\mathbb{Q}}
\newcommand{\bbR}{\mathbb{R}}
\newcommand{\bbZ}{\mathbb{Z}}
\newcommand{\bfC}{\mathbf{C}}
\newcommand{\bfN}{\mathbf{N}}
\newcommand{\bfQ}{\mathbf{Q}}
\newcommand{\bfR}{\mathbf{R}}
\newcommand{\bfZ}{\mathbf{Z}}

\newcommand{\bfu}{\mathbf{u}}
\newcommand{\bfv}{\mathbf{v}}
\newcommand{\bfw}{\mathbf{w}}

\begin{document}
\author{\href{mailto:\authoremail}{\documentauthor}}
\title{\documenttitle}
\date{\today}
\maketitle
\chapter{Notes: Vectors and the Geometry of Spaces}
Material found in Stewart \S12.
\\\\
\section{Three-Dimensional Coordinate Systems}
Here are some of the most important concepts, equations, and theorems from
this section. I know. I know. These are very boring concepts that you have
probably seen all your life and you know how to do. But we must start
somewhere and here is a perfect place.

The distance between two points $P_1(x,y,z)$ and $P_2(x,y,z)$ in $\bbR^3$
is given by the formula
\begin{equation}
  \label{eq:distance-between-points}
|P_1P_2|=\sqrt{(x_2-x_1)^2+(y_2-y_1)^2+(z_2-z_1)^2}.
\end{equation}
This is also called the \emph{Euclidean norm} and generalizes to all
dimensions. Note that equation (\ref{eq:distance-between-points}) is
equivalent to
\[
\sqrt{(x_1-x_2)^2+(y_1-y_2)^2+(z_1-z_2)^2}=|P_2P_1|
\]
so that the distance between point does not depend on your point-of-view,
i.e, whether you think of the line starting connecting $P_1$ and $P_2$ as
starting at $P_1$ and ending at $P_2$ or vice-a-versa.

We often refer to the point $P_1(x,y,z)$ as the tuple $(x_1,y_1,z_2)$ and
$P_2(x,y,z)$ as $(x_2,y_2,z_2)$, $P_3(x,y,z)$ as $(x_3,y_3,z_3)$ and so on.
\\\\
The equation of a sphere with $C(h,k,l)$ and radius $r$ is
\begin{equation}
  \label{eq:equation-of-a-sphere}
(x-h)^2+(y-k)^2+(z-l)^2=r^2.
\end{equation}
In particular, if the center is the origin $O$, then the equation
(\ref{eq:equation-of-a-sphere}) reduces to
\[
x^2+y^2+z^2=r^2.
\]
\section{Vector}
A particle moves along a line segment from point $A$ to point $B$. The
corresponding displacement vector $\bfv$ has initial point $A$ and terminal
point $B$ and is written $\bfv=\overrightarrow{AB}$.
\subsection{Combining Vectors}
If $\bfu$ and $\bfv$ are vectors positioned so the initial point of $\bfv$
is at the terminal point of $\bfu$, then the sum $\bfu+\bfv$ is the vector
from the initial point of $\bfu$ to the terminal point of $\bfv$.

%%% Local Variables:
%%% mode: latex
%%% TeX-master: "../MA166-Recitation"
%%% End:

\chapter{Exercises Week 2}
%%% Local Variables:
%%% mode: latex
%%% TeX-master: "../MA166-Recitation"
%%% End:

\include{exercises/exercises-2}
\end{document}

%%% Local Variables:
%%% mode: latex
%%% TeX-master: t
%%% End:
