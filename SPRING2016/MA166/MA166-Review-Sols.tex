\def\documentauthor{Carlos Salinas}
\def\documenttitle{MA 166 Review Sheet Solutions}
\def\shorttitle{MA 166 Review}
\def\coursename{MA166}
\def\documentsubject{calculus ii}
\def\authoremail{salinac@purdue.edu}

\documentclass[10pt]{article}
\usepackage{geometry}
\usepackage[dvipsnames]{xcolor}
\usepackage[
    breaklinks,
    bookmarks=true,
    % colorlinks=true,
    pageanchor=false,
    linkcolor=black,
    anchorcolor=black,
    citecolor=black,
    filecolor=black,
    menucolor=black,
    runcolor=black,
    urlcolor=black,
    hyperindex=false,
    hyperfootnotes=true,
    pdftitle={\shorttitle},
    pdfauthor={\documentauthor},
    pdfkeywords={\documentsubject},
    pdfsubject={\coursename}
    ]{hyperref}

% Use symbols instead of numbers
\renewcommand*{\thefootnote}{\fnsymbol{footnote}}

%% Math
\usepackage{amsmath}
\usepackage{amsthm}
\usepackage{amssymb}
\usepackage{mathtools}
% \usepackage{eucal}
% \usepackage{mathrsfs}
% \usepackage[nointegrals]{wasysym}

%% Language
\usepackage{cmap}
\usepackage[LAE,LFE,T2A,T1]{fontenc}
\usepackage[utf8]{inputenc}
\usepackage[farsi,french,german,spanish,russian,english]{babel}
\babeltags{fr=french,
           de=german,
           en=english,
           es=spanish,
           pa=farsi,
           ru=russian
           }
\def\spanishoptions{mexico}

\selectlanguage{english}

\newcommand{\textfa}[1]{\beginR\textpa{#1}\endR}

\usepackage{CJKutf8}
\newcommand{\textkr}[1]{\begin{CJK}{UTF8}{mj}#1\end{CJK}}
\newcommand{\textjp}[1]{\begin{CJK}{UTF8}{min}#1\end{CJK}}
\newcommand{\textzh}[1]{\begin{CJK}{UTF8}{bsmi}#1\end{CJK}}

%% Misc
\usepackage{graphicx}
\graphicspath{{figures/}}

\usepackage{microtype}
\usepackage{lineno}
\usepackage{multicol}
\usepackage[inline]{enumitem}
\usepackage{listings}
\usepackage{mleftright}
\mleftright
\usepackage{carlos-variables}

% % %% Unicode math and Polyglossia
% \usepackage{unicode-math}
% \usepackage{unicode-minionmath}

% \setmainfont[Ligatures=TeX]{Libertinus Serif}
% \setsansfont{Libertinus Sans}
% \setmonofont{Libertinus Mono}
% \setmathfont{Minion Math}
% \setmathfont[range={\mathfrak}]{XITS Math}
% \setmathfont[range={\mathcal},StylisticSet=1]{XITS Math}
% \setmathfont[range={\mathscr}]{XITS Math}
% \setmathfont[range={}]{Minion Math}

% \usepackage{polyglossia}

% \newfontfamily\cyrillicfont[Script=Cyrillic]{Libertinus Serif}
% \newfontfamily\cyrillicfontsf[Script=Cyrillic]{Libertinus Sans}

% \newfontfamily\farsifont[Script=Arabic,
%                          Scale=MatchUppercase]{Amiri}

% \setmainlanguage[variant=american]{english}
% \setotherlanguage{farsi}
% \setotherlanguage{french}
% \setotherlanguage[spelling=new,latesthyphen,babelshorthands]{german}
% \setotherlanguage{spanish}
% \setotherlanguage[spelling=modern,babelshorthands]{russian}

% \usepackage{xeCJK}
% \usepackage[overlap]{ruby}
% \renewcommand\rubysep{-0.2ex}
% \xeCJKDeclareSubCJKBlock{Kana}{"3040 -> "309F, "30A0 -> "30FF, "31F0 -> "31FF, "1B000 -> "1B0FF}
% \xeCJKDeclareSubCJKBlock{Hangul}{"1100 -> "11FF, "3130 -> "318F, "A960 -> "A97F, "AC00 -> "D7AF, "D7B0 -> "D7FF}

% \setCJKmainfont{HanaMinA}
% \setCJKmainfont[Kana]{HanaMinA}
% \setCJKmainfont[Hangul]{NanumMyeongjo}
% \setCJKsansfont[Hangul]{NanumGothic}

%% Theorems and definitions
\theoremstyle{plain}
\newtheorem{theorem}{Theorem}
\newtheorem{proposition}[theorem]{Proposition}
\newtheorem{corollary}[theorem]{Corollary}
\newtheorem{claim}[theorem]{Claim}
\newtheorem{lemma}[theorem]{Lemma}
\newtheorem{axiom}[theorem]{Axiom}

\newtheorem*{corollary*}{Corollary}
\newtheorem*{claim*}{Claim}
\newtheorem*{lemma*}{Lemma}
\newtheorem*{proposition*}{Proposition}
\newtheorem*{theorem*}{Theorem}

\theoremstyle{definition}
\newtheorem{definition}{Definition}
\newtheorem{example}{Examples}
\newtheorem{examples}[example]{Examples}
\newtheorem{exercise}{Exercise}
\newtheorem{problem}[exercise]{Problem}

\newtheorem*{definition*}{Definition}
\newtheorem*{example*}{Examples}
\newtheorem*{examples*}{Examples}
\newtheorem*{exercise*}{Exercise}
\newtheorem*{problem*}{Problem}

\theoremstyle{remark}
\newtheorem{remark}{Remark}
\newtheorem{remarks}[remark]{Remarks}
\newtheorem{observation}[remark]{Observation}
\newtheorem{observations}[remark]{Observations}

\newtheorem*{remark*}{**Remark**}
\newtheorem*{remarks*}{**Remarks**}
\newtheorem*{observation*}{**Observation**}
\newtheorem*{observations*}{**Observations**}

\DeclareMathOperator{\proj}{proj}
\DeclareMathOperator{\comp}{comp}

\begin{document}
\author{TA: \href{mailto:\authoremail}{\documentauthor}}
\title{\documenttitle}
\date{\today}
\maketitle
Students, I have typed up the solutions to some of the exercises in the
review sheet with some commentary for your benefit. It's possible that I
will not get to every last problem on the review sheet, but it may be worth
while for you to peruse these solutions.

\begin{problem*}[26]
Which of the following series converge?
\[
\text{(i)}\sum_{n=1}^\infty\frac{(-1)^n}{n^{1/4}}\qquad
\text{(ii)}\sum_{n=1}^\infty\frac{n!}{1\cdot 3\cdot 5\dotsm (2n-1)}\qquad
\text{(iii)}\sum_{n=1}^\infty\frac{4}{3}\left(\frac{1}{2}\right)^n.
\]
\end{problem*}
\begin{proof}[Solution]
Kevin, you said that for (i) and (iii) it is easy to verify that they
converge right? So let's focus on (ii) here. I claim that you can show the
series
\[
\sum_{n=1}^\infty\frac{n!}{1\cdot 3\cdot 5\dotsm (2n-1)}
\]
converges by applying the
\href{https://en.wikipedia.org/wiki/Ratio_test}{ratio test}. Here's how you
do it. Just to get the notation straight, set $a_n\coloneqq n!/(1\cdot
3\cdot 5\dotsm (2n-1))$. Then, for the ratio test, we need to check that
the limit of the quotient $|a_{n+1}/a_n|$ converges to some number which is
less than $1$. Let's show this:
\begin{align*}
\left|\frac{a_{n+1}}{a_n}\right|
&=\left|
\frac{(n+1)!/(1\cdot 3\cdot 5\dotsm (2(n+1)-1))}
{n!/(1\cdot 3\cdot 5\dotsm (2n-1))}\right|
\shortintertext{since the terms are all positive, we can remove the
  absolute value signs}
&=
\frac{(n+1)!/(1\cdot 3\cdot 5\dotsm (2(n+1)-1))}
{n!/(1\cdot 3\cdot 5\dotsm (2n-1))}
\shortintertext{rearrange it (remember that $1/(1/x)=x$, i.e., the inverse
  of the inverse of something is the original thing you started with, for
  example if you invert $2$ you get $1/2$, if you invert $1/2$ you get $2$,
  some people still get confused about this; I just want to make sure you
  are on board with what I am about to do)}
&=\frac{(n+1)!(1\cdot 3\cdot 5\dotsm(2n-1))}
{n!(1\cdot 3\cdot 5\dotsm(2(n+1)-1))}
\shortintertext{remember that $n!$ is defined recursively for example
  $5!=5\cdot 4\cdot 3\cdot 2\cdot 1$, but what is $4\cdot 3\cdot 2\cdot 1$?
  It's nothing other than $4!$, so $5!=5\cdot 4!$. In particular,
  $(n+1)!=(n+1)\cdot n!$ so we have}
&=\frac{(n+1)\cdot n!(1\cdot 3\cdot 5\dotsm(2n-1))}
{n!(1\cdot 3\cdot 5\dotsm(2(n+1)-1))}
\shortintertext{the $n!$ in the denominator then cancels the $n!$ in the
  numerator, and we have}
&=\frac{(n+1)(1\cdot 3\cdot 5\dotsm(2n-1))}
{(1\cdot 3\cdot 5\dotsm(2(n+1)-1))}\\
\shortintertext{Now, see those $1$'s, $3$'s, and $5$'s in the numerator and
  denominator? We are going to cancel them. But when do we stop? Well, if
  you see the pattern, we are taking all of the odd numbers up to
  $2n$. Now, let's rewrite the equation we got above by expanding the
  $2(n+1)-1$ to $2n+2-1=2n+1$}
&=\frac{(n+1)(1\cdot 3\cdot 5\dotsm(2n-1))}
{(1\cdot 3\cdot 5\dotsm(2n+1))}
\shortintertext{What is the odd number preceding $2n+1$? Well, if we remove
  $1$ from $2n+1$, we get $2n$ which is even, so we must take $2$ from
  $2n+1$ to get the odd number just before $2n+1$, this is
  $2n+1-2=2n-1$. This gives us}
&=\frac{(n+1)(1\cdot 3\cdot 5\dotsm(2n-1))}
{(1\cdot 3\cdot 5\dotsm(2n-1)\cdot(2n+1))}\\
\shortintertext{Now, it gets easy. See how we have $1\cdot 3\cdot
  5\dotsm(2n-1)$ on the top and the bottom? We can cancel these and we have}
&=\frac{n+1}{2n+1}
\shortintertext{What is the limit of this sequence as $n\to\infty$? Use
  l'Hôpital's rule; replace $n$ by $x$ and differentiate}
&=\frac{x+1}{2x+1}
\end{align*}
Now, by l'Hôpital's rule
\begin{align*}
\lim_{n\to\infty}\left|\frac{a_{n+1}}{a_n}\right|
&=\lim_{n\to\infty}\frac{x+1}{2x+1}\\
&=\lim_{x\to\infty}\frac{(x+1)'}{(2x+1)'}\\
&=\lim_{x\to\infty}\frac{1}{2}\\
&=\frac{1}{2}
\end{align*}
Thus, the limit is $1/2<1$ so the series
\[
\sum_{n=1}^\infty\frac{n!}{1\cdot 3\cdot 5\dotsm (2n-1)}
\]
converges.
\end{proof}
\end{document}

%%% Local Variables:
%%% mode: latex
%%% TeX-master: t
%%% End:
