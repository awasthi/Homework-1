\chapter{Exercises Week 2}
Interesting exercises (by that, the professor usually means, tricky or had
exercises).
\begin{exercise}[Stewart {\S}12.3, 31]
Find the acute angles at the points of intersection
\[
y=x^2,\quad y=x^3.
\]
\end{exercise}
\begin{proof}[Solution]
Remember that the angle between two curves at a point of intersection is
the angle made by the tangent lines to those curves at that point.

Let me explain how to do this. In the given exercise we are asked to find
the acute angles at the points of intersection of the curves $y=x^2$ and
$y=x^3$. We first need to find the points where they intersect, i.e., where
\begin{equation}
\label{eq:ex-12-3-31-1}
x^3=x^2.
\end{equation}
A little algebra and we can turn (\ref{eq:ex-12-3-31-1}) into
$0=x^3-x^2=x^2(x-1)$ so the solutions to (\ref{eq:ex-12-3-31-1}) are $x=0$
or $x=1$. Now we need to find the tangent line to the curve $y=x^3$ and
$y=x^2$ at $x=0$ and $x=1$.

Remember that to find the tangent line of a function $f(x)$ whose
derivative exists, at a point $x_0$ we have to first, compute its
derivative, and second, compute $f(x_0)$. After that, the tangent line of
$f(x)$ at $x_0$ will have the form
\[
y-f(x_0)=f'(x_0)(x-x_0).
\]

To avoid confusing $y=x^3$, $y=x^2$ with their tangent lines, let's call
the first one $f_1(x)$ and the second one $f_2(x)$; I mean, $f_1(x)=x^3$
and $f_2(x)=x^2$; this is function notation and you should be familiar with
it.

Moving on, we'll start with the intersection at $x=1$. At $x=1$, $f_1(1)=1$
and $f_2(1)=1$. Moreover, the derivatives of $f_1$ and $f_2$ are
\begin{align*}
f_1'(x)&=3x^2&f_2'(x)&=2x\\
f_1'(1)&=3\cdot 1^2&f_2'(1)&=2\cdot 1.
\end{align*}
Hence, $f_1'(1)=3$ and $f_2'(1)=2$. So the tangent lines to $f_1$ and $f_2$
at $1$ are
\begin{align}
\label{eq:ex-12-3-31-2}
y-1&=3(x-1)&y-1&=2(x-1)\nonumber\\
\intertext{which, in more standard form, is}
y&=3x-2&y&=2x-1.
\end{align}

As if that wasn't enough we still have to find the $y$-intercept for these
tangent lines and find the vector pointing from the their $y$-intercept to
the point of intersection. To find the $y$-intercept of the tangent lines
in \ref{eq:ex-12-3-31-2} we simply plug in $0$ for $x$ and we have $y=-2$
and $y=-1$ so the $y$-intercepts are $(0,-2)$ and $(0,-1)$ for the tangent
line of $f_1(x)$ at $1$ and the tangent line of $f_2(x)$ at $1$,
respectively. Then, we have the vectors $\langle 1,1\rangle-\langle
0,-2\rangle=\langle1,3\rangle$ and $\langle 1,1\rangle-\langle
0,-1\rangle=\langle 1,2\rangle$ so we compute the angle between them by the
dot-product formula, i.e.,
\[
\cos\theta=\frac{\langle\langle1,3\rangle\cdot\langle 1,2\rangle\rangle}
                {|\langle1,3\rangle|\cdot |\langle
                  1,2\rangle|}=\frac{1+6}{\sqrt{10}\cdot\sqrt{5}}=\frac{7}{5\sqrt{2}}.
\]
Now just take $\cos^{-1}$ of both sides and we have
$\theta=\tan^{-1}(7/5\sqrt{2})\approx 8.13^\circ$.
\end{proof}

%%% Local Variables:
%%% mode: latex
%%% TeX-master: "../MA166-Recitation"
%%% End:
