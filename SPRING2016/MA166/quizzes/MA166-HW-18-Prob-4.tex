\def\documentauthor{Carlos Salinas}
\def\documenttitle{MA 166: HW {\hwnum} Problem 4 Solution}
\def\hwnum{18}
\def\shorttitle{MA 166 HW {\hwnum} Sol}
\def\coursename{MA166}
\def\documentsubject{calculus ii}
\def\authoremail{salinac@purdue.edu}

\documentclass[10pt]{article}
\usepackage{geometry}
\usepackage[dvipsnames]{xcolor}
\usepackage[
    breaklinks,
    bookmarks=true,
    % colorlinks=true,
    pageanchor=false,
    linkcolor=black,
    anchorcolor=black,
    citecolor=black,
    filecolor=black,
    menucolor=black,
    runcolor=black,
    urlcolor=black,
    hyperindex=false,
    hyperfootnotes=true,
    pdftitle={\shorttitle},
    pdfauthor={\documentauthor},
    pdfkeywords={\documentsubject},
    pdfsubject={\coursename}
    ]{hyperref}

% Use symbols instead of numbers
\renewcommand*{\thefootnote}{\fnsymbol{footnote}}

%% Math
\usepackage{amsmath}
\usepackage{amsthm}
\usepackage{amssymb}
\usepackage{mathtools}

%%PDFTeX specific
\usepackage[mathcal]{euscript}
\usepackage{mathrsfs}
\usepackage{dsfont}
\usepackage{wasysym}

\usepackage[LAE,LFE,T2A,T1]{fontenc}
\usepackage[utf8]{inputenc}
\usepackage[farsi,french,german,spanish,russian,english]{babel}
\babeltags{pa=farsi,
           fr=french,
           de=german,
           es=spanish,
           ru=russian,
           en=english}
\def\spanishoptions{mexico}

\selectlanguage{english}

\newcommand{\textfa}[1]{\beginR\textpa{#1}\endR}

\usepackage{cmap}
\usepackage{CJKutf8}
\newcommand{\textkr}[1]{\begin{CJK}{UTF8}{mj}#1\end{CJK}}
\newcommand{\textjp}[1]{\begin{CJK}{UTF8}{min}#1\end{CJK}}
\newcommand{\textzh}[1]{\begin{CJK}{UTF8}{bsmi}#1\end{CJK}}

%% Misc
\usepackage{graphicx}
\usepackage{cutwin}
\graphicspath{{figures/}}

\usepackage{microtype}
\usepackage{multicol}
\usepackage[inline]{enumitem}
\usepackage{listings}
\usepackage{mleftright}
\mleftright

%% Theorems and definitions
\theoremstyle{plain}
\newtheorem{theorem}{Theorem}
\newtheorem{proposition}[theorem]{Proposition}
\newtheorem{corollary}[theorem]{Corollary}
\newtheorem{claim}[theorem]{Claim}
\newtheorem{lemma}[theorem]{Lemma}
\newtheorem{axiom}[theorem]{Axiom}

\newtheorem*{corollary*}{Corollary}
\newtheorem*{claim*}{Claim}
\newtheorem*{lemma*}{Lemma}
\newtheorem*{proposition*}{Proposition}
\newtheorem*{theorem*}{Theorem}

\theoremstyle{definition}
\newtheorem{definition}{Definition}
\newtheorem{example}{Examples}
\newtheorem{examples}[example]{Examples}
\newtheorem{exercise}{Exercise}
\newtheorem{problem}[exercise]{Problem}

\newtheorem*{definition*}{Definition}
\newtheorem*{example*}{Examples}
\newtheorem*{examples*}{Examples}
\newtheorem*{exercise*}{Exercise}
\newtheorem*{problem*}{Problem}

\theoremstyle{remark}
\newtheorem{remark}{Remark}
\newtheorem{remarks}[remark]{Remarks}
\newtheorem{observation}[remark]{Observation}
\newtheorem{observations}[remark]{Observations}

\newtheorem*{remark*}{**Remark**}
\newtheorem*{remarks*}{**Remarks**}
\newtheorem*{observation*}{**Observation**}
\newtheorem*{observations*}{**Observations**}

%% Commands and operators
%% Redefinitions & commands
\newcommand{\nsubset}{\ensuremath{\not\subset}}
\newcommand{\nsupset}{\ensuremath{\not\supset}}
\newcommand\minus{\ensuremath{\null\smallsetminus}}
\renewcommand\qedsymbol{\ensuremath{\null\hfill\smiley}}

%% Commands and operators
\DeclareMathOperator{\id}{id}
\DeclareMathOperator{\im}{im}

%% Linear algebra
\DeclareMathOperator{\proj}{proj}
\DeclareMathOperator{\comp}{comp}

%% Differential operators
\DeclareMathOperator{\Curl}{curl}
\DeclareMathOperator{\Div}{div}
\DeclareMathOperator{\Grad}{grad}
\DeclareMathOperator{\Lap}{\Delta}
\DeclareMathOperator{\diff}{d\!}

%% Misc
\newcommand{\bbC}{\mathbb{C}}
\newcommand{\bbCP}{\mathbb{CP}}
\newcommand{\bbH}{\mathbb{H}}
\newcommand{\bbN}{\mathbb{N}}
\newcommand{\bbQ}{\mathbb{Q}}
\newcommand{\bbR}{\mathbb{R}}
\newcommand{\bbRP}{\mathbb{RP}}
\newcommand{\bbZ}{\mathbb{Z}}

\newcommand{\bfC}{\mathbf{C}}
\newcommand{\bfCP}{\mathbf{CP}}
\newcommand{\bfH}{\mathbf{H}}
\newcommand{\bfN}{\mathbf{N}}
\newcommand{\bfQ}{\mathbf{Q}}
\newcommand{\bfR}{\mathbf{R}}
\newcommand{\bfRP}{\mathbf{RP}}
\newcommand{\bfZ}{\mathbf{Z}}

\newcommand{\calA}{\mathcal{A}}
\newcommand{\calB}{\mathcal{B}}
\newcommand{\calC}{\mathcal{C}}
\newcommand{\calS}{\mathcal{S}}
\newcommand{\calT}{\mathcal{T}}
\newcommand{\calU}{\mathcal{U}}
\newcommand{\calV}{\mathcal{V}}

\newcommand{\scrL}{\mathscr{L}}
\newcommand{\scrO}{\mathscr{O}}
\newcommand{\scrS}{\mathscr{S}}

\newcommand{\bfa}{\mathbf{a}}
\newcommand{\bfb}{\mathbf{b}}
\newcommand{\bfc}{\mathbf{c}}
\newcommand{\bfu}{\mathbf{u}}
\newcommand{\bfv}{\mathbf{v}}
\newcommand{\bfw}{\mathbf{w}}

\begin{document}
\author{TA: \href{mailto:\authoremail}{\documentauthor}}
\title{\documenttitle}
\date{\today}
\maketitle
\begin{problem}[HW \#18, \#4]
Find the centroid of the region bounded by the given curves.
\[
y=6\sin 4x,\qquad y=6\cos 4x,\qquad x=0,\qquad x=\pi/16.
\]
\end{problem}
\begin{proof}[Solution]
So I made a mistake when I calculated $M_x$ on the board, but this should
be correct now. Hope that you can adapt it to your specific problem.

First we find the area
\begin{align*}
A&=\int_0^{\pi/16}6\cos 4x-6\sin 4x\;dx\\
 &=6\int_0^{\pi/16}\cos 4x-\sin 4x\;dx\\
 &=6\left[\frac{\sin 4x+\cos 4x}{4}\right]_0^{\pi/16}\\
 &=\frac{6}{4}\left[\frac{\sin 4x+\cos 4x}{4}\right]_0^{\pi/16}\\
 &=\frac{3}{2}\left(\sin \left(4\pi/16\right)+
   \cos\left(4\pi/16\right)-(\sin 4\cdot 0+\cos 4\cdot
   0)\right)\\
 &=\frac{3}{2}\left(\sin \pi/4+\cos \pi/4-\sin 0 -\cos 0\right)\\
 &=\frac{3}{2}\left(\frac{\sqrt{2}}{2}+\frac{\sqrt{2}}{2}-0-1\right)\\
 &=\boxed{\frac{3}{2}\left(\sqrt{2}-1\right).}
\end{align*}

Now we find $\bar x$ and $\bar y$
\begin{align*}
\bar x&=\frac{1}{A}\int_0^{\pi/16}x(6\cos 4x-6\sin 4x)\;dx\\
      &=\frac{6}{A}\int_0^{\pi/16}x\cos +x-x\sin 4x\;dx
\intertext{Now use integration by parts or tabular integration to get that
        $\int x\cos x=(x\sin 4x)/4+(\cos 4x)/16$ and $\int -x\sin 4x=(x\cos
        4x)/4-(\sin 4x)/16$}
      &=\frac{6}{A}\left[\frac{x\sin 4 x}{4}+\frac{\cos 4x}{16}+\frac{x\cos
        4x}{4}-\frac{\sin 4x}{16}\right]_0^{\pi/16}\\
      &=\frac{6}{A}\left[\frac{1}{4}\left(x\sin 4 x+\frac{\cos
        4x}{4}+x\cos 4x-\frac{\sin 4x}{4}\right)\right]_0^{\pi/16}\\
      &=\frac{3}{2A}\left[x\sin 4x +\frac{\cos
        4x}{4}+x\cos 4x-\frac{\sin 4x}{4}\right]_0^{\pi/16}\\
      &=\frac{3}{2A}\left[x(\sin 4x+\cos 4x)-\frac{1}{4}(\cos 4x-\sin
        4x)\right]_0^{\pi/16}\\
      &=\frac{3}{2\bigl(3/2(\sqrt{2}-1)\bigr)}\left[x(\sin 4x+\cos
        4x)-\frac{1}{4}(\cos 4x-\sin 4x)\right]_0^{\pi/16}\\
      &=\frac{1}{\sqrt{2}-1}\left(\frac{\pi\sqrt{2}}{16}-\frac{1}{4}\right)\\
      &=\frac{1}{\sqrt{2}-1}\left(\frac{\pi\sqrt{2}}{16}-\frac{4}{16}\right)\\
      &=\frac{1}{\sqrt{2}-1}\frac{\pi\sqrt{2}-4}{16}\\
      &=\boxed{\frac{\pi\sqrt{2}-4}{16\bigl(\sqrt{2}-1\bigr)}}
\end{align*}
and
\begin{align*}
\bar y&=\frac{1}{A}\int_0^{\pi/16}\frac{(6\cos 4x)^2-(6\sin 4x)^2}{2}\;dx\\
      &=\frac{1}{A}\int_0^{\pi/16}\frac{36\cos^2 4x-36\sin^2 4x}{2}\;dx\\
      &=\frac{1}{A}\int_0^{\pi/16}\frac{36}{2}\left(\cos^2 4x-\sin^2 4x\right)\;dx\\
      &=\frac{18}{A}\int_0^{\pi/16}\cos^2 4x-\sin 4x^2\;dx\\
\intertext{use the identity $\cos^2\theta-\sin^2\theta=\cos 2\theta$}
      &=\frac{18}{A}\int_0^{\pi/16}\cos 8x\;dx\\
      &=\frac{18}{A}\left[\frac{\sin 8x}{8}\right]_0^{\pi/16}\\
      &=\frac{18}{(3/2)\bigl(\sqrt{2}-1\bigr)}\left(\frac{\sin(8(\pi/16))}{8}-\frac{\sin
       (8\cdot 0)}{8}\right)\\
      &=\frac{18}{(3/2)\bigl(\sqrt{2}-1\bigr)}\left(\frac{\sin(\pi/2)}{8}-\frac{\sin
        0}{8}\right)\\
      &=\frac{18}{(3/2)\bigl(\sqrt{2}-1\bigr)}\left(\frac{1}{8}-0\right)\\
      &=\frac{2\cdot 3\cdot 3}{(3/2)2\cdot 2\cdot
        2\bigl(\sqrt{2}-1\bigr)}\\
      &=\boxed{\frac{3}{2\bigl(\sqrt{2}-1\bigr)}.}
\end{align*}
\end{proof}
\end{document}
%%% Local Variables:
%%% mode: latex
%%% TeX-master: t
%%% End:
