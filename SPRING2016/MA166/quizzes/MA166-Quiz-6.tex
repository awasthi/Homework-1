\def\documentauthor{Carlos Salinas}
\def\documenttitle{MA 166: Quiz {\hwnum} Solutions}
\def\hwnum{6}
\def\shorttitle{MA 166 HW {\hwnum} Sols}
\def\coursename{MA166}
\def\documentsubject{calculus ii}
\def\authoremail{salinac@purdue.edu}

\documentclass[12pt]{article}
\usepackage{geometry}
\usepackage[dvipsnames]{xcolor}
\usepackage[
    breaklinks,
    bookmarks=true,
    % colorlinks=true,
    pageanchor=false,
    linkcolor=black,
    anchorcolor=black,
    citecolor=black,
    filecolor=black,
    menucolor=black,
    runcolor=black,
    urlcolor=black,
    hyperindex=false,
    hyperfootnotes=true,
    pdftitle={\shorttitle},
    pdfauthor={\documentauthor},
    pdfkeywords={\documentsubject},
    pdfsubject={\coursename}
    ]{hyperref}

% Use symbols instead of numbers
\renewcommand*{\thefootnote}{\fnsymbol{footnote}}

%% Math
\usepackage{amsmath}
\usepackage{amsthm}
\usepackage{amssymb}
\usepackage{mathtools}

%%PDFTeX specific
\usepackage[mathcal]{euscript}
\usepackage{mathrsfs}
\usepackage{dsfont}
\usepackage{wasysym}

\usepackage[LAE,LFE,T2A,T1]{fontenc}
\usepackage[utf8]{inputenc}
\usepackage[farsi,french,german,spanish,russian,english]{babel}
\babeltags{pa=farsi,
           fr=french,
           de=german,
           es=spanish,
           ru=russian,
           en=english}
\def\spanishoptions{mexico}

\selectlanguage{english}

\newcommand{\textfa}[1]{\beginR\textpa{#1}\endR}

\usepackage{cmap}
\usepackage{CJKutf8}
\newcommand{\textkr}[1]{\begin{CJK}{UTF8}{mj}#1\end{CJK}}
\newcommand{\textjp}[1]{\begin{CJK}{UTF8}{min}#1\end{CJK}}
\newcommand{\textzh}[1]{\begin{CJK}{UTF8}{bsmi}#1\end{CJK}}

%% Misc
\usepackage{graphicx}
\usepackage{cutwin}
\graphicspath{{figures/}}

\usepackage{microtype}
\usepackage{multicol}
\usepackage[inline]{enumitem}
\usepackage{listings}
\usepackage{mleftright}
\mleftright

%% Theorems and definitions
\theoremstyle{plain}
\newtheorem{theorem}{Theorem}
\newtheorem{proposition}[theorem]{Proposition}
\newtheorem{corollary}[theorem]{Corollary}
\newtheorem{claim}[theorem]{Claim}
\newtheorem{lemma}[theorem]{Lemma}
\newtheorem{axiom}[theorem]{Axiom}

\newtheorem*{corollary*}{Corollary}
\newtheorem*{claim*}{Claim}
\newtheorem*{lemma*}{Lemma}
\newtheorem*{proposition*}{Proposition}
\newtheorem*{theorem*}{Theorem}

\theoremstyle{definition}
\newtheorem{definition}{Definition}
\newtheorem{example}{Examples}
\newtheorem{examples}[example]{Examples}
\newtheorem{exercise}{Exercise}
\newtheorem{problem}[exercise]{Problem}

\newtheorem*{definition*}{Definition}
\newtheorem*{example*}{Examples}
\newtheorem*{examples*}{Examples}
\newtheorem*{exercise*}{Exercise}
\newtheorem*{problem*}{Problem}

\theoremstyle{remark}
\newtheorem{remark}{Remark}
\newtheorem{remarks}[remark]{Remarks}
\newtheorem{observation}[remark]{Observation}
\newtheorem{observations}[remark]{Observations}

\newtheorem*{remark*}{**Remark**}
\newtheorem*{remarks*}{**Remarks**}
\newtheorem*{observation*}{**Observation**}
\newtheorem*{observations*}{**Observations**}

%% Commands and operators
%% Redefinitions & commands
\newcommand{\nsubset}{\ensuremath{\not\subset}}
\newcommand{\nsupset}{\ensuremath{\not\supset}}
\newcommand\minus{\ensuremath{\null\smallsetminus}}
\renewcommand\qedsymbol{\ensuremath{\null\hfill\smiley}}

%% Commands and operators
\DeclareMathOperator{\id}{id}
\DeclareMathOperator{\im}{im}

%% Linear algebra
\DeclareMathOperator{\proj}{proj}
\DeclareMathOperator{\comp}{comp}

%% Differential operators
\DeclareMathOperator{\Curl}{curl}
\DeclareMathOperator{\Div}{div}
\DeclareMathOperator{\Grad}{grad}
\DeclareMathOperator{\Lap}{\Delta}
\DeclareMathOperator{\diff}{d\!}

%% Misc
\newcommand{\bbC}{\mathbb{C}}
\newcommand{\bbCP}{\mathbb{CP}}
\newcommand{\bbH}{\mathbb{H}}
\newcommand{\bbN}{\mathbb{N}}
\newcommand{\bbQ}{\mathbb{Q}}
\newcommand{\bbR}{\mathbb{R}}
\newcommand{\bbRP}{\mathbb{RP}}
\newcommand{\bbZ}{\mathbb{Z}}

\newcommand{\bfC}{\mathbf{C}}
\newcommand{\bfCP}{\mathbf{CP}}
\newcommand{\bfH}{\mathbf{H}}
\newcommand{\bfN}{\mathbf{N}}
\newcommand{\bfQ}{\mathbf{Q}}
\newcommand{\bfR}{\mathbf{R}}
\newcommand{\bfRP}{\mathbf{RP}}
\newcommand{\bfZ}{\mathbf{Z}}

\newcommand{\calA}{\mathcal{A}}
\newcommand{\calB}{\mathcal{B}}
\newcommand{\calC}{\mathcal{C}}
\newcommand{\calS}{\mathcal{S}}
\newcommand{\calT}{\mathcal{T}}
\newcommand{\calU}{\mathcal{U}}
\newcommand{\calV}{\mathcal{V}}

\newcommand{\scrL}{\mathscr{L}}
\newcommand{\scrO}{\mathscr{O}}
\newcommand{\scrS}{\mathscr{S}}

\newcommand{\bfa}{\mathbf{a}}
\newcommand{\bfb}{\mathbf{b}}
\newcommand{\bfc}{\mathbf{c}}
\newcommand{\bfu}{\mathbf{u}}
\newcommand{\bfv}{\mathbf{v}}
\newcommand{\bfw}{\mathbf{w}}

\begin{document}
\author{TA: \href{mailto:\authoremail}{\documentauthor}}
\title{\documenttitle}
\date{\today}
\maketitle

You have \textbf{15 minutes} to complete this quiz. You may work in groups,
but you are not allowed to use any other resources.
\\\\
\begin{problem}
Compute \textbf{two} of the following integrals of your choice
\begin{enumerate}[label=(\alph*)]
\item $\displaystyle\int\frac{x^2}{x^2-1}\;dx$
\item $\displaystyle\int\frac{x}{x^2+6x+9}\;dx$
\item $\displaystyle\int\frac{dx}{x^3+x}.$
\end{enumerate}
\end{problem}
\newpage
\section*{Solutions}
(a)  First, rewrite the integral as
\begin{proof}[Solution]
\begin{equation}
\label{eq:rewrite-a}
\int\frac{x^2}{x^2-1}\;dx=
\int\frac{(x^2-1)+1}{x^2-1}\;dx=
\underbrace{\int 1\;dx}_{I_1}-\underbrace{\int\frac{dx}{x^2-1}}_{I_2}.
\end{equation}
It's easy to calculate $I_1=x+C_1$. To calculate $I_2$ we need to use
partial fractions. Write
\begin{align*}
\frac{1}{x^2-1}
&=\frac{1}{(x-1)(x+1)}\\
&=\frac{A}{x-1}+\frac{B}{x+1}.
\end{align*}
Now we clear denominators
\begin{align*}
1&=\frac{(x-1)(x+1)}{(x-1)(x+1)}\\
&=\frac{A}{x-1}(x-1)(x+1)+\frac{B}{(x+1)}(x-1)(x+1)\\
&=\frac{A}{x-1}(x-1)(x+1)+\frac{B}{(x+1)}(x-1)(x+1)\\
&=A(x+1)+B(x-1)\\
0x+1&=(A+B)x+(A-B)
\end{align*}
and we have $A+B=0$, $A-B=1$ so $A=-B$ and $B=-1/2$, $A=1/2$. Hence, we
have
\begin{align*}
I_2
&=\int\frac{dx}{x^2-1}\\
&=\int\frac{1/2}{x-1}\;dx-\int\frac{1/2}{x+1}\;dx\\
&=\frac{1}{2}\ln|x-1|-\frac{1}{2}\ln|x+1|+C_2\\
&=\frac{1}{2}\ln\left|\frac{x-1}{x+1}\right|+C_2.
\end{align*}
Writing $C=C_1-C_2$ and putting $I_1$ and $I_2$, i.e, taking the difference
as in \eqref{eq:rewrite-a} we have
\[
\boxed{I_1-I_2=x-\frac{1}{2}\ln\left|\frac{x-1}{x+1}\right|+C.}
\]
By using log rules, you can also write this as
\[
x+\frac{1}{2}\ln\left|\frac{x+1}{x-1}\right|+C
=x+\ln\left|\sqrt{\frac{x+1}{x-1}}\right|+C
\]
and so on.
\\\\
(b) Begin by factoring the denominator
\begin{equation}
  \label{eq:factor-denom-b}
x^2+6x+9=(x+3)^2.
\end{equation}
Now use partial fractions
\begin{align*}
\frac{x}{(x+3)^2}&=\frac{A}{x+3}+\frac{B}{(x+3)^2}\\
x&=A(x+3)+B\\
x+0&=Ax+B+3.
\end{align*}
This tells us that $B=-3$ and $A=1$ so the integral turns into
\begin{align*}
\int\frac{x}{x^2+6x+9}\;dx
&=\int\frac{1}{x+3}-\frac{3}{(x+3)^2}\;dx\\
&=\boxed{\ln|x+3|+\frac{3}{x+3}+C.}
\end{align*}

Another way you could have done this problem is by noting that
\[
\frac{x}{(x+3)^2}=\frac{(x+3)-3}{(x+3)^2}=\frac{1}{x+3}-\frac{3}{(x+3)^2}
\]
and you immediately have the partial fraction decomposition, but you can't
always pull this trick on quotients of polynomials. I should be careful
with what I am saying, you can do this, but many times it's much much
messier than this systematic method of partial fractions.
\\\\
(c) Factor the denominator into
\begin{equation}
\label{eq:factor-denom-c}
x^3+x=x(x^2+1).
\end{equation}
Since we cannot factor $x^2+1$ into a product of real numbers, we must have
a numerator of $Bx+C$ over its portion of the partial fraction so, using
partial fractions, we have
\begin{align*}
\frac{1}{x(x^2+1)}&=\frac{A}{x}+\frac{Bx+C}{x^2+1}\\
1&=A(x^2+1)+(Bx+C)x\\
&=(A+B)x^2+Cx+A.
\end{align*}
Hence, $A=1$, $C=0$ and $A+B=0$ so $B=-1$ and our integral turns into
\begin{align*}
\int\frac{dx}{x^3+x}
&=\int\frac{1}{x}-\frac{x}{x^2+1}\;dx\\
&=\boxed{\ln|x|-\frac{1}{2}\ln\left(x^2+1\right)+C.}
\end{align*}
That last bit, the integral of $x/(x^2+1)$, can be computed by using the
$u$-substitution $u=x^2+1$. Then $du=2x\;dx$ and we have
\[
\int\frac{x}{x^2+1}\;dx=\frac{1}{2}\int\frac{1}{u}=\frac{1}{2}\ln
u=\frac{1}{2}\ln\left(x^2+1\right).
\]
\end{proof}
\end{document}
%%% Local Variables:
%%% mode: latex
%%% TeX-master: t
%%% End:
