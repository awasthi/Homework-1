\section*{Solutions}
Here are the solutions to the quiz.
\begin{proof}[Solution]
For the following problems, let $t$ be a dummy variable.

(a) Take the limit as $t\to 0$ of the indefinite integral
\begin{align*}
I_1&=\lim_{t\to 0}\int_t^1\frac{dx}{\sqrt{x}}\\
   &=\lim_{t\to 0}\int_t^1x^{-1/2}\;dx\\
   &=\lim_{t\to 0}\left[\frac{1}{2}x^{1/2}\right]_t^1\\
   &=\lim_{t\to 0}\frac{1}{2}-\frac{1}{2}t^{1/2}\\
   &=\frac{1}{2}.
\end{align*}
So the integral converges and its value is $1/2$.
\\\\
(b) Take the limit as $t\to 5$ of the indefinite integral
\begin{align*}
I_2&=\lim_{t\to 5}\int_t^1\frac{dx}{(5-x)^2}\\
   &=\lim_{t\to 5}\int_1^t(5-x)^{-2}\;dx\\
\shortintertext{make the substitution $u=5-x$, $du=-dx$}
   &=\lim_{t\to 5}-\int_{5-t}^4u^{-2}\;du\\
   &=\lim_{t\to 5}\int_{5-t}^4 u^{-2}\;du\\
   &=\lim_{t\to 5}-\left[-u^{-1}\right]_{5-t}^4\\
   &=\lim_{t\to 5}\left[u^{-1}\right]_{5-t}^4\\
   &=\lim_{t\to 5}\frac{1}{4}-\frac{1}{5-t}\\
   &=\frac{1}{4}-\lim_{t\to 5}\frac{1}{5-t}\\
\end{align*}
So the integral does not exist because for any number you can think of $N$,
we can pick a value $t$ such that $(5-t)^{-1}$ is bigger than $N$, in
fact let's see just when $(5-t)^{-1}=N$
\begin{align*}
\frac{1}{5-t}&=N\\
5-t&=\frac{1}{N}\\
t&=5-\frac{1}{N}.
\end{align*}
Now let $M$ be a number bigger than $N$ and set $t=5-1/M$ then
\[
\frac{1}{5-t}=\frac{1}{5-\left(5-1/M\right)}=\frac{1}{1/M}=M>N.
\]
This is what it means for an integral to not converge.
\\\\
(c) Take the limit as $t\to 0$ of the indefinite integral
\begin{align*}
I_3&=\lim_{t\to\infty}\int_0^t xe^{-x^2}\;dx\\
\shortintertext{use the substitution $u=x^2$, $du=2x\;dx$}
   &=\lim_{t^2\to\infty}\frac{1}{2}\int_0^{t^2} e^{-u}\;du\\
   &=\frac{1}{2}\lim_{t^2\to\infty}\int_0^{t^2}e^{-u}\;du\\
   &=\frac{1}{2}\lim_{t^2\to\infty}\left[-e^{-u}\right]_0^{t^2}\\
   &=\frac{1}{2}\lim_{t^2\to\infty}\left[e^{-u}\right]_{t^2}^0\\
   &=\frac{1}{2}\lim_{t^2\to\infty}\left(e^0-e^{t^2}\right)\\
   &=\frac{1}{2}\lim_{t^2\to\infty}\left(1-e^{-t^2}\right)\\
   &=\frac{1}{2}-\frac{1}{2}\left(\lim_{t^2\to\infty}e^{-t^2}\right)\\
   &=\frac{1}{2}
\end{align*}
So the integral converges and its value is $1/2$.
\end{proof}
\begin{proof}[Solution]
Remember the formula for the arc-length of a curve
\begin{equation}
  \label{eq:arclength}
L=\int_a^b\sqrt{1+\left(\frac{dy}{dx}\right)^2}\;dx
\end{equation}
So we need to find the derivative of $y=\ln(x^2-1)$
\[
\frac{d}{dx}\left(\ln\left(x^2-1\right)\right)=\frac{2x}{x^2-1}.
\]
Next we plug this into our equation and we have
\begin{align*}
  L&=\int_2^5\sqrt{1+\left(\frac{dy}{dx}\right)^2}\;dx\\
   &=\int_2^5\sqrt{1+\left(\frac{2x}{x^2-1}\right)^2}\;dx\\
   &=\int_2^5\sqrt{1+\frac{4x^2}{\left(x^2-1\right)^2}}\;dx\\
   &=\int_2^5\sqrt{\frac{\left(x^2-1\right)^2}{\left(x^2-1\right)^2}
     +\frac{4x^2}{\left(x^2-1\right)^2}}\;dx\\
   &=\int_2^5\sqrt{\frac{\left(x^2-1\right)^2+4x^2}{\left(x^2-1\right)^2}}\;dx\\
   &=\int_2^5\sqrt{\frac{x^4-2x^2+1+4x^2}{\left(x^2-1\right)^2}}\;dx\\
   &=\int_2^5\sqrt{\frac{x^4+2x^2+1}{\left(x^2-1\right)^2}}\;dx\\
   &=\int_2^5\sqrt{\frac{\left(x^2+1\right)^2}{\left(x^2-1\right)^2}}\;dx\\
   &=\int_2^5\frac{x^2+1}{x^2-1}\;dx\\
   &=\int_2^5\frac{x^2+1-1+1}{x^2-1}\;dx\\
   &=\int_2^5\frac{x^2-1+1+1}{x^2-1}\;dx\\
   &=\int_2^5\frac{\left(x^2-1\right)+2}{x^2-1}\;dx\\
   &=\int_2^5\frac{x^2-1}{x^2-1}+\frac{2}{x^2-1}\;dx\\
   &=\underbrace{\int_2^51\;dx}_{I_1}+\underbrace{\int_2^5\frac{2}{x^2-1}\;dx}_{I_2}.
\end{align*}
$I_1=3$, $I_2$ requires a bit more work. First note that $x^2-1=(x-1)(x+1)$
so by the partial fractions decomposition we have
\begin{align*}
\frac{2}{(x-1)(x+1)}&=\frac{A}{x-1}+\frac{B}{x+1}\\
  2&=A(x+1)+B(x-1)\\
  0x+2&=(A+B)x(A-B)
\end{align*}
so $A+B=0$ and $A-B=2$ hence $A-(-B)=2A=2$ gives us that $A=1$ and
$B=-1$. Now we can find $I_2$
\begin{align*}
I_2&=\int_2^5\frac{2}{x^2-1}\\
   &=\int_2^5 \frac{1}{x-1}-\frac{1}{x+1}\;dx\\
   &=\left[\ln|x-1|-\ln|x+1|\right]_2^5\\
   &=\left[\ln\left|\frac{x-1}{x+1}\right|\right]_2^5\\
   &=\ln\left|\frac{4}{6}\right|-\ln\left|\frac{1}{3}\right|\\
   &=\ln 2.
\end{align*}
Hence
\[
L=I_1+I_2=\boxed{3+\ln 2.}
\]
\end{proof}

%%% Local Variables:
%%% mode: latex
%%% TeX-master: "../MA166-Quiz"
%%% End:
