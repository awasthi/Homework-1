\section*{Solutions}
Here are the solutions to the quiz.
\begin{proof}[Solution]
For the following problems, let $t$ be a dummy variable.

(a) Take the limit as $t\to 0$ of the indefinite integral
\begin{align*}
I_1&=\lim_{t\to 0}\int_t^1\frac{dx}{\sqrt{x}}\\
   &=\lim_{t\to 0}\int_t^1x^{-1/2}\;dx\\
   &=\lim_{t\to 0}\left[\frac{1}{2}x^{1/2}\right]_t^1\\
   &=\lim_{t\to 0}\frac{1}{2}-\frac{1}{2}t^{1/2}\\
   &=\frac{1}{2}.
\end{align*}
So the integral converges and its value is $1/2$.
\\\\
(b) Take the limit as $t\to 5$ of the indefinite integral
\begin{align*}
I_2&=\lim_{t\to 5}\int_t^1\frac{dx}{(5-x)^2}\\
   &=\lim_{t\to 5}\int_1^t(5-x)^{-2}\;dx\\
\shortintertext{make the substitution $u=5-x$, $du=-dx$}
   &=\lim_{t\to 5}-\int_{5-t}^4u^{-2}\;du\\
   &=\lim_{t\to 5}\int_{5-t}^4 u^{-2}\;du\\
   &=\lim_{t\to 5}-\left[-u^{-1}\right]_{5-t}^4\\
   &=\lim_{t\to 5}\left[u^{-1}\right]_{5-t}^4\\
   &=\lim_{t\to 5}\frac{1}{4}-\frac{1}{5-t}\\
   &=\frac{1}{4}-\lim_{t\to 5}\frac{1}{5-t}\\
\end{align*}
So the integral does not exist because for any number you can think of $N$,
we can pick a value $t$ such that $(5-t)^{-1}$ is bigger than $N$, in
fact let's see just when $(5-t)^{-1}=N$
\begin{align*}
\frac{1}{5-t}&=N\\
5-t&=\frac{1}{N}\\
t&=5-\frac{1}{N}.
\end{align*}

\\\\
(c) Take the limit as $t\to 0$ of the indefinite integral
\begin{align*}
I_3&=\lim_{t\to 0}\int_t^1\frac{dx}{\sqrt{x}}\\
   &=\lim_{t\to 0}\int_t^1x^{-1/2}\;dx\\
   &=\lim_{t\to 0}\left[\frac{1}{2}x^{1/2}\right]_t^1\\
   &=\lim_{t\to 0}\frac{1}{2}-\frac{1}{2}t^{1/2}\\
   &=\frac{1}{2}.
\end{align*}
So the integral converges and its value is $1/2$.
\end{proof}
\begin{proof}[Solution]

\end{proof}

%%% Local Variables:
%%% mode: latex
%%% TeX-master: "../MA166-Quiz"
%%% End:
