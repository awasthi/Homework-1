\def\documentauthor{Carlos Salinas}
\def\documenttitle{MA 166: Exam {\hwnum} Solutions}
\def\hwnum{1}
\def\shorttitle{MA 166 Exam {\hwnum} Sols}
\def\coursename{MA166}
\def\documentsubject{calculus ii}
\def\authoremail{salinac@purdue.edu}

\documentclass{article}
\usepackage{geometry}
\usepackage[dvipsnames]{xcolor}
\usepackage[
    breaklinks,
    bookmarks=true,
    % colorlinks=true,
    pageanchor=false,
    linkcolor=black,
    anchorcolor=black,
    citecolor=black,
    filecolor=black,
    menucolor=black,
    runcolor=black,
    urlcolor=black,
    hyperindex=false,
    hyperfootnotes=true,
    pdftitle={\shorttitle},
    pdfauthor={\documentauthor},
    pdfkeywords={\documentsubject},
    pdfsubject={\coursename}
    ]{hyperref}

% Use symbols instead of numbers
\renewcommand*{\thefootnote}{\fnsymbol{footnote}}

%% Math
\usepackage{amsmath}
\usepackage{amsthm}
\usepackage{amssymb}
\usepackage{mathtools}

%%PDFTeX specific
\usepackage[mathcal]{euscript}
\usepackage{mathrsfs}
\usepackage{dsfont}
\usepackage{wasysym}

\usepackage[LAE,LFE,T2A,T1]{fontenc}
\usepackage[utf8]{inputenc}
\usepackage[farsi,french,german,spanish,russian,english]{babel}
\babeltags{pa=farsi,
           fr=french,
           de=german,
           es=spanish,
           ru=russian,
           en=english}
\def\spanishoptions{mexico}

\selectlanguage{english}

\newcommand{\textfa}[1]{\beginR\textpa{#1}\endR}

\usepackage{cmap}
\usepackage{CJKutf8}
\newcommand{\textkr}[1]{\begin{CJK}{UTF8}{mj}#1\end{CJK}}
\newcommand{\textjp}[1]{\begin{CJK}{UTF8}{min}#1\end{CJK}}
\newcommand{\textzh}[1]{\begin{CJK}{UTF8}{bsmi}#1\end{CJK}}

%% Misc
\usepackage{graphicx}
\usepackage{cutwin}
\graphicspath{{figures/}}

\usepackage{microtype}
\usepackage{multicol}
\usepackage[inline]{enumitem}
\usepackage{listings}
\usepackage{mleftright}
\mleftright

%% Theorems and definitions
\theoremstyle{plain}
\newtheorem{theorem}{Theorem}
\newtheorem{proposition}[theorem]{Proposition}
\newtheorem{corollary}[theorem]{Corollary}
\newtheorem{claim}[theorem]{Claim}
\newtheorem{lemma}[theorem]{Lemma}
\newtheorem{axiom}[theorem]{Axiom}

\newtheorem*{corollary*}{Corollary}
\newtheorem*{claim*}{Claim}
\newtheorem*{lemma*}{Lemma}
\newtheorem*{proposition*}{Proposition}
\newtheorem*{theorem*}{Theorem}

\theoremstyle{definition}
\newtheorem{definition}{Definition}
\newtheorem{example}{Examples}
\newtheorem{examples}[example]{Examples}
\newtheorem{exercise}{Exercise}
\newtheorem{problem}[exercise]{Problem}

\newtheorem*{definition*}{Definition}
\newtheorem*{example*}{Examples}
\newtheorem*{examples*}{Examples}
\newtheorem*{exercise*}{Exercise}
\newtheorem*{problem*}{Problem}

\theoremstyle{remark}
\newtheorem{remark}{Remark}
\newtheorem{remarks}[remark]{Remarks}
\newtheorem{observation}[remark]{Observation}
\newtheorem{observations}[remark]{Observations}

\newtheorem*{remark*}{**Remark**}
\newtheorem*{remarks*}{**Remarks**}
\newtheorem*{observation*}{**Observation**}
\newtheorem*{observations*}{**Observations**}

%% Commands and operators
%% Redefinitions & commands
\newcommand{\nsubset}{\ensuremath{\not\subset}}
\newcommand{\nsupset}{\ensuremath{\not\supset}}
\newcommand\minus{\ensuremath{\null\smallsetminus}}
\renewcommand\qedsymbol{\ensuremath{\null\hfill\blacksquare}}

%% Commands and operators
\DeclareMathOperator{\id}{id}
\DeclareMathOperator{\im}{im}

%% Linear algebra
\DeclareMathOperator{\proj}{proj}
\DeclareMathOperator{\comp}{comp}

%% Differential operators
\DeclareMathOperator{\Curl}{curl}
\DeclareMathOperator{\Div}{div}
\DeclareMathOperator{\Grad}{grad}
\DeclareMathOperator{\Lap}{\Delta}
\DeclareMathOperator{\diff}{d\!}

%% Misc
\newcommand{\bbC}{\mathbb{C}}
\newcommand{\bbCP}{\mathbb{CP}}
\newcommand{\bbH}{\mathbb{H}}
\newcommand{\bbN}{\mathbb{N}}
\newcommand{\bbQ}{\mathbb{Q}}
\newcommand{\bbR}{\mathbb{R}}
\newcommand{\bbRP}{\mathbb{RP}}
\newcommand{\bbZ}{\mathbb{Z}}

\newcommand{\bfC}{\mathbf{C}}
\newcommand{\bfCP}{\mathbf{CP}}
\newcommand{\bfH}{\mathbf{H}}
\newcommand{\bfN}{\mathbf{N}}
\newcommand{\bfQ}{\mathbf{Q}}
\newcommand{\bfR}{\mathbf{R}}
\newcommand{\bfRP}{\mathbf{RP}}
\newcommand{\bfZ}{\mathbf{Z}}

\newcommand{\calA}{\mathcal{A}}
\newcommand{\calB}{\mathcal{B}}
\newcommand{\calC}{\mathcal{C}}
\newcommand{\calS}{\mathcal{S}}
\newcommand{\calT}{\mathcal{T}}
\newcommand{\calU}{\mathcal{U}}
\newcommand{\calV}{\mathcal{V}}

\newcommand{\scrL}{\mathscr{L}}
\newcommand{\scrO}{\mathscr{O}}
\newcommand{\scrS}{\mathscr{S}}

\newcommand{\bfa}{\mathbf{a}}
\newcommand{\bfb}{\mathbf{b}}
\newcommand{\bfc}{\mathbf{c}}
\newcommand{\bfu}{\mathbf{u}}
\newcommand{\bfv}{\mathbf{v}}
\newcommand{\bfw}{\mathbf{w}}

\begin{document}
\author{TA: \href{mailto:\authoremail}{\documentauthor}}
\title{\documenttitle}
\date{\today}
\maketitle
Here are the solutions to Exam 1. I've provided as much detail as I can
muster. If you find any mistake or you feel that I am skipping a step,
please be sure to let me know. The {\color{Green}green booklet} and
{\color{Red}red booklet} are similar so I've highlighted the corresponding
problem number in {\color{Green}green} for the {\color{Green}green booklet}
and in {\color{Red}red} for the {\color{Red}red booklet}.

\begin{problem}[{\color{Green}\#1}, {\color{Red}\#11}]
If $a$ and $b$ are the values of $k$ for which the angle between $\langle
1,2,2 \rangle$ and $\langle 1,0,k \rangle$ equals $\pi/4$, then $a+b=$?
\end{problem}
\begin{proof}[Solution]
All you needed to do for this problem was remember the
\href{https://en.wikipedia.org/wiki/Law_of_cosines#Vector_formulation}{law
  of cosines} for vectors which tells us that
\begin{equation}
  \label{eq:law-of-cosines}
\vec u\cdot\vec v=\left|\vec u\right|\left|\vec v\right|\cos\theta.
\end{equation}
Applying the equation above to our vectors, we get
\begingroup
\allowdisplaybreaks
\begin{align*}
\langle 1,2,2\rangle\cdot\langle 1,0,k\rangle
&=\left(\sqrt{1^2+2^2+2^2}\right)
\left(\sqrt{1^2+0^2+k^2}\right)\cos(\pi/4)\\
1\cdot 1+2\cdot 0+2\cdot k&=\left(\sqrt{1+4+4}\right)
\left(\sqrt{1+0+k^2}\right)1/\sqrt{2}\\
1+2k&=\left(\sqrt{1+4+4}\right)
\left(\sqrt{1+0+k^2}\right)\\
\sqrt{2}(1+2k)&=\sqrt{9}
\left(\sqrt{1+k^2}\right)\\
&=3\sqrt{1+k^2},
\end{align*}
\endgroup
now, squaring both sides, we get
\begingroup
\allowdisplaybreaks
\begin{align}
\label{eq:quadratic-in-k}
2(1+2k)^2&=9\left(1+k^2\right)\nonumber\\
2\left(1+4k+4k^2\right)&=9+9k^2\nonumber\\
2+8k+8k^2&=9+9k^2\nonumber\\
\intertext{now move everything on the right to the left and reorder by
  the highest exponent of $k$}
0&=k^2-8k+7.
\end{align}
\endgroup
Can you see what $a+b$ is already? No? Well consider the following
quadratic polynomial $(x-a)(x-b)$. What are the roots of $(x-a)(x-b)$?
Well, they are $a$ and $b$ of course. Now, expand $(x-a)(x-b)$ like so
\[
(x-a)(x-b)=x^2-ax-bx+ab=x^2-(a+b)x+ab
\]
so $a+b$ is the same as the negative of the coefficient in front of $x$ in
our quadratic polynomial. In this case, it would be $\boxed{a+b=-(-8)=8}$.

Is that not satisfying? Well, we can go ahead and compute the roots of
equation (\ref{eq:quadratic-in-k}) by using the quadratic formula. From
doing that, we get the roots
\begingroup
\allowdisplaybreaks
\begin{align*}
x&=\frac{-(-8)\pm\sqrt{(-8)^2-4\cdot 7}}{2}\\
&=4\pm\sqrt{\frac{8^2-4\cdot 7}{4}}\\
&=4\pm\sqrt{\frac{8^2-4\cdot 7}{4}}\\
&=4\pm\sqrt{\frac{4\cdot 2\cdot 8-4\cdot 7}{4}}\\
&=4\pm\sqrt{2\cdot 8-7}\\
&=4\pm\sqrt{16-7}\\
&=4\pm\sqrt{9}\\
&=4\pm 3
\end{align*}
\endgroup
so $a=7$ and $b=1$. Hence, $\boxed{a+b=8}$ like we said before.
\end{proof}

\begin{problem}[{\color{Green}\#2}, {\color{Red}\#1}]
Let $\langle a,b,c\rangle$ be the vector projection of $\vec u=\langle
2,-1,9\rangle$ onto $\vec v=\langle 1,2,2\rangle$. Compute $a+b+c$.
\end{problem}
\begin{proof}[Solution]
Recall the formula for the projection of $\vec u$ onto $\vec v$:
\begin{equation}
  \label{eq:projection-onto}
\proj_{\vec v}\vec u=\frac{\vec u\cdot \vec v}{\left|\vec v\right|^2}\vec v.
\end{equation}
Plugging in our values of $\vec u$ and $\vec v$ into this equation, we have
\begingroup
\allowdisplaybreaks
\begin{align*}
\proj_{\vec v}\vec u
&=\frac{\langle 2,-1,9\rangle\cdot\langle 1,2,2\rangle}
  {\left|\langle 1,2,2\rangle\right|^2}\langle 1,2,2\rangle\\
&=\frac{2\cdot 1-1\cdot2+9\cdot 2}
{\left(\sqrt{1^2+2^2+2^2}\right)^2}\langle 1,2,2\rangle\\
&=\frac{2-2+18}{\left(\sqrt{9}\right)^2}\langle 1,2,2\rangle\\
&=\frac{18}{9}\langle 1,2,2\rangle\\
&=\text{$2\langle 1,2,2\rangle$ or $\langle 2,4,4\rangle$.}
\end{align*}
\endgroup
So $\boxed{a+b+c=2+4+4=10}$.
\end{proof}

\begin{problem}[{\color{Green}\#3}, {\color{Red}\#3}]
Let $\vec u=\langle 0,1,2\rangle$, $\vec v=\langle 3,1,0\rangle$, and $\vec
w=\langle a,b,c\rangle$. Suppose $\vec w$ is a unit vector with $c>0$ that
is perpendicular to both $\vec u$ and $\vec v$. Compute $a+b+c$.
\end{problem}
\begin{proof}[Solution]
Now, you all remember that to find a vector that is perpendicular to both
$\vec u$ and $\vec v$ all we need to do is find their cross product $\vec
u\times \vec v$, right? Let's start by finding this
\begingroup
\allowdisplaybreaks
\begin{align*}
\vec u\times\vec v
&=\langle 0,1,2\rangle\times\langle 3,1,0\rangle\\
&=\begin{bmatrix}
\hat\imath&\hat\jmath&\hat k\\
0&1&2\\
3&1&0
\end{bmatrix}\\
&=\begin{bmatrix}
1&2\\1&0
\end{bmatrix}\hat\imath
+\begin{bmatrix}
2&0\\
0&3
\end{bmatrix}\hat\jmath
+\begin{bmatrix}
0&1\\
3&1
\end{bmatrix}\hat k\\
&=(1\cdot 0-2\cdot 1)\hat\imath+(2\cdot 3-0\cdot 0)\hat\jmath+(0\cdot
  1-1\cdot 3)\hat k\\
&=-2\hat\imath+6\hat\jmath-3\hat k\\
&=\langle -2,6,-3\rangle.
\end{align*}
\endgroup
We are not quite done yet since we want a unit vector. All we need to do is
divide by the magnitude of $\vec u\times\vec v$ and we are one step closer
to the solution
\begingroup
\allowdisplaybreaks
\begin{align*}
\frac{\vec u\times\vec v}{\left|\vec u\times\vec v\right|}
&=\frac{\langle -2,6,-3\rangle}{\sqrt{(-2)^2+6^2+(-3)^2}}\\
&=\frac{\langle -2,6,-3\rangle}{\sqrt{4+36+9}}\\
&=\frac{\langle -2,6,-3\rangle}{\sqrt{49}}\\
&=\frac{\langle -2,6,-3\rangle}{7}\\
&=\left<-\frac{2}{7},\frac{6}{7},-\frac{3}{7}\right>.
\end{align*}
\endgroup
Notice that the third entry $-3/7$ on the vector above is negative so we
need to take the negative of the vector above and we call this $\vec w$
\[
\vec w
=\langle a,b,c\rangle=-\left<-\frac{2}{7},\frac{6}{7},-\frac{3}{7}\right>
=\left<-\left(-\frac{2}{7}\right),-\frac{6}{7},-\left(-\frac{3}{7}\right)\right>
=\left<\frac{2}{7},-\frac{6}{7},\frac{3}{7}\right>.
\]
Now, all we need to do is take the sum of the entries of the vector
$\vec w$ above and we are done!
\[
\boxed{a+b+c\frac{2}{7}-\frac{6}{7}+\frac{3}{7}=
\frac{2-6+3}{7}=
-\frac{1}{7}.}\qedhere
\]
\end{proof}

\begin{problem}[{\color{Green}\#4}, {\color{Red}\#5}]
Find the area of the region by the curves $y=x^2-2$ and $y=|x|$.
\end{problem}
\begin{proof}[Solution]
Alright! Remember the definition of the absolute value of a function
anyone? Here it is: If $f(x)$ is a function of $x$, i.e., $f(x)=\cos x$ or
$f(x)=e^x+\cos\pi x-x^{\pi}$ or what have you, then
\begin{equation}
  \label{eq:absolute-value}
\left|f(x)\right|=
\begin{cases}
f(x)&\text{if $f(x)\geq 0$}\\
-f(x)&\text{if $f(x)<0$}
\end{cases}.
\end{equation}
All this is saying is that if we plug in a value into our function $f(x)$
and it returns a negative value we make it positive and if the value is
positive we leave it positive. What does this mean for $y=|x|$? It means
that
\[
|x|=\begin{cases}
x&\text{if $f(x)\geq 0$}\\
-x&\text{if $f(x)<0$}
\end{cases},
\]
i.e., $|x|$ is $-x$ from $-\infty$ to $0$ and $x$ from $0$ to $+\infty$,
if this notation makes sense to you. This means that we must consider two
cases when solving for the intersection of $|x|$ with $x^2-2$: We must
consider the possibility that $x^2-2$ intersects $|x|$ for some value $x<0$
and $x^2-2$ intersects $|x|$ for some value $x\geq 0$. So we must solve
both equations
\begingroup
\allowdisplaybreaks
\begin{align*}
x&=x^2-2\\
-x&=x^2-2.
\end{align*}
\endgroup
By some simple algebra, we can rearrange the equations above into
\begingroup
\allowdisplaybreaks
\begin{align}
\label{eq:the-right-abs}
0&=x^2-x-2\\
\label{eq:the-left-abs}
0&=x^2+x-2
\end{align}
\endgroup
and solve for $x$. By the quadratic equation on equation
(\ref{eq:the-right-abs}) we have
\[
x=\frac{-(-1)\pm\sqrt{(-1)^2-4(-2)}}{2}=\frac{1\pm\sqrt{1+8}}{2}=\frac{1\pm
3}{2}=\text{$-1$ or $2$.}
\]
Since we are only looking at positive values of $x$, $-1$ makes no sense
so we throw it out and $2$ remains behind.

We do the same thing for equation (\ref{eq:the-left-abs})
\[
x=\frac{-1\pm\sqrt{1^2-4(-2)}}{2}=\frac{-1\pm\sqrt{1+8}}{2}=\frac{-1\pm
3}{2}=\text{$-2$ or $1$.}
\]
Since we are only only looking at negative values of, $1$ makes no sense so
we throw it out and keep $-2$.

Now all we need to do is observe that for $0\leq x\leq 2$ the equation
$x>x^2-2$ and for $-2\leq x\leq 0$ the equation $-x>x^2-2$ so the area
enclosed by $y=|x|$ and $y=x^2-2$ is given by the integral
\begingroup
\allowdisplaybreaks
\begin{align*}
\int_{-2}^2\left||x|-\left(x^2-2\right)\right|\diff x
&=\int_{-2}^0\left||x|-\left(x^2-2\right)\right|\diff x
+\int_0^2\left||x|-\left(x^2-2\right)\right|\diff x\\
&=\underbrace{\int_{-2}^0-x-\left(x^2-2\right)\diff x}_{\text{Int.\@ 1}}
+\underbrace{\int_0^2x-\left(x^2-2\right)\diff x}_{\text{Int.\@ 2}}.
\end{align*}
\endgroup
Let's compute Int.\@ 1 and Int.\@ 2 separately
\begingroup
\allowdisplaybreaks
\begin{align*}
\text{Int.\@ 1}
&=\int_{-2}^0 -x-\left(x^2-2\right)\diff x\\
&=\int_{-2}^0 -x-x^2+2\diff x\\
&=\int_{-2}^0 -x^2-x+2\diff x\\
&=\left.-\tfrac{1}{3}x^3-\tfrac{1}{2}x^2+2x\right|_{-2}^0\\
&=-\tfrac{1}{3}\cdot 0^3-\tfrac{1}{2}\cdot 0^2+2\cdot 0\\
&\phantom{{}={}}-\left(-\tfrac{1}{3}(-2)^3-\tfrac{1}{2}(-2)^2+2(-2)\right)\\
&=6-\tfrac{8}{3}\\
&=\frac{6\cdot 3-8}{3}\\
&=\frac{18-8}{3}\\
&=\frac{10}{3}.
\end{align*}
\endgroup
Now, you can either compute Int.\@ 2 and add that quantity to Int.\@ 1 to
get the area of your bounded region, or you can plot the curves and notice
that the areas are symmetric and double Int.\@ 2 to get our total area
$\boxed{2(10/3)=20/3}$.

Let's compute Int.\@ 1 just to make sure
\begingroup
\allowdisplaybreaks
\begin{align*}
\text{Int.\@ 2}
&=\int_0^2 x-\left(x^2-2\right)\diff x\\
&=\int_0^2x-x^2+2\diff x\\
&=\int_0^2-x^2+x+2\diff x\\
&=\left.-\tfrac{1}{3}x^3+\tfrac{1}{2}x^2+2x\right|_0^2\\
&=-\tfrac{1}{3}2^3+\tfrac{1}{2}2^2+2\cdot 2\\
&\phantom{{}={}}-\left(-\tfrac{1}{3}\cdot 0^3+\tfrac{1}{2}\cdot 0^2+2\cdot
  0\right)\\
&=-\frac{8}{3}+2+4\\
&=-\frac{8}{3}+6\\
&=\frac{-8+2\cdot 6}{3}\\
&=\frac{-8+18}{3}\\
&=\frac{10}{3}.
\end{align*}
\endgroup
Then
\[
\int_{-2}^2\left||x|-\left(x^2-2\right)\right|\diff x=\text{Int.\@
  1}+\text{Int.\@ 2}=\frac{10}{3}+\frac{10}{3}=\boxed{\frac{20}{3}.}\qedhere
\]
\end{proof}

\begin{problem}[{\color{Green}\#5}, {\color{Red}\#4}]
What is the radius of the sphere $x^2+y^2+z^2+8x-2y-4z=15$?
\end{problem}
\begin{proof}[Solution]
Remember the standard equation for the sphere of radius $r$ with center
$C=(a,b,c)$? Here it is
\begin{equation}
\label{eq:sphere-C-r}
(x-a)^2+(y-b)^2+(z-c)^2=r^2
\end{equation}
So all we need to do is to manipulate our equation
$x^2+y^2+z^2+8x-2y-4z=15$ until it looks like the equation
(\ref{eq:sphere-C-r})
\begingroup
\allowdisplaybreaks
\begin{align*}
x^2+y^2+z^2+8x-2y-4z&=15\\
\left(x^2+8x\right)
+\left(y^2-2y\right)
+\left(z^2-4z\right)&=15\\
\intertext{now we complete the square, not forgetting to balance the
  equation on the right-hand side}
\left(x^2+8x+16\right)
+\left(y^2-2y+1\right)
+\left(z^2-4z+4\right)&=15+16+1+4\\
(x+4)^2+(y-1)^2+(z-2)^2&=36.
\end{align*}
We didn't need to factor the left-hand side, but why not do it anyway? Now,
looking at our equation (\ref{eq:sphere-C-r}) we see that our radius
$r^2=36$ so $\boxed{r=\sqrt{36}=6}$.
\endgroup
\end{proof}

\begin{problem}[{\color{Green}\#6}, {\color{Red}\#6}]
Consider the region enclosed by the graph of the function $y=x^4$ and the
$x$-axis between $x=0$ and $x=1$. Find the volume of the solid obtained by
rotating the region about the $x$-axis using the disks/washers method.
\end{problem}
\begin{proof}[Solution]
This one is easy enough. All we need to do is find an equation for the area
of the perpendicular cross section $A$ which (as a rule of thumb, you want
to express in terms of the axis which is perpendicular to your cross
section) will be in terms of $x$
\[
A(x)=\pi y^2=\pi\left(x^4\right)^2=\pi x^8.
\]
Now, compute
\begingroup
\allowdisplaybreaks
\begin{align*}
\int_0^1\pi A(x)\diff x
&=\int_0^1\pi x^8\diff x\\
&=\pi\int_0^1 x^8\diff x\\
&=\pi\left.\tfrac{1}{9}x^9\right|_0^1\\
&=\pi\left(\tfrac{1}{9}1^9-\left(\tfrac{1}{9}0^9\right)\right)\\
&=\boxed{\frac{\pi}{9}.}\qedhere
\end{align*}
\endgroup
\end{proof}

\begin{problem}[{\color{Green}\#7}, {\color{Red}\#9}]
Consider the region enclosed by the graph of the function $y=x-x^4$ and the
$x$-axis. Find the volume of the solid obtained by rotating the region
about the $y$-axis using the cylindrical shells method.
\end{problem}
\begin{proof}[Solution]
This one is also easy. All we need to do is find the cylindrical
area. Since we are revolving about the $y$-axis, the length of our
cylinder will be $x$ and point along the $x$-axis so we probably want to
express our cross sectional area $A$ in terms of $x$ like so
\[
A(x)=2\pi x(x-x^4).
\]
Now, we need to find when $x-x^4$ intersects the line $y=0$. This happens
when $x=0$ since $0-0^4=0$ and, factoring, $x(1-x^3)$ when $x=1$. Using the
formula for our cross section, we integrate $A(x)$ from $0$ to $1$ to find
our volume
\begin{align*}
V&=\int_0^12\pi x\left(x-x^4\right)\diff x\\
&=2\pi\int_0^1x^2-x^5\diff x\\
&=2\pi\left(\left.\tfrac{1}{3}x^3-\tfrac{1}{6}x^6\right|_0^1\right)\\
&=2\pi\left(\tfrac{1}{3}1^3-\tfrac{1}{6}1^6-\left(\tfrac{1}{3}0^3-\tfrac{1}{6}0^6\right)\right)\\
&=2\pi\left(\frac{2-1}{6}\right)\\
&=\frac{2\pi}{6}\\
&=\boxed{\frac{\pi}{3}.}\qedhere
\end{align*}
\end{proof}

\begin{problem}[{\color{Green}\#8}, {\color{Red}\#8}]
Let $\langle a,b,c\rangle$ be the unit vector of length $6$ in the opposite
direction to $\langle -2,1,-2\rangle$. Compute $a+b+c$.
\end{problem}
\begin{proof}[Solution]
The wording of this question is very wonky. What was meant, I believe, was
``Let $\langle a,b,c\rangle$ be the vector which is $6$ times as long as
the unit vector pointing in the opposite direction to $\langle
-2,1,-2\rangle$''.

First, make turn the vector $\langle -2,1,-2\rangle$ into a unit vector
like so
\[
\frac{\langle -2,1,-2 \rangle}{\left|\langle -2,1,-2\rangle\right|}=
\frac{\langle -2,1,-2 \rangle}{\sqrt{(-2)^2+1^2+(-2)^2}}=
\frac{\langle -2,1,-2 \rangle}{\sqrt{4+1+4}}=
\frac{\langle -2,1,-2 \rangle}{\sqrt{9}}=
\left<-\frac{2}{3},\frac{1}{3},-\frac{2}{3}\right>.
\]
To get a unit vector pointing in the opposite way, we just multiply by
$-1$.\footnote{Why? Well, recall the law of cosines, equation
  (\ref{eq:law-of-cosines}), which says that
  $\vec a\cdot\vec b=\left|\vec a\right|\bigl|\vec b\bigr|\cos\theta$. If
  two vectors are pointing in opposite directions, that means that the
  angle between them is $\pi$ or $180^\circ$ so
  $\vec a\cdot\vec b=-\left|\vec a\right|\bigl|\vec b\bigr|$. Now, divide
  on both sides and we get $(\vec a/|\vec a|)\cdot(\vec b/|\vec b|)=-1$. It
  can be shown that, in fact, $\vec a/|\vec a|=-\vec b/|\vec b|$, but you
  have to solve a system of equations and that requires a bit more math
  that I am willing to write on this footnote.}
Thus, we have
\[
\left<\frac{2}{3},-\frac{1}{3},\frac{2}{3}\right>.
\]
We are told that our vector must be $6$ times as long as the unit vector so
\[
\langle a,b,c \rangle=
6\left<\frac{2}{3},-\frac{1}{3},\frac{2}{3}\right>=
\left<4,-2,4\right>,
\]
and $\boxed{a+b+c=4-2+4=6}$.
\end{proof}

\begin{problem}[{\color{Green}\#9}, {\color{Red}\#10}]
A force of $8$ lb is required to hold a spring stretched $4$ in beyond its
natural length. How much work is done in stretching the same spring from
its natural length to $6$ in?
\end{problem}
\begin{proof}[Solution]
Recall the definition for the force required to move a spring a distance
$x$ from its natural length
\begin{equation}
\label{eq:hookes-law}
F(x)=kx.
\end{equation}
This is called \href{https://en.wikipedia.org/wiki/Hooke's_law}{Hooke's
  law} and, without a doubt, you will see it in physics and, should you
decide to become a mechanical engineer, you will see it again\footnote{The
  analogue of the spring in electrical engineering is the
  \href{https://en.wikipedia.org/wiki/Inductor}{inductor}. By analogue, I
  mean that, mathematically, the inductor and the spring behave the same
  way in their appropriate contexts.} Now, to find the work needed to move
the spring from $x_1$ to $x_2$ is given by taking the integral
\begin{equation}
  \label{eq:work-spring}
W(x_1,x_2)=\int_{x_1}^{x_2}kx\diff x=\left.\tfrac{1}{2}kx^2\right|_{x_1}^{x_2}=\tfrac{1}{2}k\left({x_2}^2-{x_1}^2\right).
\end{equation}

Since they want the work in terms of lb-ft, it would be best to convert
from in to ft now. Let's do that: So initially the spring is stretched to
$4$ in which is $4/12=1/3$ ft and we want to know how much work is required
to stretch it from its natural length $0$ ft to $6/12=1/2$ ft. To proceed,
we need to find out what the value of $k$ is
\[
k=\frac{8}{1/3}=3\cdot 8=\text{$24$ lb$/$ft}.
\]
Now, plug in our values into equation (\ref{eq:work-spring}) and we have
\begin{align*}
W(0,1/2)&=\tfrac{1}{2}\cdot 24\left((1/2)^2-0^2\right)\\
&=12\cdot\frac{1}{4}\\
&=\boxed{3\text{ lb-ft}.}\qedhere
\end{align*}
\end{proof}

\begin{problem}[{\color{Green}\#10}, {\color{Red}\#12}]
Evaluate $\int_1^e x\ln x\diff x$ using integration by parts.
\end{problem}
\begin{proof}[Solution]
Since it's hard to take the integral of $\ln x$ and easy to take the
integral of $x$ take $\diff v=x$ and $u=\ln x$, then $\diff u=x^{-1}\diff
x$ and $v=\tfrac{1}{2}x^2$ so
\begingroup
\allowdisplaybreaks
\begin{align*}
\int_1^e u\diff v
&=\left.\tfrac{1}{2}x^2\ln x\right|_1^e
-\int\tfrac{1}{2}x^{-1}x^2\diff x\\
&=\left.\tfrac{1}{2}x^2\ln x\right|_1^e
-\int\tfrac{1}{2}x\diff x\\
&=\left.\tfrac{1}{2}x^2\ln x-\tfrac{1}{4}x^2\right|_1^e\\
&=\left.\tfrac{1}{4}x^2(2\ln x-1)\right|_1^e\\
&=\tfrac{1}{4}e^2(2-1)-\tfrac{1}{4}1(2\ln 1-1)\\
&=\tfrac{1}{4}e^2+\tfrac{1}{4}\\
&=\boxed{\frac{e^2+1}{4}.}\qedhere
\end{align*}
\endgroup
\end{proof}

\begin{problem}[{\color{Green}\#11}, {\color{Red}\#2}]
Evaluate $\int_0^\pi\sin^3 x\diff x$.
\end{problem}
\begin{proof}[Solution]
Using the Pythagorean identity
\begin{equation}
\label{eq:pythagorean-identity}
\cos^2 x+\sin^2 x=1,
\end{equation}
we get $\sin^2 x=1-\cos^2 x$ so
\begin{align*}
\int_0^\pi\sin^3x\diff x
&=\int_0^\pi\sin^2x\sin x\diff x\\
&=\int_0^\pi\left(1-\cos^2x\right)\sin x\diff x.
\end{align*}
Now, what is a good substitution to use at this point? We want to get rid
of the $\sin x$ so let's do $u=\cos x$. Then $\diff u=-\sin x\diff x$ and
the integral above turns into
\begin{align*}
\int_1^{-1}\left(1-u^2\right)\sin x\frac{\diff u}{-\sin x}
&=-\int_{-1}^11-u^2\diff u.
\end{align*}
Now, remember the identity about the integral that says that
$\int_a^b f(x)\diff x=-\int_b^a f(x)\diff x$, so the above turns into
\begin{align*}
-\int_1^{-1}1-u^2\diff u
&=\int_1^{-1}1-u^2\diff u\\
&=\left.u-\tfrac{1}{3}u^3\right|_{-1}^1\\
&=1-\tfrac{1}{3}-\left(-1-\tfrac{1}{3}(-1)^3\right)\\
&=\tfrac{2}{3}-\left(-1+\tfrac{1}{3}\right)\\
&=\frac{2}{3}-\left(-\tfrac{2}{3}\right)\\
&=\frac{2}{3}+\frac{2}{3}\\
&=\boxed{\frac{4}{3}.}\qedhere
\end{align*}
Why did the limits of integration change from $0\leq x\leq\pi$ to $1\geq
u\geq -1$, well $u$ is the new variable we are integration with respect to
and the relation ship between $u$ and $x$ is that $u=\cos x$ so the limits
of $u$ will be from $\cos 0=1$ to $\cos\pi=-1$. Makes sense, right?
\end{proof}

\begin{problem}[{\color{Green}\#12}, {\color{Red}\#7}]
Evaluate $\int_0^{\pi/4}\cos^2x\diff x$. Hint:
$\cos(2x)=\cos^2x-\sin^2x=2\cos^2x-1=1-2\sin^2x$.
\end{problem}
\begin{proof}[Solution]
All we need to do is modify the hint to express $\cos^2 x$ in terms of
$\cos 2x$ like so
\begin{align*}
\cos 2x&=2\cos^2 x-1\\
\cos 2x+1&=2\cos^2 x\\
\frac{\cos 2x+1}{2}&=\cos^2x
\end{align*}
so our integral turns into
\begin{align*}
\int_0^{\pi/4}\cos^2 x\diff x
&=\int_0^{\pi/4}\frac{\cos 2x+1}{2}\diff x\\
&=\frac{1}{2}\int_0^{\pi/4}\cos 2x+1\diff x\\
&=\frac{1}{2}\left(\left.\tfrac{1}{2}\sin 2x+x\right|_0^{\pi/4}\right)\\
&=\frac{1}{2}\left(\tfrac{1}{2}\sin 2(\pi/4)+\tfrac{pi}{4}-(\tfrac{1}{2}\sin 2\cdot
  0-0)\right)\\
&=\frac{1}{2}\left(\tfrac{1}{2}\cdot 1+\tfrac{\pi}{4}-(0-0)\right)\\
&=\tfrac{1}{2}\left(\frac{2}{4}+\frac{\pi}{4}\right)\\
&=\tfrac{1}{2}\left(\frac{2+\pi}{4}\right)\\
&=\frac{2+\pi}{8}\\
&=\boxed{\frac{1}{4}+\frac{\pi}{8}\text{ or }\frac{\pi}{8}+\frac{1}{4}.}\qedhere
\end{align*}
\end{proof}
\end{document}

%%% Local Variable
%%% mode: latex
%%% TeX-master: t
%%% End:
