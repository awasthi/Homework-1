\def\documentauthor{Carlos Salinas}
\def\hwnum{5}
\def\documenttitle{MA166: Recitation {\hwnum} Notes}
\def\shorttitle{MA166-Recitation-{\hwnum}-Notes}
\def\coursename{MA166}
\def\documentsubject{calculus ii}
\def\authoremail{salinac@purdue.edu}

\documentclass{amsart}
\usepackage{geometry}
\usepackage[dvipsnames]{xcolor}

\usepackage[
    breaklinks,
    bookmarks=true,
    % colorlinks=true,
    pageanchor=false,
    linkcolor=black,
    anchorcolor=black,
    citecolor=black,
    filecolor=black,
    menucolor=black,
    runcolor=black,
    urlcolor=black,
    hyperindex=false,
    hyperfootnotes=true,
    pdftitle={\shorttitle},
    pdfauthor={\documentauthor},
    pdfkeywords={\documentsubject},
    pdfsubject={\coursename}
    ]{hyperref}

% Use symbols instead of numbers
\renewcommand*{\thefootnote}{\fnsymbol{footnote}}

%% Math
\usepackage{amsmath}
\usepackage{amssymb}
\usepackage{amsthm}
\usepackage{mathtools}
\usepackage[mathcal]{euscript}
\usepackage{mathrsfs}
\usepackage{wasysym}

\usepackage{cmap}
\usepackage{CJKutf8}
\newcommand{\textkr}[1]{\begin{CJK}{UTF8}{mj}#1\end{CJK}}
\newcommand{\textjp}[1]{\begin{CJK}{UTF8}{min}#1\end{CJK}}
\newcommand{\textzh}[1]{\begin{CJK}{UTF8}{bsmi}#1\end{CJK}}

\usepackage[T2A,T1]{fontenc}
\usepackage[utf8]{inputenc}
\usepackage[french,german,spanish,russian,english]{babel}
\babeltags{fr=french,
           de=german,
           en=english,
           es=spanish,
           ru=russian
           }
\def\spanishoptions{mexico}

\selectlanguage{english}

\usepackage{graphicx}
\graphicspath{{figures/}}

% Misc
\usepackage{microtype}
\usepackage{soul}
\usepackage{multicol}
\usepackage[inline]{enumitem}
\usepackage{listings}
\usepackage{mleftright}
\mleftright

\theoremstyle{plain}
\newtheorem{theorem}{Theorem}
\newtheorem{proposition}[theorem]{Proposition}
\newtheorem{corollary}[theorem]{Corollary}
\newtheorem{claim}[theorem]{Claim}
\newtheorem{lemma}[theorem]{Lemma}
\newtheorem{axiom}[theorem]{Axiom}

\newtheorem*{corollary*}{Corollary}
\newtheorem*{claim*}{Claim}
\newtheorem*{lemma*}{Lemma}
\newtheorem*{proposition*}{Proposition}
\newtheorem*{theorem*}{Theorem}

\theoremstyle{definition}
\newtheorem{definition}{Definition}
\newtheorem{example}{Examples}
\newtheorem{examples}[example]{Examples}
\newtheorem{exercise}{Exercise}
\newtheorem{problem}[exercise]{Problem}

\newtheorem*{example*}{Example}
\newtheorem*{exercise*}{Exercise}
\newtheorem*{problem*}{Problem}

%% Redefinitions & commands
\newcommand{\nle}{\ensuremath{\not<}}
\newcommand{\nge}{\ensuremath{\not>}}
\newcommand{\nsubset}{\ensuremath{\not\subset}}
\newcommand{\nsupset}{\ensuremath{\not\supset}}
\newcommand\minus{\ensuremath{\null\smallsetminus}}
\renewcommand\qedsymbol{\ensuremath{\null\hfill\smiley{}}}

%% Commands and operators
\DeclareMathOperator{\comp}{comp}
\DeclareMathOperator{\proj}{proj}
\DeclareMathOperator{\diff}{d}

\newcommand{\bbC}{\mathbb{C}}
\newcommand{\bbN}{\mathbb{N}}
\newcommand{\bbQ}{\mathbb{Q}}
\newcommand{\bbR}{\mathbb{R}}
\newcommand{\bbZ}{\mathbb{Z}}

\newcommand{\bfu}{\mathbf{u}}
\newcommand{\bfv}{\mathbf{v}}
\newcommand{\bfw}{\mathbf{w}}

\begin{document}
\author{\href{mailto:\authoremail}{\documentauthor}}
\title{\documenttitle}
\date{\today}
\maketitle
\section{Fun with Euler's Formula}
In class today, I talked a little bit about
\href{https://en.wikipedia.org/wiki/Euler's_formula}{Euler's formula}
\begin{equation}
\label{eq:eulers-formula}
e^{i\theta}=\cos\theta+i\theta,
\end{equation}
where $i=\sqrt{-1}$. Later in the semester (once we define
\href{https://en.wikipedia.org/wiki/Taylor_series}{Taylor series}) you will
learn why equation \eqref{eq:eulers-formula} is true, but for now we leave it
as a \href{https://en.wikipedia.org/wiki/Black_box}{black box}. We'll use
equation \eqref{eq:eulers-formula} to derive some trigonometric identities
and compute some integrals.

First, let us derive some results about cosine and sine in terms of the
complex exponential $e^{i\theta}$. Associated with every complex number
$z=x+iy$ is a (usually complex) number called the \emph{real part}
$\Re(z)=x$ and a number called the \emph{imaginary part} $\Im(z)=y$ and a
number $\bar z=x-iy$ called the \emph{conjugate} of $z$.

Naturally, if we think of $z=x+iy$ as a vector in the
\href{https://en.wikipedia.org/wiki/Complex_plane}{complex plane $\bbC$},
conjugation of $z$ with itself will give you the magnitude of $z$ squared
\begin{align}
\label{eq:magnitude-square}
z\bar z&=(x+iy)(x-iy)\nonumber\\
       &=x^2-ixy+ixy-i^2y^2\nonumber\\
       &=x^2-i^2y^2\nonumber\\
       &=x^2-\left(\sqrt{-1}\right)^2y^2\nonumber\\
       &=x^2-(-1)y^2\nonumber\\
       &=x^2+y^2.
\end{align}
So $|z|=\sqrt{z\bar z}$.

Naturally, the conjugate of $\bar z=x-iy$ is again $z$ since
\begin{equation}
\label{eq:double-conjugate}
\bar{\bar z}=\overline{x-iy}=x+iy=z.
\end{equation}
Moreover, for any complex number $z$ the conjugate of $z$, $\bar z$, has
the property that
\begin{align*}
z+\bar z&= x+iy+x-iy&
z-\bar z&=x+iy-(x-iy)\\
        &=2x
        &&=2iy\\
        &=2\Re(z)
        &&=2i\Im(z).
\end{align*}
Hence, we have a nice formula for expressing $\Re(z)$ and $\Im(z)$ in terms
of $z$ alone with no reference to $x,y$ in $z=x+iy$
\begin{align}
  \label{eq:real-imaginary-part}
\Re(z)&=\frac{z+\bar z}{2}&
\Im(z)&=\frac{z-\bar z}{2i}=\frac{z-\bar z}{2i}\frac{-i}{-i}=-\frac{z-\bar z}{2}i
\end{align}

Now, we may ask, given the complex function $f(z)$ which takes a complex
number $z=x+iy$, what is $\Re f(z)$ and $\Im f(z)$? Well, $f(z)$ is just
another complex number so equation \eqref{eq:real-imaginary-part} works
and we have
\begin{align*}
\Re f(z)&=\frac{f(z)+\bar f(z)}{2}&
\Im f(z)&=-\frac{f(z)-\bar f(z)}{2}i
\end{align*}

Why go through all of this trouble? Well, we want to be able to express
sine and cosine in terms of the complex exponential $e^{i\theta}$ so we can
do useful things with it like simplify all of our integral calculations.

\subsection{Complex $\sin$ and $\cos$}
Using equation \eqref{eq:real-imaginary-part} together with Euler's
formula, equation \eqref{eq:eulers-formula}, we have
\begin{align*}
\Re\left(e^{i\theta}\right)&=\frac{e^{i\theta}+e^{-i\theta}}{2}&
\Im\left(e^{i\theta}\right)&=-\frac{e^{i\theta}-e^{-i\theta}}{2}i\nonumber\\
\Re(\cos\theta+i\sin\theta)&=\cos\theta&
\Im(\cos\theta+i\sin\theta)&=\sin\theta
\end{align*}
so we have the amazing identity
\begin{equation}
\label{eq:complex-sin-cos}
\cos\theta=\frac{e^{i\theta}+e^{-i\theta}}{2}
\qquad
\sin\theta=-\frac{e^{i\theta}+e^{-i\theta}}{2}i.
\end{equation}

Now, let's verify some trigonometric identities using equation
\eqref{eq:complex-sin-cos}.
\begin{example}[Sum of angles formula]
Consider the sum of angles formula  for either sine or cosine
\begin{equation}
  \label{eq:sum-of-angles}
\cos(\theta+\varphi)=\cos\theta\cos\varphi-\sin\theta\sin\varphi\qquad
\sin(\theta+\varphi)=\sin\theta\cos\varphi+\sin\varphi\cos\theta
\end{equation}
for some angle $0\leq\theta,\varphi\leq 2\pi$. Using equation
\eqref{eq:complex-sin-cos} we can express $\sin(\theta+\varphi)$ and
$\cos(\theta+\varphi)$ as the the sum of complex exponentials
\begingroups
\allowdisplaybreaks
\begin{align*}
\cos(\theta+\varphi)&=\frac{e^{i(\theta+\varphi)}+e^{-i(\theta+\varphi)}}{2}&
\sin(\theta+\varphi)&=-\frac{e^{i(\theta+\varphi)}-e^{-i(\theta+\varphi)}}{2}i\\
&=\frac{e^{i\theta}e^{i\varphi}+e^{-i\theta}e^{-i\varphi}}{2}&
&=-\frac{e^{i(\theta+\varphi)}-e^{-i(\theta+\varphi)}}{2}i\\
&=
\end{align*}
\endgroup

\begin{example}[Derivative of $\cos$, $\sin$]
Here is a fun one I bet you haven't seen yet. We know what the derivative
of $e^{i\theta}$ is with respect to $\theta$, right? It's just
$ie^{i\theta}$. By Euler's formula, we have
\begin{align*}
\frac{\diff}{\diff\theta}\left(e^{i\theta}\right)
&=\frac{\diff}{\diff\theta}(\cos\theta+i\sin\theta)\\
ie^{i\theta}&=\frac{\diff}{\diff\theta}\cos\theta
+i\frac{\diff}{\diff\theta}\sin\theta,\\
\intertext{but by using equation \eqref{eq:eulers-formula} on the left we have}
i(\cos\theta+i\sin\theta)&=\frac{\diff}{\diff\theta}\cos\theta
+i\frac{\diff}{\diff\theta}\sin\theta\\
i\cos\theta+i^2\sin\theta)&=\frac{\diff}{\diff\theta}\cos\theta
+i\frac{\diff}{\diff\theta}\sin\theta\\
-\sin\theta+i\cos\theta&=\frac{\diff}{\diff\theta}\cos\theta
+i\frac{\diff}{\diff\theta}\sin\theta\\
-\sin\theta+i\cos\theta&=\frac{\diff}{\diff\theta}\cos\theta
+i\frac{\diff}{\diff\theta}\sin\theta
\end{align*}
Putting real with real and complex with complex, we see that
\[
\frac{\diff}{\diff\theta}\cos\theta=-\sin\theta\qquad\text{and}\qquad
\frac{\diff}{\diff\theta}\sin\theta=\cos\theta.
\]
Nice, right?
\end{example}
\end{example}
\end{document}

%%% Local Variables:
%%% mode: latex
%%% TeX-master: t
%%% End:
