\chapter{Exam III (Spring 2016)}
Here are the solutions to Exam III.
\begin{quote}
{\color{Green} Version 01}: CCEA ECCD BADC

{\color{Red} Version  02}: DECA ECBC DABE
\end{quote}

\begin{problem}[{\color{Green} \# 1}, {\color{Red} \# 11}]
Test the following series for convergence:
\begin{enumerate}[label=(\MakeUppercase{\roman*})]
\item $\sum_{n=1}^\infty e^{-n}$;
\item $\sum_{n=1}^\infty2^n/n$;
\item $\sum_{n=1}^\infty3/n^2$.
\end{enumerate}
\end{problem}
\begin{proof}[Solution]
(I) By the integral test, we have
\begin{align*}
\sum_{n=1}^\infty e^{-n}
&\sim\int_1^\infty e^{-x}\\
&=\left[-e^{-x}\right]_1^\infty\\
&=-0-(-e^{-1})\\
&=e^{-1}\\
&<\infty.
\end{align*}
Hence, $\sum_{n=1}^\infty e^{-n}$ converges.
\\\\
(II) This series does not converge by the test for divergence since the
limit of $2^n/n$ as $n\to\infty$ is no $0$. In particular, you can use
l'Hôpital's rule to get $\lim_{x\to\infty}2^x/x=\lim_{x\to\infty}2^x\ln
2/1=\infty\neq 0$.
\\\\
(III) Write $\sum_{n=1}^\infty 3/n^2$ as the product $3\sum_{n=1}^\infty
n^2$. Then, since $\sum_{n=1}^\infty 1/n^2$ is a $p$-series with $p=2>1$,
it converges. Hence $3\sum_{n=1}^\infty 1/n^2=\sum_{n=1}^\infty 3/n^2$
converges.
\\\\
Answer: {\color{Green} C}, {\color{Red} B}.
\end{proof}

\begin{problem}[{\color{Green} \# 2}, {\color{Red} \# 4}]
Determine which of the following statements are true and which are false.
\begin{enumerate}[label=(\MakeUppercase{\roman*})]
\item If $\sum_{n=1}^\infty a_n$ is convergent, then
  $\sum_{n=1}^\infty|a_n|$ is convergent.
\item If $a_n>0$, $b_n>0$, $\sum_{n=1}^\infty b_n$ is convergent, and
  $b_n\leq a_n$, then $\sum_{n=1}^\infty a_n$ is convergent.
\item If $a_n>0$, $b_n>0$, $\sum_{n=1}^\infty b_n$ is convergent, and
  $\lim_{n\to\infty} a_n/b_n=5$, then $\sum_{n=1}^\infty a_n$ is convergent.
\end{enumerate}
\end{problem}
\begin{proof}[Solution]
(I) This is false. Consider the series $\sum_{n=1}^\infty(-1)^n/n$. This
series is convergent by the alternating series test, but $\sum_{n=1}^\infty
1/n$ is divergent since it is the harmonic series.
\\\\
(II) This is false. Consider the series $\sum_{n=1}^\infty1/n^2$ and
$\sum_{n=1}^\infty n$; $1/n^2<n$, but $\sum_{n=1}^\infty n$ diverges,
whereas $\sum_{n=1}^\infty 1/n^2$ converges since it is a $p$-series with
$p=2>1$.
\\\\
(III) This is true by the limit comparison test.
\\\\
Answer: {\color{Green} C}, {\color{Red} A}.
\end{proof}

\begin{problem}[{\color{Green} \# 3}, {\color{Red} \# 6}]
The series $\sum_{n=1}^\infty (-1)^n/n^2$ is convergent by the alternating
series test. According to the alternating series estimation theorem, what
is the smallest number of terms needed to find he sum of the series with
error less than $1/15$
\end{problem}
\begin{proof}[Solution]
Compute the first few terms of the series
\[
\sum_{n=1}^\infty\frac{(-1)^n}{n^2}\approx
\underbrace{-1+\frac{1}{4}-\frac{1}{9}}_{\text{terms $>1/15$}}+\frac{1}{16}-\cdots.
\]
Since the term following $-1/9$ is less that $1/15$, we only need $3$, by
the alternating series estimation theorem, we only need $3$ terms.
\\\\
Answer: {\color{Green} E}, {\color{Red} C}
\end{proof}

\begin{problem}[{\color{Green} \# 4}, {\color{Red} \# 1}]
Test the following series for convergence:
\begin{enumerate}[label=(\MakeUppercase{\roman*})]
\item $\sum_{n=1}^\infty e^{2n}/n!$;
\item $\sum_{n=1}^\infty n/(2n+4)$;
\item $\sum_{n=1}^\infty (-1)^nn/(2n+4)$.
\end{enumerate}
\end{problem}
\begin{proof}[Solution]
(I) By the ratio test, we have
\[\begin{aligned}
\left(\frac{e^{2(n+1)}}{(n+1)!}\right)\left(\frac{n!}{e^{(2n+1)}}\right)&=
\frac{e^{2n+2}n!}{(n+1)n!e^{2n}}\\
&=\frac{e^2}{n+1}
\end{aligned}\]
which goes to $0$ as $n\to\infty$. Thus, the series in (I) converges.
\\\\
(II) By the integral test, we have
\begin{align*}
\sum_{n=1}^\infty\frac{n}{(2n+4)}
&\sim\int_1^\infty\frac{x}{2x+4}\diff x\\
\intertext{maket he substitution $u=2x+4$, $du=2dx$}
&=\frac{1}{2}\int_6^\infty\frac{(u-4)/2}{u}\diff u\\
&=\frac{1}{4}\int_6^\infty\frac{u-4}{u}\diff u\\
&=\frac{1}{4}\int_6^\infty1-\frac{4}{u}\diff u\\
&=\int_6^\infty \frac{1}{4}-\int_6^\infty\frac{du}{u}\\
&=\infty.
\end{align*}
Therefore, the sum in (II) does not converge.
\\\\
(III) Apply the alternating test. Note that $a_{n+1}<a_n$, but
\[
\lim_{n\to\infty}\frac{n}{2n+4}=\lim_{x\to\infty}\frac{x}{2x+4}=\lim_{x\to\infty}\frac{1}{2}=\frac{1}{2}\neq 0.
\]
Hence, the series does not converge.
\\\\
Answer: {\color{Green} A}, {\color{Red} D}.
\end{proof}

\begin{problem}[{\color{Green} \# 5}, {\color{Red} \# 3}]
What is the interval of convergence of the series $\sum_{n=1}^\infty
n(x-1)^n/(n^2+1)$?
\end{problem}
\begin{proof}[Solution]
Either the ratio test or the root test are good methods to apply here. We
will use the ratio test. By the ratio test, we have
\begin{align*}
\lim_{n\to\infty}\left|\frac{a_{n+1}}{a_n}\right|
&=\lim_{n\to\infty}\left|\left[\frac{(n+1)(x-1)^{n+1}}{((n+1)^2+1)}\right]
\left[\frac{(n^2+1)}{n(x-1)^n}\right]\right|\\
&=\lim_{n\to\infty}\frac{(n+1)(n^2+1)|x-1|}{n(n^2+2n+2)}\\
&=\lim_{n\to\infty}\left[\frac{(n+1)(n^2+1)}{n(n^2+2n+2)}\right]|x-1|
\intertext{for those of you who have a good understanding of asymptotics
  (which terms grow largest), it is easy to see that the limit of that
  nasty expression in $n$ will be $1$ (since the highest term in the
  numerator is $n^3$ and the highest term in the denominator is $n^3$)  be
  $|x-1|$, otherwise, we observe that $n^2+2n+2>n^2$ so}
&\leq\lim_{n\to\infty}\left[\frac{(n+1)(n^2+1)}{n\cdot n^2}\right]|x-1|\\
&=\lim_{n\to\infty}\left[\frac{n^3+n^2+n+1}{n^3}\right]|x-1|\\
&=\left[\lim_{n\to\infty}\frac{n^3}{n^3}
+\lim_{n\to\infty}\frac{n^2}{n^3}
+\lim_{n\to\infty}\frac{n}{n^3}
+\lim_{n\to\infty}\frac{1}{n^3}\right]|x-1|\\
&=|x-1|.
\end{align*}
We then want to bound this limit by $1$, so $|x-1|<1$. This gives us an
interval $(0,2)$ and we must check the end points for convergence. Now,
note that for $x=0$ the series converges by the AST. For $x=2$, by the
comparison test, since $n^2+1<2n^2$, we have
\[
\sum_{n=1}^\infty\frac{n}{n^2+1}>
\sum_{n=1}^\infty\frac{n}{2n^2}=
\frac{1}{2}\sum_{n=1}^\infty\frac{n}{n^2}=
\frac{1}{2}\sum_{n=1}^\infty\frac{1}{n}=\infty.
\]
Thus, the interval of convergence is $[0,1)$.
\\\\
Answer: {\color{Green} E}, {\color{Red} C}.
\end{proof}

\begin{problem}[{\color{Green} \# 6}, {\color{Red} \# 7}]
Determine whether the following series are absolutely convergent,
conditionally convergent, or divergent:
\begin{enumerate}[label=(\MakeUppercase{\roman*})]
\item $\sum_{n=1}^\infty\cos(n)/(n^2+1)$;
\item $\sum_{n=1}^\infty(-1)^n/(2n)$.
\end{enumerate}
\end{problem}
\begin{proof}[Solution]
(I) Since $|\cos(n)|\leq 1$, by the comparison test, we have
\[
\sum_{n=1}^\infty\left|\frac{\cos(n)}{n^2+1}\right|
\leq\sum_{n=1}^\infty\frac{1}{n^2+1}\leq\sum_{n=1}^\infty\frac{1}{n^2}<\infty.
\]
Thus, the series is absolutely convergent and hence, conditionally
convergent. Remember, if a series is absolutely convergent it is also
convergent in the traditional sense.
\\\\
(II) By the alternating series test, this series is convergent since
$1/2(n+1)<1/2n$ and $\lim_{n\to\infty} 1/2n=0$. However,
\[
\sum_{n=1}^\infty\left|\frac{(-1)^n}{2n}\right|
=\sum_{n=1}^\infty\frac{1}{2n}=\frac{1}{2}\sum_{n=1}^\infty\frac{1}{n}=\infty.
\]
Hence, the series is not absolutely convergent, so it is conditionally
convergent.
\\\\
Answer: {\color{Green}C}, {\color{Red}B}.
\end{proof}

\begin{problem}[{\color{Green} \# 7}, {\color{Red} \# 9}]
Determine which of the following statements are true and which are false.
\begin{enumerate}[label=(\MakeUppercase{\roman*})]
\item $\sum_{n=1}^\infty 1/(n^2+1)$ is convergent by the ratio test.
\item $\sum_{n=1}^\infty 1/(n^2+1)$ is convergent by the limit comparison
  test with $\sum_{n=1}^\infty 1/n^2$.
\item $\sum_{n=1}^\infty 1/(n^2+1)$ is convergent by the direct comparison
  test with $\sum_{n=1}^\infty 1/n^2$.
\end{enumerate}
\end{problem}
\begin{proof}[Solution]
(I) This is false. Applying the ratio test, we have
\begin{align*}
\frac{a_{n+1}}{a_n}
&=\frac{n^2+1}{(n+1)^2+1}\\
&=\frac{n^2+1}{n^2+2n+2}
\intertext{applying l'Hôpital's rule twice, we get}
&=\lim_{x\to\infty}1\\
&=1,
\end{align*}
which is inconclusive.
\\\\
(II) By the limit comparison test, we have
\[
\lim_{n\to\infty}\frac{n^2}{n^2+1}=1<\infty
\]
so, since $\sum_{n=1}^\infty 1/n^2$ converges, so must
$\sum_{n=1}^\infty1/(n^2+1)$.
\\\\
(III) This is true. Since $n^2+1>n^2$, it follows that $1/(n^2+1)<1/n^2$ so
by the comparison test
\[
\sum_{n=1}^\infty\frac{1}{n^2+1}<\sum_{n=1}^\infty\frac{1}{n^2}
\]
which we know converges since it is a $p$-series with $p>1$.
\\\\
Answer: {\color{Green}C}, {\color{Red}D}
\end{proof}

\begin{problem}[{\color{Green} \# 8}, {\color{Red} \# 10}]
The series $\sum_{n=1}^\infty(-1)^nnx^n$ converges for $|x|<1$ to ?
\end{problem}
\begin{proof}[Solution]
Recall that
\[
\frac{1}{1-x}=\sum_{n=0}^\infty x^n.
\]
so
\[
\frac{1}{1+x}=\frac{1}{1-(-x)}=\sum_{n=0}^\infty (-1)^nx^n.
\]
The key here is recognize that the sum above looks like the derivative of
something. Namely,
\begin{align*}
\sum_{n=1}^\infty(-1)^nnx^n
&=x\sum_{n=1}^\infty(-1)^nnx^{n-1}\\
&=x\frac{d}{dx}\left[\sum_{n=1}^\infty(-1)^nx^n\right]\\
&=x\frac{d}{dx}\left[\frac{1}{1+x}\right]\\
&=x\left(\frac{-1}{(1+x)^2}\right)\\
&=-\frac{x}{(1+x)^2}.
\end{align*}
\\\\
Answer: {\color{Green}D}, {\color{Red}A}.
\end{proof}

\begin{problem}[{\color{Green} \# 9}, {\color{Red} \# 12}]
Let $\sin x=\sum_{n=0}^\infty a_n\left(x-\pi/6\right)^n$ be the Taylor
series for $\sin x$ at $a=\pi/6$. Then $a_2=$ ?
\end{problem}
\begin{proof}[Solution]
Remember that to find the Taylor series expansion of a function $f$ about
$x$, we use the following formula
\[
T(f)\coloneqq\sum_{n=0}^\infty\frac{f^{(n)}(a)}{n!}(x-a)^n
\]
where $f^{(n)}(a)$ is the $n$-th derivative of $f$ evaluated at $a$. Now,
if we put this formula in the form that the statement is asking, we see
that the coefficient $a_n$ has to be equal to the coefficient
$f^{(n)}(a)/n!$. Thus, we need to compute the $2$-nd derivative of $f$ at
$\pi/6$ and divide it by $2!$ to find $a_2$.
\[
a_2=\frac{-\sin(\pi/6)}{2!}=\frac{-1/2}{2}=-\frac{1}{4}
\]
since the second derivative of $\sin x$ is $-\sin x$.
\\\\
Answer: {\color{Green}B}, {\color{Red}E}.
\end{proof}

\begin{problem}[{\color{Green} \# 10}, {\color{Red} \# 5}]
What is the coefficient of  $x^8$ in the power series representation for
$\int x\cos(x^3)\diff x$? You may assume $\cos(x)=\sum_{n=0}^\infty(-1)^n
x^{2n}/(2n!)$.
\end{problem}
\begin{proof}[Solution]
All we need to do is replace $\cos$ by its power series represetation in
the integral and expand the first few terms of the power series
representation for the integral until we get to the term containing
$x^8$. Thus, we have
\begin{align*}
\int x\cos(x^3)\diff x
&=\int x\sum_{n=0}^\infty(-1)^n\frac{(x^3)^{2n}}{(2n)!}\\
&=\int x\sum_{n=0}^\infty(-1)^n\frac{x^{6n}}{(2n)!}\\
&=\int\sum_{n=0}^\infty(-1)^n\frac{x^{6n+1}}{(2n)!}\\
&=\sum_{n=0}^\infty\int(-1)^n\frac{x^{6n+1}}{(2n)!}\\
&=\sum_{n=0}^\infty(-1)^n\frac{x^{6n+2}}{(2n)!(6n+2)}.
\end{align*}
Thus, the term containing $x^8$, has coefficient
\[
a_1=\frac{(-1)^1}{2!\cdot(6\cdot 1+2)}=\frac{-1}{2\cdot 8}=-\frac{1}{16}.
\]
\\\\
Answer: {\color{Green}A}, {\color{Red}E}.
\end{proof}

\begin{problem}[{\color{Green} \# 11}, {\color{Red} \# 8}]
A particle moves along  the curve $x=\tan t$, $y=\sec t$, $0\leq t\leq
\pi/4$. The trajectory of the particle is a segment of
\end{problem}
\begin{proof}[Solution]
This is easily seen to be a hyperbola by the $\tan$ identity
\begin{align*}
1&=\tan^2t-\sec^2t\\
 &=x^2-y^2
\end{align*}
which is the equation for a hyperbola.
\\\\
Answer: {\color{Green}D}, {\color{Red}C}.
\end{proof}

\begin{problem}[{\color{Green} \# 12}, {\color{Red} \# 2}]
The length of the curve $x=\cos\theta+\theta\sin\theta$,
$y=\sin\theta-\theta\cos\theta$, $0\leq\theta\leq\pi$, is
\end{problem}
\begin{proof}[Solution]
Recall that the length of a parametric curve $(x(\theta),y(\theta))$ from
time $\alpha\leq\theta\leq\beta$ is given by
\[
\ell\coloneqq\int_\alpha^\beta\sqrt{\left(\frac{dx(\theta)}{d\theta}\right)^2+
\left(\frac{dy(\theta)}{d\theta}\right)^2}.
\]
First, we find the derivatives of $x$ and $y$ with respect to
$\theta$. They are
\begin{align*}
\frac{dx}{d\theta}
&=-\sin\theta+\theta\cos\theta+\sin\theta&
\frac{dy}{d\theta}
&=\cos\theta+\theta\sin\theta-\cos\theta\\
&=\theta\cos\theta&&=\theta\sin\theta.
\end{align*}
Hence, the length is given by
\[
\int_0^\pi\sqrt{(\theta\cos\theta)^2+(\theta\sin\theta)^2}\diff\theta=
\int_0^\pi\sqrt{\theta^2(\cos^2\theta+\sin^2\theta)}\diff\theta=
\int_0^\pi\theta\diff\theta=\left[\frac{\theta^2}{2}\right]_0^\pi=\frac{\pi^2}{2}.
\]
\\\\
Answer: {\color{Green}C}, {\color{Red}E}.
\end{proof}

%%% Local Variables:
%%% mode: latex
%%% TeX-master: "../MA166-Recitation"
%%% End:
