\chapter{Exam III (Spring 2016)}
Here are the solutions to Exam III.
\begin{quote}
{\color{Green} Version 01}: CCEA ECCD BADC

{\color{Red} Version  02}: DECA ECBC DABE
\end{quote}

\begin{problem}[{\color{Green} \# 1}, {\color{Red} \# 11}]
Test the following series for convergence:
\begin{enumerate}[label=(\MakeUppercase{\roman*})]
\item $\sum_{n=1}^\infty e^{-n}$;
\item $\sum_{n=1}^\infty2^n/n$;
\item $\sum_{n=1}^\infty3/n^2$.
\end{enumerate}
\end{problem}
\begin{proof}[Solution]
(I) By the integral test, we have
\begin{align*}
\sum_{n=1}^\infty e^{-n}
&\sim\int_1^\infty e^{-x}\\
&=\left[-e^{-x}\right]_1^\infty\\
&=-0-(-e^{-1})\\
&=e^{-1}\\
&<\infty.
\end{align*}
Hence, $\sum_{n=1}^\infty e^{-n}$ converges.
\\\\
(II) This series does not converge by the test for divergence since the
limit of $2^n/n$ as $n\to\infty$ is no $0$. In particular, you can use
l'Hôpital's rule to get $\lim_{x\to\infty}2^x/x=\lim_{x\to\infty}2^x\ln
2/1=\infty\neq 0$.
\\\\
(III) Write $\sum_{n=1}^\infty 3/n^2$ as the product $3\sum_{n=1}^\infty
n^2$. Then, since $\sum_{n=1}^\infty 1/n^2$ is a $p$-series with $p=2>1$,
it converges. Hence $3\sum_{n=1}^\infty 1/n^2=\sum_{n=1}^\infty 3/n^2$
converges.
\\\\
Answer: {\color{Green} C}, {\color{Red} B}.
\end{proof}

\begin{problem}[{\color{Green} \# 2}, {\color{Red} \# 4}]
Determine which of the following statements are true and which are false.
\begin{enumerate}[label=(\MakeUppercase{\roman*})]
\item If $\sum_{n=1}^\infty a_n$ is convergent, then
  $\sum_{n=1}^\infty|a_n|$ is convergent.
\item If $a_n>0$, $b_n>0$, $\sum_{n=1}^\infty b_n$ is convergent, and
  $b_n\leq a_n$, then $\sum_{n=1}^\infty a_n$ is convergent.
\item If $a_n>0$, $b_n>0$, $\sum_{n=1}^\infty b_n$ is convergent, and
  $\lim_{n\to\infty} a_n/b_n=5$, then $\sum_{n=1}^\infty a_n$ is convergent.
\end{enumerate}
\end{problem}
\begin{proof}[Solution]
(I) This is false. Consider the series $\sum_{n=1}^\infty(-1)^n/n$. This
series is convergent by the alternating series test, but $\sum_{n=1}^\infty
1/n$ is divergent since it is the harmonic series.
\\\\
(II) This is false. Consider the series $\sum_{n=1}^\infty1/n^2$ and
$\sum_{n=1}^\infty n$; $1/n^2<n$, but $\sum_{n=1}^\infty n$ diverges,
whereas $\sum_{n=1}^\infty 1/n^2$ converges since it is a $p$-series with
$p=2>1$.
\\\\
(III) This is true by the limit comparison test.
\\\\
Answer: {\color{Green} C}, {\color{Red} A}.
\end{proof}

\begin{problem}[{\color{Green} \# 3}, {\color{Red} \# 6}]
The series $\sum_{n=1}^\infty (-1)^n/n^2$ is convergent by the alternating
series test. According to the alternating series estimation theorem, what
is the smallest number of terms needed to find he sum of the series with
error less than $1/15$
\end{problem}
\begin{proof}[Solution]
Compute the first few terms of the series
\[
\sum_{n=1}^\infty\frac{(-1)^n}{n^2}\approx
-1+\frac{1}{4}-\frac{1}{9}+\frac{1}{16}-\cdots.
\]
Since the term following $-1/9$ is less that $1/15$, we only need $3$, by
the alternating series estimation theorem, we only need $3$ terms.
\\\\
Answer: {\color{Green} E}, {\color{Red} C}
\end{proof}

\begin{problem}[{\color{Green} \# 4}, {\color{Red} \# 1}]
Test the following series for convergence:
\begin{enumerate}[label=(\MakeUppercase{\roman*})]
\item $\sum_{n=1}^\infty e^{2n}/n!$;
\item $\sum_{n=1}^\infty n/(2n+4)$;
\item $\sum_{n=1}^\infty (-1)^nn/(2n+4)$.
\end{enumerate}
\end{problem}
\begin{proof}[Solution]
(I) By the ratio test, we have
\[\begin{aligned}
\left(\frac{e^{2(n+1)}}{(n+1)!}\right)\left(\frac{n!}{e^{(2n+1)}}\right)&=
\frac{e^{2n+2}n!}{(n+1)n!e^{2n}}\\
&=\frac{e^2}{n+1}
\end{aligned}\]
which goes to $0$ as $n\to\infty$. Thus, the series in (I) converges.
\\\\
(II) By the ratio test, we have
\begin{align*}
\left(\frac{n+1}{2(n+1)+4}\right)
\left(\frac{2n+4}{n}\right)
&=\frac{(n+1)(2n+4)}{(2n+6)n}\\
&=\frac{2n^2+6n+4}{2n^2+6n}
\end{align*}
which, by l'Hôpital's
\\\\
(III)
\\\\
Answer: {\color{Green}}, {\color{Red}}
\end{proof}

\begin{problem}[{\color{Green} \# 5}, {\color{Red} \# 3}]
What is the interval of convergence of the series $\sum_{n=1}^\infty
n(x-1)^n/(n^2+1)$?
\end{problem}
\begin{proof}[Solution]
Answer: {\color{Green}}, {\color{Red}}
\end{proof}

\begin{problem}[{\color{Green} \# 6}, {\color{Red} \# 7}]
Determine whether the following series are absolutely convergent,
conditionally convergent, or divergent:
\begin{enumerate}[label=(\MakeUppercase{\roman*})]
\item $\sum_{n=1}^\infty\cos(n)/(n^2+1)$;
\item $\sum_{n=1}^\infty(-1)^n/(2n)$.
\end{enumerate}
\end{problem}
\begin{proof}[Solution]
Answer: {\color{Green}}, {\color{Red}}
\end{proof}

\begin{problem}[{\color{Green} \# 7}, {\color{Red} \# 9}]
Determine which of the following statements are true and which are false.
\begin{enumerate}[label=(\MakeUppercase{\roman*})]
\item $\sum_{n=1}^\infty 1/(n^2+1)$ is convergent by the ratio test.
\item $\sum_{n=1}^\infty 1/(n^2+1)$ is convergent by the limit comparison
  test with $\sum_{n=1}^\infty 1/n^2$.
\item $\sum_{n=1}^\infty 1/(n^2+1)$ is convergent by the direct comparison
  test with $\sum_{n=1}^\infty 1/n^2$.
\end{enumerate}
\end{problem}
\begin{proof}[Solution]
Answer: {\color{Green}}, {\color{Red}}
\end{proof}

\begin{problem}[{\color{Green} \# 8}, {\color{Red} \# 10}]
The series $\sum_{n=1}^\infty(-1)^nnx^n$ converges for $|x|<1$ to ?
\end{problem}
\begin{proof}[Solution]
Answer: {\color{Green}}, {\color{Red}}
\end{proof}

\begin{problem}[{\color{Green} \# 9}, {\color{Red} \# 12}]
Let $\sin x=\sum_{n=0}^\infty a_n\left(x-\pi/6\right)^n$ be the Taylor
series for $\sin x$ at $a=\pi/6$. Then $a_2=$ ?
\end{problem}
\begin{proof}[Solution]
Answer: {\color{Green}}, {\color{Red}}
\end{proof}

\begin{problem}[{\color{Green} \# 10}, {\color{Red} \# 5}]
What is the coefficient of  $x^8$ in the power series representation for
$\int x\cos(x^3)\diff x$? You may assume $\cos(x)=\sum_{n=0}^\infty(-1)^n
x^{2n}/(2n!)$.
\end{problem}
\begin{proof}[Solution]
Answer: {\color{Green}}, {\color{Red}}
\end{proof}

\begin{problem}[{\color{Green} \# 11}, {\color{Red} \# 8}]
A particle moves along  the curve $x=\tan t$, $y=\sec t$, $0\leq t\leq
\pi/4$. The trajectory of the particle is a segment of
\end{problem}
\begin{proof}[Solution]
Answer: {\color{Green}}, {\color{Red}}
\end{proof}

\begin{problem}[{\color{Green} \# 12}, {\color{Red} \# 2}]
The length of the curve $x=\cos\theta+\theta\sin\theta$,
$y=\sin\theta-\theta\cos\theta$, $0\leq\theta\leq\pi$, is
\end{problem}
\begin{proof}[Solution]
Answer: {\color{Green}}, {\color{Red}}
\end{proof}

%%% Local Variables:
%%% mode: latex
%%% TeX-master: "../MA166-Recitation"
%%% End:
