As I promised, here are somewhat detailed solutions to two of the last Exam
2's for MA 166.
\chapter{MA 166 Exam 2, Spring 2014}
\begin{problem}
Evaluate the following integral
\[
\int_0^\pi \sin^2x \cos^2 x\;dx
\]
\end{problem}
\begin{proof}[Solution]
The following trig identities are useful
\begin{equation}
  \label{eq:sine-double-angle-formula}
\sin 2\theta=2\sin\theta\cos\theta
\end{equation}
\begin{equation}
\label{eq:sine-square-formula}
\sin^2\theta=\frac{1-\cos 2\theta}{2}.
\end{equation}
With that in mind we compute the integral
\begin{align*}
\int_0^\pi \sin^2x \cos^2 x\;dx
&=\int_0^\pi\left(\sin x\cos x\right)^2\;dx\\
&=\int_0^\pi\left(\frac{1}{2}\sin 2x\right)^2\;dx\\
&=\frac{1}{4}\int_0^\pi\sin^2 2x\;dx\\
&=\frac{1}{4}\int_0^\pi\frac{1-\cos 4x}{2}\;dx\\
&=\frac{1}{8}\int_0^\pi 1-\cos 4x\;dx\\
&=\frac{1}{8}\left[x-\frac{1}{4}\sin 4x\right]_0^\pi\\
&=\frac{1}{8}\left[\pi-0-(0-0)\right]\\
&=\boxed{\frac{\pi}{8}.}
\end{align*}
Answer: B.
\end{proof}
\begin{problem}
Evaluate the following integral
\[
\int_0^{\pi/4}\sec^4 x\tan x\;dx.
\]
\end{problem}
\begin{proof}[Solution]
The following identities are useful
\begin{equation}
  \label{eq:tangent-pythagorean-identity}
\sec^2\theta-\tan^2\theta=1.
\end{equation}
Substitute $u=\tan x$, $du=\sec^2 x\;dx$
\begin{align*}
\int_0^{\pi/4}\sec^4 x\tan x\;dx
&=\int_0^{\pi/4}\sec^2 x\sec^2 x\tan x\;dx\\
&=\int_0^{\pi/4}\sec^2 x(1+\tan^2 x)\tan x\;dx\\
&=\int_0^1(1+u^2)u\;dx\\
&=\int_0^1u+u^3\;dx\\
&=\left[\frac{1}{2}u^2+\frac{1}{4}u^4\right]_0^1\\
&=\frac{1}{2}+\frac{1}{4}-0-0\\
&=\boxed{\frac{3}{4}.}
\end{align*}
Answer: A.
\end{proof}
\begin{problem}
After the appropriate trigonometric substitution, the integral
\[
\int_2^5\frac{dx}{\sqrt{x^2-4x+13}}
\]
becomes
\end{problem}
\begin{proof}[Solution]
The general strategy for these types of problems is to complete the square
in the denominator and make some sort of trig substitution
\[
x^2-4x+13=
\left(x^2-4x+4\right)+9=
(x-2)^2+9.
\]
Make the $u$-substitution $u=(x-2)/3$, $du=dx/3$
\begin{align*}
\int_2^5\frac{dx}{\sqrt{x^2-4x+13}}
&=\int_2^5\frac{dx}{3\sqrt{(x-2)^2/9+1}}\\
&=\frac{1}{3}\int_2^5\frac{dx}{\sqrt{\left(\frac{x-2}{3}\right)^2+1}}\\
&=\int_0^1\frac{du}{\sqrt{u^2+1}}\\
\intertext{follow it up with the trig substitution $\tan\theta=u$,
  $\sec^2\theta\;d\theta=du$, $0\leq\theta\leq\pi/4$}
&=\int_0^{\pi/4}\sec^2\theta\cos\theta\;d\theta\\
&=\int_0^{\pi/4}\sec\theta\;d\theta\\
&=\left[\ln|\sec\theta+\tan\theta|\right]_0^{\pi/4}\\
&=\ln\left|\frac{\sec\pi/4+\tan\pi/4}{\sec 0+\tan 0}\right|\\
&=\ln\left|\frac{\sqrt{2}+1}{1+0}\right|\\
&=\boxed{\ln\left|\sqrt{2}+1\right|}
\end{align*}
Answer: A.
\end{proof}
\begin{problem}
Compute
\[
\int_3^4\frac{3}{x^2-x-2}\;dx.
\]
\end{problem}
\begin{proof}[Solution]
Factor
\[
x^2-x-2=(x-2)(x+1)
\]
and use partial fractions
\begin{align*}
\frac{3}{(x-2)(x+1)}&=\frac{A}{x-2}+\frac{B}{x+1}\\
3&=A(x+1)+B(x-2)\\
0x+3&=(A+B)x+A-2B
\end{align*}
gives you $A-2B=3$, $A+B=0$ so $A=-B$, $-B-2B=3$, $B=-1$ and $A=1$. Now we
can compute the integral
\begin{align*}
\int_3^4\frac{3}{x^2-x-2}\;dx
&=\int_3^4\left[\frac{1}{x-2}-\frac{1}{x+1}\right]dx\\
&=\left[\ln|x-2|-\ln|x+1|\right]_3^4\\
&=\left[\ln\left|\frac{x-2}{x+1}\right|\right]_3^4\\
&=\ln\left|\frac{2}{5}\right|-\ln\left|\frac{1}{4}\right|\\
&=\ln\left|\frac{2/5}{1/4}\right|\\
&=\boxed{\ln\left|\frac{8}{5}\right|.}
\end{align*}
Remember your log properties!
\\\\
Answer: B.
\end{proof}
\begin{problem}
It is known that
\[
\int\frac{2x-3}{x\left(x^2+1\right)}dx=a\ln x+b\ln\left(x^2+1\right)+c\tan x+C
\]
for some constants $a$, $b$, $c$ and $C$. What is $b$?
\end{problem}
\begin{proof}[Solution]
There is a typo in the original problem; instead of $c\tan x$ it should be
a $c\tan^{-1}x$. One thing you can do is use the fundamental theorem of
calculus
\begin{equation}
  \label{eq:fundamental-theorem-of-calculus}
f(x)=\frac{d}{dt}\int_a^x f(t)\;dt.
\end{equation}
Applying the fundamental theorem on our function, we get
\begin{align*}
\frac{2x-3}{x\left(x^2+1\right)}
&=\frac{a}{x}+\frac{2bx}{x^2+1}+\frac{c}{x^2+1}\\
&=\frac{a}{x}+\frac{2bx+c}{x^2+1}\\
&=\frac{a\left(x^2+1\right)+(2bx+c)x}{x\left(x^2+1\right)}\\
&=\frac{(a+2b)x^2+cx+a}{x\left(x^2+1\right)}.
\end{align*}
Now we solve for the values in the numerator by noting that $a+2b=0$, $c=2$
and $a=-3$, so $b=3/2$.
\\\\
Answer: E.
\end{proof}
\begin{problem}
Evaluate the integral
\[
\int\frac{x^2+5x+1}{\left(x^2+1\right)^2}dx.
\]
\end{problem}
\begin{proof}[Solution]
Rewrite the function
\[
\frac{x^2+5x+1}{\left(x^2+1\right)^2}=
\frac{\left(x^2+1\right)+5x}{\left(x^2+1\right)^2}=
\underbrace{\frac{1}{x^2+1}}_{f_1}
+\underbrace{\frac{5x}{\left(x^2+1\right)^2}}_{f_2}.
\]
Let's compute these separately. If you recognize the integral of
$1/\left(x^2+1\right)$ as being $\tan^{-1}(x)$, good for you; if not we can
use the trig substitution $\tan\theta=x$,
$\sec^2\theta\;d\theta=dx$
\begin{align*}
I_1&=\int\frac{dx}{x^2+1}\\
   &=\int\sec^2\theta\cos^2\theta\;d\theta\\
   &=\int 1\;d\theta\\
   &=\theta
\intertext{substitute back $\theta=\tan^{-1}(x)$ and we have}
   &=\tan^{-1}(x).
\end{align*}

Now we compute $I_2$ by using the substitution $u=x^2$, $du=2x\;dx$
\begin{align*}
I_2&=\int\frac{5x}{\left(x^2+1\right)^2}\\
   &=\frac{5}{2}\int\frac{1}{(u+1)^2}du\\
   &=-\frac{5}{2}(u+1)\\
   &=-\frac{5}{2}(x^2+1)
\end{align*}
so the integral is
\[
I_1+I_2=\boxed{\tan^{-1}(x)-\frac{5}{2}\left(x^2+1\right)^{-1}+C.}
\]
\\\\
Answer: C.
\end{proof}
\begin{problem}
Approximate $\int_{-1}^3 x^4\;dx$ using Simpson's rule with $n=4$
subintervals.
\end{problem}
\begin{proof}[Solution]
Remember Simpson's rule? Neither do I, so here it is
\begin{equation}
\label{eq:simpsons-rule}
\int_{x_0}^{x_1}f(x)\;dx=
\int_{x_0}^{x_0+2\Delta x}\approx
\frac{\Delta x}{3}\left(
f\left(x_0\right)
+4f\left(x_0+\Delta x\right)
+f\left(x_0+2\Delta x\right)
\right).
\end{equation}
Now, let's partition the interval $-1\leq x\leq 3$ into $4$ subintervals,
$\Delta x=(3-(-1))/4=1$ so $x_0=-1$, $x_1=0$, $x_2=1$, $x_3=2$, $x_4=3$. We
have
\begin{align*}
\int_{-1}^3 x^4\;dx
&=\int_{-1}^1x^4\;dx
  +\int_{1}^3x^4\;dx\\
&\approx \frac{1}{3}
\left(\left({x_0}^4+4{x_1}^4+{x_2}^4\right)
+\left({x_2}^4+4{x_3}^4+{x_4}^4\right)\right)\\
&=\frac{1}{3}\left(1+0+1+1+64+81\right)\\
&=\boxed{\frac{148}{3}.}
\end{align*}
\\\\
Answer: E.
\end{proof}
\begin{problem}
Which of the following improper integrals converge?
\begin{enumerate}[label=\MakeUppercase{\roman*}.]
\item $\displaystyle\int_0^\infty xe^{-x^2}\;dx$
\item $\displaystyle\int_{-\infty}^\infty\frac{dx}{x}$
\item $\displaystyle\int_{-1}^1\frac{dx}{\sqrt[3]{x}}$
\end{enumerate}
\end{problem}
\begin{proof}[Solution]
First, let's compute the integrals I, II and III. Here's I
\begin{align*}
I_1&=\int_0^\infty xe^{-x^2}\;dx\\
   &=\frac{1}{2}\int_0^\infty e^{-u}\;du\\
   &=\left[-\frac{1}{2}e^{-u}\right]_0^\infty\\
   &=\frac{1}{2}.
\end{align*}
Here's II
\begin{align*}
I_2&=\int_{-\infty}^\infty\frac{dx}{x}\\
&=\int_{-\infty}^0\frac{dx}{x}+\int_0^\infty\frac{dx}{x}\\
&=-\int_\infty^0\frac{du}{u}+\int_0^\infty\frac{dx}{x}\\
&=\int_0^\infty\frac{du}{u}+\int_0^\infty\frac{dx}{x}\\
&=\left[\ln u\right]_0^\infty+\left[\ln x\right]_0^\infty
\end{align*}
this clearly divereges since $\ln x\to\-\infty$ as $x\to 0$ and $ln
x\to\infty$ as $x\to\infty$. The same goes for $\ln u$. You can't
win. Here's III
\begin{align*}
I_3&=\int_{-1}^1\frac{dx}{\sqrt[3]{x}}\\
   &=\int_{-1}^1x^{1/3}\;dx\\
   &=\frac{3}{4}\left[x^{4/3}\right]_{-1}^1\\
   &=0.
\end{align*}
Hence, I and III converge, but III does not.
\end{proof}
\begin{problem}
Find the exact length of the curve $y=\ln(\sec x)$, $0\leq x\leq\pi/3$.
\end{problem}
\begin{proof}[Solution]
First find the derivative with respect to $x$
\[
\frac{dy}{dx}=\tan x.
\]
Then
\begin{align*}
S(0,\pi/3)
&=\int_0^{\pi/3}\sqrt{1+\tan^2x}\;dx\\
&=\int_0^{\pi/3}\sec x\;dx\\
&=\left[\ln|\sec x+\tan x|\right]_0^{\pi/3}\\
&=\ln\left(2+\sqrt{3}\right)-\ln(1-0)\\
&=\boxed{\ln\left(2+\sqrt{3}\right).}
\end{align*}
\end{proof}
\begin{problem}
The curve $y=2-x^2$, $0\leq x\leq 1$, is rotated around the $y$-axis to
generate the surface $S$. Which is of the following formulas represents the
area of the surface $S$?
\end{problem}
\begin{proof}[Solution]
First we find the derivative
\[
\frac{dy}{dy}=-2x.
\]
Now the arc will be
\begin{align*}
ds&=\sqrt{1+\left(\frac{dy}{dx}\right)^2}dx\\
  &=\sqrt{1+\left(-2x\right)^2}dx\\
  &=\sqrt{1+4x^2}dx
\end{align*}
Then
\begin{align*}
\int_0^2 2\pi x\;ds
&=\int_0^2 2\pi x\sqrt{1+4x^2}dx
\intertext{substitute $u=1+4x^2$, $du=8x\;dx$}
&=\frac{\pi}{4}\int_0^{17}\sqrt{u}\;du\\
&=\frac{\pi}{4}\left[\frac{2}{3}x^{3/2}\right]_0^{17}\\
&=\frac{\pi}{6}\left[x^{3/2}\right]_0^{17}\\
&=\frac{17\sqrt{17}\pi}{6}.
\end{align*}
\\\\
Answer: D.
\end{proof}
\begin{problem}
Let $(\bar x,\bar y)$ denote the coordinates of the center of mass of a
region bounded by the curves $y=x^4$, $y=0$ and $x=1$, with density
$\rho=1$. What is $\bar x$?
\end{problem}
\begin{proof}[Solution]
Remember the formulas for the moments? Here they are
\begin{equation}
  \label{eq:moment-x-y}
M_x=\rho\int_a^b\frac{f(x)^2-g(x)^2}{2}dx
\qquad
M_y=\rho\int_a^bx(f(x)-g(x))\;dx.
\end{equation}
Now, compute $M_x$ and $M_y$
\begin{align*}
M_x&=\frac{1}{2}\int_0^1x^8\;dx&
M_y&=\int_0^1 x^5\;dx\\
&=\frac{1}{18}\left[x^9\right]_0^1&
&=\frac{1}{6}\left[ x^6 \right]_0^1\\
&=\frac{1}{18}&
&=\frac{1}{6}.
\end{align*}
And the area under the curve is
\begin{align*}
A&=\int_0^1 x^4\;dx\\
 &=\frac{1}{5}\left[x^5\right]_0^1\\
 &=\frac{1}{5}.
\end{align*}
So
\[
\bar x =\frac{M_y}{A}=\frac{1/6}{1/5}=\frac{5}{6}.
\]
\\\\
Answer: D.
\end{proof}
\begin{problem}
Determine whether the following sequences are convergent or divergent.
\begin{enumerate}[label=(\arabic*)]
\item $\displaystyle\left\{\,a_n=\frac{2n}{3n+1}\,\right\}$
\item $\displaystyle\left\{\,a_n=\cos n\pi\,\right\}$
\item $\displaystyle\left\{\,a_n=n\sin\left(\frac{1}{n}\right)\,\right\}$.
\end{enumerate}
\end{problem}
\begin{proof}[Solution]
(2) Recall that $a_n\cos n\pi=(-1)^n$. This sequence clearly does not converge
because for its value goes from $-1$ to $1$ and the distance between any
two member $|a_n-a_{n-1}|=2$, i.e., never gets any smaller.
\\\\
(1) This converges. Set $n=x$ and by L'Hôpital's rule we have
\[
\lim_{x\to\infty} \frac{2x}{3x+1}=\lim_{x\to\infty}\frac{2}{3}=1.
\]
So the sequence converges to $1$.
\\\\
(3) Lastly, we can show this sequence converges by making the substitution
as $m=1/n$ and now $1/n\to 0$ as $n\to\infty$ so we want
\[
\lim_{m\to 0}\frac{1}{m}\sin(m).
\]
This is a well known limit and you can use some geometry to show that
\[
\cos m\leq \frac{\sin m}{m}\leq 1
\]
so by the squeeze theorem, $\lim_{m\to 0}\sin(m)/m=1$.
\\\\
Answer: C
\end{proof}

\chapter{Formula Sheet}
Here is some useful stuff you should know before you take the exam
\section{Trigonometric identities}
\subsection{Pythagorean identities}
\begin{align}
\label{eq:trig-identities}
\sin^2\theta+\cos^2\theta&=1\\
\sec^2\theta-\tan^2\theta&=1\\
\csc^2\theta-\cot^2\theta&=1.
\end{align}
\subsection{Square and double-angle formulas}
\begin{align}
\label{eq:dobule-angle}
\sin^2\theta&=\frac{1-\cos 2\theta}{2}\\
\cos^2\theta&=\frac{1+\cos 2\theta}{2}\\
\sin 2\theta&=2\sin\theta\cos\theta\\
\cos 2\theta&=\cos^2\theta-\sin^2\theta
\end{align}

\section{Approximate Integrals}
For the integral
\[
\int_a^b f(x)\;dx
\]
set $\Delta x=(b-a)/n$ where $n$ is the number of desired steps and $\bar
x_i=(x_{i-1}+x_i)/2$, i.e., the midpoint and $x_i=x_{i-1}+\Delta x$. Then
\begin{align}
  \label{eq:approx-integrals}
M_n&=\Delta\left[f(\bar x_1)+\cdots+f(\bar x_n)\right]\\
T_n&=\frac{\Delta x}{2}\left[f(x_0)+2f(x_1)+\cdots+2
  f(x_{n-1})+f(x_n)\right]\\
S_n&=\frac{\Delta x}{3}\left[f(x_0)+4f(x_1)+2f(x_2)+4f(x_3)+\cdots
  +2f(x_{n-2})+4f(x_{n-1})+f(x_n)\right].
\end{align}
Where $M_n$ is the midpoint rule, $T_n$ is the trapezoidal rule and $S_n$
is Simpson's rule with $n$ steps.

\section{Arc Length and Surface Area}
Let $f(x)$ be an integrable function (yes, there are functions you cannot
integrate) of $x$ and $a\leq x\leq b$ then
\subsection{Arc-length}
The arc-length of $f$ is
\begin{equation}
\label{eq:arc-length}
L=\int_a^b\sqrt{1+\left(\frac{dx}{dy}\right)^2}\;dx.
\end{equation}
\subsection{Surface area}
The surface area of $f$ is
\begin{equation}
\label{eq:surface-area-y}
\int 2\pi x\sqrt{1+\left(\frac{dy}{dx}\right)^2}dx
\end{equation}
about the $y$-axis and
\begin{equation}
\label{eq:surface-area-x}
\int 2\pi f(x)\sqrt{1+\left(\frac{dy}{dx}\right)^2}dx.
\end{equation}
\\\\
If $f(y)$ is a function of $x$,
\begin{equation}
\label{eq:surface-area-y-2}
\int 2\pi f(y)\sqrt{1+\left(\frac{dx}{dy}\right)^2}dx
\end{equation}
about the $y$-axis and
\begin{equation}
\label{eq:surface-area-x-2}
\int 2\pi y\sqrt{1+\left(\frac{dx}{dy}\right)^2}dx.
\end{equation}
\section{Moments and centroids}
Let $\rho$ be the density and $f$, $g$ be the curves for $a\leq x\leq
b$. Then the area is
\begin{equation}
  \label{eq:under-curve}
A=\int_a^b f(x)-g(x)\;dx,
\end{equation}
the mass is
\begin{equation}
  \label{eq:mass}
m=\rho A,
\end{equation}
the moments are
\begin{equation}
  \label{eq:moments}
M_x=\rho\int_a^b\frac{f(x)^2-g(x)^2}{2}dx
\qquad
M_y=\rho\int_a^bx(f(x)-g(x))\;dx
\end{equation}
and the centroids are
\begin{equation}
\label{eq:centroids}
\bar x=\frac{M_y}{m}
\qquad
\bar y=\frac{M_x}{m}.
\end{equation}

%%% Local Variables:
%%% TeX-master: "../MA166-Recitation"
%%% End:
