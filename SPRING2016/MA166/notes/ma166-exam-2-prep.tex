As I promised, here are somewhat detailed solutions to two of the last Exam
2's for MA 166.
\chapter{MA 166 Exam 2, Spring 2014}
\begin{problem}
Evaluate the following integral
\[
\int_0^\pi \sin^2x \cos^2 x\;dx
\]
\end{problem}
\begin{proof}[Solution]
The following trig identities are useful
\begin{equation}
  \label{eq:sine-double-angle-formula}
\sin 2\theta=2\sin\theta\cos\theta
\end{equation}
\begin{equation}
\label{eq:sine-square-formula}
\sin^2\theta=\frac{1-\cos 2\theta}{2}.
\end{equation}
With that in mind we compute the integral
\begin{align*}
\int_0^\pi \sin^2x \cos^2 x\;dx
&=\int_0^\pi\left(\sin x\cos x\right)^2\;dx\\
&=\int_0^\pi\left(\frac{1}{2}\sin 2x\right)^2\;dx\\
&=\frac{1}{4}\int_0^\pi\sin^2 2x\;dx\\
&=\frac{1}{4}\int_0^\pi\frac{1-\cos 4x}{2}\;dx\\
&=\frac{1}{8}\int_0^\pi 1-\cos 4x\;dx\\
&=\frac{1}{8}\left[x-\frac{1}{4}\sin 4x\right]_0^\pi\\
&=\frac{1}{8}\left[\pi-0-(0-0)\right]\\
&=\boxed{\frac{\pi}{8}.}
\end{align*}
Answer: B.
\end{proof}
\begin{problem}
Evaluate the following integral
\[
\int_0^{\pi/4}\sec^4 x\tan x\;dx.
\]
\end{problem}
\begin{proof}[Solution]
The following identities are useful
\begin{equation}
  \label{eq:tangent-pythagorean-identity}
\sec^2\theta-\tan^2\theta=1.
\end{equation}
Substitute $u=\tan x$, $du=\sec^2 x\;dx$
\begin{align*}
\int_0^{\pi/4}\sec^4 x\tan x\;dx
&=\int_0^{\pi/4}\sec^2 x\sec^2 x\tan x\;dx\\
&=\int_0^{\pi/4}\sec^2 x(1+\tan^2 x)\tan x\;dx\\
&=\int_0^1(1+u^2)u\;dx\\
&=\int_0^1u+u^3\;dx\\
&=\left[\frac{1}{2}u^2+\frac{1}{4}u^4\right]_0^1\\
&=\frac{1}{2}+\frac{1}{4}-0-0\\
&=\boxed{\frac{3}{4}.}
\end{align*}
Answer: A.
\end{proof}
\begin{problem}
After the appropriate trigonometric substitution, the integral
\[
\int_2^5\frac{dx}{\sqrt{x^2-4x+13}}
\]
becomes
\end{problem}
\begin{proof}[Solution]
The general strategy for these types of problems is to complete the square
in the denominator and make some sort of trig substitution
\[
x^2-4x+13=
\left(x^2-4x+4\right)+9=
(x-2)^2+9.
\]
Make the $u$-substitution $u=(x-2)/3$, $du=dx/3$
\begin{align*}
\int_2^5\frac{dx}{\sqrt{x^2-4x+13}}
&=\int_2^5\frac{dx}{3\sqrt{(x-2)^2/9+1}}\\
&=\frac{1}{3}\int_2^5\frac{dx}{\sqrt{\left(\frac{x-2}{3}\right)^2+1}}\\
&=\int_0^1\frac{du}{\sqrt{u^2+1}}\\
\intertext{follow it up with the trig substitution $\tan\theta=u$,
  $\sec^2\theta\;d\theta=du$, $0\leq\theta\leq\pi/4$}
&=\int_0^{\pi/4}\sec^2\theta\cos\theta\;d\theta\\
&=\int_0^{\pi/4}\sec\theta\;d\theta\\
&=\left[\ln|\sec\theta+\tan\theta|\right]_0^{\pi/4}\\
&=\ln\left|\frac{\sec\pi/4+\tan\pi/4}{\sec 0+\tan 0}\right|\\
&=\ln\left|\frac{\sqrt{2}+1}{1+0}\right|\\
&=\boxed{\ln\left|\sqrt{2}+1\right|}
\end{align*}
Answer: A.
\end{proof}
\begin{problem}
Compute
\[
\int_3^4\frac{3}{x^2-x-2}\;dx.
\]
\end{problem}
\begin{proof}[Solution]
Factor
\[
x^2-x-2=(x-2)(x+1)
\]
and use partial fractions
\begin{align*}
\frac{3}{(x-2)(x+1)}&=\frac{A}{x-2}+\frac{B}{x+1}\\
3&=A(x+1)+B(x-2)\\
0x+3&=(A+B)x+A-2B
\end{align*}
gives you $A-2B=3$, $A+B=0$ so $A=-B$, $-B-2B=3$, $B=-1$ and $A=1$. Now we
can compute the integral
\begin{align*}
\int_3^4\frac{3}{x^2-x-2}\;dx
&=\int_3^4\left[\frac{1}{x-2}-\frac{1}{x+1}\right]dx\\
&=\left[\ln|x-2|-\ln|x+1|\right]_3^4\\
&=\left[\ln\left|\frac{x-2}{x+1}\right|\right]_3^4\\
&=\ln\left|\frac{2}{5}\right|-\ln\left|\frac{1}{4}\right|\\
&=\ln\left|\frac{2/5}{1/4}\right|\\
&=\boxed{\ln\left|\frac{8}{5}\right|.}
\end{align*}
Remember your log properties!
\\\\
Answer: B.
\end{proof}
\begin{problem}
It is known that
\[
\int\frac{2x-3}{x\left(x^2+1\right)}dx=a\ln x+b\ln\left(x^2+1\right)+c\tan x+C
\]
for some constants $a$, $b$, $c$ and $C$. What is $b$?
\end{problem}
\begin{proof}[Solution]
There is a typo in the original problem; instead of $c\tan x$ it should be
a $c\tan^{-1}x$. One thing you can do is use the fundamental theorem of
calculus
\begin{equation}
  \label{eq:fundamental-theorem-of-calculus}
f(x)=\frac{d}{dt}\int_a^x f(t)\;dt.
\end{equation}
Applying the fundamental theorem on our function, we get
\begin{align*}
\frac{2x-3}{x\left(x^2+1\right)}
&=\frac{a}{x}+\frac{2bx}{x^2+1}+\frac{c}{x^2+1}\\
&=\frac{a}{x}+\frac{2bx+c}{x^2+1}\\
&=\frac{a\left(x^2+1\right)+(2bx+c)x}{x\left(x^2+1\right)}\\
&=\frac{(a+2b)x^2+cx+a}{x\left(x^2+1\right)}.
\end{align*}
Now we solve for the values in the numerator by noting that $a+2b=0$, $c=2$
and $a=-3$, so $b=3/2$.
\\\\
Answer: E.
\end{proof}
\begin{problem}
Evaluate the integral
\[
\int\frac{x^2+5x+1}{\left(x^2+1\right)^2}dx.
\]
\end{problem}
\begin{proof}[Solution]
Remember, you cannot use the method of partial fractions if the highest
term in the numerator is greater than or equal to the highest term in the
denominator, here we have $x^2$ on the top and $x^4$ on the bottom, if you
expand the square, so we should be okay. Then by partial fractions, we can
write
\[
\frac{x^2+5x+1}{\left(x^2+1\right)^2}=\frac{A}{x^2+1}+\frac{B}{\left(x^2+1\right)^2}
\]
\end{proof}
\begin{problem}
\end{problem}
\begin{proof}[Solution]
\end{proof}
\begin{problem}
\end{problem}
\begin{proof}[Solution]
\end{proof}
\begin{problem}
\end{problem}
\begin{proof}[Solution]
\end{proof}
\begin{problem}
\end{problem}
\begin{proof}[Solution]
\end{proof}
\begin{problem}
\end{problem}
\begin{proof}[Solution]
\end{proof}
\begin{problem}
\end{problem}
\begin{proof}[Solution]
\end{proof}

\chapter{MA 166 Exam 2, Spring 2015}
\begin{problem}
\end{problem}
\begin{proof}[Solution]
\end{proof}
\begin{problem}
\end{problem}
\begin{proof}[Solution]
\end{proof}
\begin{problem}
\end{problem}
\begin{proof}[Solution]
\end{proof}
\begin{problem}
\end{problem}
\begin{proof}[Solution]
\end{proof}
\begin{problem}
\end{problem}
\begin{proof}[Solution]
\end{proof}
\begin{problem}
\end{problem}
\begin{proof}[Solution]
\end{proof}
\begin{problem}
\end{problem}
\begin{proof}[Solution]
\end{proof}
\begin{problem}
\end{problem}
\begin{proof}[Solution]
\end{proof}
\begin{problem}
\end{problem}
\begin{proof}[Solution]
\end{proof}
\begin{problem}
\end{problem}
\begin{proof}[Solution]
\end{proof}
\begin{problem}
\end{problem}
\begin{proof}[Solution]
\end{proof}
\begin{problem}
\end{problem}
\begin{proof}[Solution]
\end{proof}

%%% Local Variables:
%%% TeX-master: "../MA166-Recitation"
%%% End:
