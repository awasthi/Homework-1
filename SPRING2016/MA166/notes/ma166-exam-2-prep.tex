As I promised, here are somewhat detailed solutions to two of the last Exam
2's for MA 166.
\chapter{MA 166 Exam 2, Spring 2015}
\begin{problem}
Evaluate the following integral
\[
\int_0^\pi \sin^2x \cos^2 x\;dx
\]
\end{problem}
\begin{proof}[Solution]
The following trig identities are useful
\begin{equation}
  \label{eq:sine-double-angle-formula}
\sin 2\theta=2\sin\theta\cos\theta
\end{equation}
\begin{equation}
\label{eq:sine-square-formula}
\sin^2\theta=\frac{1-\cos 2\theta}{2}.
\end{equation}
With that in mind we compute the integral
\begin{align*}
\int_0^\pi \sin^2x \cos^2 x\;dx
&=\int_0^\pi\left(\sin x\cos x\right)^2\;dx\\
&=\int_0^\pi\left(\frac{1}{2}\sin 2x\right)^2\;dx\\
&=\frac{1}{4}\int_0^\pi\sin^2 2x\;dx\\
&=\frac{1}{4}\int_0^\pi\frac{1-\cos 4x}{2}\;dx\\
&=\frac{1}{8}\int_0^\pi 1-\cos 4x\;dx\\
&=\frac{1}{8}\left[x-\frac{1}{4}\sin 4x\right]_0^\pi\\
&=\frac{1}{8}\left[\pi-0-(0-0)\right]\\
&=\boxed{\frac{\pi}{8}.}
\end{align*}
Answer: B.
\end{proof}
\begin{problem}
Evaluate the following integral
\[
\int_0^{\pi/4}\sec^4 x\tan x\;dx.
\]
\end{problem}
\begin{proof}[Solution]
The following identities are useful
\begin{equation}
  \label{eq:tangent-pythagorean-identity}
\sec^2\theta-\tan^2\theta=1.
\end{equation}
Substitute $u=\tan x$, $du=\sec^2 x\;dx$
\begin{align*}
\int_0^{\pi/4}\sec^4 x\tan x\;dx
&=\int_0^{\pi/4}\sec^2 x\sec^2 x\tan x\;dx\\
&=\int_0^{\pi/4}\sec^2 x(1+\tan^2 x)\tan x\;dx\\
&=\int_0^1(1+u^2)u\;dx\\
&=\int_0^1u+u^3\;dx\\
&=\left[\frac{1}{2}u^2+\frac{1}{4}u^4\right]_0^1\\
&=\frac{1}{2}+\frac{1}{4}-0-0\\
&=\boxed{\frac{3}{4}.}
\end{align*}
Answer: A.
\end{proof}
\begin{problem}
After the appropriate trigonometric substitution, the integral
\[
\int_2^5\frac{dx}{\sqrt{x^2-4x+13}}
\]
becomes
\end{problem}
\begin{proof}[Solution]
The general strategy for these types of problems is to complete the square
in the denominator and make some sort of trig substitution
\[
x^2-4x+13=
\left(x^2-4x+4\right)+9=
(x-2)^2+9.
\]
Make the $u$-substitution $u=(x-2)/3$, $du=dx/3$
\begin{align*}
\int_2^5\frac{dx}{\sqrt{x^2-4x+13}}
&=\int_2^5\frac{dx}{3\sqrt{(x-2)^2/9+1}}\\
&=\frac{1}{3}\int_2^5\frac{dx}{\sqrt{\left(\frac{x-2}{3}\right)^2+1}}\\
&=\int_0^1\frac{du}{\sqrt{u^2+1}}\\
\intertext{follow it up with the trig substitution $\sec\theta=u$,
  $\sec\theta\tan\theta\;d\theta=du$}
&=\int_{\pi/2}^0
\end{align*}
\end{proof}
\begin{problem}
\end{problem}
\begin{proof}[Solution]
\end{proof}
\begin{problem}
\end{problem}
\begin{proof}[Solution]
\end{proof}
\begin{problem}
\end{problem}
\begin{proof}[Solution]
\end{proof}
\begin{problem}
\end{problem}
\begin{proof}[Solution]
\end{proof}
\begin{problem}
\end{problem}
\begin{proof}[Solution]
\end{proof}
\begin{problem}
\end{problem}
\begin{proof}[Solution]
\end{proof}
\begin{problem}
\end{problem}
\begin{proof}[Solution]
\end{proof}
\begin{problem}
\end{problem}
\begin{proof}[Solution]
\end{proof}
\begin{problem}
\end{problem}
\begin{proof}[Solution]
\end{proof}

\chapter{MA 166 Exam 2, Spring 2014}
\begin{problem}
\end{problem}
\begin{proof}[Solution]
\end{proof}
\begin{problem}
\end{problem}
\begin{proof}[Solution]
\end{proof}
\begin{problem}
\end{problem}
\begin{proof}[Solution]
\end{proof}
\begin{problem}
\end{problem}
\begin{proof}[Solution]
\end{proof}
\begin{problem}
\end{problem}
\begin{proof}[Solution]
\end{proof}
\begin{problem}
\end{problem}
\begin{proof}[Solution]
\end{proof}
\begin{problem}
\end{problem}
\begin{proof}[Solution]
\end{proof}
\begin{problem}
\end{problem}
\begin{proof}[Solution]
\end{proof}
\begin{problem}
\end{problem}
\begin{proof}[Solution]
\end{proof}
\begin{problem}
\end{problem}
\begin{proof}[Solution]
\end{proof}
\begin{problem}
\end{problem}
\begin{proof}[Solution]
\end{proof}
\begin{problem}
\end{problem}
\begin{proof}[Solution]
\end{proof}

%%% Local Variables:
%%% TeX-master: "../MA166-Recitation"
%%% End:
