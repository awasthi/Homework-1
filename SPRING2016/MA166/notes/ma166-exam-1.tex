\chapter{Review Notes for Exam 1}
This is a review sheet for Exam 1 for MA 166. This is by no means the only
resource you should use to study for the exam, but I hope it will serve as
a good review for some of the techniques you have learned thus far.
\section{Vectors and the Geometry of Space}
\subsection{Three-Dimensional Coordinate Systems}
In this section you first learn about the
\href{https://en.wikipedia.org/wiki/Right-hand_rule}{right-hand rule} and
right handed coordinate systems. This is really just a mathematical
\emph{convention} that we follow because we like the cross product of two
``positive'' vectors, i.e.\,vectors in the first quadrant of the
$xy$-plane, to point out of the plane. Keep this in mind as you continue
studying the natural sciences.

A cute way to figure out whether you have a right-handed coordinate system
is this:
\begin{quote}
If you are right-handed, imagine holding a coffee mug with your
right hand, your thumb pointing up towards the $z$-axis, then your fingers
wrapped around the handle of the mug will traverse first the $x$-axis, then
end up on the $y$-axis.
\end{quote}
I noticed a lot of students were having trouble with describing equations
and inequalities in $\bbR^2$ and $\bbR^3$. When you see an equation like
``What does the equation $x=3$ represent in $\bbR^3$?'' your first thought
should be, what is a point in the graph of this equation? The equation is
telling us, no matter what choice of $y$ and $z$ we make, $x$ will always
be $3$. Thus, the points $(3,1,0)$ and $(3,0,1)$ are in the graph of the
equation $x=3$, but $(2,0,0)$ is not because we have the constraint that
$x$ must equal $3$. You can already more or less see what this is going to
look like. If you draw the line from $(3,1,0)$ to $(3,0,1)$ every point on
the line will be in the graph of $x=3$ and if you pick any other point in
$x=3$ and draw the line from $(3,1,0)$ to it, the same will be true, so
$x=3$ must be a plane perpendicular to the $x$-axis which intersects the
$x$-axis at $(3,0,0)$.

Now, what does the equation $x=3$ represent in $\bbR^2$?

Of course, you should also know the general equation of a sphere centered at
$(x_0,y_0,z_0)$ with radius $r$:
\begin{equation}
  \label{eq:sphere}
(x-x_0)^2+(y-y_0)^2+(z-z_0)^2=r^2.
\end{equation}
When you see a quadratic equation, i.e.,an equation with terms like $x^2$,
$y^2$, $z^2$, you should try completing the square and simplifying it. For
example, suppose we are asked what the following expression represents
\[
x^2+y^2+z^2-6x-4y+6z=0?
\]
First, you gather all your like terms and put them next to each other like
this
\[
\left(x^2-6x\right)+\left(y^2-4y\right)+\left(z^2+6z\right)=0.
\]
Next, you complete the square, i.e., you add whatever terms you need to add
to the parenthesized polynomials to turn it into the square of a linear
polynomial (a linear polynomial looks like $ax+b$, or $a'y+b'$, or
$a''z+b''$, etc.) so we have
\[
(x^2-6x+9)^2+(y^2-4y+4)+(z^2+6z+9)=9+4+9.
\]
\textbf{Don't forget} than when you are completing the square, you are
adding terms, so you are changing your original equation, you must add the
same terms to the right-hand side to balance the equation! Now you just
need to recognize that, because the coefficient in front of $x$ is negative
(the same for $y$) and $(x+a)^2=x^2+2ax+a^2$, then we must be looking at
the square of negative $-3$ (the same is true of the coefficient in front
of $y$) so we have
\[
(x-3)^2+(y-2)^2+(z+3)^2=22.
\]
Now we can read off the values: The equation $x^2+y^2+z^2-6x-4y+6z=0$
represents a sphere of radius $\sqrt{22}$ centered at $(3,2,-3)$

\subsection{}
Now, if you find it a little difficult to remember \underline{$\comp$} and
\underline{$\proj$} perhaps the following equation will help you see the
relationship between the scalar projection and the projection
\begin{equation}
  \label{eq:comp-to-proj}
\proj_{\bfv}\bfw=
\frac{\bfv\cdot\bfw}{|\bfv|^2}\bfv=
\left(\frac{\bfv\cdot\bfw}{|\bfv|}\right)\frac{\bfv}{|\bfv|}
=\comp_{\bfv}\bfw\frac{\bfv}{|\bfv|}.
\end{equation}
In fact, the scalar projection is just the \emph{signed} magnitude of
$\proj_{\bfv}\bfw$ since

\section{Integration}

%%% Local Variables:
%%% mode: latex
%%% TeX-master: "../MA166-Recitation"
%%% End:
