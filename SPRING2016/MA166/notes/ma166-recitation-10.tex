\chapter{Homework}
Here is this week's WebAssign homework and outlines of the solutions to the
problems.

\bigskip

\textkr{성준아}, here is the material you need to know to give this
recitation tomorrow. The subject for tomorrow's recitation are the
different convergence tests for series found in sections \S 11.6
to 11.8 of Stewart's book. Here is a rough summary of the material on those
sections:
\section*{Stewart \S 11.6--\S 11.8 Summary}
\subsection*{\S 11.6: Absolute Convergence and the Ratio and Root Tests}
\begin{definition}
A series $\sum a_n$ is called \emph{absolutely convergent} if the series of
absolute values $\sum |a_n|$ is convergent.
\end{definition}
\begin{definition}
A series $\sum a_n$ is called \emph{conditionally convergent} if it is
convergent but not absolutely convergent.
\end{definition}
\begin{theorem}
If a series $\sum a_n$ is absolutely convergent, then it is conditionally
convergent.
\end{theorem}
\begin{theorem}[The Ratio Test]
\begin{enumerate}[label=\textnormal{(\roman*)},noitemsep]
\item If $\lim\left|a_{n+1}/a_n\right|=L<1$, then the series
  $\sum a_n$ is absolutely convergent (and therefore convergent).
\item If $\lim\left|a_{n+1}/a_n\right|=L>1$ or $\infty$, then the series
  $\sum a_n$ is divergent.
\item If $\lim\left|a_{n+1}/a_n\right|=L=1$, the \emph{ratio test} is
  inconclusive; that is, conclusion can be drawn about the convergence or
  divergence of $sum a_n$.
\end{enumerate}
\end{theorem}
\begin{theorem}[The Root Test]
\begin{enumerate}[label=\textnormal{(\roman*)},noitemsep]
\item If $\lim\sqrt[n]{|a_n|}=L<1$, the series $\sum a_n$ is is absolutely
  convergent (and therefore convergent).
\item If $\lim\sqrt[n]{|a_n|}=L>1$ or $\infty$, then the series $\sum
  a_n$ is divergent.
\item If $\lim\sqrt[n]{|a_n|}=L=1$, the \emph{root test} is inconclusive.
\end{enumerate}
\end{theorem}
\subsection*{\S 11.7: Strategy for Testing Series}
\begin{enumerate}[label=\textbf{\arabic*.}]
\item If the series is of the form $\sum 1/n^p$, it is a $p$-series, which
  we know to be convergent if $p>1$ and divergent if $p\leq 1$.
\item If the series has the form $\sum ar^{n-1}$ or $\sum ar^n$, it is a
  geometric series, which converges if $|r|<1$ and diverges if $|r|\geq
  1$.
\item If the series has a form similar to a $p$-series or geometric series,
  then one of the comparison tests should be considered.
\item If you can see $\lim a_n\neq 0$, then the test for divergence should
  be used.
\item If the series has the form $\sum(-1)^{n+1} b_n$ or $\sum (-1)^n b_n$,
  the alternating test is an obvious choice.
\item Series that involve factorials and other products are handled
  conveniently with the ratio test.
\item If $a_n$ has the form ${b_n}^n$, then the root test may be useful.
\item If $a_n=f(n)$, where $\int_1^\infty f(x)\diff x$ is easily evaluated,
  the integral test is effective.
\end{enumerate}
\subsection*{\S 11.8: Power Series}
\begin{definition}
A \emph{power series} is a series of the form
\[
\sum_{n=0}^\infty c_nx^n=c_0+c_1x+c_2x^2+\dotsb
\]
where $x$ is a variable and the $c_n$'s are constants called
\emph{coefficients} of the series.
\end{definition}
\begin{definition}
More generally, a series of the form
\[
\sum_{n=0}^\infty c_n(x-a)^n=c_0+c_1(x-a)+c_2+(x-a)^2+\dotsb
\]
is called a \emph{power series in $(x-a)$} or a \emph{power series centered
at $a$} or a \emph{power series series about $a$}.
\end{definition}
\begin{theorem}
For a given power series $\sum c_n(x-a)^n$ there are only three
possibilities:
\begin{enumerate}[label=\textnormal{(\roman*)},noitemsep]
\item The series converges only when $x=a$.
\item The series converges for all $x$.
\item There is a positive number $R$ such that the series converges if
  $|x-a|<R$ and diverges if $|x-a|>R$.
\end{enumerate}
\end{theorem}
The number $R$ in (iii) is called the \emph{radius of convergence} of the
power series. By convention, $R=0$ in case (i) and $\infty$ in case (ii)>
The \emph{interval of convergence} of a power series is the interval that
consists of all values of $x$ fr which the series converges.
\begin{center}
\begin{tabular}{|c|c|c|c|}
\hline
&Series&Radius of convergence&Interval of convergence\\
\hline
&&&\\
Geometric series&$\displaystyle\sum x^n$&$R=1$&$(-1,1)$\\
Example 1&$\displaystyle\sum n!x^n$&$r=0$&$\{0\}$\\
Example 2&$\displaystyle\sum\frac{(x-3)^n}{n}$&$R=1$&$[2,4)$\\
Example
  3&$\displaystyle\sum\frac{(-1)^nx^{2n}}{2^{2n}(n!)^2}$&$R=\infty$&$(-\infty,\infty)$\\
&&&\\
\hline
\end{tabular}
\end{center}
Now
\newpage
\section*{Homework 25}
\begin{problem}[WebAssign HW 25, \# 1]
Determine whether the series is absolutely convergent, conditionally
convergent, or divergent.
\[
\sum_{n=1}^\infty\frac{n}{\sqrt{n^3-9}}.
\]
\end{problem}
\begin{problem}[WebAssign HW 25, \# 2]
Determine whether the series is absolutely convergent, conditionally
convergent, or divergent.
\[
\sum_{n=1}^\infty\frac{\sin 7n}{5^n}.
\]
\end{problem}
\begin{problem}[WebAssign HW 25, \# 3]
Determine whether the series is absolutely convergent, conditionally
convergent, or divergent.
\[
\sum_{n=1}^\infty\frac{12^n}{(n+1) 5^{2n+1}}.
\]
\end{problem}
\begin{problem}[WebAssign HW 25, \# 4]
Determine whether the series is absolutely convergent, conditionally
convergent, or divergent.
\[
\sum_{n=2}^\infty\frac{(-1)^n}{\ln 6n}.
\]
\end{problem}
\begin{problem}[WebAssign HW 25, \# 5]
The terms of a series are defined recursively by the equations
\[
a_1=4\qquad a_{n+1}=\frac{7n+1}{3n+9}\cdot a_n.
\]
Determine whether $\sum a_n$ is absolutely convergent, conditionally
convergent, or divergent.
\end{problem}

\section*{Homework 27}
\begin{problem}[WebAssign HW 27, \# 1]
Determine whether the series is absolutely convergent, conditionally
convergent, or divergent.
\[
\sum_{n=1}^\infty\left(\frac{n^2+4}{5n^2+2}\right)^n.
\]
\end{problem}
\begin{problem}[WebAssign HW 27, \# 2]
Test the series for convergence or divergence.
\[
\sum_{n=1}^\infty\frac{1}{n+6^n}
\]
\end{problem}
\begin{problem}[WebAssign HW 27, \# 3]
Test the series for convergence or divergence.
\[
\sum_{n=1}^\infty\frac{(4n+1)^n}{n^{5n}}.
\]
\end{problem}
\begin{problem}[WebAssign HW 27, \# 4]
Test the series for convergence or divergence.
\[
\sum_{n=1}^\infty (-1)^n\frac{5n}{n+3}.
\]
\end{problem}
\begin{problem}[WebAssign HW 27, \# 5]
Test the series for convergence or divergence.
\[
\sum_{n=1}^\infty\frac{n^28^{n-1}}{(-9)^n}.
\]
\end{problem}
\begin{problem}[WebAssign HW 27, \# 6]
Test the series for convergence or divergence.
\[
\sum_{n=1}^\infty \frac{1}{8n+5}.
\]
\end{problem}
\begin{problem}[WebAssign HW 27, \# 7]
Test the series for convergence or divergence.
\[
\sum_{n=2}^\infty\frac{1}{n\sqrt{\ln 7n}}.
\]
\end{problem}
\begin{problem}[WebAssign HW 27, \# 8]
Test the series for convergence or divergence.
\[
\sum_{k=1}^\infty\frac{6^kk!}{(k+2)!}.
\]
\end{problem}
\begin{problem}[WebAssign HW 27, \# 9]
Test the series for convergence or divergence.
\[
\sum_{n=1}^\infty\frac{6n!}{e^{n^2}}.
\]
\end{problem}
\section*{Homework 28}
\begin{problem}[WebAssign HW 28, \# 1]
Find the radius of convergence, $R$, of the series.
\[
\sum_{n=2}^\infty\frac{x^{n+1}}{2n!}.
\]
Find the interval, $I$, of convergence of the series.
\end{problem}
\begin{problem}[WebAssign HW 28, \# 2]
Find the radius of convergence, $R$, of the series.
\[
\sum_{n=1}^\infty\frac{8^nx^n}{n^3}.
\]
Find the interval, $I$, of convergence of the series.
\end{problem}
\begin{problem}[WebAssign HW 28, \# 3]
Find the radius of convergence, $R$, of the series.
\[
\sum_{n=1}^\infty\frac{(-5)^n}{n\sqrt{n}}x^n.
\]
Find the interval, $I$, of convergence of the series.
\end{problem}
\begin{problem}[WebAssign HW 28, \# 4]
Find the radius of convergence, $R$, of the series.
\[
\sum_{n=2}^\infty(-1)^n\frac{x^n}{12^n\ln n}.
\]
Find the interval, $I$, of convergence of the series.
\end{problem}
\begin{problem}[WebAssign HW 28, \# 5]
Find the radius of convergence, $R$, of the series.
\[
\sum_{n=0}^\infty\frac{(x-4)^n}{n^6+1}.
\]
Find the interval, $I$, of convergence of the series.
\end{problem}
\begin{problem}[WebAssign HW 28, \# 6]
Find the radius of convergence, $R$, of the series.
\[
\sum_{n=1}^\infty\frac{6^n(x+4)^n}{\sqrt{n}}.
\]
Find the interval, $I$, of convergence of the series.
\end{problem}
\begin{problem}[WebAssign HW 28, \# 7]
Find the radius of convergence, $R$, of the series.
\[
\sum_{n=1}^\infty\frac{(x-6)^n}{n^n}.
\]
Find the interval, $I$, of convergence of the series.
\end{problem}
\begin{problem}[WebAssign HW 28, \# 8]
Find the radius of convergence, $R$, of the series.
\[
\sum_{n=1}^\infty\frac{n}{b^n}(x-a)^n, \qquad b>0.
\]
Find the interval, $I$, of convergence of the series.
\end{problem}
\begin{problem}[WebAssign HW 28, \# 9]
Find the radius of convergence, $R$, of the series.
\[
\sum_{n=1}^\infty n!(3x-1)^n.
\]
Find the interval, $I$, of convergence of the series.
\end{problem}
\begin{problem}[WebAssign HW 28, \# 10]
Find the radius of convergence, $R$, of the series.
\[
\sum_{n=2}^\infty\frac{x^{6n}}{n(\ln n)^8}.
\]
Find the interval, $I$, of convergence of the series.
\end{problem}
\newpage
\chapter{Relevant Exam Problems}
If you run out of things to talk about within the first few minutes, talk
about these problems
\begin{problem}[Exam 3, Spring 2015, \# 2]
The series
\[
\sum_{n=1}^\infty\frac{1}{\left( n^{3\alpha}+9 \right)^{1/8}}
\]
if and only if $\alpha$ is?
\end{problem}
\begin{proof}[Solution]
\end{proof}
\begin{problem}[Exam 3, Spring 2015, \# 3]
Test the following series for convergence or divergence.
\begin{enumerate}[label=(\alph*)]
\item $\displaystyle\sum_{n=1}^\infty n\sin\left(\frac{1}{n}\right)$.
\item $\displaystyle\sum_{n=1}^\infty(-1)^n\arctan(\pi/2n)$.
\item $\displaystyle\sum_{n=1}^\infty\frac{n^2+9}{(n^3+4)\sqrt{n}}$.
\end{enumerate}
\end{problem}
\begin{proof}[Solution]
\end{proof}
\begin{problem}[Exam 3, Spring 2015, \# 7]
Suppose that the powers
\[
\sum_{n=0}^\infty c_n(x-5)^n
\]
converges when $x=2$ and diverges when $x=10$.

From the above information, which of the following statements can we
conclude to be true?
\begin{enumerate}[label=\MakeUppercase{\roman*}.]
\item The radius of convergence $R$ satisfies $3\leq R\leq 5$.
\item We \emph{cannot} determine the interval of convergence from the above
  information only.
\item The derivative of the power series is $\sum_{n=1}^\infty
  nc_n(x-5)^{n-1}$ which converges when $x=3$.
\end{enumerate}
\end{problem}
\begin{proof}[Solution]
\end{proof}

\begin{center}
{\large\textkr{고마워,성준!}}
\end{center}

%%% Local Variables:
%%% mode: latex
%%% TeX-master: "../MA166-Recitation"
%%% End:
