\chapter{Last Recitation}
For the last recitation, we'll go over some exam problems and potentially
review some of the material you need to know for the final. We will pull
exam questions from the last couple of MA 166 finals.

\section{Exam 1 Material}
Perhaps the best way to start this section is by working out some problems.
\begin{problem}
If the sphere given by $x^2+y^2+z^2+8x-4y+6z=C$ passes through the point
$(-1,2,1)$. Then the radius is:
\end{problem}
\begin{proof}[Solution]
This problem may seem tricky at first, but you know how to do this. Before
we do anything at all, note that the equation
\[
x^2+y^2+z^2+8x-4y+6z=C
\]
contains an unknown constant, namely, $C$. We are told something about the
``sphere'', namely, that it passes through the point $(-1,2,1)$. Therefore,
if we plug in $(-1,2,1)$ into our equation, we should be able to solve for
$C$
\begin{align*}
C&=(-1)^2+2^2+1^2+8\cdot(-1)-4\cdot 2+6\cdot 1\\
 &=1+4+1-8-8+6\\
 &=-4.
\end{align*}
Now we can solve for the radius by putting our equation in the form
\[
r^2=(x-x_0)^2+(y-y_0)^2+(z-z_0)^2.
\]
Let's do this. Using the method of
\href{https://en.wikipedia.org/wiki/Completing_the_square}{\emph{completing
  the square}}, we have
\begin{align*}
0&=x^2+y^2+z^2+8x-4y+6z+4\\
 &=(x^2+8x)+(y^2-4y)+(z^2+6z)+4\\
 &=(x^2+8x+16)+(y^2-4y+4)+(z^2+6z+9)+2-(16+4+9)\\
16+4+9-4&=(x+4)^2+(y-2)^2+(z+3)^2\\
25&=(x+4)^2+(y-2)^2+(z+3)^2.
\end{align*}
Hence, the radius is $r=\sqrt{25}=5$. Answer: \textbf{E}.
\end{proof}

\begin{problem}
If the two vectors $\langle y,3,2 \rangle$ and $\langle y,-4,-2y \rangle$
are perpendicular to each other, then the value of $y$ is:
\end{problem}
\begin{proof}[Solution]
Recall what it means for two vectors to be
\href{https://en.wikipedia.org/wiki/Dot_product#Properties}{\emph{perpendicular}}:
\begin{quote}
Two vectors $\bfv$ and $\bfu$ are perpendicular if and only if
$\bfv\cdot\bfu=0$.
\end{quote}
Hence, we have
\[
\langle y,3,2 \rangle\cdot\langle y,-4,-2y \rangle=y^2-4y-12.
\]
Using the
\href{https://en.wikipedia.org/wiki/Quadratic_formula}{\emph{quadratic
    formula}}, we have
\begin{align*}
r_1,r_2&=\frac{-(-4)\pm\sqrt{(-4)^2-4(-12)}}{2}\\
       &=\frac{4}{2}\pm\frac{\sqrt{4\cdot 4+4\cdot 12}}{2}\\
       &=2\pm\sqrt{\frac{4\cdot 4+4\cdot 12}{4}}\\
       &=2\pm\sqrt{4+12}\\
       &=2\pm 4.
\end{align*}
Hence, $y=-2,6$. Answer: \textbf{A}.
\end{proof}

\section{Exam 2 Material}
\section{Exam 3 Material}
\section{Final Material}
This section is for material that doesn't quite fit into the last three
sections.


%%% Local Variables:
%%% mode: latex
%%% TeX-master: "../MA166-Recitation"
%%% End:
