\chapter{Last Recitation}
For the last recitation, we'll go over some exam problems and potentially
review some of the material you need to know for the final. We will pull
exam questions from the last couple of MA 166 finals.

\section{Exam 1 Material}
Perhaps the best way to start this section is by working out some problems.
\begin{problem}
If the sphere given by $x^2+y^2+z^2+8x-4y+6z=C$ passes through the point
$(-1,2,1)$. Then the radius is:
\end{problem}
\begin{proof}
This problem may seem tricky at first, but you know how to do this. Before
we do anything at all, note that the equation
\[
x^2+y^2+z^2+8x-4y+6z=C
\]
contains an unknown constant, namely, $C$. We are told something about the
``sphere'', namely, that it passes through the point $(-1,2,1)$. Therefore,
if we plug in $(-1,2,1)$ into our equation, we should be able to solve for
$C$
\begin{align*}
C&=(-1)^2+2^2+1^2+8\cdot(-1)-4\cdot 2+6\cdot 1\\
 &=1+4+1-8-8+6\\
 &=-2.
\end{align*}
Now we can solve for the radius by putting our equation in the form
\[
r^2=(x-x_0)^2+(y-y_0)^2+(z-z_0)^2.
\]
Let's do this. Define
\[
f(x,y,z)\coloneqq
\begin{cases}
x^2&x<y\\
y^2&y=x\\
0&z=0.
\end{cases}
\]
\end{proof}


\begin{problem}
\end{problem}

\section{Exam 2 Material}
\section{Exam 3 Material}
\section{Final Material}
This section is for material that doesn't quite fit into the last three
sections.


%%% Local Variables:
%%% mode: latex
%%% TeX-master: "../MA166-Recitation"
%%% End:
