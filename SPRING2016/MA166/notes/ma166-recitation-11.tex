\chapter{Homework}
\section{This Week's Summary}
Here's a summary of the material that was (presumably) covered this week.
From Stewart, \S 11.8 to \S 11.10, we have
\subsection{\S 11.8: Power Series}
A \emph{power series} is a series of the form
\begin{equation}
  \label{eq:power-series-def}
\sum_{n=0}^\infty c_nx^n=c_0+c_1x+c_2x^2+c_3x^3+\dotsb
\end{equation}
where $x$ is a variable and the $c_n$'s are constant called the
\emph{coefficients} of the series. For each fixed $x$, the series
\eqref{eq:power-series-def} is a series of constants that we can test for
convergence or divergence. A power series may converge for some values of
$x$ and diverge for other values of $x$. The sum of the series is a
function
\[
f(x)=c_0+c_1x+c_2x^2+\dotsb+c_nx^x+\dotsb
\]
whose domain is the set of all $x$ for which the series converges. Notice
that $f$ resembles a polynomial. The only difference is that $f$ has
infinitely many terms.

\begin{theorem}
For a given power series $\sum_{n=0}^\infty c_n(x-a)^n$ there are only
three possibilities:
\begin{enumerate}[label=\textnormal{(\roman*)}]
\item The series converges only when $x=a$.
\item The series converges for all $x$.
\item There is a positive number $R$ such that the series converges if
  $|x-a|<R$ and diverges if $|x-a|>R$.
\end{enumerate}
\end{theorem}

The number $R$ in case (iii) is called the \emph{radius of convergence} of
the power series. By convention, the radius of convergence is $R=0$ in case
(i) and $R=\infty$ in case (ii). The \emph{interval of convergence} of a
power series is the interval that consists of all values of $x$ for which
the series converges. In

\subsection{\S 11.9: Representation of Functions as Power Series}
We start with an equation that we've seen before
\begin{equation}
\label{eq:geometri-series-representation}
\frac{1}{1-x}=1+x+x^2+x^3+\dotsb=\sum_{n=0}^\infty x^n\qquad |x|<1.
\end{equation}
\begin{theorem}
If the power series $\sum c_n(x-a)^n$ has radius of convergence $R>0$ then
the function $f$ defined by
\[
f(x)=c_0+c_1(x-a)+c_2(x-a)^2+\dotsb=\sum_{n=0}^\infty c_n(x-a)^n
\]
is differentiable (and therefore continuous) on the interval $(a-R,a+R)$
and
\begin{enumerate}[label=\textnormal{(\roman*)}]
\item $f'(x)=c_1+2c_2(x-a)+3c_3(x-a)^2+\dotsb=\sum_{n=1}^\infty
  nc_n(x-a)^{n-1}$.
\item $\int
  f(x)dx=C+c_0(x-a)+c_1(x-a)^2/2+c_2(x-a)^3/3+\dotsb=C+\sum_{n=0}^\infty
  (x-a)^{n+1}/(n+1)$.
\end{enumerate}
The radii of convergence of the power series in Equations (i) and (ii) are
both $R$.
\end{theorem}

\subsection{\S 11.10: Taylor and Maclaurin Series}
\begin{theorem}
If $f$ has a power series representation (expansion) at $a$, that is, if
\[
f(x)=\sum_{n=0}^\infty c_n(x-a)^n\qquad |x-a|<R
\]
then its coefficients are given by the formula
\[
c_n=\frac{f^{(n)}(a)}{n!}.
\]
\end{theorem}

\begin{equation}
\label{eq:taylor-series-expansion}
\begin{aligned}
f(x)&=\sum_{n=0}^\infty\frac{f^{(n)}(a)}{n!}(x-a)^n\\
&=f(a)+\frac{f'(a)}{1!}(x-a)+\frac{f''(a)}{2!}(x-a)^2+\frac{f'''(a)}{3!}(x-a)^3+\dotsb.
\end{aligned}
\end{equation}
The series in \eqref{eq:taylor-series-expansion} is called the \emph{Taylor
series of the function $f$ at $a$} (or \emph{about $a$} or \emph{centered
at $a$}). For the special case
\[
f(x)=\sum_{n=0}^\infty\frac{f^{(n)}(0)}{n!}x^n=f(0)+\frac{f'(0)}{1!}x+\frac{f''(0)}{2!}x^2+\dotsb.
\]
This case arises frequently enough that it is given the special name
\emph{Maclaurin series}.

\begin{theorem}
If $f(x)=T_n(x)+R_n(x)$, where $T_n$ is the $n$th degree Taylor polynomial
of $f$ at $a$ and
\[
\lim_{n\to\infty}R_n(x)=0
\]
for $|x-a|<R$, then $f$ is equal to the sum of its Taylor series on the
interval $|x-a|<R$.
\end{theorem}

\begin{theorem}[Taylor's inequality]
If $|f^{(n+1)}(x)|\leq M$ for $|x-a|\leq d$, then the remainder $R_n(x)$ of
the Taylor series satisfies the inequality
\[
|R_n(x)|\leq \frac{M}{(n+1)!}|x-a|^{n+1}\qquad\text{for $|x-a|\leq d$.}
\]
\end{theorem}
\begin{equation}
\label{eq:n-factorial-grows-quickly}
\lim_{n\to\infty}\frac{x^n}{n!}=0\qquad\text{for all $x$.}
\end{equation}

\begin{equation}
  \label{eq:exp-equals-its-taylor-ser}
e^x=\sum_{n=0}^\infty\frac{x^n}{n!}\qquad\text{for all $x$.}
\end{equation}

\begin{equation}
\label{eq:sin-taylor-expansion}
\sin x=\sum_{n=0}^\infty (-1)^n\frac{x^{2n+1}}{(2n+1)!}\qquad\text{for all
  $x$.}
\end{equation}

\begin{equation}
  \label{eq:cos-taylor-series}
\cos x=\sum_{n=0}^\infty(-1)^n\frac{x^{2n}}{(2n)!}\qquad\text{for all $x$.}
\end{equation}

The traditional notation for the coefficients in the binomial series is
\[
\binom k n=\frac{k(k-1)(k-2)\dotsm(k-n+1)}{n!}
\]
and these numbers are called the \emph{binomial coefficients.}

\begin{theorem}[The binomial series]
If $k$  is any real number and $|x|<1$, then
\[
(1+x)^k=\sum_{n=0}^\infty\binom k n x^n.
\]
\end{theorem}
Table of Taylor series:

\begin{table}
\centering
\begin{tabular}{|l|l|l|}
\hline
Function&Taylor series&Radius of convergence\\
\hline
$\displaystyle\frac{1}{1-x}$&
$\displaystyle\sum_{n=0}^\infty x^n$&
$R=1$\\
$\displaystyle e^x$&
$\displaystyle\sum_{n=0}^\infty \frac{x^n}{n!}$&
$R=\infty$\\
$\displaystyle\sin x$&
$\displaystyle\sum_{n=0}^\infty (-1)^n\frac{x^{2n+1}}{(2n+1)!}$&
$R=\infty$\\
$\displaystyle\cos x$&
$\displaystyle\sum_{n=0}^\infty(-1)^n\frac{x^{2n}}{(2n)!}$&
$R=\infty$\\
$\displaystyle\arctan x$&
$\displaystyle\sum_{n=0}^\infty(-1)^n\frac{x^{2n+1}}{2n+1}$&
$R=1$\\
$\displaystyle\ln(1+x)$&
$\displaystyle\sum_{n=0}^\infty(-1)^{n-1}\frac{x^n}{n}$&
$R=1$\\
$\displaystyle(1+x)^k$&
$\displaystyle\sum_{n=0}^\infty\binom k n x^n$&
$R=1$\\
\hline
\end{tabular}
\caption{Table of Taylor series.}
\label{tab:taylor-series}
\end{table}

\section{WebAssign Homework}
Solutions to selected problems
\subsection{Homework 28}
\begin{problem}[WebAssign HW 28, \# 1]
Find a power series representation for the function.
\[
f(x)=\frac{1}{9+x}.
\]
Determine the interval of convergence.
\end{problem}

\begin{problem}[WebAssign HW 28, \# 2]
Find a power series representation for the function.
\[
f(x)=\frac{8}{9-x}.
\]
Determine the interval of convergence.
\end{problem}

\begin{problem}[WebAssign HW 28, \# 3]
Find a power series representation for the function.
\[
f(x)=\frac{x}{4+x^2}.
\]
Determine the interval of convergence.
\end{problem}

\begin{problem}[WebAssign HW 28, \# 4]
Find a power series representation for the function.
\[
f(x)=\frac{10}{x^2-2x-24}.
\]
Determine the interval of convergence.
\end{problem}

\begin{problem}[WebAssign HW 28, \# 5]
Find a power series representation for the function.
\[
f(x)=\frac{1}{(7+x)^2}.
\]
Determine the interval of convergence.
\end{problem}

\begin{problem}[WebAssign HW 28, \# 6]
Find a power series representation for the function.
\[
f(x)=\ln(3-x).
\]
Determine the interval of convergence.
\end{problem}

\begin{problem}[WebAssign HW 28, \# 7]
Evaluate the indefinite integral as a power series.
\[
\int\frac{t}{1-t^{11}}dt.
\]
What is the radius of convergence $R$?
\end{problem}

\begin{problem}[WebAssign HW 28, \# 8]
Use a power series to approximate the definite integral, $I$, to six
decimal places.
\[
\int_0^{0.1}\frac{1}{1+x^6}dx.
\]
\end{problem}

\subsection{Homework 29}
\begin{problem}[WebAssign HW 29, \# 1]
Find the Maclaurin series for $f(x)$ using the definition of a Maclaurin
series. [Assume that $f$ has a power series expansion. Do not show that
$R_n(x)\to 0$.]
\[
f(x)=\sin\left(\frac{\pi x}{3}\right).
\]
Find the associated radius of convergence $R$.
\end{problem}

\begin{problem}[WebAssign HW 29, \# 2]
Find the Maclaurin series for $f(x)$ using the definition of a Maclaurin
series. [Assume that $f$ has a power series expansion. Do not show that
$R_n(x)\to 0$.]
\[
f(x)=e^{-5x}.
\]
Find the associated radius of convergence $R$.
\end{problem}

\begin{problem}[WebAssign HW 29, \# 3]
Find the Taylor series for $f(x)$ centered at the given value of
$a$. [Assume that $f$ has a power series expansion. Do not show that
$R_n(x)\to 0$.]
\[
f(x)=x^4-4x^2+2,\qquad a=2.
\]
Find the associated radius of convergence $R$.
\end{problem}

\begin{problem}[WebAssign HW 29, \# 4]
Find the Taylor series for $f(x)$ centered at the given value of
$a$. [Assume that $f$ has a power series expansion. Do not show that
$R_n(x)\to 0$.]
\[
f(x)=\ln x,\qquad a=6.
\]
Find the associated radius of convergence $R$.
\end{problem}

\begin{problem}[WebAssign HW 29, \# 5]
Find the Taylor series for $f(x)$ centered at the given value of
$a$. [Assume that $f$ has a power series expansion. Do not show that
$R_n(x)\to 0$.]
\[
f(x)=\frac{10}{x},\qquad a=-2.
\]
Find the associated radius of convergence $R$.
\end{problem}

\subsection{Homework 30}
\begin{problem}[WebAssign HW 30, \# 1]
Use the Maclaurin series in the table (it's somewhere in the book, I'll put
a link here \ref{tab:taylor-series}) to obtain the Maclaurin series for the
given function.
\[
f(x)=6e^x+e^{4x}.
\]
\end{problem}

\begin{problem}[WebAssign HW 30, \# 2]
Use the Maclaurin series in the table to obtain the Maclaurin series for
the given function.
\[
f(x)=4x\cos\left(\frac{x^2}{9}\right).
\]
\end{problem}

\begin{problem}[WebAssign HW 30, \# 3]
Use the Maclaurin series in the table to obtain the Maclaurin series for
the given function.
\[
f(x)=9\sin^2 x.
\]
[\emph{Hint}: Use $\sin^2 x=(1-\cos 2x)/2$.]
\end{problem}

\begin{problem}[WebAssign HW 30, \# 4]
Use series to approximate the definite integral $I$ to within the indicated
accuracy.
\[
I=\int_0^{0.5} x^4 e^{-x^2}dx
\]
($|\mathrm{error}|<0.001$).
\end{problem}

\begin{problem}[WebAssign HW 30, \# 5]
Use series to evaluate the limit.
\[
\lim_{x\to 0}\frac{1-\cos 3x}{1+3x-e^{3x}}.
\]
\end{problem}

\begin{problem}[WebAssign HW 30, \# 6]
Use multiplication or division of power series to find the first three
nonzero terms in the Maclaurin series for the function.
\[
y=e^{-x^2}\cos x.
\]
\end{problem}
\newpage
\chapter{Exam 3: Problems}

%%% Local Variables:
%%% mode: latex
%%% TeX-master: "../MA166-Recitation"
%%% End:
