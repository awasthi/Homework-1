\chapter{Homework}
\section{This Week's Summary}
Here's a summary of the material that was (presumably) covered this week.
From Stewart, we have
\subsection{}

\section{WebAssign Homework}
Solutions to selected problems
\subsection{Homework 28}
\begin{problem}[WebAssign HW 28, \# 1]
Find a power series representation for the function.
\[
f(x)=\frac{1}{9+x}.
\]
Determine the interval of convergence.
\end{problem}

\begin{problem}[WebAssign HW 28, \# 2]
Find a power series representation for the function.
\[
f(x)=\frac{8}{9-x}.
\]
Determine the interval of convergence.
\end{problem}

\begin{problem}[WebAssign HW 28, \# 3]
Find a power series representation for the function.
\[
f(x)=\frac{x}{4+x^2}.
\]
Determine the interval of convergence.
\end{problem}

\begin{problem}[WebAssign HW 28, \# 4]
Find a power series representation for the function.
\[
f(x)=\frac{10}{x^2-2x-24}.
\]
Determine the interval of convergence.
\end{problem}

\begin{problem}[WebAssign HW 28, \# 5]
Find a power series representation for the function.
\[
f(x)=\frac{1}{(7+x)^2}.
\]
Determine the interval of convergence.
\end{problem}

\begin{problem}[WebAssign HW 28, \# 6]
Find a power series representation for the function.
\[
f(x)=\ln(3-x).
\]
Determine the interval of convergence.
\end{problem}

\begin{problem}[WebAssign HW 28, \# 7]
Evaluate the indefinite integral as a power series.
\[
\int\frac{t}{1-t^{11}}dt.
\]
What is the radius of convergence $R$?
\end{problem}

\begin{problem}[WebAssign HW 28, \# 8]
Use a power series to approximate the definite integral, $I$, to six
decimal places.
\[
\int_0^{0.1}\frac{1}{1+x^6}dx.
\]
\end{problem}

\subsection{Homework 29}
\begin{problem}[WebAssign HW 29, \# 1]
Find the MacLaurin series for $f(x)$ using the definition of a MacLaurin
series. [Assume that $f$ has a power series expansion. Do not show that
$R_n(x)\to 0$.]
\[
f(x)=\sin\left(\frac{\pi x}{3}\right).
\]
Find the associated radius of convergence $R$.
\end{problem}

\begin{problem}[WebAssign HW 29, \# 2]
Find the MacLaurin series for $f(x)$ using the definition of a MacLaurin
series. [Assume that $f$ has a power series expansion. Do not show that
$R_n(x)\to 0$.]
\[
f(x)=e^{-5x}.
\]
Find the associated radius of convergence $R$.
\end{problem}

\begin{problem}[WebAssign HW 29, \# 3]
Find the Taylor series for $f(x)$ centered at the given value of
$a$. [Assume that $f$ has a power series expansion. Do not show that
$R_n(x)\to 0$.]
\[
f(x)=x^4-4x^2+2,\qquad a=2.
\]
Find the associated radius of convergence $R$.
\end{problem}

\begin{problem}[WebAssign HW 29, \# 4]
Find the Taylor series for $f(x)$ centered at the given value of
$a$. [Assume that $f$ has a power series expansion. Do not show that
$R_n(x)\to 0$.]
\[
f(x)=\ln x,\qquad a=6.
\]
Find the associated radius of convergence $R$.
\end{problem}

\begin{problem}[WebAssign HW 29, \# 5]
Find the Taylor series for $f(x)$ centered at the given value of
$a$. [Assume that $f$ has a power series expansion. Do not show that
$R_n(x)\to 0$.]
\[
f(x)=\frac{10}{x},\qquad a=-2.
\]
Find the associated radius of convergence $R$.
\end{problem}

\subsection{Homework 30}
\begin{problem}[WebAssign HW 30, \# 1]
Use the MacLaurin series in the table (it's somewhere in the book, I'll put
a link here ) to obtain the MacLaurin series for the given function.
\[
f(x)=6e^x+e^{4x}.
\]
\end{problem}

\begin{problem}[WebAssign HW 30, \# 2]
Use the MacLaurin series in the table to obtain the MacLaurin series for
the given function.
\[
f(x)=4x\cos\left(\frac{x^2}{9}\right).
\]
\end{problem}

\begin{problem}[WebAssign HW 30, \# 3]
Use the MacLaurin series in the table to obtain the MacLaurin series for
the given function.
\[
f(x)=9\sin^2 x.
\]
[\emph{Hint}: Use $\sin^2 x=(1-\cos 2x)/2$.]
\end{problem}

\begin{problem}[WebAssign HW 30, \# 4]
Use series to approximate the definite integral $I$ to within the indicated
accuracy.
\[
I=\int_0^{0.5} x^4 e^{-x^2}dx
\]
($|\mathrm{error}|<0.001$).
\end{problem}

\begin{problem}[WebAssign HW 30, \# 5]
Use series to evaluate the limit.
\[
\lim_{x\to 0}\frac{1-\cos 3x}{1+3x-e^{3x}}.
\]
\end{problem}

\begin{problem}[WebAssign HW 30, \# 6]
Use multiplication or division of power series to find the first three
nonzero terms in the MacLaurin series for the function.
\[
y=e^{-x^2}\cos x.
\]
\end{problem}
\newpage
\chapter{Exam III Problems}

%%% Local Variables:
%%% mode: latex
%%% TeX-master: "../MA166-Recitation"
%%% End:
