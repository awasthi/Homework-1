\chapter{Recitation 6 Prep}
Recitation average for Exam 1
\begin{table}[htbp]
\caption{Section averages for Exam 1.}
\label{tab:sec-avg}
\centering
\begin{tabular}{|c|c|}
\hline
section&average\\
\hline
294&71.0\\
151&76.66\\
112&69.82\\
\hline
\end{tabular}.
\end{table}
\section{Homework Solutions}
Here are the homework solutions for this week.
\subsection{Homework 12}
\begin{problem}[WebAssign, HW 12, \# 1]
Evaluate the integral
\[
\int_{3\sqrt{2}}^6\frac{1}{t^3\sqrt{t^2-9}}\;dt.
\]
\end{problem}
\begin{proof}[Solution]
Make the substitution
\begin{equation}
\label{eq:hw-12-1-subst-1}
3\sec\theta=t,
\end{equation}
then $3\sec\theta\tan\theta\;d\theta=dt$ and substituting this and
\eqref{eq:hw-12-1-subst-1} into the integral, making sure to solve for the
appropriate values of $\theta$, i.e., the lower bound is
$\sec^{-1}\left(\sqrt{2}\right)=\pi/4$ and the upper bound is
$\sec^{-1}(2)=\pi/3$
\begin{align*}
\int_{3\sqrt{2}}^6\frac{1}{t^3\sqrt{t^2-9}}\;dt
&=\int_{\pi/4}^{\pi/3}
\frac{3\sec\theta\tan\theta}{3^3\sec^3\theta\sqrt{3^2\sec^2\theta-9}}\;d\theta\\
&=\int_{\pi/4}^{\pi/3}
\frac{\tan\theta}{3^2\sec^3\theta\sqrt{3^2\sec^2\theta-3^2}}\;d\theta\\
&=\int_{\pi/4}^{\pi/3}
\frac{\tan\theta}{3^2\sec^2\theta\sqrt{3^2(\sec^2\theta-1)}}\;d\theta\\
&=\frac{1}{27}\int_{\pi/4}^{\pi/3}
\frac{\tan\theta}{\sec^2\theta\sqrt{\tan^2\theta}}\;d\theta\\
&=\frac{1}{27}\int_{\pi/4}^{\pi/3}
\frac{1}{\sec^2\theta}\;d\theta\\
&=\frac{1}{27}\int_{\pi/4}^{\pi/3}\cos^2\theta\;d\theta\\
&=\frac{1}{27}\int_{\pi/4}^{\pi/3}\frac{1+\cos2\theta}{2}\;d\theta\\
&=\frac{1}{54}\int_{\pi/4}^{\pi/3}1+\cos2\theta\;d\theta\\
&=\frac{1}{54}\left(\left.\theta+\frac{\sin
  2\theta}{2}\right|_{\pi/4}^{\pi/3}\right)\\
&=\boxed{\frac{\pi}{648}+\frac{\sqrt{3}-2}{108}.}\qedhere
\end{align*}
\end{proof}

\begin{problem}[WebAssign, HW 12, \# 2]
Evaluate the integral. (Use $C$ for the constant of integration.)
\[
\int\sqrt{1-25x^2}\;dx.
\]
\end{problem}
\begin{proof}[Solution]
First, make the substitution $u=5x$. Then the integral above turns into
\[
\frac{1}{5}\int\sqrt{1-u^2}\;du.
\]
Now we make the trig substitution $\cos\theta=u$ so
$-\sin\theta\;d\theta=du$ and the integral above turns into
\begin{align*}
\frac{1}{5}\int\sqrt{1-u^2}\;du&=\frac{1}{5}\int\sin\theta(-\sin\theta)\;d\theta\\
&=-\frac{1}{5}\int\sin^2\theta\;d\theta\\
&=-\frac{1}{5}\int\left(\frac{1-\cos2\theta}{2}\right)\;d\theta\\
&=-\frac{1}{10}\int 1-\cos2\theta\;d\theta\\
&=\frac{1}{10}\int\cos2\theta-1\;d\theta\\
&=\frac{1}{10}\left(\frac{1}{2}\sin2\theta-\theta\right)\\
&=\frac{1}{20}\sin 2\theta-\frac{1}{2}\theta+C\\
&=\frac{1}{20}2\sin\theta\cos\theta-\frac{1}{2}\theta+C\\
\intertext{substituting back $u$ then $x$, we have}
&=\frac{1}{10}\sqrt{1-u^2}u-\frac{1}{2}\cos^{-1}(u)+C\\
&=\frac{1}{10}\sqrt{1-25x^2}(5x)-\frac{1}{2}\cos^{-1}(5x)+C\\
&=\boxed{\frac{\sqrt{1-25x^2}x-\cos^{-1}(5x)}{2}+C.}\qedhere
\end{align*}
\end{proof}

\begin{problem}[WebAssign, HW 12, \# 3]
Evaluate the integral. (Use $C$ for the constant of integration.)
\[
\int\sqrt{16+6x-x^2}\;dx.
\]
\end{problem}
\begin{proof}[Solution]
First we need to complete the square
\begin{align*}
(x-3)^2-9&=x^2-6x+9-9\\
         &=x^2-6x.
\end{align*}
Then, the integral above turns into
\[
\int\sqrt{16+x-x^2}\;dx=\int\sqrt{25+(x-3)^2}\;dx
\]
and now we can use the substitution $5u=x-3$ to simplify our integral into
\[
\frac{1}{5}\int\sqrt{25-(5u)^2}\;du=
\frac{1}{5}\int\sqrt{5^2-5^2u^2}\;du=
\int\sqrt{1-u^2}\;du.
\]
Now we can use the substitution $\cos\theta=u$ and
$-\sin\theta\;d\theta=du$ to get
\begin{align*}
\int\sqrt{1-u^2}\;du
&=\int\sqrt{1-\cos^2\theta}(-\sin\theta)\;d\theta\\
&=-\int\sqrt{\sin^2\theta}\sin\theta\;d\theta\\
&=-\int\sin^2\theta\;d\theta\\
\intertext{which, from our previous problem, we know to be}
&=\frac{1}{2}\sin 2\theta-\theta+C.
\end{align*}
Substututing back first $u$ then $x$ we get
\begin{align*}
\frac{1}{2}\sin 2\theta-\theta+C
&=\sin\theta\cos\theta-\theta+C\\
&=\sqrt{1-u^2}u-\cos^{-1}(u)+C\\
&=\sqrt{1-\left(\frac{x-3}{5}\right)^2}\left(\frac{x-3}{5}\right)-\cos^{-1}\left(\frac{x-3}{5}\right)+C\\
&=\boxed{\frac{x-3}{25}\sqrt{16+6x-x^2}-\cos^{-1}\left(\frac{x-3}{5}\right)+C.}\qedhere
\end{align*}
\end{proof}

\begin{problem}[WebAssign, HW 12, \# 4]
Evaluate the integral. (Use $C$ for the constant of integration.)
\[
\int\frac{1}{\sqrt{t^2-12t+40}}\;dt.
\]
\end{problem}
\begin{proof}[Solution]
\end{proof}

\begin{problem}[WebAssign, HW 12, \# 5]
Evaluate the integral. (Use $C$ for the constant of integration.)
\[
\int\sqrt{x^2+6x}\;dx.
\]
\end{problem}
\begin{proof}[Solution]
\end{proof}

\subsection{Homework 13}
\begin{problem}[WebAssign, HW 13, \# 1]
\end{problem}
\begin{proof}[Solution]
\end{proof}

\begin{problem}[WebAssign, HW 13, \# 2]
\end{problem}
\begin{proof}[Solution]
\end{proof}

\begin{problem}[WebAssign, HW 13, \# 3]
\end{problem}
\begin{proof}[Solution]
\end{proof}

\begin{problem}[WebAssign, HW 13, \# 4]
\end{problem}
\begin{proof}[Solution]
\end{proof}

\begin{problem}[WebAssign, HW 13, \# 5]
\end{problem}
\begin{proof}[Solution]
\end{proof}

\begin{problem}[WebAssign, HW 13, \# 6]
\end{problem}
\begin{proof}[Solution]
\end{proof}


\subsection{Homework 14}
\begin{problem}[WebAssign, HW 14, \# 1]
\end{problem}
\begin{proof}[Solution]
\end{proof}

\begin{problem}[WebAssign, HW 14, \# 2]
\end{problem}
\begin{proof}[Solution]
\end{proof}

\begin{problem}[WebAssign, HW 14, \# 3]
\end{problem}
\begin{proof}[Solution]
\end{proof}

\begin{problem}[WebAssign, HW 14, \# 4]
\end{problem}
\begin{proof}[Solution]
\end{proof}

\begin{problem}[WebAssign, HW 14, \# 5]
\end{problem}
\begin{proof}[Solution]
\end{proof}

\section{Exam 2 Problems}

%%% Local Variables:
%%% mode: latex
%%% TeX-master: "../MA166-Recitation"
%%% End:
