\chapter{Recitation 6 Prep}
Recitation average for Exam 1
\begin{table}[htbp]
\caption{Section averages for Exam 1.}
\label{tab:sec-avg}
\centering
\begin{tabular}{|c|c|}
\hline
section&average\\
\hline
294&71.0\\
151&76.66\\
112&69.82\\
\hline
\end{tabular}.
\end{table}
\section{Homework Solutions}
Here are the homework solutions for this week.
\subsection{Homework 12}
\begin{problem}[WebAssign, HW 12, \# 1]
Evaluate the integral
\[
\int_{3\sqrt{2}}^6\frac{1}{t^3\sqrt{t^2-9}}\;dt.
\]
\end{problem}
\begin{proof}[Solution]
Make the substitution
\begin{equation}
\label{eq:hw-12-1-subst-1}
3\sec\theta=t,
\end{equation}
then $3\sec\theta\tan\theta\;d\theta=dt$ and substituting this and
\eqref{eq:hw-12-1-subst-1} into the integral, making sure to solve for the
appropriate values of $\theta$, i.e., the lower bound is
$\sec^{-1}\left(\sqrt{2}\right)=\pi/4$ and the upper bound is
$\sec^{-1}(2)=\pi/3$
\begin{align*}
\int_{3\sqrt{2}}^6\frac{1}{t^3\sqrt{t^2-9}}\;dt
&=\int_{\pi/4}^{\pi/3}
\frac{3\sec\theta\tan\theta}{3^3\sec^3\theta\sqrt{3^2\sec^2\theta-9}}\;d\theta\\
&=\int_{\pi/4}^{\pi/3}
\frac{\tan\theta}{3^2\sec^3\theta\sqrt{3^2\sec^2\theta-3^2}}\;d\theta\\
&=\int_{\pi/4}^{\pi/3}
\frac{\tan\theta}{3^2\sec^2\theta\sqrt{3^2(\sec^2\theta-1)}}\;d\theta\\
&=\frac{1}{27}\int_{\pi/4}^{\pi/3}
\frac{\tan\theta}{\sec^2\theta\sqrt{\tan^2\theta}}\;d\theta\\
&=\frac{1}{27}\int_{\pi/4}^{\pi/3}
\frac{1}{\sec^2\theta}\;d\theta\\
&=\frac{1}{27}\int_{\pi/4}^{\pi/3}\cos^2\theta\;d\theta\\
&=\frac{1}{27}\int_{\pi/4}^{\pi/3}\frac{1+\cos2\theta}{2}\;d\theta\\
&=\frac{1}{54}\int_{\pi/4}^{\pi/3}1+\cos2\theta\;d\theta\\
&=\frac{1}{54}\left(\left.\theta+\frac{\sin
  2\theta}{2}\right|_{\pi/4}^{\pi/3}\right)\\
&=\boxed{\frac{\pi}{648}+\frac{\sqrt{3}-2}{108}.}\qedhere
\end{align*}
\end{proof}

\begin{problem}[WebAssign, HW 12, \# 2]
Evaluate the integral. (Use $C$ for the constant of integration.)
\[
\int\sqrt{1-25x^2}\;dx.
\]
\end{problem}
\begin{proof}[Solution]
First, make the substitution $u=5x$. Then the integral above turns into
\[
\frac{1}{5}\int\sqrt{1-u^2}\;du.
\]
Now we make the trig substitution $\cos\theta=u$ so
$-\sin\theta\;d\theta=du$ and the integral above turns into
\begin{align*}
\frac{1}{5}\int\sqrt{1-u^2}\;du&=\frac{1}{5}\int\sin\theta(-\sin\theta)\;d\theta\\
&=-\frac{1}{5}\int\sin^2\theta\;d\theta\\
&=-\frac{1}{5}\int\left(\frac{1-\cos2\theta}{2}\right)\;d\theta\\
&=-\frac{1}{10}\int 1-\cos2\theta\;d\theta\\
&=\frac{1}{10}\int\cos2\theta-1\;d\theta\\
&=\frac{1}{10}\left(\frac{1}{2}\sin2\theta-\theta\right)\\
&=\frac{1}{20}\sin 2\theta-\frac{1}{2}\theta+C\\
&=\frac{1}{20}2\sin\theta\cos\theta-\frac{1}{2}\theta+C\\
\intertext{substituting back $u$ then $x$, we have}
&=\frac{1}{10}\sqrt{1-u^2}u-\frac{1}{2}\cos^{-1}(u)+C\\
&=\frac{1}{10}\sqrt{1-25x^2}(5x)-\frac{1}{2}\cos^{-1}(5x)+C\\
&=\boxed{\frac{\sqrt{1-25x^2}x-\cos^{-1}(5x)}{2}+C.}\qedhere
\end{align*}
\end{proof}

\begin{problem}[WebAssign, HW 12, \# 3]
Evaluate the integral. (Use $C$ for the constant of integration.)
\[
\int\sqrt{16+6x-x^2}\;dx.
\]
\end{problem}
\begin{proof}[Solution]
First we need to complete the square
\begin{align*}
(x-3)^2-9&=x^2-6x+9-9\\
         &=x^2-6x.
\end{align*}
Then, the integral above turns into
\[
\int\sqrt{16+x-x^2}\;dx=\int\sqrt{25+(x-3)^2}\;dx
\]
and now we can use the substitution $5u=x-3$ to simplify our integral into
\[
\frac{1}{5}\int\sqrt{25-(5u)^2}\;du=
\frac{1}{5}\int\sqrt{5^2-5^2u^2}\;du=
\int\sqrt{1-u^2}\;du.
\]
Now we can use the substitution $\cos\theta=u$ and
$-\sin\theta\;d\theta=du$ to get
\begin{align*}
\int\sqrt{1-u^2}\;du
&=\int\sqrt{1-\cos^2\theta}(-\sin\theta)\;d\theta\\
&=-\int\sqrt{\sin^2\theta}\sin\theta\;d\theta\\
&=-\int\sin^2\theta\;d\theta\\
\intertext{which, from our previous problem, we know to be}
&=\frac{1}{2}\sin 2\theta-\theta+C.
\end{align*}
Substututing back first $u$ then $x$ we get
\begin{align*}
\frac{1}{2}\sin 2\theta-\theta+C
&=\sin\theta\cos\theta-\theta+C\\
&=\sqrt{1-u^2}u-\cos^{-1}(u)+C\\
&=\sqrt{1-\left(\frac{x-3}{5}\right)^2}\left(\frac{x-3}{5}\right)-\cos^{-1}\left(\frac{x-3}{5}\right)+C\\
&=\boxed{\frac{x-3}{25}\sqrt{16+6x-x^2}-\cos^{-1}\left(\frac{x-3}{5}\right)+C.}\qedhere
\end{align*}
\end{proof}

\begin{problem}[WebAssign, HW 12, \# 4]
Evaluate the integral. (Use $C$ for the constant of integration.)
\[
\int\frac{1}{\sqrt{t^2-12t+40}}\;dt.
\]
\end{problem}
\begin{proof}[Solution]
This problem is similar to the last one. The first thing we need to do is
to complete the square
\begin{align*}
(t-6)^2-36&=t^2-12t+36-36\\
          &=t^2-12t.
\end{align*}
This turns our integral into
$$
\int\frac{1}{\sqrt{t^2-12t+40}}\;dt=
\int\frac{1}{\sqrt{(t-6)^2-36+40}}\;dt=
\int\frac{1}{\sqrt{(t-6)^2+4}}\;dt.
$$
Then, making the substitution $u=(t-6)/2$ we have $2\;du=dt$ and our
integral turns into
\begin{align*}
\int\frac{1}{\sqrt{(t-6)^2+4}}\;dt
&=2\int\frac{1}{\sqrt{2^2u^2+2^2}}\;du\\
&=\int\frac{1}{\sqrt{u^2+1}}\;du.
\end{align*}
From here we can make the substitution $\tan\theta=u$ so
$\sec\theta\tan\theta\;d\theta=du$ and our integral turns into
\begin{align*}
\int\frac{1}{\sqrt{u^2+1}}\;du
&=\int\frac{\sec\theta\tan\theta}{\sec\theta}\;d\theta\\
&=\int\tan\theta\;d\theta\\
&=\ln\left|\cos\theta\right|+C.
\end{align*}
Now we substitute $u$ back into the equation by $\cos\theta=1/\sqrt{1+u^2}$
and $u=(t-6)/2$ giving us
\begin{align*}
\ln\left|\cos\theta\right|+C
&=\ln\left|\frac{1}{\sqrt{1+u^2}}\right|+C\\
&=\ln\left|\frac{1}{\sqrt{1+((t-6)/2)^2}}\right|+C\\
&=\boxed{\ln\left|\frac{2}{\sqrt{t^2-12t+40}}\right|+C.}
\end{align*}
\end{proof}

\begin{problem}[WebAssign, HW 12, \# 5]
Evaluate the integral. (Use $C$ for the constant of integration.)
\[
\int\sqrt{x^2+6x}\;dx.
\]
\end{problem}
\begin{proof}[Solution]
Again, nothing special is going on here, all we need to do is first
complete the square and then proceed to make a $u$-substitution or
trigonometric substitution where appropriate
\begin{align*}
(x+3)^2-9&=x^2+6x+9-9\\
         &=x^2+6x.
\end{align*}
Then our integral turns into
$$
\int\sqrt{(x+3)^2-9}\;dx.
$$
Now, making the $u=(x+3)/3$ we have $3\;du=dx$ and
\begin{align*}
\int\sqrt{(x+3)^2-9}
&=\int\sqrt{(x+3)^2-3^2}\;dx\\
&=3\int\sqrt{3^2u^2-3^2}\;du\\
&=3\int\sqrt{3^2(u^2-1)}\;du\\
&=9\int\sqrt{u^2-1}\;du.
\end{align*}
To continue we need to make a trigonometric substitution. The following is
the easiest substitution to make $\sec\theta=u$ then
$\sec\theta\tan\theta\;d\theta=du$ and we have
\begin{align*}
9\int\sqrt{u^2-1}\;du
&=9\int\tan\theta\sec\theta\tan\theta\;d\theta\\
&=9\int\tan^2\theta\sec\theta\;d\theta\\
&=9\int\left(\sec^2\theta-1\right)\sec\theta\;d\theta\\
&=9\int\sec^3\theta-\sec\theta\;d\theta\\
&=9\left(\vphantom{\int\sec^3\;d\theta}\right.
\underbrace{\int\sec^3\;d\theta}_{I_1}
-\underbrace{\int\sec\theta\;d\theta}_{I_2}
\left.\vphantom{\int\sec^3\;d\theta}\right).
\end{align*}
Let's compute $I_1$ and $I_2$ separately. For $I_1$ we use integration by
parts to get to
\begin{align*}
I_1&=\int\sec^3\theta\;d\theta\\
&=\int\sec^2\theta\sec\theta\;d\theta
\intertext{let $u=\sec\theta$, $dv=\sec^2\theta\;d\theta$ so
  $du=\sec\theta\tan\theta\;d\theta$ and $v=\tan\theta$ giving us}
&=\sec\theta\tan\theta-\int\sec\theta\tan^2\theta\;d\theta\\
&=\sec\theta\tan\theta-\int\sec\theta(\sec^2\theta-1)\;d\theta\\
&=\sec\theta\tan\theta-\int\sec^3\theta+\int\sec\theta\;d\theta\\
&=\sec\theta\tan\theta-I_1+I_2\\
I_1+I_1&=\sec\theta\tan\theta+I_2\\
I_1&=\frac{1}{2}\left(\sec\theta\tan\theta+I_2\right)+C_1.
\end{align*}
This depends on $I_2$ so let's compute that
\begin{align*}
I_2&=\int\sec\theta\;d\theta\\
\intertext{rewrite the integral as}
&=\int\sec\theta\frac{\sec\theta+\tan\theta}{\sec\theta+\tan\theta}\;d\theta\\
&=\int\frac{\sec^2\theta+\sec\theta\tan\theta}{\sec\theta+\tan\theta}\\
\intertext{and use the substitution $u=\sec\theta+\tan\theta$ since
  $du=(\sec\theta\tan\theta+\sec^2\theta)\;d\theta$}
&=\int\frac{\sec^2\theta+\sec\theta\tan\theta}{u}\frac{1}{\sec^2\theta+\sec\theta\tan\theta}\;du\\
&=\int\frac{1}{u}\;du\\
&=\ln|u|+C_2\\
&=\ln\left|\sec\theta+\tan\theta\right|+C_2
\end{align*}
Then
$$
I_1=\frac{1}{2}(\sec\theta\tan\theta+\ln|\sec\theta+\tan\theta|+C_2)+C_1
$$
and we have
\begin{align*}
9(I_1-I_2)
&=9\left(
\frac{1}{2}(\sec\theta\tan\theta+\ln|\sec\theta+\tan\theta|+C_2)+C_1
-\ln|\sec\theta+\tan\theta|
\right)\\
&=\frac{9}{2}\sec\theta\tan\theta-\frac{9}{2}\ln|\sec\theta+\tan\theta|+
\underbrace{9C_1-\frac{9}{2}C_2}_{\text{call this $C$}}\\
&=\frac{9}{2}\sec\theta\tan\theta-\frac{9}{2}\ln|\sec\theta+\tan\theta|+C.
\end{align*}
Now we substitute back our value of $u$ then $x$ as follows
\begin{align*}
\frac{9}{2}\sec\theta\tan\theta-\frac{9}{2}\ln|\sec\theta+\tan\theta|+C
&=\frac{9}{2}u\sqrt{u^2-1}-\frac{9}{2}\ln\left|u+\sqrt{u^2-1}\right|+C\\
&=\frac{9}{2}\frac{x+3}{3}\sqrt{\left(\frac{x+3}{3}\right)^2-1}
-\frac{9}{2}\ln\left|\frac{x+3}{3}+\sqrt{\left(\frac{x+3}{3}\right)^2-1}\right|+C\\
&=\boxed{\frac{1}{2}(x+3)\sqrt{x^2+6x}
+\frac{9}{2}\ln\left|\frac{1}{3}\left(x+3+\sqrt{x^2+6x}\right)\right|
+C.}\qedhere
\end{align*}
\end{proof}

\subsection{Homework 13}
\begin{problem}[WebAssign, HW 13, \# 1]
Write out the form of the partial fraction decomposition of the
function. Do not determine the numerical values of the coefficients.
\begin{enumerate}[label=(\alph*)]
\item $\displaystyle\frac{x}{x^2+x-2}$
\item $\displaystyle\frac{x^2}{x^2+x+3}$
\end{enumerate}
\end{problem}
\begin{proof}[Solution]
\textbf{(a)} Let's start by factoring the denominator
\[
x^2+x-2=(x+2)(x-1).
\]
This factorization is very easy to see. If you don't believe me, use the
quadratic equation and find the roots and you'll see that I'm right. Now,
we know that the partial fraction decomposition will look like
\[
\boxed{\frac{A}{x+2}+\frac{B}{x-1}.}
\]
\\\\
\textbf{(b)} The polynomial $x^2+x+3$ has no real roots. Let's confirm
this. If we put the coefficients of $x^2+x+3$ into the quadratic equation
we have
\begin{align*}
\frac{-1\pm\sqrt{1^2-4\cdot 3}}{2}
&=\frac{-1\pm\sqrt{1-12}}{2}\\
&=\frac{-1\pm\sqrt{-11}}{2}.
\end{align*}
$\sqrt{-11}$ is not a real number so we cannot proceed. In this case, the
best we cannot get rid of the $x^2+x+3$ at the bottom. Now, since the
degree of the top $x^2$ is not less that the degree of $x^2+x+3$ we need to
remove enough terms to get something of lower degree on top
\begin{align*}
\frac{x^2}{x^2+x+3}
&=\frac{(x^2+x+3)-x-3}{x^2+x+3}\\
&=\frac{x^2+x+3}{x^2+x+3}-\frac{x-3}{x^2+x+3}\\
&=\boxed{1+\frac{Ax+B}{x^2+x+3}.}\qedhere
\end{align*}
\end{proof}

\begin{problem}[WebAssign, HW 13, \# 2]
Write out the form of the partial fraction decomposition of the
function. Do not determine the numerical values of the coefficients.
\begin{enumerate}[label=(\alph*)]
\item $\displaystyle\frac{x^4+1}{x^5+7x^3}$
\item $\displaystyle\frac{3}{(x^2-25)^2}$
\end{enumerate}
\end{problem}
\begin{proof}[Solution]
\textbf{(a)} Let's start by factoring the denominator into something
simpler
\[
x^5+7x^3=x^3(x^2+7).
\]
Then, since the top has degree less than the bottom and we cannot factor
$x^2+7$, we have the partial fraction decomposition
\[
\frac{3}{x^5+7x^3}=
\boxed{\frac{A}{x}+\frac{B}{x^2}+\frac{C}{x^3}+\frac{Dx+E}{x^2+7}.}
\]
\\\\
\textbf{(c)} Let's factor the denominator
\[
\left(x^2-25\right)^2=(x-5)^2(x+5)^2.
\]
Then we have the partial fraction decomposition
\[
\frac{3}{\left(x^2-25\right)^2}=
\boxed{\frac{A}{x+5}+\frac{B}{(x+5)^2}+\frac{C}{x-5}+\frac{D}{(x-5)^2}.}
\]
\end{proof}
\begin{problem}[WebAssign, HW 13, \# 3]
Evaluate the integral
\[
\int_0^1\frac{x-8}{x^2-7x+10}\;dx.
\]
\end{problem}
\begin{proof}[Solution]
First let's find the partial fraction decomposition of
\[
\frac{x-8}{x^2-7x+10}.
\]
To that end, we need to factor the denominator
\[
x^2-7x+10=(x-5)(x-2).
\]
Then
\[
\frac{x-8}{(x-5)(x-2)}=\frac{A}{x-5}+\frac{B}{x-2}.
\]
Solving for $A$ we have
\[
A=\frac{5-8}{5-2}+\frac{B}{5-2}(5-5)=\frac{-3}{3}=-1
\]
and
\[
B=\frac{2-8}{2-5}+\frac{A}{2-5}(2-2)=-2=2.
\]
Hence, the integral turns into
\begin{align*}
\int_0^1\frac{x-8}{x^2-7x+10}\;dx
&=\int_0^1-\frac{1}{x-5}+\frac{2}{x-2}\;dx\\
&=\left.-\ln|x-5|+2\ln|x-2|\right|_0^1\\
&=-\ln|4|+2\ln|1|+\ln|5|-\ln|2|\\
&=-2\ln|2|-\ln|2|+\ln|5|\\
&=\boxed{\ln|5|-3\ln|2|.}
\end{align*}
\end{proof}

\begin{problem}[WebAssign, HW 13, \# 4]
Evaluate the integral. (Remember to use $\ln|u|$ where appropriate. Use $C$
for the constant of integration.)
\[
\int\frac{ax}{x^2-bx}\;dx.
\]
\end{problem}
\begin{proof}[Solution]
First, let's write down the partial fraction decomposition for
\[
\frac{ax}{x^2-bx}.
\]
The denominator factors as $x^2-bx=x(x-b)$ so
\[
\frac{ax}{x(x-b)}=\frac{A}{x}+\frac{B}{x-b}.
\]
Then we have
\[
ax=A(x-b)+Bx=(A+B)x-Ab.
\]
This tells us that $A=0$ and $B=a$. Well, actually, we didn't have to do
partial fractions since $x^2-bx$ factors neatly. Let's proceed
\begin{align*}
\int\frac{ax}{x^2-bx}\;dx
&=\int\frac{a}{x-b}\;dx\\
\intertext{use the substitution $u=x-5$ then $du=dx$ and we have}
&=a\int\frac{1}{u}\;du\\
&=a\ln|u|+C\\
&=\boxed{a\ln|x-5|+C.}
\end{align*}
\end{proof}

\begin{problem}[WebAssign, HW 13, \# 5]
Evaluate the integral. (Remember to use $\ln|u|$ where appropriate. Use $C$
for the constant of integration.)
\[
\int\frac{7x^2+2x-7}{x^3-x}\;dx
\]
\end{problem}
\begin{proof}[Solution]
First note that the denominator factors nicely as
\[
x^3-x=x(x^2-1)=x(x+1)(x-1).
\]
Then we can find the partial fraction decomposition of the quotient by
\begin{align*}
\frac{7x^2+2x-7}{x(x+1)(x-1)}
&=\frac{A}{x}+\frac{B}{x+1}+\frac{C}{x-1}\\
7x^2+2x-7&=A(x+1)(x-1)+Bx(x-1)+Cx(x+1)\\
&=Ax^2-A+Bx^2-Bx+Cx^2+Cx\\
&=(A+B+C)x^2+Cx-A.
\end{align*}
Solving for $A$, $B$ and $C$ we have $A=7$, $C=2$ and $B=-C-A=-2-7=-10$ so
\[
\frac{7x^2+2x-7}{x(x+1)(x-1)}
=\frac{7}{x}-\frac{10}{x+1}+\frac{2}{x-1}.
\]
Now, integrating this we have
\begin{align*}
\int\frac{7x^2+2x-7}{x(x+1)(x-1)}\;dx
&=\int\frac{7}{x}-\frac{10}{x+1}+\frac{2}{x-1}\\
&=\boxed{7\ln|x|-\ln|x+1|+2\ln|x-1|+C.}\qedhere
\end{align*}
\end{proof}

\begin{problem}[WebAssign, HW 13, \# 6]
Evaluate the integral. (Remember to use $\ln|u|$ where appropriate. Use $C$
for the constant of integration.)
\[
\int\frac{3x^2-20x+33}{(2x+1)(x-2)^2}\;dx.
\]
\end{problem}
\begin{proof}[Solution]
First, let's find the partial fraction decomposition
\[
\frac{3x^2-20x+33}{(2x+1)(x-2)^2}
=\frac{A}{2x+1}+\frac{B}{x-2}+\frac{C}{(x-2)^2}.
\]
Then, we solve for $A$, $B$, $C$ and $D$ by
\begin{align*}
3x^2-20x+33&=A(x-2)^2+B(2x+1)(x-2)+C(2x+1)\\
&=Ax^2-4Ax+4A+2Bx^2-3Bx-2B+2Cx+C\\
&=(A+2B)x^2+(-4A-3B+2C)x+(4A-2B+C)
\end{align*}
Then we have
\begin{align*}
A+2B&=3\\
-4A-3B+2C&=-20\\
4A-2B+C&=33\\
\end{align*}
Now to solve this linear system of equations we use Gaussian elimination on
the augmented matrix
$$
\begin{abmatrix}{3}
1&2&0&3\\
-4&-3&2&-20\\
4&-2&1&33
\end{abmatrix}
\longrightarrow
\begin{abmatrix}{3}
1&0&0&7\\
0&1&0&-2\\
0&0&1&1
\end{abmatrix}
$$
This tells us that $A=7$, $B=-2$ and $C=1$ so our partial fraction
decomposition is
\[
\frac{3x^2-20x+33}{(2x+1)(x-2)^2}
=\frac{7}{2x+1}-\frac{2}{x-2}+\frac{1}{(x-2)^2}
\]
and our integral becomes
\begin{align*}
\int\frac{3x^2-20x+33}{(2x+1)(x-2)^2}\;dx
&=
\int\frac{7}{2x+1}-\frac{2}{x-2}+\frac{1}{(x-2)^2}\\
&=\boxed{\frac{7}{2}\ln|2x+1|+2\ln|x-2|-\frac{1}{x-2}+C.}
\end{align*}
\end{proof}

\subsection{Homework 14}
\begin{problem}[WebAssign, HW 14, \# 1]
Evaluate the integral. (Remember to use $\ln|u|$ where appropriate. Use $C$
for the constant of integration.)
$$
\int\frac{5}{(x-1)(x^2+4)}\;dx
$$
\end{problem}
\begin{proof}[Solution]
Write
\[
\frac{5}{(x-1)(x^2+4)}=\frac{A}{x-1}+\frac{Bx+C}{x^2+4}.
\]
Then
\begin{align*}
5&=A(x^2+4)+(Bx+C)(x-1)\\
 &=(A+B)x^2+(C-B)x+(4A-C)
\end{align*}
and we have
\begin{align*}
A+B&=0&C-B&=0&4A-C&=5.
\end{align*}
This tells us that $A=-B=-C=5-4A$ so $A=1$, $B=-1$ and $C=-1$. Hence, our
integral becomes
\begin{align*}
\int\frac{5}{(x-1)(x^2-4)}\;dx
&=\int\frac{1}{x-1}-\frac{x+1}{x^2+4}\;dx\\
&=\int\frac{1}{x-1}-\frac{x}{x^2+4}-\frac{1}{x^2+4}\;dx\\
&=\underbrace{\int\frac{1}{x-1}\;dx}_{I_1}
-\underbrace{\int\frac{x}{x^2+4}\;dx}_{I_2}
-\underbrace{\int\frac{1}{x^2+4}\;dx}_{I_3}.
\end{align*}
Let's compute these separately. We know what $I_1$ and $I_3$ are, they are
\begin{align*}
I_1&=\ln|x-1|+C_1&I_3&=\frac{1}{2}\tan^{-1}\left(\frac{1}{2}\right)
\end{align*}
from our handy integral table. That leaves us only $I_2$ to figure out. We
do this by substitution letting $u=x^2$ so $du=2x\;dx$ giving us
\begin{align*}
I_2&=\int\frac{x}{x^2+4}\;dx\\
&=\int\frac{x}{u+4}\frac{dx}{2x}\\
&=\frac{1}{2}\int\frac{1}{u+4}\\
&=\frac{1}{2}\ln\left|x^2+4\right|+C_3.
\end{align*}
Hence, setting $C\coloneqq C_1-C_2-C_3$, our integral is
\[
I_1-I_2-I_3=\boxed{\ln|x-1|+\frac{1}{2}\ln\left|x^2+4\right|-\frac{1}{2}\tan^{-1}\left(\frac{x}{2}\right)+C.}\qedhere
\]
\end{proof}

\begin{problem}[WebAssign, HW 14, \# 2]
Evaluate the integral. (Use $C$ for the constant of integration.)
\[
\int\frac{4x^2+3x+4}{(x^2+1)^2}\;dx.
\]
\end{problem}
\begin{proof}[Solution]
Instead of finding the partial fraction decomposition split the numerator
into
\[
4(x^2+1)+3x.
\]
Then we can split the integral into
\begin{align*}
\int\frac{4x^2+3x+4}{(x^2+1)^2}\;dx
&=\int\frac{4(x^2+1)+3x}{(x^2+1)^2}\;dx\\
&=\int\frac{4(x^2+1)+3x}{(x^2+1)^2}\;dx\\
&=4\int\frac{x^2+1}{(x^2+1)^2}\;dx
+3\int\frac{x}{(x^2+1)^2}\;dx\\
&=4\underbrace{\int\frac{1}{x^2+1}\;dx}_{I_1}
+3\underbrace{\int\frac{x}{(x^2+1)^2}\;dx}_{I_2}.
\end{align*}
Calculating $I_1$ is easy from the table of integrals
$I_1=\tan(x)+C_1$. $I_2$ takes a little more work and requires the
substitution $u=x^2$ with $du=2x\;dx$
\begin{align*}
I_2
&=\int\frac{x}{(x^2+1)^2}\;dx\\
&=\int\frac{x}{(u+1)^2}\frac{du}{2x}\\
&=\frac{1}{2}\int\frac{1}{(u+1)^2}\;du\\
&=-\frac{1}{2(u+1)}+C_2\\
&=-\frac{1}{2(x^2+1)}+C_2.
\end{align*}
Then, letting $C\coloneqq 4C_1-3C_2$, the integral is
\[
4I_1+3I_2=\boxed{4\tan^{-1} x-\frac{3}{2(x^2+1)}.}\qedhere
\]
\end{proof}

\begin{problem}[WebAssign, HW 14, \# 3]
Evaluate the integral. (Remember to use $\ln|u|$ where appropriate. Use $C$
for the constant of integration.)
\[
\int\frac{3}{x(x^2+4)^2}\;dx.
\]
\end{problem}
\begin{proof}[Solution]
For this problem  we need to express the quotient in its partial fraction
decomposition. First let's rewrite the integral, putting the coefficient $3$
aside
\[
\int\frac{3}{x(x^2+4)^2}\;dx=
3\int\frac{1}{x(x^2+4)^2}\;dx.
\]
Now the denominator factors as $x$ and $(x^2+4)^2$. But $x^2+4$ cannot be
factored into a polynomial of lower degree so our decomposition will look
like
\[
\frac{1}{x(x^2+4)^2}=
\frac{A}{x}+\frac{Bx+C}{x^2+4}+\frac{Dx+E}{(x^2+4)^2}.
\]
Let's solve for $A$ first. Multiply across by $x$ on the left and right and
plug in $0$ gives us
\[
A=\frac{1}{(0+4)^2}=\frac{1}{16}.
\]
Now, multiply by $x(x^2+4)^2$ across and we have
\begin{align*}
1&=\frac{1}{16}(x^2+4)^2+(Bx^2+Cx)(x^2+4)+Dx+E\\
&=\frac{1}{16}x^4+\frac{1}{2}x^2+1+\left(Bx^4+Cx^3+4Bx^2+4Cx\right)+(Dx+E)x\\
&=\left(\frac{1}{16}+B\right)x^4+Cx^3
+\left(\frac{1}{2}+4B+D\right)x^2+\left(4C+E\right)x+1.
\end{align*}
This tells us that $C=0$ so $E=0$, $B=-1/16$ so $D=-1/2+1/4=-1/4$.

Now we can at last compute the integral
\begin{align*}
3\int\frac{1}{x(x^2+4)^2}\;dx
&=3\int\left(\frac{1}{16x}-\frac{1}{16(x^2+4)}
-\frac{x}{4(x^2+4)^2}\right)\;dx\\
&=3\underbrace{\int\frac{1}{16x}\;dx}_{I_1}
-3\underbrace{\int\frac{1}{16(x^2+4)}\;dx}_{I_2}
-3\underbrace{\int\frac{x}{4(x^2+4)^2}\;dx}_{I_3}\\
&=3(I_1-I_2-I_3).
\end{align*}
Let's compute these separately. It's easy to see that
\[
I_1=\frac{1}{16}\ln|x|+C_1.
\]
The integral $I_2$ is also easy by $u$-substitution using $u=x^2+4$
\begin{align*}
I_2&=\frac{1}{16}\int\frac{x}{x^2+4}\;dx\\
   &=\frac{1}{32}\int\frac{1}{u}\;du\\
   &=\frac{1}{32}\ln|u|+C_2\\
   &=\frac{1}{32}\ln\left|x^2+4\right|+C_2.
\end{align*}
The integral $I_3$ can also be computed using the substitution $u=x^2+4$
and we have
\begin{align*}
I_3&=\int\frac{x}{4(x^2+4)^2}\;dx\\
&=\frac{1}{4}\int\frac{1}{u^2}\;du\\
&=-\frac{1}{4u}+C_3\\
&=-\frac{1}{4}\frac{1}{x^2+4}+C_3.
\end{align*}
Then, putting $C=3(C_1-C_2-C_3)$ , the integral is
\[
3(I_1-I_2-I_3)=\boxed{\frac{3}{16}\ln|x|-\frac{3}{32}\ln\left|x^2+4\right|
+\frac{3}{8}\frac{1}{x^2+4}+C.}\qedhere
\]
\end{proof}

\begin{problem}[WebAssign, HW 14, \# 4]
Make a substitution to express the integrand as a rational function and
then evaluate the integral. (Remember to use $\ln|u|$ where
appropriate. Use $C$ for the constant of integration.)
\[
\int\frac{\sqrt{x+49}}{x}\;dx.
\]
\end{problem}
\begin{proof}[Solution]
Let's make the substitution $u=\sqrt{x+49}$ so we have $u^2=x+49$ and
$2u\;du=dx$. Then our integral turns into
\begin{align*}
\int\frac{\sqrt{x+49}}{x}\;dx
&=\int\frac{u}{u^2-49}2u\;du\\
&=\int\frac{2u^2}{u^2-49}\;du\\
&=\int\frac{2u^2-98+98}{u^2-49}\;du\\
&=\int\left(2+\frac{98}{u^2-49}\right)\;du\\
&=\int2\;du+\int\frac{98}{u^2-49}\;du\\
&=2u+C_1+98\underbrace{\int\frac{1}{u^2-49}\;du}_{I_2}.
\end{align*}
Here's a good time to do some partial fraction decomposition. We can break
up $u^2-49=(u-7)(u+7)$ and we have
\[
\frac{1}{(u-7)(u+7)}=\frac{A}{u-7}+\frac{B}{u+7}.
\]
Then
\[
1=A(u+7)+B(u-7)=(A+B)u+7A-7B.
\]
This tells us that $A-B=1/7$ and $A+B=0$ so $A=-B$ and $A-(-A)=1/7$ implies
$A=1/14$ and $B=-1/14$. Thus, we have
\begin{align*}
I_2
&=\int\frac{1}{u^2-49}\;du\\
&=\frac{1}{14}\int\frac{1}{u-7}\;du
-\frac{1}{14}\int\frac{1}{u+7}\;du\\
&=\frac{1}{14}\ln|u-7|-\frac{1}{14}\ln|u+7|+C_2.
\end{align*}
So, letting $C=C_1+98C_2$ and substituting back $u=\sqrt{x+49}$ our
integral is
\[
2\sqrt{x+49}+C_1+98I_2=
\boxed{
2\sqrt{x+49}
+7\ln\left|\sqrt{x+49}-7\right|
-7\ln\left|\sqrt{x+49}+7\right|+C
.
}\qedhere
\]
\end{proof}

\begin{problem}[WebAssign, HW 14, \# 5]
Find the area of the region under the given curve from $1$ to $2$
\[
y=\frac{13}{x^3+3x}.
\]
\end{problem}
\begin{proof}[Solution]
Using the partial fraction decomposition write $x^3+3x=x(x^2+3)$ and we
have
\[
\frac{1}{x(x^2+3)}=\frac{A}{x}+\frac{Bx+C}{x^2+3}.
\]
Multiplying on both sides by $x(x^2+3)$ we get
\[
1=A(x^2+3)+(Bx+C)x=(A+B)x^2+Cx+3A
\]
giving us that $A=1/3$, $C=0$ and $B=-1/3$. Hence, we can write the
integral as
\begin{align*}
\int_1^2\frac{13}{x^3+3x}\;dx
&=13\int_1^2\frac{1}{3x}-13\frac{x}{3(x^2+3)}\;dx\\
&=\left.\frac{13}{3}\ln|x|-\frac{13}{6}\ln\left|x^2+3\right|\right|_1^2\\
&=\frac{13}{3}\ln 2-\frac{13}{6}\ln|7|
-\left(\frac{13}{3}\ln 1-\frac{13}{6}\ln|4|\right)\\
&=\frac{13}{3}\ln 2-\frac{13}{6}\ln|7|+\frac{13}{6}\ln|4|\\
&=\frac{13}{6}\ln|4|-\frac{13}{6}\ln|7|+\frac{13}{6}\ln|4|\\
&=\frac{13}{6}\ln|16|-\frac{13}{6}\ln|7|\\
&=\boxed{\frac{13}{6}\ln\left|\frac{16}{17}\right|.}\qedhere
\end{align*}
\end{proof}

\section{Exam 2 Problems}

%%% Local Variables:
%%% mode: latex
%%% TeX-master: "../MA166-Recitation"
%%% End:
