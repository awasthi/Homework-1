\chapter{Homework}
\section{This Week's Summary}
\subsection{Homework Problems}
Solutions to selected problems:
\subsubsection{Homework 35}
\begin{problem}[WebAssign HW 35, \# 1]
Sketch the curve with the given polar equation by first sketching the graph
of $r$ as a function of $\theta$ in Cartesian coordinates
\[
\begin{aligned}
r&=\theta,&\theta &\geq 0
\end{aligned}.
\]
\end{problem}
\begin{problem}[WebAssign HW 35, \# 2]
Sketch the curve with the given polar equation by first sketching the graph
of $r$ as a function of $\theta$ in Cartesian coordinates.
\[
  \begin{aligned}
    r&=\ln\theta,&\theta\geq 1.
  \end{aligned}
\]
\end{problem}
\begin{problem}[WebAssign HW 35, \# 3]
Sketch the curve with the given polar equation by first sketching the graph
of $r$ as a function of $\theta$ in Cartesian coordinates.
\[
\begin{aligned}
r=6\sin 4\theta.
\end{aligned}
\]
\end{problem}
\begin{problem}[WebAssign HW 35, \# 4]
Sketch the curve with the given polar equation by first sketching the graph
of $r$ as a function of $\theta$ in Cartesian coordinates.
\[
r=\cos 4\theta.
\]
\end{problem}
\begin{problem}[WebAssign HW 35, \# 5]
Sketch the curve with the given polar equation by first sketching the graph
of $r$ as a function of $\theta$ in Cartesian coordinates.
\[
r=6\cos 4\theta.
\]
\end{problem}
\begin{problem}[WebAssign HW 35, \# 6]
Sketch the curve with the given polar equation by first sketching the graph
of $r$ as a function of $\theta$ in Cartesian coordinates.
\[
r=1-2\sin\theta.
\]
\end{problem}
\begin{problem}[WebAssign HW 35, \# 7]
Sketch the curve with the given polar equation by first sketching the graph
of $r$ as a function of $\theta$ in Cartesian coordinates.
\[
r=5+3\sin\theta.
\]
\end{problem}
\begin{problem}[WebAssign HW 35, \# 8]
Evaluate the expression and write your answer in the form $a+bi$.
\[
\left(3+\tfrac{5}{2}i\right)-\left(8+\tfrac{9}{2}i\right)
\]
\end{problem}
\begin{problem}[WebAssign HW 35, \# 9]
Evaluate the expression and write your answer in the form $a+bi$.
\[
(6+7i)(9-4i).
\]
\end{problem}
\begin{problem}[WebAssign HW 35, \# 10]
Evaluate the expression and write your answer in the form $a+bi$.
\[
\overline{3+4i}.
\]
\end{problem}
\begin{problem}[WebAssign HW 35, \# 11]
Evaluate the expression and write your answer in the form $a+bi$.
\[
\frac{6+5i}{4-7i}.
\]
\end{problem}
\begin{problem}[WebAssign HW 35, \# 12]
Evaluate the expression and write your answer in the form $a+bi$.
\[
5i^3.
\]
\end{problem}
\begin{problem}[WebAssign HW 35, \# 13]
Evaluate the expression and write your answer in the form $a+bi$.
\[
8i^{100}.
\]
\end{problem}
\begin{problem}[WebAssign HW 35, \# 14]
Evaluate the expression and write your answer in the form $a+bi$.
\[
\sqrt{-81}.
\]
\end{problem}
\begin{problem}[WebAssign HW 35, \# 15]
Find the complex conjugate of the number $-4+6\sqrt{5}i$. Find the modulus
of the number.
\end{problem}
\begin{problem}[WebAssign HW 35, \# 16]
Find all solutions to the equation.
\[
4x^2+16=0.
\]
\end{problem}
\begin{problem}[WebAssign HW 35, \# 17]
Find all solutions to the equation.
\[
x^4=256.
\]
\end{problem}
\begin{problem}[WebAssign HW 35, \# 18]
Find all solutions to the equation.
\[
x^2+5x+7=0.
\]
\end{problem}

\subsubsection{Homework 36}
\begin{problem}[WebAssign HW 36, \# 1]
Write the number in polar form with argument between $0$ and $2\pi$.
\[
-6+6i.
\]
\end{problem}
\begin{problem}[WebAssign HW 36, \# 2]
Write the number in polar form with argument between $0$ and $2\pi$.
\[
2i.
\]
\end{problem}
\begin{problem}[WebAssign HW 36, \# 3]
Find polar forms for $zw$, $z/w$, and $1/z$ by first putting $z$ and $w$
into polar form.
\[
\begin{aligned}
z&=3\sqrt{3}+3i&
w&=3+3\sqrt{3}i.
\end{aligned}
\]
\end{problem}
\begin{problem}[WebAssign HW 36, \# 4]
Write the number in the form $a+bi$.
\[
3e^{i\pi/2}.
\]
\end{problem}
\begin{problem}[WebAssign HW 36, \# 5]
Write the number in the form $a+bi$.
\[
6e^{i\pi}.
\]
\end{problem}
\begin{problem}[WebAssign HW 36, \# 6]
Write the number in the form $a+bi$.
\[
6e^{5+i\pi}.
\]
\end{problem}



%%% Local Variables:
%%% mode: latex
%%% TeX-master: "../MA166-Recitation"
%%% End:
