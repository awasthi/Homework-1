\section*{Exam II Solutions}
Students, here are the solutions to Exam II. I have color-coded the
solutions so that they match the color of the respective version of the
exam.

Students, here are the solutions to Exam II.


The following few pages go into

\begin{problem}[{\color{Red} \# 1}, {\color{Green} \#}]
Find the sum of the series
\[
\sum_{n=1}^\infty \frac{4^n}{5\left(3^{2n-1}\right)}
\]
\end{problem}
\begin{proof}[Solution]
This is a geometric series and it's not hard to see that. The first thing
you should do is factor it
\begin{align*}
\sum_{n=1}^\infty \frac{4^n}{5\left(3^{2n-1}\right)}
&=\frac{4}{15}\sum_{n=1}^\infty\frac{4^{n-1}}{3^{2n-2}}\\
&=\frac{4}{15}\sum_{n=1}^\infty\frac{4^{n-1}}{3^{2(n-1)}}\\
\intertext{now shift it back to turn it into a geometric series}
&=\frac{4}{15}\sum_{n=0}^\infty\frac{4^{n}}{3^{2n}}\\
&=\frac{4}{15}\sum_{n=0}^\infty\left(\frac{4}{3^2}\right)^n\\
&=\frac{4}{15}\sum_{n=0}^\infty\left(\frac{4}{9}\right)^n
\intertext{since $|4/9|<1$, this sequence converges and it converges to }
&=\frac{4}{15}\left(\frac{1}{1-\frac{4}{9}}\right)\\
&=\frac{4}{15}\left(\frac{1}{\frac{5}{9}}\right)\\
&=\boxed{\frac{12}{25}.}
\end{align*}
Answer: {\color{Red} D}, {\color{Green}}
\end{proof}

\begin{problem}[{\color{Red} \# 2}, {\color{Green} \#}]
Evaluate the integral
\[
\int_0^1\frac{x^2+1}{(x+1)^2}dx.
\]
\end{problem}
\begin{proof}[Solution]
Rewrite the integral and use partial fractions
\begin{align*}
\int_0^1\frac{x^2+1}{(x+1)^2}dx
&=\int_0^1\frac{(x^2+1+2x)-2x}{(x+1)^2}dx\\
&=\int_0^1\left[\frac{(x^2+1+2x)}{(x+1)^2}-\frac{2x}{(x+1)^2}\right]dx\\
&=\int_0^1\left[\frac{(x+1)^2}{(x+1)^2}-\frac{2x}{(x+1)^2}\right]dx\\
&=\int_0^1\left[1-\frac{2x}{(x+1)^2}\right]dx\\
&=\underbrace{\int_0^11dx}_{I_1}-\underbrace{\int_0^1\frac{2x}{(x+1)^2}dx}_{I_2}.
\end{align*}
Now, rewrite $I_1=1$ and that's easy. To find $I_2$ we use partial
fractions
\begin{align*}
\frac{2x}{(x+1)^2}&=\frac{A}{x+1}\frac{B}{(x+1)^2}\\
2x&=A(x+1)+B\\
 &=Ax+(A+B).
\end{align*}
So $A+B=0$ and $A=2$ so $B=-2$. Now we compute $I_2$
\begin{align*}
I_2&=\int_0^1\frac{2x}{(x+1)^2}dx\\
   &=\int_0^1\left[\frac{2}{x+1}-\frac{2}{(x+1)^2}\right]dx\\
   &=\int_0^1\frac{2}{x+1}dx-\int_0^1\frac{2}{(x+1)^2}dx\\
   &=\left[2\ln|x+1|+\frac{2}{x+1}\right]_0^1\\
   &=2\ln 2-1.
\end{align*}
Hence the integral is
\[
I_1-I_2=1-(2\ln 2-1)=\boxed{2-2\ln 2.}
\]
Answer: {\color{Red} E}.
\end{proof}

\begin{problem}[{\color{Red} \# 3}, {\color{Green} \#}]
Evaluate the integral
\[
\int_0^1 \frac{x^2}{x^2+1}dx.
\]
\end{problem}
\begin{proof}[Solution]
Factor and use partial fractions
\begin{align*}
\int_0^1 \frac{x^2}{x^2+1}dx
&=\int_0^1\frac{x^2+1-1}{x^2+1}dx\\
&=\int_0^1\frac{\left(x^2+1\right)-1}{x^2+1}dx\\
&=\int_0^1\left[\frac{x^2+1}{x^2+1}-\frac{1}{x^2+1}\right]dx\\
&=\int_0^1\left[1-\frac{1}{x^2+1}\right]dx\\
&=\underbrace{\int_0^11dx}_{I_1}-\underbrace{\int_0^1\frac{1}{x^2+1}dx}_{I_2}.
\end{align*}
It's easy to compute $I_1=1$. To compute $I_2$ you can either use a trig
substitution or realize that the integral of $1/(x^2+1)$ is
$\tan^{-1}(x)$.

Using the trig substitution, let $x=\tan\theta$, $dx=\sec^2\theta d\theta$
we have
\begin{align*}
\int_0^{/pi/4}\frac{1}{x^2+1}dx
&=\int_0^{\pi/4}\sec^2\theta\cos^2\theta d\theta\\
&=\int_0^11d\theta\\
&=\left[\theta\right]_0^{\pi/4}\\
&=\frac{\pi}{4}.
\end{align*}
Then the integral is
\[
I_1-I_2=\boxed{1-\frac{\pi}{4}.}
\]
Answer: {\color{Red} B}.
\end{proof}

\begin{problem}[{\color{Red} \#}, {\color{Green} \#}]
Which of the following statements are true?
\begin{enumerate}[label=(\MakeUppercase{\roman*})]
\item The sequence $\displaystyle a_n=\sin(n\pi)$ is convergent.
\item The sequence $\displaystyle a_n=\frac{2n^3+1}{n-n^3}$ is divergent.
\item The sequence $\displaystyle a_n=e^{\left(\frac{2n}{n+2}\right)}$ is
  convergent.
\end{enumerate}
\end{problem}
\begin{proof}[Solution]
(II) clearly converges. First rewrite
\begin{align*}
\frac{2n^3+1}{n-n^3}
&=-\frac{2n^3+1}{n^3-n}
\intertext{make the substitution $n=x$ and use l'Hôpital's rule}
&=-\frac{6x^2}{3x^2-1}\\
&=-\frac{12x}{6x}\\
&=-2.
\end{align*}
\\\\
(III) converges because the sequence $2n/(n+2)$ converges to $2$, so
$a_n\to e^2$.
\\\\
(I) is well known to not converge since $\sin
\pi x$ changes value from $-1$ to $1$ and as we get closer and closer to
infinity, it keeps on moving between these two values.
\\\\
Answer: {\color{Red} E.}
\end{proof}

\begin{problem}[{\color{Red} \# 5}, {\color{Green} \#}]
Which of the following statements are true?
\begin{enumerate}[label=(\MakeUppercase{\roman*})]
\item Every positive bounded sequence is convergent.
\item The sequence $\displaystyle a_n=\frac{n\cos n}{n^2+3}$ is
  convergent.
\item The sequence $\displaystyle a_n=\frac{3^n}{2^{n+1}}$ is convergent.
\end{enumerate}
\end{problem}
\begin{proof}[Solution]
(I) is false. Just consider $|\sin(\pi n/2)|$. This sequence goes from $0$
to $1$, $0$ to $1$, $0$ to $1$ indefinitely. This sequence is positive and
bounded, but it does not converge.
\\\\
(II) By l'Hôpital's as $n\to\infty$, $1+3/n^2\to 1$ and
$n(1+3/n^2)\to\infty$ as $n\to\infty$ so
\[
\lim_{n\to\infty}\frac{\cos n}{n(1+\frac{3}{n^2})}\to 0.
\]
\end{proof}

\begin{problem}[{\color{Red} \# 6}, {\color{Green} \#}]
Evaluate the integral
$\displaystyle\int_0^1\frac{x^2}{\sqrt{1-x^2}}dx$. Hint:
$\cos(2t)=1-2\sin^2 t$.
\end{problem}
\begin{proof}[Solution]
Use a trigonometric substitution $\sin t=x$, $\cos t\;dt= x$ so $0\leq
t\leq\pi/2$
\begin{align*}
\int_0^1\frac{x^2}{\sqrt{1-x^2}}dx
&=\int_0^{\pi/2}\frac{\sin^2 t}{\cos t}\cos t\;dt\\
&=\int_0^{\pi/2}\sin^2 t\;dt\\
&=\frac{1}{2}\left[\int_0^{\pi/2}1-\cos 2t\right]\;dt\\
&=\frac{1}{2}\left[\int_0^{\pi/2}1\;dt-\int_0^{\pi/2}\cos 2t\;dt\right]\\
&=\frac{1}{2}\left[t-\frac{1}{2}\cos 2t\right]_0^{\pi/2}\\
&=\frac{1}{2}\left[\frac{\pi}{2}-(-1)-\left(0-1\right)\right]_0^{\pi/2}\\
&=\boxed{\frac{\pi}{4}.}
\end{align*}
Answer: {\color{Red} E.}
\end{proof}

\begin{problem}[{\color{Red} \# 7}, {\color{Green} \#}]
Evaluate the integral
\[
\int_4^9\frac{\sqrt{x}}{x-1}dx.
\]
Hints: $\displaystyle\frac{d}{dx}\sqrt{x}=\frac{1}{2\sqrt{x}}$,
$\displaystyle\frac{2}{u^2-1}=\frac{1}{u-1}-\frac{1}{u+1}$.
\end{problem}
\begin{proof}[Solution]
Make the substitution $u^2=x$, $2u\;du=dx$. Then
\begin{align*}
\int_4^9\frac{\sqrt{x}}{x-1}dx
&=\int_2^3\frac{u}{u^2-1}2u\;du\\
&=\int_2^3\frac{2u^2}{u^2-1}du\\
&=2\int_2^3\frac{u^2}{u^2-1}du\\
&=2\int_2^3\frac{u^2-1+1}{u^2-1}du\\
&=2\left[\int_2^3\left(1+\frac{1}{u^2-1}\right)\;du\right]\\
&=2\int_2^3 1\;du+\int_2^3\frac{2}{u^2-1}du\\
&=2\int_2^3 1\;du+\int_2^3\left[\frac{1}{u-1}-\frac{1}{u+1}\right]du\\
&=\left[2u+\ln\left|\frac{u-1}{u+1}\right|\right]_2^3\\
&=\left[6+\ln\left|\frac{2}{4}\right|-4-\ln\left|\frac{1}{3}\right|\right]\\
&=\boxed{2+\ln(3/2)}.
\end{align*}
Answer: {\color{Red} A.}
\end{proof}

\begin{problem}[{\color{Red} \# 8}, {\color{Green} \#}]
Find the arc length of the curve $y=2x^{3/2}$, $0\leq x\leq 3$.
\end{problem}
\begin{proof}[Solution]
Find the derivative
\[
\frac{dy}{dx}=3\sqrt{x}.
\]
Then
\begin{align*}
\int_0^3\sqrt{1+\left(\frac{dy}{dx}\right)^2}\;dx
&=\int_0^3\sqrt{1+\left(3\sqrt{x}\right)^2}\;dx\\
&=\int_0^3\sqrt{1+9x}\;dx\\
\intertext{make the substitution $u=1+9x$, $du=9\;dx$, $1\leq u\leq 28$}
&=\frac{1}{9}\int_1^{28}\sqrt{u}\;du\\
&=\int_1^{28}u^{1/2}\;du\\
&=\frac{2}{27}\left[u^{3/2}\right]_1^{28}\\
&=\boxed{\frac{2}{27}\left(28^{3/2}-1\right).}
\end{align*}
Answer: {\color{Red} E.}
\end{proof}

\begin{problem}[{\color{Red} \# 9}, {\color{Green} \#}]
The curve
\[
x=\frac{1}{3}\left(y^2+2\right)^{3/2},\qquad 1\leq y\leq 2,
\]
is rotated about the $y$-axis. The area of the resulting surface is
\[
\int_1^2\frac{2\pi}{3}\left( y^2+2\right)^{3/2}(y^2+k)\;dy
\]
for some constant $k$. What is $k$?
\end{problem}
\begin{proof}[Solution]
What we are really looking for is the simplification of
\[
\sqrt{1+\left(\frac{dx}{dy}\right)^2}.
\]
We need to find
\[
\frac{dx}{dy}=y\sqrt{y^2+1}
\]
so
\begin{align*}
\sqrt{1+\left(\frac{dx}{dy}\right)^2}
&=\sqrt{1+\left(y\sqrt{y^2+2}\right)^2}\\
&=\sqrt{1+y^2\left(y^2+2\right)}\\
&=\sqrt{1+y^4+2y^2}\\
&=\sqrt{y^4+2y^2+1}\\
&=\sqrt{\left(y^2+1\right)^2}\\
&=y^2+1.
\end{align*}
If we compare this to
$\displaystyle\int_1^2\frac{2\pi}{3}\left(y^2+2\right)^{3/2}\left(y^2+k\right)$
we see that $k=1$.
\\\\
Answer: {\color{Red} C.}
\end{proof}

\begin{problem}[{\color{Red} \# 10}, {\color{Green} \#}]
Find the $x$-coordinate, $\bar x$, of the centroid of the region bounded by
$y=-2x+3$, $y=0$, $x=0$ and $x=1$.
\end{problem}
\begin{proof}[Solution]
First we compute the area of the region
\begin{align*}
A&=\int_0^1-2x+3\\
 &=\left[-x^2+3x\right]_0^1\\
 &=2.
\end{align*}
Then the mass is $2\rho $ and the moment about the $y$-axis is
\begin{align*}
M_y&=\rho\int_0^1 x(-2x+3)\;dx\\
   &=\rho\int_0^1 -2x^2+3x\;dx\\
   &=\rho\left[-\frac{2}{3}x^3+\frac{3}{2}x^2\right]_0^1\\
   &=\rho\left[-\frac{2}{3}+\frac{3}{2}\right]_0^1\\
   &=\frac{5}{6}\rho.
\end{align*}
So
\[
\bar x=\frac{M_y}{m}=\frac{(5/6)rho}{2\rho}=\boxed{\frac{5}{12}.}
\]
Answer: {\color{Red} D}.
\end{proof}

\begin{problem}[{\color{Red} \#}, {\color{Green} \#}]
Evaluate the integral
\[
\int_0^{\pi/3}\tan^3 x\sec x\;dx.
\]
\end{problem}
\begin{proof}[Solution]
Use the following trig identity
\[
\sec^2 x-\tan ^2x=1.
\]
Rewrite the integral
\begin{align*}
\int_0^{\pi/3}\tan^3 x\sec x\;dx
&=\int_0^{\pi/3}\left(\tan^2 x\right)\tan x\sec x\;dx\\
&=\int_0^{\pi/3}\left(\sec^2 x-1\right)\tan x\sec x\;dx\\
\intertext{make the substitution $u=\sec x$, $du =\tan x\sec x\;dx$}
&=\int_1^2\left(u^2-1\right)\tan x\sec x\frac{du}{\tan x\sec x}\\
&=\int_1^2 u^2-1\;du\\
&=\left[\frac{1}{3}u^3-u \right]_1^2\\
&=\frac{8}{3}-2-\frac{1}{3}+1\\
&=\frac{7}{3}-1\\
&=\boxed{\frac{4}{3}.}
\end{align*}
Answer: {\color{Red} C.}
\end{proof}

\begin{problem}[{\color{Red} \# 12}, {\color{Green} \#}]
Evaluate the integral $\displaystyle\int_0^{\pi/2}\frac{\cos
  t}{\sqrt{1+\sin^2 t}}\;dt$ using the table of integrals formula
$\displaystyle\int \frac{du}{1+u^2}=\ln\left(u+\sqrt{1+u^2}\right)+C$.
\end{problem}
\begin{proof}[Solution]
Set $u=\sin t$, $du=\cos t\;dt$, then we have the integral
\begin{align*}
\int_0^{\pi/2}\frac{\cos t}{\sqrt{1+\sin^2 t}}\;dt
&=\int_0^1\frac{1}{1+u^2}\;dt\\
&=\left[\ln\left(u+\sqrt{1+u^2}\right)\right]_0^1\\
&=\ln\left(1+\sqrt{2}\right)-\ln \left(0+\sqrt{1+0}\right)\\
&=\boxed{\ln\left( 1+\sqrt{2} \right).}
\end{align*}
Answer: {\color{Red} A.}
\end{proof}

%%% Local Variables:
%%% mode: latex
%%% TeX-master: "../MA166-Recitation"
%%% End:
