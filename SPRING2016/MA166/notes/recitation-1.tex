\chapter{Notes: Vectors and the Geometry of Spaces}
Material found in Stewart \S12.
\\\\
\section{Three-Dimensional Coordinate Systems}
Here are some of the most important concepts, equations, and theorems from
this section. I know. I know. These are very boring concepts that you have
probably seen all your life and you know how to do. But we must start
somewhere and here is a perfect place.

The distance between two points $P_1(x,y,z)$ and $P_2(x,y,z)$ in $\bbR^3$
is given by the formula
\begin{equation}
  \label{eq:distance-between-points}
|P_1P_2|=\sqrt{(x_2-x_1)^2+(y_2-y_1)^2+(z_2-z_1)^2}.
\end{equation}
This is also called the \emph{Euclidean norm} and generalizes to all
dimensions. Note that equation (\ref{eq:distance-between-points}) is
equivalent to
\[
\sqrt{(x_1-x_2)^2+(y_1-y_2)^2+(z_1-z_2)^2}=|P_2P_1|
\]
so that the distance between point does not depend on your point-of-view,
i.e, whether you think of the line starting connecting $P_1$ and $P_2$ as
starting at $P_1$ and ending at $P_2$ or vice-a-versa.

We often refer to the point $P_1(x,y,z)$ as the tuple $(x_1,y_1,z_2)$ and
$P_2(x,y,z)$ as $(x_2,y_2,z_2)$, $P_3(x,y,z)$ as $(x_3,y_3,z_3)$ and so on.
\\\\
The equation of a sphere with $C(h,k,l)$ and radius $r$ is
\begin{equation}
  \label{eq:equation-of-a-sphere}
(x-h)^2+(y-k)^2+(z-l)^2=r^2.
\end{equation}
In particular, if the center is the origin $O$, then the equation
(\ref{eq:equation-of-a-sphere}) reduces to
\[
x^2+y^2+z^2=r^2.
\]
\section{Vector}
A particle moves along a line segment from point $A$ to point $B$. The
corresponding displacement vector $\bfv$ has initial point $A$ and terminal
point $B$ and is written $\bfv=\overrightarrow{AB}$.
\subsection{Combining Vectors}
If $\bfu$ and $\bfv$ are vectors positioned so the initial point of $\bfv$
is at the terminal point of $\bfu$, then the sum $\bfu+\bfv$ is the vector
from the initial point of $\bfu$ to the terminal point of $\bfv$.

\section{The Cross Product}
\begin{definition}
If $\bfu=\langle u_1,u_2,u_3\rangle$ and $\bfv=\langle v_1,v_2,v_3\rangle$,
then the \emph{cross product} of $\bfu$ and $\bfv$ is the vector
\[
\bfu\times\bfv=\langle u_2v_3-u_3v_2,u_3v_1-u_1v_3,u_1v_2-u_2v_1\rangle.
\]
\end{definition}
\begin{theorem}
The vector $\bfw=\bfu\times\bfv$ is orthogonal to both $\bfu$ and $\bfv$,
i.e., $\bfw\cdot\bfv=0$  and $\bfw\cdot\bfu$.
\end{theorem}
\begin{theorem}
If $\theta$ is the angle between $\bfu$ and $\bfv$ (so
$0\leq\theta\leq\pi$), then
\[
|\bfu\times\bfv|=|\bfu||\bfv|\sin\theta.
\]
\end{theorem}
\begin{theorem}
Two nonzero vectors $\bfu$ and $\bfv$ are parallel if and only if
\[
\bfv\times\bfu=\mathbf{0}
\]
\end{theorem}
The length of the product $\bfv\times\bfu$ is equal to the area of the
parallelogram determined by $\bfu$ and $\bfv$.
\begin{theorem}
If $\bfu$, $\bfv$, and $c$ are vectors and $c$ is a scalar, then
\begin{enumerate}[label=\textnormal{(\alph*)}]
\item $\bfu\times\bfv=-\bfv\times\bfu$.
\item $(c\bfu)\times\bfv=c(\bfu\times\bfv)=\bfu\times(c\bfv)$.
\item $\bfu\times(\bfv+\bfw)=\bfu\times\bfv+\bfu\times\bfw$.
\item $(\bfu+\bfv)\times\bfw=\bfu\times\bfw+\bfv\times\bfw$.
\item $\bfu\cdot(\bfv\times\bfw)=(\bfu\times\bfv)\cdot\bfw$.
\item $\bfu\times(\bfv\times\bfw)=(\bfu\cdot\bfw)\bfv-(\bfu\cdot\bfv)\bfw$.
\end{enumerate}
\end{theorem}
\bigskip
The volume of a parallelepiped determined by $\bfu$, $\bfv$, and $\bfv$ and
is the magnitude of their scalar triple product:
\begin{equation}
\label{eq:volume-of-a-parallelepiped}
V=|\bfu\cdot(\bfv\times\bfw)|.
\end{equation}
\subsection{Torque}
Consider a force $\mathbf{F}$ acting on a rigid body at a point given by
the position vector $\mathbf{r}$. The torque (relative to the origin) is
defined to be
\begin{equation}
\label{eq:torque}
\bm{\mathrm{\tau}}=\mathbf{r}\times\mathbf{F}.
\end{equation}

%%% Local Variables:
%%% mode: latex
%%% TeX-master: "../MA166-Recitation"
%%% End:
