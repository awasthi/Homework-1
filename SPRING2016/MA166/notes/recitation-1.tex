\chapter{Notes: Vectors and the Geometry of Spaces}
Material found in Stewart \S12.
\\\\
\section{Three-Dimensional Coordinate Systems}
Here are some of the most important concepts, equations, and theorems from
this section. I know. I know. These are very boring concepts that you have
probably seen all your life and you know how to do. But we must start
somewhere and here is a perfect place.

The distance between two points $P_1(x,y,z)$ and $P_2(x,y,z)$ in $\bbR^3$
is given by the formula
\begin{equation}
  \label{eq:distance-between-points}
|P_1P_2|=\sqrt{(x_2-x_1)^2+(y_2-y_1)^2+(z_2-z_1)^2}.
\end{equation}
This is also called the \emph{Euclidean norm} and generalizes to all
dimensions. Note that equation (\ref{eq:distance-between-points}) is
equivalent to
\[
\sqrt{(x_1-x_2)^2+(y_1-y_2)^2+(z_1-z_2)^2}=|P_2P_1|
\]
so that the distance between point does not depend on your point-of-view,
i.e, whether you think of the line starting connecting $P_1$ and $P_2$ as
starting at $P_1$ and ending at $P_2$ or vice-a-versa.

We often refer to the point $P_1(x,y,z)$ as the tuple $(x_1,y_1,z_2)$ and
$P_2(x,y,z)$ as $(x_2,y_2,z_2)$, $P_3(x,y,z)$ as $(x_3,y_3,z_3)$ and so on.
\\\\
The equation of a sphere with $C(h,k,l)$ and radius $r$ is
\begin{equation}
  \label{eq:equation-of-a-sphere}
(x-h)^2+(y-k)^2+(z-l)^2=r^2.
\end{equation}
In particular, if the center is the origin $O$, then the equation
(\ref{eq:equation-of-a-sphere}) reduces to
\[
x^2+y^2+z^2=r^2.
\]
\section{Vector}
A particle moves along a line segment from point $A$ to point $B$. The
corresponding displacement vector $\bfv$ has initial point $A$ and terminal
point $B$ and is written $\bfv=\overrightarrow{AB}$.
\subsection{Combining Vectors}
If $\bfu$ and $\bfv$ are vectors positioned so the initial point of $\bfv$
is at the terminal point of $\bfu$, then the sum $\bfu+\bfv$ is the vector
from the initial point of $\bfu$ to the terminal point of $\bfv$.

%%% Local Variables:
%%% mode: latex
%%% TeX-master: "../MA166-Recitation"
%%% End:
