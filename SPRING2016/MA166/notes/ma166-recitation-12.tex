\chapter{Homework}
\section{This Week's Summary}
Here's a summary of the material that was (presumably) covered this week.
Sections from Stewart
\subsection{\S 10.1: Parametric Equations and Polar Coordinates}
Suppose that $x$ and $y$ are defined in terms of another third variable
$t$ (called the \emph{parameter}) by the equations
\[
\begin{aligned}
x&=f(x)&y&=g(t).
\end{aligned}
\]
(called \emph{parametric equations}). Each value of $t$ determines a point
$(x,y)$ which we can plot in the coordinate plane $\bbR^2$. As $t$ varies,
the point $(x,y)=(f(t),g(t))$ varies and traces out a curve $C$, which we
call the \emph{parametric curve}. The parameter $f$ does not necessarily
represent time, and in fact,

\section{Homework Problems}
Solutions to selected problems:
\subsection{Homework 31}
\begin{problem}[WebAssign HW 31, \# 1]
Select the curve generated by the parametric equations. Indicate with an
arrow the direction in which the curve is traced as $t$ increases.
\[
  \begin{aligned}
    x&=e^{-t}+t,&y&=e^t-t,&-2&\leq t\leq 2.
  \end{aligned}
\]
\end{problem}
\begin{problem}[WebAssign HW 31, \# 2]
Consider the following equations.
\[
  \begin{aligned}
    x&=1-t^2,&y&=t-3,&-2&\leq t\leq 2.
  \end{aligned}
\]
\begin{enumerate}[label=(\alph*)]
\item Sketch the curve using the parametric equations to plot
  points. Indicate with an arrow the direction in which the curve is traced
  out as $t$ increases.
\item Eliminate the parameter to find a Cartesian equation of the curve for
  $-5\leq y\leq -1$.
\end{enumerate}
\end{problem}
\begin{problem}[WebAssign HW 31, \# 3]
Consider the parametric equations below.
\[
  \begin{aligned}
    x&=\sqrt{t},&y&=11-t.
  \end{aligned}
\]
\begin{enumerate}[label=(\alph*)]
\item Sketch the curve using the parametric equations to plot
  points. Indicate with an arrow the direction in which the curve is traced
  out as $t$ increases.
\item Eliminate the parameter to find a Cartesian equation of the curve for
  $x\geq 0$.
\end{enumerate}
\end{problem}
\begin{problem}[WebAssign HW 31, \# 4]
Consider the following.
\[
  \begin{aligned}
    x&=\sin\tfrac{1}{2}\theta,
    &y&=\cos\tfrac{1}{2}\theta,
    &-\pi&\leq\theta\leq\pi.
  \end{aligned}
\]
\begin{enumerate}[label=(\alph*)]
\item Sketch the curve using the parametric equations to plot
  points. Indicate with an arrow the direction in which the curve is traced
  out as $t$ increases.
\item Eliminate the parameter to find a Cartesian equation of the curve for
  $-5\leq y\leq -1$.
\end{enumerate}
\end{problem}
\begin{problem}[WebAssign HW 31, \# 5]
Consider the following.
\[
  \begin{aligned}
    x&=\sin t,&
    y&=\csc t&
    0&<t<\pi/2.
  \end{aligned}
\]
(a) Eliminate the parameter to find a Cartesian equation of the curve.

(b) Sketch the curve and indicate with an arrow the direction in which the
curve is traced as the parameter increases.
\end{problem}
\begin{problem}[WebAssign HW 31, \# 6]
Describe the motion of a particle with position (x, y) as t varies in the
given interval.
\[
  \begin{aligned}
    x&=2+2\cos t,&
    y&=1+2\sin t,&
    \pi/2&\leq t\leq 3\pi/2.
  \end{aligned}
\]
\end{problem}
\begin{problem}[WebAssign HW 31, \# 7]
Describe the motion of a particle with position $(x,y)$ as t varies in the
given interval.
\[
  \begin{aligned}
    x&=2\sin t,&
    y&=1+\cos t,&
    0&\leq t \leq 3\pi/2.
  \end{aligned}
\]
\end{problem}
\begin{problem}[WebAssign HW 31, \# 8]
Describe the motion of a particle with position $(x,y)$ as t varies in the
given interval.
\[
  \begin{aligned}
    x&=4\sin t,&
    y&=9\cos t,&
    -\pi&\leq t\leq 9\pi.
  \end{aligned}
\]
\end{problem}
\begin{problem}[WebAssign HW 31, \# 9]
 Match the graphs of the parametric equations $x=f(t)$ and $y=g(t)$ in
 (a)--(d) with the parametric curves labeled I--IV.
\end{problem}
\begin{problem}[WebAssign HW 31, \# 10]
 Use the graphs of $x=f(t)$ and $y=g(t)$ to sketch the parametric curve
 $x=f(t)$, $y=g(t)$.  Indicate with arrows the direction in which the curve
 is traced as $t$ increases.
\end{problem}
\subsection{Homework 32}
\begin{problem}[WebAssign HW 32, \# 1]
Find $dy/dx$.
\[
  \begin{aligned}
    x&=t\sin t,&y&=t^2+3t.
  \end{aligned}
\]
\end{problem}
\begin{problem}[WebAssign HW 32, \# 2]
Find $dy/dx$.
\[
  \begin{aligned}
    x&=7/t,&y&=\sqrt{t}e^{-t}.
  \end{aligned}
\]
\end{problem}
\begin{problem}[WebAssign HW 32, \# 3]
Find an equation of the tangent to the curve at the point corresponding to
the given value of the parameter.
\[
\begin{aligned}
x&=t-t^{-1},&y&=9+t^2,&t=1.
\end{aligned}
\]
\end{problem}
\begin{problem}[WebAssign HW 32, \# 4]
Find $dy/dx$ and $d^2y/dx^2$.
\[
  \begin{aligned}
    x&=e^t,&y&=te^{-t}.
  \end{aligned}
\]
For which values of $t$ is the curve concave upward?
\end{problem}
\begin{problem}[WebAssign HW 32, \# 5]
Find $dy/dx$ and $d^2y/dx^2$.
\[
  \begin{aligned}
    x&=2\sin t,&y&=3\cos t,&0&<t<2\pi.
  \end{aligned}
\]
For which values of $t$ is the curve concave upward?
\end{problem}
\begin{problem}[WebAssign HW 32, \# 6]
Find the exact length of the curve.
\[
  \begin{aligned}
    x&=3+t^2,&y&=3+2t^3,&0&\leq t\leq 2.
  \end{aligned}
\]
\end{problem}
\begin{problem}[WebAssign HW 32, \# 7]
Find the exact length of the curve
\[
  \begin{aligned}
    x&=e^t+e^{-t},&y&=5-2t,&0&\leq t\leq 4.
  \end{aligned}
\]
\end{problem}
\begin{problem}[WebAssign HW 32, \# 8]
Find the distance traveled by a particle with position $(x,y)$ as $t$
varies in the given time interval.
\[
  \begin{aligned}
    x&=3\sin^2 t,&y&=3\cos^2 t,&0&\leq t\leq 3\pi.
  \end{aligned}
\]
What is the length of the curve?
\end{problem}
\begin{problem}[WebAssign HW 33, \# 1]
Find two other pairs of polar coordinates of the given polar coordinate,
one with $r>0$ and one with $r<0$. Then plot the point.
\begin{enumerate}[label=(\alph*)]
\item $(5,\pi/4)$
\item $(4,-2\pi/3)$
\item $(-4,\pi/6)$
\end{enumerate}
\end{problem}
\begin{problem}[WebAssign HW 33, \# 2]
Find the Cartesian coordinates of the given polar coordinates. Then plot
the point.
\begin{enumerate}[label=(\alph*)]
\item $(5,\pi)$
\item $(6,-2\pi/3)$
\item $(-6,3\pi/4)$
\end{enumerate}
\end{problem}
\begin{problem}[WebAssign HW 33, \# 3]
The Cartesian coordinates of a point are given.
\begin{enumerate}[label=(\alph*)]
\item $(4,-4)$
  \begin{enumerate}[label=(\roman*)]
  \item Find polar coordinates $(r,\theta)$ of the point, where $r>0$ and
    $0\leq\theta<2\pi$.
  \item Find polar coordinates $(r,\theta)$ of the point, where $r<0$ and
    $0\leq\theta<2\pi$.
  \end{enumerate}
\item $(-1,\sqrt{3})$
  \begin{enumerate}[label=(\roman*)]
  \item Find polar coordinates $(r,\theta)$ of the point, where $r>0$ and
    $0\leq\theta<2\pi$.
  \item Find polar coordinates $(r,\theta)$ of the point, where $r<0$ and
    $0\leq\theta<2\pi$.
  \end{enumerate}
\end{enumerate}
\end{problem}
\begin{problem}[WebAssign HW 33, \# 4]
The Cartesian coordinates of a point are given.
\begin{enumerate}[label=(\alph*)]
\item $(2\sqrt{3},2)$
  \begin{enumerate}[label=(\roman*)]
  \item Find polar coordinates $(r,\theta)$ of the point, where $r>0$ and
    $0\leq\theta<2\pi$.
  \item Find polar coordinates $(r,\theta)$ of the point, where $r<0$ and
    $0\leq\theta<2\pi$.
  \end{enumerate}
\item $(1,-3)$
  \begin{enumerate}[label=(\roman*)]
  \item Find polar coordinates $(r,\theta)$ of the point, where $r>0$ and
    $0\leq\theta<2\pi$.
  \item Find polar coordinates $(r,\theta)$ of the point, where $r<0$ and
    $0\leq\theta<2\pi$.
  \end{enumerate}
\end{enumerate}
\end{problem}
\begin{problem}[WebAssign HW 33, \# 5]
Sketch the region in the plane consisting of points whose polar coordinates
satisfy the given conditions
\[
  \begin{aligned}
    2&<r<5,&3\pi/2&\leq\theta\leq 5\pi/2.
  \end{aligned}
\]
\end{problem}
\begin{problem}[WebAssign HW 33, \# 6]
Sketch the region in the plane consisting of points whose polar coordinates
satisfy the given conditions
\[
  \begin{aligned}
    r&\geq 5,&\pi&\leq\theta\leq 2\pi
  \end{aligned}
\]
\end{problem}
\begin{problem}[WebAssign HW 33, \# 7]
Find a Cartesian equation for the curve and identify it.
\[
r^2\cos 2\theta=1.
\]
\end{problem}
\begin{problem}[WebAssign HW 33, \# 8]
Find a Cartesian equation for the curve and identify it.
\[
r=4\tan\theta\sec\theta.
\]
\end{problem}
\begin{problem}[WebAssign HW 33, \# 9]
Find a polar equation for the curve represented by the given Cartesian
equation.
\[
  x^2+y^2=10cx.
\]
\end{problem}
\begin{problem}[WebAssign HW 33, \# 10]
Find a polar equation for the curve represented by the given Cartesian
equation.
\[
xy=11.
\]
\end{problem}

\chapter{Exam Problems}
\section{Exam 3: Spring 2014}
\setcounter{exercise}{0}
\begin{problem}
Compute
\[
\sum_{n=1}^\infty \frac{2^{n-1}-(-3)^{n+1}}{4^n}.
\]
\end{problem}
\begin{problem}
This series
\[
\sum_{n=1}^\infty\frac{1}{({n^{2p}+1})^{1/6}}
\]
is convergent if and only if
\end{problem}
\begin{problem}
Test the following series for convergence or divergence.
\begin{enumerate}[label=(\alph*)]
\item $\displaystyle\sum_{n=1}^\infty \sin\left(\frac{1}{n}\right)$.
\item $\displaystyle\sum_{n=1}^\infty(-1)^n\arctan n$.
\item $\displaystyle\sum_{n=1}^\infty\frac{n+1}{n\sqrt{n}}$.
\end{enumerate}
\end{problem}
\begin{problem}
Determine whether the following series are absolutely convergent,
conditionally convergent or divergent.
\begin{enumerate}[label=(\alph*)]
\item $\displaystyle\sum_{n=1}^\infty(-1)^{n-1}\frac{2n}{3n+5}$.
\item $\displaystyle\sum_{n=1}^\infty(-1)^{n-1}\frac{1}{n^3}$.
\item $\displaystyle\sum_{n=2}^\infty(-1)^{n-1}\frac{\ln n}{n}$.
\end{enumerate}
\end{problem}
\begin{problem}
Test the following series for convergence or divergence.
\begin{enumerate}[label=(\alph*)]
\item $\displaystyle\sum_{n=1}^\infty\frac{n!}{n^n}$.
\item $\displaystyle\sum_{n=1}^\infty\left(\frac{3n+2}{2n+3}\right)^n$.
\item $\displaystyle\sum_{n=1}^\infty\frac{n}{5^n}$.
\end{enumerate}
\end{problem}
\begin{problem}
Which of the following statements are \emph{always true}?
\begin{enumerate}[label=(\MakeUppercase{\roman*})]
\item If $\lim_{n\to\infty} a_n=0$, then $\sum_{n=1}^\infty a_n$
  converges.
\item If $\lim_{n\to\infty} n^3|a_n|=0$, then
  $\sum_{n=1}^\infty(-1)^{n+1}a_n$ converges.
\item $\sum_{n=1}^\infty (e^n+c)/e^{2n}$ converges for any positive value
  $c$.
\end{enumerate}
\end{problem}
\begin{problem}
Given the following series
\[
\sum_{n=1}^\infty\frac{3}{2^n+n-1}.
\]
Mark, Nancy and David provide the following ingredient of the arguments for
convergence or divergence of the series:
\begin{enumerate}[label=(\alph*)]
\item the name of the test to use,
\item the conclusion for convergence or divergence
\end{enumerate}
\begin{itemize}
\item[Mark:] (a) $b_n=3/2^n$, comparison test ($0\leq a_n\leq b_n$); (b)
  convergent
\item[Nancy:] (a) $b_n=1/n$, limit comparison test ($\lim_{n\to\infty}
  a_n/b_n=3$); (b) divergent
\item[David:] (a) ratio test ($\lim_{n\to\infty}|a_{n+1}|/|a_n|=1/2$); (b)
  convergent.
\end{itemize}
Choose the name(s) of the person(s) with correct arguments.
\end{problem}
\begin{problem}
Consider the Maclaurin series for $e^x$
\[
e^x=\sum_{n=0}^\infty\frac{x^n}{n!}=1+\frac{x}{1!}+\frac{x^2}{2!}+\frac{x^3}{3!}+\dotsb.
\]
By plugging in $x=-1$, one obtains the alternating series
\[
e^{-1}=1-\frac{1}{1!}+\frac{1}{2!}-\frac{1}{3!}+\frac{1}{4!}-\frac{1}{5!}+\dotsb.
\]
If we compute the sum of the \emph{fewest} terms necessary to guarantee
that the error is less than $0.05$, \emph{using the estimation theorem for
  alternating series}, then what is the estimate for $e^{-1}$?
\end{problem}
\begin{problem}
Suppose the power series
\[
\sum_{n=0}^\infty c_n(x-3)^n
\]
converges when $x=1$, but diverges when $x=7$.

From the above information, which of the following statements can we
conclude to be true?
\begin{enumerate}[label=(\MakeUppercase{\roman*})]
\item The radius of convergence is $R\geq 2$.
\item The power series converges at $x=4.5$.
\item The power series diverges at $x=6.5$.
\end{enumerate}
\end{problem}
\begin{problem}
What is the coefficient of $x^6$ in the power series expansion
$2/(1+2x^2)$?
\end{problem}
\begin{problem}
Determine the interval of convergence for the following power series
\[
\sum_{n=0}^\infty\frac{(-5)^n}{\sqrt{n+2}}(x-3)^n.
\]
\end{problem}
\begin{problem}
The power series representation (centered at $a=0$) for $g(x)=x/(4-x^2)$ is
given by
\[
g(x)=\sum_{n=0}^\infty\frac{x^{2n+1}}{4^{n+1}}
\]
with the interval of convergence $(-2,2)$.

Find
\begin{enumerate}[label=(\alph*)]
\item the power series representation (centered at $a=0$), and
\item the interval of convergence
\end{enumerate}
for the function
\[
f(x)=\ln|4-x^2|.
\]
\end{problem}
\section{Exam 3 Spring 2013}
\setcounter{exercise}{0}

%%% Local Variables:
%%% mode: latex
%%% TeX-master: "../MA166-Recitation"
%%% End:
