\chapter{Homework Problems}
\section{Homework 15}
\begin{problem}[WebAssign, HW 15, \# 1]
Use the table of integrals to evaluate the integral. (Remember to use
$\ln|u|$ where appropriate. Use $C$ for the constant of integration.)
\[
\int\frac{\cos x}{\sin^2x-36}\;dx.
\]
\end{problem}
\begin{proof}[Solution]
First make the substitution $u=\sin x$. Then from the table, we have the
formula
\begin{equation}
\label{eq:table-formula-20}
\int\frac{du}{u^2-a^2}=\frac{1}{2a}\ln\left|\frac{u-a}{u+a}\right|+C.
\end{equation}
So
\begin{align*}
\int\frac{\cos x}{\sin^2x-36}
&=\int\frac{du}{u^2-6^2}\;du\\
&=\frac{1}{12}\ln\left|\frac{u-6}{u+6}\right|+C\\
&=\boxed{\frac{1}{12}\ln\left|\frac{\sin x-6}{\sin x+6}\right|+C.}
\end{align*}
\end{proof}

\begin{problem}[WebAssign, HW 15, \# 2]
Use the table of integrals to evaluate the integral. (Remember to use
$\ln|u|$ where appropriate. Use $C$ for the constant of integration.)
\[
\int\frac{1}{x^2\sqrt{81x^2+4}}\;dx.
\]
\end{problem}
\begin{proof}[Solution]
First make the substitution $u=9x$. Then from the table, we have the formula
\begin{equation}
  \label{eq:table-formula-28}
\int\frac{du}{u^2\sqrt{a^2+u^2}}=-\frac{\sqrt{a^2+u^2}}{a^2u}+C.
\end{equation}
So
\begin{align*}
\int\frac{1}{x^2\sqrt{81x^2+4}}\;dx
&=\frac{1}{9}\int\frac{1}{\frac{1}{81}u^2\sqrt{u^2+2^2}}\;dx\\
&=9\int\frac{1}{u^2\sqrt{u^2+2^2}}\;du\\
&=-\frac{9\sqrt{u^2+4}}{4u}+C\\
&=\boxed{-\frac{\sqrt{81x^2+4}}{4x}+C.}
\end{align*}
\end{proof}

\begin{problem}[WebAssign, HW 15, \# 3]
Use the table of integrals to evaluate the integral. (Remember to use
$\ln|u|$ where appropriate. Use $C$ for the constant of integration.)
\[
\int\frac{\tan^3(6/z)}{z^2}\;dz.
\]
\end{problem}
\begin{proof}[Solution]
First make the substitution $u=6/z$. Then from the table, we have the
formula
\begin{equation}
  \label{eq:table-formula-69}
\int\tan^3 u\;du=\frac{1}{2}\tan^2 u+\ln|\cos u|+C.
\end{equation}
So
\begin{align*}
\int\frac{\tan^3(6/z)}{z^2}\;dz
&=-\frac{1}{6}\int\tan^3 u\;du\\
&=-\frac{1}{12}\tan^2 u-\frac{1}{6}\ln|\cos u|+C\\
&=\boxed{-\frac{1}{12}\tan^2(6/z)-\frac{1}{6}\ln|\cos(6/z)|+C.}
\end{align*}
\end{proof}

\begin{problem}[WebAssign, HW 15, \# 4]
Use the table of integrals to evaluate the integral. (Remember to use
$\ln|u|$ where appropriate. Use $C$ for the constant of integration.)
\[
\int\frac{e^{2x}}{13-e^{4x}}\;dx.
\]
\end{problem}
\begin{proof}[Solution]
Make the substitution $u=e^{2x}$. Then, from the table of integrals we have
the formula
\begin{equation}
  \label{eq:table-formula-19}
\int\frac{du}{a^2-u^2}=\frac{1}{2a}\ln\left|\frac{u+a}{u-a}\right|+C.
\end{equation}
So
\begin{align*}
\int\frac{e^{2x}}{13-e^{4x}}\;dx
&=\frac{1}{2}\int\frac{1}{13-u^2}\;du\\
&=\frac{1}{4\sqrt{13}}\ln\left|\frac{u+\sqrt{13}}{u-\sqrt{13}}\right|+C.\\
&=\boxed{\frac{1}{4\sqrt{13}}\ln\left|\frac{e^{2x}+\sqrt{13}}{e^{2x}-\sqrt{13}}\right|+C.}
\end{align*}
\end{proof}

\begin{problem}[WebAssign, HW 15, \# 5]
Use the trapezoidal rule, the midpoint rule, and Simpson's rule to
approximate the given integral with specified value $n$. (Round your anwser
to six decimal places).
\[
\int_1^42\sqrt{\ln}\;dx,\qquad n=6
\]
\end{problem}
\begin{proof}[Solution]
These problems take too long, just remember the formulas
\begin{equation}
\label{eq:trapezoidal-rule}
\int_a^b f(x)\;dx\approx(b-a)\left[\frac{f(a)+f(b)}{2}\right]
\end{equation}
for the trapezoidal rule,
\begin{equation}
  \label{eq:midpoint-rule}
\int_a^b\approx\frac{b-a}{N}\sum_{n=0}^{N-1}f\left(a+\tfrac{n(b-a)}{N}\right)
\end{equation}
for the midpoint rule and
\begin{equation}
  \label{eq:simpsons-rule}
\int_a^b f(x)\;dx\approx
\frac{b-a}{6}\left[f(a)+4f\left(\tfrac{a+b}{2}\right)+f(b)\right].
\end{equation}
for Simpson's rule.
\end{proof}

\begin{problem}[WebAssign, HW 15, \# 6]
Use the trapezoidal rule, the midpoint rule, and Simpson's rule to
approximate the given integral with specified value $n$. (Round your anwser
to six decimal places).
\[
\int_0^4 e^{2\sqrt{t}}\;dt,\qquad n=8.
\]
\end{problem}
\begin{proof}[Solution]
Look at my comment above.
\end{proof}

\section{Homework 16}
\begin{problem}[WebAssign, HW 16, \# 1]
Determine whether the integral is convergent or divergent.
\[
\int_{-\infty}^0\frac{dx}{4-7x}.
\]
\end{problem}
\begin{proof}[Solution]
Let's develop some strategies for attacking these problems. First of all,
we like to work with positive numbers whenever possible, so let's make the
substitution $u=-x$. This changes the bounds  from $(-\infty,0)$ to
$(\infty,0)$. Now, remember that
\begin{equation}
\label{eq:reverse-integration}
\int_a^b f(x)\;dx=-\int_b^a f(x)\;dx.
\end{equation}
Thus, our integral turns into
\[
\int_{-\infty}^0\frac{dx}{4-7x}=
-\int_0^\infty\frac{-du}{4+7u}=
\int_0^\infty\frac{du}{4+7u}.
\]
Finding the integral of this, we have
\[
\int_0^\infty\frac{du}{4+7u}=\left.\frac{1}{7}\ln|4+7u|\right|_0^\infty.
\]
Now, what happens as $u\to\infty$? The value of $\frac{1}{7}\ln|4+7u|$ gets
bigger and bigger so the integral diverges.
\end{proof}

\begin{problem}[WebAssign, HW 16, \# 2]
Determine whether the integral is convergent or divergent.
\[
\int_2^{\infty} e^{-9p}\;dp.
\]
\end{problem}
\begin{proof}[Solution]
Compute the integral
\[
\int_2^\infty
e^{-9p}\;dp=\left.-\frac{1}{9}e^{-9p}\right|_2^\infty=-\frac{1}{9}e^{-9p}+\frac{1}{9}e^{-18}.
\]
What happens as $p\to\infty$? The value of $-\frac{1}{9}e^{-9p}$ gets
closer and closer to $0$ so the integral converges and is equal to
\[
\boxed{\frac{1}{9e^{18}}.}
\]
\end{proof}

\begin{problem}[WebAssign, HW 16, \# 3]
Determine whether the integral is convergent or divergent.
\[
\int_{-\infty}^\infty 3xe^{-x^2}\;dx.
\]
\end{problem}
\begin{proof}[Solution]
Remember that if we have three points $a<c<b$ in the real line, we can
rewrite the integral of $f(x)$ as
\begin{equation}
  \label{eq:integral-midpoint}
\int_a^b f(x)\;dx=\int_a^cf(x)\;dx+\int_c^bf(x)\;dx.
\end{equation}
Rewrite the integral above as
\[
\int_{-\infty}^\infty 3xe^{-x^2}\;dx=
\int_{-\infty}^0 3xe^{-x^2}\;dx+\int_0^\infty 3e^{-x^2}\;dx
-\underbrace{\int_0^\infty 3ue^{-u^2}\;du}_{I_1}+\underbrace{\int_0^\infty
  3xe^{-x^2}\;dx}_{I_2}
\]
where $u=-x$. For the same reasons as the previous problem, $I_1$ and $I_2$
converge. Moreover, it is easy to see that $I_1=I_2$ so the integral
$\boxed{-I_1+I_2=0}$.
\end{proof}

\begin{problem}[WebAssign, HW 16, \# 4]
Determine whether the integral is convergent or divergent.
\[
\int_1^\infty 37\frac{e^{-\sqrt{x}}}{\sqrt{x}}\;dx.
\]
\end{problem}
\begin{proof}[Solution]
Compute the integral by using the substitution $u=\sqrt{x}$
\[
\int_1^\infty 37\frac{e^{-\sqrt{x}}}{\sqrt{x}}\;dx
=\int_1^\infty 74e^{-u}\;du
=\left.-74e^{-u}\right|_1^\infty
\]
which converges for similar reasons as problem 1 from this homework. Thus,
the integral is $\boxed{74/e}$.
\end{proof}

\begin{problem}[WebAssign, HW 16, \# 5]
Determine whether the integral is convergent or divergent.
\[
\int_{-\infty}^\infty 31\cos\pi t\;dt.
\]
\end{problem}
\begin{proof}[Solution]
\end{proof}

\begin{problem}[WebAssign, HW 16, \# 6]
Determine whether the integral is convergent or divergent.
\[
\int_2^\infty\frac{dv}{v^2+5v-6}.
\]
\end{problem}
\begin{proof}[Solution]
Factor and use partial fractions
\begin{align*}
\int_2^\infty\frac{dv}{v^2+5v-6}
&=\int_2^\infty\frac{dv}{(v+6)(v-1)}\\
&=\int_2^{\infty}\left(\frac{-\frac{1}{7}}{v+6}+\frac{\frac{1}{7}}{v-1}\right)\;dv\\
&=\left.-\frac{1}{7}\ln|v+6|+\frac{1}{7}\ln|v-1|\right|_2^\infty\\
&=\left.\frac{1}{7}\ln\left|\frac{v-1}{v+6}\right|\right|_2^\infty\\
&=-\frac{1}{7}\ln\left|\frac{1}{8}\right|\\
&=\boxed{\frac{1}{7}\ln 8.}
\end{align*}
\end{proof}

\begin{problem}[WebAssign, HW 16, \# 7]
Determine whether the integral is convergent or divergent.
\[
\int_1^\infty 25\frac{\ln x}{x}\;dx.
\]
\end{problem}
\begin{proof}[Solution]
Use the substitution $u=\ln x$ then rewrite the integral and compute
\begin{align*}
\int_1^\infty 25\frac{\ln x}{x}\;dx
&=25\int_0^\infty u\;du\\
&=25\left.\frac{1}{2}u^2\right|_0^\infty\\
&=\frac{25}{2}\left.u^2\right|_0^\infty
\end{align*}
which clearly diverges as $u\to\infty$ since $u^2$ keeps getting bigger and
bigger.
\end{proof}

\begin{problem}[WebAssign, HW 16, \# 8]
Determine whether the integral is convergent or divergent.
\[
\int_{-2}^3\frac{45}{x^4}\;dx.
\]
\end{problem}
\begin{proof}[Solution]
Rewrite the integral as
\[
\int_{-2}^3\frac{45}{x^4}\;dx=
\int_{-2}^0\frac{45}{x^4}\;dx+\int_0^3\frac{45}{x^4}\;dx=
\underbrace{45\int_{-2}^0\frac{dx}{x^4}}_{I_1}+\underbrace{45\int_3^0\frac{du}{u^4}}_{I_2}
\]
where we let $u=-x$. Now, computing $I_1$ and $I_2$ we have
\[
I_1=-\frac{45}{3}\left.x^{-3}\right|_{-2}^0.
\]
It is clear that as $x\to 0$, the integral grows closer and closer to
$-\infty$. The same is true of $I_2$ so the integral diverges.
\end{proof}

\begin{problem}[WebAssign, HW 16, \# 9]
Determine whether the integral is convergent or divergent.
\[
\int_0^9\frac{7}{\sqrt[3]{x-1}}\;dx.
\]
\end{problem}
\begin{proof}[Solution]
By straight calculation
\begin{align*}
\int_0^9\frac{7}{\sqrt[3]{x-1}}\;dx
&=7\int_0^0(x-1)^{-1/3}\;dx\\
&=\frac{21}{2}\left.\sqrt{x-1}\right|_0^9\\
&=\frac{21}{2}\left(4-0\right)\\
&=\boxed{48.}
\end{align*}
\end{proof}

\section{Homework 17}
\begin{problem}[WebAssign, HW 17, \# 1]
Find the exact length of the curve.
\[
y=2+2x^{3/2},\qquad 0\leq x\leq 1.
\]
\end{problem}
\begin{proof}[Solution]
Remember the arclength formula
\begin{equation}
  \label{eq:arclength}
L(a,b)=\int_a^b\sqrt{1+\frac{dy}{dx}^2}\;dx.
\end{equation}
So let's find $dy/dx$, $dy/dx=3x^{1/2}$ by the power rule so we have
\begin{align*}
L(0,1)&=\int_0^1\sqrt{1+\left(3x^{1/2}\right)^2}\;dx\\
&=\int_0^1\sqrt{1+9x}\;dx\\
&=\int_0^1(1+9x)^{1/2}\;dx\\
&=\frac{2}{3}\left.(1+9x)^{3/2}\right|_0^1\\
      &=\boxed{\frac{2}{3}\left(10\sqrt{10}+1\right).}
\end{align*}
\end{proof}

\begin{problem}[WebAssign, HW 17, \# 2]
Find the exact length of the curve.
\[
x=\frac{\sqrt{y}(y-3)}{3},\qquad 9\leq y\leq 25.
\]
\end{problem}
\begin{proof}[Solution]
By straight computation,
\[
\frac{dx}{dy}=
\frac{1}{3}\left(\tfrac{3}{2}y^{1/2}-\frac{3}{2}y^{-1/2}\right)=
\frac{1}{2}y^{1/2}-\frac{1}{2}y^{-1/2}.
\]
Thus, the archlength is
\begin{align*}
L(9,25)
&=\int_9^{25}\sqrt{1+\left(\frac{1}{2}y^{1/2}-\frac{1}{2}y^{-1/2}\right)^2}\;dy\\
&=\frac{1}{2}\int_9^{25}\left(y^{1/2}+y^{-1/2}\right)\;dy\\
&=\left.\frac{1}{3}y^{3/2}+y^{1/2}\right|_9^{25}\\
&=\boxed{\frac{104}{3}.}
\end{align*}
\end{proof}

\begin{problem}[WebAssign, HW 17, \# 3]
Find the exact length of the curve.
\[
y=\ln|\sec x|,\qquad 0\leq x\leq\frac{\pi}{3}.
\]
\end{problem}
\begin{proof}[Solution]
Compute
\[
\frac{dy}{dx}=\frac{\sec x\tan x}{\sec x}=\tan x.
\]
So
\begin{align*}
L(0,\pi/3)
&=\int_0^{\pi/3}\sqrt{1+\tan^2 x}\;dx\\
&=\int_0^{\pi/3}\sec x\;dx\\
&=\left.\ln|\sec x+\tan x|\right|_0^{\pi/3}\\
&=\boxed{\ln 3}.
\end{align*}
\end{proof}

\begin{problem}[WebAssign, HW 17, \# 4]
Find the exact length of the curve.
\[
y=\ln\left(1-x^2\right),\qquad 0\leq x\leq\frac{1}{3}.
\]
\end{problem}
\begin{proof}[Solution]
Find
\[
\frac{dy}{dx}=-\frac{2x}{1-x^2}.
\]
Then, by partial fractions etc., we have
\begin{align*}
L(0,1/3)
&=\int_0^{1/3}\sqrt{1+\frac{4x^2}{\left(1-x^2\right)^2}}\;dx\\
&=\int_0^{1/3}\frac{1+x^2}{1-x^2}\;dx\\
&=\int_0^{1/3}\left(-1+\frac{2}{1-x^2}\right)\\
&=\int_0^{1/3}-1+\frac{1}{1+x}+\frac{1}{1-x}\;dx\\
&=\left.-x+\ln\left|\frac{1+x}{1-x}\right|\right|_0^{1/3}\\
&=\boxed{\ln(2)-\frac{1}{3}.}
\end{align*}
\end{proof}

\begin{problem}[WebAssign, HW 17, \# 5]
Find the exact area of the surface obtained by rotating the curve about the
$x$-axis.
\[
y=x^3,\qquad 0\leq x\leq 3.
\]
\end{problem}
\begin{proof}[Solution]
Using the cylinder method, set $y=x^3$ to your length and the change in the
arc will be $\sqrt{1+(dy/dx)^2}\;dx=\sqrt{1+3x^2}$ so our surface area will be
\begin{align*}
S(0,3)
&=\int_0^32\pi x^3\sqrt{1+9x^4}\;dx\\
\intertext{make the substitution $u=1+9x^4$}
&=\frac{2\pi}{36}\int_1^{730}\sqrt{u}\;du\\
&=\frac{\pi}{18}\left.\frac{2}{3}u^{3/2}\right|_1^{730}\\
&=\boxed{\frac{\pi}{27}(730\sqrt{730}-1).}
\end{align*}
\end{proof}

\begin{problem}[WebAssign, HW 17, \# 6]
Find the exact area of the surface obtained by rotating the curve about the
$x$-axis.
\[
y=\sin\left(\frac{\pi x}{3}\right),\qquad 0\leq x\leq 3.
\]
\end{problem}
\begin{proof}[Solution]
Find
\[
\frac{dy}{dx}=\frac{\pi}{3}\cos\frac{\pi x}{3}.
\]
So
\begin{align*}
S(0,3)
&=2\pi\int_0^3y\sqrt{1+(dy/dx)^2}\;dx\\
&=2\pi\int_0^3\sin\left(\frac{\pi
  x}{3}\right)\sqrt{1+\frac{\pi^2}{9}\cos^2\left(\frac{\pi
  x}{3}\right)}\;dx\\
\intertext{make the substitution $u=(\pi/3)\cos(\pi x/3)$}
&=-\frac{18}{\pi}\int_{\pi/3}^{-\pi/3}\sqrt{1+u^2}\;du\\
&=\frac{18}{\pi}\int_{-\pi/3}^{\pi/3}\sqrt{1+u^2}\;du\\
&=\frac{36}{\pi}\int_0^{\pi/3}\sqrt{1+u^2}\;du\\
&=\frac{36}{\pi}\int_0^{\pi/3}\sqrt{1+u^2}\;du\\
\intertext{use a trig substitution}
&=\frac{36}{\pi}\left[\frac{u\sqrt{1+u^2}}{2}-\frac{1}{2}\ln\left(u+\sqrt{1+u^2}\right)\right]_0^{\pi/3}\\
&=\boxed{6\sqrt{1+\frac{\pi^2}{9}}+\frac{18}{\pi}\left(\frac{\pi}{3}+\sqrt{1+\frac{\pi^2}{9}}\right).}
\end{align*}
\end{proof}

\begin{problem}[WebAssign, HW 17, \# 7]
The given curve is rotated about the $y$-axis. Find the area of the
resulting surface.
\[
y=\sqrt[3]{x},\qquad 2\leq y\leq 4.
\]
\end{problem}
\begin{proof}[Solution]
Express $x$ in terms of $y$, $x=y^3$ and find
\[
\frac{dx}{dy}=3y^2.
\]
Then
\begin{align*}
S&=2\pi\int_2^4x\sqrt{1+(dx/dy)^2}\;dy\\
&=2\pi\int_2^4y^3\sqrt{1+9y^4}\;dy\\
&=\frac{2\pi}{36}\int_2^4(36y^3)\sqrt{1+9y^4}\;dy\\
\intertext{make the substitution $u=1+9y^4$, then}
&=\frac{\pi}{18}\int_{145}^{2305}\sqrt{1+u}\;du\\
&=\frac{\pi}{27}\left[u^{3/2}\right]_{145}^{2305}\\
&=\boxed{\frac{\pi}{2}\left(2305\sqrt{2305}-145\sqrt{145}\right).}
\end{align*}
\end{proof}

\begin{problem}[WebAssign, HW 17, \# 8]
The given curve is rotated about the $y$-axis. Find the area of the
resulting surface.
\[
  y=4-x^2,\qquad 0\leq x\leq 5.
\]
\end{problem}
\begin{proof}[Solution]
Fird
\[
\frac{dy}{dx}=2x.
\]
Then
\begin{align*}
S(0,5)
&=2\pi\int_0^5 x\sqrt{1+4x^2}\;dx\\
\intertext{make the substitution $u=1+4x^2$, then}
&=\frac{\pi}{4}\int_1^{101}\sqrt{u}\;du\\
&=\frac{3\pi}{8}\left[u^{3/2}\right]_1^{101}\\
&=\boxed{\frac{3\pi}{8}(101\sqrt{101}-1).}
\end{align*}
\end{proof}
\chapter{Exam II Problems}
Relevant exam problems
\begin{problem}[Spring 2014, \# 8]
Which of the following improper integrals converge?
\begin{enumerate}[label=\MakeUppercase{\roman*}.]
\item $\displaystyle\int_0^\infty xe^{-x^2}\;dx$
\item $\displaystyle\int_{-\infty}^\infty\frac{dx}{x}$
\item $\displaystyle\int_{-1}^1\frac{dx}{\sqrt[3]{x}}$.
\end{enumerate}
\end{problem}
\begin{proof}[Solution]
First, let's compute the integrals I, II and III. Here's I
\begin{align*}
I_1&=\int_0^\infty xe^{-x^2}\;dx\\
   &=\frac{1}{2}\int_0^\infty e^{-u}\;du\\
   &=\left[-\frac{1}{2}e^{-u}\right]_0^\infty\\
   &=\frac{1}{2}.
\end{align*}
Here's II
\begin{align*}
I_2&=\int_{-\infty}^\infty\frac{dx}{x}\\
&=\int_{-\infty}^0\frac{dx}{x}+\int_0^\infty\frac{dx}{x}\\
&=-\int_\infty^0\frac{du}{u}+\int_0^\infty\frac{dx}{x}\\
&=\int_0^\infty\frac{du}{u}+\int_0^\infty\frac{dx}{x}\\
&=\left[\ln u\right]_0^\infty+\left[\ln x\right]_0^\infty
\end{align*}
this clearly divereges since $\ln x\to\-\infty$ as $x\to 0$ and $ln
x\to\infty$ as $x\to\infty$. The same goes for $\ln u$. You can't
win. Here's III
\begin{align*}
I_3&=\int_{-1}^1\frac{dx}{\sqrt[3]{x}}\\
   &=\int_{-1}^1x^{1/3}\;dx\\
   &=\frac{3}{4}\left[x^{4/3}\right]_{-1}^1\\
   &=0.
\end{align*}
Hence, I and III converge, but III does not.
\end{proof}

\begin{problem}[Spring 2014, \# 9]
Find the exact length of the curve $y=\ln(\sec x)$, $0\leq x\leq\pi/3$.
\end{problem}
\begin{proof}[Solution]
First find the derivative with respect to $x$
\[
\frac{dy}{dx}=\tan x.
\]
Then
\begin{align*}
S(0,\pi/3)
&=\int_0^{\pi/3}\sqrt{1+\tan^2x}\;dx\\
&=\int_0^{\pi/3}\sec x\;dx\\
&=\left[\ln|\sec x+\tan x|\right]_0^{\pi/3}\\
&=\ln\left(2+\sqrt{3}\right)-\ln(1-0)\\
&=\boxed{\ln\left(2+\sqrt{3}\right).}
\end{align*}
\end{proof}

\begin{problem}[Spring 2015, \# 9]
If the upper part of the ellipse $y^2/4+x^2/16=1$ is revolved around the
$x$-axis to generate an ellipsoid $S$, then the surface area of $S$ is
given by?
\end{problem}
\begin{proof}[Solution]
Let's just consider the portion on that lies on the first quadrant and
double the result. Using the cylindrical method, we have to rewrite
\[
y=\sqrt{4-\frac{x^2}{4}}
\]
and
\[
\frac{dy}{dx}=\frac{x}{4\sqrt{4-\frac{x^2}{4}}}.
\]
\begin{align*}
S(0,1)
&=\int_0^1\sqrt{1+\left(\frac{x}{4\sqrt{4-\frac{x^2}{4}}}\right)^2}\;dx\\
&=\int_0^1\sqrt{1+\frac{x^2}{16(4-x^2/4)^2}}\;dx\\
&=\int_0^1\sqrt{1+\frac{x^2}{8(16-x^2)^2}}\;dx\\
&=\int_0^1\sqrt{\frac{8x^4-287x^4+2048}{8(16-x^2)^2}}\;dx\\
&=
\end{align*}
\end{proof}

%%% Local Variables:
%%% mode: latex
%%% TeX-master: "../MA166-Recitation"
%%% End:
