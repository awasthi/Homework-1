\chapter{Homework Problems}
\section{Homework 15}
\begin{problem}[WebAssign, HW 15, \# 1]
Use the table of integrals to evaluate the integral. (Remember to use
$\ln|u|$ where appropriate. Use $C$ for the constant of integration.)
\[
\int\frac{\cos x}{\sin^2x-36}\;dx.
\]
\end{problem}
\begin{proof}[Solution]
\end{proof}

\begin{problem}[WebAssign, HW 15, \# 2]
Use the table of integrals to evaluate the integral. (Remember to use
$\ln|u|$ where appropriate. Use $C$ for the constant of integration.)
\[
\int\frac{1}{x^2\sqrt{81x^2+4}}\;dx.
\]
\end{problem}
\begin{proof}[Solution]
\end{proof}

\begin{problem}[WebAssign, HW 15, \# 3]
Use the table of integrals to evaluate the integral. (Remember to use
$\ln|u|$ where appropriate. Use $C$ for the constant of integration.)
\[
\int\frac{\tan^3(6/z)}{z^2}\;dz.
\]
\end{problem}
\begin{proof}[Solution]
\end{proof}

\begin{problem}[WebAssign, HW 15, \# 4]
Use the table of integrals to evaluate the integral. (Remember to use
$\ln|u|$ where appropriate. Use $C$ for the constant of integration.)
\[
\int\frac{e^{2x}}{13-e^{4x}}\;dx.
\]
\end{problem}
\begin{proof}[Solution]
\end{proof}

\begin{problem}[WebAssign, HW 15, \# 5]
Use the trapezoidal rule, the midpoint rule, and Simpson's rule to
approximate the given integral with specified value $n$. (Round your anwser
to six decimal places).
\[
\int_1^42\sqrt{\ln}\;dx,\qquad n=6
\]
\end{problem}
\begin{proof}[Solution]
\end{proof}

\begin{problem}[WebAssign, HW 15, \# 6]
Use the trapezoidal rule, the midpoint rule, and Simpson's rule to
approximate the given integral with specified value $n$. (Round your anwser
to six decimal places).
\[
\int_0^4 e^{2\sqrt{t}}\;dt,\qquad n=8.
\]
\end{problem}
\begin{proof}[Solution]
\end{proof}

\section{Homework 16}
\begin{problem}[WebAssign, HW 16, \# 1]
Determine whether the integral is convergent or divergent.
\[
\int_{-\infty}^0\frac{1}{4-7x}\;dx.
\]
\end{problem}
\begin{proof}[Solution]
\end{proof}

\begin{problem}[WebAssign, HW 16, \# 2]
Determine whether the integral is convergent or divergent.
\[
\int_2^{\infty} e^{-9p}\;dp.
\]
\end{problem}
\begin{proof}[Solution]
\end{proof}

\begin{problem}[WebAssign, HW 16, \# 3]
Determine whether the integral is convergent or divergent.
\[
\int_{-\infty}^\infty 3xe^{-x^2}\;dx.
\]
\end{problem}
\begin{proof}[Solution]
\end{proof}

\begin{problem}[WebAssign, HW 16, \# 4]
Determine whether the integral is convergent or divergent.
\[
\int_1^\infty 37\frac{e^{-\sqrt{x}}}{\sqrt{x}}\;dx.
\]
\end{problem}
\begin{proof}[Solution]
\end{proof}

\begin{problem}[WebAssign, HW 16, \# 5]
Determine whether the integral is convergent or divergent.
\[
\int_{-\infty}^\infty 31\cos\pi t\;dt.
\]
\end{problem}
\begin{proof}[Solution]
\end{proof}

\begin{problem}[WebAssign, HW 16, \# 6]
Determine whether the integral is convergent or divergent.
\[
\int_2^\infty\frac{1}{v^2+5v-6}\;dv.
\]
\end{problem}
\begin{proof}[Solution]
\end{proof}

\begin{problem}[WebAssign, HW 16, \# 7]
Determine whether the integral is convergent or divergent.
\[
\int_1^\infty 25\frac{\ln x}{x}\;dx.
\]
\end{problem}
\begin{proof}[Solution]
\end{proof}

\begin{problem}[WebAssign, HW 16, \# 8]
Determine whether the integral is convergent or divergent.
\[
\int_{-2}^3\frac{45}{x^4}\;dx.
\]
\end{problem}
\begin{proof}[Solution]
\end{proof}

\begin{problem}[WebAssign, HW 16, \# 9]
Determine whether the integral is convergent or divergent.
\[
\int_0^9\frac{7}{\sqrt{x-1}}\;dx.
\]
\end{problem}
\begin{proof}[Solution]
\end{proof}

\section{Homework 17}
\begin{problem}[WebAssign, HW 17, \# 1]
Find the exact length of the curve.
\[
y=2+2x^{3/2},\qquad 0\leq x\leq 1.
\]
\end{problem}
\begin{proof}[Solution]
\end{proof}

\begin{problem}[WebAssign, HW 17, \# 2]
Find the exact length of the curve.
\[
x=\frac{\sqrt{y}(y-3)}{3},\qquad 9\leq y\leq 25.
\]
\end{problem}
\begin{proof}[Solution]
\end{proof}

\begin{problem}[WebAssign, HW 17, \# 3]
Find the exact length of the curve.
\[
y=\ln|\sec x|,\qquad 0\leq x\leq\frac{\pi}{3}.
\]
\end{problem}
\begin{proof}[Solution]
\end{proof}

\begin{problem}[WebAssign, HW 17, \# 4]
Find the exact length of the curve.
\[
y=\ln\left(1-x^2\right),\qquad 0\leq x\leq\frac{1}{3}.
\]
\end{problem}
\begin{proof}[Solution]
\end{proof}

\begin{problem}[WebAssign, HW 17, \# 5]
Find the exact area of the surface obtained by rotating the curve about the
$x$-axis.
\[
y=x^3,\qquad 0\leq x\leq 3.
\]
\end{problem}
\begin{proof}[Solution]
\end{proof}

\begin{problem}[WebAssign, HW 17, \# 6]
Find the exact area of the surface obtained by rotating the curve about the
$x$-axis.
\[
y=\sin\left(\frac{\pi x}{3}\right),\qquad 0\leq x\leq 3.
\]
\end{problem}
\begin{proof}[Solution]
\end{proof}

\begin{problem}[WebAssign, HW 17, \# 7]
The given curve is rotated about the $y$-axis. Find the area of the
resulting surface.
\[
y=\sqrt[3]{x},\qquad 2\leq y\leq 4.
\]
\end{problem}
\begin{proof}[Solution]
\end{proof}

\begin{problem}[WebAssign, HW 17, \# 8]
The given curve is rotated about the $y$-axis. Find the area of the
resulting surface.
\[
  y=4-x^2,\qquad 0\leq x\leq 5.
\]
\end{problem}
\begin{proof}[Solution]
\end{proof}

\chapter{Exam II Problems}

%%% Local Variables:
%%% mode: latex
%%% TeX-master: "../MA166-Recitation"
%%% End:
