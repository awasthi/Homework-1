\chapter{Solutions to this week's homework}
\section{Homework \# 22}
\begin{problem}[WebAssign HW \# 22, \# 1]
Determine whether the series is convergent or divergent.
\[
1+\frac{1}{4}+\frac{1}{9}+\frac{1}{16}+\frac{1}{25}+\cdots.
\]
\end{problem}
\begin{proof}[Solution]
The series is convergent. It is in fact this is a very important $p$-series
whose sum is equal to $\pi^2/6$.
\end{proof}

\begin{problem}[WebAssign HW \# 22, \# 2]
Determine whether the series is convergent or divergent.
\[
1+\frac{1}{2\sqrt{2}}+\frac{1}{4\sqrt{4}}+\frac{1}{5\sqrt{5}}+\cdots.
\]
\end{problem}
\begin{proof}[Solution]
This is again convergent because it is a $p$-series with form
\[
\sum_{n=0}^\infty\frac{1}{n^{3/2}}
\]
with $p=3/2>1$, so the series converges.
\end{proof}

\begin{problem}[WebAssign HW \# 22, \# 3]
Determine whether the series is convergent or divergent.
\[
\sum_{n=1}^\infty\frac{\sqrt{n}+5}{n^2}.
\]
\end{problem}
\begin{proof}[Solution]
Expand the series
\begin{align*}
\sum_{n=1}^\infty\frac{\sqrt{n}+5}{n^2}
&=\sum_{n=1}^\infty\left[\frac{\sqrt{n}}{n^2}+\frac{5}{n^2}\right]\\
&=\sum_{n=1}^\infty\left[\frac{1}{n^{3/2}}+\frac{5}{n^2}\right]\\
&=\underbrace{\sum_{n=1}^\infty\frac{1}{n^{3/2}}}_{S_1}
+\underbrace{\sum_{n=1}^\infty\frac{5}{n^2}}_{S_2}.
\end{align*}
The series $S_1$ is finite by the last problem since it is a $p$-series
with $p=3/2>1$ and the same goes for $S_2$. Thus, the series we were given
is convergent.
\end{proof}

\begin{problem}[WebAssign HW \# 22, \# 4]
Suppose $\sum a_n$ and $\sum b_n$ are series with positive terms and $\sum
b_n$ is known to be convergent.
\begin{enumerate}[label=(\alph*)]
\item If $a_n>b_n$ for all $n$, what can you say about $\sum a_n$? Why?
\item If $a_n<b_n$ for all $n$, what can you say about $\sum a_n$? Why?
\end{enumerate}
\end{problem}
\begin{proof}[Solution]
\end{proof}

\begin{problem}[WebAssign HW \# 22, \# 5]
Suppose $\sum a_n$ and $\sum b_n$ are series with positive terms and $\sum
b_n$ is known to be divergent.
\begin{enumerate}[label=(\alph*)]
\item If $a_n>b_n$ for all $n$, what can you say about $\sum a_n$? Why?
\item If $a_n<b_n$ for all $n$, what can you say about $\sum a_n$? Why?
\end{enumerate}
\end{problem}
\begin{proof}[Solution]
\end{proof}

\begin{problem}[WebAssign HW \# 22, \# 6]
Determine whether the series converges or diverges.
\[
\sum_{n=1}^\infty\frac{n}{5n^3+1}.
\]
\end{problem}
\begin{proof}[Solution]
Use the comparison test to determine whether the series converges
\begin{align*}
\frac{n}{5n^3+1}&<\frac{n}{5n^3}\\
                &=\frac{1}{5n^2}\\
                &<\frac{1}{n^2}.
\end{align*}
Hence, we have
\[
\sum_{n=1}^\infty\frac{n}{5n^3+1}<\sum_{n=1}^\infty\frac{1}{n^2}
\]
the right of which converges since it is a $p$-series with $p=2>1$.
\end{proof}

\begin{problem}[WebAssign HW \# 22, \# 7]
Determine whether the series converges or diverges.
\[
\sum_{n=1}^\infty\frac{n^4}{2n^5-1}.
\]
\end{problem}
\begin{proof}[Solution]
By the integral test we have
\begin{align*}
\int_1^\infty\frac{x^4}{2x^5-1}\diff x
&=\frac{1}{2}\int_1^\infty\frac{x^4}{x^5-1/2}\diff x
\intertext{make the $u$-substitution $u=x^5-1/2$, $\diff u=5x^4\diff x$ so}
&=\frac{1}{10}\int_{1/2}^\infty\frac{\diff u}{u}\\
&=\left[\frac{1}{10}\ln|u|\right]_{1/2}^\infty\\
&=\infty.
\end{align*}
\end{proof}

\begin{problem}[WebAssign HW \# 22, \# 8]
Determine whether the series converges or diverges.
\[
\sum_{n=1}^\infty\frac{n+6}{n\sqrt{n}}.
\]
\end{problem}
\begin{proof}[Solution]
Expand the series and simplify
\begin{align*}
\sum_{n=1}^\infty\frac{n+6}{n\sqrt{n}}
&=\sum_{n=1}^\infty\frac{n+6}{n^{3/2}}\\
&=\sum_{n=1}^\infty\left[\frac{n}{n^{3/2}}+\frac{6}{n^{3/2}}\right]\\
&=\sum_{n=1}^\infty\left[\frac{1}{n^{1/2}}+\frac{6}{n^{3/2}}\right]\\
&=\underbrace{\sum_{n=1}^\infty\frac{1}{n^{1/2}}}_{S_1}
+6\underbrace{\sum_{n=1}^\infty\frac{1}{n^{3/2}}}_{S_2}.
\end{align*}
The series $S_2$ is a convergent $p$-series with $p=2>1$, but $S_1$ is not
convergent since $p=1/2<1$. Hence, the sum of $S_1$ and $S_2$ is
divergent.
\end{proof}

\begin{problem}[WebAssign HW \# 22, \# 9]
Determine whether the series converges or diverges.
\[
\sum_{n=1}^\infty\frac{7^n}{1+11^n}.
\]
\end{problem}
\begin{proof}[Solution]
Apply the comparison test
\[
\frac{7^n}{1+11^n}<\frac{7^n}{11^n}=\left(\frac{7}{11}\right)^n.
\]
Then
\[
\sum_{n=1}^\infty\frac{7^n}{1+11^n}<\sum_{n=1}^\infty\left(\frac{7}{11}\right)^n
\]
which is a geometric series with radius $r=|7/11|<1$, hence convergent so
the original series is convergent.
\end{proof}

\section{Homework \# 23}
\begin{problem}[WebAssign HW \# 23, \# 1]
Determine whether the series converges or diverges.
\[
\sum_{n=9}^\infty\frac{3\sqrt{n}}{n-8}.
\]
\end{problem}
\begin{proof}[Solution]
By the comparison test
\begin{align*}
\frac{3\sqrt{n}}{n-8}&>\frac{3\sqrt{n}}{n}\\
                     &=\frac{3}{\sqrt{n}}
\end{align*}
so
\[
\sum_{n=1}^\infty\frac{3\sqrt{n}}{n-8}>\sum_{n=1}^\infty\frac{3}{\sqrt{n}}
\]
the right of which diverges since it is a $p$-series with $p=1/2<1$. Hence,
the original series diverges.
\end{proof}

\begin{problem}[WebAssign HW \# 23, \# 2]
Determine whether the series converges or diverges.
\[
\sum_{n=1}^\infty\frac{9}{\sqrt{n^2+2}}.
\]
\end{problem}
\begin{proof}[Solution]
By the comparison test
\begin{align*}
\frac{9}{\sqrt{n^2+2}}&>\frac{9}{\sqrt{n^2}}\\
                      &=\frac{9}{n}
\end{align*}
so
\[
\sum_{n=1}^\infty\frac{9}{\sqrt{n^2+2}}>\sum_{n=1}^\infty\frac{9}{n}
\]
where the series on the right diverges, so the original series diverges.
\end{proof}

\begin{problem}[WebAssign HW \# 23, \# 3]
Determine whether the series converges or diverges.
\[
\sum_{n=1}^\infty\frac{5+8^n}{5+7^n}.
\]
\end{problem}
\begin{proof}[Solution]
By the comparison test we have
\begin{align*}
\frac{5+8^n}{5+7^n}&>\frac{5+8^n}{7^n}\\
                   &=\frac{5}{7^n}+\left(\frac{8}{7}\right)^n.
\end{align*}
Then we have
\begin{align*}
\sum_{n=1}^\infty\frac{5+8^n}{5+7^n}
&>\sum_{n=1}^\infty\left[\frac{5}{7^n}+\left(\frac{8}{7}\right)^n\right]\\
&=\underbrace{\sum_{n=1}^\infty\frac{5}{7^n}}_{S_1}
+\underbrace{\sum_{n=1}^\infty\left(\frac{8}{7}\right)^n}_{S_2}
\end{align*}
where the series on the right is a geometric series with radius $r=|8/7|>1$
hence diverges. Thus, the sum $S_1+S_2$ diverges.
\end{proof}

\begin{problem}[WebAssign HW \# 23, \# 4]
Determine whether the series converges or diverges.
\[
\sum_{n=1}^\infty\frac{n+2^n}{n+8^n}.
\]
\end{problem}
\begin{proof}[Solution]
By the comparison test we have
\begin{align*}
\frac{n+2^n}{n+8^n}
&<\frac{n+2^n}{8^n}\\
&=\frac{n}{8^n}+\left(\frac{2}{8}\right)^n.
\end{align*}
Then we have
\begin{align*}
\sum_{i=1}^\infty\frac{n+2^n}{n+8^n}
&<\sum_{i=1}^\infty\left[\frac{n}{8^n}+\left(\frac{2}{8}\right)^n\right]\\
&=\underbrace{\sum_{i=1}^\infty\frac{n}{8^n}}_{S_1}+
  \underbrace{\sum_{i=1}^\infty\left(\frac{2}{8}\right)^n}_{S_2}.
\end{align*}
$S_2$ converges because it is a geometric series with radius
$r=|2/8|=1/4<1$. $S_1$ converges by the integral test
\begin{align*}
\int_1^\infty\frac{x}{8^x}\diff x&=\int_1^\infty\frac{x}{e^{x\ln 8}}\diff
                                   x
\intertext{by integration by parts}
                                 &=\left[-xe^{-x\ln 8}/\ln 8-e^{-x\ln
                                   8}/(\ln 8)^2\right]_1^\infty\\
                                 &=0-(-8/\ln 8-8/(\ln 8)^2)\\
                                 &<\infty
\end{align*}
(We don't care what the value is, just that it is not infinite.) Thus,
$S_1+S_2$ converges.at
\end{proof}

\begin{problem}[WebAssign HW \# 23, \# 5]
Determine whether the series converges or diverges.
\[
\sum_{n=1}^\infty 5\sin\left(\frac{3}{n}\right).
\]
\end{problem}
\begin{proof}[Solution]
Set $b_n\coloneqq 15/n$. By the limit comparison test, we have
\[
\lim_{n\to\infty}\frac{5\sin(3/n)}{15/n}=\lim_{n\to\infty}\frac{\sin(3/n)}{n/3}=\lim_{k\to
0}\frac{\sin k}{k}=1.
\]
The above limit is well known. The reason we make a change of variables
here is because we have $3/n\to 0$ as $n\to\infty$ so we may as well define
$k=3/n$ and look at the sequence of $k$ as they go to $0$. Since the series
\[
\sum_{i=1}^\infty\frac{15}{n}
\]
is harmonic, it diverges so the original series diverges.
\end{proof}

\section{Homework \# 24}
\begin{problem}[WebAssign HW \# 24, \# 1]
Test the series for convergence or divergence.
\[
\sum_{n=1}^\infty(-1)^{n-1}b_n=\frac{1}{\sqrt{4}}-\frac{1}{\sqrt{5}}+\frac{1}{\sqrt{6}}-\frac{1}{\sqrt{7}}+\frac{1}{\sqrt{8}}-\cdots.
\]
\end{problem}
\begin{proof}[Solution]
First we find $b_n$. Looking at the pattern we quickly see that
\[
b_n=\frac{1}{\sqrt{n+3}}.
\]
Then $b_n>0$ is a positive sequence with limit $0$ so by the alternating
series test, the series converges.
\end{proof}

\begin{problem}[WebAssign HW \# 24, \# 2]
Test the series for convergence or divergence.
\[
\sum_{n=1}^\infty\frac{(-1)^{n-1}}{5n+1}.
\]
\end{problem}
\begin{proof}[Solution]
The sequence
\[
b_n=\frac{1}{5n+1}
\]
is positive (always greater than or equal to $0$) and converges to $0$ so
by the alternating series test, the series converges.
\end{proof}

\begin{problem}[WebAssign HW \# 24, \# 3]
Test the series for convergence or divergence.
\[
\sum_{n=1}^\infty (-1)^n\frac{3n-1}{5n+1}.
\]
Identify $b_n$ and evaluate the limit $\lim_{n\to\infty}b_n$.
\end{problem}
\begin{proof}[Solution]
Identifying $b_n$ is the easiest part. It is the part of the series which
does not have the alternating power $(-1)^n$, i.e.,
$b_n=(3n-1)/(5n+1)$. Now to compute the limit you can use l'Hôpital's rule
on the functions $3x-1/5x+1$ to get $3/5$.

Since $\lim_{n\to\infty}b_n\neq 0$ and $b_{n+1}\geq b_n$ for all $n$, the
series diverges.
\end{proof}

\begin{problem}[WebAssign HW \# 24, \# 4]
Test the series for convergence or divergence.
\[
\sum_{n=1}^\infty\frac{(-1)^{n+1}}{3n^5}.
\]
\end{problem}
\begin{proof}[Solution]
The series converges since it satisfies
\[
b_n=\frac{1}{3n^5}\geq \frac{1}{3(n+1)^{5}}
\]
and
\[
\lim_{n\to\infty} b_n=0.
\]
\end{proof}

\begin{problem}[WebAssign HW \# 24, \#]
Approximate the sum of the series correct to four decimal places.
\[
\sum_{n=1}^\infty\frac{(-1)^{n-1}n^2}{7^n}.
\]
\end{problem}
\begin{proof}[Solution]
\end{proof}

\begin{problem}[WebAssign HW \# 24, \#]
\end{problem}
\begin{proof}[Solution]
\end{proof}

\chapter{Exam problems}

%%% Local Variables:
%%% mode: latex
%%% TeX-master: "../MA166-Recitation"
%%% End:
