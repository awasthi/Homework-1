\chapter{Homework Solutions}
\section{Homework 18}
\begin{problem}
The masses $m_i$ are located at the points $P_i$. Find the moments $M_x$
and $M_y$ and the center of mass of the system.
\begin{align*}
m_1=2,&&m_2=1,&&m_3=7;\\
P_1(2,-5),&&P_2(-3,1),&&P_3(3,5).
\end{align*}
\end{problem}
\begin{proof}[Solution]
The definitions for the moment of the system about the $y$-axis is
\begin{equation}
  \label{eq:moment-about-y}
M_y=\sum_{i=1}^n m_ix_i,
\end{equation}
and for the moment of the system about the $x$-axis is
\begin{equation}
  \label{eq:moment-about-x}
M_x=\sum_{i=1}^n m_iy_i.
\end{equation}

So all you need to do for this problem is to plug in the values
\[
M_y=m_1x_1+m_2x_2+m_3x_3=\boxed{-4,}
\]
and
\[
M_x=m_1y_1+m_2y_2+m_3y_3=\boxed{-2.}
\]
Then the total mass is $M=10$ so
\[
(\bar x,\bar y)=\boxed{\left(-\frac{2}{5},-\frac{1}{5}\right).}
\]
\end{proof}

\begin{problem}
Sketch the region bounded by the curves, and visually estimate the location
of the centroid.
\[
y=4x,\qquad y=0,\qquad x=1.
\]
\end{problem}
\begin{proof}[Solution]
The image you can find yourself. It's at the
\href{https://en.wikipedia.org/wiki/Centroid}{centroid} of the triangle
(assuming uniform distribution of mass) and there's a very simple formula
for finding the centroid of a triangle, from a purely geometric
perspective, it is
\begin{equation}
  \label{eq:centroid-of-triangle}
(\bar x,\bar y)=\frac{1}{3}(x_1+x_2+x_3,y_1+y_2+y_3)
\end{equation}
where $(x_1,y_1)$, $(x_2,y_2)$ and $(x_3,y_3)$ are the vertices of the
triangle. The vertices are very clearly $(0,0)$, $(1,0)$ and $(1,4)$ hence
\[
(\bar x,\bar y)=\boxed{\left(\frac{2}{3},\frac{4}{3}\right).}
\]
\end{proof}

\begin{problem}
Sketch the region bounded by the curves, and visually estimate the location
of the centroid. Find the exact coordinates of the centroid.
\[
y=e^x,\qquad
y=0,\qquad
x=5.
\]
Find the exact coordinates of the centroid.
\end{problem}
\begin{proof}[Solution]
I'll assume you can plot this on your own. Having me do it is asking for
too much this late at night :-). Now, recall the definition of the moments
about the axes
\begin{equation}
\label{eq:moment-about-x-continuous}
M_y=\int_a^b x(f(x)-g(x))\;dx
\end{equation}
and
\begin{equation}
\label{eq:moment-about-y-continuous}
M_x=\int \frac{(f(x)-g(x))^2}{2}\;dx
\end{equation}
and of course the formula for the centroid
\begin{equation}
  \label{eq:center-of-mass}
(\bar x,\bar y)=\left(\frac{M_y}{A},\frac{M_x}{A}\right).
\end{equation}
Now the first thing we need to do is to calculate the area
\[
A=\int_0^5 e^x\;dx=e^5-1.
\]
Next, we calculate $M_y$ and $M_x$ like so
\begin{align*}
M_x&=\int_0^5 xe^x\;dx&M_y&=\int_0^5\frac{e^{2x}}{2}\;dx\\
&=\left[xe^x-e^x\right]_0^5&&=\frac{1}{4}\left[e^{2x}\right]_0^5\\
&=4e^5+1&&=\frac{e^{10}-1}{4}.
\end{align*}
So the centroid is
\[
\boxed{
(\bar x,\bar y)
=\left(\frac{1+4e^5}{e^5-1},\frac{e^5+1}{4}\right)
.
}
\]
\end{proof}

\begin{problem}
Find the centroid of the region bounded by the given curves.
\[
y=6\sin 5x,\qquad
y=6\cos 5x,\qquad
x=0,\qquad
x=\frac{\pi}{20}.
\]
\end{problem}
\begin{proof}[Solution]
What a horrible calculation. Spare my poor fingers having to type this out
in details :\textasciicircum).

The area is
\begin{align*}
A&=6\int_0^{\pi/12}\cos 3x-\sin 3x\;dx\\
 &=2\left[ \sin 3x+\cos 3x \right]_0^{\pi/12}\\
 &=\boxed{2(\sqrt{2}-1).}
\end{align*}
Skipping straight to the centroid, we have the following
\begin{align*}
\bar x&=\frac{1}{2(\sqrt{2}-1)}\int_0^{\pi/12}x\cos 3x-x\sin 3x\;dx&
\bar y&=\frac{1}{4(\sqrt{2}-1)}\int_0^{\pi/12}\cos^2 3x-\sin^2 3x\;dx\\
&=
\end{align*}
\end{proof}

\begin{problem}
Find the centroid of the region bounded by the given curves.
\[
y=x^3,\qquad
x+y=10,\qquad
y=0.
\]
\end{problem}
\begin{proof}[Solution]
\end{proof}

\begin{problem}
Calculate the moments $M_x$, $M_y$ and the center of mass of the lamina
with the given density and shape.
\end{problem}
\begin{proof}[Solution]
\end{proof}

\begin{problem}
\end{problem}
\begin{proof}[Solution]
\end{proof}


\section{Homework 19}
\begin{problem}
\end{problem}
\begin{proof}[Solution]
\end{proof}

\begin{problem}
\end{problem}
\begin{proof}[Solution]
\end{proof}

\begin{problem}
\end{problem}
\begin{proof}[Solution]
\end{proof}

\begin{problem}
\end{problem}
\begin{proof}[Solution]
\end{proof}

\begin{problem}
\end{problem}
\begin{proof}[Solution]
\end{proof}

\begin{problem}
\end{problem}
\begin{proof}[Solution]
\end{proof}

\begin{problem}
\end{problem}
\begin{proof}[Solution]
\end{proof}

\begin{problem}
\end{problem}
\begin{proof}[Solution]
\end{proof}

\begin{problem}
\end{problem}
\begin{proof}[Solution]
\end{proof}

\begin{problem}
\end{problem}
\begin{proof}[Solution]
\end{proof}

\begin{problem}
\end{problem}
\begin{proof}[Solution]
\end{proof}


\section{Homework 20}
\begin{problem}
\end{problem}
\begin{proof}[Solution]
\end{proof}

\begin{problem}
\end{problem}
\begin{proof}[Solution]
\end{proof}

\begin{problem}
\end{problem}
\begin{proof}[Solution]
\end{proof}

\begin{problem}
\end{problem}
\begin{proof}[Solution]
\end{proof}

\begin{problem}
\end{problem}
\begin{proof}[Solution]
\end{proof}

\begin{problem}
\end{problem}
\begin{proof}[Solution]
\end{proof}

\begin{problem}
\end{problem}
\begin{proof}[Solution]
\end{proof}

\begin{problem}
\end{problem}
\begin{proof}[Solution]
\end{proof}

\begin{problem}
\end{problem}
\begin{proof}[Solution]
\end{proof}

\chapter{Past Exam Problems}
\begin{problem}
\end{problem}
\begin{proof}[Solution]
\end{proof}

%%% Local Variables:
%%% TeX-master: "../MA166-Recitation"
%%% End:
