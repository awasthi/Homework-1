\chapter{Homework Solutions}
\section{Homework 18}
\begin{problem}
The masses $m_i$ are located at the points $P_i$. Find the moments $M_x$
and $M_y$ and the center of mass of the system.
\begin{align*}
m_1=2,&&m_2=1,&&m_3=7;\\
P_1(2,-5),&&P_2(-3,1),&&P_3(3,5).
\end{align*}
\end{problem}
\begin{proof}[Solution]
The definitions for the moment of the system about the $y$-axis is
\begin{equation}
  \label{eq:moment-about-y}
M_y=\sum_{i=1}^n m_ix_i,
\end{equation}
and for the moment of the system about the $x$-axis is
\begin{equation}
  \label{eq:moment-about-x}
M_x=\sum_{i=1}^n m_iy_i.
\end{equation}

So all you need to do for this problem is to plug in the values
\[
M_y=m_1x_1+m_2x_2+m_3x_3=\boxed{-4,}
\]
and
\[
M_x=m_1y_1+m_2y_2+m_3y_3=\boxed{-2.}
\]
Then the total mass is $M=10$ so
\[
(\bar x,\bar y)=\boxed{\left(-\frac{2}{5},-\frac{1}{5}\right).}
\]
\end{proof}

\begin{problem}
Sketch the region bounded by the curves, and visually estimate the location
of the centroid.
\[
y=4x,\qquad y=0,\qquad x=1.
\]
\end{problem}
\begin{proof}[Solution]
The image you can find yourself. It's at the
\href{https://en.wikipedia.org/wiki/Centroid}{centroid} of the triangle
(assuming uniform distribution of mass) and there's a very simple formula
for finding the centroid of a triangle, from a purely geometric
perspective, it is
\begin{equation}
  \label{eq:centroid-of-triangle}
(\bar x,\bar y)=\frac{1}{3}(x_1+x_2+x_3,y_1+y_2+y_3)
\end{equation}
where $(x_1,y_1)$, $(x_2,y_2)$ and $(x_3,y_3)$ are the vertices of the
triangle. The vertices are very clearly $(0,0)$, $(1,0)$ and $(1,4)$ hence
\[
(\bar x,\bar y)=\boxed{\left(\frac{2}{3},\frac{4}{3}\right).}
\]
\end{proof}

\begin{problem}
Sketch the region bounded by the curves, and visually estimate the location
of the centroid. Find the exact coordinates of the centroid.
\[
y=e^x,\qquad
y=0,\qquad
x=5.
\]
Find the exact coordinates of the centroid.
\end{problem}
\begin{proof}[Solution]
I'll assume you can plot this on your own. Having me do it is asking for
too much this late at night :-). Now, recall the definition of the moments
about the axes
\begin{equation}
\label{eq:moment-about-x-continuous}
M_y=\int_a^b x(f(x)-g(x))\;dx
\end{equation}
and
\begin{equation}
\label{eq:moment-about-y-continuous}
M_x=\int \frac{(f(x)-g(x))^2}{2}\;dx
\end{equation}
and of course the formula for the centroid
\begin{equation}
  \label{eq:center-of-mass}
(\bar x,\bar y)=\left(\frac{M_y}{A},\frac{M_x}{A}\right).
\end{equation}
Now the first thing we need to do is to calculate the area
\[
A=\int_0^5 e^x\;dx=e^5-1.
\]
Next, we calculate $M_y$ and $M_x$ like so
\begin{align*}
M_x&=\int_0^5 xe^x\;dx&M_y&=\int_0^5\frac{e^{2x}}{2}\;dx\\
&=\left[xe^x-e^x\right]_0^5&&=\frac{1}{4}\left[e^{2x}\right]_0^5\\
&=4e^5+1&&=\frac{e^{10}-1}{4}.
\end{align*}
So the centroid is
\[
\boxed{
(\bar x,\bar y)
=\left(\frac{1+4e^5}{e^5-1},\frac{e^5+1}{4}\right)
.
}
\]
\end{proof}

\begin{problem}
Find the centroid of the region bounded by the given curves.
\[
y=6\sin 5x,\qquad
y=6\cos 5x,\qquad
x=0,\qquad
x=\frac{\pi}{20}.
\]
\end{problem}
\begin{proof}[Solution]
What a horrible calculation. Spare my poor fingers having to type this out
in details :\textasciicircum).

The area is
\begin{align*}
A&=6\int_0^{\pi/12}\cos 3x-\sin 3x\;dx\\
 &=2\left[ \sin 3x+\cos 3x \right]_0^{\pi/12}\\
 &=\boxed{2(\sqrt{2}-1).}
\end{align*}
Skipping straight to the centroid, we have the following
\begin{align*}
\bar x&=\frac{1}{2(\sqrt{2}-1)}\int_0^{\pi/12}x\cos 3x-x\sin 3x\;dx&
\bar y&=\frac{1}{4(\sqrt{2}-1)}\int_0^{\pi/12}\cos^2 3x-\sin^2 3x\;dx\\
&=\frac{3}{\sqrt{2}-1}\int_0^{\pi/12} x\cos 3x-x\sin 3x\;dx&
&=\frac{9}{\sqrt{2}-1}\int_0^{\pi/12}\cos^23x-\sin^23x\;dx\\
&=\frac{3}{\sqrt{2}-1}\int_0^{\pi/12}\left[\frac{x\sin 3x+x\cos
  3x}{3}\right.\\
&\phantom{{}=\frac{3}{\sqrt{2}-1}\int_0^{\pi/12}{}}\left.+\frac{\cos 3x-\sin 3x}{9}\right]_0^{\pi/12}&
&=\frac{3}{2(\sqrt{2}-1)}\left[ \sin 6x \right]_0^{\pi/12}\\
&=\frac{\pi\sqrt{2}-4}{12(\sqrt{2}-1)}&
&=\frac{3}{2(\sqrt{2}-1)}.
\end{align*}
So the answer is
\[
\boxed{
(\bar x,\bar y)=
\left(
\frac{\pi\sqrt{2}-4}{12(\sqrt{2}-1)},
\frac{3}{2(\sqrt{2}-1)}.
\right)
.}
\]
\end{proof}

\begin{problem}
Find the centroid of the region bounded by the given curves.
\[
y=x^3,\qquad
x+y=30,\qquad
y=0.
\]
\end{problem}
\begin{proof}[Solution]
The curves are simple enough to sketch. We compute the area
\begin{align*}
  A&=\int_0^3 x^3\;dx+\int_3^{30}(30-x)\;dx\\
   &=\frac{1539}{4}.
\end{align*}
Next we compute directly $\bar x$ and $\bar y$
\begin{align*}
\bar x&=\frac{2}{1539}\int_0^{27}\left((30-y)^2-y^{2/3}\right)^2\;dx&
\bar y&=\frac{4}{1539}\int_0^{27}y\left(30-y-y^{1/3}\right)\;dy\\
&=\frac{2}{1539}\int_0^{27}900-60y+y^2-y^{2/3}\;dy&
&=\frac{4}{1539}\int_0^{27}30y-y^2-y^{4/3}\;dy\\
&=\frac{2}{1539}\left[900y-30y^2+\tfrac{1}{3}y^3-\tfrac{3}{5}y^{5/3}\right]_0^{27}&
&=\frac{4}{1539}\left[15y^2-\tfrac{1}{3}y^3-\tfrac{3}{7}y^{7/3}\right]_0^{27}\\
&=\frac{1092}{95}&&=\frac{1188}{133}.
\end{align*}
So the answer is
\[
\boxed{(\bar x,\bar y)=
\left(
\frac{1092}{95},\frac{1188}{133}
\right).
}
\]
\end{proof}

\begin{problem}
Calculate the moments $M_x$, $M_y$ and the center of mass of the lamina
with the given density and shape. $\rho=3$.
\end{problem}
\begin{proof}[Solution]
Just a silly calculation. Using plain old geometry, we can compute the area
of the region by inspection $A=\frac{1}{4}\pi^2+\frac{1}{2}$. We still have
to parameterize the quarter-circle and find the equation for the
line. These are
\[
f(x)=\sqrt{1-x^2}\qquad\text{and}\qquad g(x)=x-1.
\]
Hence
\begin{align*}
M_x&=\frac{3}{2}\int_0^1\left(\sqrt{1-x^2}\right)^2-(x-1)^2\;dx&
M_y&=3\int_0^1x\sqrt{1-x^2}-x(x-1)\;dx\\
   &=\frac{3}{2}\int_0^1(1-x^2)-(x^2-2x+1)\;dx\\
   &=\frac{3}{2}\int_0^1-2x^2+2x\;dx\\
   &=\frac{3}{2}\left[-\tfrac{2}{3}x^3+x^2\right]_0^1&
   &=3\left[-\tfrac{1}{3}\left(1-x^3\right)^{3/2}-\tfrac{1}{3}x^3+\tfrac{1}{2}x^2\right]_0^1\\
   &=\frac{1}{2}&
   &=\frac{3}{2}.
\end{align*}
Then
\[
\boxed{
(\bar x,\bar y)=
\left(\frac{2}{\pi+2},\frac{2}{3\pi+6}\right).
}
\]
\end{proof}

\begin{problem}
\end{problem}
\begin{proof}[Solution]
\end{proof}


\section{Homework 19}
\begin{problem}[HW 19, \# 1]
List the first five terms of a sequence
\[
a_n=\frac{(-1)^{n-1}}{6^n}.
\]
\end{problem}
\begin{proof}[Solution]
Just plug in the values $n=1,2,3,4,5$ into the equation.
\end{proof}

\begin{problem}[HW 19, \# 2]
List the first five terms of the sequence
\[
a_1=4,\qquad a_{n+1}=5a_n-1
\]
\end{problem}
\begin{proof}[Solution]
The sequence is recursive and depends on the value of the previous terms.
\end{proof}

\begin{problem}[HW 19, \# 3]
Find a formula for the general term $a_n$ of the sequence, assuming that
the pattern of the first few terms continues. (Assume that $n$ begins with
$1$).
\end{problem}
\begin{proof}[Solution]
The denominator of the $n$th is the $n$th odd integer; odd integers are not
divisible by $2$ so odd integers have the form $2n-1$. Hence,
$\boxed{a_n=1/(2n-1)}$.
\end{proof}

\begin{problem}[HW 19, \# 4]
Determine whether the sequence converges or diverges. If it converges, find
the limit.
\[
a_n=2-(0.3)^n.
\]
\end{problem}
\begin{proof}[Solution]
Since the $2$ part of $a_n$ is constant, we may ignore it for the
moment. What happens to $0.3^n$ as $n\to\infty$? The sequence is geometric,
i.e., of the form $r^n$ and we have that the limit exits if $-1<r<1$.
\end{proof}

\begin{problem}[HW 19, \# 5]
Determine whether the sequence converges or diverges. If it converges, find
the limit.
\[
a_n=\frac{n^3}{5n^3+1}
\]
\end{problem}
\begin{proof}[Solution]
Do the following
\begin{align*}
a_n&=\frac{n^3}{5n^3+1}\\
   &=\frac{n^3/n^3}{(5n^5+1)/n^3}\\
   &=\frac{1}{5+\frac{1}{n^3}}
\end{align*}
where $5+\frac{1}{n^3}\to 5+0$ as $n\to\infty$ so the limit is $\boxed{1/5}$.
\end{proof}

\begin{problem}[HW 19, \# 7]
Determine whether the sequence converges or diverges. If it converges, fin
the limit.
\end{problem}
\begin{proof}[Solution]
As $n\to\infty$, $8n\to 0$ so $\lim_{n\to\infty}a_n=\cos(0)=1$.
\end{proof}

\begin{problem}[HW 19, \# 8]
Determine whether the sequence converges or diverges. If it converges, find
the limit.
\[
a_n=\frac{(8n-1)!}{(8n+1)!}.
\]
\end{problem}
\begin{proof}[Solution]
By the definition of the factorial we have $n!=n(n-1)!$, henec
\begin{align*}
a_n&=\frac{(8n-1)!}{(8n+1)!}\\
   &=\frac{(8n-1)!}{(8n+1)8n(8n-1)!}\\
   &=\frac{1}{(8n+1)8n}\\
   &=\frac{1}{64n^2+8n}
\end{align*}
which clearly goes to $0$ as $n\to\infty$.
\end{proof}

\begin{problem}[HW 19, \# 9]
Determine whether the sequence converges or diverges. If it converges, find
the limit.
\[
a_n=n^2e^{-3n}.
\]
\end{problem}
\begin{proof}[Solution]
Write
\begin{align*}
\lim_{n\to\infty }a_n&=\lim_{n\to\infty} n^2e^{-3n}\\
   &=\lim_{n\to\infty}\frac{n}{e^{3n}}\\
\shortintertext{by L'Hôpital's rule twice}
   &=\lim_{n\to\infty}\frac{2x}{3e^{3x}}\\
   &=\lim_{n\to\infty}\frac{2}{9e^{3x}}\\
   &=0.
\end{align*}
\end{proof}

\begin{problem}[HW 19, \# 10]
Determine whether the sequence converges or diverges. If it converges, find
the limit.
\[
a_n=\tfrac{n}{4}\sin\left(\tfrac{4}{n}\right).
\]
\end{problem}
\begin{proof}[Solution]
Let $m=4/n$ and rewrite
\begin{align*}
a_n&=\tfrac{n}{4}\sin\left(4/n\right)\\
   &=\frac{\sin(4/n)}{4/n}\\
   &=\frac{\sin m}{m}.
\end{align*}
Now as $n\to\infty$, $m\to 0$ so
\[
  \lim_{n\to\infty}\tfrac{1}{4}\sin(4/n)=1
\]
by some theorem in the book.
\end{proof}

\begin{problem}[HW 19, \# 11]
Determine whether the sequence converges or diverges. If it converges, find
the limit.
\[
a_n=\frac{\sin 3n}{1+\sqrt{n}}.
\]
\end{problem}
\begin{proof}[Solution]
Since $\sin$ is periodic, $-1\leq \sin 3n\leq 1$ so by the squeeze theorem we
have
\[
\frac{-1}{1+\sqrt{n}}\leq\frac{\sin 3n}{1+\sqrt{n}}\leq\frac{1}{1+\sqrt{n}}.
\]
Letting $n\to\infty$, we see that the limit of $\sin 3n/(1+\sqrt{n})$ is
$0$.
\end{proof}

\section{Homework 20}
\begin{problem}[HW 20, \# 1]
Determine whether the sequence is increasing, decreasing, or monotonic.
\[
a_n=\frac{3n-7}{7n+3}.
\]
\end{problem}
\begin{proof}[Solution]
Clearly increasing and bounded, just check that $a_n$ never exceeds say
$5$. You can check that this is increasing by replacing $n$ by $x$ and
taking the derivative
\[
\frac{d}{dx}\left(\frac{3n-7}{7n+3}\right)=
\frac{3(7x+3)-7(3x-7)}{(7x+3)^2}=\frac{58}{(7x+3)^2}>0.
\]
\end{proof}

\begin{problem}[HW 20, \# 2]
Determine whether the sequence is increasing, decreasing, or monotonic.
\[
a_n=6ne^{-5n}.
\]
\end{problem}
\begin{proof}[Solution]
The function is decreasing since $e^{-5n}$ is decreasing. You can check by
the derivative test. Moreover the sequence is bounded by $\frac{6}{e^5}$
above and $0$ below.
\end{proof}

\begin{problem}[HW 20, \# 3]
Determine whether the sequence is increasing, decreasing, or monotonic.
\[
a_n=\frac{n}{5n^2+3}.
\]
\end{problem}
\begin{proof}[Solution]
The sequence is clearly decreasing. Take the derivative and check. It is
bounded since $0<a_n\leq 1/8$.
\end{proof}

\begin{problem}[HW 20, \# 4]
\begin{enumerate}[label=(\alph*),noitemsep]
\item What is the difference between a sequence and a series?
\item What is a convergent series? What is a divergent series?
\end{enumerate}
\end{problem}
\begin{proof}[Solution]
(a) A sequence is an ordered list of numbers whereas a series is the sum of
a list of numbers.
\\\\
(b) A series is convergent if the sequence of partial sums is a convergent
sequence. A series is divergent if it is not convergent.
\end{proof}

\begin{problem}[HW 20, \# 5]
Determine whether the series is increasing, decreasing, or monotonic.
\[
\left(7-9+\frac{81}{7}-\frac{729}{49}+\cdots\right).
\]
\end{problem}
\begin{proof}[Solution]
The series can be written
\[
7\sum_{n=1}^\infty\left(-\frac{9}{7}\right)^{n-1}.
\]
This is a geometric series with $|9/7|>1$ so it diverges.
\end{proof}

\begin{problem}[HW 20, \# 6]
Determine whether the series is increasing, decreasing, or monotonic.
\[
\sum_{n=1}^\infty 6\left(\frac{1}{2}\right)^{n-1}.
\]
\end{problem}
\begin{proof}[Solution]
This is yet another geometric series. Note that $|1/2|<1$ so this series
converges. Remember the formula
\begin{equation}
  \label{eq:geometric-series}
\frac{a}{1-r}
\end{equation}
for the convergence of a geometric series. Plugging in $a=6$ and $r=1/2$
into this equation, we get $12$.
\end{proof}

\begin{problem}[HW 20, \# 7]
Determine whether the series is increasing, decreasing, or monotonic.
\[
\sum_{n=1}^\infty \frac{6^n}{(-2)^{n-1}}.
\]
\end{problem}
\begin{proof}[Solution]
Rewrite the sum as
\begin{align*}
\sum_{n=1}^\infty \frac{6^n}{(-2)^{n-1}}
&=6\sum_{n=1}^\infty \frac{6^{n-1}}{(-2)^{n-1}}\\
&=6\sum_{n=1}^\infty \left(-\frac{6}{2}\right)^{n-1}
&=6\sum_{n=1}^\infty \left(-3\right)^{n-1}.
\end{align*}
This clearly does not converge since $|-3|>1$.
\end{proof}

\begin{problem}[HW 20, \# 8]
Determine whether the series is increasing, decreasing, or monotonic.
\[
\sum_{n=1}^\infty\frac{(-7)^{n-1}}{8^n}.
\]
\end{problem}
\begin{proof}[Solution]
Rewrite the sequence as
\begin{align*}
\sum_{n=1}^\infty\frac{(-7)^{n-1}}{8^n}
&=\frac{1}{8}\sum_{n=1}^\infty\left(-\frac{7}{8}\right)^{n-1}
\end{align*}
this converges since $|-7/8|<1$ and by the formula
\eqref{eq:geometric-series} it converges to $1/15$.
\end{proof}

\begin{problem}[HW 20, \# 9]
Determine whether the series is increasing, decreasing, or monotonic.
\[
\sum_{n=0}^\infty\frac{1}{\left(\sqrt{14}\right)^n}.
\]
\end{problem}
\begin{proof}[Solution]
Rewrite it as
\[
\sum_{n=0}^\infty\left(\frac{1}{\sqrt{14}}\right)^n.
\]
You can see that since $\left|1/\sqrt{14}\right|<1$, the series converges
and by \eqref{eq:geometric-series} it converges to
$\left(14+\sqrt{14}\right)/13$.
\end{proof}

\chapter{Past Exam Problems}
\begin{problem}
\end{problem}
\begin{proof}[Solution]
\end{proof}

%%% Local Variables:
%%% TeX-master: "../MA166-Recitation"
%%% End:
