\def\documentauthor{Carlos Salinas}
\def\documenttitle{MA 598 PG: Homework \hwnum}
\def\hwnum{1}
\def\shorttitle{MA 598 PG HW \hwnum}
\def\coursename{MA598-PG}
\def\documentsubject{profinite groups}
\def\authoremail{salinac@purdue.edu}

\documentclass[article,oneside,11pt]{memoir}
\usepackage{geometry}
\usepackage[dvipsnames]{xcolor}
\usepackage[
    breaklinks,
    bookmarks=true,
    colorlinks=true,
    pageanchor=false,
    linkcolor=black,
    anchorcolor=black,
    citecolor=black,
    filecolor=black,
    menucolor=black,
    runcolor=black,
    urlcolor=black,
    hyperindex=false,
    hyperfootnotes=true,
    pdftitle={\shorttitle},
    pdfauthor={\documentauthor},
    pdfkeywords={\documentsubject},
    pdfsubject={\coursename}
    ]{hyperref}

% Use symbols instead of numbers
\renewcommand*{\thefootnote}{\fnsymbol{footnote}}

%% Math
\usepackage{amsmath}
\usepackage{amsthm}
\usepackage{amssymb}
\usepackage{mathtools}

%%PDFTeX specific
\usepackage[mathcal]{euscript}
\usepackage{mathrsfs}
\usepackage{dsfont}
\usepackage{wasysym}

\usepackage[LAE,LFE,T2A,T1]{fontenc}
\usepackage[utf8]{inputenc}
\usepackage[farsi,french,german,spanish,dutch,russian,swedish,english]{babel}
\babeltags{pa=farsi,
           fr=french,
           de=german,
           es=spanish,
           nl=dutch,
           ru=russian,
           sv=swedish,
           en=english}
\def\spanishoptions{mexico}

\selectlanguage{english}

\newcommand{\textfa}[1]{\beginR\textpa{#1}\endR}

\usepackage{cmap}
\usepackage{CJKutf8}
\newcommand{\textha}[1]{\begin{CJK}{UTF8}{mj}#1\end{CJK}}
\newcommand{\textni}[1]{\begin{CJK}{UTF8}{min}#1\end{CJK}}
\newcommand{\textzh}[1]{\begin{CJK}{UTF8}{bsmi}#1\end{CJK}}

%% Misc
\usepackage{graphicx}
\graphicspath{{figures/}}

\usepackage{microtype}
\usepackage{multicol}
\usepackage[inline]{enumitem}
\usepackage{listings}
\usepackage{mleftright}
\mleftright

%% Theorems and definitions
\theoremstyle{plain}
\newtheorem{theorem}{Theorem}
\newtheorem{proposition}[theorem]{Proposition}
\newtheorem{corollary}[theorem]{Corollary}
\newtheorem{claim}[theorem]{Claim}
\newtheorem{lemma}[theorem]{Lemma}
\newtheorem{axiom}[theorem]{Axiom}

\newtheorem*{corollary*}{Corollary}
\newtheorem*{claim*}{Claim}
\newtheorem*{lemma*}{Lemma}
\newtheorem*{proposition*}{Proposition}
\newtheorem*{theorem*}{Theorem}

\theoremstyle{definition}
\newtheorem{definition}{Definition}
\newtheorem{example}{Examples}
\newtheorem{examples}[example]{Examples}
% \newtheorem{exercise}{Exercise}[section]
% \newtheorem{problem}[exercise]{Problem}

\newcounter{problem}
\newenvironment{problem}[1][]% environment name
{% begin code
  \stepcounter{problem}
  \par\vspace{\baselineskip}\noindent
  \ifx &#1&%
  {\normalfont\Large\bfseries\scshape Problem~\hwnum.\theproblem}
  \global\def\exercisename{Problem~\hwnum.\theproblem}%
  \else
  {\normalfont\Large\bfseries\scshape Problem~\hwnum.\theproblem~(#1)}
  \global\def\exercisename{Problem~\hwnum.\theproblem(#1)}
  \fi
  \par\vspace{\baselineskip}%
  \noindent\ignorespaces
}%
{% end code
  \par\vspace{\baselineskip}%
  \noindent\ignorespacesafterend
}

\newtheorem*{definition*}{Definition}
\newtheorem*{example*}{Examples}
\newtheorem*{examples*}{Examples}
\newtheorem*{exercise*}{Exercise}
\newtheorem*{problem*}{Problem}

\theoremstyle{remark}
\newtheorem{remark}{Remark}
\newtheorem{remarks}[remark]{Remarks}
\newtheorem{observation}[remark]{Observation}
\newtheorem{observations}[remark]{Observations}

\newtheorem*{remark*}{**Remark**}
\newtheorem*{remarks*}{**Remarks**}
\newtheorem*{observation*}{**Observation**}
\newtheorem*{observations*}{**Observations**}

%% Commands and operators
%% Redefinitions & commands
\newcommand{\nsubset}{\ensuremath{\not\subset}}
\newcommand{\nsupset}{\ensuremath{\not\supset}}
\newcommand\minus{\ensuremath{\null\smallsetminus}}
\renewcommand\qedsymbol{\ensuremath{\null\hfill\blacksquare}}

%% Commands and operators
\DeclareMathOperator{\id}{id}
\DeclareMathOperator{\im}{im}
\DeclareMathOperator{\Int}{int}
\DeclareMathOperator{\Cl}{cl}

%% Calculus operators
\newcommand{\diff}{\;\mathrm{d}}

%% Misc operators
\newcommand\restrict[2]{{
  \left.\kern-\nulldelimiterspace
  #1
  % \vphantom{\big|}
  \,\right|\,{#2}
  }}

%% Symbols
\newcommand{\bbC}{\mathbb{C}}
\newcommand{\bbN}{\mathbb{N}}
\newcommand{\bbQ}{\mathbb{Q}}
\newcommand{\bbR}{\mathbb{R}}
\newcommand{\bbZ}{\mathbb{Z}}

\newcommand{\bfC}{\mathbf{C}}
\newcommand{\bfN}{\mathbf{N}}
\newcommand{\bfQ}{\mathbf{Q}}
\newcommand{\bfR}{\mathbf{R}}
\newcommand{\bfZ}{\mathbf{Z}}

\newcommand{\calA}{\mathcal{A}}
\newcommand{\calB}{\mathcal{B}}
\newcommand{\calC}{\mathcal{C}}
\newcommand{\calF}{\mathcal{F}}
\newcommand{\calO}{\mathcal{O}}
\newcommand{\calP}{\mathcal{P}}
\newcommand{\calQ}{\mathcal{Q}}
\newcommand{\calS}{\mathcal{S}}
\newcommand{\calT}{\mathcal{T}}
\newcommand{\calU}{\mathcal{V}}
\newcommand{\calV}{\mathcal{U}}

\newcommand{\scrF}{\mathscr{F}}
\newcommand{\scrL}{\mathscr{L}}
\newcommand{\scrO}{\mathscr{O}}
\newcommand{\scrS}{\mathscr{S}}
\newcommand{\scrT}{\mathscr{T}}

\begin{document}
\frontmatter
\aliaspagestyle{title}{empty}
\pagestyle{title}
\author{\href{mailto:\authoremail}{\documentauthor}}
\title{\documenttitle}
\date{\today}
\maketitle
% \cleartooddpage
% \makepagestyle{my-headings}
% \makeoddhead{my-headings}
%         {\small{\MakeUppercase{\itshape\documentauthor}}}
%         {}
%         {\small{\MakeUppercase{\itshape\exercisename}}}
% \makeoddfoot{my-headings}{{\itshape\documenttitle}}
%                       {}
%                       {\thepage}
% \makeevenhead{my-headings}
%         {\small{\MakeUppercase{\itshape\documentauthor}}}
%         {}
%         {\small{\MakeUppercase{\itshape\exercisename}}}
% \makeevenfoot{my-headings}{{\itshape\documenttitle}}
%                       {}
%                       {\thepage}
% \makeheadrule{my-headings}{\textwidth}{.25pt}
% \pagestyle{my-headings}
% \makerunningwidth{headings}{1.15\textwidth}
\mainmatter

% \begin{problem}
\begin{enumerate}[label=(\alph*)]
\item Every filter $\calF$ is contained in an ultrafilter.
\item A filter $\calF$ in $X$ is an ultrafilter if and only if for each
  $Y\subset X$, either $Y\in\calF$ or $X\minus Y\in\calF$.
\item For any $x\in X$, the principal filter $\calF_{\{x\}}$ is an
  ultrafilter.
\item If $X$ is finite, every ultrafilter $\calF$ in $X$ is principal.
\item If $X$ is infinite and $\calF$ is a nonprincipal ultrafilter, then
  $\calF$ contains $\calF_{\mathrm{cf}}$.
\end{enumerate}
\end{problem}
\begin{proof}
\end{proof}
\newpage

\begin{problem}
There exists an open subset $U\subset C$ such that $1\in U$ and
$U=U^{-1}$.
\end{problem}
\begin{proof}
\end{proof}
\newpage

\begin{problem}
Prove that $\bigcap_{i\leq j}E_{i,j}\neq\emptyset$.
\end{problem}
\begin{proof}
\end{proof}

%%% Local Variables:
%%% mode: latex
%%% TeX-master: "../MA598-PG-HW-Current"
%%% End:

\chapter{Notes}
Let's just turn this file into notes.

\section{Preliminaries}
\subsection{Topological spaces}
By a topological space we mean a pair $(X,\calT)$, where $X$ is a set and
$\calT$ is a set of subsets of $X$ satisfying:
\begin{enumerate}[label=(\roman*)]
\item $\emptyset,X\in\calT$.
\item If $U_1,U_2\in\calT$, then $U_1\cap U_2\in\calT$.
\item For any subset $\calS\subset\calT$, $\bigcup_{U\in\calS} U\in\calT$.
\end{enumerate}

The sets $U\in\calT$ are referred to as \emph{$\calT$-open sets} or,
simply, \emph{open sets} when the topology $\calT$ is understood. A subset
$C\subset X$ is \emph{$\calT$-closed} or, simply, \emph{closed} if
$X\minus C\in\calT$. Given a subset $Y\subset X$, we define the
\emph{closure} of $Y$ in $X$ to be the intersection of all closed sets
$C\subset X$ such that $Y\subset C$. We denote the closure of $Y$ by
$\overline Y$. We say $Y\subset X$ is \emph{dense} if $\overline Y=X$. For
each $x\in X$, we say $U\in\calT$ is an \emph{open neighborhood} of $x$ or,
simply, a \emph{neighborhood} of $x$ if $x\in U$. A \emph{base} for a
topology $\calT$ is any subset $\calB$ of $\calT$ such that every
$U\in\calT$ can be expressed as a union of open sets in $\calB$. A
\emph{neighborhood base} at $x$ is any collection $\calB_x$ of
neighborhoods of $x$ such that every every neighborhood of $x$ can be
expressed as a union of sets in $\calB_x$.

\begin{example}[Discrete topology]
If $\calT$ is the power set $\calP(x)$ of $X$ then $\calT$ is a
topology. This topology is called the \emph{discrete topology}. Every set
of $X$ is both open and closed.
\end{example}

Given a topological space $(X,\calT)$ and a subset $Y\subset X$, we define
the \emph{subspace topology $\calT_{X,Y}$} on $Y$ by
\begin{equation}
  \label{eq:subspace-topology}
\calT_{X,Y}\coloneqq\left\{\,U\cap Y:U\in\calT\,\right\}.
\end{equation}
We will refer to $(Y,\calT_{X,Y})$ as a subspace of $(X,\calT)$. We say
that $(X,\calT)$ is \emph{compact} if given any subset $\calS\subset\calT$
such that $X=\bigcup_{U\in\calS} U$, there exists a finite subset
$\calS_0\subset\calS$ such that $X=\bigcup_{U\in\calS_0} U$. We will say
that a subset $Y\subset(X,\calT)$ is compact if $(Y,\calT_{X,Y})$ is a
compact space. The following lemma is immediate from the definition of
closed and compact.

\begin{lemma}
Let $(X,\calT)$ be a compact space. If $\calS$ is a collection of closed
sets of $X$ such that for any finite subset $\calS_0\subset\calS$, we have
$\bigcap_{C\in\calS_0}C\neq\emptyset$, then $\bigcap_{C\in\calS}C\neq\emptyset$.
\end{lemma}

We say a space $(X,\calT)$ is \emph{Hausdorff} if given distinct
$x_1,x_2\in X$, there exist disjoint open sets $U_1,U_2\in\calT$ such that
$x_i\in U_i$ for $i=1,2$. If $X$ is a Hausdorff space, then $\{x\}$ is
closed for all $x\in X$. We say a space $(X,\calT)$ is \emph{connected} if
$X$ cannot be expressed as the union of two disjoint closed sets. We say a
space $(X,\calT)$ is totally disconnected if every connected subspace has
at most one element.

\begin{lemma}
Let $X$ be a compact Hausdorff space.
\begin{enumerate}[label=(\alph*)]
\item If $C_1,C_2$ are disjoint closed subsets of $X$, then there exists
  disjoint open subsets $U_1,U_2$ of $X$ such that $C_i\subset U_i$ for
  $i=1,2$.
\item If $x\in X$ and $A_x$ is the intersection of all sets $U$ containing
  $x$ such that are both open and closed, then $A_x$ is connected.
\item If $X$ is also totally disconnected, then every open set is a union
  of sets that are both open and closed.
\end{enumerate}
\end{lemma}
\begin{proof}
We start with (a). First, we assert that for each $x\in C_1$ there exists
disjoint open sets $U_x$ and $V_x$ such that $x\in U_x$ and $C_2\subset
V_x$. For each $y\in C_2$, there exists disjoint open sets $U_{x,y}$ and
$V_{x,y}$ such that $x\in U_{x,y}$, and $y\in V_{x,y}$. The set of open
sets $\calC_x=\left\{X\minus C_2\right\}\cup\left\{V_{x,y}\right\}_{y\in
  C_2}$ is an open cover of $X$. Since $X$ is compact, there exists a
finite subset $\left\{y_1,...,y_n\right\}$ of $C_2$ such that $X$ is a
union of $X-C_2$ and the sets $V_{x,y_i}$. Taking
\[
U_x\coloneqq\bigcap_{i=1}^n U_{x,y_i}\quad\text{, and}\quad
V_x\coloneqq\bigcup_{i=1}^n V_{x,y_i},
\]
verifies our first assertion.
\end{proof}

%%% Local Variables:
%%% mode: latex
%%% TeX-master: "../MA598-PG-HW-Current"
%%% End:

\end{document}

%%% Local Variables:
%%% mode: latex
%%% TeX-master: t
%%% End:
