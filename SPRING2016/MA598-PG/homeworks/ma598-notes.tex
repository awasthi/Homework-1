\chapter{Notes}
Let's just turn this file into notes.

\section{Preliminaries}
\subsection{Topological spaces}
By a topological space we mean a pair $(X,\calT)$, where $X$ is a set and
$\calT$ is a set of subsets of $X$ satisfying:
\begin{enumerate}[label=(\roman*)]
\item $\emptyset,X\in\calT$.
\item If $U_1,U_2\in\calT$, then $U_1\cap U_2\in\calT$.
\item For any subset $\calS\subset\calT$, $\bigcup_{U\in\calS} U\in\calT$.
\end{enumerate}

The sets $U\in\calT$ are referred to as \emph{$\calT$-open sets} or,
simply, \emph{open sets} when the topology $\calT$ is understood. A subset
$C\subset X$ is \emph{$\calT$-closed} or, simply, \emph{closed} if
$X\minus C\in\calT$. Given a subset $Y\subset X$, we define the
\emph{closure} of $Y$ in $X$ to be the intersection of all closed sets
$C\subset X$ such that $Y\subset C$. We denote the closure of $Y$ by
$\overline Y$. We say $Y\subset X$ is \emph{dense} if $\overline Y=X$. For
each $x\in X$, we say $U\in\calT$ is an \emph{open neighborhood} of $x$ or,
simply, a \emph{neighborhood} of $x$ if $x\in U$. A \emph{base} for a
topology $\calT$ is any subset $\calB$ of $\calT$ such that every
$U\in\calT$ can be expressed as a union of open sets in $\calB$. A
\emph{neighborhood base} at $x$ is any collection $\calB_x$ of
neighborhoods of $x$ such that every every neighborhood of $x$ can be
expressed as a union of sets in $\calB_x$.

\begin{example}[Discrete topology]
If $\calT$ is the power set $\calP(x)$ of $X$ then $\calT$ is a
topology. This topology is called the \emph{discrete topology}. Every set
of $X$ is both open and closed.
\end{example}

Given a topological space $(X,\calT)$ and a subset $Y\subset X$, we define
the \emph{subspace topology $\calT_{X,Y}$} on $Y$ by
\begin{equation}
  \label{eq:subspace-topology}
\calT_{X,Y}\coloneqq\left\{\,U\cap Y:U\in\calT\,\right\}.
\end{equation}
We will refer to $(Y,\calT_{X,Y})$ as a subspace of $(X,\calT)$. We say
that $(X,\calT)$ is \emph{compact} if given any subset $\calS\subset\calT$
such that $X=\bigcup_{U\in\calS} U$, there exists a finite subset
$\calS_0\subset\calS$ such that $X=\bigcup_{U\in\calS_0} U$. We will say
that a subset $Y\subset(X,\calT)$ is compact if $(Y,\calT_{X,Y})$ is a
compact space. The following lemma is immediate from the definition of
closed and compact.

\begin{lemma}
Let $(X,\calT)$ be a compact space. If $\calS$ is a collection of closed
sets of $X$ such that for any finite subset $\calS_0\subset\calS$, we have
$\bigcap_{C\in\calS_0}C\neq\emptyset$, then $\bigcap_{C\in\calS}C\neq\emptyset$.
\end{lemma}

We say a space $(X,\calT)$ is \emph{Hausdorff} if given distinct
$x_1,x_2\in X$, there exist disjoint open sets $U_1,U_2\in\calT$ such that
$x_i\in U_i$ for $i=1,2$. If $X$ is a Hausdorff space, then $\{x\}$ is
closed for all $x\in X$. We say a space $(X,\calT)$ is \emph{connected} if
$X$ cannot be expressed as the union of two disjoint closed sets. We say a
space $(X,\calT)$ is totally disconnected if every connected subspace has
at most one element.

\begin{lemma}
Let $X$ be a compact Hausdorff space.
\begin{enumerate}[label=(\alph*)]
\item If $C_1,C_2$ are disjoint closed subsets of $X$, then there exists
  disjoint open subsets $U_1,U_2$ of $X$ such that $C_i\subset U_i$ for
  $i=1,2$.
\item If $x\in X$ and $A_x$ is the intersection of all sets $U$ containing
  $x$ such that are both open and closed, then $A_x$ is connected.
\item If $X$ is also totally disconnected, then every open set is a union
  of sets that are both open and closed.
\end{enumerate}
\end{lemma}
\begin{proof}
We start with (a). First, we assert that for each $x\in C_1$ there exists
disjoint open sets $U_x$ and $V_x$ such that $x\in U_x$ and $C_2\subset
V_x$. For each $y\in C_2$, there exists disjoint open sets $U_{x,y}$ and
$V_{x,y}$ such that $x\in U_{x,y}$, and $y\in V_{x,y}$. The set of open
sets $\calC_x=\left\{X\minus C_2\right\}\cup\left\{V_{x,y}\right\}_{y\in
  C_2}$ is an open cover of $X$. Since $X$ is compact, there exists a
finite subset $\left\{y_1,...,y_n\right\}$ of $C_2$ such that $X$ is a
union of $X-C_2$ and the sets $V_{x,y_i}$. Taking
\[
U_x\coloneqq\bigcap_{i=1}^n U_{x,y_i}\quad\text{, and}\quad
V_x\coloneqq\bigcup_{i=1}^n V_{x,y_i},
\]
verifies our first assertion.
\end{proof}

%%% Local Variables:
%%% mode: latex
%%% TeX-master: "../MA598-PG-HW-Current"
%%% End:
