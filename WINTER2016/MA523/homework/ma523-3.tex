\begin{problem}
  Consider the initial value problem
  \[
    u_t=\sin u_x;\qquad u(x,0)=\frac{\pi}{4}x.
  \]
  Verify that the assumptions of the Cauchy--Kovalevskaya theorem are
  satisfied and obtain the taylor series of the solution about the origin.
\end{problem}
\begin{solution}
  The initial value problem certainly satisfies the assumptions of the
  Cauchy--Kovalevskaya theorem, that is, setting
  \(\bfu\defeq(u,u_x,u_t,t)\), the \(\bfb\) are all identically \(0\), and
  \(\bfc(\bfu,x)=\sin u_x(x,t)\) is analytic. Next we show that the Taylor
  series of \(u\) at \((0,0)\),
  \[
    u(x,t)=\sum_{\alpha,\beta} \frac{a_{\alpha,\beta}}{\alpha!\beta!}
    x^\alpha t^\beta
  \]
  is a solution to our PDE.

  First, we must compute the coefficients \(a_{\alpha,\beta}\). To this
  end, we must find the partial derivatives \(u_{\alpha,\beta}\) and
  potentially, relations among them which will help us to find these
  coefficients. Naïvely listing the partials with respect to \(t\) and
  \(x\), we have
  \begin{align*}
    u(0,0)&=0\\
    u_x(0,0)&=\frac{\pi}{4}\\
    u_t(0,0)&=\sin u_x(0,0)=\frac{\sqrt{2}}{2}\\
    u_{xx}(0,0)&=0\\
    u_{tx}(0,0)&=0\\
    u_{tt}(0,0)&=-\cos\bigl(u_x(0,0)\bigr)u_{xt}(0,0)=0\\
    u_{xxx}(0,0)&=0\\
    u_{ttx}(0,0)&=0,
  \end{align*}
  etc. Thus,
  \begin{equation}
    \label{eq:taylor-exp}
    u=\frac{\pi}{4}x+\frac{\sqrt{2}}{2}t.
  \end{equation}
  Plugging in Eq.\@ \eqref{eq:taylor-exp} into our PDE, we have
  \[
    u_t-\sin u_x=\frac{\sqrt{2}}{2}-\sin(\pi/4)=0,
  \]
  as desired.
\end{solution}
\newpage

\begin{problem}
  Consider the Cauchy problem for \(u(x,y)\)
  \begin{align*}
    u_y&=a(x, y, u)u_x+b(x,y,u)\\
    u(x,0)&=0
  \end{align*}
  let \(a\) and \(b\) be analytic functions of their arguments. Assume that
  \(d^\alpha a(0,0,0)\geq 0\) and \(d^\alpha b(0,0,0)\geq 0\) for all
  \(\alpha\). (Remember by definition, if \(\alpha=0\) then
  \(D^\alpha f=f\).)
  \begin{enumerate}[label=(\alph*),noitemsep]
  \item Show that \(D^\beta u(0,0)\geq 0\) for all \(|\beta|\leq 2\).
  \item Prove that \(D^\beta u(0,0)\geq 0\) for all
    \(\beta=(\beta_1,\beta_2)\). (\emph{Hint:} Argue as in the proof of the
    Cauchy--Kovalevskaya theorem; i.e., use induction in \(\beta_2\))
  \end{enumerate}
\end{problem}
\begin{solution}
  Write
  \[
    a(x,y,u)=\sum_{\alpha,\beta,\gamma} a_{\alpha,\beta,\gamma}x^\alpha
    y^\beta u^\gamma,%
    \qquad%
    b(x,y,u)=\sum_{\alpha,\beta,\gamma} b_{\alpha,\beta,\gamma}x^\alpha
    y^\beta u^\gamma
  \]
  where the right-hand side of the expressions above converge to the
  left-hand side for \(|x|+|y|+|u|<r\) for some sufficiently small \(r\).

  For part (a) we show this explicitly by considering all cases. The case
  \(\beta=(0,0)\) is obvious as are the cases \(\beta=(0,1)\) and
  \(\beta=(1,0)\) since \(u_x(0,0)=0\) and
  \begin{align*}
    u_y(0,0)
    &=a\bigl(0,0,u(0,0)\bigr)u_x(0,0)+b\bigl( 0,0,u(0,0) \bigr)\\
    &=a(0,0,0)u_x(0,0)+b(0,0,0)\\
    &=b(0,0,0)\\
    &\geq 0
  \end{align*}
  since \(b\) is a series of strictly positive numbers. For
  \(\beta=(2,0)\), we have \(u_{xx}(0,0)=0\). For \(\beta=(1,1)\), we have
  \begin{align*}
    u_{xy}(0,0)&=a\bigl(0,0,u(0,0)\bigr)u_{xx}(0,0)+\frac{\partial}{\partial
                 x}a\bigl( 0,0,u(0,0) \bigr)u_x(0,0)
                 +\frac{\partial}{\partial x} b\bigl(0,0,u(0,0)\bigr)\\
               &=\frac{\partial}{\partial x} b(0,0,0)\\
               &\geq 0.
  \end{align*}
  For \(\beta=(0,2)\), we have
  \begin{align*}
    u_{yy}(0,0)&=a\bigl(0,0,u(0,0)\bigr)u_{xy}(0,0)+\frac{\partial}{\partial
                 y}a\bigl( 0,0,u(0,0) \bigr)u_x(0,0)
                 +\frac{\partial}{\partial y} b\bigl(0,0,u(0,0)\bigr)\\
               &=a(0,0,0)\frac{\partial}{\partial
                 y}b(0,0,0)+\frac{\partial}{\partial y}b(0,0,0)\\
               &\geq 0
  \end{align*}
  since the latter is a sum of positive numbers.

  For part (b), in the proof of the Cauchy--Kovalevskaya theorem, for
  \(\beta_2=0\), we have
  \[
    D^\beta u(0,0)=0
  \]
  since \(u\) is constant on the hypersurface \(\{\,y=0\,\}\). In
  particular, \(D^\beta u(0,0)\geq 0\).

  Now, suppose \(D^\beta u(0,0)\geq 0\) for all \(\beta_2\leq n-1\). Then,
  for \(\beta=(m,n)\), we have
  \begin{align*}
    D^\beta u(0,0)
    &=d^{(m,n-1)} u_y(0,0)\\
    &=d^{(m,n-1)}\bigl(au_x+b\bigr)(0,0)\\
    &=
  \end{align*}

  This is essentially proven in Evans. The expression above will be given
  by a polynomial in derivatives of lower order (as in equation
  (23)). Since these derivatives are products of positive numbers since
  \(D^\beta a(0,0),D^\beta b(0,0)\geq 0\), this shows that \(D^\beta
  u(0,0)\geq 0\). I tried finding an expression for this, but it was not
  very nice.
\end{solution}
\newpage

\begin{problem}
  (Kovalevskaya's example) show that the line \(\{\,t=0\,\}\) is
  characteristic for the heat equation \(u_t=u_{xx}\). Show there does not
  exist an analytic solution \(u\) of the heat equation in
  \(\bfR\times\bfR\), with \(u=1/(1+x^2)\) on \(\{t=0\}\). (\emph{Hint:}
  assume there is an analytic solution, compute its coefficients, and show
  instead that the resulting power series diverges in any neighborhood of
  \((0,0)\).)
\end{problem}
\begin{solution}
  First we show that the line \(\gamma\defeq\{\,t=0\,\}\) is characteristic
  for the heat equation. With \(\bfnu=(1,0)\) the normal to the line
  \(\gamma\), the noncharacteristic condition reads
  \[
    \sum_{|\alpha|=2} a_\alpha\bfnu^\alpha\neq 0.
  \]
  However,
  \[
    \sum_{|\alpha|=2} a_\alpha\bfnu^\alpha=%
    1\cdot 1+a_{0,2}\cdot 0=%
    1\neq%
    0.
  \]
  Thus, \(\gamma\) is characteristic for \(u_t=u_{xx}\).

  Next suppose \(u\) is an analytic solution to the heat equation with
  \[
    u(x,t)=\sum_{m,n} \frac{a_{m,n}}{m!n!}x^mt^n
  \]
  on \(\bfR\times\bfR\).

  Let us compute the coefficients \(a_{m,n}\) near \((0,0)\). From the PDE,
  we have the relation
  \begin{equation}
    \label{eq:3:pde-relation}
    \begin{aligned}
      a_{m,n}
      &=d^{(m,n)} u(0,0)\\
      &=d^{(m,n-1)}u_t(0,0)\\
      &=d^{(m,n-1)}u_{xx}(0,0)\\
      &=d^{(m+2,n-1)}u(0,0)\\
      &=a_{m+2,n-1}.
    \end{aligned}
  \end{equation}

  Form the boundary condition, we have
  \begin{equation}
    \label{eq:3:taylor-exp-g}
    u(x,0)=\sum_{k=1}^\infty (-1)^k x^{2k}
  \end{equation}
  for a sufficiently small neighborhood about \((0,0)\), where the
  right-hand side is given taylor series of \(1/(1+x^2)\). Taking the
  \(m\)\textsuperscript{th} \(x\)-partial derivative at \((0,0)\), with the
  help of Eq.\@ \eqref{eq:3:taylor-exp-g} we find the coefficients
  \begin{equation}
    \label{eq:3:boundary-relation}
    a_{m,0}=
    \begin{cases}
      0&\text{if \(m=2k+1\) is odd}\\
      (-1)^k(2k)!&\text{if \(m=2k\) is even.}
    \end{cases}
  \end{equation}

  Putting all of this information together, we deduce that
  \[
    a_{2m+1,n}=0
  \]
  for all \(m,n\) and, recursively,
  \[
    a_{2m,n}=a_{2m+2,n-1}=\dotsb=
    a_{2(m+n),0}=(-1)^{m+n}\bigl(2(m+n)\bigr)!.
  \]
  From this we see that the coefficients of the form \(a_{2n,n}\) grow very
  quickly, that is,
  \begin{align*}
    \frac{a_{2n,n}}{(2n)!n!}
    &=(-1)^{2n}\frac{\bigl(2(n+n)\bigr)!}{(2n)!n!}\\
    &=\frac{(4n)!}{(2n)!n!}
      \intertext{which, by Stirling's formula, is asymptotically equal to}
    &\asymp\frac{\sqrt{2\pi n}(4n/\rme)^{4n}}{\sqrt{4\pi
      n}(2n/\rme)^{2n}\sqrt{2\pi n}(n/\rme)^n}\\
    &=\frac{\sqrt{2\pi n}(4n/\rme)^{4n}}{\sqrt{8\pi
      n^2}(2n/\rme)^{2n}(n/\rme)^n}\\
    &=\frac{\bigl(\sqrt{\pi/n}\bigr)4^{4n}}{2\cdot
      2^{2n}}\left(\frac{n}{\rme}\right)^{4n-3n-n}\\
    &=\frac{\sqrt{\pi/n}}{2}\left(\frac{16}{2}\right)^{2n}
      \left(\frac{n}{\rme}\right)^n\\
    &=\bigl(\sqrt{\pi/n}\bigr)2^{6n-1}\left(\frac{n}{\rme}\right)^n\\
    &=\alpha\beta_nn^{n+1/2}
  \end{align*}
  which approaches \(\infty\) as \(n\to\infty\). This shows that for
  \(x,t>0\), the terms \(a_{2n,n}\) grow arbitrarily large.% ; taking
  % \(X=\min\{x,t\}\), we have
  % \begin{align*}
  %   \frac{a_{2n,n}}{(2n)!n!}x^{2n}t^n
  %   &\asymp\alpha\beta_nn^{n+1/2}x^{2n}t^n\\
  %   &\geq\alpha\beta_nn^{n+1/2}X^{3n}.
  % \end{align*}
  % If \(X\geq 1\), these terms clearly grow arbitrarily large so suppose
  % that \(X<1\). Then we can write \(X=1/Y\) for some \(Y>1\) and we have
  % \begin{align*}
  %   \alpha\beta_n \frac{n^{n+1/2}}{Y^{3n}}
  %   &\geq M
  %     \intertext{if and only if}
  %     \frac{\alpha\beta_n n^{n+1/2}}{M}
  %   &\geq Y^{3n}\\
  %   \log_Y
  %   \left(
  %   \frac{\alpha\beta_n n^{n+1/2}}{M}
  %   \right)
  %   &\geq 3n.
  % \end{align*}
  % (I'm sure this can be achieved somehow).

  In particular, for any small \(t>0\), we want the series
  \[
    \sum_{n}a_{0,n}t^n
  \]
  to converge to \(0\) but the size of the coefficients \(a_{0,n}\) prevent
  us from doing this. This leads to a contradiction.
\end{solution}

%%% Local Variables:
%%% mode: latex
%%% TeX-master: "../MA523-Current-HW"
%%% End:
