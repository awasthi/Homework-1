\subsection{Homework 5}
\subsubsection{Problems}
\begin{problem}[Ross, \S 5, \# 16]

\end{problem}
\begin{solution*}
\end{solution*}

\begin{problem}[Ross, \S 5, \# 20]
\end{problem}
\begin{solution*}
\end{solution*}

\begin{problem}[Ross, \S 5, \# 23]
\end{problem}
\begin{solution*}
\end{solution*}

\begin{problem}[Ross, \S 5, \# 30]
\end{problem}
\begin{solution*}
\end{solution*}

\subsubsection{Theoretical exercises}
\begin{problem}[Ross, \S 5, \# 11]
\end{problem}
\begin{solution*}
\end{solution*}

\begin{problem}[Ross, \S 5, \# 31]
\end{problem}
\begin{solution*}
\end{solution*}

\subsubsection{Self-test problems}
\begin{problem}
  Let \(f(x)=\frac{1}{\sqrt{2\pi\sigma_1^2}}\rme^{-x^2/(2\sigma_1^2)}\) and
  \(g(x)=\frac{1}{\sqrt{2\pi \sigma_2^2}}\rme^{-x^2/(2\sigma_2^2)}\). Show
  that
  \[
    \int_{-\infty}^\infty f(y)g(x-y)\diff y=
    \tfrac{1}{\sqrt{2\pi(\sigma_1^2+\sigma_2^2)}}\rme^{-x^2/2(\sigma_1^2+\sigma_2^2)}.
  \]

  \noindent\emph{Hint:} there are many methods to do this problem. One,
  outlined in class is to first combine the exponentials and then compute
  the square for the quadratic function in the exponential.
\end{problem}
\begin{solution*}
\end{solution*}

%%% Local Variables:
%%% mode: latex
%%% TeX-master: "../MA519-HW-ALL"
%%% End:
