\section{Homework Solutions (Yip)}
These are the solutions to Yip's Math/Stat 519 homework for the fall
semester of 2016.

Our main reference is \cite{ross}.

\subsection{Homework 1}
\subsubsection{Problems}
\begin{problem}[Ross, \S 1, \# 7]
  \hfill
  \begin{alphlist}
  \item In how many different ways can \(3\) boys and \(3\) girls sit in a
    row?
  \item In how many ways can \(3\) boys and \(3\) girls sit in a row if the
    boys and the girls are each to sit together?
  \item In how many ways if only the boys must sit together?
  \item In how many ways if no two people of the same sex are allowed to
    sit together?
  \end{alphlist}
\end{problem}
\begin{solution*}
\end{solution*}

\begin{problem}[Ross, \S 1, \# 11]
  In how many ways can \(3\) novels, \(2\) mathematics books, and \(1\)
  chemistry book be arranged on a bookshelf if
  \begin{alphlist}
  \item the books can be arranged in any order?
  \item the mathematics books must be together and the novels must be
    together?
  \item the novels must be together, but the other books can be arranged in
    any order?
  \end{alphlist}
\end{problem}
\begin{solution*}
\end{solution*}

\begin{problem}[Ross, \S 1, \# 19]
  From a group of \(8\) women and \(6\) men, a committee consisting of
  \(3\) men and \(3\) women is to be formed. How many different committees
  are possible if
  \begin{alphlist}
  \item \(2\) of the men refuse to serve together?
  \item \(2\) of the women refuse to serve together?
  \item \(1\) mand and \(1\) woman refuse to serve together?
  \end{alphlist}
\end{problem}
\begin{solution*}
\end{solution*}

\begin{problem}[Ross, \S 1, \# 21]
  Consider the grid of points show at the top of the next column (in the
  book; we draw it here using Asymptote).
  \[
    \includegraphics{yip-1-1}
  \]
  Suppose that, starting at the point labeled \(A\), you can go one step up
  or one step to the right at each move. This procedure is continued until
  the point labeled \(B\) is reached. How many different paths from \(A\)
  to \(B\) are possible?

  \noindent\emph{Hint:} Note that to reach \(B\) from \(A\) you must take
  \(4\) steps to the right and \(3\) steps up.
\end{problem}
\begin{solution*}
\end{solution*}

\begin{problem}[Ross, \S 1, \# 22]
  In Problem \# 21, how many different paths are there from \(A\) to \(B\)
  that go through the point circled in the following lattice?
  \[
    \includegraphics{yip-1-2}
  \]
\end{problem}
\begin{solution*}
\end{solution*}

\begin{problem}[Ross, \S 1, \# 33]
  We have \(\$ 20,000\) that must be invested among \(4\) possible
  opportunities. Each investment must be integral in units of \(\$1000\),
  and there are minimal investment that need to be made if one is to invest
  these opportunities. The minimal investments are \(\$2000\), \(\$2000\),
  \(\$3000\), and \(\$4000\). How many different investment strategies are
  available if
  \begin{alphlist}
  \item an investment must be made in each opportunity?
  \item investments must be made in at least \(3\) of the \(4\)
    opportunities?
  \end{alphlist}
\end{problem}
\begin{solution*}
\end{solution*}

\subsubsection{Theoretical exercises}
\begin{problem}[Ross, \S 1, \# 5]
  Determine the number of vectors \((x_1,\dotsc,x_n)\) such that each
  \(x_k\) is either \(0\) or \(1\) and
  \[
    \sum_{k=1}^n x_k\geq l.
  \]
\end{problem}
\begin{solution*}
\end{solution*}

\begin{problem}[Ross, \S 1, \# 6]
  How many vectors \(x_1,\dotsc,x_k\) are there for which each \(x_k\) is a
  positive integer such that \(1\leq x_k\leq n\) and \(x_1<x_2<\dotsb<
  x_n\)?
\end{problem}
\begin{solution*}
\end{solution*}

\begin{problem}[Ross, \S 1, \# 8]
  Prove that
  \[
    \binom{n+m}{r}=%
    \binom{n}{0}\binom{m}{r}+\binom{n}{1}\binom{m}{r-1}%
    +\dotsb+\binom{n}{r}\binom{m}{0}.
  \]
\end{problem}
\begin{solution*}
\end{solution*}

\begin{problem}[Ross, \S 1, \# 9]
  Use Theoretical Exercise 8 to prove that
  \[
    \binom{2}{n}=\sum_{k=0}^n\binom{n}{k}^2.
  \]
\end{problem}
\begin{solution*}
\end{solution*}

\begin{problem}[Ross, \S 1, \# 12]
  Consider the following combinatorial identity:
  \[
    \sum_{k=1}^n k\binom{n}{k}=n2^{n-1}.
  \]
  \begin{alphlist}
  \item Present a combinatorial argument for this identity by considering
    the set of \(n\) people and determining, in two ways, the number of
    possible selections of a committee of any size and a chairperson for
    the committee.

    \noindent\emph{Hint:}
    \begin{enumerate}[label=(\roman*)]
    \item How many possible elections are there of a committee of size
      \(k\) and its chairperson?
    \item How many possible selections are there of a chairperson and the
      other committee members?
    \end{enumerate}
  \item Verify the following identity for \(n=1,2,3,4,5\):
    \[
      \sum_{k=1}^n \binom{n}{k}k^2=2^{n-2}n(n+1).
    \]
    For a combinatorial proof of the preceding, consider the set of \(n\)
    people and argue that both sides of the identity represent the number
    of different selections of a committee, its chairperson, and its
    secretary (possibly the same as the chairperson).

    \noindent\emph{Hint:}
    \begin{enumerate}[label=(\roman*)]
    \item How many different selection result in the committee containing
      exactly \(k\) people?
    \item How many different selections are there in which the chairperson
      and the secretary are the same? (Answer: \(n2^{n-1}\).)
    \item How many different selections result in a chairperson and the
      secretary being different?
    \end{enumerate}
  \item Now argue that
    \[
      \sum_{k=1}^n\binom{n}{k}k^3=2^{n-3}n^2(n+3).
    \]
  \end{alphlist}
\end{problem}
\begin{solution*}
\end{solution*}

\begin{problem}[Ross, \S 1, \# 23]
  Determine the number of vectors \((x_1,\dotsc,x_n)\) of \(n\) variables
  such that each \(x_k\) is a nonnegative integer and
  \[
    \sum_{k=1}^n x_k\leq l.
  \]
\end{problem}
\begin{solution*}
\end{solution*}

\subsubsection{Problems}
\begin{problem}[Ross, \S 2, \# 25]
\end{problem}
\begin{solution*}
\end{solution*}

\begin{problem}[Ross, \S 2, \# 29]
\end{problem}
\begin{solution*}
\end{solution*}

\begin{problem}[Ross, \S 2, \# 35]
\end{problem}
\begin{solution*}
\end{solution*}

\begin{problem}[Ross, \S 2, \# 44]
\end{problem}
\begin{solution*}
\end{solution*}

\begin{problem}[Ross, \S 2, \# 49]
\end{problem}
\begin{solution*}
\end{solution*}

\subsubsection{Theoretical exercises}
\begin{problem}[Ross, \S 2, \# 5]
\end{problem}
\begin{solution*}
\end{solution*}

\begin{problem}[Ross, \S 2, \# 14]
\end{problem}
\begin{solution*}
\end{solution*}

\begin{problem}[Ross, \S 2, \# 19]
\end{problem}
\begin{solution*}
\end{solution*}

%%% Local Variables:
%%% mode: latex
%%% TeX-master: "../MA519-HW-ALL"
%%% End:
