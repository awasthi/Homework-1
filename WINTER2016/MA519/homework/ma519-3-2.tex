\subsection{Homework 3}
\subsubsection{Problems}
\begin{problem}[Ross, \S 4, \# 4]
\end{problem}
\begin{solution*}
\end{solution*}

\begin{problem}[Ross, \S 4, \# 21]
\end{problem}
\begin{solution*}
\end{solution*}

\begin{problem}[Ross, \S 4, \# 22]
\end{problem}
\begin{solution*}
\end{solution*}

\begin{problem}[Ross, \S 4, \# 36]
\end{problem}
\begin{solution*}
\end{solution*}

\begin{problem}[Ross, \S 4, \# 37]
\end{problem}
\begin{solution*}
\end{solution*}

\begin{problem}[Ross, \S 4, \# 42]
\end{problem}
\begin{solution*}
\end{solution*}

\begin{problem}[Ross, \S 4, \# 43]
\end{problem}
\begin{solution*}
\end{solution*}

\subsubsection{Theoretical exercises}
\begin{problem}[Ross, \S 4, \# 4]
\end{problem}
\begin{solution*}
\end{solution*}

\begin{problem}[Ross, \S 4, \# 5]
\end{problem}
\begin{solution*}
\end{solution*}

\begin{problem}[Ross, \S 4, \# 10]
\end{problem}
\begin{solution*}
\end{solution*}

\begin{problem}[Ross, \S 4, \# 13]
\end{problem}
\begin{solution*}
\end{solution*}

\begin{problem}[Ross, \S 4, \# 14]
\end{problem}
\begin{solution*}
\end{solution*}

\begin{problem}[Ross, \S 4, \# 18]
\end{problem}
\begin{solution*}
\end{solution*}

\begin{problem}[Ross, \S 4, \# 25]
\end{problem}
\begin{solution*}
\end{solution*}

\begin{problem}[Ross, \S 4, \# 27]
\end{problem}
\begin{solution*}
\end{solution*}

\begin{problem}[Ross, \S 4, \# 30]
\end{problem}
\begin{solution*}
\end{solution*}

\begin{problem}[Ross, \S 4, \# 32]
\end{problem}
\begin{solution*}
\end{solution*}

\begin{problem}[Ross, \S 4, \# 34]
\end{problem}
\begin{solution*}
\end{solution*}

\subsubsection{Self-test problems}
\begin{problem}
  Let \(X\) be a positive-integer valued random varibale; i.e., \(X\) takes
  values in \(\Z^+\). Suppose the distribution of \(X\) satisfies the
  following: for all \(m\), \(n\) it holds that
  \[
    P(X>n)=P(X>m+n).
  \]
  (The above property is called the memoryless property.)

  Show that \(X\) must be a geometric random variable; i.e., there exists a
  \(p\) (\(0<p<1\)) such that for all \(n\)
  \[
    P(X=n)=(1-p)^{n-1}p.
  \]
  (Combined with what is shown in class, you will have actually proved that
  a positive-integer valued random variable is geometric \emph{if and only
    if} it satisfies the memoryless property.)
\end{problem}
\begin{solution*}
\end{solution*}

%%% Local Variables:
%%% mode: latex
%%% TeX-master: "../MA519-HW-ALL"
%%% End:
