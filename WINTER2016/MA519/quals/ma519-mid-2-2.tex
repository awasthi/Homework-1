\subsection{Midterm 2}
\begin{problem}
  Consider the infinitely many independent, identical experiments of
  throwing a pair of dice and the outcome of each experiment is the sum of
  the two numbers. Let \(N\geq 1\) be the number of experiments such that
  the number \(5\) or \(7\) first appears as the outcome. Let furher \(X\)
  be that number at the \(N\)\textsup{th} experiment; i.e., \(X\) can be
  either a \(5\) or a \(7\).
  \begin{alphlist}
  \item Find \(P(N=n)\) for \(n\geq 1\).
  \item Find \(P(X=5)\).
  \item Prove or disprove that \(N\) and \(X\) are independent.
  \end{alphlist}
\end{problem}
\begin{solution*}
\end{solution*}

\begin{problem}
  Consider a city in which the male and female drivers occupy \(\alpha\)
  and \(1-\alpha\) fractions of the whole city driver population.  In any
  given year, a male and female driver will have an accident with
  probability \(p_M\) and \(p_F\). Assume that the behavior of each driver
  is independent from year to year.

  Now a driver is randomly chosen. Let \(A_k\) be the event that this
  driver will have an accident in the \(k\)\textsup{th} year. Let \(M\) be
  the event that the randomly chosen driver is male.
  \begin{alphlist}
  \item Suppose \(p_M>p_F\). Show that \(P(M\mid A_k)>p(M)\).
  \item Suppose \(p_M\neq p_F\). Show that \(P(A_2\mid A_1)>p(A_1)\).
  \end{alphlist}
\end{problem}
\begin{solution*}
  First, the author of these problems writes ungrammatical sentences and
  second he give inconsistent notation. This is driving me nuts!
\end{solution*}

\begin{problem}
\end{problem}
\begin{solution*}
\end{solution*}

\begin{problem}
\end{problem}
\begin{solution*}
\end{solution*}

\begin{problem}
\end{problem}
\begin{solution*}
\end{solution*}

%%% Local Variables:
%%% TeX-master: "../MA519-HW-ALL"
%%% End:
