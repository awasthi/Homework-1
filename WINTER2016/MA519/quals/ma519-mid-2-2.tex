\subsection{Midterm 2}
This is actually the past final.
\begin{problem}
  Consider the infinitely many independent, identical experiments of
  throwing a pair of dice and the outcome of each experiment is the sum of
  the two numbers. Let \(N\geq 1\) be the number of experiments such that
  the number \(5\) or \(7\) first appears as the outcome. Let furher \(X\)
  be that number at the \(N\)\textsup{th} experiment; i.e., \(X\) can be
  either a \(5\) or a \(7\).
  \begin{alphlist}
  \item Find \(P(N=n)\) for \(n\geq 1\).
  \item Find \(P(X=5)\).
  \item Prove or disprove that \(N\) and \(X\) are independent.
  \end{alphlist}
\end{problem}
\begin{solution*}
  For part (a), first let us determine the number of ways in which a \(5\)
  or a \(7\) arise. To form a \(5\) the pair of dice must come up as either
  \((4,1)\) or \((2,3)\) and vice versa; giving us a probability of
  \(\frac{4}{36}=\frac{1}{9}\). To form a \(7\) the pair of dice must come
  up as either \((6,1)\), \((5,2)\)
  out
  \\\\
  For part (b),
  \\\\
  For part (c),
\end{solution*}

\begin{problem}
  Consider a city in which the male and female drivers occupy \(\alpha\)
  and \(1-\alpha\) fractions of the whole city driver population.  In any
  given year, a male and female driver will have an accident with
  probability \(p_M\) and \(p_F\). Assume that the behavior of each driver
  is independent from year to year.

  Now a driver is randomly chosen. Let \(A_k\) be the event that this
  driver will have an accident in the \(k\)\textsup{th} year. Let \(M\) be
  the event that the randomly chosen driver is male.
  \begin{alphlist}
  \item Suppose \(p_M>p_F\). Show that \(P(M\mid A_k)>p(M)\).
  \item Suppose \(p_M\neq p_F\). Show that \(P(A_2\mid A_1)>p(A_1)\).
  \end{alphlist}
\end{problem}
\begin{solution*}
\end{solution*}

\begin{problem}
  Let \(X_1,\dotsc,X_n\) be a collection of IID exponential random
  variables with parameter \(\lambda\). Let
  \begin{align*}
    Y_1&=X_1,\\
    Y_2&=X_1+X_2,\\
       &\vdotswithin{{}={}}\\
    Y_n&=X_1+\dotsb+X_n.
  \end{align*}
  Find the joint PDF \(p(y_1,\dotsc,y_n)\) of \(Y_1,\dotsc,Y_n\).
\end{problem}
\begin{solution*}
\end{solution*}

\begin{problem}
  The PDF \(p(x)\) of the gamma distribution with parameter (\(\alpha>0\),
  \(\lambda>0\)) is given by
  \[
    p(x)=
    \begin{cases}
      \frac{\lambda}{\Gamma(\alpha)}\rme^{-\lambda x}(\lambda x)^{\alpha
        -1}
      &\text{for \(x\geq 0\),}\\
      0&\text{otherwise.}
    \end{cases}
  \]
  Let \(X\) and \(Y\) be independent gamma distributed random variables
  with parameters \((\alpha,\lambda)\) and \((\beta,\lambda)\). Show
  analitically that \(X+Y\) has a gamma distribution with parameter
  \((\alpha+\beta,\lambda)\).

  Show how as a byproduct that the above conclusion leads to the following
  integration identity for \(\alpha,\beta>0\)
  \[
    \int_0^1 x^{\alpha-1}(1-x)^{\beta-1}\diff x
    =\frac{\Gamma(\alpha)\Gamma(\beta)}{\Gamma(\alpha+\beta)}.
  \]
\end{problem}
\begin{solution*}
\end{solution*}

\begin{problem}
  Let \(X_1,\dotsc,X_n\) be a collection of IID random variables with
  expectations and variances equal to \(\mu\) and \(\sigma^2\). Define the
  \emph{sample mean} \(\bar X\) and \emph{sample variance} \(S^2\) as
  \[
    \bar X=\tfrac{1}{n}(X_1+\dotsb+X_n),\quad
    S^2=\frac{1}{n}\sum_{k=1}^n(X_k-\bar X)^2.
  \]
  Compute \(\Var(\bar X)\) and \(E(S^2)\).
\end{problem}
\begin{solution*}
\end{solution*}

\begin{problem}[Estimation of the length of an interval]
  Let \(l>0\) be some unknown but fixed length. Let \(X_1,X_2,\dotsc\), be
  a sequence of IID random variables uniformly distributed on
  \([0,l]\). The goal is to use the \(X_k\) to estimate \(l\).
  \begin{alphlist}
  \item Let \(A_n=\tfrac{2}{n}(X_1+\dotsb+X_n).\) Show that \(A_n\) is an
    unbiased estimation in the sense that \(E(A_n)=l\).
  \item Let \(B_n=\gamma_n\max\{X_1,\dotsc,X_n\}\) where \(\gamma_n\) is
    some number. Find the correct value of \(\gamma_n\) such that \(B_n\)
    is also an unbiased estimator.
  \item Find \(\Var(A_n)\) and \(\Var(B_n)\).
  \item Which
  \end{alphlist}
\end{problem}
\begin{solution*}
\end{solution*}

%%% Local Variables:
%%% TeX-master: "../MA519-HW-ALL"
%%% End:
