\begin{problem}[Munkres \S26, Ex.\,8]
\begin{theorem*}
Let $f\colon X\to Y$; let $Y$ be compact Hausdorff. Then $f$ is
continuous if and only if the \emph{graph} of $f$,
\[
G_f=\left\{\,(x,f(x))\;\middle|\;x\in X\,\right\},
\]
is closed in $X\times Y$.
\end{theorem*}
[\emph{Hint:} If $G_f$ is closed and $V$ is a neighborhood of
$f(x_0)$, then the intersection of $G_f$ and $X\times(Y-V)$ is
closed. Apply Exercise 7.]
\end{problem}
\begin{proof}
As we demonstrated in Problem 2.7 (Munkres \S18, Ex.\,17) $Y$ is
Hausdorff if and only if the diagonal,
$\Delta_Y=\left\{\,(y,y)\;\middle|\;y\in Y\,\right\}$, is a closed
subset of $Y\times Y$. Consider the map $F\colon X\times Y\to Y\times
Y$ defined by $(x,y)\mapsto (f(x),y)$. This map is continuous by
Theorem 18.4 as $f$ is, by assumption, continuous and $\id_Y$ is
continuous by 18.2(b) (since it is the inclusion $Y\hookrightarrow
Y$). Then
\begin{align*}
F^{-1}(\Delta_Y)
&=\left\{\,(x,y)\;\middle|\;\text{$F(x,y)\in\Delta_Y$, $x\in X$, $y\in
  Y$}\,\right\}\\
&=\left\{\,(x,y)\;\middle|\;\text{$(f(x),y)\in\Delta_Y$, $x\in X$,
  $y\in Y$}\,\right\}\\
&=\left\{\,(x,y)\;\middle|\;\text{$f(x)=y$, $x\in X$, $y\in Y$}\,\right\}\\
&=\left\{\,(x,f(x))\;\middle|\;\text{$x\in X$, $y\in Y$}\,\right\}\\
&=G_f
\end{align*}
is closed by Theorem 18.1(3).

Conversely, suppose $G_f$ is closed in $X\times Y$. Fix a point
$x_0\in X$ and let $V\subset Y$ be an arbitrary neighborhood of
$f(x_0)$. Then $Y-V$ is a closed subset of $Y$ so, by Problem 2.1
(Munkres \S17, Ex.\,3), the product $X\times (Y-V)$ is closed in
$Y\times Y$. In particular, by Theorem 17.1(2), the intersection
$B=G_f\cap X\times(Y-V)$ is closed in $X\times Y$. Thus, by Problem
6.5 (Munkres \S26, Ex.\,7), since $Y$ is a compact Hausdorff space,
the projection $\pi_1(B)$ onto $X$ is a closed subset of $X$. But
\begin{align*}
B
&=\left\{\,(x,y)\;\middle|\;\text{$(x,y)\in G_f$ and $(x,y)\in X\times
  (Y-V)$}\,\right\}\\
&=\left\{\,(x,y)\;\middle|\;\text{$y=f(x)$ and $(x,y)\in X\times
  (Y-V)$}\right\}\\
&=\left\{\,(x,f(x))\;\middle|\;f(x)\in Y-V\,\right\}
\end{align*}
so we have that $\pi_1(B)=f^{-1}(Y-V)=X-f^{-1}(V)$. One
containment is easy to see, namely ``$\subset$'': if $x\in B$ then
$x=\pi_1(x,f(x))$ for at least one $f(x)\in Y-V$. To see the reverse
inclusion, take $x\in f^{-1}(Y-V)$, then $f(x)\in Y-V$ so $(x,f(x))\in
B$, hence $x\in\pi_1(B)$. Thus, $X-\pi_1(B)=f^{-1}(V)$ is open
so $f$ is continuous.
\end{proof}
\newpage
\begin{problem}[Munkres \S26, Ex.\,9]
Generalize the tube lemma as follows:
\begin{theorem*}
Let $A$ and $B$ be subspaces of $X$ and $Y$, respectively; let
$N$ be an open set in $X\times Y$ containing $A\times B$. If $A$
and $B$ are compact, then there exist open sets $U$ and $V$ in
$X$ and $Y$, respectively, such that
\[A\times B\subset U\times V\subset N.\]
\end{theorem*}
\end{problem}
\begin{proof}
The idea is to construct an appropriate covering of $A\times B$ using
both compactness of $A$ and compactness of $B$ that will give us the
open sets that we want. Fix an $a\in A$. Then, for every $b\in B$
there exists neighborhoods $U_b\subset X$ and $V_b\subset Y$ of $a$
and $b$, respectively, such that $U_b\times V_b\subset N$ (by the
definition of the product topology and since $N$ is open). Then, since
$B$ is compact, by Lemma 26.1, there exists a finite subcollection,
say $\{V_i\}_{i=1}^{n_a}$, that covers $B$. Let $U_a=\bigcup_{i=1}^{n_a} U_i$
and $V_a=\bigcup_{i=1}^{n_a} V_i$. Varying this over every $a\in A$,
we obtain an open cover $\left\{U_a\times V_a\right\}_{a\in A}$; let's
verify this: Let $(a,b)\in A\times B$, then $a\in
U_a=\bigcup_{i=1}^{n_a}U_i$ (since each $U_i$ is in fact a
neighborhood of $a$) and $b\in V_a=\bigcup_{i=1}^{n_a}V_i$ so $b\in
V_i$ for some $1\leq q\leq n_a$. Thus, by Theorem 26.7, there exists a
finite subcollection $\left\{ U_i\times V_i \right\}_{i=1}^n$ covering
$A\times B$. Take $U=\bigcup_{i=1}^n U_i$ and $V=\bigcap_{i=1}^n
V_i$. Then, we claim that $A\times B\subset U\times V\subset N$.

It is clear, by construction of $U$ and $V$, that $U\times V\subset N$
(and this follows from Lemma 5 proved on Homework 2, i.e.,
if $A,B\subset C$ then $A\cup B,A\cap B\subset C$). To see that
$A\times B\subset U\times V$ take $(a,b)\in A\times B$. Then $a\in
U_i$ for some $1\leq i\leq n$ and $b\in V_i$ for all $i$ (since
$V_i\supset B$ for all $1\leq i\leq n$)so $(a,b)\in U\times V$. Thus,
we have
\[A\times B\subset U\times V\subset N\]
as desired.
\end{proof}
\newpage
\begin{problem}[Munkres \S26, Ex.\,12]
Let $p\colon X\to Y$ be a closed continuous surjective map
such that $p^{-1}(y)$ is compact, for each $y\in Y$. (Such a
map is called a \emph{perfect map}.) Show that if $Y$ is compact,
then $X$ is compact.

[\emph{Hint:} If $U$ is an open set containing $p^{-1}(y)$, there
is a neighborhood $W$ of $y$ such that $p^{-1}(W)$ is contained
in $U$.]
\end{problem}
\begin{proof}
First we shall prove Munkres's hint:
\begin{claim*}
Let $p\colon X\to Y$ be a closed map. If $U$ is an open subset
containing $p^{-1}(y)$ for some $y\in Y$, there exists a neighborhood
$W$ of $y$ such that $p^{-1}(W)\subset U$.
\end{claim*}
\begin{proof}[Proof of claim]
\renewcommand\qedsymbol{$\clubsuit$}
Let $y\in Y$. Suppose that $U$ is an open subset containing
$p^{-1}(y)$. Then, $X-U$ is closed so $p(X-U)$ is closed. In
particular, $y\notin p(X-U)$ (for if it were, we would have
$p^{-1}(y)\subset X-U$, but $U\supset p^{-1}(y)$). Thus $Y-p(X-U)$ is
a neighborhood of $y$ so
\[
p^{-1}(Y-p(X-U))=p^{-1}(Y)-p^{-1}(p(X-U))=X-p^{-1}(p(X-U))\subset U
\]
since, by Problem 1.1(a) (Munkres \S2, Ex.\,1(a)), we have that
$p^{-1}(p(X-U))\supset X-U$.
\end{proof}
Now let $\left\{ U_\alpha \right\}$ be an open cover of $X$. Then,
since $p^{-1}(y)\subset X=\bigcup U_\alpha$ is compact, by Lemma 26.1,
there exists a finite subcollection, say $\{U_i\}_{i=1}^{n_y}$, that
covers $p^{-1}(y)$. Let $U_y=\bigcup_{i=1}^{n_y} U_i$. Then, by the
claim, there exists $W_y$ neighborhood of $y$ such that
$p^{-1}(W_y)\subset\bigcup_{i=1}^{n_y}U_i$. We can do this for every
$y\in Y$. In particular, we see that the collection $\{W_y\}_{y\in Y}$
is an open cover of $Y$ so, since $Y$ is compact, there exists a
finite subcollection, say $\{W_{y_i}\}_{i=1}^n$, that covers $Y$. Then
$p^{-1}(W_{y_i})\subset U_{y_i}$ and
\[
X=p^{-1}(Y)=\bigcup_{i=1}^{n} p^{-1}(W_{y_i})\subset\bigcup_{i=1}^n U_{y_i}.
\]
Thus, $X$ is compact.
\end{proof}
\newpage
\begin{problem}[Munkres \S27, Ex.\,2(b,d)]
Let $X$ be a metric space with metric $d$; let $A\subset X$ be
nonempty.
\begin{itemize}
\item[(b)] Show that if $A$ is compact, $d(x,A)=d(x,a)$ for some
  $a\in A$.
\item[(d)] Assume that $A$ is compact; let $U$ be an open set
  containing $A$. Show that some $\epsilon$-neighborhood of $A$
  is contained in $U$.
\end{itemize}
\end{problem}
\begin{proof}
(b) Fix $x\in X$ and consider the map $d_x\colon A\to\RR$ given by
$a\mapsto d(x,a)$. We claim that $d_x$ is continuous so, assuming this
has been proven, by the extreme value theorem there exists points
$a,b\in A$ such that $d_x(a)\leq d_x(y)\leq d_x(b)$ for every $y\in
A$. In particular, we have that $d(x,A)=\inf_{y\in
  A}d(x,y)=d(x,a)=d_x(a)$ ((i) $d_x(a)\leq d_x(y)$ for all $y$; (ii)
if $d_x(a')\leq d_x(y)$ for all $y\in A$ then $d_x(a)=d_x(a')$ since
$d_x(a)\leq d_x(y)$ for all $y\in A$).

That $d_x$ is continuous follows from the following lemma (which we
shall prove if we have time):
\begin{lemma*}[Munkres \S18, Ex\.,11]
Let $f\colon X\times Y\to Z$. We say that $F$ is \emph{continuous in
  each variable separately} if for each $y_0$ in $Y$, the map $h\colon
X\to Z$ defined by
\end{lemma*}
% (d) Suppose not. Then for every $\epsilon>0$, the open set
% $U(A,\epsilon)$ contains a point $x\notin U$. By part (c),
% $U(A,\epsilon)$ is a union of oen balls $B_d(\a,\epsilon)$ for all
% $a\in A$, so, in particular, this collection covers $A$. Hence, by
% Lemma 26.1, there is a finite subcollection, say
% $\left\{B_d(a_i,\epsilon)\right\}_{i=1}^n$ that covers $A$.
(d) For every $a\in A$, $r>0$ such that $B_d(a,2r)\subset U$ (we are
guaranteed these exist since $U$ is open in the metric topology)
consider the collection $\left\{B_d(a,2r)\right\}$. This collection is an open
cover of $A$, so, by Lemma 26.1, there exists a finite subcollection,
say $\left\{B_d(a_i,2r_i)\right\}_{i=1}^n$ that covers $A$. Let
$r=\min\{r_1,...,r_n\}$ and $a$ be the corresponding basepoint for the
open ball. Then for every $b\in B_d(a_i,r_i)$, we have that
\[B_d(b,r)\subset B_d(a_i,r_i+\epsilon)\subset B_d(a_i,2r_i)\subset U\]
so, by part (c), $U(A,\epsilon)=\bigcup_{b\in A}B_d(b,r)\subset U$
\end{proof}
\newpage
\begin{problem}[Munkres \S27, Ex.\,5]
Let $X$ be a compact Hausdorff space; let $\left\{A_n\right\}$ be
a countable collection of closed sets of $X$. Show that if each
set $A_n$ has empty interior in $X$, then the union $\bigcup A_n$
has empty interior in $X$. [\emph{Hint:} Imitate the proof of
Theorem 27.7.]

This is a special case of the \emph{Baire category theorem},
which we shall study in Chapter 8.
\end{problem}
\begin{proof}
Mimicking the proof of Theorem 27.7, suppose $A\subset X$ is
closed and $U\subset X$ is a nonempty open subset such that
$U\nsubset X$. Then, since $U-A\neq\emptyset$ and $X$ is a
compact Hausdorff space, by Theorem 26.2, the union $A\cup (X-U)$
is compact so, by Theorem 26.4, there exist disjoint
neighborhoods $W$ and $V$ about $A\cup(X-U)$ and $x$,
respectively, such that
\[\overline V\subset X-(A\cup(X-U))=(X-A)\cap U=U-A.\]
Now we show that any nonempty open set, $U_0$, has a point that
is not in the union $\bigcup A_n$. For $A_i$, $i\geq 1$,
$U_{i-1}$ is a nonempty open subset such that $U_{i-1}\nsubset
A_i$, hence, there is a nonempty open set $U_i\subset X$ such
that $\overline U_i\subset U_{i-1}-A_i$. We thus have a nested
sequence of nonempty closed subsets
\[
\overline{U_1}\subset\overline{U_2}\subset\cdots
\]
and their intersection is nonempty since $X$ is compact, such
that any point $x\in\bigcap\overline{U_i}$ belongs to $U_0$, but
not to $\bigcup A_n$.
\end{proof}
\newpage
\begin{problem}[Munkres \S29, Ex.\,2(a)]
Let $\left\{X_\alpha\right\}$ be an indexed family of nonempty
spaces.
\begin{itemize}
\item[(a)] Show that if $\prod X_\alpha$ is locally compact, then
  each $X_\alpha$ is locally compact and $X_\alpha$ is compact
  for all but finitely many values of $\alpha$.
\end{itemize}
\end{problem}
\begin{proof}[Proof of (a)]
Suppose $X=\prod X_\alpha$ is locally compact. Then, for every
$\mathbf{x}\in X$, there exist a compact set $C$ containing an open
neighborhood $U$ of $\mathbf{x}$. We may, without loss of generality,
assume $U=\prod U_\alpha$ where $U_\alpha=X_\alpha$ for all but
finitely many $X_\alpha$. Suppose $U_\beta=X_\beta$. Then
$\pi_\beta(C)=X_\beta$ is compact by Theorem 26.5. It follows that
each $X_\alpha$ is compact for all but finitely man $\alpha$. To see
that each $X_\alpha$ is also locally compact we prove the following
stronger result:
\begin{lemma}[Munkres \S20, Ex.\,3]
If $f\colon X\to Y$ is a continuous and open, then $f(X)$ is locally
compact.
\end{lemma}
\begin{proof}[Proof of lemma]
\renewcommand\qedsymbol{$\clubsuit$}
Since $X$ is locally compact, then for every $x\in X$ there exists a
compact set $C$ containing a neighborhood $U$ of $x$. Then,
$f(U)\subset f(C)$ is a compact set, by Theorem 26.5, containing a
neighborhood, namely $f(U)$, of $f(x)$. Thus, $f(X)$ is locally compact.
\end{proof}
Now, since $\pi_\alpha$ is an open map (generalization of Munkres
\S16, Ex.\,4), it follows that $\pi_\alpha(X)=X_\alpha$ is locally
compact by Lemma 15.
\end{proof}
\newpage
\begin{problem}[Munkres \S29, Ex.\,10]
Show that if $X$ is a Hausdorff space that is locally compact at
the point $x$, then for each neighborhood $U$ of $x$, there is a
neighborhood $V$ of $x$ such that $\overline V$ is compact and
$\overline V\subset U$.
\end{problem}
\begin{proof}
Since $x$ is locally compact, there exists a compact set $C$
containing a neighborhood, say $W$, of $x$. Let $U$ be an
arbitrary neighborhood of $x$. Then $C-U\cap W$ is a closed
subset of $C$, hence compact in the subspace topology on $C$ so,
by Lemma 26.1, it is compact in $X$. Moreover, $x\notin C-U\cap
W$ so by Lemma 26.4, since $X$ is Hausdorff, there exists
disjoint neighborhoods $V_1$ and $V_2$ of $x$ and $C-U\cap W$,
respectively. Now, note that $\overline{V_1}\cap V_2=\emptyset$ for
otherwise for any point $x\in\overline{V_1}\cap V_2$, $V_2$ is a
neighborhood of $x$ so $V_1\cap V_2\neq\emptyset$, this is a
contradiction. Let $V=V_1\cap U\cap W$. Then $V\subset U$ and
$V\subset C$ and, by Lemma B, $\overline V\subset C$, by Theorem
26.2, $\overline V$ is compact as desired.
\end{proof}
\newpage
\begin{problem}[A]
Let $S^1$ denote the circle
\[
S^1=\left\{\,(x,y)\in\RR^2\;\middle|\;x^2+y^2=1\,\right\}
\]
and let $B^2$ denote the closed disk
\[
B^2=\left\{\,(x,y)\in\RR^2\;\middle|\;x^2+y^2\leq 1\,\right\}.
\]
Prove that the quotient space $(S^1\times[0,1])/(S^1\times 0)$
(see HW \#4 for the notation) is homeomorphic to $B^2$.
\end{problem}
\begin{proof}
Note that, by Theorem 26.6, it suffices to find a bijective
continuous function $\bar\phi$, since $B^2$ is Hausdorff and $CS^1$ is
compact, by Theorem 26.5. Consider the map $\phi\colon
S^1\times[0,1]\to B^2$ given by $(x,y,z)\mapsto (zx,zy)$. We will show
that the induced map on the quotient space $\bar\phi$ is a continuous
bijection.

To see that $\bar\phi$ is continuous, let $\Phi\colon
S^1\times[0,1]\to\RR^2$ be $\phi$ with whose codomain has been
extended. Then, note that $\pi_1\circ\Phi(x,y)=zx$ and
$\pi_2\circ\Phi(x,y)=zy$ are continuous by Theorem 21.4, so by Theorem
18.4 and 18.2(e), $\phi$ is continuous. Moreover, if
$(x_1,y_1,z_1)\sim(x_2,y_2,z_2)$ then $(x_1,y_1,z_1)=(x_2,y_2,z_2)$,
in which case $\phi(x_1,y_1,z_1)=\phi(x_2,y_2,z_2)$, or
$(x_1,y_1,0)=(x_2,y_2,0)$ for any $(x_1,y_1,0)\in S^1\times 0$, in
which case $\phi(x_1,y_1,0)=0=\phi(x_2,y_2,0)$, $\phi$ preserves the
equivalence relation so by Theorem Q.3, $\bar\phi$ is continuous.

Now we show that $\bar\phi$ is bijective. Surjectivity of $\bar\phi$
follows from surjectivity of $\phi$. Let $(x,y)\in B^2$ and put
$z_0=\sqrt{x^2+y^2}$, $y_0=y/z_0$ and $x_0=x/x_0$. Note that
$x_0^2+y_0^2=x^2/(x^2+y^2)+y^2/(x^2+y^2)=1$ and $\sqrt{x^2+y^2}\leq 1$
for all $x,y$ so $z\leq 1$ so $(x_0,y_0,z_0)\in S^1\times[0,1]$. Thus,
$\phi(x_0,y_0,z_0)=(x,y)$ so $\phi$ is surjective. This implies that
$\bar\phi$ is surjective.

Finally, to see that $\bar\phi$ is injective suppose
$\bar\phi([x_1,y_1,z_1])=\bar\phi([x_2,y_2,z_2])$ then
\[
\tag{(*)}
(z_1x_1,z_1y_1)=z_1(x_1,y_1)=z_2(x_2,y_2)=(z_2x_2,z_2y_2)
\]
so, if $z_1=0$ we must have $z_2=0$ since $(0,0)\notin S^1$, hence
$(x_1,y_1,z_1)=(x_1,y_1,0)$ and $(x_2,y_2,z_2)=(x_2,y_2,0)$ so
$(x_1,y_1,z_1)\sim(x_2,y_2,z_2)$, if $z_1\neq 0$ then we can divide by
$z_1$ on both sides and we must have $z_2=z_1$ since
$\sqrt{(z_2x_2/z_1)^2+(z_2y/z_1)^2}=|z_2/z_1|\sqrt{x_2^2+y_2^2}=1$ and
$z_1,z_2\geq 0$ so, in fact, we have $(x_1,y_1,z_1)=(x_2,y_2,z_2)$ so
in particular $(x_1,y_1,z_1)\sim(x_2,y_2,z_2)$. Thus, $\bar\phi$ is
injective.

We conclude, by Theorem 26.6, that $\phi$ is a homeomorphism and
$CS^1\cong B^2$.
\end{proof}

%%% Local Variables:
%%% mode: latex
%%% TeX-master: "../MA571-HW-Current"
%%% End:
